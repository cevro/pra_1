\documentclass[a4paper,10pt]{article}
%\usepackage[IL2]{fontenc}
\usepackage[utf8x]{inputenc}
\usepackage[czech]{babel}
\usepackage{listings}  
\usepackage{amsfonts,amsmath,amssymb,graphicx,color}
%\usepackage[total={17cm,27cm}, top=2cm, left=2cm, includefoot]{geometry}
%\usepackage{fancyhdr}
\usepackage{fkssugar}
\usepackage{hyperref}

%\usepackage{caption}
\renewcommand{\popi}[2]{$\frac{#1}{[\jd{#2}]}$}
\renewcommand{\figurename}{Obr.}
\addto\captionsczech{\renewcommand{\figurename}{Obr.}}
\addto\captionsczech{\renewcommand{\tablename}{Tab.}}

\begin{document}
\def\mean#1{\left< #1 \right>}
\noindent
{\large Fyzikální praktikum 1.} \hfil {\large FJFI ČVUT V Praze}\\
\noindent
{\large\textbf{pracovní úkol \# 1}}
\begin{center}
{\large\textit{Mechanické pokusy na vzduchové dráze}}
\end{center}
\noindent
\rule{\textwidth}{1px}
\vspace{\baselineskip}

\emph{Michal Červeňák}
\par
\vspace{\baselineskip}
\begin{minipage}[l]{0.5\textwidth}%
\textit{dátum merania:}~05.12. 2016\\%
%\vspace{\baselineskip}%
\par%
\noindent%
\textit{skupina:}~4\\%
%\vspace{\baselineskip}%
\par%
\noindent%
\textit{Klasifikace:}\dotfill\\%
\end{minipage}

\section{Pracovní úkol}

\begin{enumerate}
\item DU: Zopakujte si, jak sčítáme, odečítáme a násobíme veličiny s chybou.
\item Elastické srážky na vzduchové dráze. Při měření použijete 2 vozíčky o různých hmotnostech.
Zvažte je na digitálních vahách. Jeden z nich ponechte před srážkou v klidu, druhému udělte
nenulovou počáteční rychlost pomocí startovacího zařízení. Pro každou ze 3 startovacích rychlostí
proveďte minimálně 10 měření. Poté obraťte konfiguraci vozíčků a měření opakujte. Celkově tedy
máte alespoň 60 měření. Počet měření je nutné dodržet, bez potřebné statistiky nebudete schopni
zpracovat výsledky měření.
\item Z naměřených dat rychlostí prvního vozíčku zjistěte, s jakou přesností jste schopni měřit rychlost
v. Tuto chybu určete zvlášť pro každou startovací rychlost a obě hmotnosti vozíčků. Do grafu
naneste závislost relativní a absolutní chyby rychlosti v závislosti na velikosti startovací rychlosti.
Rozmyslete si, jaký je rozdíl mezi systematickou a statistickou chybou a která z nich je pro
vaše měření zásadní, t.j. která vám výsledky více ovlivňuje. Pokuste se odhadnou systematickou
chybu měření - co je jejím hlavním zdrojem? Odhadněte brzdný koeficient vozíčku na dráze.
\item S použitím získané přesnosti měření rychlosti zjistěte s jakou přesností můžete měřit hybnost
p a energii E (přesnost měření hmotnosti berte dle použitého přístroje). Určete, jak se vámi
změřené celkové hybnosti resp. energie před a po srážce musí lišit, abyste je v rámci chyby
měření mohli prohlásit za shodné. Pečlivě si rozmyslete, kolikrát se vám do finálního výsledku
chyba rychlosti, potažmo hybnosti a energie promítne.
\item Z rychlosti vozíčků před srážkou a po srážce zjistěte změnu hybnosti ∆p a změnu energie ∆E
pro každou ze startovacích rychlostí a obě konfigurace vozíčků. Diskutujte, zda výsledek odpovídá
očekávanému, tedy zda je změna hybnosti a energie v rámci předpokládaného chybového intervalu
(s přesností $1\sigma$, $2\sigma$ nebo $3\sigma$). Rozhodněte, zda můžete zákony zachování považovat za ověřené.
\item Do grafu vyneste závislost celkové hybnosti po srážce $p_0$ na celkové hybnosti před srážkou p
a závislost celkové energie po srážce E0 na celkové energii před srážkou $E$. V obou závislostech
zobrazte i errorbary (viz poznámky na konci návodu) s předpokládanou chybou měření. Do grafu
zaneste přímku ideálního případu, kdy $\Delta p = 0$, $\Delta E = 0$ a diskutujte, zda jste graficky dokázali
či nedokázali zákony zachování. Pokud budete získaná data fitovat, nespoléhejte se pouze na
Gnuplot, že vám výsledky sám správně nafituje, ale zjitěte a v protokolu uveďte jakou metodu
fitu jste použili.
\item Pomocí tlakového senzoru změřte průběh síly při odrazu vozíku. Vypočtěte změnu hybnosti
pomocí integrálu průběhu síly a srovnejte ji se změnou hybnosti změřené pohybovým senzorem.
1
Opakujte měření pro každou startovací rychlost alespoň 10x. Vyneste do grafu změnu hybnosti
naměřenou silovým senzorem ∆pint v závislosti na změně hybnosti určené pohybovým senzorem
∆prychl, opět i s errorbary. Body proložte přímkou $\Delta p_{int} = a \Delta p_{rychl} + b$ a diskutujte rozdíl
směrnice a posunu přímky oproti ideálnímu případu $a = 1$, $b = 0$.

\end{enumerate}



\section{Pomôcky}
Vzduchová dráha s příslušenstvím, digitální váhy, 2x pohybový senzor PASCO, silový
senzor PASCO, PC (DataStudio)

\section{Teória}
Odvodenie vzťahov nájdete na \cite{C_1}.

Hybnosť telesa pre RPP je 
\eq{
p = m v\,, \lbl{R_1}
}
kde $m$ je hmotnosť vozíku a $v$ je jeho rýchlosť.

Celkovú hybnosť sústavy určíme podľa vzťahu
\eq{
\vect p = \sum_i \vect{p_i} \,, \lbl{R_2}
}
kde $\vect p_i$ je hybnosť telesa $i$.

Energiu telesa v našom prípade určíme ako
\eq{
E = \frac{1}{2} m v^2 \,,\lbl{R_3}
}
kde $m$ je hmotnosť vozíku a $v$ je jeho rýchlosť.

Celkovú energiu sústavy získame ako 
\eq{
E = \sum_i E_i \,, \lbl{R_4}
}
kde $E_1$ je energia $i$-teho telesa.



\subsubsection{Spracovanie chýb merania}

Označme $\mean{t}$ aritmetický priemer nameraných hodnôt $t_i$, a $\Delta t$ hodnotu $\mean{t}-t$, pričom 
\eq{
\mean{t} = \frac{1}{n}\sum_{i=1}^n t_i \,, \lbl{SCH_1}
}  
a chybu aritmetického priemeru 
\eq{
  \sigma_0=\sqrt{\frac{\sum_{i=1}^n \(t_i - \mean{t}\)^2}{n\(n-1\)}}\,, \lbl{SCH_2}
}
pričom $n$ je počet meraní.

\section{Postup merania}
\begin{enumerate}
\item Obidva vozíčky boli odvážené na digitálnych váhach, jeden samostatne druhý spolu s použitými závažiami
\item Oba boli položené na vzduchoví dráhu odrazovými plochami proti sebe
\item Jeden z vozíkov sa nastavil o odpaľovaciemu zariadeniu a bol pomocou neho uvedený do pohybu
\item Pomocou DataStudio boli zaznamenávané polohy obch vozíčkov pred a po náraze.
\item pomocou programu DataStudio boli fitnuté rýchlosti vozíčkov pred a vzájomnom náraze
\item postup bol opakovaný pre každú kombináciu oddaľovaný vozíček -- poloha odpaľovacieho zariadenia $10 \times$.
\item Následne bol jeden z pohybových senzorov nahradený senzorom sily
\item pre každú pozíciu odpaľovacieho zariadenia bolo nameraných 5 odpalov a odrazov od senzoru.
\item pri každom odpale sa zaznamenávala poloha s sila. z ktorých s následne určila rýchlosť a impulz.
\end{enumerate}



\section{Výsledky merania}
Označením $i\rightarrow j$ označujeme pokus, kde vozíček č. $i$ naráža do stojaceho vozíku č $j$.
Do tabuliek Tab. \ref{T_1}, Tab. \ref{T_2}, Tab. \ref{T_3}, Tab. \ref{T_4}, Tab. \ref{T_5} a Tab. \ref{T_6} 
sú vynesené namerané dáta rýchlosti pred $v^B$ a po $v^A$ náraze pre vozíčky č.1 s hmotnosťou $m_1="247.98 g"$ a č.2 $m = "208.54 g"$.

Z nich boli dopočítané podľa \ref{R_1} a \ref{R_2} hybnosti pred $p^B$ a po $p^A$ náraze a ich závislosť bola vynesená od grafu Obr. \ref{G_1}. 
Ďalej bola vypočítaná podľa \ref{R_3} a \ref{R_4} celková energia sústavy pred $E^B$ a po $E^A$ náraze a ich závislosť bola vynesená od grafu Obr. \ref{G_2}.

Scriptom v v prílohe bol vypočítaný impulz sily a vynesený do tabuľky Tab. \ref{T_7} a následne do grafu Obr. \ref{G_3}.

Z tabuliek Tab. \ref{T_1}, Tab. \ref{T_2}, Tab. \ref{T_3}, Tab. \ref{T_4}, Tab. \ref{T_5} a Tab. \ref{T_6} boli za pomoci vzťahov \ref{SCH_1} a \ref{SCH_2}.
boli vypočítané hodnoty $\Delta v_1^B$ a $\mean{v_1^B}$ pre jednotlivé kombinácie a vynesené do tabuľky \ref{T_8} a následne do grafu \ref{G_4}.



\begin{figure}
% GNUPLOT: LaTeX picture
\setlength{\unitlength}{0.240900pt}
\ifx\plotpoint\undefined\newsavebox{\plotpoint}\fi
\begin{picture}(1500,900)(0,0)
\sbox{\plotpoint}{\rule[-0.200pt]{0.400pt}{0.400pt}}%
\put(171.0,131.0){\rule[-0.200pt]{4.818pt}{0.400pt}}
\put(151,131){\makebox(0,0)[r]{ 20}}
\put(1419.0,131.0){\rule[-0.200pt]{4.818pt}{0.400pt}}
\put(171.0,212.0){\rule[-0.200pt]{4.818pt}{0.400pt}}
\put(151,212){\makebox(0,0)[r]{ 40}}
\put(1419.0,212.0){\rule[-0.200pt]{4.818pt}{0.400pt}}
\put(171.0,293.0){\rule[-0.200pt]{4.818pt}{0.400pt}}
\put(151,293){\makebox(0,0)[r]{ 60}}
\put(1419.0,293.0){\rule[-0.200pt]{4.818pt}{0.400pt}}
\put(171.0,374.0){\rule[-0.200pt]{4.818pt}{0.400pt}}
\put(151,374){\makebox(0,0)[r]{ 80}}
\put(1419.0,374.0){\rule[-0.200pt]{4.818pt}{0.400pt}}
\put(171.0,455.0){\rule[-0.200pt]{4.818pt}{0.400pt}}
\put(151,455){\makebox(0,0)[r]{ 100}}
\put(1419.0,455.0){\rule[-0.200pt]{4.818pt}{0.400pt}}
\put(171.0,535.0){\rule[-0.200pt]{4.818pt}{0.400pt}}
\put(151,535){\makebox(0,0)[r]{ 120}}
\put(1419.0,535.0){\rule[-0.200pt]{4.818pt}{0.400pt}}
\put(171.0,616.0){\rule[-0.200pt]{4.818pt}{0.400pt}}
\put(151,616){\makebox(0,0)[r]{ 140}}
\put(1419.0,616.0){\rule[-0.200pt]{4.818pt}{0.400pt}}
\put(171.0,697.0){\rule[-0.200pt]{4.818pt}{0.400pt}}
\put(151,697){\makebox(0,0)[r]{ 160}}
\put(1419.0,697.0){\rule[-0.200pt]{4.818pt}{0.400pt}}
\put(171.0,778.0){\rule[-0.200pt]{4.818pt}{0.400pt}}
\put(151,778){\makebox(0,0)[r]{ 180}}
\put(1419.0,778.0){\rule[-0.200pt]{4.818pt}{0.400pt}}
\put(171.0,859.0){\rule[-0.200pt]{4.818pt}{0.400pt}}
\put(151,859){\makebox(0,0)[r]{ 200}}
\put(1419.0,859.0){\rule[-0.200pt]{4.818pt}{0.400pt}}
\put(171.0,131.0){\rule[-0.200pt]{0.400pt}{4.818pt}}
\put(171,90){\makebox(0,0){ 40}}
\put(171.0,839.0){\rule[-0.200pt]{0.400pt}{4.818pt}}
\put(330.0,131.0){\rule[-0.200pt]{0.400pt}{4.818pt}}
\put(330,90){\makebox(0,0){ 60}}
\put(330.0,839.0){\rule[-0.200pt]{0.400pt}{4.818pt}}
\put(488.0,131.0){\rule[-0.200pt]{0.400pt}{4.818pt}}
\put(488,90){\makebox(0,0){ 80}}
\put(488.0,839.0){\rule[-0.200pt]{0.400pt}{4.818pt}}
\put(647.0,131.0){\rule[-0.200pt]{0.400pt}{4.818pt}}
\put(647,90){\makebox(0,0){ 100}}
\put(647.0,839.0){\rule[-0.200pt]{0.400pt}{4.818pt}}
\put(805.0,131.0){\rule[-0.200pt]{0.400pt}{4.818pt}}
\put(805,90){\makebox(0,0){ 120}}
\put(805.0,839.0){\rule[-0.200pt]{0.400pt}{4.818pt}}
\put(964.0,131.0){\rule[-0.200pt]{0.400pt}{4.818pt}}
\put(964,90){\makebox(0,0){ 140}}
\put(964.0,839.0){\rule[-0.200pt]{0.400pt}{4.818pt}}
\put(1122.0,131.0){\rule[-0.200pt]{0.400pt}{4.818pt}}
\put(1122,90){\makebox(0,0){ 160}}
\put(1122.0,839.0){\rule[-0.200pt]{0.400pt}{4.818pt}}
\put(1281.0,131.0){\rule[-0.200pt]{0.400pt}{4.818pt}}
\put(1281,90){\makebox(0,0){ 180}}
\put(1281.0,839.0){\rule[-0.200pt]{0.400pt}{4.818pt}}
\put(1439.0,131.0){\rule[-0.200pt]{0.400pt}{4.818pt}}
\put(1439,90){\makebox(0,0){ 200}}
\put(1439.0,839.0){\rule[-0.200pt]{0.400pt}{4.818pt}}
\put(171.0,131.0){\rule[-0.200pt]{0.400pt}{175.375pt}}
\put(171.0,131.0){\rule[-0.200pt]{305.461pt}{0.400pt}}
\put(1439.0,131.0){\rule[-0.200pt]{0.400pt}{175.375pt}}
\put(171.0,859.0){\rule[-0.200pt]{305.461pt}{0.400pt}}
\put(30,495){\makebox(0,0){\popi{p^A}{mN\cdot s}}}
\put(805,29){\makebox(0,0){\popi{p^B}{mN\cdot s}}}
\put(771,819){\makebox(0,0)[r]{$1\rightarrow2$ pre \uv{najvačšiu} pozíciu}}
\put(791.0,819.0){\rule[-0.200pt]{24.090pt}{0.400pt}}
\put(791.0,809.0){\rule[-0.200pt]{0.400pt}{4.818pt}}
\put(891.0,809.0){\rule[-0.200pt]{0.400pt}{4.818pt}}
\put(1200.0,652.0){\rule[-0.200pt]{0.400pt}{22.163pt}}
\put(1190.0,652.0){\rule[-0.200pt]{4.818pt}{0.400pt}}
\put(1190.0,744.0){\rule[-0.200pt]{4.818pt}{0.400pt}}
\put(953.0,522.0){\rule[-0.200pt]{0.400pt}{22.163pt}}
\put(943.0,522.0){\rule[-0.200pt]{4.818pt}{0.400pt}}
\put(943.0,614.0){\rule[-0.200pt]{4.818pt}{0.400pt}}
\put(1029.0,570.0){\rule[-0.200pt]{0.400pt}{22.163pt}}
\put(1019.0,570.0){\rule[-0.200pt]{4.818pt}{0.400pt}}
\put(1019.0,662.0){\rule[-0.200pt]{4.818pt}{0.400pt}}
\put(1120.0,617.0){\rule[-0.200pt]{0.400pt}{22.163pt}}
\put(1110.0,617.0){\rule[-0.200pt]{4.818pt}{0.400pt}}
\put(1110.0,709.0){\rule[-0.200pt]{4.818pt}{0.400pt}}
\put(1106.0,593.0){\rule[-0.200pt]{0.400pt}{22.404pt}}
\put(1096.0,593.0){\rule[-0.200pt]{4.818pt}{0.400pt}}
\put(1096.0,686.0){\rule[-0.200pt]{4.818pt}{0.400pt}}
\put(994.0,510.0){\rule[-0.200pt]{0.400pt}{22.163pt}}
\put(984.0,510.0){\rule[-0.200pt]{4.818pt}{0.400pt}}
\put(984.0,602.0){\rule[-0.200pt]{4.818pt}{0.400pt}}
\put(1069.0,605.0){\rule[-0.200pt]{0.400pt}{22.163pt}}
\put(1059.0,605.0){\rule[-0.200pt]{4.818pt}{0.400pt}}
\put(1059.0,697.0){\rule[-0.200pt]{4.818pt}{0.400pt}}
\put(872.0,532.0){\rule[-0.200pt]{0.400pt}{22.163pt}}
\put(862.0,532.0){\rule[-0.200pt]{4.818pt}{0.400pt}}
\put(862.0,624.0){\rule[-0.200pt]{4.818pt}{0.400pt}}
\put(872.0,547.0){\rule[-0.200pt]{0.400pt}{22.163pt}}
\put(862.0,547.0){\rule[-0.200pt]{4.818pt}{0.400pt}}
\put(862.0,639.0){\rule[-0.200pt]{4.818pt}{0.400pt}}
\put(1125.0,596.0){\rule[-0.200pt]{0.400pt}{22.163pt}}
\put(1115.0,596.0){\rule[-0.200pt]{4.818pt}{0.400pt}}
\put(1115.0,688.0){\rule[-0.200pt]{4.818pt}{0.400pt}}
\put(1106.0,624.0){\rule[-0.200pt]{0.400pt}{22.163pt}}
\put(1096.0,624.0){\rule[-0.200pt]{4.818pt}{0.400pt}}
\put(1096.0,716.0){\rule[-0.200pt]{4.818pt}{0.400pt}}
\put(1093.0,698.0){\rule[-0.200pt]{51.553pt}{0.400pt}}
\put(1093.0,688.0){\rule[-0.200pt]{0.400pt}{4.818pt}}
\put(1307.0,688.0){\rule[-0.200pt]{0.400pt}{4.818pt}}
\put(846.0,568.0){\rule[-0.200pt]{51.553pt}{0.400pt}}
\put(846.0,558.0){\rule[-0.200pt]{0.400pt}{4.818pt}}
\put(1060.0,558.0){\rule[-0.200pt]{0.400pt}{4.818pt}}
\put(922.0,616.0){\rule[-0.200pt]{51.553pt}{0.400pt}}
\put(922.0,606.0){\rule[-0.200pt]{0.400pt}{4.818pt}}
\put(1136.0,606.0){\rule[-0.200pt]{0.400pt}{4.818pt}}
\put(1013.0,663.0){\rule[-0.200pt]{51.553pt}{0.400pt}}
\put(1013.0,653.0){\rule[-0.200pt]{0.400pt}{4.818pt}}
\put(1227.0,653.0){\rule[-0.200pt]{0.400pt}{4.818pt}}
\put(999.0,640.0){\rule[-0.200pt]{51.553pt}{0.400pt}}
\put(999.0,630.0){\rule[-0.200pt]{0.400pt}{4.818pt}}
\put(1213.0,630.0){\rule[-0.200pt]{0.400pt}{4.818pt}}
\put(887.0,556.0){\rule[-0.200pt]{51.553pt}{0.400pt}}
\put(887.0,546.0){\rule[-0.200pt]{0.400pt}{4.818pt}}
\put(1101.0,546.0){\rule[-0.200pt]{0.400pt}{4.818pt}}
\put(962.0,651.0){\rule[-0.200pt]{51.312pt}{0.400pt}}
\put(962.0,641.0){\rule[-0.200pt]{0.400pt}{4.818pt}}
\put(1175.0,641.0){\rule[-0.200pt]{0.400pt}{4.818pt}}
\put(765.0,578.0){\rule[-0.200pt]{51.553pt}{0.400pt}}
\put(765.0,568.0){\rule[-0.200pt]{0.400pt}{4.818pt}}
\put(979.0,568.0){\rule[-0.200pt]{0.400pt}{4.818pt}}
\put(765.0,593.0){\rule[-0.200pt]{51.553pt}{0.400pt}}
\put(765.0,583.0){\rule[-0.200pt]{0.400pt}{4.818pt}}
\put(979.0,583.0){\rule[-0.200pt]{0.400pt}{4.818pt}}
\put(1018.0,642.0){\rule[-0.200pt]{51.553pt}{0.400pt}}
\put(1018.0,632.0){\rule[-0.200pt]{0.400pt}{4.818pt}}
\put(1232.0,632.0){\rule[-0.200pt]{0.400pt}{4.818pt}}
\put(999.0,670.0){\rule[-0.200pt]{51.553pt}{0.400pt}}
\put(999.0,660.0){\rule[-0.200pt]{0.400pt}{4.818pt}}
\put(1200,698){\makebox(0,0){$+$}}
\put(953,568){\makebox(0,0){$+$}}
\put(1029,616){\makebox(0,0){$+$}}
\put(1120,663){\makebox(0,0){$+$}}
\put(1106,640){\makebox(0,0){$+$}}
\put(994,556){\makebox(0,0){$+$}}
\put(1069,651){\makebox(0,0){$+$}}
\put(872,578){\makebox(0,0){$+$}}
\put(872,593){\makebox(0,0){$+$}}
\put(1125,642){\makebox(0,0){$+$}}
\put(1106,670){\makebox(0,0){$+$}}
\put(841,819){\makebox(0,0){$+$}}
\put(1213.0,660.0){\rule[-0.200pt]{0.400pt}{4.818pt}}
\put(771,778){\makebox(0,0)[r]{$1\rightarrow2$ pre \uv{strednú} pozíciu}}
\multiput(791,778)(20.756,0.000){5}{\usebox{\plotpoint}}
\put(891,778){\usebox{\plotpoint}}
\put(791.00,788.00){\usebox{\plotpoint}}
\put(791,768){\usebox{\plotpoint}}
\put(891.00,788.00){\usebox{\plotpoint}}
\put(891,768){\usebox{\plotpoint}}
\multiput(459,277)(0.000,20.756){7}{\usebox{\plotpoint}}
\put(459,402){\usebox{\plotpoint}}
\put(449.00,277.00){\usebox{\plotpoint}}
\put(469,277){\usebox{\plotpoint}}
\put(449.00,402.00){\usebox{\plotpoint}}
\put(469,402){\usebox{\plotpoint}}
\multiput(754,417)(0.000,20.756){7}{\usebox{\plotpoint}}
\put(754,542){\usebox{\plotpoint}}
\put(744.00,417.00){\usebox{\plotpoint}}
\put(764,417){\usebox{\plotpoint}}
\put(744.00,542.00){\usebox{\plotpoint}}
\put(764,542){\usebox{\plotpoint}}
\multiput(872,472)(0.000,20.756){7}{\usebox{\plotpoint}}
\put(872,597){\usebox{\plotpoint}}
\put(862.00,472.00){\usebox{\plotpoint}}
\put(882,472){\usebox{\plotpoint}}
\put(862.00,597.00){\usebox{\plotpoint}}
\put(882,597){\usebox{\plotpoint}}
\multiput(827,453)(0.000,20.756){7}{\usebox{\plotpoint}}
\put(827,578){\usebox{\plotpoint}}
\put(817.00,453.00){\usebox{\plotpoint}}
\put(837,453){\usebox{\plotpoint}}
\put(817.00,578.00){\usebox{\plotpoint}}
\put(837,578){\usebox{\plotpoint}}
\multiput(764,424)(0.000,20.756){6}{\usebox{\plotpoint}}
\put(764,548){\usebox{\plotpoint}}
\put(754.00,424.00){\usebox{\plotpoint}}
\put(774,424){\usebox{\plotpoint}}
\put(754.00,548.00){\usebox{\plotpoint}}
\put(774,548){\usebox{\plotpoint}}
\multiput(793,447)(0.000,20.756){7}{\usebox{\plotpoint}}
\put(793,572){\usebox{\plotpoint}}
\put(783.00,447.00){\usebox{\plotpoint}}
\put(803,447){\usebox{\plotpoint}}
\put(783.00,572.00){\usebox{\plotpoint}}
\put(803,572){\usebox{\plotpoint}}
\multiput(567,327)(0.000,20.756){6}{\usebox{\plotpoint}}
\put(567,451){\usebox{\plotpoint}}
\put(557.00,327.00){\usebox{\plotpoint}}
\put(577,327){\usebox{\plotpoint}}
\put(557.00,451.00){\usebox{\plotpoint}}
\put(577,451){\usebox{\plotpoint}}
\multiput(811,447)(0.000,20.756){7}{\usebox{\plotpoint}}
\put(811,572){\usebox{\plotpoint}}
\put(801.00,447.00){\usebox{\plotpoint}}
\put(821,447){\usebox{\plotpoint}}
\put(801.00,572.00){\usebox{\plotpoint}}
\put(821,572){\usebox{\plotpoint}}
\multiput(801,461)(0.000,20.756){6}{\usebox{\plotpoint}}
\put(801,585){\usebox{\plotpoint}}
\put(791.00,461.00){\usebox{\plotpoint}}
\put(811,461){\usebox{\plotpoint}}
\put(791.00,585.00){\usebox{\plotpoint}}
\put(811,585){\usebox{\plotpoint}}
\multiput(760,416)(0.000,20.756){7}{\usebox{\plotpoint}}
\put(760,541){\usebox{\plotpoint}}
\put(750.00,416.00){\usebox{\plotpoint}}
\put(770,416){\usebox{\plotpoint}}
\put(750.00,541.00){\usebox{\plotpoint}}
\put(770,541){\usebox{\plotpoint}}
\multiput(839,471)(0.000,20.756){7}{\usebox{\plotpoint}}
\put(839,596){\usebox{\plotpoint}}
\put(829.00,471.00){\usebox{\plotpoint}}
\put(849,471){\usebox{\plotpoint}}
\put(829.00,596.00){\usebox{\plotpoint}}
\put(849,596){\usebox{\plotpoint}}
\multiput(335,339)(20.756,0.000){12}{\usebox{\plotpoint}}
\put(584,339){\usebox{\plotpoint}}
\put(335.00,329.00){\usebox{\plotpoint}}
\put(335,349){\usebox{\plotpoint}}
\put(584.00,329.00){\usebox{\plotpoint}}
\put(584,349){\usebox{\plotpoint}}
\multiput(630,480)(20.756,0.000){12}{\usebox{\plotpoint}}
\put(878,480){\usebox{\plotpoint}}
\put(630.00,470.00){\usebox{\plotpoint}}
\put(630,490){\usebox{\plotpoint}}
\put(878.00,470.00){\usebox{\plotpoint}}
\put(878,490){\usebox{\plotpoint}}
\multiput(748,535)(20.756,0.000){12}{\usebox{\plotpoint}}
\put(996,535){\usebox{\plotpoint}}
\put(748.00,525.00){\usebox{\plotpoint}}
\put(748,545){\usebox{\plotpoint}}
\put(996.00,525.00){\usebox{\plotpoint}}
\put(996,545){\usebox{\plotpoint}}
\multiput(702,516)(20.756,0.000){12}{\usebox{\plotpoint}}
\put(951,516){\usebox{\plotpoint}}
\put(702.00,506.00){\usebox{\plotpoint}}
\put(702,526){\usebox{\plotpoint}}
\put(951.00,506.00){\usebox{\plotpoint}}
\put(951,526){\usebox{\plotpoint}}
\multiput(639,486)(20.756,0.000){12}{\usebox{\plotpoint}}
\put(888,486){\usebox{\plotpoint}}
\put(639.00,476.00){\usebox{\plotpoint}}
\put(639,496){\usebox{\plotpoint}}
\put(888.00,476.00){\usebox{\plotpoint}}
\put(888,496){\usebox{\plotpoint}}
\multiput(669,510)(20.756,0.000){12}{\usebox{\plotpoint}}
\put(918,510){\usebox{\plotpoint}}
\put(669.00,500.00){\usebox{\plotpoint}}
\put(669,520){\usebox{\plotpoint}}
\put(918.00,500.00){\usebox{\plotpoint}}
\put(918,520){\usebox{\plotpoint}}
\multiput(443,389)(20.756,0.000){12}{\usebox{\plotpoint}}
\put(692,389){\usebox{\plotpoint}}
\put(443.00,379.00){\usebox{\plotpoint}}
\put(443,399){\usebox{\plotpoint}}
\put(692.00,379.00){\usebox{\plotpoint}}
\put(692,399){\usebox{\plotpoint}}
\multiput(687,510)(20.756,0.000){12}{\usebox{\plotpoint}}
\put(936,510){\usebox{\plotpoint}}
\put(687.00,500.00){\usebox{\plotpoint}}
\put(687,520){\usebox{\plotpoint}}
\put(936.00,500.00){\usebox{\plotpoint}}
\put(936,520){\usebox{\plotpoint}}
\multiput(677,523)(20.756,0.000){12}{\usebox{\plotpoint}}
\put(926,523){\usebox{\plotpoint}}
\put(677.00,513.00){\usebox{\plotpoint}}
\put(677,533){\usebox{\plotpoint}}
\put(926.00,513.00){\usebox{\plotpoint}}
\put(926,533){\usebox{\plotpoint}}
\multiput(636,478)(20.756,0.000){12}{\usebox{\plotpoint}}
\put(884,478){\usebox{\plotpoint}}
\put(636.00,468.00){\usebox{\plotpoint}}
\put(636,488){\usebox{\plotpoint}}
\put(884.00,468.00){\usebox{\plotpoint}}
\put(884,488){\usebox{\plotpoint}}
\multiput(714,534)(20.756,0.000){12}{\usebox{\plotpoint}}
\put(963,534){\usebox{\plotpoint}}
\put(714.00,524.00){\usebox{\plotpoint}}
\put(714,544){\usebox{\plotpoint}}
\put(963.00,524.00){\usebox{\plotpoint}}
\put(963,544){\usebox{\plotpoint}}
\put(459,339){\makebox(0,0){$\times$}}
\put(754,480){\makebox(0,0){$\times$}}
\put(872,535){\makebox(0,0){$\times$}}
\put(827,516){\makebox(0,0){$\times$}}
\put(764,486){\makebox(0,0){$\times$}}
\put(793,510){\makebox(0,0){$\times$}}
\put(567,389){\makebox(0,0){$\times$}}
\put(811,510){\makebox(0,0){$\times$}}
\put(801,523){\makebox(0,0){$\times$}}
\put(760,478){\makebox(0,0){$\times$}}
\put(839,534){\makebox(0,0){$\times$}}
\put(841,778){\makebox(0,0){$\times$}}
\sbox{\plotpoint}{\rule[-0.400pt]{0.800pt}{0.800pt}}%
\sbox{\plotpoint}{\rule[-0.200pt]{0.400pt}{0.400pt}}%
\put(771,737){\makebox(0,0)[r]{$1\rightarrow2$ pre \uv{najmenšiu} pozíciu}}
\sbox{\plotpoint}{\rule[-0.400pt]{0.800pt}{0.800pt}}%
\put(791.0,737.0){\rule[-0.400pt]{24.090pt}{0.800pt}}
\put(791.0,727.0){\rule[-0.400pt]{0.800pt}{4.818pt}}
\put(891.0,727.0){\rule[-0.400pt]{0.800pt}{4.818pt}}
\put(428.0,262.0){\rule[-0.400pt]{0.800pt}{9.877pt}}
\put(418.0,262.0){\rule[-0.400pt]{4.818pt}{0.800pt}}
\put(418.0,303.0){\rule[-0.400pt]{4.818pt}{0.800pt}}
\put(347.0,271.0){\rule[-0.400pt]{0.800pt}{10.118pt}}
\put(337.0,271.0){\rule[-0.400pt]{4.818pt}{0.800pt}}
\put(337.0,313.0){\rule[-0.400pt]{4.818pt}{0.800pt}}
\put(448.0,269.0){\rule[-0.400pt]{0.800pt}{9.877pt}}
\put(438.0,269.0){\rule[-0.400pt]{4.818pt}{0.800pt}}
\put(438.0,310.0){\rule[-0.400pt]{4.818pt}{0.800pt}}
\put(412.0,309.0){\rule[-0.400pt]{0.800pt}{9.877pt}}
\put(402.0,309.0){\rule[-0.400pt]{4.818pt}{0.800pt}}
\put(402.0,350.0){\rule[-0.400pt]{4.818pt}{0.800pt}}
\put(383.0,287.0){\rule[-0.400pt]{0.800pt}{10.118pt}}
\put(373.0,287.0){\rule[-0.400pt]{4.818pt}{0.800pt}}
\put(373.0,329.0){\rule[-0.400pt]{4.818pt}{0.800pt}}
\put(446.0,328.0){\rule[-0.400pt]{0.800pt}{9.877pt}}
\put(436.0,328.0){\rule[-0.400pt]{4.818pt}{0.800pt}}
\put(436.0,369.0){\rule[-0.400pt]{4.818pt}{0.800pt}}
\put(394.0,301.0){\rule[-0.400pt]{0.800pt}{9.877pt}}
\put(384.0,301.0){\rule[-0.400pt]{4.818pt}{0.800pt}}
\put(384.0,342.0){\rule[-0.400pt]{4.818pt}{0.800pt}}
\put(373.0,261.0){\rule[-0.400pt]{0.800pt}{10.118pt}}
\put(363.0,261.0){\rule[-0.400pt]{4.818pt}{0.800pt}}
\put(363.0,303.0){\rule[-0.400pt]{4.818pt}{0.800pt}}
\put(398.0,276.0){\rule[-0.400pt]{0.800pt}{10.118pt}}
\put(388.0,276.0){\rule[-0.400pt]{4.818pt}{0.800pt}}
\put(388.0,318.0){\rule[-0.400pt]{4.818pt}{0.800pt}}
\put(337.0,274.0){\rule[-0.400pt]{0.800pt}{9.877pt}}
\put(327.0,274.0){\rule[-0.400pt]{4.818pt}{0.800pt}}
\put(327.0,315.0){\rule[-0.400pt]{4.818pt}{0.800pt}}
\put(363.0,281.0){\rule[-0.400pt]{0.800pt}{9.877pt}}
\put(353.0,281.0){\rule[-0.400pt]{4.818pt}{0.800pt}}
\put(353.0,322.0){\rule[-0.400pt]{4.818pt}{0.800pt}}
\put(391.0,283.0){\rule[-0.400pt]{17.827pt}{0.800pt}}
\put(391.0,273.0){\rule[-0.400pt]{0.800pt}{4.818pt}}
\put(465.0,273.0){\rule[-0.400pt]{0.800pt}{4.818pt}}
\put(310.0,292.0){\rule[-0.400pt]{17.827pt}{0.800pt}}
\put(310.0,282.0){\rule[-0.400pt]{0.800pt}{4.818pt}}
\put(384.0,282.0){\rule[-0.400pt]{0.800pt}{4.818pt}}
\put(410.0,290.0){\rule[-0.400pt]{18.067pt}{0.800pt}}
\put(410.0,280.0){\rule[-0.400pt]{0.800pt}{4.818pt}}
\put(485.0,280.0){\rule[-0.400pt]{0.800pt}{4.818pt}}
\put(375.0,330.0){\rule[-0.400pt]{17.827pt}{0.800pt}}
\put(375.0,320.0){\rule[-0.400pt]{0.800pt}{4.818pt}}
\put(449.0,320.0){\rule[-0.400pt]{0.800pt}{4.818pt}}
\put(345.0,308.0){\rule[-0.400pt]{18.067pt}{0.800pt}}
\put(345.0,298.0){\rule[-0.400pt]{0.800pt}{4.818pt}}
\put(420.0,298.0){\rule[-0.400pt]{0.800pt}{4.818pt}}
\put(408.0,348.0){\rule[-0.400pt]{18.067pt}{0.800pt}}
\put(408.0,338.0){\rule[-0.400pt]{0.800pt}{4.818pt}}
\put(483.0,338.0){\rule[-0.400pt]{0.800pt}{4.818pt}}
\put(357.0,321.0){\rule[-0.400pt]{18.067pt}{0.800pt}}
\put(357.0,311.0){\rule[-0.400pt]{0.800pt}{4.818pt}}
\put(432.0,311.0){\rule[-0.400pt]{0.800pt}{4.818pt}}
\put(336.0,282.0){\rule[-0.400pt]{17.827pt}{0.800pt}}
\put(336.0,272.0){\rule[-0.400pt]{0.800pt}{4.818pt}}
\put(410.0,272.0){\rule[-0.400pt]{0.800pt}{4.818pt}}
\put(361.0,297.0){\rule[-0.400pt]{18.067pt}{0.800pt}}
\put(361.0,287.0){\rule[-0.400pt]{0.800pt}{4.818pt}}
\put(436.0,287.0){\rule[-0.400pt]{0.800pt}{4.818pt}}
\put(300.0,295.0){\rule[-0.400pt]{18.067pt}{0.800pt}}
\put(300.0,285.0){\rule[-0.400pt]{0.800pt}{4.818pt}}
\put(375.0,285.0){\rule[-0.400pt]{0.800pt}{4.818pt}}
\put(326.0,302.0){\rule[-0.400pt]{17.827pt}{0.800pt}}
\put(326.0,292.0){\rule[-0.400pt]{0.800pt}{4.818pt}}
\put(428,283){\makebox(0,0){$\ast$}}
\put(347,292){\makebox(0,0){$\ast$}}
\put(448,290){\makebox(0,0){$\ast$}}
\put(412,330){\makebox(0,0){$\ast$}}
\put(383,308){\makebox(0,0){$\ast$}}
\put(446,348){\makebox(0,0){$\ast$}}
\put(394,321){\makebox(0,0){$\ast$}}
\put(373,282){\makebox(0,0){$\ast$}}
\put(398,297){\makebox(0,0){$\ast$}}
\put(337,295){\makebox(0,0){$\ast$}}
\put(363,302){\makebox(0,0){$\ast$}}
\put(841,737){\makebox(0,0){$\ast$}}
\put(400.0,292.0){\rule[-0.400pt]{0.800pt}{4.818pt}}
\sbox{\plotpoint}{\rule[-0.500pt]{1.000pt}{1.000pt}}%
\sbox{\plotpoint}{\rule[-0.200pt]{0.400pt}{0.400pt}}%
\put(771,696){\makebox(0,0)[r]{$2\rightarrow1$ pre \uv{najvačšiu} pozíciu}}
\sbox{\plotpoint}{\rule[-0.500pt]{1.000pt}{1.000pt}}%
\multiput(791,696)(20.756,0.000){5}{\usebox{\plotpoint}}
\put(891,696){\usebox{\plotpoint}}
\put(791.00,706.00){\usebox{\plotpoint}}
\put(791,686){\usebox{\plotpoint}}
\put(891.00,706.00){\usebox{\plotpoint}}
\put(891,686){\usebox{\plotpoint}}
\multiput(1082,619)(0.000,20.756){4}{\usebox{\plotpoint}}
\put(1082,698){\usebox{\plotpoint}}
\put(1072.00,619.00){\usebox{\plotpoint}}
\put(1092,619){\usebox{\plotpoint}}
\put(1072.00,698.00){\usebox{\plotpoint}}
\put(1092,698){\usebox{\plotpoint}}
\multiput(1027,581)(0.000,20.756){4}{\usebox{\plotpoint}}
\put(1027,660){\usebox{\plotpoint}}
\put(1017.00,581.00){\usebox{\plotpoint}}
\put(1037,581){\usebox{\plotpoint}}
\put(1017.00,660.00){\usebox{\plotpoint}}
\put(1037,660){\usebox{\plotpoint}}
\multiput(1128,645)(0.000,20.756){4}{\usebox{\plotpoint}}
\put(1128,725){\usebox{\plotpoint}}
\put(1118.00,645.00){\usebox{\plotpoint}}
\put(1138,645){\usebox{\plotpoint}}
\put(1118.00,725.00){\usebox{\plotpoint}}
\put(1138,725){\usebox{\plotpoint}}
\multiput(1098,634)(0.000,20.756){4}{\usebox{\plotpoint}}
\put(1098,713){\usebox{\plotpoint}}
\put(1088.00,634.00){\usebox{\plotpoint}}
\put(1108,634){\usebox{\plotpoint}}
\put(1088.00,713.00){\usebox{\plotpoint}}
\put(1108,713){\usebox{\plotpoint}}
\multiput(1082,622)(0.000,20.756){4}{\usebox{\plotpoint}}
\put(1082,702){\usebox{\plotpoint}}
\put(1072.00,622.00){\usebox{\plotpoint}}
\put(1092,622){\usebox{\plotpoint}}
\put(1072.00,702.00){\usebox{\plotpoint}}
\put(1092,702){\usebox{\plotpoint}}
\multiput(1151,630)(0.000,20.756){4}{\usebox{\plotpoint}}
\put(1151,709){\usebox{\plotpoint}}
\put(1141.00,630.00){\usebox{\plotpoint}}
\put(1161,630){\usebox{\plotpoint}}
\put(1141.00,709.00){\usebox{\plotpoint}}
\put(1161,709){\usebox{\plotpoint}}
\multiput(1102,540)(0.000,20.756){4}{\usebox{\plotpoint}}
\put(1102,620){\usebox{\plotpoint}}
\put(1092.00,540.00){\usebox{\plotpoint}}
\put(1112,540){\usebox{\plotpoint}}
\put(1092.00,620.00){\usebox{\plotpoint}}
\put(1112,620){\usebox{\plotpoint}}
\multiput(1146,643)(0.000,20.756){4}{\usebox{\plotpoint}}
\put(1146,722){\usebox{\plotpoint}}
\put(1136.00,643.00){\usebox{\plotpoint}}
\put(1156,643){\usebox{\plotpoint}}
\put(1136.00,722.00){\usebox{\plotpoint}}
\put(1156,722){\usebox{\plotpoint}}
\multiput(1137,689)(0.000,20.756){4}{\usebox{\plotpoint}}
\put(1137,769){\usebox{\plotpoint}}
\put(1127.00,689.00){\usebox{\plotpoint}}
\put(1147,689){\usebox{\plotpoint}}
\put(1127.00,769.00){\usebox{\plotpoint}}
\put(1147,769){\usebox{\plotpoint}}
\multiput(1140,642)(0.000,20.756){4}{\usebox{\plotpoint}}
\put(1140,722){\usebox{\plotpoint}}
\put(1130.00,642.00){\usebox{\plotpoint}}
\put(1150,642){\usebox{\plotpoint}}
\put(1130.00,722.00){\usebox{\plotpoint}}
\put(1150,722){\usebox{\plotpoint}}
\multiput(1043,658)(20.756,0.000){4}{\usebox{\plotpoint}}
\put(1121,658){\usebox{\plotpoint}}
\put(1043.00,648.00){\usebox{\plotpoint}}
\put(1043,668){\usebox{\plotpoint}}
\put(1121.00,648.00){\usebox{\plotpoint}}
\put(1121,668){\usebox{\plotpoint}}
\multiput(989,620)(20.756,0.000){4}{\usebox{\plotpoint}}
\put(1066,620){\usebox{\plotpoint}}
\put(989.00,610.00){\usebox{\plotpoint}}
\put(989,630){\usebox{\plotpoint}}
\put(1066.00,610.00){\usebox{\plotpoint}}
\put(1066,630){\usebox{\plotpoint}}
\multiput(1089,685)(20.756,0.000){4}{\usebox{\plotpoint}}
\put(1167,685){\usebox{\plotpoint}}
\put(1089.00,675.00){\usebox{\plotpoint}}
\put(1089,695){\usebox{\plotpoint}}
\put(1167.00,675.00){\usebox{\plotpoint}}
\put(1167,695){\usebox{\plotpoint}}
\multiput(1060,673)(20.756,0.000){4}{\usebox{\plotpoint}}
\put(1137,673){\usebox{\plotpoint}}
\put(1060.00,663.00){\usebox{\plotpoint}}
\put(1060,683){\usebox{\plotpoint}}
\put(1137.00,663.00){\usebox{\plotpoint}}
\put(1137,683){\usebox{\plotpoint}}
\multiput(1043,662)(20.756,0.000){4}{\usebox{\plotpoint}}
\put(1121,662){\usebox{\plotpoint}}
\put(1043.00,652.00){\usebox{\plotpoint}}
\put(1043,672){\usebox{\plotpoint}}
\put(1121.00,652.00){\usebox{\plotpoint}}
\put(1121,672){\usebox{\plotpoint}}
\multiput(1112,669)(20.756,0.000){4}{\usebox{\plotpoint}}
\put(1190,669){\usebox{\plotpoint}}
\put(1112.00,659.00){\usebox{\plotpoint}}
\put(1112,679){\usebox{\plotpoint}}
\put(1190.00,659.00){\usebox{\plotpoint}}
\put(1190,679){\usebox{\plotpoint}}
\multiput(1063,580)(20.756,0.000){4}{\usebox{\plotpoint}}
\put(1141,580){\usebox{\plotpoint}}
\put(1063.00,570.00){\usebox{\plotpoint}}
\put(1063,590){\usebox{\plotpoint}}
\put(1141.00,570.00){\usebox{\plotpoint}}
\put(1141,590){\usebox{\plotpoint}}
\multiput(1108,683)(20.756,0.000){4}{\usebox{\plotpoint}}
\put(1185,683){\usebox{\plotpoint}}
\put(1108.00,673.00){\usebox{\plotpoint}}
\put(1108,693){\usebox{\plotpoint}}
\put(1185.00,673.00){\usebox{\plotpoint}}
\put(1185,693){\usebox{\plotpoint}}
\multiput(1098,729)(20.756,0.000){4}{\usebox{\plotpoint}}
\put(1175,729){\usebox{\plotpoint}}
\put(1098.00,719.00){\usebox{\plotpoint}}
\put(1098,739){\usebox{\plotpoint}}
\put(1175.00,719.00){\usebox{\plotpoint}}
\put(1175,739){\usebox{\plotpoint}}
\multiput(1101,682)(20.756,0.000){4}{\usebox{\plotpoint}}
\put(1179,682){\usebox{\plotpoint}}
\put(1101.00,672.00){\usebox{\plotpoint}}
\put(1101,692){\usebox{\plotpoint}}
\put(1179.00,672.00){\usebox{\plotpoint}}
\put(1179,692){\usebox{\plotpoint}}
\put(1082,658){\raisebox{-.8pt}{\makebox(0,0){$\Box$}}}
\put(1027,620){\raisebox{-.8pt}{\makebox(0,0){$\Box$}}}
\put(1128,685){\raisebox{-.8pt}{\makebox(0,0){$\Box$}}}
\put(1098,673){\raisebox{-.8pt}{\makebox(0,0){$\Box$}}}
\put(1082,662){\raisebox{-.8pt}{\makebox(0,0){$\Box$}}}
\put(1151,669){\raisebox{-.8pt}{\makebox(0,0){$\Box$}}}
\put(1102,580){\raisebox{-.8pt}{\makebox(0,0){$\Box$}}}
\put(1146,683){\raisebox{-.8pt}{\makebox(0,0){$\Box$}}}
\put(1137,729){\raisebox{-.8pt}{\makebox(0,0){$\Box$}}}
\put(1140,682){\raisebox{-.8pt}{\makebox(0,0){$\Box$}}}
\put(841,696){\raisebox{-.8pt}{\makebox(0,0){$\Box$}}}
\sbox{\plotpoint}{\rule[-0.600pt]{1.200pt}{1.200pt}}%
\sbox{\plotpoint}{\rule[-0.200pt]{0.400pt}{0.400pt}}%
\put(771,655){\makebox(0,0)[r]{$2\rightarrow1$ pre \uv{strednú} pozíciu}}
\sbox{\plotpoint}{\rule[-0.600pt]{1.200pt}{1.200pt}}%
\put(791.0,655.0){\rule[-0.600pt]{24.090pt}{1.200pt}}
\put(791.0,645.0){\rule[-0.600pt]{1.200pt}{4.818pt}}
\put(891.0,645.0){\rule[-0.600pt]{1.200pt}{4.818pt}}
\put(771.0,526.0){\rule[-0.600pt]{1.200pt}{13.249pt}}
\put(761.0,526.0){\rule[-0.600pt]{4.818pt}{1.200pt}}
\put(761.0,581.0){\rule[-0.600pt]{4.818pt}{1.200pt}}
\put(791.0,478.0){\rule[-0.600pt]{1.200pt}{13.249pt}}
\put(781.0,478.0){\rule[-0.600pt]{4.818pt}{1.200pt}}
\put(781.0,533.0){\rule[-0.600pt]{4.818pt}{1.200pt}}
\put(773.0,456.0){\rule[-0.600pt]{1.200pt}{13.249pt}}
\put(763.0,456.0){\rule[-0.600pt]{4.818pt}{1.200pt}}
\put(763.0,511.0){\rule[-0.600pt]{4.818pt}{1.200pt}}
\put(778.0,462.0){\rule[-0.600pt]{1.200pt}{13.249pt}}
\put(768.0,462.0){\rule[-0.600pt]{4.818pt}{1.200pt}}
\put(768.0,517.0){\rule[-0.600pt]{4.818pt}{1.200pt}}
\put(766.0,431.0){\rule[-0.600pt]{1.200pt}{13.249pt}}
\put(756.0,431.0){\rule[-0.600pt]{4.818pt}{1.200pt}}
\put(756.0,486.0){\rule[-0.600pt]{4.818pt}{1.200pt}}
\put(751.0,428.0){\rule[-0.600pt]{1.200pt}{13.249pt}}
\put(741.0,428.0){\rule[-0.600pt]{4.818pt}{1.200pt}}
\put(741.0,483.0){\rule[-0.600pt]{4.818pt}{1.200pt}}
\put(798.0,477.0){\rule[-0.600pt]{1.200pt}{13.249pt}}
\put(788.0,477.0){\rule[-0.600pt]{4.818pt}{1.200pt}}
\put(788.0,532.0){\rule[-0.600pt]{4.818pt}{1.200pt}}
\put(753.0,474.0){\rule[-0.600pt]{1.200pt}{13.249pt}}
\put(743.0,474.0){\rule[-0.600pt]{4.818pt}{1.200pt}}
\put(743.0,529.0){\rule[-0.600pt]{4.818pt}{1.200pt}}
\put(738.0,466.0){\rule[-0.600pt]{1.200pt}{13.249pt}}
\put(728.0,466.0){\rule[-0.600pt]{4.818pt}{1.200pt}}
\put(728.0,521.0){\rule[-0.600pt]{4.818pt}{1.200pt}}
\put(763.0,455.0){\rule[-0.600pt]{1.200pt}{13.249pt}}
\put(753.0,455.0){\rule[-0.600pt]{4.818pt}{1.200pt}}
\put(753.0,510.0){\rule[-0.600pt]{4.818pt}{1.200pt}}
\put(753.0,554.0){\rule[-0.600pt]{8.672pt}{1.200pt}}
\put(753.0,544.0){\rule[-0.600pt]{1.200pt}{4.818pt}}
\put(789.0,544.0){\rule[-0.600pt]{1.200pt}{4.818pt}}
\put(773.0,506.0){\rule[-0.600pt]{8.672pt}{1.200pt}}
\put(773.0,496.0){\rule[-0.600pt]{1.200pt}{4.818pt}}
\put(809.0,496.0){\rule[-0.600pt]{1.200pt}{4.818pt}}
\put(755.0,483.0){\rule[-0.600pt]{8.672pt}{1.200pt}}
\put(755.0,473.0){\rule[-0.600pt]{1.200pt}{4.818pt}}
\put(791.0,473.0){\rule[-0.600pt]{1.200pt}{4.818pt}}
\put(760.0,490.0){\rule[-0.600pt]{8.672pt}{1.200pt}}
\put(760.0,480.0){\rule[-0.600pt]{1.200pt}{4.818pt}}
\put(796.0,480.0){\rule[-0.600pt]{1.200pt}{4.818pt}}
\put(748.0,459.0){\rule[-0.600pt]{8.672pt}{1.200pt}}
\put(748.0,449.0){\rule[-0.600pt]{1.200pt}{4.818pt}}
\put(784.0,449.0){\rule[-0.600pt]{1.200pt}{4.818pt}}
\put(733.0,456.0){\rule[-0.600pt]{8.913pt}{1.200pt}}
\put(733.0,446.0){\rule[-0.600pt]{1.200pt}{4.818pt}}
\put(770.0,446.0){\rule[-0.600pt]{1.200pt}{4.818pt}}
\put(779.0,505.0){\rule[-0.600pt]{8.913pt}{1.200pt}}
\put(779.0,495.0){\rule[-0.600pt]{1.200pt}{4.818pt}}
\put(816.0,495.0){\rule[-0.600pt]{1.200pt}{4.818pt}}
\put(735.0,502.0){\rule[-0.600pt]{8.672pt}{1.200pt}}
\put(735.0,492.0){\rule[-0.600pt]{1.200pt}{4.818pt}}
\put(771.0,492.0){\rule[-0.600pt]{1.200pt}{4.818pt}}
\put(720.0,493.0){\rule[-0.600pt]{8.672pt}{1.200pt}}
\put(720.0,483.0){\rule[-0.600pt]{1.200pt}{4.818pt}}
\put(756.0,483.0){\rule[-0.600pt]{1.200pt}{4.818pt}}
\put(745.0,482.0){\rule[-0.600pt]{8.672pt}{1.200pt}}
\put(745.0,472.0){\rule[-0.600pt]{1.200pt}{4.818pt}}
\put(771,554){\makebox(0,0){$\blacksquare$}}
\put(791,506){\makebox(0,0){$\blacksquare$}}
\put(773,483){\makebox(0,0){$\blacksquare$}}
\put(778,490){\makebox(0,0){$\blacksquare$}}
\put(766,459){\makebox(0,0){$\blacksquare$}}
\put(751,456){\makebox(0,0){$\blacksquare$}}
\put(798,505){\makebox(0,0){$\blacksquare$}}
\put(753,502){\makebox(0,0){$\blacksquare$}}
\put(738,493){\makebox(0,0){$\blacksquare$}}
\put(763,482){\makebox(0,0){$\blacksquare$}}
\put(841,655){\makebox(0,0){$\blacksquare$}}
\put(781.0,472.0){\rule[-0.600pt]{1.200pt}{4.818pt}}
\sbox{\plotpoint}{\rule[-0.500pt]{1.000pt}{1.000pt}}%
\sbox{\plotpoint}{\rule[-0.200pt]{0.400pt}{0.400pt}}%
\put(771,614){\makebox(0,0)[r]{$2\rightarrow1$ pre \uv{najmenšiu} pozíciu}}
\sbox{\plotpoint}{\rule[-0.500pt]{1.000pt}{1.000pt}}%
\multiput(791,614)(41.511,0.000){3}{\usebox{\plotpoint}}
\put(891,614){\usebox{\plotpoint}}
\put(791.00,624.00){\usebox{\plotpoint}}
\put(791,604){\usebox{\plotpoint}}
\put(891.00,624.00){\usebox{\plotpoint}}
\put(891,604){\usebox{\plotpoint}}
\multiput(376,305)(0.000,41.511){3}{\usebox{\plotpoint}}
\put(376,398){\usebox{\plotpoint}}
\put(366.00,305.00){\usebox{\plotpoint}}
\put(386,305){\usebox{\plotpoint}}
\put(366.00,398.00){\usebox{\plotpoint}}
\put(386,398){\usebox{\plotpoint}}
\multiput(411,302)(0.000,41.511){3}{\usebox{\plotpoint}}
\put(411,394){\usebox{\plotpoint}}
\put(401.00,302.00){\usebox{\plotpoint}}
\put(421,302){\usebox{\plotpoint}}
\put(401.00,394.00){\usebox{\plotpoint}}
\put(421,394){\usebox{\plotpoint}}
\multiput(396,298)(0.000,41.511){3}{\usebox{\plotpoint}}
\put(396,390){\usebox{\plotpoint}}
\put(386.00,298.00){\usebox{\plotpoint}}
\put(406,298){\usebox{\plotpoint}}
\put(386.00,390.00){\usebox{\plotpoint}}
\put(406,390){\usebox{\plotpoint}}
\multiput(300,178)(0.000,41.511){3}{\usebox{\plotpoint}}
\put(300,271){\usebox{\plotpoint}}
\put(290.00,178.00){\usebox{\plotpoint}}
\put(310,178){\usebox{\plotpoint}}
\put(290.00,271.00){\usebox{\plotpoint}}
\put(310,271){\usebox{\plotpoint}}
\multiput(406,325)(0.000,41.511){3}{\usebox{\plotpoint}}
\put(406,418){\usebox{\plotpoint}}
\put(396.00,325.00){\usebox{\plotpoint}}
\put(416,325){\usebox{\plotpoint}}
\put(396.00,418.00){\usebox{\plotpoint}}
\put(416,418){\usebox{\plotpoint}}
\multiput(408,331)(0.000,41.511){3}{\usebox{\plotpoint}}
\put(408,424){\usebox{\plotpoint}}
\put(398.00,331.00){\usebox{\plotpoint}}
\put(418,331){\usebox{\plotpoint}}
\put(398.00,424.00){\usebox{\plotpoint}}
\put(418,424){\usebox{\plotpoint}}
\multiput(404,308)(0.000,41.511){3}{\usebox{\plotpoint}}
\put(404,401){\usebox{\plotpoint}}
\put(394.00,308.00){\usebox{\plotpoint}}
\put(414,308){\usebox{\plotpoint}}
\put(394.00,401.00){\usebox{\plotpoint}}
\put(414,401){\usebox{\plotpoint}}
\multiput(413,331)(0.000,41.511){3}{\usebox{\plotpoint}}
\put(413,423){\usebox{\plotpoint}}
\put(403.00,331.00){\usebox{\plotpoint}}
\put(423,331){\usebox{\plotpoint}}
\put(403.00,423.00){\usebox{\plotpoint}}
\put(423,423){\usebox{\plotpoint}}
\multiput(391,340)(0.000,41.511){3}{\usebox{\plotpoint}}
\put(391,433){\usebox{\plotpoint}}
\put(381.00,340.00){\usebox{\plotpoint}}
\put(401,340){\usebox{\plotpoint}}
\put(381.00,433.00){\usebox{\plotpoint}}
\put(401,433){\usebox{\plotpoint}}
\multiput(370,287)(0.000,41.511){3}{\usebox{\plotpoint}}
\put(370,380){\usebox{\plotpoint}}
\put(360.00,287.00){\usebox{\plotpoint}}
\put(380,287){\usebox{\plotpoint}}
\put(360.00,380.00){\usebox{\plotpoint}}
\put(380,380){\usebox{\plotpoint}}
\multiput(342,351)(41.511,0.000){2}{\usebox{\plotpoint}}
\put(410,351){\usebox{\plotpoint}}
\put(342.00,341.00){\usebox{\plotpoint}}
\put(342,361){\usebox{\plotpoint}}
\put(410.00,341.00){\usebox{\plotpoint}}
\put(410,361){\usebox{\plotpoint}}
\multiput(377,348)(41.511,0.000){2}{\usebox{\plotpoint}}
\put(445,348){\usebox{\plotpoint}}
\put(377.00,338.00){\usebox{\plotpoint}}
\put(377,358){\usebox{\plotpoint}}
\put(445.00,338.00){\usebox{\plotpoint}}
\put(445,358){\usebox{\plotpoint}}
\multiput(362,344)(41.511,0.000){2}{\usebox{\plotpoint}}
\put(430,344){\usebox{\plotpoint}}
\put(362.00,334.00){\usebox{\plotpoint}}
\put(362,354){\usebox{\plotpoint}}
\put(430.00,334.00){\usebox{\plotpoint}}
\put(430,354){\usebox{\plotpoint}}
\multiput(266,224)(41.511,0.000){2}{\usebox{\plotpoint}}
\put(334,224){\usebox{\plotpoint}}
\put(266.00,214.00){\usebox{\plotpoint}}
\put(266,234){\usebox{\plotpoint}}
\put(334.00,214.00){\usebox{\plotpoint}}
\put(334,234){\usebox{\plotpoint}}
\multiput(372,371)(41.511,0.000){2}{\usebox{\plotpoint}}
\put(440,371){\usebox{\plotpoint}}
\put(372.00,361.00){\usebox{\plotpoint}}
\put(372,381){\usebox{\plotpoint}}
\put(440.00,361.00){\usebox{\plotpoint}}
\put(440,381){\usebox{\plotpoint}}
\multiput(374,378)(41.511,0.000){2}{\usebox{\plotpoint}}
\put(442,378){\usebox{\plotpoint}}
\put(374.00,368.00){\usebox{\plotpoint}}
\put(374,388){\usebox{\plotpoint}}
\put(442.00,368.00){\usebox{\plotpoint}}
\put(442,388){\usebox{\plotpoint}}
\multiput(370,355)(41.511,0.000){2}{\usebox{\plotpoint}}
\put(438,355){\usebox{\plotpoint}}
\put(370.00,345.00){\usebox{\plotpoint}}
\put(370,365){\usebox{\plotpoint}}
\put(438.00,345.00){\usebox{\plotpoint}}
\put(438,365){\usebox{\plotpoint}}
\multiput(379,377)(41.511,0.000){2}{\usebox{\plotpoint}}
\put(447,377){\usebox{\plotpoint}}
\put(379.00,367.00){\usebox{\plotpoint}}
\put(379,387){\usebox{\plotpoint}}
\put(447.00,367.00){\usebox{\plotpoint}}
\put(447,387){\usebox{\plotpoint}}
\multiput(357,387)(41.511,0.000){2}{\usebox{\plotpoint}}
\put(425,387){\usebox{\plotpoint}}
\put(357.00,377.00){\usebox{\plotpoint}}
\put(357,397){\usebox{\plotpoint}}
\put(425.00,377.00){\usebox{\plotpoint}}
\put(425,397){\usebox{\plotpoint}}
\multiput(336,334)(41.511,0.000){2}{\usebox{\plotpoint}}
\put(404,334){\usebox{\plotpoint}}
\put(336.00,324.00){\usebox{\plotpoint}}
\put(336,344){\usebox{\plotpoint}}
\put(404.00,324.00){\usebox{\plotpoint}}
\put(404,344){\usebox{\plotpoint}}
\put(376,351){\makebox(0,0){$\circ$}}
\put(411,348){\makebox(0,0){$\circ$}}
\put(396,344){\makebox(0,0){$\circ$}}
\put(300,224){\makebox(0,0){$\circ$}}
\put(406,371){\makebox(0,0){$\circ$}}
\put(408,378){\makebox(0,0){$\circ$}}
\put(404,355){\makebox(0,0){$\circ$}}
\put(413,377){\makebox(0,0){$\circ$}}
\put(391,387){\makebox(0,0){$\circ$}}
\put(370,334){\makebox(0,0){$\circ$}}
\put(841,614){\makebox(0,0){$\circ$}}
\sbox{\plotpoint}{\rule[-0.200pt]{0.400pt}{0.400pt}}%
\put(771,573){\makebox(0,0)[r]{teoretická závyslosť}}
\put(791.0,573.0){\rule[-0.200pt]{24.090pt}{0.400pt}}
\put(266,261){\usebox{\plotpoint}}
\multiput(266.00,261.59)(1.155,0.477){7}{\rule{0.980pt}{0.115pt}}
\multiput(266.00,260.17)(8.966,5.000){2}{\rule{0.490pt}{0.400pt}}
\multiput(277.00,266.59)(1.044,0.477){7}{\rule{0.900pt}{0.115pt}}
\multiput(277.00,265.17)(8.132,5.000){2}{\rule{0.450pt}{0.400pt}}
\multiput(287.00,271.59)(0.943,0.482){9}{\rule{0.833pt}{0.116pt}}
\multiput(287.00,270.17)(9.270,6.000){2}{\rule{0.417pt}{0.400pt}}
\multiput(298.00,277.59)(1.044,0.477){7}{\rule{0.900pt}{0.115pt}}
\multiput(298.00,276.17)(8.132,5.000){2}{\rule{0.450pt}{0.400pt}}
\multiput(308.00,282.59)(1.155,0.477){7}{\rule{0.980pt}{0.115pt}}
\multiput(308.00,281.17)(8.966,5.000){2}{\rule{0.490pt}{0.400pt}}
\multiput(319.00,287.59)(0.852,0.482){9}{\rule{0.767pt}{0.116pt}}
\multiput(319.00,286.17)(8.409,6.000){2}{\rule{0.383pt}{0.400pt}}
\multiput(329.00,293.59)(1.155,0.477){7}{\rule{0.980pt}{0.115pt}}
\multiput(329.00,292.17)(8.966,5.000){2}{\rule{0.490pt}{0.400pt}}
\multiput(340.00,298.59)(1.044,0.477){7}{\rule{0.900pt}{0.115pt}}
\multiput(340.00,297.17)(8.132,5.000){2}{\rule{0.450pt}{0.400pt}}
\multiput(350.00,303.59)(0.943,0.482){9}{\rule{0.833pt}{0.116pt}}
\multiput(350.00,302.17)(9.270,6.000){2}{\rule{0.417pt}{0.400pt}}
\multiput(361.00,309.59)(1.044,0.477){7}{\rule{0.900pt}{0.115pt}}
\multiput(361.00,308.17)(8.132,5.000){2}{\rule{0.450pt}{0.400pt}}
\multiput(371.00,314.59)(0.943,0.482){9}{\rule{0.833pt}{0.116pt}}
\multiput(371.00,313.17)(9.270,6.000){2}{\rule{0.417pt}{0.400pt}}
\multiput(382.00,320.59)(1.155,0.477){7}{\rule{0.980pt}{0.115pt}}
\multiput(382.00,319.17)(8.966,5.000){2}{\rule{0.490pt}{0.400pt}}
\multiput(393.00,325.59)(1.044,0.477){7}{\rule{0.900pt}{0.115pt}}
\multiput(393.00,324.17)(8.132,5.000){2}{\rule{0.450pt}{0.400pt}}
\multiput(403.00,330.59)(0.943,0.482){9}{\rule{0.833pt}{0.116pt}}
\multiput(403.00,329.17)(9.270,6.000){2}{\rule{0.417pt}{0.400pt}}
\multiput(414.00,336.59)(1.044,0.477){7}{\rule{0.900pt}{0.115pt}}
\multiput(414.00,335.17)(8.132,5.000){2}{\rule{0.450pt}{0.400pt}}
\multiput(424.00,341.59)(1.155,0.477){7}{\rule{0.980pt}{0.115pt}}
\multiput(424.00,340.17)(8.966,5.000){2}{\rule{0.490pt}{0.400pt}}
\multiput(435.00,346.59)(0.852,0.482){9}{\rule{0.767pt}{0.116pt}}
\multiput(435.00,345.17)(8.409,6.000){2}{\rule{0.383pt}{0.400pt}}
\multiput(445.00,352.59)(1.155,0.477){7}{\rule{0.980pt}{0.115pt}}
\multiput(445.00,351.17)(8.966,5.000){2}{\rule{0.490pt}{0.400pt}}
\multiput(456.00,357.59)(1.044,0.477){7}{\rule{0.900pt}{0.115pt}}
\multiput(456.00,356.17)(8.132,5.000){2}{\rule{0.450pt}{0.400pt}}
\multiput(466.00,362.59)(0.943,0.482){9}{\rule{0.833pt}{0.116pt}}
\multiput(466.00,361.17)(9.270,6.000){2}{\rule{0.417pt}{0.400pt}}
\multiput(477.00,368.59)(1.044,0.477){7}{\rule{0.900pt}{0.115pt}}
\multiput(477.00,367.17)(8.132,5.000){2}{\rule{0.450pt}{0.400pt}}
\multiput(487.00,373.59)(0.943,0.482){9}{\rule{0.833pt}{0.116pt}}
\multiput(487.00,372.17)(9.270,6.000){2}{\rule{0.417pt}{0.400pt}}
\multiput(498.00,379.59)(1.044,0.477){7}{\rule{0.900pt}{0.115pt}}
\multiput(498.00,378.17)(8.132,5.000){2}{\rule{0.450pt}{0.400pt}}
\multiput(508.00,384.59)(1.155,0.477){7}{\rule{0.980pt}{0.115pt}}
\multiput(508.00,383.17)(8.966,5.000){2}{\rule{0.490pt}{0.400pt}}
\multiput(519.00,389.59)(0.852,0.482){9}{\rule{0.767pt}{0.116pt}}
\multiput(519.00,388.17)(8.409,6.000){2}{\rule{0.383pt}{0.400pt}}
\multiput(529.00,395.59)(1.155,0.477){7}{\rule{0.980pt}{0.115pt}}
\multiput(529.00,394.17)(8.966,5.000){2}{\rule{0.490pt}{0.400pt}}
\multiput(540.00,400.59)(1.044,0.477){7}{\rule{0.900pt}{0.115pt}}
\multiput(540.00,399.17)(8.132,5.000){2}{\rule{0.450pt}{0.400pt}}
\multiput(550.00,405.59)(0.943,0.482){9}{\rule{0.833pt}{0.116pt}}
\multiput(550.00,404.17)(9.270,6.000){2}{\rule{0.417pt}{0.400pt}}
\multiput(561.00,411.59)(1.044,0.477){7}{\rule{0.900pt}{0.115pt}}
\multiput(561.00,410.17)(8.132,5.000){2}{\rule{0.450pt}{0.400pt}}
\multiput(571.00,416.59)(0.943,0.482){9}{\rule{0.833pt}{0.116pt}}
\multiput(571.00,415.17)(9.270,6.000){2}{\rule{0.417pt}{0.400pt}}
\multiput(582.00,422.59)(1.044,0.477){7}{\rule{0.900pt}{0.115pt}}
\multiput(582.00,421.17)(8.132,5.000){2}{\rule{0.450pt}{0.400pt}}
\multiput(592.00,427.59)(1.155,0.477){7}{\rule{0.980pt}{0.115pt}}
\multiput(592.00,426.17)(8.966,5.000){2}{\rule{0.490pt}{0.400pt}}
\multiput(603.00,432.59)(0.852,0.482){9}{\rule{0.767pt}{0.116pt}}
\multiput(603.00,431.17)(8.409,6.000){2}{\rule{0.383pt}{0.400pt}}
\multiput(613.00,438.59)(1.155,0.477){7}{\rule{0.980pt}{0.115pt}}
\multiput(613.00,437.17)(8.966,5.000){2}{\rule{0.490pt}{0.400pt}}
\multiput(624.00,443.59)(1.044,0.477){7}{\rule{0.900pt}{0.115pt}}
\multiput(624.00,442.17)(8.132,5.000){2}{\rule{0.450pt}{0.400pt}}
\multiput(634.00,448.59)(0.943,0.482){9}{\rule{0.833pt}{0.116pt}}
\multiput(634.00,447.17)(9.270,6.000){2}{\rule{0.417pt}{0.400pt}}
\multiput(645.00,454.59)(1.044,0.477){7}{\rule{0.900pt}{0.115pt}}
\multiput(645.00,453.17)(8.132,5.000){2}{\rule{0.450pt}{0.400pt}}
\multiput(655.00,459.59)(1.155,0.477){7}{\rule{0.980pt}{0.115pt}}
\multiput(655.00,458.17)(8.966,5.000){2}{\rule{0.490pt}{0.400pt}}
\multiput(666.00,464.59)(0.852,0.482){9}{\rule{0.767pt}{0.116pt}}
\multiput(666.00,463.17)(8.409,6.000){2}{\rule{0.383pt}{0.400pt}}
\multiput(676.00,470.59)(1.155,0.477){7}{\rule{0.980pt}{0.115pt}}
\multiput(676.00,469.17)(8.966,5.000){2}{\rule{0.490pt}{0.400pt}}
\multiput(687.00,475.59)(0.852,0.482){9}{\rule{0.767pt}{0.116pt}}
\multiput(687.00,474.17)(8.409,6.000){2}{\rule{0.383pt}{0.400pt}}
\multiput(697.00,481.59)(1.155,0.477){7}{\rule{0.980pt}{0.115pt}}
\multiput(697.00,480.17)(8.966,5.000){2}{\rule{0.490pt}{0.400pt}}
\multiput(708.00,486.59)(1.044,0.477){7}{\rule{0.900pt}{0.115pt}}
\multiput(708.00,485.17)(8.132,5.000){2}{\rule{0.450pt}{0.400pt}}
\multiput(718.00,491.59)(0.943,0.482){9}{\rule{0.833pt}{0.116pt}}
\multiput(718.00,490.17)(9.270,6.000){2}{\rule{0.417pt}{0.400pt}}
\multiput(729.00,497.59)(1.044,0.477){7}{\rule{0.900pt}{0.115pt}}
\multiput(729.00,496.17)(8.132,5.000){2}{\rule{0.450pt}{0.400pt}}
\multiput(739.00,502.59)(1.155,0.477){7}{\rule{0.980pt}{0.115pt}}
\multiput(739.00,501.17)(8.966,5.000){2}{\rule{0.490pt}{0.400pt}}
\multiput(750.00,507.59)(0.852,0.482){9}{\rule{0.767pt}{0.116pt}}
\multiput(750.00,506.17)(8.409,6.000){2}{\rule{0.383pt}{0.400pt}}
\multiput(760.00,513.59)(1.155,0.477){7}{\rule{0.980pt}{0.115pt}}
\multiput(760.00,512.17)(8.966,5.000){2}{\rule{0.490pt}{0.400pt}}
\multiput(771.00,518.59)(1.155,0.477){7}{\rule{0.980pt}{0.115pt}}
\multiput(771.00,517.17)(8.966,5.000){2}{\rule{0.490pt}{0.400pt}}
\multiput(782.00,523.59)(0.852,0.482){9}{\rule{0.767pt}{0.116pt}}
\multiput(782.00,522.17)(8.409,6.000){2}{\rule{0.383pt}{0.400pt}}
\multiput(792.00,529.59)(1.155,0.477){7}{\rule{0.980pt}{0.115pt}}
\multiput(792.00,528.17)(8.966,5.000){2}{\rule{0.490pt}{0.400pt}}
\multiput(803.00,534.59)(0.852,0.482){9}{\rule{0.767pt}{0.116pt}}
\multiput(803.00,533.17)(8.409,6.000){2}{\rule{0.383pt}{0.400pt}}
\multiput(813.00,540.59)(1.155,0.477){7}{\rule{0.980pt}{0.115pt}}
\multiput(813.00,539.17)(8.966,5.000){2}{\rule{0.490pt}{0.400pt}}
\multiput(824.00,545.59)(1.044,0.477){7}{\rule{0.900pt}{0.115pt}}
\multiput(824.00,544.17)(8.132,5.000){2}{\rule{0.450pt}{0.400pt}}
\multiput(834.00,550.59)(0.943,0.482){9}{\rule{0.833pt}{0.116pt}}
\multiput(834.00,549.17)(9.270,6.000){2}{\rule{0.417pt}{0.400pt}}
\multiput(845.00,556.59)(1.044,0.477){7}{\rule{0.900pt}{0.115pt}}
\multiput(845.00,555.17)(8.132,5.000){2}{\rule{0.450pt}{0.400pt}}
\multiput(855.00,561.59)(1.155,0.477){7}{\rule{0.980pt}{0.115pt}}
\multiput(855.00,560.17)(8.966,5.000){2}{\rule{0.490pt}{0.400pt}}
\multiput(866.00,566.59)(0.852,0.482){9}{\rule{0.767pt}{0.116pt}}
\multiput(866.00,565.17)(8.409,6.000){2}{\rule{0.383pt}{0.400pt}}
\multiput(876.00,572.59)(1.155,0.477){7}{\rule{0.980pt}{0.115pt}}
\multiput(876.00,571.17)(8.966,5.000){2}{\rule{0.490pt}{0.400pt}}
\multiput(887.00,577.59)(1.044,0.477){7}{\rule{0.900pt}{0.115pt}}
\multiput(887.00,576.17)(8.132,5.000){2}{\rule{0.450pt}{0.400pt}}
\multiput(897.00,582.59)(0.943,0.482){9}{\rule{0.833pt}{0.116pt}}
\multiput(897.00,581.17)(9.270,6.000){2}{\rule{0.417pt}{0.400pt}}
\multiput(908.00,588.59)(1.044,0.477){7}{\rule{0.900pt}{0.115pt}}
\multiput(908.00,587.17)(8.132,5.000){2}{\rule{0.450pt}{0.400pt}}
\multiput(918.00,593.59)(0.943,0.482){9}{\rule{0.833pt}{0.116pt}}
\multiput(918.00,592.17)(9.270,6.000){2}{\rule{0.417pt}{0.400pt}}
\multiput(929.00,599.59)(1.044,0.477){7}{\rule{0.900pt}{0.115pt}}
\multiput(929.00,598.17)(8.132,5.000){2}{\rule{0.450pt}{0.400pt}}
\multiput(939.00,604.59)(1.155,0.477){7}{\rule{0.980pt}{0.115pt}}
\multiput(939.00,603.17)(8.966,5.000){2}{\rule{0.490pt}{0.400pt}}
\multiput(950.00,609.59)(0.852,0.482){9}{\rule{0.767pt}{0.116pt}}
\multiput(950.00,608.17)(8.409,6.000){2}{\rule{0.383pt}{0.400pt}}
\multiput(960.00,615.59)(1.155,0.477){7}{\rule{0.980pt}{0.115pt}}
\multiput(960.00,614.17)(8.966,5.000){2}{\rule{0.490pt}{0.400pt}}
\multiput(971.00,620.59)(1.044,0.477){7}{\rule{0.900pt}{0.115pt}}
\multiput(971.00,619.17)(8.132,5.000){2}{\rule{0.450pt}{0.400pt}}
\multiput(981.00,625.59)(0.943,0.482){9}{\rule{0.833pt}{0.116pt}}
\multiput(981.00,624.17)(9.270,6.000){2}{\rule{0.417pt}{0.400pt}}
\multiput(992.00,631.59)(1.044,0.477){7}{\rule{0.900pt}{0.115pt}}
\multiput(992.00,630.17)(8.132,5.000){2}{\rule{0.450pt}{0.400pt}}
\multiput(1002.00,636.59)(0.943,0.482){9}{\rule{0.833pt}{0.116pt}}
\multiput(1002.00,635.17)(9.270,6.000){2}{\rule{0.417pt}{0.400pt}}
\multiput(1013.00,642.59)(1.044,0.477){7}{\rule{0.900pt}{0.115pt}}
\multiput(1013.00,641.17)(8.132,5.000){2}{\rule{0.450pt}{0.400pt}}
\multiput(1023.00,647.59)(1.155,0.477){7}{\rule{0.980pt}{0.115pt}}
\multiput(1023.00,646.17)(8.966,5.000){2}{\rule{0.490pt}{0.400pt}}
\multiput(1034.00,652.59)(0.852,0.482){9}{\rule{0.767pt}{0.116pt}}
\multiput(1034.00,651.17)(8.409,6.000){2}{\rule{0.383pt}{0.400pt}}
\multiput(1044.00,658.59)(1.155,0.477){7}{\rule{0.980pt}{0.115pt}}
\multiput(1044.00,657.17)(8.966,5.000){2}{\rule{0.490pt}{0.400pt}}
\multiput(1055.00,663.59)(1.044,0.477){7}{\rule{0.900pt}{0.115pt}}
\multiput(1055.00,662.17)(8.132,5.000){2}{\rule{0.450pt}{0.400pt}}
\multiput(1065.00,668.59)(0.943,0.482){9}{\rule{0.833pt}{0.116pt}}
\multiput(1065.00,667.17)(9.270,6.000){2}{\rule{0.417pt}{0.400pt}}
\multiput(1076.00,674.59)(1.044,0.477){7}{\rule{0.900pt}{0.115pt}}
\multiput(1076.00,673.17)(8.132,5.000){2}{\rule{0.450pt}{0.400pt}}
\multiput(1086.00,679.59)(1.155,0.477){7}{\rule{0.980pt}{0.115pt}}
\multiput(1086.00,678.17)(8.966,5.000){2}{\rule{0.490pt}{0.400pt}}
\multiput(1097.00,684.59)(0.852,0.482){9}{\rule{0.767pt}{0.116pt}}
\multiput(1097.00,683.17)(8.409,6.000){2}{\rule{0.383pt}{0.400pt}}
\multiput(1107.00,690.59)(1.155,0.477){7}{\rule{0.980pt}{0.115pt}}
\multiput(1107.00,689.17)(8.966,5.000){2}{\rule{0.490pt}{0.400pt}}
\multiput(1118.00,695.59)(0.852,0.482){9}{\rule{0.767pt}{0.116pt}}
\multiput(1118.00,694.17)(8.409,6.000){2}{\rule{0.383pt}{0.400pt}}
\multiput(1128.00,701.59)(1.155,0.477){7}{\rule{0.980pt}{0.115pt}}
\multiput(1128.00,700.17)(8.966,5.000){2}{\rule{0.490pt}{0.400pt}}
\multiput(1139.00,706.59)(1.044,0.477){7}{\rule{0.900pt}{0.115pt}}
\multiput(1139.00,705.17)(8.132,5.000){2}{\rule{0.450pt}{0.400pt}}
\multiput(1149.00,711.59)(0.943,0.482){9}{\rule{0.833pt}{0.116pt}}
\multiput(1149.00,710.17)(9.270,6.000){2}{\rule{0.417pt}{0.400pt}}
\multiput(1160.00,717.59)(1.155,0.477){7}{\rule{0.980pt}{0.115pt}}
\multiput(1160.00,716.17)(8.966,5.000){2}{\rule{0.490pt}{0.400pt}}
\multiput(1171.00,722.59)(1.044,0.477){7}{\rule{0.900pt}{0.115pt}}
\multiput(1171.00,721.17)(8.132,5.000){2}{\rule{0.450pt}{0.400pt}}
\multiput(1181.00,727.59)(0.943,0.482){9}{\rule{0.833pt}{0.116pt}}
\multiput(1181.00,726.17)(9.270,6.000){2}{\rule{0.417pt}{0.400pt}}
\multiput(1192.00,733.59)(1.044,0.477){7}{\rule{0.900pt}{0.115pt}}
\multiput(1192.00,732.17)(8.132,5.000){2}{\rule{0.450pt}{0.400pt}}
\multiput(1202.00,738.59)(1.155,0.477){7}{\rule{0.980pt}{0.115pt}}
\multiput(1202.00,737.17)(8.966,5.000){2}{\rule{0.490pt}{0.400pt}}
\multiput(1213.00,743.59)(0.852,0.482){9}{\rule{0.767pt}{0.116pt}}
\multiput(1213.00,742.17)(8.409,6.000){2}{\rule{0.383pt}{0.400pt}}
\multiput(1223.00,749.59)(1.155,0.477){7}{\rule{0.980pt}{0.115pt}}
\multiput(1223.00,748.17)(8.966,5.000){2}{\rule{0.490pt}{0.400pt}}
\multiput(1234.00,754.59)(0.852,0.482){9}{\rule{0.767pt}{0.116pt}}
\multiput(1234.00,753.17)(8.409,6.000){2}{\rule{0.383pt}{0.400pt}}
\multiput(1244.00,760.59)(1.155,0.477){7}{\rule{0.980pt}{0.115pt}}
\multiput(1244.00,759.17)(8.966,5.000){2}{\rule{0.490pt}{0.400pt}}
\multiput(1255.00,765.59)(1.044,0.477){7}{\rule{0.900pt}{0.115pt}}
\multiput(1255.00,764.17)(8.132,5.000){2}{\rule{0.450pt}{0.400pt}}
\multiput(1265.00,770.59)(0.943,0.482){9}{\rule{0.833pt}{0.116pt}}
\multiput(1265.00,769.17)(9.270,6.000){2}{\rule{0.417pt}{0.400pt}}
\multiput(1276.00,776.59)(1.044,0.477){7}{\rule{0.900pt}{0.115pt}}
\multiput(1276.00,775.17)(8.132,5.000){2}{\rule{0.450pt}{0.400pt}}
\multiput(1286.00,781.59)(1.155,0.477){7}{\rule{0.980pt}{0.115pt}}
\multiput(1286.00,780.17)(8.966,5.000){2}{\rule{0.490pt}{0.400pt}}
\multiput(1297.00,786.59)(0.852,0.482){9}{\rule{0.767pt}{0.116pt}}
\multiput(1297.00,785.17)(8.409,6.000){2}{\rule{0.383pt}{0.400pt}}
\put(171.0,131.0){\rule[-0.200pt]{0.400pt}{175.375pt}}
\put(171.0,131.0){\rule[-0.200pt]{305.461pt}{0.400pt}}
\put(1439.0,131.0){\rule[-0.200pt]{0.400pt}{175.375pt}}
\put(171.0,859.0){\rule[-0.200pt]{305.461pt}{0.400pt}}
\end{picture}

\caption{Závislosť hybnosti pred $p^B$ a hybnosti po zrážke $p^A$ pre jednotlivé kombinácie hmotností a štartovacích impulzov}  \label{G_1}
\end{figure}

\begin{figure}
% GNUPLOT: LaTeX picture
\setlength{\unitlength}{0.240900pt}
\ifx\plotpoint\undefined\newsavebox{\plotpoint}\fi
\begin{picture}(1500,900)(0,0)
\sbox{\plotpoint}{\rule[-0.200pt]{0.400pt}{0.400pt}}%
\put(151.0,131.0){\rule[-0.200pt]{4.818pt}{0.400pt}}
\put(131,131){\makebox(0,0)[r]{ 0}}
\put(1419.0,131.0){\rule[-0.200pt]{4.818pt}{0.400pt}}
\put(151.0,222.0){\rule[-0.200pt]{4.818pt}{0.400pt}}
\put(131,222){\makebox(0,0)[r]{ 10}}
\put(1419.0,222.0){\rule[-0.200pt]{4.818pt}{0.400pt}}
\put(151.0,313.0){\rule[-0.200pt]{4.818pt}{0.400pt}}
\put(131,313){\makebox(0,0)[r]{ 20}}
\put(1419.0,313.0){\rule[-0.200pt]{4.818pt}{0.400pt}}
\put(151.0,404.0){\rule[-0.200pt]{4.818pt}{0.400pt}}
\put(131,404){\makebox(0,0)[r]{ 30}}
\put(1419.0,404.0){\rule[-0.200pt]{4.818pt}{0.400pt}}
\put(151.0,495.0){\rule[-0.200pt]{4.818pt}{0.400pt}}
\put(131,495){\makebox(0,0)[r]{ 40}}
\put(1419.0,495.0){\rule[-0.200pt]{4.818pt}{0.400pt}}
\put(151.0,586.0){\rule[-0.200pt]{4.818pt}{0.400pt}}
\put(131,586){\makebox(0,0)[r]{ 50}}
\put(1419.0,586.0){\rule[-0.200pt]{4.818pt}{0.400pt}}
\put(151.0,677.0){\rule[-0.200pt]{4.818pt}{0.400pt}}
\put(131,677){\makebox(0,0)[r]{ 60}}
\put(1419.0,677.0){\rule[-0.200pt]{4.818pt}{0.400pt}}
\put(151.0,768.0){\rule[-0.200pt]{4.818pt}{0.400pt}}
\put(131,768){\makebox(0,0)[r]{ 70}}
\put(1419.0,768.0){\rule[-0.200pt]{4.818pt}{0.400pt}}
\put(151.0,859.0){\rule[-0.200pt]{4.818pt}{0.400pt}}
\put(131,859){\makebox(0,0)[r]{ 80}}
\put(1419.0,859.0){\rule[-0.200pt]{4.818pt}{0.400pt}}
\put(151.0,131.0){\rule[-0.200pt]{0.400pt}{4.818pt}}
\put(151,90){\makebox(0,0){ 0}}
\put(151.0,839.0){\rule[-0.200pt]{0.400pt}{4.818pt}}
\put(335.0,131.0){\rule[-0.200pt]{0.400pt}{4.818pt}}
\put(335,90){\makebox(0,0){ 10}}
\put(335.0,839.0){\rule[-0.200pt]{0.400pt}{4.818pt}}
\put(519.0,131.0){\rule[-0.200pt]{0.400pt}{4.818pt}}
\put(519,90){\makebox(0,0){ 20}}
\put(519.0,839.0){\rule[-0.200pt]{0.400pt}{4.818pt}}
\put(703.0,131.0){\rule[-0.200pt]{0.400pt}{4.818pt}}
\put(703,90){\makebox(0,0){ 30}}
\put(703.0,839.0){\rule[-0.200pt]{0.400pt}{4.818pt}}
\put(887.0,131.0){\rule[-0.200pt]{0.400pt}{4.818pt}}
\put(887,90){\makebox(0,0){ 40}}
\put(887.0,839.0){\rule[-0.200pt]{0.400pt}{4.818pt}}
\put(1071.0,131.0){\rule[-0.200pt]{0.400pt}{4.818pt}}
\put(1071,90){\makebox(0,0){ 50}}
\put(1071.0,839.0){\rule[-0.200pt]{0.400pt}{4.818pt}}
\put(1255.0,131.0){\rule[-0.200pt]{0.400pt}{4.818pt}}
\put(1255,90){\makebox(0,0){ 60}}
\put(1255.0,839.0){\rule[-0.200pt]{0.400pt}{4.818pt}}
\put(1439.0,131.0){\rule[-0.200pt]{0.400pt}{4.818pt}}
\put(1439,90){\makebox(0,0){ 70}}
\put(1439.0,839.0){\rule[-0.200pt]{0.400pt}{4.818pt}}
\put(151.0,131.0){\rule[-0.200pt]{0.400pt}{175.375pt}}
\put(151.0,131.0){\rule[-0.200pt]{310.279pt}{0.400pt}}
\put(1439.0,131.0){\rule[-0.200pt]{0.400pt}{175.375pt}}
\put(151.0,859.0){\rule[-0.200pt]{310.279pt}{0.400pt}}
\put(30,495){\makebox(0,0){\popi{E^A}{mJ}}}
\put(795,29){\makebox(0,0){\popi{E^B}{mJ}}}
\put(771,819){\makebox(0,0)[r]{$1\rightarrow2$ pre \uv{najvačšiu} pozíciu}}
\put(791.0,819.0){\rule[-0.200pt]{24.090pt}{0.400pt}}
\put(791.0,809.0){\rule[-0.200pt]{0.400pt}{4.818pt}}
\put(891.0,809.0){\rule[-0.200pt]{0.400pt}{4.818pt}}
\put(1222.0,514.0){\rule[-0.200pt]{0.400pt}{24.572pt}}
\put(1212.0,514.0){\rule[-0.200pt]{4.818pt}{0.400pt}}
\put(1212.0,616.0){\rule[-0.200pt]{4.818pt}{0.400pt}}
\put(864.0,394.0){\rule[-0.200pt]{0.400pt}{24.572pt}}
\put(854.0,394.0){\rule[-0.200pt]{4.818pt}{0.400pt}}
\put(854.0,496.0){\rule[-0.200pt]{4.818pt}{0.400pt}}
\put(967.0,419.0){\rule[-0.200pt]{0.400pt}{24.331pt}}
\put(957.0,419.0){\rule[-0.200pt]{4.818pt}{0.400pt}}
\put(957.0,520.0){\rule[-0.200pt]{4.818pt}{0.400pt}}
\put(1097.0,496.0){\rule[-0.200pt]{0.400pt}{24.572pt}}
\put(1087.0,496.0){\rule[-0.200pt]{4.818pt}{0.400pt}}
\put(1087.0,598.0){\rule[-0.200pt]{4.818pt}{0.400pt}}
\put(1077.0,474.0){\rule[-0.200pt]{0.400pt}{24.572pt}}
\put(1067.0,474.0){\rule[-0.200pt]{4.818pt}{0.400pt}}
\put(1067.0,576.0){\rule[-0.200pt]{4.818pt}{0.400pt}}
\put(918.0,382.0){\rule[-0.200pt]{0.400pt}{24.572pt}}
\put(908.0,382.0){\rule[-0.200pt]{4.818pt}{0.400pt}}
\put(908.0,484.0){\rule[-0.200pt]{4.818pt}{0.400pt}}
\put(1022.0,470.0){\rule[-0.200pt]{0.400pt}{24.572pt}}
\put(1012.0,470.0){\rule[-0.200pt]{4.818pt}{0.400pt}}
\put(1012.0,572.0){\rule[-0.200pt]{4.818pt}{0.400pt}}
\put(763.0,380.0){\rule[-0.200pt]{0.400pt}{24.572pt}}
\put(753.0,380.0){\rule[-0.200pt]{4.818pt}{0.400pt}}
\put(753.0,482.0){\rule[-0.200pt]{4.818pt}{0.400pt}}
\put(763.0,400.0){\rule[-0.200pt]{0.400pt}{24.572pt}}
\put(753.0,400.0){\rule[-0.200pt]{4.818pt}{0.400pt}}
\put(753.0,502.0){\rule[-0.200pt]{4.818pt}{0.400pt}}
\put(1106.0,463.0){\rule[-0.200pt]{0.400pt}{24.572pt}}
\put(1096.0,463.0){\rule[-0.200pt]{4.818pt}{0.400pt}}
\put(1096.0,565.0){\rule[-0.200pt]{4.818pt}{0.400pt}}
\put(1077.0,501.0){\rule[-0.200pt]{0.400pt}{24.572pt}}
\put(1067.0,501.0){\rule[-0.200pt]{4.818pt}{0.400pt}}
\put(1067.0,603.0){\rule[-0.200pt]{4.818pt}{0.400pt}}
\put(1074.0,565.0){\rule[-0.200pt]{71.065pt}{0.400pt}}
\put(1074.0,555.0){\rule[-0.200pt]{0.400pt}{4.818pt}}
\put(1369.0,555.0){\rule[-0.200pt]{0.400pt}{4.818pt}}
\put(717.0,445.0){\rule[-0.200pt]{70.825pt}{0.400pt}}
\put(717.0,435.0){\rule[-0.200pt]{0.400pt}{4.818pt}}
\put(1011.0,435.0){\rule[-0.200pt]{0.400pt}{4.818pt}}
\put(820.0,470.0){\rule[-0.200pt]{70.825pt}{0.400pt}}
\put(820.0,460.0){\rule[-0.200pt]{0.400pt}{4.818pt}}
\put(1114.0,460.0){\rule[-0.200pt]{0.400pt}{4.818pt}}
\put(950.0,547.0){\rule[-0.200pt]{70.825pt}{0.400pt}}
\put(950.0,537.0){\rule[-0.200pt]{0.400pt}{4.818pt}}
\put(1244.0,537.0){\rule[-0.200pt]{0.400pt}{4.818pt}}
\put(930.0,525.0){\rule[-0.200pt]{70.825pt}{0.400pt}}
\put(930.0,515.0){\rule[-0.200pt]{0.400pt}{4.818pt}}
\put(1224.0,515.0){\rule[-0.200pt]{0.400pt}{4.818pt}}
\put(771.0,433.0){\rule[-0.200pt]{71.065pt}{0.400pt}}
\put(771.0,423.0){\rule[-0.200pt]{0.400pt}{4.818pt}}
\put(1066.0,423.0){\rule[-0.200pt]{0.400pt}{4.818pt}}
\put(875.0,521.0){\rule[-0.200pt]{70.825pt}{0.400pt}}
\put(875.0,511.0){\rule[-0.200pt]{0.400pt}{4.818pt}}
\put(1169.0,511.0){\rule[-0.200pt]{0.400pt}{4.818pt}}
\put(616.0,431.0){\rule[-0.200pt]{70.825pt}{0.400pt}}
\put(616.0,421.0){\rule[-0.200pt]{0.400pt}{4.818pt}}
\put(910.0,421.0){\rule[-0.200pt]{0.400pt}{4.818pt}}
\put(616.0,451.0){\rule[-0.200pt]{70.825pt}{0.400pt}}
\put(616.0,441.0){\rule[-0.200pt]{0.400pt}{4.818pt}}
\put(910.0,441.0){\rule[-0.200pt]{0.400pt}{4.818pt}}
\put(959.0,514.0){\rule[-0.200pt]{70.825pt}{0.400pt}}
\put(959.0,504.0){\rule[-0.200pt]{0.400pt}{4.818pt}}
\put(1253.0,504.0){\rule[-0.200pt]{0.400pt}{4.818pt}}
\put(930.0,552.0){\rule[-0.200pt]{70.825pt}{0.400pt}}
\put(930.0,542.0){\rule[-0.200pt]{0.400pt}{4.818pt}}
\put(1222,565){\makebox(0,0){$+$}}
\put(864,445){\makebox(0,0){$+$}}
\put(967,470){\makebox(0,0){$+$}}
\put(1097,547){\makebox(0,0){$+$}}
\put(1077,525){\makebox(0,0){$+$}}
\put(918,433){\makebox(0,0){$+$}}
\put(1022,521){\makebox(0,0){$+$}}
\put(763,431){\makebox(0,0){$+$}}
\put(763,451){\makebox(0,0){$+$}}
\put(1106,514){\makebox(0,0){$+$}}
\put(1077,552){\makebox(0,0){$+$}}
\put(841,819){\makebox(0,0){$+$}}
\put(1224.0,542.0){\rule[-0.200pt]{0.400pt}{4.818pt}}
\put(771,778){\makebox(0,0)[r]{$1\rightarrow2$ pre \uv{strednú} pozíciu}}
\multiput(791,778)(20.756,0.000){5}{\usebox{\plotpoint}}
\put(891,778){\usebox{\plotpoint}}
\put(791.00,788.00){\usebox{\plotpoint}}
\put(791,768){\usebox{\plotpoint}}
\put(891.00,788.00){\usebox{\plotpoint}}
\put(891,768){\usebox{\plotpoint}}
\multiput(367,175)(0.000,20.756){5}{\usebox{\plotpoint}}
\put(367,277){\usebox{\plotpoint}}
\put(357.00,175.00){\usebox{\plotpoint}}
\put(377,175){\usebox{\plotpoint}}
\put(357.00,277.00){\usebox{\plotpoint}}
\put(377,277){\usebox{\plotpoint}}
\multiput(630,286)(0.000,20.756){5}{\usebox{\plotpoint}}
\put(630,388){\usebox{\plotpoint}}
\put(620.00,286.00){\usebox{\plotpoint}}
\put(640,286){\usebox{\plotpoint}}
\put(620.00,388.00){\usebox{\plotpoint}}
\put(640,388){\usebox{\plotpoint}}
\multiput(763,336)(0.000,20.756){5}{\usebox{\plotpoint}}
\put(763,438){\usebox{\plotpoint}}
\put(753.00,336.00){\usebox{\plotpoint}}
\put(773,336){\usebox{\plotpoint}}
\put(753.00,438.00){\usebox{\plotpoint}}
\put(773,438){\usebox{\plotpoint}}
\multiput(710,325)(0.000,20.756){5}{\usebox{\plotpoint}}
\put(710,427){\usebox{\plotpoint}}
\put(700.00,325.00){\usebox{\plotpoint}}
\put(720,325){\usebox{\plotpoint}}
\put(700.00,427.00){\usebox{\plotpoint}}
\put(720,427){\usebox{\plotpoint}}
\multiput(640,292)(0.000,20.756){5}{\usebox{\plotpoint}}
\put(640,394){\usebox{\plotpoint}}
\put(630.00,292.00){\usebox{\plotpoint}}
\put(650,292){\usebox{\plotpoint}}
\put(630.00,394.00){\usebox{\plotpoint}}
\put(650,394){\usebox{\plotpoint}}
\multiput(672,309)(0.000,20.756){5}{\usebox{\plotpoint}}
\put(672,411){\usebox{\plotpoint}}
\put(662.00,309.00){\usebox{\plotpoint}}
\put(682,309){\usebox{\plotpoint}}
\put(662.00,411.00){\usebox{\plotpoint}}
\put(682,411){\usebox{\plotpoint}}
\multiput(452,214)(0.000,20.756){5}{\usebox{\plotpoint}}
\put(452,316){\usebox{\plotpoint}}
\put(442.00,214.00){\usebox{\plotpoint}}
\put(462,214){\usebox{\plotpoint}}
\put(442.00,316.00){\usebox{\plotpoint}}
\put(462,316){\usebox{\plotpoint}}
\multiput(692,315)(0.000,20.756){5}{\usebox{\plotpoint}}
\put(692,416){\usebox{\plotpoint}}
\put(682.00,315.00){\usebox{\plotpoint}}
\put(702,315){\usebox{\plotpoint}}
\put(682.00,416.00){\usebox{\plotpoint}}
\put(702,416){\usebox{\plotpoint}}
\multiput(681,318)(0.000,20.756){5}{\usebox{\plotpoint}}
\put(681,420){\usebox{\plotpoint}}
\put(671.00,318.00){\usebox{\plotpoint}}
\put(691,318){\usebox{\plotpoint}}
\put(671.00,420.00){\usebox{\plotpoint}}
\put(691,420){\usebox{\plotpoint}}
\multiput(636,290)(0.000,20.756){5}{\usebox{\plotpoint}}
\put(636,392){\usebox{\plotpoint}}
\put(626.00,290.00){\usebox{\plotpoint}}
\put(646,290){\usebox{\plotpoint}}
\put(626.00,392.00){\usebox{\plotpoint}}
\put(646,392){\usebox{\plotpoint}}
\multiput(724,334)(0.000,20.756){5}{\usebox{\plotpoint}}
\put(724,436){\usebox{\plotpoint}}
\put(714.00,334.00){\usebox{\plotpoint}}
\put(734,334){\usebox{\plotpoint}}
\put(714.00,436.00){\usebox{\plotpoint}}
\put(734,436){\usebox{\plotpoint}}
\multiput(250,226)(20.756,0.000){12}{\usebox{\plotpoint}}
\put(485,226){\usebox{\plotpoint}}
\put(250.00,216.00){\usebox{\plotpoint}}
\put(250,236){\usebox{\plotpoint}}
\put(485.00,216.00){\usebox{\plotpoint}}
\put(485,236){\usebox{\plotpoint}}
\multiput(512,337)(20.756,0.000){12}{\usebox{\plotpoint}}
\put(747,337){\usebox{\plotpoint}}
\put(512.00,327.00){\usebox{\plotpoint}}
\put(512,347){\usebox{\plotpoint}}
\put(747.00,327.00){\usebox{\plotpoint}}
\put(747,347){\usebox{\plotpoint}}
\multiput(645,387)(20.756,0.000){12}{\usebox{\plotpoint}}
\put(881,387){\usebox{\plotpoint}}
\put(645.00,377.00){\usebox{\plotpoint}}
\put(645,397){\usebox{\plotpoint}}
\put(881.00,377.00){\usebox{\plotpoint}}
\put(881,397){\usebox{\plotpoint}}
\multiput(592,376)(20.756,0.000){12}{\usebox{\plotpoint}}
\put(828,376){\usebox{\plotpoint}}
\put(592.00,366.00){\usebox{\plotpoint}}
\put(592,386){\usebox{\plotpoint}}
\put(828.00,366.00){\usebox{\plotpoint}}
\put(828,386){\usebox{\plotpoint}}
\multiput(522,343)(20.756,0.000){12}{\usebox{\plotpoint}}
\put(758,343){\usebox{\plotpoint}}
\put(522.00,333.00){\usebox{\plotpoint}}
\put(522,353){\usebox{\plotpoint}}
\put(758.00,333.00){\usebox{\plotpoint}}
\put(758,353){\usebox{\plotpoint}}
\multiput(555,360)(20.756,0.000){12}{\usebox{\plotpoint}}
\put(790,360){\usebox{\plotpoint}}
\put(555.00,350.00){\usebox{\plotpoint}}
\put(555,370){\usebox{\plotpoint}}
\put(790.00,350.00){\usebox{\plotpoint}}
\put(790,370){\usebox{\plotpoint}}
\multiput(334,265)(20.756,0.000){12}{\usebox{\plotpoint}}
\put(569,265){\usebox{\plotpoint}}
\put(334.00,255.00){\usebox{\plotpoint}}
\put(334,275){\usebox{\plotpoint}}
\put(569.00,255.00){\usebox{\plotpoint}}
\put(569,275){\usebox{\plotpoint}}
\multiput(574,366)(20.756,0.000){12}{\usebox{\plotpoint}}
\put(810,366){\usebox{\plotpoint}}
\put(574.00,356.00){\usebox{\plotpoint}}
\put(574,376){\usebox{\plotpoint}}
\put(810.00,356.00){\usebox{\plotpoint}}
\put(810,376){\usebox{\plotpoint}}
\multiput(563,369)(20.756,0.000){12}{\usebox{\plotpoint}}
\put(799,369){\usebox{\plotpoint}}
\put(563.00,359.00){\usebox{\plotpoint}}
\put(563,379){\usebox{\plotpoint}}
\put(799.00,359.00){\usebox{\plotpoint}}
\put(799,379){\usebox{\plotpoint}}
\multiput(518,341)(20.756,0.000){12}{\usebox{\plotpoint}}
\put(754,341){\usebox{\plotpoint}}
\put(518.00,331.00){\usebox{\plotpoint}}
\put(518,351){\usebox{\plotpoint}}
\put(754.00,331.00){\usebox{\plotpoint}}
\put(754,351){\usebox{\plotpoint}}
\multiput(606,385)(20.756,0.000){12}{\usebox{\plotpoint}}
\put(841,385){\usebox{\plotpoint}}
\put(606.00,375.00){\usebox{\plotpoint}}
\put(606,395){\usebox{\plotpoint}}
\put(841.00,375.00){\usebox{\plotpoint}}
\put(841,395){\usebox{\plotpoint}}
\put(367,226){\makebox(0,0){$\times$}}
\put(630,337){\makebox(0,0){$\times$}}
\put(763,387){\makebox(0,0){$\times$}}
\put(710,376){\makebox(0,0){$\times$}}
\put(640,343){\makebox(0,0){$\times$}}
\put(672,360){\makebox(0,0){$\times$}}
\put(452,265){\makebox(0,0){$\times$}}
\put(692,366){\makebox(0,0){$\times$}}
\put(681,369){\makebox(0,0){$\times$}}
\put(636,341){\makebox(0,0){$\times$}}
\put(724,385){\makebox(0,0){$\times$}}
\put(841,778){\makebox(0,0){$\times$}}
\sbox{\plotpoint}{\rule[-0.400pt]{0.800pt}{0.800pt}}%
\sbox{\plotpoint}{\rule[-0.200pt]{0.400pt}{0.400pt}}%
\put(771,737){\makebox(0,0)[r]{$1\rightarrow2$ pre \uv{najmenšiu} pozíciu}}
\sbox{\plotpoint}{\rule[-0.400pt]{0.800pt}{0.800pt}}%
\put(791.0,737.0){\rule[-0.400pt]{24.090pt}{0.800pt}}
\put(791.0,727.0){\rule[-0.400pt]{0.800pt}{4.818pt}}
\put(891.0,727.0){\rule[-0.400pt]{0.800pt}{4.818pt}}
\put(345.0,180.0){\rule[-0.400pt]{0.800pt}{5.782pt}}
\put(335.0,180.0){\rule[-0.400pt]{4.818pt}{0.800pt}}
\put(335.0,204.0){\rule[-0.400pt]{4.818pt}{0.800pt}}
\put(295.0,181.0){\rule[-0.400pt]{0.800pt}{5.782pt}}
\put(285.0,181.0){\rule[-0.400pt]{4.818pt}{0.800pt}}
\put(285.0,205.0){\rule[-0.400pt]{4.818pt}{0.800pt}}
\put(359.0,177.0){\rule[-0.400pt]{0.800pt}{5.782pt}}
\put(349.0,177.0){\rule[-0.400pt]{4.818pt}{0.800pt}}
\put(349.0,201.0){\rule[-0.400pt]{4.818pt}{0.800pt}}
\put(335.0,207.0){\rule[-0.400pt]{0.800pt}{5.782pt}}
\put(325.0,207.0){\rule[-0.400pt]{4.818pt}{0.800pt}}
\put(325.0,231.0){\rule[-0.400pt]{4.818pt}{0.800pt}}
\put(316.0,192.0){\rule[-0.400pt]{0.800pt}{5.541pt}}
\put(306.0,192.0){\rule[-0.400pt]{4.818pt}{0.800pt}}
\put(306.0,215.0){\rule[-0.400pt]{4.818pt}{0.800pt}}
\put(358.0,212.0){\rule[-0.400pt]{0.800pt}{5.782pt}}
\put(348.0,212.0){\rule[-0.400pt]{4.818pt}{0.800pt}}
\put(348.0,236.0){\rule[-0.400pt]{4.818pt}{0.800pt}}
\put(324.0,198.0){\rule[-0.400pt]{0.800pt}{5.541pt}}
\put(314.0,198.0){\rule[-0.400pt]{4.818pt}{0.800pt}}
\put(314.0,221.0){\rule[-0.400pt]{4.818pt}{0.800pt}}
\put(310.0,178.0){\rule[-0.400pt]{0.800pt}{5.541pt}}
\put(300.0,178.0){\rule[-0.400pt]{4.818pt}{0.800pt}}
\put(300.0,201.0){\rule[-0.400pt]{4.818pt}{0.800pt}}
\put(326.0,184.0){\rule[-0.400pt]{0.800pt}{5.541pt}}
\put(316.0,184.0){\rule[-0.400pt]{4.818pt}{0.800pt}}
\put(316.0,207.0){\rule[-0.400pt]{4.818pt}{0.800pt}}
\put(289.0,182.0){\rule[-0.400pt]{0.800pt}{5.782pt}}
\put(279.0,182.0){\rule[-0.400pt]{4.818pt}{0.800pt}}
\put(279.0,206.0){\rule[-0.400pt]{4.818pt}{0.800pt}}
\put(304.0,189.0){\rule[-0.400pt]{0.800pt}{5.782pt}}
\put(294.0,189.0){\rule[-0.400pt]{4.818pt}{0.800pt}}
\put(294.0,213.0){\rule[-0.400pt]{4.818pt}{0.800pt}}
\put(323.0,192.0){\rule[-0.400pt]{10.840pt}{0.800pt}}
\put(323.0,182.0){\rule[-0.400pt]{0.800pt}{4.818pt}}
\put(368.0,182.0){\rule[-0.400pt]{0.800pt}{4.818pt}}
\put(273.0,193.0){\rule[-0.400pt]{10.600pt}{0.800pt}}
\put(273.0,183.0){\rule[-0.400pt]{0.800pt}{4.818pt}}
\put(317.0,183.0){\rule[-0.400pt]{0.800pt}{4.818pt}}
\put(337.0,189.0){\rule[-0.400pt]{10.600pt}{0.800pt}}
\put(337.0,179.0){\rule[-0.400pt]{0.800pt}{4.818pt}}
\put(381.0,179.0){\rule[-0.400pt]{0.800pt}{4.818pt}}
\put(313.0,219.0){\rule[-0.400pt]{10.600pt}{0.800pt}}
\put(313.0,209.0){\rule[-0.400pt]{0.800pt}{4.818pt}}
\put(357.0,209.0){\rule[-0.400pt]{0.800pt}{4.818pt}}
\put(294.0,204.0){\rule[-0.400pt]{10.600pt}{0.800pt}}
\put(294.0,194.0){\rule[-0.400pt]{0.800pt}{4.818pt}}
\put(338.0,194.0){\rule[-0.400pt]{0.800pt}{4.818pt}}
\put(336.0,224.0){\rule[-0.400pt]{10.600pt}{0.800pt}}
\put(336.0,214.0){\rule[-0.400pt]{0.800pt}{4.818pt}}
\put(380.0,214.0){\rule[-0.400pt]{0.800pt}{4.818pt}}
\put(302.0,210.0){\rule[-0.400pt]{10.600pt}{0.800pt}}
\put(302.0,200.0){\rule[-0.400pt]{0.800pt}{4.818pt}}
\put(346.0,200.0){\rule[-0.400pt]{0.800pt}{4.818pt}}
\put(288.0,190.0){\rule[-0.400pt]{10.600pt}{0.800pt}}
\put(288.0,180.0){\rule[-0.400pt]{0.800pt}{4.818pt}}
\put(332.0,180.0){\rule[-0.400pt]{0.800pt}{4.818pt}}
\put(304.0,196.0){\rule[-0.400pt]{10.600pt}{0.800pt}}
\put(304.0,186.0){\rule[-0.400pt]{0.800pt}{4.818pt}}
\put(348.0,186.0){\rule[-0.400pt]{0.800pt}{4.818pt}}
\put(267.0,194.0){\rule[-0.400pt]{10.600pt}{0.800pt}}
\put(267.0,184.0){\rule[-0.400pt]{0.800pt}{4.818pt}}
\put(311.0,184.0){\rule[-0.400pt]{0.800pt}{4.818pt}}
\put(282.0,201.0){\rule[-0.400pt]{10.600pt}{0.800pt}}
\put(282.0,191.0){\rule[-0.400pt]{0.800pt}{4.818pt}}
\put(345,192){\makebox(0,0){$\ast$}}
\put(295,193){\makebox(0,0){$\ast$}}
\put(359,189){\makebox(0,0){$\ast$}}
\put(335,219){\makebox(0,0){$\ast$}}
\put(316,204){\makebox(0,0){$\ast$}}
\put(358,224){\makebox(0,0){$\ast$}}
\put(324,210){\makebox(0,0){$\ast$}}
\put(310,190){\makebox(0,0){$\ast$}}
\put(326,196){\makebox(0,0){$\ast$}}
\put(289,194){\makebox(0,0){$\ast$}}
\put(304,201){\makebox(0,0){$\ast$}}
\put(841,737){\makebox(0,0){$\ast$}}
\put(326.0,191.0){\rule[-0.400pt]{0.800pt}{4.818pt}}
\sbox{\plotpoint}{\rule[-0.500pt]{1.000pt}{1.000pt}}%
\sbox{\plotpoint}{\rule[-0.200pt]{0.400pt}{0.400pt}}%
\put(771,696){\makebox(0,0)[r]{$2\rightarrow1$ pre \uv{najvačšiu} pozíciu}}
\sbox{\plotpoint}{\rule[-0.500pt]{1.000pt}{1.000pt}}%
\multiput(791,696)(20.756,0.000){5}{\usebox{\plotpoint}}
\put(891,696){\usebox{\plotpoint}}
\put(791.00,706.00){\usebox{\plotpoint}}
\put(791,686){\usebox{\plotpoint}}
\put(891.00,706.00){\usebox{\plotpoint}}
\put(891,686){\usebox{\plotpoint}}
\multiput(1210,551)(0.000,20.756){7}{\usebox{\plotpoint}}
\put(1210,679){\usebox{\plotpoint}}
\put(1200.00,551.00){\usebox{\plotpoint}}
\put(1220,551){\usebox{\plotpoint}}
\put(1200.00,679.00){\usebox{\plotpoint}}
\put(1220,679){\usebox{\plotpoint}}
\multiput(1118,478)(0.000,20.756){7}{\usebox{\plotpoint}}
\put(1118,605){\usebox{\plotpoint}}
\put(1108.00,478.00){\usebox{\plotpoint}}
\put(1128,478){\usebox{\plotpoint}}
\put(1108.00,605.00){\usebox{\plotpoint}}
\put(1128,605){\usebox{\plotpoint}}
\multiput(1291,576)(0.000,20.756){7}{\usebox{\plotpoint}}
\put(1291,703){\usebox{\plotpoint}}
\put(1281.00,576.00){\usebox{\plotpoint}}
\put(1301,576){\usebox{\plotpoint}}
\put(1281.00,703.00){\usebox{\plotpoint}}
\put(1301,703){\usebox{\plotpoint}}
\multiput(1239,522)(0.000,20.756){7}{\usebox{\plotpoint}}
\put(1239,650){\usebox{\plotpoint}}
\put(1229.00,522.00){\usebox{\plotpoint}}
\put(1249,522){\usebox{\plotpoint}}
\put(1229.00,650.00){\usebox{\plotpoint}}
\put(1249,650){\usebox{\plotpoint}}
\multiput(1210,585)(0.000,20.756){7}{\usebox{\plotpoint}}
\put(1210,713){\usebox{\plotpoint}}
\put(1200.00,585.00){\usebox{\plotpoint}}
\put(1220,585){\usebox{\plotpoint}}
\put(1200.00,713.00){\usebox{\plotpoint}}
\put(1220,713){\usebox{\plotpoint}}
\multiput(1333,568)(0.000,20.756){7}{\usebox{\plotpoint}}
\put(1333,695){\usebox{\plotpoint}}
\put(1323.00,568.00){\usebox{\plotpoint}}
\put(1343,568){\usebox{\plotpoint}}
\put(1323.00,695.00){\usebox{\plotpoint}}
\put(1343,695){\usebox{\plotpoint}}
\multiput(1245,413)(0.000,20.756){7}{\usebox{\plotpoint}}
\put(1245,541){\usebox{\plotpoint}}
\put(1235.00,413.00){\usebox{\plotpoint}}
\put(1255,413){\usebox{\plotpoint}}
\put(1235.00,541.00){\usebox{\plotpoint}}
\put(1255,541){\usebox{\plotpoint}}
\multiput(1324,569)(0.000,20.756){7}{\usebox{\plotpoint}}
\put(1324,697){\usebox{\plotpoint}}
\put(1314.00,569.00){\usebox{\plotpoint}}
\put(1334,569){\usebox{\plotpoint}}
\put(1314.00,697.00){\usebox{\plotpoint}}
\put(1334,697){\usebox{\plotpoint}}
\multiput(1306,643)(0.000,20.756){7}{\usebox{\plotpoint}}
\put(1306,770){\usebox{\plotpoint}}
\put(1296.00,643.00){\usebox{\plotpoint}}
\put(1316,643){\usebox{\plotpoint}}
\put(1296.00,770.00){\usebox{\plotpoint}}
\put(1316,770){\usebox{\plotpoint}}
\multiput(1312,581)(0.000,20.756){7}{\usebox{\plotpoint}}
\put(1312,708){\usebox{\plotpoint}}
\put(1302.00,581.00){\usebox{\plotpoint}}
\put(1322,581){\usebox{\plotpoint}}
\put(1302.00,708.00){\usebox{\plotpoint}}
\put(1322,708){\usebox{\plotpoint}}
\multiput(1144,615)(20.756,0.000){7}{\usebox{\plotpoint}}
\put(1276,615){\usebox{\plotpoint}}
\put(1144.00,605.00){\usebox{\plotpoint}}
\put(1144,625){\usebox{\plotpoint}}
\put(1276.00,605.00){\usebox{\plotpoint}}
\put(1276,625){\usebox{\plotpoint}}
\multiput(1052,541)(20.756,0.000){7}{\usebox{\plotpoint}}
\put(1184,541){\usebox{\plotpoint}}
\put(1052.00,531.00){\usebox{\plotpoint}}
\put(1052,551){\usebox{\plotpoint}}
\put(1184.00,531.00){\usebox{\plotpoint}}
\put(1184,551){\usebox{\plotpoint}}
\multiput(1225,639)(20.756,0.000){7}{\usebox{\plotpoint}}
\put(1358,639){\usebox{\plotpoint}}
\put(1225.00,629.00){\usebox{\plotpoint}}
\put(1225,649){\usebox{\plotpoint}}
\put(1358.00,629.00){\usebox{\plotpoint}}
\put(1358,649){\usebox{\plotpoint}}
\multiput(1173,586)(20.756,0.000){7}{\usebox{\plotpoint}}
\put(1305,586){\usebox{\plotpoint}}
\put(1173.00,576.00){\usebox{\plotpoint}}
\put(1173,596){\usebox{\plotpoint}}
\put(1305.00,576.00){\usebox{\plotpoint}}
\put(1305,596){\usebox{\plotpoint}}
\multiput(1144,649)(20.756,0.000){7}{\usebox{\plotpoint}}
\put(1276,649){\usebox{\plotpoint}}
\put(1144.00,639.00){\usebox{\plotpoint}}
\put(1144,659){\usebox{\plotpoint}}
\put(1276.00,639.00){\usebox{\plotpoint}}
\put(1276,659){\usebox{\plotpoint}}
\multiput(1267,631)(20.756,0.000){7}{\usebox{\plotpoint}}
\put(1399,631){\usebox{\plotpoint}}
\put(1267.00,621.00){\usebox{\plotpoint}}
\put(1267,641){\usebox{\plotpoint}}
\put(1399.00,621.00){\usebox{\plotpoint}}
\put(1399,641){\usebox{\plotpoint}}
\multiput(1178,477)(20.756,0.000){7}{\usebox{\plotpoint}}
\put(1311,477){\usebox{\plotpoint}}
\put(1178.00,467.00){\usebox{\plotpoint}}
\put(1178,487){\usebox{\plotpoint}}
\put(1311.00,467.00){\usebox{\plotpoint}}
\put(1311,487){\usebox{\plotpoint}}
\multiput(1258,633)(20.756,0.000){7}{\usebox{\plotpoint}}
\put(1390,633){\usebox{\plotpoint}}
\put(1258.00,623.00){\usebox{\plotpoint}}
\put(1258,643){\usebox{\plotpoint}}
\put(1390.00,623.00){\usebox{\plotpoint}}
\put(1390,643){\usebox{\plotpoint}}
\multiput(1240,706)(20.756,0.000){7}{\usebox{\plotpoint}}
\put(1373,706){\usebox{\plotpoint}}
\put(1240.00,696.00){\usebox{\plotpoint}}
\put(1240,716){\usebox{\plotpoint}}
\put(1373.00,696.00){\usebox{\plotpoint}}
\put(1373,716){\usebox{\plotpoint}}
\multiput(1246,645)(20.756,0.000){7}{\usebox{\plotpoint}}
\put(1378,645){\usebox{\plotpoint}}
\put(1246.00,635.00){\usebox{\plotpoint}}
\put(1246,655){\usebox{\plotpoint}}
\put(1378.00,635.00){\usebox{\plotpoint}}
\put(1378,655){\usebox{\plotpoint}}
\put(1210,615){\raisebox{-.8pt}{\makebox(0,0){$\Box$}}}
\put(1118,541){\raisebox{-.8pt}{\makebox(0,0){$\Box$}}}
\put(1291,639){\raisebox{-.8pt}{\makebox(0,0){$\Box$}}}
\put(1239,586){\raisebox{-.8pt}{\makebox(0,0){$\Box$}}}
\put(1210,649){\raisebox{-.8pt}{\makebox(0,0){$\Box$}}}
\put(1333,631){\raisebox{-.8pt}{\makebox(0,0){$\Box$}}}
\put(1245,477){\raisebox{-.8pt}{\makebox(0,0){$\Box$}}}
\put(1324,633){\raisebox{-.8pt}{\makebox(0,0){$\Box$}}}
\put(1306,706){\raisebox{-.8pt}{\makebox(0,0){$\Box$}}}
\put(1312,645){\raisebox{-.8pt}{\makebox(0,0){$\Box$}}}
\put(841,696){\raisebox{-.8pt}{\makebox(0,0){$\Box$}}}
\sbox{\plotpoint}{\rule[-0.600pt]{1.200pt}{1.200pt}}%
\sbox{\plotpoint}{\rule[-0.200pt]{0.400pt}{0.400pt}}%
\put(771,655){\makebox(0,0)[r]{$2\rightarrow1$ pre \uv{strednú} pozíciu}}
\sbox{\plotpoint}{\rule[-0.600pt]{1.200pt}{1.200pt}}%
\put(791.0,655.0){\rule[-0.600pt]{24.090pt}{1.200pt}}
\put(791.0,645.0){\rule[-0.600pt]{1.200pt}{4.818pt}}
\put(891.0,645.0){\rule[-0.600pt]{1.200pt}{4.818pt}}
\put(742.0,430.0){\rule[-0.600pt]{1.200pt}{13.972pt}}
\put(732.0,430.0){\rule[-0.600pt]{4.818pt}{1.200pt}}
\put(732.0,488.0){\rule[-0.600pt]{4.818pt}{1.200pt}}
\put(768.0,370.0){\rule[-0.600pt]{1.200pt}{13.972pt}}
\put(758.0,370.0){\rule[-0.600pt]{4.818pt}{1.200pt}}
\put(758.0,428.0){\rule[-0.600pt]{4.818pt}{1.200pt}}
\put(744.0,350.0){\rule[-0.600pt]{1.200pt}{13.972pt}}
\put(734.0,350.0){\rule[-0.600pt]{4.818pt}{1.200pt}}
\put(734.0,408.0){\rule[-0.600pt]{4.818pt}{1.200pt}}
\put(750.0,356.0){\rule[-0.600pt]{1.200pt}{13.972pt}}
\put(740.0,356.0){\rule[-0.600pt]{4.818pt}{1.200pt}}
\put(740.0,414.0){\rule[-0.600pt]{4.818pt}{1.200pt}}
\put(736.0,324.0){\rule[-0.600pt]{1.200pt}{13.972pt}}
\put(726.0,324.0){\rule[-0.600pt]{4.818pt}{1.200pt}}
\put(726.0,382.0){\rule[-0.600pt]{4.818pt}{1.200pt}}
\put(717.0,330.0){\rule[-0.600pt]{1.200pt}{13.972pt}}
\put(707.0,330.0){\rule[-0.600pt]{4.818pt}{1.200pt}}
\put(707.0,388.0){\rule[-0.600pt]{4.818pt}{1.200pt}}
\put(777.0,365.0){\rule[-0.600pt]{1.200pt}{13.972pt}}
\put(767.0,365.0){\rule[-0.600pt]{4.818pt}{1.200pt}}
\put(767.0,423.0){\rule[-0.600pt]{4.818pt}{1.200pt}}
\put(719.0,361.0){\rule[-0.600pt]{1.200pt}{13.972pt}}
\put(709.0,361.0){\rule[-0.600pt]{4.818pt}{1.200pt}}
\put(709.0,419.0){\rule[-0.600pt]{4.818pt}{1.200pt}}
\put(700.0,340.0){\rule[-0.600pt]{1.200pt}{13.972pt}}
\put(690.0,340.0){\rule[-0.600pt]{4.818pt}{1.200pt}}
\put(690.0,398.0){\rule[-0.600pt]{4.818pt}{1.200pt}}
\put(731.0,343.0){\rule[-0.600pt]{1.200pt}{13.972pt}}
\put(721.0,343.0){\rule[-0.600pt]{4.818pt}{1.200pt}}
\put(721.0,401.0){\rule[-0.600pt]{4.818pt}{1.200pt}}
\put(718.0,459.0){\rule[-0.600pt]{11.563pt}{1.200pt}}
\put(718.0,449.0){\rule[-0.600pt]{1.200pt}{4.818pt}}
\put(766.0,449.0){\rule[-0.600pt]{1.200pt}{4.818pt}}
\put(744.0,399.0){\rule[-0.600pt]{11.563pt}{1.200pt}}
\put(744.0,389.0){\rule[-0.600pt]{1.200pt}{4.818pt}}
\put(792.0,389.0){\rule[-0.600pt]{1.200pt}{4.818pt}}
\put(720.0,379.0){\rule[-0.600pt]{11.563pt}{1.200pt}}
\put(720.0,369.0){\rule[-0.600pt]{1.200pt}{4.818pt}}
\put(768.0,369.0){\rule[-0.600pt]{1.200pt}{4.818pt}}
\put(727.0,385.0){\rule[-0.600pt]{11.322pt}{1.200pt}}
\put(727.0,375.0){\rule[-0.600pt]{1.200pt}{4.818pt}}
\put(774.0,375.0){\rule[-0.600pt]{1.200pt}{4.818pt}}
\put(712.0,353.0){\rule[-0.600pt]{11.322pt}{1.200pt}}
\put(712.0,343.0){\rule[-0.600pt]{1.200pt}{4.818pt}}
\put(759.0,343.0){\rule[-0.600pt]{1.200pt}{4.818pt}}
\put(693.0,359.0){\rule[-0.600pt]{11.563pt}{1.200pt}}
\put(693.0,349.0){\rule[-0.600pt]{1.200pt}{4.818pt}}
\put(741.0,349.0){\rule[-0.600pt]{1.200pt}{4.818pt}}
\put(753.0,394.0){\rule[-0.600pt]{11.563pt}{1.200pt}}
\put(753.0,384.0){\rule[-0.600pt]{1.200pt}{4.818pt}}
\put(801.0,384.0){\rule[-0.600pt]{1.200pt}{4.818pt}}
\put(695.0,390.0){\rule[-0.600pt]{11.563pt}{1.200pt}}
\put(695.0,380.0){\rule[-0.600pt]{1.200pt}{4.818pt}}
\put(743.0,380.0){\rule[-0.600pt]{1.200pt}{4.818pt}}
\put(676.0,369.0){\rule[-0.600pt]{11.563pt}{1.200pt}}
\put(676.0,359.0){\rule[-0.600pt]{1.200pt}{4.818pt}}
\put(724.0,359.0){\rule[-0.600pt]{1.200pt}{4.818pt}}
\put(707.0,372.0){\rule[-0.600pt]{11.563pt}{1.200pt}}
\put(707.0,362.0){\rule[-0.600pt]{1.200pt}{4.818pt}}
\put(742,459){\makebox(0,0){$\blacksquare$}}
\put(768,399){\makebox(0,0){$\blacksquare$}}
\put(744,379){\makebox(0,0){$\blacksquare$}}
\put(750,385){\makebox(0,0){$\blacksquare$}}
\put(736,353){\makebox(0,0){$\blacksquare$}}
\put(717,359){\makebox(0,0){$\blacksquare$}}
\put(777,394){\makebox(0,0){$\blacksquare$}}
\put(719,390){\makebox(0,0){$\blacksquare$}}
\put(700,369){\makebox(0,0){$\blacksquare$}}
\put(731,372){\makebox(0,0){$\blacksquare$}}
\put(841,655){\makebox(0,0){$\blacksquare$}}
\put(755.0,362.0){\rule[-0.600pt]{1.200pt}{4.818pt}}
\sbox{\plotpoint}{\rule[-0.500pt]{1.000pt}{1.000pt}}%
\sbox{\plotpoint}{\rule[-0.200pt]{0.400pt}{0.400pt}}%
\put(771,614){\makebox(0,0)[r]{$2 \rightarrow 1$ pre \uv{najmenšiu} pozíciu}}
\sbox{\plotpoint}{\rule[-0.500pt]{1.000pt}{1.000pt}}%
\multiput(791,614)(41.511,0.000){3}{\usebox{\plotpoint}}
\put(891,614){\usebox{\plotpoint}}
\put(791.00,624.00){\usebox{\plotpoint}}
\put(791,604){\usebox{\plotpoint}}
\put(891.00,624.00){\usebox{\plotpoint}}
\put(891,604){\usebox{\plotpoint}}
\multiput(343,193)(0.000,41.511){2}{\usebox{\plotpoint}}
\put(343,238){\usebox{\plotpoint}}
\put(333.00,193.00){\usebox{\plotpoint}}
\put(353,193){\usebox{\plotpoint}}
\put(333.00,238.00){\usebox{\plotpoint}}
\put(353,238){\usebox{\plotpoint}}
\multiput(369,205)(0.000,41.511){2}{\usebox{\plotpoint}}
\put(369,250){\usebox{\plotpoint}}
\put(359.00,205.00){\usebox{\plotpoint}}
\put(379,205){\usebox{\plotpoint}}
\put(359.00,250.00){\usebox{\plotpoint}}
\put(379,250){\usebox{\plotpoint}}
\multiput(357,192)(0.000,41.511){2}{\usebox{\plotpoint}}
\put(357,238){\usebox{\plotpoint}}
\put(347.00,192.00){\usebox{\plotpoint}}
\put(367,192){\usebox{\plotpoint}}
\put(347.00,238.00){\usebox{\plotpoint}}
\put(367,238){\usebox{\plotpoint}}
\multiput(291,133)(0.000,41.511){2}{\usebox{\plotpoint}}
\put(291,178){\usebox{\plotpoint}}
\put(281.00,133.00){\usebox{\plotpoint}}
\put(301,133){\usebox{\plotpoint}}
\put(281.00,178.00){\usebox{\plotpoint}}
\put(301,178){\usebox{\plotpoint}}
\multiput(365,210)(0.000,41.511){2}{\usebox{\plotpoint}}
\put(365,255){\usebox{\plotpoint}}
\put(355.00,210.00){\usebox{\plotpoint}}
\put(375,210){\usebox{\plotpoint}}
\put(355.00,255.00){\usebox{\plotpoint}}
\put(375,255){\usebox{\plotpoint}}
\multiput(366,207)(0.000,41.511){2}{\usebox{\plotpoint}}
\put(366,252){\usebox{\plotpoint}}
\put(356.00,207.00){\usebox{\plotpoint}}
\put(376,207){\usebox{\plotpoint}}
\put(356.00,252.00){\usebox{\plotpoint}}
\put(376,252){\usebox{\plotpoint}}
\multiput(364,193)(0.000,41.511){2}{\usebox{\plotpoint}}
\put(364,239){\usebox{\plotpoint}}
\put(354.00,193.00){\usebox{\plotpoint}}
\put(374,193){\usebox{\plotpoint}}
\put(354.00,239.00){\usebox{\plotpoint}}
\put(374,239){\usebox{\plotpoint}}
\multiput(370,207)(0.000,41.511){2}{\usebox{\plotpoint}}
\put(370,253){\usebox{\plotpoint}}
\put(360.00,207.00){\usebox{\plotpoint}}
\put(380,207){\usebox{\plotpoint}}
\put(360.00,253.00){\usebox{\plotpoint}}
\put(380,253){\usebox{\plotpoint}}
\multiput(354,211)(0.000,41.511){2}{\usebox{\plotpoint}}
\put(354,256){\usebox{\plotpoint}}
\put(344.00,211.00){\usebox{\plotpoint}}
\put(364,211){\usebox{\plotpoint}}
\put(344.00,256.00){\usebox{\plotpoint}}
\put(364,256){\usebox{\plotpoint}}
\multiput(338,193)(0.000,41.511){2}{\usebox{\plotpoint}}
\put(338,238){\usebox{\plotpoint}}
\put(328.00,193.00){\usebox{\plotpoint}}
\put(348,193){\usebox{\plotpoint}}
\put(328.00,238.00){\usebox{\plotpoint}}
\put(348,238){\usebox{\plotpoint}}
\multiput(319,216)(41.511,0.000){2}{\usebox{\plotpoint}}
\put(366,216){\usebox{\plotpoint}}
\put(319.00,206.00){\usebox{\plotpoint}}
\put(319,226){\usebox{\plotpoint}}
\put(366.00,206.00){\usebox{\plotpoint}}
\put(366,226){\usebox{\plotpoint}}
\multiput(345,227)(41.511,0.000){2}{\usebox{\plotpoint}}
\put(393,227){\usebox{\plotpoint}}
\put(345.00,217.00){\usebox{\plotpoint}}
\put(345,237){\usebox{\plotpoint}}
\put(393.00,217.00){\usebox{\plotpoint}}
\put(393,237){\usebox{\plotpoint}}
\multiput(334,215)(41.511,0.000){2}{\usebox{\plotpoint}}
\put(381,215){\usebox{\plotpoint}}
\put(334.00,205.00){\usebox{\plotpoint}}
\put(334,225){\usebox{\plotpoint}}
\put(381.00,205.00){\usebox{\plotpoint}}
\put(381,225){\usebox{\plotpoint}}
\multiput(267,156)(41.511,0.000){2}{\usebox{\plotpoint}}
\put(315,156){\usebox{\plotpoint}}
\put(267.00,146.00){\usebox{\plotpoint}}
\put(267,166){\usebox{\plotpoint}}
\put(315.00,146.00){\usebox{\plotpoint}}
\put(315,166){\usebox{\plotpoint}}
\multiput(341,232)(41.511,0.000){2}{\usebox{\plotpoint}}
\put(389,232){\usebox{\plotpoint}}
\put(341.00,222.00){\usebox{\plotpoint}}
\put(341,242){\usebox{\plotpoint}}
\put(389.00,222.00){\usebox{\plotpoint}}
\put(389,242){\usebox{\plotpoint}}
\multiput(342,229)(41.511,0.000){2}{\usebox{\plotpoint}}
\put(390,229){\usebox{\plotpoint}}
\put(342.00,219.00){\usebox{\plotpoint}}
\put(342,239){\usebox{\plotpoint}}
\put(390.00,219.00){\usebox{\plotpoint}}
\put(390,239){\usebox{\plotpoint}}
\multiput(340,216)(41.511,0.000){2}{\usebox{\plotpoint}}
\put(388,216){\usebox{\plotpoint}}
\put(340.00,206.00){\usebox{\plotpoint}}
\put(340,226){\usebox{\plotpoint}}
\put(388.00,206.00){\usebox{\plotpoint}}
\put(388,226){\usebox{\plotpoint}}
\multiput(346,230)(41.511,0.000){2}{\usebox{\plotpoint}}
\put(394,230){\usebox{\plotpoint}}
\put(346.00,220.00){\usebox{\plotpoint}}
\put(346,240){\usebox{\plotpoint}}
\put(394.00,220.00){\usebox{\plotpoint}}
\put(394,240){\usebox{\plotpoint}}
\multiput(330,234)(41.511,0.000){2}{\usebox{\plotpoint}}
\put(378,234){\usebox{\plotpoint}}
\put(330.00,224.00){\usebox{\plotpoint}}
\put(330,244){\usebox{\plotpoint}}
\put(378.00,224.00){\usebox{\plotpoint}}
\put(378,244){\usebox{\plotpoint}}
\multiput(314,215)(41.511,0.000){2}{\usebox{\plotpoint}}
\put(362,215){\usebox{\plotpoint}}
\put(314.00,205.00){\usebox{\plotpoint}}
\put(314,225){\usebox{\plotpoint}}
\put(362.00,205.00){\usebox{\plotpoint}}
\put(362,225){\usebox{\plotpoint}}
\put(343,216){\makebox(0,0){$\circ$}}
\put(369,227){\makebox(0,0){$\circ$}}
\put(357,215){\makebox(0,0){$\circ$}}
\put(291,156){\makebox(0,0){$\circ$}}
\put(365,232){\makebox(0,0){$\circ$}}
\put(366,229){\makebox(0,0){$\circ$}}
\put(364,216){\makebox(0,0){$\circ$}}
\put(370,230){\makebox(0,0){$\circ$}}
\put(354,234){\makebox(0,0){$\circ$}}
\put(338,215){\makebox(0,0){$\circ$}}
\put(841,614){\makebox(0,0){$\circ$}}
\sbox{\plotpoint}{\rule[-0.200pt]{0.400pt}{0.400pt}}%
\put(771,573){\makebox(0,0)[r]{teoretická závyslosť}}
\put(791.0,573.0){\rule[-0.200pt]{24.090pt}{0.400pt}}
\put(250,180){\usebox{\plotpoint}}
\multiput(250.00,180.59)(0.943,0.482){9}{\rule{0.833pt}{0.116pt}}
\multiput(250.00,179.17)(9.270,6.000){2}{\rule{0.417pt}{0.400pt}}
\multiput(261.00,186.59)(1.267,0.477){7}{\rule{1.060pt}{0.115pt}}
\multiput(261.00,185.17)(9.800,5.000){2}{\rule{0.530pt}{0.400pt}}
\multiput(273.00,191.59)(0.943,0.482){9}{\rule{0.833pt}{0.116pt}}
\multiput(273.00,190.17)(9.270,6.000){2}{\rule{0.417pt}{0.400pt}}
\multiput(284.00,197.59)(1.033,0.482){9}{\rule{0.900pt}{0.116pt}}
\multiput(284.00,196.17)(10.132,6.000){2}{\rule{0.450pt}{0.400pt}}
\multiput(296.00,203.59)(1.267,0.477){7}{\rule{1.060pt}{0.115pt}}
\multiput(296.00,202.17)(9.800,5.000){2}{\rule{0.530pt}{0.400pt}}
\multiput(308.00,208.59)(0.943,0.482){9}{\rule{0.833pt}{0.116pt}}
\multiput(308.00,207.17)(9.270,6.000){2}{\rule{0.417pt}{0.400pt}}
\multiput(319.00,214.59)(1.033,0.482){9}{\rule{0.900pt}{0.116pt}}
\multiput(319.00,213.17)(10.132,6.000){2}{\rule{0.450pt}{0.400pt}}
\multiput(331.00,220.59)(1.033,0.482){9}{\rule{0.900pt}{0.116pt}}
\multiput(331.00,219.17)(10.132,6.000){2}{\rule{0.450pt}{0.400pt}}
\multiput(343.00,226.59)(1.155,0.477){7}{\rule{0.980pt}{0.115pt}}
\multiput(343.00,225.17)(8.966,5.000){2}{\rule{0.490pt}{0.400pt}}
\multiput(354.00,231.59)(1.033,0.482){9}{\rule{0.900pt}{0.116pt}}
\multiput(354.00,230.17)(10.132,6.000){2}{\rule{0.450pt}{0.400pt}}
\multiput(366.00,237.59)(0.943,0.482){9}{\rule{0.833pt}{0.116pt}}
\multiput(366.00,236.17)(9.270,6.000){2}{\rule{0.417pt}{0.400pt}}
\multiput(377.00,243.59)(1.033,0.482){9}{\rule{0.900pt}{0.116pt}}
\multiput(377.00,242.17)(10.132,6.000){2}{\rule{0.450pt}{0.400pt}}
\multiput(389.00,249.59)(1.267,0.477){7}{\rule{1.060pt}{0.115pt}}
\multiput(389.00,248.17)(9.800,5.000){2}{\rule{0.530pt}{0.400pt}}
\multiput(401.00,254.59)(0.943,0.482){9}{\rule{0.833pt}{0.116pt}}
\multiput(401.00,253.17)(9.270,6.000){2}{\rule{0.417pt}{0.400pt}}
\multiput(412.00,260.59)(1.033,0.482){9}{\rule{0.900pt}{0.116pt}}
\multiput(412.00,259.17)(10.132,6.000){2}{\rule{0.450pt}{0.400pt}}
\multiput(424.00,266.59)(0.943,0.482){9}{\rule{0.833pt}{0.116pt}}
\multiput(424.00,265.17)(9.270,6.000){2}{\rule{0.417pt}{0.400pt}}
\multiput(435.00,272.59)(1.267,0.477){7}{\rule{1.060pt}{0.115pt}}
\multiput(435.00,271.17)(9.800,5.000){2}{\rule{0.530pt}{0.400pt}}
\multiput(447.00,277.59)(1.033,0.482){9}{\rule{0.900pt}{0.116pt}}
\multiput(447.00,276.17)(10.132,6.000){2}{\rule{0.450pt}{0.400pt}}
\multiput(459.00,283.59)(0.943,0.482){9}{\rule{0.833pt}{0.116pt}}
\multiput(459.00,282.17)(9.270,6.000){2}{\rule{0.417pt}{0.400pt}}
\multiput(470.00,289.59)(1.033,0.482){9}{\rule{0.900pt}{0.116pt}}
\multiput(470.00,288.17)(10.132,6.000){2}{\rule{0.450pt}{0.400pt}}
\multiput(482.00,295.59)(1.267,0.477){7}{\rule{1.060pt}{0.115pt}}
\multiput(482.00,294.17)(9.800,5.000){2}{\rule{0.530pt}{0.400pt}}
\multiput(494.00,300.59)(0.943,0.482){9}{\rule{0.833pt}{0.116pt}}
\multiput(494.00,299.17)(9.270,6.000){2}{\rule{0.417pt}{0.400pt}}
\multiput(505.00,306.59)(1.033,0.482){9}{\rule{0.900pt}{0.116pt}}
\multiput(505.00,305.17)(10.132,6.000){2}{\rule{0.450pt}{0.400pt}}
\multiput(517.00,312.59)(0.943,0.482){9}{\rule{0.833pt}{0.116pt}}
\multiput(517.00,311.17)(9.270,6.000){2}{\rule{0.417pt}{0.400pt}}
\multiput(528.00,318.59)(1.267,0.477){7}{\rule{1.060pt}{0.115pt}}
\multiput(528.00,317.17)(9.800,5.000){2}{\rule{0.530pt}{0.400pt}}
\multiput(540.00,323.59)(1.033,0.482){9}{\rule{0.900pt}{0.116pt}}
\multiput(540.00,322.17)(10.132,6.000){2}{\rule{0.450pt}{0.400pt}}
\multiput(552.00,329.59)(0.943,0.482){9}{\rule{0.833pt}{0.116pt}}
\multiput(552.00,328.17)(9.270,6.000){2}{\rule{0.417pt}{0.400pt}}
\multiput(563.00,335.59)(1.033,0.482){9}{\rule{0.900pt}{0.116pt}}
\multiput(563.00,334.17)(10.132,6.000){2}{\rule{0.450pt}{0.400pt}}
\multiput(575.00,341.59)(1.155,0.477){7}{\rule{0.980pt}{0.115pt}}
\multiput(575.00,340.17)(8.966,5.000){2}{\rule{0.490pt}{0.400pt}}
\multiput(586.00,346.59)(1.033,0.482){9}{\rule{0.900pt}{0.116pt}}
\multiput(586.00,345.17)(10.132,6.000){2}{\rule{0.450pt}{0.400pt}}
\multiput(598.00,352.59)(1.033,0.482){9}{\rule{0.900pt}{0.116pt}}
\multiput(598.00,351.17)(10.132,6.000){2}{\rule{0.450pt}{0.400pt}}
\multiput(610.00,358.59)(0.943,0.482){9}{\rule{0.833pt}{0.116pt}}
\multiput(610.00,357.17)(9.270,6.000){2}{\rule{0.417pt}{0.400pt}}
\multiput(621.00,364.59)(1.267,0.477){7}{\rule{1.060pt}{0.115pt}}
\multiput(621.00,363.17)(9.800,5.000){2}{\rule{0.530pt}{0.400pt}}
\multiput(633.00,369.59)(1.033,0.482){9}{\rule{0.900pt}{0.116pt}}
\multiput(633.00,368.17)(10.132,6.000){2}{\rule{0.450pt}{0.400pt}}
\multiput(645.00,375.59)(0.943,0.482){9}{\rule{0.833pt}{0.116pt}}
\multiput(645.00,374.17)(9.270,6.000){2}{\rule{0.417pt}{0.400pt}}
\multiput(656.00,381.59)(1.033,0.482){9}{\rule{0.900pt}{0.116pt}}
\multiput(656.00,380.17)(10.132,6.000){2}{\rule{0.450pt}{0.400pt}}
\multiput(668.00,387.59)(1.155,0.477){7}{\rule{0.980pt}{0.115pt}}
\multiput(668.00,386.17)(8.966,5.000){2}{\rule{0.490pt}{0.400pt}}
\multiput(679.00,392.59)(1.033,0.482){9}{\rule{0.900pt}{0.116pt}}
\multiput(679.00,391.17)(10.132,6.000){2}{\rule{0.450pt}{0.400pt}}
\multiput(691.00,398.59)(1.033,0.482){9}{\rule{0.900pt}{0.116pt}}
\multiput(691.00,397.17)(10.132,6.000){2}{\rule{0.450pt}{0.400pt}}
\multiput(703.00,404.59)(0.943,0.482){9}{\rule{0.833pt}{0.116pt}}
\multiput(703.00,403.17)(9.270,6.000){2}{\rule{0.417pt}{0.400pt}}
\multiput(714.00,410.59)(1.267,0.477){7}{\rule{1.060pt}{0.115pt}}
\multiput(714.00,409.17)(9.800,5.000){2}{\rule{0.530pt}{0.400pt}}
\multiput(726.00,415.59)(0.943,0.482){9}{\rule{0.833pt}{0.116pt}}
\multiput(726.00,414.17)(9.270,6.000){2}{\rule{0.417pt}{0.400pt}}
\multiput(737.00,421.59)(1.033,0.482){9}{\rule{0.900pt}{0.116pt}}
\multiput(737.00,420.17)(10.132,6.000){2}{\rule{0.450pt}{0.400pt}}
\multiput(749.00,427.59)(1.033,0.482){9}{\rule{0.900pt}{0.116pt}}
\multiput(749.00,426.17)(10.132,6.000){2}{\rule{0.450pt}{0.400pt}}
\multiput(761.00,433.59)(1.155,0.477){7}{\rule{0.980pt}{0.115pt}}
\multiput(761.00,432.17)(8.966,5.000){2}{\rule{0.490pt}{0.400pt}}
\multiput(772.00,438.59)(1.033,0.482){9}{\rule{0.900pt}{0.116pt}}
\multiput(772.00,437.17)(10.132,6.000){2}{\rule{0.450pt}{0.400pt}}
\multiput(784.00,444.59)(0.943,0.482){9}{\rule{0.833pt}{0.116pt}}
\multiput(784.00,443.17)(9.270,6.000){2}{\rule{0.417pt}{0.400pt}}
\multiput(795.00,450.59)(1.267,0.477){7}{\rule{1.060pt}{0.115pt}}
\multiput(795.00,449.17)(9.800,5.000){2}{\rule{0.530pt}{0.400pt}}
\multiput(807.00,455.59)(1.033,0.482){9}{\rule{0.900pt}{0.116pt}}
\multiput(807.00,454.17)(10.132,6.000){2}{\rule{0.450pt}{0.400pt}}
\multiput(819.00,461.59)(0.943,0.482){9}{\rule{0.833pt}{0.116pt}}
\multiput(819.00,460.17)(9.270,6.000){2}{\rule{0.417pt}{0.400pt}}
\multiput(830.00,467.59)(1.033,0.482){9}{\rule{0.900pt}{0.116pt}}
\multiput(830.00,466.17)(10.132,6.000){2}{\rule{0.450pt}{0.400pt}}
\multiput(842.00,473.59)(1.267,0.477){7}{\rule{1.060pt}{0.115pt}}
\multiput(842.00,472.17)(9.800,5.000){2}{\rule{0.530pt}{0.400pt}}
\multiput(854.00,478.59)(0.943,0.482){9}{\rule{0.833pt}{0.116pt}}
\multiput(854.00,477.17)(9.270,6.000){2}{\rule{0.417pt}{0.400pt}}
\multiput(865.00,484.59)(1.033,0.482){9}{\rule{0.900pt}{0.116pt}}
\multiput(865.00,483.17)(10.132,6.000){2}{\rule{0.450pt}{0.400pt}}
\multiput(877.00,490.59)(0.943,0.482){9}{\rule{0.833pt}{0.116pt}}
\multiput(877.00,489.17)(9.270,6.000){2}{\rule{0.417pt}{0.400pt}}
\multiput(888.00,496.59)(1.267,0.477){7}{\rule{1.060pt}{0.115pt}}
\multiput(888.00,495.17)(9.800,5.000){2}{\rule{0.530pt}{0.400pt}}
\multiput(900.00,501.59)(1.033,0.482){9}{\rule{0.900pt}{0.116pt}}
\multiput(900.00,500.17)(10.132,6.000){2}{\rule{0.450pt}{0.400pt}}
\multiput(912.00,507.59)(0.943,0.482){9}{\rule{0.833pt}{0.116pt}}
\multiput(912.00,506.17)(9.270,6.000){2}{\rule{0.417pt}{0.400pt}}
\multiput(923.00,513.59)(1.033,0.482){9}{\rule{0.900pt}{0.116pt}}
\multiput(923.00,512.17)(10.132,6.000){2}{\rule{0.450pt}{0.400pt}}
\multiput(935.00,519.59)(1.155,0.477){7}{\rule{0.980pt}{0.115pt}}
\multiput(935.00,518.17)(8.966,5.000){2}{\rule{0.490pt}{0.400pt}}
\multiput(946.00,524.59)(1.033,0.482){9}{\rule{0.900pt}{0.116pt}}
\multiput(946.00,523.17)(10.132,6.000){2}{\rule{0.450pt}{0.400pt}}
\multiput(958.00,530.59)(1.033,0.482){9}{\rule{0.900pt}{0.116pt}}
\multiput(958.00,529.17)(10.132,6.000){2}{\rule{0.450pt}{0.400pt}}
\multiput(970.00,536.59)(0.943,0.482){9}{\rule{0.833pt}{0.116pt}}
\multiput(970.00,535.17)(9.270,6.000){2}{\rule{0.417pt}{0.400pt}}
\multiput(981.00,542.59)(1.267,0.477){7}{\rule{1.060pt}{0.115pt}}
\multiput(981.00,541.17)(9.800,5.000){2}{\rule{0.530pt}{0.400pt}}
\multiput(993.00,547.59)(1.033,0.482){9}{\rule{0.900pt}{0.116pt}}
\multiput(993.00,546.17)(10.132,6.000){2}{\rule{0.450pt}{0.400pt}}
\multiput(1005.00,553.59)(0.943,0.482){9}{\rule{0.833pt}{0.116pt}}
\multiput(1005.00,552.17)(9.270,6.000){2}{\rule{0.417pt}{0.400pt}}
\multiput(1016.00,559.59)(1.033,0.482){9}{\rule{0.900pt}{0.116pt}}
\multiput(1016.00,558.17)(10.132,6.000){2}{\rule{0.450pt}{0.400pt}}
\multiput(1028.00,565.59)(1.155,0.477){7}{\rule{0.980pt}{0.115pt}}
\multiput(1028.00,564.17)(8.966,5.000){2}{\rule{0.490pt}{0.400pt}}
\multiput(1039.00,570.59)(1.033,0.482){9}{\rule{0.900pt}{0.116pt}}
\multiput(1039.00,569.17)(10.132,6.000){2}{\rule{0.450pt}{0.400pt}}
\multiput(1051.00,576.59)(1.033,0.482){9}{\rule{0.900pt}{0.116pt}}
\multiput(1051.00,575.17)(10.132,6.000){2}{\rule{0.450pt}{0.400pt}}
\multiput(1063.00,582.59)(0.943,0.482){9}{\rule{0.833pt}{0.116pt}}
\multiput(1063.00,581.17)(9.270,6.000){2}{\rule{0.417pt}{0.400pt}}
\multiput(1074.00,588.59)(1.267,0.477){7}{\rule{1.060pt}{0.115pt}}
\multiput(1074.00,587.17)(9.800,5.000){2}{\rule{0.530pt}{0.400pt}}
\multiput(1086.00,593.59)(0.943,0.482){9}{\rule{0.833pt}{0.116pt}}
\multiput(1086.00,592.17)(9.270,6.000){2}{\rule{0.417pt}{0.400pt}}
\multiput(1097.00,599.59)(1.033,0.482){9}{\rule{0.900pt}{0.116pt}}
\multiput(1097.00,598.17)(10.132,6.000){2}{\rule{0.450pt}{0.400pt}}
\multiput(1109.00,605.59)(1.033,0.482){9}{\rule{0.900pt}{0.116pt}}
\multiput(1109.00,604.17)(10.132,6.000){2}{\rule{0.450pt}{0.400pt}}
\multiput(1121.00,611.59)(1.155,0.477){7}{\rule{0.980pt}{0.115pt}}
\multiput(1121.00,610.17)(8.966,5.000){2}{\rule{0.490pt}{0.400pt}}
\multiput(1132.00,616.59)(1.033,0.482){9}{\rule{0.900pt}{0.116pt}}
\multiput(1132.00,615.17)(10.132,6.000){2}{\rule{0.450pt}{0.400pt}}
\multiput(1144.00,622.59)(1.033,0.482){9}{\rule{0.900pt}{0.116pt}}
\multiput(1144.00,621.17)(10.132,6.000){2}{\rule{0.450pt}{0.400pt}}
\multiput(1156.00,628.59)(0.943,0.482){9}{\rule{0.833pt}{0.116pt}}
\multiput(1156.00,627.17)(9.270,6.000){2}{\rule{0.417pt}{0.400pt}}
\multiput(1167.00,634.59)(1.267,0.477){7}{\rule{1.060pt}{0.115pt}}
\multiput(1167.00,633.17)(9.800,5.000){2}{\rule{0.530pt}{0.400pt}}
\multiput(1179.00,639.59)(0.943,0.482){9}{\rule{0.833pt}{0.116pt}}
\multiput(1179.00,638.17)(9.270,6.000){2}{\rule{0.417pt}{0.400pt}}
\multiput(1190.00,645.59)(1.033,0.482){9}{\rule{0.900pt}{0.116pt}}
\multiput(1190.00,644.17)(10.132,6.000){2}{\rule{0.450pt}{0.400pt}}
\multiput(1202.00,651.59)(1.033,0.482){9}{\rule{0.900pt}{0.116pt}}
\multiput(1202.00,650.17)(10.132,6.000){2}{\rule{0.450pt}{0.400pt}}
\multiput(1214.00,657.59)(1.155,0.477){7}{\rule{0.980pt}{0.115pt}}
\multiput(1214.00,656.17)(8.966,5.000){2}{\rule{0.490pt}{0.400pt}}
\multiput(1225.00,662.59)(1.033,0.482){9}{\rule{0.900pt}{0.116pt}}
\multiput(1225.00,661.17)(10.132,6.000){2}{\rule{0.450pt}{0.400pt}}
\multiput(1237.00,668.59)(0.943,0.482){9}{\rule{0.833pt}{0.116pt}}
\multiput(1237.00,667.17)(9.270,6.000){2}{\rule{0.417pt}{0.400pt}}
\multiput(1248.00,674.59)(1.033,0.482){9}{\rule{0.900pt}{0.116pt}}
\multiput(1248.00,673.17)(10.132,6.000){2}{\rule{0.450pt}{0.400pt}}
\multiput(1260.00,680.59)(1.267,0.477){7}{\rule{1.060pt}{0.115pt}}
\multiput(1260.00,679.17)(9.800,5.000){2}{\rule{0.530pt}{0.400pt}}
\multiput(1272.00,685.59)(0.943,0.482){9}{\rule{0.833pt}{0.116pt}}
\multiput(1272.00,684.17)(9.270,6.000){2}{\rule{0.417pt}{0.400pt}}
\multiput(1283.00,691.59)(1.033,0.482){9}{\rule{0.900pt}{0.116pt}}
\multiput(1283.00,690.17)(10.132,6.000){2}{\rule{0.450pt}{0.400pt}}
\multiput(1295.00,697.59)(1.267,0.477){7}{\rule{1.060pt}{0.115pt}}
\multiput(1295.00,696.17)(9.800,5.000){2}{\rule{0.530pt}{0.400pt}}
\multiput(1307.00,702.59)(0.943,0.482){9}{\rule{0.833pt}{0.116pt}}
\multiput(1307.00,701.17)(9.270,6.000){2}{\rule{0.417pt}{0.400pt}}
\multiput(1318.00,708.59)(1.033,0.482){9}{\rule{0.900pt}{0.116pt}}
\multiput(1318.00,707.17)(10.132,6.000){2}{\rule{0.450pt}{0.400pt}}
\multiput(1330.00,714.59)(0.943,0.482){9}{\rule{0.833pt}{0.116pt}}
\multiput(1330.00,713.17)(9.270,6.000){2}{\rule{0.417pt}{0.400pt}}
\multiput(1341.00,720.59)(1.267,0.477){7}{\rule{1.060pt}{0.115pt}}
\multiput(1341.00,719.17)(9.800,5.000){2}{\rule{0.530pt}{0.400pt}}
\multiput(1353.00,725.59)(1.033,0.482){9}{\rule{0.900pt}{0.116pt}}
\multiput(1353.00,724.17)(10.132,6.000){2}{\rule{0.450pt}{0.400pt}}
\multiput(1365.00,731.59)(0.943,0.482){9}{\rule{0.833pt}{0.116pt}}
\multiput(1365.00,730.17)(9.270,6.000){2}{\rule{0.417pt}{0.400pt}}
\multiput(1376.00,737.59)(1.033,0.482){9}{\rule{0.900pt}{0.116pt}}
\multiput(1376.00,736.17)(10.132,6.000){2}{\rule{0.450pt}{0.400pt}}
\multiput(1388.00,743.59)(1.155,0.477){7}{\rule{0.980pt}{0.115pt}}
\multiput(1388.00,742.17)(8.966,5.000){2}{\rule{0.490pt}{0.400pt}}
\put(151.0,131.0){\rule[-0.200pt]{0.400pt}{175.375pt}}
\put(151.0,131.0){\rule[-0.200pt]{310.279pt}{0.400pt}}
\put(1439.0,131.0){\rule[-0.200pt]{0.400pt}{175.375pt}}
\put(151.0,859.0){\rule[-0.200pt]{310.279pt}{0.400pt}}
\end{picture}

\caption{Závislosť energie pred $E^B$ a energie po zrážke $E^A$ pre jednotlivé kombinácie hmotností a štartovacích impulzov}  \label{G_2}
\end{figure}

\begin{figure}
% GNUPLOT: LaTeX picture
\setlength{\unitlength}{0.240900pt}
\ifx\plotpoint\undefined\newsavebox{\plotpoint}\fi
\begin{picture}(1500,900)(0,0)
\sbox{\plotpoint}{\rule[-0.200pt]{0.400pt}{0.400pt}}%
\put(171.0,131.0){\rule[-0.200pt]{4.818pt}{0.400pt}}
\put(151,131){\makebox(0,0)[r]{ 0.2}}
\put(1419.0,131.0){\rule[-0.200pt]{4.818pt}{0.400pt}}
\put(171.0,222.0){\rule[-0.200pt]{4.818pt}{0.400pt}}
\put(151,222){\makebox(0,0)[r]{ 0.3}}
\put(1419.0,222.0){\rule[-0.200pt]{4.818pt}{0.400pt}}
\put(171.0,313.0){\rule[-0.200pt]{4.818pt}{0.400pt}}
\put(151,313){\makebox(0,0)[r]{ 0.4}}
\put(1419.0,313.0){\rule[-0.200pt]{4.818pt}{0.400pt}}
\put(171.0,404.0){\rule[-0.200pt]{4.818pt}{0.400pt}}
\put(151,404){\makebox(0,0)[r]{ 0.5}}
\put(1419.0,404.0){\rule[-0.200pt]{4.818pt}{0.400pt}}
\put(171.0,495.0){\rule[-0.200pt]{4.818pt}{0.400pt}}
\put(151,495){\makebox(0,0)[r]{ 0.6}}
\put(1419.0,495.0){\rule[-0.200pt]{4.818pt}{0.400pt}}
\put(171.0,586.0){\rule[-0.200pt]{4.818pt}{0.400pt}}
\put(151,586){\makebox(0,0)[r]{ 0.7}}
\put(1419.0,586.0){\rule[-0.200pt]{4.818pt}{0.400pt}}
\put(171.0,677.0){\rule[-0.200pt]{4.818pt}{0.400pt}}
\put(151,677){\makebox(0,0)[r]{ 0.8}}
\put(1419.0,677.0){\rule[-0.200pt]{4.818pt}{0.400pt}}
\put(171.0,768.0){\rule[-0.200pt]{4.818pt}{0.400pt}}
\put(151,768){\makebox(0,0)[r]{ 0.9}}
\put(1419.0,768.0){\rule[-0.200pt]{4.818pt}{0.400pt}}
\put(171.0,859.0){\rule[-0.200pt]{4.818pt}{0.400pt}}
\put(151,859){\makebox(0,0)[r]{ 1}}
\put(1419.0,859.0){\rule[-0.200pt]{4.818pt}{0.400pt}}
\put(171.0,131.0){\rule[-0.200pt]{0.400pt}{4.818pt}}
\put(171,90){\makebox(0,0){ 0.2}}
\put(171.0,839.0){\rule[-0.200pt]{0.400pt}{4.818pt}}
\put(330.0,131.0){\rule[-0.200pt]{0.400pt}{4.818pt}}
\put(330,90){\makebox(0,0){ 0.3}}
\put(330.0,839.0){\rule[-0.200pt]{0.400pt}{4.818pt}}
\put(488.0,131.0){\rule[-0.200pt]{0.400pt}{4.818pt}}
\put(488,90){\makebox(0,0){ 0.4}}
\put(488.0,839.0){\rule[-0.200pt]{0.400pt}{4.818pt}}
\put(647.0,131.0){\rule[-0.200pt]{0.400pt}{4.818pt}}
\put(647,90){\makebox(0,0){ 0.5}}
\put(647.0,839.0){\rule[-0.200pt]{0.400pt}{4.818pt}}
\put(805.0,131.0){\rule[-0.200pt]{0.400pt}{4.818pt}}
\put(805,90){\makebox(0,0){ 0.6}}
\put(805.0,839.0){\rule[-0.200pt]{0.400pt}{4.818pt}}
\put(963.0,131.0){\rule[-0.200pt]{0.400pt}{4.818pt}}
\put(963,90){\makebox(0,0){ 0.7}}
\put(963.0,839.0){\rule[-0.200pt]{0.400pt}{4.818pt}}
\put(1122.0,131.0){\rule[-0.200pt]{0.400pt}{4.818pt}}
\put(1122,90){\makebox(0,0){ 0.8}}
\put(1122.0,839.0){\rule[-0.200pt]{0.400pt}{4.818pt}}
\put(1281.0,131.0){\rule[-0.200pt]{0.400pt}{4.818pt}}
\put(1281,90){\makebox(0,0){ 0.9}}
\put(1281.0,839.0){\rule[-0.200pt]{0.400pt}{4.818pt}}
\put(1439.0,131.0){\rule[-0.200pt]{0.400pt}{4.818pt}}
\put(1439,90){\makebox(0,0){ 1}}
\put(1439.0,839.0){\rule[-0.200pt]{0.400pt}{4.818pt}}
\put(171.0,131.0){\rule[-0.200pt]{0.400pt}{175.375pt}}
\put(171.0,131.0){\rule[-0.200pt]{305.461pt}{0.400pt}}
\put(1439.0,131.0){\rule[-0.200pt]{0.400pt}{175.375pt}}
\put(171.0,859.0){\rule[-0.200pt]{305.461pt}{0.400pt}}
\put(30,495){\makebox(0,0){\popi{I_k}{mA}}}
\put(805,29){\makebox(0,0){\popi{I_x}{mA}}}
\put(1211,819){\makebox(0,0)[r]{Namerané hodnoty}}
\put(1376,785){\makebox(0,0){$+$}}
\put(1281,739){\makebox(0,0){$+$}}
\put(1185,687){\makebox(0,0){$+$}}
\put(1122,659){\makebox(0,0){$+$}}
\put(963,578){\makebox(0,0){$+$}}
\put(932,557){\makebox(0,0){$+$}}
\put(837,509){\makebox(0,0){$+$}}
\put(488,310){\makebox(0,0){$+$}}
\put(330,219){\makebox(0,0){$+$}}
\put(171,133){\makebox(0,0){$+$}}
\put(1281,819){\makebox(0,0){$+$}}
\put(1211,778){\makebox(0,0)[r]{Preložená závyslosť $I_k = "0.95"\cdot I_x +"0.017 mA"$}}
\multiput(1231,778)(20.756,0.000){5}{\usebox{\plotpoint}}
\put(1331,778){\usebox{\plotpoint}}
\put(171,137){\usebox{\plotpoint}}
\put(171.00,137.00){\usebox{\plotpoint}}
\put(189.14,147.07){\usebox{\plotpoint}}
\put(207.50,156.73){\usebox{\plotpoint}}
\put(225.64,166.82){\usebox{\plotpoint}}
\put(243.78,176.87){\usebox{\plotpoint}}
\put(261.91,186.96){\usebox{\plotpoint}}
\put(280.28,196.61){\usebox{\plotpoint}}
\put(298.41,206.70){\usebox{\plotpoint}}
\put(316.56,216.75){\usebox{\plotpoint}}
\put(334.90,226.44){\usebox{\plotpoint}}
\put(353.06,236.49){\usebox{\plotpoint}}
\put(371.42,246.16){\usebox{\plotpoint}}
\put(389.34,256.62){\usebox{\plotpoint}}
\put(407.68,266.32){\usebox{\plotpoint}}
\put(425.84,276.37){\usebox{\plotpoint}}
\put(444.20,286.03){\usebox{\plotpoint}}
\put(462.52,295.76){\usebox{\plotpoint}}
\put(480.47,306.19){\usebox{\plotpoint}}
\put(498.98,315.53){\usebox{\plotpoint}}
\put(516.95,325.89){\usebox{\plotpoint}}
\put(535.27,335.63){\usebox{\plotpoint}}
\put(553.22,346.05){\usebox{\plotpoint}}
\put(571.72,355.41){\usebox{\plotpoint}}
\put(589.88,365.44){\usebox{\plotpoint}}
\put(608.05,375.45){\usebox{\plotpoint}}
\put(626.15,385.58){\usebox{\plotpoint}}
\put(644.54,395.21){\usebox{\plotpoint}}
\put(662.65,405.33){\usebox{\plotpoint}}
\put(680.83,415.32){\usebox{\plotpoint}}
\put(698.93,425.46){\usebox{\plotpoint}}
\put(717.31,435.09){\usebox{\plotpoint}}
\put(735.68,444.73){\usebox{\plotpoint}}
\put(753.61,455.19){\usebox{\plotpoint}}
\put(771.95,464.89){\usebox{\plotpoint}}
\put(790.09,474.97){\usebox{\plotpoint}}
\put(808.47,484.61){\usebox{\plotpoint}}
\put(826.39,495.06){\usebox{\plotpoint}}
\put(844.73,504.76){\usebox{\plotpoint}}
\put(862.87,514.85){\usebox{\plotpoint}}
\put(881.25,524.48){\usebox{\plotpoint}}
\put(899.54,534.27){\usebox{\plotpoint}}
\put(917.52,544.63){\usebox{\plotpoint}}
\put(935.98,554.07){\usebox{\plotpoint}}
\put(954.00,564.34){\usebox{\plotpoint}}
\put(972.28,574.14){\usebox{\plotpoint}}
\put(990.27,584.49){\usebox{\plotpoint}}
\put(1008.72,593.95){\usebox{\plotpoint}}
\put(1026.89,603.95){\usebox{\plotpoint}}
\put(1045.10,613.89){\usebox{\plotpoint}}
\put(1063.17,624.09){\usebox{\plotpoint}}
\put(1081.57,633.69){\usebox{\plotpoint}}
\put(1099.67,643.83){\usebox{\plotpoint}}
\put(1117.88,653.76){\usebox{\plotpoint}}
\put(1135.94,663.97){\usebox{\plotpoint}}
\put(1154.34,673.57){\usebox{\plotpoint}}
\put(1172.73,683.18){\usebox{\plotpoint}}
\put(1190.66,693.64){\usebox{\plotpoint}}
\put(1209.00,703.33){\usebox{\plotpoint}}
\put(1227.12,713.45){\usebox{\plotpoint}}
\put(1245.52,723.05){\usebox{\plotpoint}}
\put(1263.44,733.51){\usebox{\plotpoint}}
\put(1281.78,743.21){\usebox{\plotpoint}}
\put(1300.21,752.71){\usebox{\plotpoint}}
\put(1318.27,762.91){\usebox{\plotpoint}}
\put(1336.53,772.76){\usebox{\plotpoint}}
\put(1354.54,783.07){\usebox{\plotpoint}}
\put(1372.95,792.59){\usebox{\plotpoint}}
\put(1376,794){\usebox{\plotpoint}}
\put(171.0,131.0){\rule[-0.200pt]{0.400pt}{175.375pt}}
\put(171.0,131.0){\rule[-0.200pt]{305.461pt}{0.400pt}}
\put(1439.0,131.0){\rule[-0.200pt]{0.400pt}{175.375pt}}
\put(171.0,859.0){\rule[-0.200pt]{305.461pt}{0.400pt}}
\end{picture}

\caption{Závislosť impulzu sily $I$ od zmeny hybnosti $p$ vyvolané a lineárny fit $\(1.43\pm0.07\)x -\(0.039\pm0.015\) $}  \label{G_3}
\end{figure}


\begin{figure}
% GNUPLOT: LaTeX picture
\setlength{\unitlength}{0.240900pt}
\ifx\plotpoint\undefined\newsavebox{\plotpoint}\fi
\begin{picture}(1500,900)(0,0)
\sbox{\plotpoint}{\rule[-0.200pt]{0.400pt}{0.400pt}}%
\put(191.0,131.0){\rule[-0.200pt]{4.818pt}{0.400pt}}
\put(171,131){\makebox(0,0)[r]{ 0}}
\put(1358.0,131.0){\rule[-0.200pt]{4.818pt}{0.400pt}}
\put(191.0,194.0){\rule[-0.200pt]{4.818pt}{0.400pt}}
\put(171,194){\makebox(0,0)[r]{ 0.02}}
\put(1358.0,194.0){\rule[-0.200pt]{4.818pt}{0.400pt}}
\put(191.0,257.0){\rule[-0.200pt]{4.818pt}{0.400pt}}
\put(171,257){\makebox(0,0)[r]{ 0.04}}
\put(1358.0,257.0){\rule[-0.200pt]{4.818pt}{0.400pt}}
\put(191.0,320.0){\rule[-0.200pt]{4.818pt}{0.400pt}}
\put(171,320){\makebox(0,0)[r]{ 0.06}}
\put(1358.0,320.0){\rule[-0.200pt]{4.818pt}{0.400pt}}
\put(191.0,383.0){\rule[-0.200pt]{4.818pt}{0.400pt}}
\put(171,383){\makebox(0,0)[r]{ 0.08}}
\put(1358.0,383.0){\rule[-0.200pt]{4.818pt}{0.400pt}}
\put(191.0,446.0){\rule[-0.200pt]{4.818pt}{0.400pt}}
\put(171,446){\makebox(0,0)[r]{ 0.1}}
\put(1358.0,446.0){\rule[-0.200pt]{4.818pt}{0.400pt}}
\put(191.0,509.0){\rule[-0.200pt]{4.818pt}{0.400pt}}
\put(171,509){\makebox(0,0)[r]{ 0.12}}
\put(1358.0,509.0){\rule[-0.200pt]{4.818pt}{0.400pt}}
\put(191.0,572.0){\rule[-0.200pt]{4.818pt}{0.400pt}}
\put(171,572){\makebox(0,0)[r]{ 0.14}}
\put(1358.0,572.0){\rule[-0.200pt]{4.818pt}{0.400pt}}
\put(191.0,131.0){\rule[-0.200pt]{0.400pt}{4.818pt}}
\put(191,90){\makebox(0,0){ 0.25}}
\put(191.0,552.0){\rule[-0.200pt]{0.400pt}{4.818pt}}
\put(299.0,131.0){\rule[-0.200pt]{0.400pt}{4.818pt}}
\put(299,90){\makebox(0,0){ 0.3}}
\put(299.0,552.0){\rule[-0.200pt]{0.400pt}{4.818pt}}
\put(407.0,131.0){\rule[-0.200pt]{0.400pt}{4.818pt}}
\put(407,90){\makebox(0,0){ 0.35}}
\put(407.0,552.0){\rule[-0.200pt]{0.400pt}{4.818pt}}
\put(515.0,131.0){\rule[-0.200pt]{0.400pt}{4.818pt}}
\put(515,90){\makebox(0,0){ 0.4}}
\put(515.0,552.0){\rule[-0.200pt]{0.400pt}{4.818pt}}
\put(623.0,131.0){\rule[-0.200pt]{0.400pt}{4.818pt}}
\put(623,90){\makebox(0,0){ 0.45}}
\put(623.0,552.0){\rule[-0.200pt]{0.400pt}{4.818pt}}
\put(731.0,131.0){\rule[-0.200pt]{0.400pt}{4.818pt}}
\put(731,90){\makebox(0,0){ 0.5}}
\put(731.0,552.0){\rule[-0.200pt]{0.400pt}{4.818pt}}
\put(838.0,131.0){\rule[-0.200pt]{0.400pt}{4.818pt}}
\put(838,90){\makebox(0,0){ 0.55}}
\put(838.0,552.0){\rule[-0.200pt]{0.400pt}{4.818pt}}
\put(946.0,131.0){\rule[-0.200pt]{0.400pt}{4.818pt}}
\put(946,90){\makebox(0,0){ 0.6}}
\put(946.0,552.0){\rule[-0.200pt]{0.400pt}{4.818pt}}
\put(1054.0,131.0){\rule[-0.200pt]{0.400pt}{4.818pt}}
\put(1054,90){\makebox(0,0){ 0.65}}
\put(1054.0,552.0){\rule[-0.200pt]{0.400pt}{4.818pt}}
\put(1162.0,131.0){\rule[-0.200pt]{0.400pt}{4.818pt}}
\put(1162,90){\makebox(0,0){ 0.7}}
\put(1162.0,552.0){\rule[-0.200pt]{0.400pt}{4.818pt}}
\put(1270.0,131.0){\rule[-0.200pt]{0.400pt}{4.818pt}}
\put(1270,90){\makebox(0,0){ 0.75}}
\put(1270.0,552.0){\rule[-0.200pt]{0.400pt}{4.818pt}}
\put(1378.0,131.0){\rule[-0.200pt]{0.400pt}{4.818pt}}
\put(1378,90){\makebox(0,0){ 0.8}}
\put(1378.0,552.0){\rule[-0.200pt]{0.400pt}{4.818pt}}
\put(191.0,131.0){\rule[-0.200pt]{0.400pt}{106.237pt}}
\put(191.0,131.0){\rule[-0.200pt]{285.948pt}{0.400pt}}
\put(1378.0,131.0){\rule[-0.200pt]{0.400pt}{106.237pt}}
\put(191.0,572.0){\rule[-0.200pt]{285.948pt}{0.400pt}}
\put(30,351){\makebox(0,0){\popi{\sigma_v}{-}}}
\put(1417,351){\makebox(0,0){\popi{\Delta v}{m/s}}}
\put(784,29){\makebox(0,0){\popi{v}{m/s}}}
\put(644,860){\makebox(0,0)[r]{rel. ch., č.1 ,\uv{V}}}
\put(946,418){\makebox(0,0){$+$}}
\put(714,860){\makebox(0,0){$+$}}
\put(644,819){\makebox(0,0)[r]{abs. ch., č.1, \uv{V} }}
\put(946,498){\makebox(0,0){$\times$}}
\put(714,819){\makebox(0,0){$\times$}}
\sbox{\plotpoint}{\rule[-0.400pt]{0.800pt}{0.800pt}}%
\sbox{\plotpoint}{\rule[-0.200pt]{0.400pt}{0.400pt}}%
\put(644,778){\makebox(0,0)[r]{rel.ch., č.1, \uv{S}}}
\sbox{\plotpoint}{\rule[-0.400pt]{0.800pt}{0.800pt}}%
\put(644,565){\makebox(0,0){$\ast$}}
\put(714,778){\makebox(0,0){$\ast$}}
\sbox{\plotpoint}{\rule[-0.500pt]{1.000pt}{1.000pt}}%
\sbox{\plotpoint}{\rule[-0.200pt]{0.400pt}{0.400pt}}%
\put(644,737){\makebox(0,0)[r]{abs. ch., č.1, \uv{S}}}
\sbox{\plotpoint}{\rule[-0.500pt]{1.000pt}{1.000pt}}%
\put(644,572){\raisebox{-.8pt}{\makebox(0,0){$\Box$}}}
\put(714,737){\raisebox{-.8pt}{\makebox(0,0){$\Box$}}}
\sbox{\plotpoint}{\rule[-0.600pt]{1.200pt}{1.200pt}}%
\sbox{\plotpoint}{\rule[-0.200pt]{0.400pt}{0.400pt}}%
\put(644,696){\makebox(0,0)[r]{rel. ch., č.1, \uv{M}}}
\sbox{\plotpoint}{\rule[-0.600pt]{1.200pt}{1.200pt}}%
\put(234,353){\makebox(0,0){$\blacksquare$}}
\put(714,696){\makebox(0,0){$\blacksquare$}}
\sbox{\plotpoint}{\rule[-0.500pt]{1.000pt}{1.000pt}}%
\sbox{\plotpoint}{\rule[-0.200pt]{0.400pt}{0.400pt}}%
\put(644,655){\makebox(0,0)[r]{abs. ch., č.1, \uv{M}}}
\sbox{\plotpoint}{\rule[-0.500pt]{1.000pt}{1.000pt}}%
\put(234,199){\makebox(0,0){$\circ$}}
\put(714,655){\makebox(0,0){$\circ$}}
\sbox{\plotpoint}{\rule[-0.200pt]{0.400pt}{0.400pt}}%
\put(1164,860){\makebox(0,0)[r]{rel. ch., č.2, \uv{V}}}
\put(1292,228){\makebox(0,0){$\bullet$}}
\put(1234,860){\makebox(0,0){$\bullet$}}
\put(1164,819){\makebox(0,0)[r]{abs. ch., č.2, \uv{V}}}
\put(1292,236){\makebox(0,0){$\triangle$}}
\put(1234,819){\makebox(0,0){$\triangle$}}
\sbox{\plotpoint}{\rule[-0.400pt]{0.800pt}{0.800pt}}%
\sbox{\plotpoint}{\rule[-0.200pt]{0.400pt}{0.400pt}}%
\put(1164,778){\makebox(0,0)[r]{rel. ch., č.2, \uv{S}}}
\sbox{\plotpoint}{\rule[-0.400pt]{0.800pt}{0.800pt}}%
\put(838,194){\makebox(0,0){$\blacktriangle$}}
\put(1234,778){\makebox(0,0){$\blacktriangle$}}
\sbox{\plotpoint}{\rule[-0.500pt]{1.000pt}{1.000pt}}%
\sbox{\plotpoint}{\rule[-0.200pt]{0.400pt}{0.400pt}}%
\put(1164,737){\makebox(0,0)[r]{abs. ch., č.2, \uv{S}}}
\sbox{\plotpoint}{\rule[-0.500pt]{1.000pt}{1.000pt}}%
\put(838,131){\makebox(0,0){$\triangledown$}}
\put(1234,737){\makebox(0,0){$\triangledown$}}
\sbox{\plotpoint}{\rule[-0.600pt]{1.200pt}{1.200pt}}%
\sbox{\plotpoint}{\rule[-0.200pt]{0.400pt}{0.400pt}}%
\put(1164,696){\makebox(0,0)[r]{rel. ch., č.2, \uv{M}}}
\sbox{\plotpoint}{\rule[-0.600pt]{1.200pt}{1.200pt}}%
\put(342,328){\makebox(0,0){$\blacktriangledown$}}
\put(1234,696){\makebox(0,0){$\blacktriangledown$}}
\sbox{\plotpoint}{\rule[-0.500pt]{1.000pt}{1.000pt}}%
\sbox{\plotpoint}{\rule[-0.200pt]{0.400pt}{0.400pt}}%
\put(1164,655){\makebox(0,0)[r]{abs. ch., č.2, \uv{M}}}
\sbox{\plotpoint}{\rule[-0.500pt]{1.000pt}{1.000pt}}%
\put(342,207){\makebox(0,0){$\lozenge$}}
\put(1234,655){\makebox(0,0){$\lozenge$}}
\sbox{\plotpoint}{\rule[-0.200pt]{0.400pt}{0.400pt}}%
\put(191.0,131.0){\rule[-0.200pt]{0.400pt}{106.237pt}}
\put(191.0,131.0){\rule[-0.200pt]{285.948pt}{0.400pt}}
\put(1378.0,131.0){\rule[-0.200pt]{0.400pt}{106.237pt}}
\put(191.0,572.0){\rule[-0.200pt]{285.948pt}{0.400pt}}
\end{picture}

\caption{Závislosť rýchlosti $v$ na absolútnej (abs. ch.) $\Delta v$ a relatívnej (rel. chyb.) $\sigma_v$ chybe, kde \uv{V} je najväčšia \uv{S} je stredná a \uv{M} je najmenšia pozícia odpaľovacieho zariadenia, č. vyjadruje číslo vozíka.}  \label{G_4}
\end{figure}


\begin{table}[h]
\begin{center}
\begin{tabular}{| c | c | c | c | c | c | c | c | c | c |}
\hline
\popi{v_1^B}{m/s} & \popi{v_1^A}{m/s} & \popi{v_2^B}{m/s} & \popi{v_2^A}{m/s} & \popi{p^B}{mN\cdot s} & \popi{p^A}{mN\cdot s} & \popi{\Delta p}{mN\cdot s} & \popi{E^B}{mJ} & \popi{E^A}{mJ} & \popi{\Delta E}{mJ}\\
\hline
$"0.69"$ & $"0.08"$ & $"0.00"$ & $"0.67"$ & $"169.87"$ & $"160.25"$ & $"9.61"$ & $"58.18"$ & $"47.66"$ & $"10.52"$\\
$"0.56"$ & $"0.03"$ & $"0.00"$ & $"0.57"$ & $"138.62"$ & $"127.98"$ & $"10.64"$ & $"38.74"$ & $"34.49"$ & $"4.25"$\\
$"0.60"$ & $"0.07"$ & $"0.00"$ & $"0.59"$ & $"148.29"$ & $"139.98"$ & $"8.31"$ & $"44.34"$ & $"37.20"$ & $"7.14"$\\
$"0.64"$ & $"0.06"$ & $"0.00"$ & $"0.66"$ & $"159.70"$ & $"151.61"$ & $"8.09"$ & $"51.42"$ & $"45.69"$ & $"5.73"$\\
$"0.64"$ & $"0.05"$ & $"0.00"$ & $"0.64"$ & $"157.96"$ & $"145.74"$ & $"12.23"$ & $"50.31"$ & $"43.26"$ & $"7.05"$\\
$"0.58"$ & $"0.03"$ & $"0.00"$ & $"0.56"$ & $"143.83"$ & $"125.14"$ & $"18.68"$ & $"41.71"$ & $"33.17"$ & $"8.54"$\\
$"0.62"$ & $"0.06"$ & $"0.00"$ & $"0.64"$ & $"153.25"$ & $"148.64"$ & $"4.62"$ & $"47.35"$ & $"42.81"$ & $"4.54"$\\
$"0.52"$ & $"0.06"$ & $"0.00"$ & $"0.56"$ & $"128.45"$ & $"130.56"$ & $"-2.11"$ & $"33.27"$ & $"32.98"$ & $"0.29"$\\
$"0.52"$ & $"0.05"$ & $"0.00"$ & $"0.58"$ & $"128.45"$ & $"134.15"$ & $"-5.70"$ & $"33.27"$ & $"35.21"$ & $"-1.94"$\\
$"0.65"$ & $"0.06"$ & $"0.00"$ & $"0.63"$ & $"160.44"$ & $"146.35"$ & $"14.09"$ & $"51.90"$ & $"42.08"$ & $"9.83"$\\
$"0.64"$ & $"0.06"$ & $"0.00"$ & $"0.66"$ & $"157.96"$ & $"153.36"$ & $"4.60"$ & $"50.31"$ & $"46.29"$ & $"4.02"$\\
\hline
\end{tabular}
\caption{Namerané hodnoty rýchlosti pred zrážkou $v_1^B$ a $v_2^B$ a 
po zrážke $v_1^A$ a $v_2^A$ a z nich vypočítané hodnoty celkovej hybnosti 
pred $p^B$ a po zrážke $p^A$, rozdiel hybnosti $\Delta p$ a celkové energie 
pred $E^B$ a po $E^A$ zrážke a ich rozdiel $\Delta E$ pre 
\uv{najväčšiu} pozíciu štartovacieho zariadenia a naráža ťažší vozík do ľahšieho} \label{T_1}
\end{center}
\end{table}


\begin{table}[h]
\begin{center}
\begin{tabular}{| c | c | c | c | c | c | c | c | c | c |}
\hline
\popi{v_1^B}{m/s} & \popi{v_1^A}{m/s} & \popi{v_2^B}{m/s} & \popi{v_2^A}{m/s} & \popi{p^B}{mN\cdot s} & \popi{p^A}{mN\cdot s} & \popi{\Delta p}{mN\cdot s} & \popi{E^B}{mJ} & \popi{E^A}{mJ} & \popi{\Delta E}{mJ}\\
\hline
$"0.31"$ & $"0.02"$ & $"0.00"$ & $"0.32"$ & $"76.38"$ & $"71.52"$ & $"4.86"$ & $"11.76"$ & $"10.41"$ & $"1.35"$\\
$"0.46"$ & $"0.04"$ & $"0.00"$ & $"0.46"$ & $"113.57"$ & $"106.19"$ & $"7.39"$ & $"26.01"$ & $"22.63"$ & $"3.38"$\\
$"0.52"$ & $"0.05"$ & $"0.00"$ & $"0.52"$ & $"128.45"$ & $"119.79"$ & $"8.66"$ & $"33.27"$ & $"28.16"$ & $"5.11"$\\
$"0.50"$ & $"0.04"$ & $"0.00"$ & $"0.51"$ & $"122.75"$ & $"115.09"$ & $"7.66"$ & $"30.38"$ & $"26.88"$ & $"3.50"$\\
$"0.46"$ & $"0.04"$ & $"0.00"$ & $"0.47"$ & $"114.81"$ & $"107.74"$ & $"7.07"$ & $"26.58"$ & $"23.31"$ & $"3.27"$\\
$"0.48"$ & $"0.05"$ & $"0.00"$ & $"0.49"$ & $"118.53"$ & $"113.61"$ & $"4.93"$ & $"28.33"$ & $"25.21"$ & $"3.12"$\\
$"0.36"$ & $"0.02"$ & $"0.00"$ & $"0.38"$ & $"90.02"$ & $"83.76"$ & $"6.26"$ & $"16.34"$ & $"14.73"$ & $"1.61"$\\
$"0.49"$ & $"0.04"$ & $"0.00"$ & $"0.50"$ & $"120.77"$ & $"113.64"$ & $"7.12"$ & $"29.41"$ & $"25.77"$ & $"3.64"$\\
$"0.48"$ & $"0.05"$ & $"0.00"$ & $"0.50"$ & $"119.53"$ & $"116.89"$ & $"2.64"$ & $"28.81"$ & $"26.11"$ & $"2.70"$\\
$"0.46"$ & $"0.03"$ & $"0.00"$ & $"0.47"$ & $"114.32"$ & $"105.91"$ & $"8.40"$ & $"26.35"$ & $"23.07"$ & $"3.28"$\\
$"0.50"$ & $"0.05"$ & $"0.00"$ & $"0.52"$ & $"124.24"$ & $"119.57"$ & $"4.66"$ & $"31.12"$ & $"27.95"$ & $"3.17"$\\
\hline
\end{tabular}
\caption{Namerané hodnoty rýchlosti pred zrážkou $v_1^B$ a $v_2^B$ a 
po zrážke $v_1^A$ a $v_2^A$ a z nich vypočítané hodnoty celkovej hybnosti 
pred $p^B$ a po zrážke $p^A$, rozdiel hybnosti $\Delta p$ a celkové energie 
pred $E^B$ a po $E^A$ zrážke a ich rozdiel $\Delta E$ pre 
\uv{strednú} pozíciu štartovacieho zariadenia a naráža ťažší vozík do ľahšieho } \label{T_2}
\end{center}
\end{table}



\begin{table}[h]
\begin{center}
\begin{tabular}{| c | c | c | c | c | c | c | c | c | c |}
\hline
\popi{v_1^B}{m/s} & \popi{v_1^A}{m/s} & \popi{v_2^B}{m/s} & \popi{v_2^A}{m/s} & \popi{p^B}{mN\cdot s} & \popi{p^A}{mN\cdot s} & \popi{\Delta p}{mN\cdot s} & \popi{E^B}{mJ} & \popi{E^A}{mJ} & \popi{\Delta E}{mJ}\\
\hline
$"0.29"$ & $"0.02"$ & $"0.00"$ & $"0.25"$ & $"72.41"$ & $"57.51"$ & $"14.90"$ & $"10.57"$ & $"6.67"$ & $"3.90"$\\
$"0.25"$ & $"0.03"$ & $"0.00"$ & $"0.25"$ & $"62.24"$ & $"59.81"$ & $"2.43"$ & $"7.81"$ & $"6.82"$ & $"0.99"$\\
$"0.30"$ & $"0.03"$ & $"0.00"$ & $"0.25"$ & $"74.89"$ & $"59.25"$ & $"15.64"$ & $"11.31"$ & $"6.39"$ & $"4.92"$\\
$"0.28"$ & $"0.02"$ & $"0.00"$ & $"0.30"$ & $"70.43"$ & $"69.16"$ & $"1.26"$ & $"10.00"$ & $"9.64"$ & $"0.36"$\\
$"0.27"$ & $"0.03"$ & $"0.00"$ & $"0.28"$ & $"66.71"$ & $"63.77"$ & $"2.94"$ & $"8.97"$ & $"7.97"$ & $"1.00"$\\
$"0.30"$ & $"0.04"$ & $"0.00"$ & $"0.31"$ & $"74.64"$ & $"73.75"$ & $"0.89"$ & $"11.23"$ & $"10.19"$ & $"1.05"$\\
$"0.28"$ & $"0.03"$ & $"0.00"$ & $"0.29"$ & $"68.19"$ & $"67.08"$ & $"1.11"$ & $"9.38"$ & $"8.64"$ & $"0.74"$\\
$"0.26"$ & $"0.02"$ & $"0.00"$ & $"0.25"$ & $"65.47"$ & $"57.31"$ & $"8.15"$ & $"8.64"$ & $"6.43"$ & $"2.21"$\\
$"0.28"$ & $"0.03"$ & $"0.00"$ & $"0.26"$ & $"68.69"$ & $"61.05"$ & $"7.64"$ & $"9.51"$ & $"7.09"$ & $"2.42"$\\
$"0.25"$ & $"0.03"$ & $"0.00"$ & $"0.26"$ & $"61.00"$ & $"60.50"$ & $"0.50"$ & $"7.50"$ & $"6.94"$ & $"0.57"$\\
$"0.26"$ & $"0.02"$ & $"0.00"$ & $"0.27"$ & $"64.23"$ & $"62.24"$ & $"1.98"$ & $"8.32"$ & $"7.72"$ & $"0.59"$\\
\hline
\end{tabular}
\caption{Namerané hodnoty rýchlosti pred zrážkou $v_1^B$ a $v_2^B$ a 
po zrážke $v_1^A$ a $v_2^A$ a z nich vypočítané hodnoty celkovej hybnosti 
pred $p^B$ a po zrážke $p^A$, rozdiel hybnosti $\Delta p$ a celkové energie 
pred $E^B$ a po $E^A$ zrážke a ich rozdiel $\Delta E$ pre 
\uv{najmenšiu} pozíciu štartovacieho zariadenia a naráža ťažší vozík do ľahšieho} \label{T_3}
\end{center}
\end{table}






\begin{table}[h]
\begin{center}
\begin{tabular}{| c | c | c | c | c | c | c | c | c | c |}
\hline
\popi{v_2^B}{m/s} & \popi{v_2^A}{m/s} & \popi{v_1^B}{m/s} & \popi{v_1^A}{m/s} & \popi{p^B}{mN\cdot s} & \popi{p^A}{mN\cdot s} & \popi{\Delta p}{mN\cdot s} & \popi{E^B}{mJ} & \popi{E^A}{mJ} & \popi{\Delta E}{mJ}\\
\hline
$"0.743"$ & $"0.0552"$ & $"0"$ & $"0.653"$ & $"154.95"$ & $"150.42"$ & $"4.53"$ & $"57.56"$ & $"53.19"$ & $"4.37"$\\
$"0.710"$ & $"0.0396"$ & $"0"$ & $"0.602"$ & $"148.06"$ & $"141.03"$ & $"7.04"$ & $"52.56"$ & $"45.10"$ & $"7.46"$\\
$"0.771"$ & $"0.0439"$ & $"0"$ & $"0.670"$ & $"160.78"$ & $"156.99"$ & $"3.79"$ & $"61.98"$ & $"55.86"$ & $"6.12"$\\
$"0.753"$ & $"0.0162"$ & $"0"$ & $"0.635"$ & $"157.03"$ & $"154.09"$ & $"2.94"$ & $"59.12"$ & $"50.02"$ & $"9.10"$\\
$"0.743"$ & $"0.0760"$ & $"0"$ & $"0.674"$ & $"154.95"$ & $"151.29"$ & $"3.66"$ & $"57.56"$ & $"56.93"$ & $"0.63"$\\
$"0.785"$ & $"0.0555"$ & $"0"$ & $"0.664"$ & $"163.70"$ & $"153.08"$ & $"10.62"$ & $"64.25"$ & $"54.99"$ & $"9.27"$\\
$"0.755"$ & $"0.0293"$ & $"0"$ & $"0.553"$ & $"157.45"$ & $"131.02"$ & $"26.42"$ & $"59.44"$ & $"38.01"$ & $"21.43"$\\
$"0.782"$ & $"0.0421"$ & $"0"$ & $"0.666"$ & $"163.08"$ & $"156.38"$ & $"6.70"$ & $"63.76"$ & $"55.18"$ & $"8.58"$\\
$"0.776"$ & $"0.0430"$ & $"0"$ & $"0.713"$ & $"161.83"$ & $"167.84"$ & $"-6.02"$ & $"62.79"$ & $"63.23"$ & $"-0.44"$\\
$"0.778"$ & $"0.051"$ & $"0"$ & $"0.673"$ &  $"162.24"$ & $"156.26"$ & $"5.99"$ & $"63.11"$ & $"56.43"$ & $"6.68"$\\
\hline
\end{tabular}
\caption{Namerané hodnoty rýchlosti pred zrážkou $v_1^B$ a $v_2^B$ a 
po zrážke $v_1^A$ a $v_2^A$ a z nich vypočítané hodnoty celkovej hybnosti 
pred $p^B$ a po zrážke $p^A$, rozdiel hybnosti $\Delta p$ a celkové energie 
pred $E^B$ a po $E^A$ zrážke a ich rozdiel $\Delta E$ pre \uv{najväčšiu} pozíciu štartovacieho zariadenia a naráža ľahší vozík do ťažšieho} \label{T_4}
\end{center}
\end{table}


\begin{table}[h]
\begin{center}
\begin{tabular}{| c | c | c | c | c | c | c | c | c | c |}
\hline
\popi{v_2^B}{m/s} & \popi{v_2^A}{m/s} & \popi{v_1^B}{m/s} & \popi{v_1^A}{m/s} & \popi{p^B}{mN\cdot s} & \popi{p^A}{mN\cdot s} & \popi{\Delta p}{mN\cdot s} & \popi{E^B}{mJ} & \popi{E^A}{mJ} & \popi{\Delta E}{mJ}\\
\hline
$"0.555"$ & $"0.0425"$ & $"0"$ & $"0.538"$ & $"115.74"$ & $"124.55"$ & $"-8.81"$ & $"32.12"$ & $"36.08"$ & $"-3.96"$\\
$"0.567"$ & $"0.0378"$ & $"0"$ & $"0.486"$ & $"118.24"$ & $"112.64"$ & $"5.61"$ & $"33.52"$ & $"29.43"$ & $"4.09"$\\
$"0.556"$ & $"0.0415"$ & $"0"$ & $"0.467"$ & $"115.95"$ & $"107.15"$ & $"8.80"$ & $"32.23"$ & $"27.22"$ & $"5.01"$\\
$"0.559"$ & $"0.0410"$ & $"0"$ & $"0.473"$ & $"116.57"$ & $"108.74"$ & $"7.83"$ & $"32.58"$ & $"27.92"$ & $"4.67"$\\
$"0.552"$ & $"0.0409"$ & $"0"$ & $"0.442"$ & $"115.11"$ & $"101.08"$ & $"14.04"$ & $"31.77"$ & $"24.40"$ & $"7.37"$\\
$"0.543"$ & $"0.0505"$ & $"0"$ & $"0.447"$ & $"113.24"$ & $"100.32"$ & $"12.92"$ & $"30.74"$ & $"25.04"$ & $"5.70"$\\
$"0.571"$ & $"0.0343"$ & $"0"$ & $"0.482"$ & $"119.08"$ & $"112.37"$ & $"6.70"$ & $"34.00"$ & $"28.93"$ & $"5.07"$\\
$"0.544"$ & $"0.0332"$ & $"0"$ & $"0.478"$ & $"113.45"$ & $"111.61"$ & $"1.83"$ & $"30.86"$ & $"28.44"$ & $"2.41"$\\
$"0.535"$ & $"0.0202"$ & $"0"$ & $"0.459"$ & $"111.57"$ & $"109.61"$ & $"1.96"$ & $"29.84"$ & $"26.16"$ & $"3.68"$\\
$"0.550"$ & $"0.0356"$ & $"0"$ & $"0.461"$ & $"114.70"$ & $"106.89"$ & $"7.80"$ & $"31.54"$ & $"26.48"$ & $"5.06"$\\
\hline
\end{tabular}
\caption{Namerané hodnoty rýchlosti pred zrážkou $v_1^B$ a $v_2^B$ a 
po zrážke $v_1^A$ a $v_2^A$ a z nich vypočítané hodnoty celkovej hybnosti 
pred $p^B$ a po zrážke $p^A$, rozdiel hybnosti $\Delta p$ a celkové energie 
pred $E^B$ a po $E^A$ zrážke a ich rozdiel $\Delta E$ pre \uv{strednú} pozíciu štartovacieho zariadenia a naráža ľahší vozík do ťažšieho} \label{T_5}
\end{center}
\end{table}


\begin{table}[h]
\begin{center}
\begin{tabular}{| c | c | c | c | c | c | c | c | c | c |}
\hline
\popi{v_2^B}{m/s} & \popi{v_2^A}{m/s} & \popi{v_1^B}{m/s} & \popi{v_1^A}{m/s} & \popi{p^B}{mN\cdot s} & \popi{p^A}{mN\cdot s} & \popi{\Delta p}{mN\cdot s} & \popi{E^B}{mJ} & \popi{E^A}{mJ} & \popi{\Delta E}{mJ}\\
\hline
$"0.316"$ & $"0.0335"$ & $"0"$ & $"0.272"$ & $"65.90"$ & $"74.44"$ & $"-8.54"$ & $"10.41"$ & $"9.29"$ & $"1.12"$\\
$"0.337"$ & $"0.0061"$ & $"0"$ & $"0.292"$ & $"70.28"$ & $"73.68"$ & $"-3.40"$ & $"11.84"$ & $"10.58"$ & $"1.27"$\\
$"0.328"$ & $"0.0251"$ & $"0"$ & $"0.272"$ & $"68.40"$ & $"72.68"$ & $"-4.28"$ & $"11.22"$ & $"9.24"$ & $"1.98"$\\
$"0.270"$ & $"0.0351"$ & $"0"$ & $"0.144"$ & $"56.31"$ & $"43.03"$ & $"13.28"$ & $"7.60"$ & $"2.70"$ & $"4.90"$\\
$"0.334"$ & $"0.0254"$ & $"0"$ & $"0.299"$ & $"69.65"$ & $"79.44"$ & $"-9.79"$ & $"11.63"$ & $"11.15"$ & $"0.48"$\\
$"0.335"$ & $"0.0399"$ & $"0"$ & $"0.293"$ & $"69.86"$ & $"80.98"$ & $"-11.12"$ & $"11.70"$ & $"10.81"$ & $"0.89"$\\
$"0.333"$ & $"0.0378"$ & $"0"$ & $"0.272"$ & $"69.44"$ & $"75.33"$ & $"-5.89"$ & $"11.56"$ & $"9.32"$ & $"2.24"$\\
$"0.338"$ & $"0.0381"$ & $"0"$ & $"0.294"$ & $"70.49"$ & $"80.85"$ & $"-10.36"$ & $"11.91"$ & $"10.87"$ & $"1.04"$\\
$"0.325"$ & $"0.0433"$ & $"0"$ & $"0.299"$ & $"67.78"$ & $"83.18"$ & $"-15.40"$ & $"11.01"$ & $"11.28"$ & $"-0.27"$\\
$"0.312"$ & $"0.0118"$ & $"0"$ & $"0.273"$ & $"65.06"$ & $"70.16"$ & $"-5.09"$ & $"10.15"$ & $"9.26"$ & $"0.89"$\\
\hline
\end{tabular}
\caption{Namerané hodnoty rýchlosti pred zrážkou $v_1^B$ a $v_2^B$ a 
po zrážke $v_1^A$ a $v_2^A$ a z nich vypočítané hodnoty celkovej hybnosti 
pred $p^B$ a po zrážke $p^A$, rozdiel hybnosti $\Delta p$ a celkové energie 
pred $E^B$ a po $E^A$ zrážke a ich rozdiel $\Delta E$
pre \uv{najmenšiu} pozíciu štartovacieho zariadenia
a naráža lahší vozík do ťažšieho} \label{T_6}
\end{center}
\end{table}


\begin{table}[h]
\begin{center}
\begin{tabular}{| c | c | c | c | c | c | }
\hline
typ & \popi{v^B}{m/s} & \popi{v^A}{m/s} &\popi{I}{N\cdot s} & \popi{p^B}{mN\cdot s} & \popi{p^A}{mN\cdot s}\\
\hline
n & $"0.790"$ & $"0.321"$ & $"0.2175"$ & $"0.16"$ & $"0.067"$\\
n & $"0.781"$ & $"0.305"$ & $"0.2145"$ & $"0.16"$ & $"0.063"$\\
n & $"0.776"$ & $"0.276"$ & $"0.2205"$ & $"0.16"$ & $"0.057"$\\
n & $"0.762"$ & $"0.275"$ & $"0.2089"$ & $"0.15"$ & $"0.057"$\\
n & $"0.795"$ & $"0.292"$ & $"0.2222"$ & $"0.16"$ & $"0.061"$\\
s & $"0.557"$ & $"0.217"$ & $"0.1562"$ & $"0.11"$ & $"0.045"$\\
s & $"0.595"$ & $"0.219"$ & $"0.1673"$ & $"0.12"$ & $"0.045"$\\
s & $"0.577"$ & $"0.209"$ & $"0.1617"$ & $"0.12"$ & $"0.043"$\\
s & $"0.491"$ & $"0.193"$ & $"0.1372"$ & $"0.10"$ & $"0.040"$\\
s & $"0.580"$ & $"0.224"$ & $"0.1638"$ & $"0.12"$ & $"0.046"$\\
m & $"0.300"$ & $"0.141"$ & $"0.0872"$ & $"0.06"$ & $"0.029"$\\
m & $"0.313"$ & $"0.142"$ & $"0.0896"$ & $"0.06"$ & $"0.029"$\\
m & $"0.285"$ & $"0.123"$ & $"0.0804"$ & $"0.05"$ & $"0.025"$\\
m & $"0.351"$ & $"0.151"$ & $"0.0957"$ & $"0.07"$ & $"0.031"$\\
m & $"0.262"$ & $"0.125"$ & $"0.0760"$ & $"0.05"$ & $"0.026"$\\
\hline
\end{tabular}
\caption{Namerané hodnoty rýchlosti pred nárazom $v^B$  a 
po náraze $v^A$ a z nich vypočítané hodnoty celkovej hybnosti 
pred $p^B$ a po zrážke $p^A$, vypočítaný impulz sily $I$.} \label{T_7}
\end{center}
\end{table}

\begin{table}[h]
\begin{center}
\begin{tabular}{| c | c | }
\hline
Tabuľka & \popi{v_1^B}{m/s}\\
\hline
Tab. \ref{T_1} & $"0.60\pm0.05"$\\
Tab. \ref{T_1} & $"0.46\pm0.06"$\\
Tab. \ref{T_1} & $"0.27\pm0.02"$\\
Tab. \ref{T_1} & $"0.76\pm0.02"$\\
Tab. \ref{T_1} & $"0.55\pm0.01"$\\
Tab. \ref{T_1} & $"0.32\pm0.02"$\\
\hline
\end{tabular}
\caption{Namerané hodnoty rýchlosti pred nárazom $v^B$  a 
po náraze $v^A$ a z nich vypočítané hodnoty celkovej hybnosti 
pred $p^B$ a po zrážke $p^A$, vypočítaný impulz sily $I$.} \label{T_8}
\end{center}
\end{table}




\section{Diskusia \& Záver}
Meraním sme zistili a vypočítali chyby resp. presnosť s akou sa dá určovať rýchlosť vozíčku.
Tieto hodnoty sú v tabuľke \ref{T_8}.

V našom meraní sú zdrojmi systematických chýb predovšetkým prehnutie dráhy, trenie vozíkov o dráhu. Prehnutie dráhy spôsobuje len veľmi malé zrýchlenie v porovnaní s rýchlosťami ktoré vozíky dosahujú.
Toto prehnutie sa v strednej časti dráhy kompenzovalo trením a tak pohyb sa stával RPP.

Odhad brzdného koeficientu dráha je dlhá $l=\sim "2 m"$ vozíček ju po odraze od tlakového senzoru skoro nedôjde a sem tam sa zastavil už na dráhe. 
Teda 
\eq{
a = \frac{v^2}{2l}\,,
}
po dosadení dostávame $a=\sim "0.004 m s^{-2}"$.

Z obrázkov Obr. \ref{G_1} a \ref{G_2}, že skoro via ako $60\%$ nameraných hodnôt padlo v $\sigma_0$ do predpokladanej závislosti a teda môžeme aj ZZE aj ZZH považovať za \textbf{nevyvrátený}.
Základný ma najväčším problémom tohoto merania je senzor. Na toto meranie je hlboko nevhodný kvôli svojej veľmi malej snímacej frekvencií pri náraze zaznamenal cca 7 hodnôt.
Na presné meranie by sme potrebovali väčšiu jemnosť dát. Teda impulz sily vychádza menší ako rozdiel hybnosti získavaných z rýchlostí. 





\clearpage

\section{Prílohy}

\begin{lstlisting}
<?php
$files = scandir(__DIR__);
echo '<pre>';
foreach ($files as $file) {
    if(is_dir($file)) continue;
    if($file == "..") continue;
    if($file == '.') continue;
    $info = pathinfo($file);
    if($info['extension'] != 'txt') continue;
    $content = str_replace(',','.',file_get_contents('./'.$file));
    $lines = explode("\n",$content);
    unset($lines[0],$lines[1]);
    $oldTime = 0;
    $sum = 0;
    foreach ($lines as $line) {
        if($line == "") continue;
        list ($newTime,$v) = explode("\t",$line);
        if($v > 0.5){
            $sum += ($newTime - $oldTime) * $v;
        }
        $oldTime = $newTime;
    }
    echo "$file\t$sum\n";
}
\end{lstlisting}



\begin{thebibliography}{2}
\bibitem{C_1}
Mechanické pokusy na vzduchové dráze [cit. 19.12.2016]Dostupné po prihlásení z Kurz: Fyzikální praktikum I:\url{https://praktikum.fjfi.cvut.cz/pluginfile.php/102/mod_resource/content/9/Vzduchova_draha-2015-Oct-01.pdf}

\end{thebibliography}

\end{document}


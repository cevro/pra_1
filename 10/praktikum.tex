\documentclass[a4paper,10pt]{article}
%\usepackage[IL2]{fontenc}
\usepackage[utf8x]{inputenc}
\usepackage[czech]{babel}
\usepackage{amsfonts,amsmath,amssymb,graphicx,color}
%\usepackage[total={17cm,27cm}, top=2cm, left=2cm, includefoot]{geometry}
%\usepackage{fancyhdr}
\usepackage{fkssugar}
\usepackage{hyperref}

%\usepackage{caption}
\renewcommand{\popi}[2]{$#1$[\jd{#2}]}
\renewcommand{\figurename}{Obr.}
\addto\captionsczech{\renewcommand{\figurename}{Obr.}}
\addto\captionsczech{\renewcommand{\tablename}{Tab.}}

\begin{document}
\def\mean#1{\left< #1 \right>}
\noindent
{\large Fyzikální praktikum 1.} \hfil {\large FJFI ČVUT V Praze}\\
\noindent
{\large\textbf{pracovní úkol \# 10}}
\begin{center}
{\large\textit{LHO}}
\end{center}
\noindent
\rule{\textwidth}{1px}
\vspace{\baselineskip}

\emph{Michal Červeňák}
\par
\vspace{\baselineskip}
\begin{minipage}[l]{0.5\textwidth}%
\textit{dátum merania:}~21.11. 2016\\%
%\vspace{\baselineskip}%
\par%
\noindent%
\textit{skupina:}~4\\%
%\vspace{\baselineskip}%
\par%
\noindent%
\textit{Klasifikace:}\dotfill\\%
\end{minipage}

\section{Pracovní úkol}

\begin{enumerate}
\item Změřrte tuhost pružiny statickou metodou a vypočtěte vlastní úhlovou frekvenci pro dvě různá
závaží.
\item Změřte časový průběh tlumených kmitů pro dvě závaží, ověřte platnost rovnice (14) proložením
dat a z parametrů proložení vypočtěte vlastní frekvenci volného oscilátoru.
\item Změřte závislost amplitudy vynucených kmitů na frekvenci vnější síly v okolí rezonance pro
dvě závaží a proložením dat ověřte platnost vztahu (19), z parametrů proložení vypočtěte vlasní
frekvenci volného oscilátoru.
\item Porovnejte výsledky vlastní frekvence ze všech tří předchozích úkolů
\item Ověřte, že (10) je řešením pohybové rovnice Pohlova kyvadla, nalezněte ´
řešení pro polohovou a rychlostní podmínku a nakreslete jejich průběh.
\item Naměřte časový vývoj výchylky kmitů kyvadla pro netlumené kmity a určete vlastní frekvenci
Pohlova kyvadla.
\item Změřte koeficient útlumu pro jednu hodnotu tlumícího proudu z intervalu (0,5 – 1,5)A.
\item Experimentálně nalezněte hodnotu tlumícího proudu, pro který nastává kritický útlum. Realizujte
polohovou i pohybovou podmínku a za pomoci domácího úkolu a úkolu 2. zjistěte, zda
platí podmínka $\omega_0 = \delta$.
\item Bonus: Vytvořte v DataStudiu fázový diagram. Prozkoumejte, jak vypadá pohyb kyvadla ve
fázovém diagramu za různých podmínek (netlumený, tlumený, kriticky tlumený). Grafy přiložte
k protokolu a diskutujte je


\end{enumerate}



\section{Pomôcky}
Experimentální stojan s pružinou a motorkem, tlumící magnety, rotační pohybové
senzory Pasco, sada závaží, regulovatelný zdroj $"0-20 V"$, voltmetr, digitální tachometr CEM AT-6,
PC, programy DataStudio a gnuplot, Pohlovo kyvadlo, nastavitelný zdroj $"0-3 A"$, rotační senzor PASCO, PC, program
DataStudio.


\section{Teória}
Pohlovo kyvadlo je tvorené medeným kotúčom a k menu pripevnená špirálovitá pružina, ktorá sa pri otáčaní skracuje alebo predlžuje. \cite{C_2}\textit{voľná citácia}

Podľa Hookovho zákona platí
\eq{
F = - kx \,, \lbl{R_1}
}
kde F je sila pôsobiaca na pružinu v našom prípade $F=mg$, 
kde $m$ je hmotnosť závaží, $g$ je tiažové zrýchlenie a $x$ je výchylka.

Závislosť uhlovej frekvencie $\omega_0$ na tuhosti $k$ platí vzťah
\eq{
\omega_0 = \sqrt{\frac{k}{m}}\,, \lbl{R_2}
}
kde $m$ je hmotnosť závažia.

Závislosť amplitúdy $B$ kmitov od frekvencie $\gamma$ budiacej sily môžeme vyjadriť ako
\eq{
B_{\(\gamma\)} = \frac{f}{m\sqrt{\(\omega_0^2-\gamma^2\)+4\lambda^2\gamma^2}}\,, \lbl{R_4}
}
kde $\omega_0$ je vlastná uhlová frekvencia oscilátoru a $\lambda$ je direkté tlmenie oscilátoru.

Označme $\omega_0$ vlastnú uhlovú frekvenciu oscilátoru, $\omega$ uhlovú frekvenciu tlmeného oscilátora a $\gamma$ útlm oscilátora. Potom medzi nimi platí nasledujúci vzťah
\eq{
\omega_0 =  \sqrt{\omega^2 + \gamma^2}\,. \lbl{V_PB_1}
}
Celá teória spolu s odvodením dostupná na \cite{C_1} a \cite{C_2}.


\subsubsection{Spracovanie chýb merania}

Označme $\mean{t}$ aritmetický priemer nameraných hodnôt $t_i$, a $\Delta t$ hodnotu $\mean{t}-t$, pričom 
\eq{
\mean{t} = \frac{1}{n}\sum_{i=1}^n t_i \,, \lbl{SCH_1}
}  
a chybu aritmetického priemeru 
\eq{
  \sigma_0=\sqrt{\frac{\sum_{i=1}^n \(t_i - \mean{t}\)^2}{n\(n-1\)}}\,, \lbl{SCH_2}
}
pričom $n$ je počet meraní.

\section{Postup merania}
\begin{enumerate}
\item Na digitálnych váhach boli odvážené jednotlivé časti protizávaží a samotné závažia kyvadla.
\item Pomocou senzoru Pasco boli odmerané jednotlivé rozdiely polôh pre rôzne hmotnosti závaží zavesených na oscilátri.
\item Oscilátor bol potiahnutím závaží nadol uvedený do pohybu a pomocou senzoru bol zaznamenávaný priebeh výchylky na čase
\item Elektromotor bol pripojený k zdroji napätia a pomocou tachometru odmeraná závislosť otáčiek na napätí. 
\item Motorček bol pripojený k oscilátoru, tak aby budil kmity. Pomocou senzoru Pasco bola zaznamenávaná veľkosť výchylky.
\item Pohlové kyvadlo bolo vychýlením z rovnovážnej polohy uvedené do pohybu, opäť bola zaznamenávaná veľkosť výchylky na čas
\item Pre prúdy v rozsahu $I= "0.5-1.5 A"$ boli opäť namerané závislosti podľa predchádzajúceho bodu.
\item Pre rôzne prúdy sa kyvadlo vychýlilo a hľadal sa prúd, v ktorom nastane kritický útlm.
\item Do kyvadla sa v rovnovážnej polohe buchlo tak aby to spôsobilo výchylku a opäť sa hľadal prúd kedy nastane kritický útlm.
\end{enumerate}

\section{Výsledky merania}
\subsection{Statická metóda}

V Tab. \ref{T_1} sú zaznamenané hmotnosti závažia a im prislúchajúcu hodnotu predĺženia. A vypočítaná tuhosť a vlastné frekvencie.
Na oscilátore okrem závaží sú zavesené aj iné predmety; menovite hliníková tyčka o hmotnosti $m_h="21.84 g"$ a kovový hák na zavesenie závaží o hmotnosti $m_t="21.16 g"$, pričom celková hmotnosť 
\eq{
m_0 = m_t + m_h = "43.0 g"\,. 
}



\begin{table}[h]

\begin{center}
\begin{tabular}{| c | c | c | c | c |}
\hline
 typ   &\popi{\Delta x}{mm} & \popi{m}{g} & \popi{k}{kg s^{-2}} & \popi{\omega_0}{rad\cdot s^{-1}} \\
\hline
Malé    & $3.9   \pm 0.01$ & $ 10.29 $ & $28.88\pm0.2$ & $23.34$\\
Stredné & $7.56  \pm 0.01$ & $ 22.30 $ & $28.93\pm0.2$ & $21.09$\\
Veľké   & $17.78 \pm 0.01$ & $ 45.56 $ & $25.14\pm0.2$ & $16.90$\\
\hline

\end{tabular}
\caption{Namerané hodnoty výchylky $x$, na hmotnosti závažia $m$, vypočítaná tuhosť pružiny $k$ podľa vzťahu \ref{R_1} a vlastná uhlová frekvencia $\omega_0$ podľa vzťahu \ref{R_2} } \label{T_1}
\end{center}
\end{table}



\subsection{Tlmené kmity}
Do grafov Obr. \ref{G_1} a Obr. \ref{G_2} boli vynesené namerané hodnoty tlmeného kmitania pre stredné závažie. 
Tie boli preložené fitmi, a z nich boli získané postupne hodnoty $\omega_0="\(18.88\pm0.007\) rad s^{-1}"$ a $\omega_0="\(18.24\pm0.02\) rad \cdot  s^{-1}"$.

Pre veľké závažia bol postup ekvivalentný, grafy boli vynesené do Obr. \ref{G_3} a Obr. \ref{G_4} a získané hodnoty $\omega_0="\(16.56\pm0.02\) rad s^{-1}"$ a $\omega_0="\(16.30\pm0.01\) rad \cdot  s^{-1}"$.


\begin{figure}
% GNUPLOT: LaTeX picture
\setlength{\unitlength}{0.240900pt}
\ifx\plotpoint\undefined\newsavebox{\plotpoint}\fi
\begin{picture}(1500,900)(0,0)
\sbox{\plotpoint}{\rule[-0.200pt]{0.400pt}{0.400pt}}%
\put(151.0,131.0){\rule[-0.200pt]{4.818pt}{0.400pt}}
\put(131,131){\makebox(0,0)[r]{-40}}
\put(1419.0,131.0){\rule[-0.200pt]{4.818pt}{0.400pt}}
\put(151.0,235.0){\rule[-0.200pt]{4.818pt}{0.400pt}}
\put(131,235){\makebox(0,0)[r]{-30}}
\put(1419.0,235.0){\rule[-0.200pt]{4.818pt}{0.400pt}}
\put(151.0,339.0){\rule[-0.200pt]{4.818pt}{0.400pt}}
\put(131,339){\makebox(0,0)[r]{-20}}
\put(1419.0,339.0){\rule[-0.200pt]{4.818pt}{0.400pt}}
\put(151.0,443.0){\rule[-0.200pt]{4.818pt}{0.400pt}}
\put(131,443){\makebox(0,0)[r]{-10}}
\put(1419.0,443.0){\rule[-0.200pt]{4.818pt}{0.400pt}}
\put(151.0,547.0){\rule[-0.200pt]{4.818pt}{0.400pt}}
\put(131,547){\makebox(0,0)[r]{ 0}}
\put(1419.0,547.0){\rule[-0.200pt]{4.818pt}{0.400pt}}
\put(151.0,651.0){\rule[-0.200pt]{4.818pt}{0.400pt}}
\put(131,651){\makebox(0,0)[r]{ 10}}
\put(1419.0,651.0){\rule[-0.200pt]{4.818pt}{0.400pt}}
\put(151.0,755.0){\rule[-0.200pt]{4.818pt}{0.400pt}}
\put(131,755){\makebox(0,0)[r]{ 20}}
\put(1419.0,755.0){\rule[-0.200pt]{4.818pt}{0.400pt}}
\put(151.0,859.0){\rule[-0.200pt]{4.818pt}{0.400pt}}
\put(131,859){\makebox(0,0)[r]{ 30}}
\put(1419.0,859.0){\rule[-0.200pt]{4.818pt}{0.400pt}}
\put(151.0,131.0){\rule[-0.200pt]{0.400pt}{4.818pt}}
\put(151,90){\makebox(0,0){ 10}}
\put(151.0,839.0){\rule[-0.200pt]{0.400pt}{4.818pt}}
\put(409.0,131.0){\rule[-0.200pt]{0.400pt}{4.818pt}}
\put(409,90){\makebox(0,0){ 10.5}}
\put(409.0,839.0){\rule[-0.200pt]{0.400pt}{4.818pt}}
\put(666.0,131.0){\rule[-0.200pt]{0.400pt}{4.818pt}}
\put(666,90){\makebox(0,0){ 11}}
\put(666.0,839.0){\rule[-0.200pt]{0.400pt}{4.818pt}}
\put(924.0,131.0){\rule[-0.200pt]{0.400pt}{4.818pt}}
\put(924,90){\makebox(0,0){ 11.5}}
\put(924.0,839.0){\rule[-0.200pt]{0.400pt}{4.818pt}}
\put(1181.0,131.0){\rule[-0.200pt]{0.400pt}{4.818pt}}
\put(1181,90){\makebox(0,0){ 12}}
\put(1181.0,839.0){\rule[-0.200pt]{0.400pt}{4.818pt}}
\put(1439.0,131.0){\rule[-0.200pt]{0.400pt}{4.818pt}}
\put(1439,90){\makebox(0,0){ 12.5}}
\put(1439.0,839.0){\rule[-0.200pt]{0.400pt}{4.818pt}}
\put(151.0,131.0){\rule[-0.200pt]{0.400pt}{175.375pt}}
\put(151.0,131.0){\rule[-0.200pt]{310.279pt}{0.400pt}}
\put(1439.0,131.0){\rule[-0.200pt]{0.400pt}{175.375pt}}
\put(151.0,859.0){\rule[-0.200pt]{310.279pt}{0.400pt}}
\put(30,495){\makebox(0,0){\popi{x}{mm}}}
\put(795,29){\makebox(0,0){\popi{t}{s}}}
\put(1279,819){\makebox(0,0)[r]{$x= f(t) $}}
\put(1299.0,819.0){\rule[-0.200pt]{24.090pt}{0.400pt}}
\put(151,167){\usebox{\plotpoint}}
\multiput(151.58,167.00)(0.492,0.582){21}{\rule{0.119pt}{0.567pt}}
\multiput(150.17,167.00)(12.000,12.824){2}{\rule{0.400pt}{0.283pt}}
\multiput(163.58,181.00)(0.492,3.124){21}{\rule{0.119pt}{2.533pt}}
\multiput(162.17,181.00)(12.000,67.742){2}{\rule{0.400pt}{1.267pt}}
\multiput(175.58,254.00)(0.492,4.977){21}{\rule{0.119pt}{3.967pt}}
\multiput(174.17,254.00)(12.000,107.767){2}{\rule{0.400pt}{1.983pt}}
\multiput(187.58,370.00)(0.492,5.752){21}{\rule{0.119pt}{4.567pt}}
\multiput(186.17,370.00)(12.000,124.522){2}{\rule{0.400pt}{2.283pt}}
\multiput(199.58,504.00)(0.492,5.451){21}{\rule{0.119pt}{4.333pt}}
\multiput(198.17,504.00)(12.000,118.006){2}{\rule{0.400pt}{2.167pt}}
\multiput(211.58,631.00)(0.492,4.115){21}{\rule{0.119pt}{3.300pt}}
\multiput(210.17,631.00)(12.000,89.151){2}{\rule{0.400pt}{1.650pt}}
\multiput(223.58,727.00)(0.492,2.133){21}{\rule{0.119pt}{1.767pt}}
\multiput(222.17,727.00)(12.000,46.333){2}{\rule{0.400pt}{0.883pt}}
\multiput(235.00,775.95)(2.472,-0.447){3}{\rule{1.700pt}{0.108pt}}
\multiput(235.00,776.17)(8.472,-3.000){2}{\rule{0.850pt}{0.400pt}}
\multiput(247.58,766.25)(0.492,-2.263){21}{\rule{0.119pt}{1.867pt}}
\multiput(246.17,770.13)(12.000,-49.126){2}{\rule{0.400pt}{0.933pt}}
\multiput(259.58,708.27)(0.492,-3.813){21}{\rule{0.119pt}{3.067pt}}
\multiput(258.17,714.64)(12.000,-82.635){2}{\rule{0.400pt}{1.533pt}}
\multiput(271.58,616.64)(0.492,-4.632){21}{\rule{0.119pt}{3.700pt}}
\multiput(270.17,624.32)(12.000,-100.320){2}{\rule{0.400pt}{1.850pt}}
\multiput(283.58,509.06)(0.492,-4.503){21}{\rule{0.119pt}{3.600pt}}
\multiput(282.17,516.53)(12.000,-97.528){2}{\rule{0.400pt}{1.800pt}}
\multiput(295.58,407.10)(0.492,-3.555){21}{\rule{0.119pt}{2.867pt}}
\multiput(294.17,413.05)(12.000,-77.050){2}{\rule{0.400pt}{1.433pt}}
\multiput(307.58,329.08)(0.492,-2.004){21}{\rule{0.119pt}{1.667pt}}
\multiput(306.17,332.54)(12.000,-43.541){2}{\rule{0.400pt}{0.833pt}}
\multiput(319.00,287.94)(1.651,-0.468){5}{\rule{1.300pt}{0.113pt}}
\multiput(319.00,288.17)(9.302,-4.000){2}{\rule{0.650pt}{0.400pt}}
\multiput(331.58,285.00)(0.492,1.530){21}{\rule{0.119pt}{1.300pt}}
\multiput(330.17,285.00)(12.000,33.302){2}{\rule{0.400pt}{0.650pt}}
\multiput(343.58,321.00)(0.492,2.952){21}{\rule{0.119pt}{2.400pt}}
\multiput(342.17,321.00)(12.000,64.019){2}{\rule{0.400pt}{1.200pt}}
\multiput(355.58,390.00)(0.492,3.684){21}{\rule{0.119pt}{2.967pt}}
\multiput(354.17,390.00)(12.000,79.843){2}{\rule{0.400pt}{1.483pt}}
\multiput(367.58,476.00)(0.492,3.727){21}{\rule{0.119pt}{3.000pt}}
\multiput(366.17,476.00)(12.000,80.773){2}{\rule{0.400pt}{1.500pt}}
\multiput(379.58,563.00)(0.492,3.038){21}{\rule{0.119pt}{2.467pt}}
\multiput(378.17,563.00)(12.000,65.880){2}{\rule{0.400pt}{1.233pt}}
\multiput(391.58,634.00)(0.492,1.832){21}{\rule{0.119pt}{1.533pt}}
\multiput(390.17,634.00)(12.000,39.817){2}{\rule{0.400pt}{0.767pt}}
\multiput(403.00,677.59)(0.669,0.489){15}{\rule{0.633pt}{0.118pt}}
\multiput(403.00,676.17)(10.685,9.000){2}{\rule{0.317pt}{0.400pt}}
\multiput(415.58,682.13)(0.492,-1.056){21}{\rule{0.119pt}{0.933pt}}
\multiput(414.17,684.06)(12.000,-23.063){2}{\rule{0.400pt}{0.467pt}}
\multiput(427.58,653.39)(0.492,-2.219){21}{\rule{0.119pt}{1.833pt}}
\multiput(426.17,657.19)(12.000,-48.195){2}{\rule{0.400pt}{0.917pt}}
\multiput(439.58,598.17)(0.492,-3.232){19}{\rule{0.118pt}{2.609pt}}
\multiput(438.17,603.58)(11.000,-63.585){2}{\rule{0.400pt}{1.305pt}}
\multiput(450.58,529.76)(0.492,-3.038){21}{\rule{0.119pt}{2.467pt}}
\multiput(449.17,534.88)(12.000,-65.880){2}{\rule{0.400pt}{1.233pt}}
\multiput(462.58,460.28)(0.492,-2.564){21}{\rule{0.119pt}{2.100pt}}
\multiput(461.17,464.64)(12.000,-55.641){2}{\rule{0.400pt}{1.050pt}}
\multiput(474.58,403.19)(0.492,-1.659){21}{\rule{0.119pt}{1.400pt}}
\multiput(473.17,406.09)(12.000,-36.094){2}{\rule{0.400pt}{0.700pt}}
\multiput(486.00,368.92)(0.496,-0.492){21}{\rule{0.500pt}{0.119pt}}
\multiput(486.00,369.17)(10.962,-12.000){2}{\rule{0.250pt}{0.400pt}}
\multiput(498.58,358.00)(0.492,0.669){21}{\rule{0.119pt}{0.633pt}}
\multiput(497.17,358.00)(12.000,14.685){2}{\rule{0.400pt}{0.317pt}}
\multiput(510.58,374.00)(0.492,1.703){21}{\rule{0.119pt}{1.433pt}}
\multiput(509.17,374.00)(12.000,37.025){2}{\rule{0.400pt}{0.717pt}}
\multiput(522.58,414.00)(0.492,2.306){21}{\rule{0.119pt}{1.900pt}}
\multiput(521.17,414.00)(12.000,50.056){2}{\rule{0.400pt}{0.950pt}}
\multiput(534.58,468.00)(0.492,2.478){21}{\rule{0.119pt}{2.033pt}}
\multiput(533.17,468.00)(12.000,53.780){2}{\rule{0.400pt}{1.017pt}}
\multiput(546.58,526.00)(0.492,2.176){21}{\rule{0.119pt}{1.800pt}}
\multiput(545.17,526.00)(12.000,47.264){2}{\rule{0.400pt}{0.900pt}}
\multiput(558.58,577.00)(0.492,1.487){21}{\rule{0.119pt}{1.267pt}}
\multiput(557.17,577.00)(12.000,32.371){2}{\rule{0.400pt}{0.633pt}}
\multiput(570.58,612.00)(0.492,0.539){21}{\rule{0.119pt}{0.533pt}}
\multiput(569.17,612.00)(12.000,11.893){2}{\rule{0.400pt}{0.267pt}}
\multiput(582.00,623.92)(0.600,-0.491){17}{\rule{0.580pt}{0.118pt}}
\multiput(582.00,624.17)(10.796,-10.000){2}{\rule{0.290pt}{0.400pt}}
\multiput(594.58,610.57)(0.492,-1.229){21}{\rule{0.119pt}{1.067pt}}
\multiput(593.17,612.79)(12.000,-26.786){2}{\rule{0.400pt}{0.533pt}}
\multiput(606.58,579.64)(0.492,-1.832){21}{\rule{0.119pt}{1.533pt}}
\multiput(605.17,582.82)(12.000,-39.817){2}{\rule{0.400pt}{0.767pt}}
\multiput(618.58,536.08)(0.492,-2.004){21}{\rule{0.119pt}{1.667pt}}
\multiput(617.17,539.54)(12.000,-43.541){2}{\rule{0.400pt}{0.833pt}}
\multiput(630.58,489.63)(0.492,-1.832){21}{\rule{0.119pt}{1.533pt}}
\multiput(629.17,492.82)(12.000,-39.817){2}{\rule{0.400pt}{0.767pt}}
\multiput(642.58,448.43)(0.492,-1.272){21}{\rule{0.119pt}{1.100pt}}
\multiput(641.17,450.72)(12.000,-27.717){2}{\rule{0.400pt}{0.550pt}}
\multiput(654.58,420.65)(0.492,-0.582){21}{\rule{0.119pt}{0.567pt}}
\multiput(653.17,421.82)(12.000,-12.824){2}{\rule{0.400pt}{0.283pt}}
\multiput(666.00,409.59)(1.267,0.477){7}{\rule{1.060pt}{0.115pt}}
\multiput(666.00,408.17)(9.800,5.000){2}{\rule{0.530pt}{0.400pt}}
\multiput(678.58,414.00)(0.492,0.927){21}{\rule{0.119pt}{0.833pt}}
\multiput(677.17,414.00)(12.000,20.270){2}{\rule{0.400pt}{0.417pt}}
\multiput(690.58,436.00)(0.492,1.401){21}{\rule{0.119pt}{1.200pt}}
\multiput(689.17,436.00)(12.000,30.509){2}{\rule{0.400pt}{0.600pt}}
\multiput(702.58,469.00)(0.492,1.616){21}{\rule{0.119pt}{1.367pt}}
\multiput(701.17,469.00)(12.000,35.163){2}{\rule{0.400pt}{0.683pt}}
\multiput(714.58,507.00)(0.492,1.530){21}{\rule{0.119pt}{1.300pt}}
\multiput(713.17,507.00)(12.000,33.302){2}{\rule{0.400pt}{0.650pt}}
\multiput(726.58,543.00)(0.492,1.099){21}{\rule{0.119pt}{0.967pt}}
\multiput(725.17,543.00)(12.000,23.994){2}{\rule{0.400pt}{0.483pt}}
\multiput(738.58,569.00)(0.492,0.582){21}{\rule{0.119pt}{0.567pt}}
\multiput(737.17,569.00)(12.000,12.824){2}{\rule{0.400pt}{0.283pt}}
\put(750,581.17){\rule{2.500pt}{0.400pt}}
\multiput(750.00,582.17)(6.811,-2.000){2}{\rule{1.250pt}{0.400pt}}
\multiput(762.58,578.37)(0.492,-0.669){21}{\rule{0.119pt}{0.633pt}}
\multiput(761.17,579.69)(12.000,-14.685){2}{\rule{0.400pt}{0.317pt}}
\multiput(774.58,561.13)(0.492,-1.056){21}{\rule{0.119pt}{0.933pt}}
\multiput(773.17,563.06)(12.000,-23.063){2}{\rule{0.400pt}{0.467pt}}
\multiput(786.58,535.30)(0.492,-1.315){21}{\rule{0.119pt}{1.133pt}}
\multiput(785.17,537.65)(12.000,-28.648){2}{\rule{0.400pt}{0.567pt}}
\multiput(798.58,504.43)(0.492,-1.272){21}{\rule{0.119pt}{1.100pt}}
\multiput(797.17,506.72)(12.000,-27.717){2}{\rule{0.400pt}{0.550pt}}
\multiput(810.58,475.40)(0.492,-0.970){21}{\rule{0.119pt}{0.867pt}}
\multiput(809.17,477.20)(12.000,-21.201){2}{\rule{0.400pt}{0.433pt}}
\multiput(822.00,454.92)(0.496,-0.492){21}{\rule{0.500pt}{0.119pt}}
\multiput(822.00,455.17)(10.962,-12.000){2}{\rule{0.250pt}{0.400pt}}
\put(834,442.67){\rule{2.891pt}{0.400pt}}
\multiput(834.00,443.17)(6.000,-1.000){2}{\rule{1.445pt}{0.400pt}}
\multiput(846.00,443.58)(0.543,0.492){19}{\rule{0.536pt}{0.118pt}}
\multiput(846.00,442.17)(10.887,11.000){2}{\rule{0.268pt}{0.400pt}}
\multiput(858.58,454.00)(0.492,0.841){21}{\rule{0.119pt}{0.767pt}}
\multiput(857.17,454.00)(12.000,18.409){2}{\rule{0.400pt}{0.383pt}}
\multiput(870.58,474.00)(0.492,1.056){21}{\rule{0.119pt}{0.933pt}}
\multiput(869.17,474.00)(12.000,23.063){2}{\rule{0.400pt}{0.467pt}}
\multiput(882.58,499.00)(0.492,1.013){21}{\rule{0.119pt}{0.900pt}}
\multiput(881.17,499.00)(12.000,22.132){2}{\rule{0.400pt}{0.450pt}}
\multiput(894.58,523.00)(0.492,0.841){21}{\rule{0.119pt}{0.767pt}}
\multiput(893.17,523.00)(12.000,18.409){2}{\rule{0.400pt}{0.383pt}}
\multiput(906.00,543.58)(0.496,0.492){21}{\rule{0.500pt}{0.119pt}}
\multiput(906.00,542.17)(10.962,12.000){2}{\rule{0.250pt}{0.400pt}}
\put(918,555.17){\rule{2.500pt}{0.400pt}}
\multiput(918.00,554.17)(6.811,2.000){2}{\rule{1.250pt}{0.400pt}}
\multiput(930.00,555.93)(0.758,-0.488){13}{\rule{0.700pt}{0.117pt}}
\multiput(930.00,556.17)(10.547,-8.000){2}{\rule{0.350pt}{0.400pt}}
\multiput(942.58,546.51)(0.492,-0.625){21}{\rule{0.119pt}{0.600pt}}
\multiput(941.17,547.75)(12.000,-13.755){2}{\rule{0.400pt}{0.300pt}}
\multiput(954.58,530.82)(0.492,-0.841){21}{\rule{0.119pt}{0.767pt}}
\multiput(953.17,532.41)(12.000,-18.409){2}{\rule{0.400pt}{0.383pt}}
\multiput(966.58,510.82)(0.492,-0.841){21}{\rule{0.119pt}{0.767pt}}
\multiput(965.17,512.41)(12.000,-18.409){2}{\rule{0.400pt}{0.383pt}}
\multiput(978.58,491.23)(0.492,-0.712){21}{\rule{0.119pt}{0.667pt}}
\multiput(977.17,492.62)(12.000,-15.616){2}{\rule{0.400pt}{0.333pt}}
\multiput(990.00,475.92)(0.600,-0.491){17}{\rule{0.580pt}{0.118pt}}
\multiput(990.00,476.17)(10.796,-10.000){2}{\rule{0.290pt}{0.400pt}}
\multiput(1002.00,465.95)(2.472,-0.447){3}{\rule{1.700pt}{0.108pt}}
\multiput(1002.00,466.17)(8.472,-3.000){2}{\rule{0.850pt}{0.400pt}}
\multiput(1014.00,464.59)(1.267,0.477){7}{\rule{1.060pt}{0.115pt}}
\multiput(1014.00,463.17)(9.800,5.000){2}{\rule{0.530pt}{0.400pt}}
\multiput(1026.00,469.58)(0.496,0.492){19}{\rule{0.500pt}{0.118pt}}
\multiput(1026.00,468.17)(9.962,11.000){2}{\rule{0.250pt}{0.400pt}}
\multiput(1037.58,480.00)(0.492,0.669){21}{\rule{0.119pt}{0.633pt}}
\multiput(1036.17,480.00)(12.000,14.685){2}{\rule{0.400pt}{0.317pt}}
\multiput(1049.58,496.00)(0.492,0.669){21}{\rule{0.119pt}{0.633pt}}
\multiput(1048.17,496.00)(12.000,14.685){2}{\rule{0.400pt}{0.317pt}}
\multiput(1061.58,512.00)(0.492,0.582){21}{\rule{0.119pt}{0.567pt}}
\multiput(1060.17,512.00)(12.000,12.824){2}{\rule{0.400pt}{0.283pt}}
\multiput(1073.00,526.58)(0.600,0.491){17}{\rule{0.580pt}{0.118pt}}
\multiput(1073.00,525.17)(10.796,10.000){2}{\rule{0.290pt}{0.400pt}}
\multiput(1085.00,536.61)(2.472,0.447){3}{\rule{1.700pt}{0.108pt}}
\multiput(1085.00,535.17)(8.472,3.000){2}{\rule{0.850pt}{0.400pt}}
\multiput(1097.00,537.95)(2.472,-0.447){3}{\rule{1.700pt}{0.108pt}}
\multiput(1097.00,538.17)(8.472,-3.000){2}{\rule{0.850pt}{0.400pt}}
\multiput(1109.00,534.93)(0.758,-0.488){13}{\rule{0.700pt}{0.117pt}}
\multiput(1109.00,535.17)(10.547,-8.000){2}{\rule{0.350pt}{0.400pt}}
\multiput(1121.58,525.79)(0.492,-0.539){21}{\rule{0.119pt}{0.533pt}}
\multiput(1120.17,526.89)(12.000,-11.893){2}{\rule{0.400pt}{0.267pt}}
\multiput(1133.58,512.79)(0.492,-0.539){21}{\rule{0.119pt}{0.533pt}}
\multiput(1132.17,513.89)(12.000,-11.893){2}{\rule{0.400pt}{0.267pt}}
\multiput(1145.00,500.92)(0.496,-0.492){21}{\rule{0.500pt}{0.119pt}}
\multiput(1145.00,501.17)(10.962,-12.000){2}{\rule{0.250pt}{0.400pt}}
\multiput(1157.00,488.93)(0.758,-0.488){13}{\rule{0.700pt}{0.117pt}}
\multiput(1157.00,489.17)(10.547,-8.000){2}{\rule{0.350pt}{0.400pt}}
\multiput(1169.00,480.94)(1.651,-0.468){5}{\rule{1.300pt}{0.113pt}}
\multiput(1169.00,481.17)(9.302,-4.000){2}{\rule{0.650pt}{0.400pt}}
\put(1181,478.17){\rule{2.500pt}{0.400pt}}
\multiput(1181.00,477.17)(6.811,2.000){2}{\rule{1.250pt}{0.400pt}}
\multiput(1193.00,480.59)(1.033,0.482){9}{\rule{0.900pt}{0.116pt}}
\multiput(1193.00,479.17)(10.132,6.000){2}{\rule{0.450pt}{0.400pt}}
\multiput(1205.00,486.58)(0.600,0.491){17}{\rule{0.580pt}{0.118pt}}
\multiput(1205.00,485.17)(10.796,10.000){2}{\rule{0.290pt}{0.400pt}}
\multiput(1217.00,496.58)(0.600,0.491){17}{\rule{0.580pt}{0.118pt}}
\multiput(1217.00,495.17)(10.796,10.000){2}{\rule{0.290pt}{0.400pt}}
\multiput(1229.00,506.58)(0.600,0.491){17}{\rule{0.580pt}{0.118pt}}
\multiput(1229.00,505.17)(10.796,10.000){2}{\rule{0.290pt}{0.400pt}}
\multiput(1241.00,516.59)(0.758,0.488){13}{\rule{0.700pt}{0.117pt}}
\multiput(1241.00,515.17)(10.547,8.000){2}{\rule{0.350pt}{0.400pt}}
\multiput(1253.00,524.61)(2.472,0.447){3}{\rule{1.700pt}{0.108pt}}
\multiput(1253.00,523.17)(8.472,3.000){2}{\rule{0.850pt}{0.400pt}}
\multiput(1277.00,525.93)(1.267,-0.477){7}{\rule{1.060pt}{0.115pt}}
\multiput(1277.00,526.17)(9.800,-5.000){2}{\rule{0.530pt}{0.400pt}}
\multiput(1289.00,520.93)(0.874,-0.485){11}{\rule{0.786pt}{0.117pt}}
\multiput(1289.00,521.17)(10.369,-7.000){2}{\rule{0.393pt}{0.400pt}}
\multiput(1301.00,513.93)(0.669,-0.489){15}{\rule{0.633pt}{0.118pt}}
\multiput(1301.00,514.17)(10.685,-9.000){2}{\rule{0.317pt}{0.400pt}}
\multiput(1313.00,504.93)(0.758,-0.488){13}{\rule{0.700pt}{0.117pt}}
\multiput(1313.00,505.17)(10.547,-8.000){2}{\rule{0.350pt}{0.400pt}}
\multiput(1325.00,496.93)(0.874,-0.485){11}{\rule{0.786pt}{0.117pt}}
\multiput(1325.00,497.17)(10.369,-7.000){2}{\rule{0.393pt}{0.400pt}}
\put(1265.0,527.0){\rule[-0.200pt]{2.891pt}{0.400pt}}
\put(1279,778){\makebox(0,0)[r]{namerané dáta}}
\put(151,175){\makebox(0,0){$\times$}}
\put(152,177){\makebox(0,0){$\times$}}
\put(152,178){\makebox(0,0){$\times$}}
\put(153,179){\makebox(0,0){$\times$}}
\put(153,181){\makebox(0,0){$\times$}}
\put(154,182){\makebox(0,0){$\times$}}
\put(154,183){\makebox(0,0){$\times$}}
\put(155,185){\makebox(0,0){$\times$}}
\put(155,187){\makebox(0,0){$\times$}}
\put(156,188){\makebox(0,0){$\times$}}
\put(156,190){\makebox(0,0){$\times$}}
\put(157,192){\makebox(0,0){$\times$}}
\put(157,194){\makebox(0,0){$\times$}}
\put(158,196){\makebox(0,0){$\times$}}
\put(158,197){\makebox(0,0){$\times$}}
\put(159,199){\makebox(0,0){$\times$}}
\put(159,201){\makebox(0,0){$\times$}}
\put(160,203){\makebox(0,0){$\times$}}
\put(160,205){\makebox(0,0){$\times$}}
\put(161,208){\makebox(0,0){$\times$}}
\put(161,210){\makebox(0,0){$\times$}}
\put(162,212){\makebox(0,0){$\times$}}
\put(162,214){\makebox(0,0){$\times$}}
\put(163,217){\makebox(0,0){$\times$}}
\put(163,219){\makebox(0,0){$\times$}}
\put(164,222){\makebox(0,0){$\times$}}
\put(164,224){\makebox(0,0){$\times$}}
\put(165,227){\makebox(0,0){$\times$}}
\put(165,229){\makebox(0,0){$\times$}}
\put(166,232){\makebox(0,0){$\times$}}
\put(166,235){\makebox(0,0){$\times$}}
\put(167,238){\makebox(0,0){$\times$}}
\put(167,241){\makebox(0,0){$\times$}}
\put(168,244){\makebox(0,0){$\times$}}
\put(169,247){\makebox(0,0){$\times$}}
\put(169,250){\makebox(0,0){$\times$}}
\put(170,253){\makebox(0,0){$\times$}}
\put(170,257){\makebox(0,0){$\times$}}
\put(171,260){\makebox(0,0){$\times$}}
\put(171,264){\makebox(0,0){$\times$}}
\put(172,267){\makebox(0,0){$\times$}}
\put(172,271){\makebox(0,0){$\times$}}
\put(173,274){\makebox(0,0){$\times$}}
\put(173,278){\makebox(0,0){$\times$}}
\put(174,282){\makebox(0,0){$\times$}}
\put(174,286){\makebox(0,0){$\times$}}
\put(175,290){\makebox(0,0){$\times$}}
\put(175,294){\makebox(0,0){$\times$}}
\put(176,298){\makebox(0,0){$\times$}}
\put(176,302){\makebox(0,0){$\times$}}
\put(177,306){\makebox(0,0){$\times$}}
\put(177,311){\makebox(0,0){$\times$}}
\put(178,315){\makebox(0,0){$\times$}}
\put(178,319){\makebox(0,0){$\times$}}
\put(179,324){\makebox(0,0){$\times$}}
\put(179,328){\makebox(0,0){$\times$}}
\put(180,333){\makebox(0,0){$\times$}}
\put(180,337){\makebox(0,0){$\times$}}
\put(181,342){\makebox(0,0){$\times$}}
\put(181,347){\makebox(0,0){$\times$}}
\put(182,352){\makebox(0,0){$\times$}}
\put(182,356){\makebox(0,0){$\times$}}
\put(183,362){\makebox(0,0){$\times$}}
\put(183,366){\makebox(0,0){$\times$}}
\put(184,371){\makebox(0,0){$\times$}}
\put(184,376){\makebox(0,0){$\times$}}
\put(185,381){\makebox(0,0){$\times$}}
\put(186,386){\makebox(0,0){$\times$}}
\put(186,391){\makebox(0,0){$\times$}}
\put(187,397){\makebox(0,0){$\times$}}
\put(187,402){\makebox(0,0){$\times$}}
\put(188,407){\makebox(0,0){$\times$}}
\put(188,412){\makebox(0,0){$\times$}}
\put(189,418){\makebox(0,0){$\times$}}
\put(189,423){\makebox(0,0){$\times$}}
\put(190,428){\makebox(0,0){$\times$}}
\put(190,433){\makebox(0,0){$\times$}}
\put(191,439){\makebox(0,0){$\times$}}
\put(191,444){\makebox(0,0){$\times$}}
\put(192,449){\makebox(0,0){$\times$}}
\put(192,454){\makebox(0,0){$\times$}}
\put(193,460){\makebox(0,0){$\times$}}
\put(193,465){\makebox(0,0){$\times$}}
\put(194,470){\makebox(0,0){$\times$}}
\put(194,475){\makebox(0,0){$\times$}}
\put(195,481){\makebox(0,0){$\times$}}
\put(195,486){\makebox(0,0){$\times$}}
\put(196,492){\makebox(0,0){$\times$}}
\put(196,497){\makebox(0,0){$\times$}}
\put(197,503){\makebox(0,0){$\times$}}
\put(197,508){\makebox(0,0){$\times$}}
\put(198,513){\makebox(0,0){$\times$}}
\put(198,519){\makebox(0,0){$\times$}}
\put(199,524){\makebox(0,0){$\times$}}
\put(199,530){\makebox(0,0){$\times$}}
\put(200,535){\makebox(0,0){$\times$}}
\put(200,540){\makebox(0,0){$\times$}}
\put(201,545){\makebox(0,0){$\times$}}
\put(201,550){\makebox(0,0){$\times$}}
\put(202,556){\makebox(0,0){$\times$}}
\put(203,561){\makebox(0,0){$\times$}}
\put(203,566){\makebox(0,0){$\times$}}
\put(204,571){\makebox(0,0){$\times$}}
\put(204,577){\makebox(0,0){$\times$}}
\put(205,582){\makebox(0,0){$\times$}}
\put(205,587){\makebox(0,0){$\times$}}
\put(206,592){\makebox(0,0){$\times$}}
\put(206,596){\makebox(0,0){$\times$}}
\put(207,601){\makebox(0,0){$\times$}}
\put(207,606){\makebox(0,0){$\times$}}
\put(208,611){\makebox(0,0){$\times$}}
\put(208,616){\makebox(0,0){$\times$}}
\put(209,621){\makebox(0,0){$\times$}}
\put(209,626){\makebox(0,0){$\times$}}
\put(210,630){\makebox(0,0){$\times$}}
\put(210,635){\makebox(0,0){$\times$}}
\put(211,640){\makebox(0,0){$\times$}}
\put(211,644){\makebox(0,0){$\times$}}
\put(212,648){\makebox(0,0){$\times$}}
\put(212,653){\makebox(0,0){$\times$}}
\put(213,657){\makebox(0,0){$\times$}}
\put(213,661){\makebox(0,0){$\times$}}
\put(214,666){\makebox(0,0){$\times$}}
\put(214,670){\makebox(0,0){$\times$}}
\put(215,674){\makebox(0,0){$\times$}}
\put(215,678){\makebox(0,0){$\times$}}
\put(216,682){\makebox(0,0){$\times$}}
\put(216,686){\makebox(0,0){$\times$}}
\put(217,690){\makebox(0,0){$\times$}}
\put(217,694){\makebox(0,0){$\times$}}
\put(218,697){\makebox(0,0){$\times$}}
\put(218,701){\makebox(0,0){$\times$}}
\put(219,704){\makebox(0,0){$\times$}}
\put(220,708){\makebox(0,0){$\times$}}
\put(220,711){\makebox(0,0){$\times$}}
\put(221,715){\makebox(0,0){$\times$}}
\put(221,718){\makebox(0,0){$\times$}}
\put(222,721){\makebox(0,0){$\times$}}
\put(222,724){\makebox(0,0){$\times$}}
\put(223,727){\makebox(0,0){$\times$}}
\put(223,730){\makebox(0,0){$\times$}}
\put(224,733){\makebox(0,0){$\times$}}
\put(224,736){\makebox(0,0){$\times$}}
\put(225,738){\makebox(0,0){$\times$}}
\put(225,741){\makebox(0,0){$\times$}}
\put(226,743){\makebox(0,0){$\times$}}
\put(226,746){\makebox(0,0){$\times$}}
\put(227,748){\makebox(0,0){$\times$}}
\put(227,750){\makebox(0,0){$\times$}}
\put(228,753){\makebox(0,0){$\times$}}
\put(228,755){\makebox(0,0){$\times$}}
\put(229,757){\makebox(0,0){$\times$}}
\put(229,759){\makebox(0,0){$\times$}}
\put(230,760){\makebox(0,0){$\times$}}
\put(230,762){\makebox(0,0){$\times$}}
\put(231,764){\makebox(0,0){$\times$}}
\put(231,766){\makebox(0,0){$\times$}}
\put(232,767){\makebox(0,0){$\times$}}
\put(232,768){\makebox(0,0){$\times$}}
\put(233,770){\makebox(0,0){$\times$}}
\put(233,771){\makebox(0,0){$\times$}}
\put(234,772){\makebox(0,0){$\times$}}
\put(234,773){\makebox(0,0){$\times$}}
\put(235,774){\makebox(0,0){$\times$}}
\put(235,775){\makebox(0,0){$\times$}}
\put(236,776){\makebox(0,0){$\times$}}
\put(237,777){\makebox(0,0){$\times$}}
\put(237,778){\makebox(0,0){$\times$}}
\put(238,778){\makebox(0,0){$\times$}}
\put(238,778){\makebox(0,0){$\times$}}
\put(239,779){\makebox(0,0){$\times$}}
\put(239,779){\makebox(0,0){$\times$}}
\put(240,779){\makebox(0,0){$\times$}}
\put(240,780){\makebox(0,0){$\times$}}
\put(241,780){\makebox(0,0){$\times$}}
\put(241,780){\makebox(0,0){$\times$}}
\put(242,780){\makebox(0,0){$\times$}}
\put(242,780){\makebox(0,0){$\times$}}
\put(243,779){\makebox(0,0){$\times$}}
\put(243,779){\makebox(0,0){$\times$}}
\put(244,778){\makebox(0,0){$\times$}}
\put(244,778){\makebox(0,0){$\times$}}
\put(245,777){\makebox(0,0){$\times$}}
\put(245,776){\makebox(0,0){$\times$}}
\put(246,776){\makebox(0,0){$\times$}}
\put(246,775){\makebox(0,0){$\times$}}
\put(247,774){\makebox(0,0){$\times$}}
\put(247,773){\makebox(0,0){$\times$}}
\put(248,772){\makebox(0,0){$\times$}}
\put(248,771){\makebox(0,0){$\times$}}
\put(249,769){\makebox(0,0){$\times$}}
\put(249,768){\makebox(0,0){$\times$}}
\put(250,767){\makebox(0,0){$\times$}}
\put(250,765){\makebox(0,0){$\times$}}
\put(251,764){\makebox(0,0){$\times$}}
\put(251,762){\makebox(0,0){$\times$}}
\put(252,760){\makebox(0,0){$\times$}}
\put(252,759){\makebox(0,0){$\times$}}
\put(253,757){\makebox(0,0){$\times$}}
\put(254,755){\makebox(0,0){$\times$}}
\put(254,753){\makebox(0,0){$\times$}}
\put(255,750){\makebox(0,0){$\times$}}
\put(255,748){\makebox(0,0){$\times$}}
\put(256,746){\makebox(0,0){$\times$}}
\put(256,743){\makebox(0,0){$\times$}}
\put(257,741){\makebox(0,0){$\times$}}
\put(257,738){\makebox(0,0){$\times$}}
\put(258,736){\makebox(0,0){$\times$}}
\put(258,733){\makebox(0,0){$\times$}}
\put(259,731){\makebox(0,0){$\times$}}
\put(259,728){\makebox(0,0){$\times$}}
\put(260,725){\makebox(0,0){$\times$}}
\put(260,722){\makebox(0,0){$\times$}}
\put(261,719){\makebox(0,0){$\times$}}
\put(261,716){\makebox(0,0){$\times$}}
\put(262,713){\makebox(0,0){$\times$}}
\put(262,710){\makebox(0,0){$\times$}}
\put(263,706){\makebox(0,0){$\times$}}
\put(263,703){\makebox(0,0){$\times$}}
\put(264,699){\makebox(0,0){$\times$}}
\put(264,696){\makebox(0,0){$\times$}}
\put(265,692){\makebox(0,0){$\times$}}
\put(265,689){\makebox(0,0){$\times$}}
\put(266,685){\makebox(0,0){$\times$}}
\put(266,682){\makebox(0,0){$\times$}}
\put(267,678){\makebox(0,0){$\times$}}
\put(267,674){\makebox(0,0){$\times$}}
\put(268,670){\makebox(0,0){$\times$}}
\put(268,667){\makebox(0,0){$\times$}}
\put(269,662){\makebox(0,0){$\times$}}
\put(269,659){\makebox(0,0){$\times$}}
\put(270,655){\makebox(0,0){$\times$}}
\put(271,650){\makebox(0,0){$\times$}}
\put(271,647){\makebox(0,0){$\times$}}
\put(272,642){\makebox(0,0){$\times$}}
\put(272,638){\makebox(0,0){$\times$}}
\put(273,634){\makebox(0,0){$\times$}}
\put(273,629){\makebox(0,0){$\times$}}
\put(274,625){\makebox(0,0){$\times$}}
\put(274,621){\makebox(0,0){$\times$}}
\put(275,617){\makebox(0,0){$\times$}}
\put(275,612){\makebox(0,0){$\times$}}
\put(276,608){\makebox(0,0){$\times$}}
\put(276,603){\makebox(0,0){$\times$}}
\put(277,599){\makebox(0,0){$\times$}}
\put(277,594){\makebox(0,0){$\times$}}
\put(278,590){\makebox(0,0){$\times$}}
\put(278,585){\makebox(0,0){$\times$}}
\put(279,581){\makebox(0,0){$\times$}}
\put(279,576){\makebox(0,0){$\times$}}
\put(280,571){\makebox(0,0){$\times$}}
\put(280,567){\makebox(0,0){$\times$}}
\put(281,562){\makebox(0,0){$\times$}}
\put(281,557){\makebox(0,0){$\times$}}
\put(282,553){\makebox(0,0){$\times$}}
\put(282,548){\makebox(0,0){$\times$}}
\put(283,544){\makebox(0,0){$\times$}}
\put(283,539){\makebox(0,0){$\times$}}
\put(284,534){\makebox(0,0){$\times$}}
\put(284,530){\makebox(0,0){$\times$}}
\put(285,525){\makebox(0,0){$\times$}}
\put(285,520){\makebox(0,0){$\times$}}
\put(286,516){\makebox(0,0){$\times$}}
\put(286,511){\makebox(0,0){$\times$}}
\put(287,506){\makebox(0,0){$\times$}}
\put(288,502){\makebox(0,0){$\times$}}
\put(288,497){\makebox(0,0){$\times$}}
\put(289,492){\makebox(0,0){$\times$}}
\put(289,488){\makebox(0,0){$\times$}}
\put(290,483){\makebox(0,0){$\times$}}
\put(290,478){\makebox(0,0){$\times$}}
\put(291,474){\makebox(0,0){$\times$}}
\put(291,469){\makebox(0,0){$\times$}}
\put(292,465){\makebox(0,0){$\times$}}
\put(292,460){\makebox(0,0){$\times$}}
\put(293,456){\makebox(0,0){$\times$}}
\put(293,451){\makebox(0,0){$\times$}}
\put(294,447){\makebox(0,0){$\times$}}
\put(294,442){\makebox(0,0){$\times$}}
\put(295,438){\makebox(0,0){$\times$}}
\put(295,433){\makebox(0,0){$\times$}}
\put(296,429){\makebox(0,0){$\times$}}
\put(296,425){\makebox(0,0){$\times$}}
\put(297,421){\makebox(0,0){$\times$}}
\put(297,416){\makebox(0,0){$\times$}}
\put(298,412){\makebox(0,0){$\times$}}
\put(298,408){\makebox(0,0){$\times$}}
\put(299,404){\makebox(0,0){$\times$}}
\put(299,400){\makebox(0,0){$\times$}}
\put(300,396){\makebox(0,0){$\times$}}
\put(300,392){\makebox(0,0){$\times$}}
\put(301,388){\makebox(0,0){$\times$}}
\put(301,384){\makebox(0,0){$\times$}}
\put(302,381){\makebox(0,0){$\times$}}
\put(302,377){\makebox(0,0){$\times$}}
\put(303,373){\makebox(0,0){$\times$}}
\put(303,369){\makebox(0,0){$\times$}}
\put(304,366){\makebox(0,0){$\times$}}
\put(305,362){\makebox(0,0){$\times$}}
\put(305,359){\makebox(0,0){$\times$}}
\put(306,355){\makebox(0,0){$\times$}}
\put(306,352){\makebox(0,0){$\times$}}
\put(307,349){\makebox(0,0){$\times$}}
\put(307,346){\makebox(0,0){$\times$}}
\put(308,342){\makebox(0,0){$\times$}}
\put(308,339){\makebox(0,0){$\times$}}
\put(309,336){\makebox(0,0){$\times$}}
\put(309,333){\makebox(0,0){$\times$}}
\put(310,330){\makebox(0,0){$\times$}}
\put(310,327){\makebox(0,0){$\times$}}
\put(311,325){\makebox(0,0){$\times$}}
\put(311,322){\makebox(0,0){$\times$}}
\put(312,319){\makebox(0,0){$\times$}}
\put(312,317){\makebox(0,0){$\times$}}
\put(313,314){\makebox(0,0){$\times$}}
\put(313,312){\makebox(0,0){$\times$}}
\put(314,309){\makebox(0,0){$\times$}}
\put(314,307){\makebox(0,0){$\times$}}
\put(315,305){\makebox(0,0){$\times$}}
\put(315,303){\makebox(0,0){$\times$}}
\put(316,301){\makebox(0,0){$\times$}}
\put(316,299){\makebox(0,0){$\times$}}
\put(317,298){\makebox(0,0){$\times$}}
\put(317,296){\makebox(0,0){$\times$}}
\put(318,294){\makebox(0,0){$\times$}}
\put(318,293){\makebox(0,0){$\times$}}
\put(319,291){\makebox(0,0){$\times$}}
\put(319,290){\makebox(0,0){$\times$}}
\put(320,288){\makebox(0,0){$\times$}}
\put(321,287){\makebox(0,0){$\times$}}
\put(321,286){\makebox(0,0){$\times$}}
\put(322,285){\makebox(0,0){$\times$}}
\put(322,284){\makebox(0,0){$\times$}}
\put(323,283){\makebox(0,0){$\times$}}
\put(323,282){\makebox(0,0){$\times$}}
\put(324,281){\makebox(0,0){$\times$}}
\put(324,281){\makebox(0,0){$\times$}}
\put(325,280){\makebox(0,0){$\times$}}
\put(325,280){\makebox(0,0){$\times$}}
\put(326,280){\makebox(0,0){$\times$}}
\put(326,279){\makebox(0,0){$\times$}}
\put(327,279){\makebox(0,0){$\times$}}
\put(327,279){\makebox(0,0){$\times$}}
\put(328,279){\makebox(0,0){$\times$}}
\put(328,279){\makebox(0,0){$\times$}}
\put(329,279){\makebox(0,0){$\times$}}
\put(329,279){\makebox(0,0){$\times$}}
\put(330,279){\makebox(0,0){$\times$}}
\put(330,280){\makebox(0,0){$\times$}}
\put(331,280){\makebox(0,0){$\times$}}
\put(331,280){\makebox(0,0){$\times$}}
\put(332,281){\makebox(0,0){$\times$}}
\put(332,282){\makebox(0,0){$\times$}}
\put(333,283){\makebox(0,0){$\times$}}
\put(333,283){\makebox(0,0){$\times$}}
\put(334,284){\makebox(0,0){$\times$}}
\put(334,285){\makebox(0,0){$\times$}}
\put(335,286){\makebox(0,0){$\times$}}
\put(335,287){\makebox(0,0){$\times$}}
\put(336,288){\makebox(0,0){$\times$}}
\put(336,290){\makebox(0,0){$\times$}}
\put(337,291){\makebox(0,0){$\times$}}
\put(338,292){\makebox(0,0){$\times$}}
\put(338,294){\makebox(0,0){$\times$}}
\put(339,295){\makebox(0,0){$\times$}}
\put(339,297){\makebox(0,0){$\times$}}
\put(340,299){\makebox(0,0){$\times$}}
\put(340,300){\makebox(0,0){$\times$}}
\put(341,302){\makebox(0,0){$\times$}}
\put(341,304){\makebox(0,0){$\times$}}
\put(342,306){\makebox(0,0){$\times$}}
\put(342,308){\makebox(0,0){$\times$}}
\put(343,311){\makebox(0,0){$\times$}}
\put(343,313){\makebox(0,0){$\times$}}
\put(344,315){\makebox(0,0){$\times$}}
\put(344,317){\makebox(0,0){$\times$}}
\put(345,320){\makebox(0,0){$\times$}}
\put(345,322){\makebox(0,0){$\times$}}
\put(346,325){\makebox(0,0){$\times$}}
\put(346,327){\makebox(0,0){$\times$}}
\put(347,330){\makebox(0,0){$\times$}}
\put(347,332){\makebox(0,0){$\times$}}
\put(348,335){\makebox(0,0){$\times$}}
\put(348,338){\makebox(0,0){$\times$}}
\put(349,341){\makebox(0,0){$\times$}}
\put(349,344){\makebox(0,0){$\times$}}
\put(350,346){\makebox(0,0){$\times$}}
\put(350,349){\makebox(0,0){$\times$}}
\put(351,353){\makebox(0,0){$\times$}}
\put(351,356){\makebox(0,0){$\times$}}
\put(352,359){\makebox(0,0){$\times$}}
\put(352,362){\makebox(0,0){$\times$}}
\put(353,365){\makebox(0,0){$\times$}}
\put(353,369){\makebox(0,0){$\times$}}
\put(354,372){\makebox(0,0){$\times$}}
\put(355,376){\makebox(0,0){$\times$}}
\put(355,379){\makebox(0,0){$\times$}}
\put(356,383){\makebox(0,0){$\times$}}
\put(356,386){\makebox(0,0){$\times$}}
\put(357,390){\makebox(0,0){$\times$}}
\put(357,393){\makebox(0,0){$\times$}}
\put(358,397){\makebox(0,0){$\times$}}
\put(358,400){\makebox(0,0){$\times$}}
\put(359,404){\makebox(0,0){$\times$}}
\put(359,408){\makebox(0,0){$\times$}}
\put(360,411){\makebox(0,0){$\times$}}
\put(360,415){\makebox(0,0){$\times$}}
\put(361,419){\makebox(0,0){$\times$}}
\put(361,423){\makebox(0,0){$\times$}}
\put(362,426){\makebox(0,0){$\times$}}
\put(362,430){\makebox(0,0){$\times$}}
\put(363,434){\makebox(0,0){$\times$}}
\put(363,438){\makebox(0,0){$\times$}}
\put(364,442){\makebox(0,0){$\times$}}
\put(364,446){\makebox(0,0){$\times$}}
\put(365,450){\makebox(0,0){$\times$}}
\put(365,454){\makebox(0,0){$\times$}}
\put(366,458){\makebox(0,0){$\times$}}
\put(366,461){\makebox(0,0){$\times$}}
\put(367,465){\makebox(0,0){$\times$}}
\put(367,469){\makebox(0,0){$\times$}}
\put(368,473){\makebox(0,0){$\times$}}
\put(368,477){\makebox(0,0){$\times$}}
\put(369,481){\makebox(0,0){$\times$}}
\put(369,485){\makebox(0,0){$\times$}}
\put(370,489){\makebox(0,0){$\times$}}
\put(370,493){\makebox(0,0){$\times$}}
\put(371,497){\makebox(0,0){$\times$}}
\put(372,501){\makebox(0,0){$\times$}}
\put(372,505){\makebox(0,0){$\times$}}
\put(373,509){\makebox(0,0){$\times$}}
\put(373,513){\makebox(0,0){$\times$}}
\put(374,517){\makebox(0,0){$\times$}}
\put(374,521){\makebox(0,0){$\times$}}
\put(375,525){\makebox(0,0){$\times$}}
\put(375,529){\makebox(0,0){$\times$}}
\put(376,533){\makebox(0,0){$\times$}}
\put(376,537){\makebox(0,0){$\times$}}
\put(377,540){\makebox(0,0){$\times$}}
\put(377,544){\makebox(0,0){$\times$}}
\put(378,548){\makebox(0,0){$\times$}}
\put(378,552){\makebox(0,0){$\times$}}
\put(379,556){\makebox(0,0){$\times$}}
\put(379,559){\makebox(0,0){$\times$}}
\put(380,563){\makebox(0,0){$\times$}}
\put(380,567){\makebox(0,0){$\times$}}
\put(381,571){\makebox(0,0){$\times$}}
\put(381,574){\makebox(0,0){$\times$}}
\put(382,578){\makebox(0,0){$\times$}}
\put(382,582){\makebox(0,0){$\times$}}
\put(383,585){\makebox(0,0){$\times$}}
\put(383,589){\makebox(0,0){$\times$}}
\put(384,592){\makebox(0,0){$\times$}}
\put(384,596){\makebox(0,0){$\times$}}
\put(385,599){\makebox(0,0){$\times$}}
\put(385,602){\makebox(0,0){$\times$}}
\put(386,606){\makebox(0,0){$\times$}}
\put(386,609){\makebox(0,0){$\times$}}
\put(387,612){\makebox(0,0){$\times$}}
\put(387,615){\makebox(0,0){$\times$}}
\put(388,619){\makebox(0,0){$\times$}}
\put(389,622){\makebox(0,0){$\times$}}
\put(389,625){\makebox(0,0){$\times$}}
\put(390,628){\makebox(0,0){$\times$}}
\put(390,631){\makebox(0,0){$\times$}}
\put(391,634){\makebox(0,0){$\times$}}
\put(391,637){\makebox(0,0){$\times$}}
\put(392,640){\makebox(0,0){$\times$}}
\put(392,642){\makebox(0,0){$\times$}}
\put(393,645){\makebox(0,0){$\times$}}
\put(393,648){\makebox(0,0){$\times$}}
\put(394,650){\makebox(0,0){$\times$}}
\put(394,653){\makebox(0,0){$\times$}}
\put(395,655){\makebox(0,0){$\times$}}
\put(395,658){\makebox(0,0){$\times$}}
\put(396,660){\makebox(0,0){$\times$}}
\put(396,662){\makebox(0,0){$\times$}}
\put(397,665){\makebox(0,0){$\times$}}
\put(397,667){\makebox(0,0){$\times$}}
\put(398,669){\makebox(0,0){$\times$}}
\put(398,671){\makebox(0,0){$\times$}}
\put(399,673){\makebox(0,0){$\times$}}
\put(399,675){\makebox(0,0){$\times$}}
\put(400,677){\makebox(0,0){$\times$}}
\put(400,679){\makebox(0,0){$\times$}}
\put(401,680){\makebox(0,0){$\times$}}
\put(401,682){\makebox(0,0){$\times$}}
\put(402,684){\makebox(0,0){$\times$}}
\put(402,685){\makebox(0,0){$\times$}}
\put(403,687){\makebox(0,0){$\times$}}
\put(403,688){\makebox(0,0){$\times$}}
\put(404,690){\makebox(0,0){$\times$}}
\put(404,691){\makebox(0,0){$\times$}}
\put(405,692){\makebox(0,0){$\times$}}
\put(406,693){\makebox(0,0){$\times$}}
\put(406,694){\makebox(0,0){$\times$}}
\put(407,695){\makebox(0,0){$\times$}}
\put(407,696){\makebox(0,0){$\times$}}
\put(408,697){\makebox(0,0){$\times$}}
\put(408,698){\makebox(0,0){$\times$}}
\put(409,699){\makebox(0,0){$\times$}}
\put(409,699){\makebox(0,0){$\times$}}
\put(410,700){\makebox(0,0){$\times$}}
\put(410,701){\makebox(0,0){$\times$}}
\put(411,701){\makebox(0,0){$\times$}}
\put(411,701){\makebox(0,0){$\times$}}
\put(412,702){\makebox(0,0){$\times$}}
\put(412,702){\makebox(0,0){$\times$}}
\put(413,702){\makebox(0,0){$\times$}}
\put(413,702){\makebox(0,0){$\times$}}
\put(414,702){\makebox(0,0){$\times$}}
\put(414,702){\makebox(0,0){$\times$}}
\put(415,702){\makebox(0,0){$\times$}}
\put(415,702){\makebox(0,0){$\times$}}
\put(416,702){\makebox(0,0){$\times$}}
\put(416,702){\makebox(0,0){$\times$}}
\put(417,701){\makebox(0,0){$\times$}}
\put(417,701){\makebox(0,0){$\times$}}
\put(418,701){\makebox(0,0){$\times$}}
\put(418,700){\makebox(0,0){$\times$}}
\put(419,699){\makebox(0,0){$\times$}}
\put(419,699){\makebox(0,0){$\times$}}
\put(420,698){\makebox(0,0){$\times$}}
\put(420,697){\makebox(0,0){$\times$}}
\put(421,696){\makebox(0,0){$\times$}}
\put(421,695){\makebox(0,0){$\times$}}
\put(422,694){\makebox(0,0){$\times$}}
\put(423,693){\makebox(0,0){$\times$}}
\put(423,692){\makebox(0,0){$\times$}}
\put(424,691){\makebox(0,0){$\times$}}
\put(424,689){\makebox(0,0){$\times$}}
\put(425,688){\makebox(0,0){$\times$}}
\put(425,687){\makebox(0,0){$\times$}}
\put(426,685){\makebox(0,0){$\times$}}
\put(426,684){\makebox(0,0){$\times$}}
\put(427,682){\makebox(0,0){$\times$}}
\put(427,681){\makebox(0,0){$\times$}}
\put(428,679){\makebox(0,0){$\times$}}
\put(428,677){\makebox(0,0){$\times$}}
\put(429,675){\makebox(0,0){$\times$}}
\put(429,673){\makebox(0,0){$\times$}}
\put(430,671){\makebox(0,0){$\times$}}
\put(430,669){\makebox(0,0){$\times$}}
\put(431,668){\makebox(0,0){$\times$}}
\put(431,666){\makebox(0,0){$\times$}}
\put(432,663){\makebox(0,0){$\times$}}
\put(432,661){\makebox(0,0){$\times$}}
\put(433,659){\makebox(0,0){$\times$}}
\put(433,657){\makebox(0,0){$\times$}}
\put(434,654){\makebox(0,0){$\times$}}
\put(434,652){\makebox(0,0){$\times$}}
\put(435,650){\makebox(0,0){$\times$}}
\put(435,647){\makebox(0,0){$\times$}}
\put(436,645){\makebox(0,0){$\times$}}
\put(436,642){\makebox(0,0){$\times$}}
\put(437,640){\makebox(0,0){$\times$}}
\put(437,637){\makebox(0,0){$\times$}}
\put(438,634){\makebox(0,0){$\times$}}
\put(438,631){\makebox(0,0){$\times$}}
\put(439,629){\makebox(0,0){$\times$}}
\put(440,626){\makebox(0,0){$\times$}}
\put(440,623){\makebox(0,0){$\times$}}
\put(441,620){\makebox(0,0){$\times$}}
\put(441,617){\makebox(0,0){$\times$}}
\put(442,615){\makebox(0,0){$\times$}}
\put(442,612){\makebox(0,0){$\times$}}
\put(443,609){\makebox(0,0){$\times$}}
\put(443,606){\makebox(0,0){$\times$}}
\put(444,603){\makebox(0,0){$\times$}}
\put(444,599){\makebox(0,0){$\times$}}
\put(445,596){\makebox(0,0){$\times$}}
\put(445,593){\makebox(0,0){$\times$}}
\put(446,590){\makebox(0,0){$\times$}}
\put(446,587){\makebox(0,0){$\times$}}
\put(447,584){\makebox(0,0){$\times$}}
\put(447,581){\makebox(0,0){$\times$}}
\put(448,578){\makebox(0,0){$\times$}}
\put(448,574){\makebox(0,0){$\times$}}
\put(449,571){\makebox(0,0){$\times$}}
\put(449,568){\makebox(0,0){$\times$}}
\put(450,564){\makebox(0,0){$\times$}}
\put(450,561){\makebox(0,0){$\times$}}
\put(451,558){\makebox(0,0){$\times$}}
\put(451,554){\makebox(0,0){$\times$}}
\put(452,551){\makebox(0,0){$\times$}}
\put(452,548){\makebox(0,0){$\times$}}
\put(453,544){\makebox(0,0){$\times$}}
\put(453,541){\makebox(0,0){$\times$}}
\put(454,538){\makebox(0,0){$\times$}}
\put(454,535){\makebox(0,0){$\times$}}
\put(455,531){\makebox(0,0){$\times$}}
\put(455,528){\makebox(0,0){$\times$}}
\put(456,524){\makebox(0,0){$\times$}}
\put(457,521){\makebox(0,0){$\times$}}
\put(457,517){\makebox(0,0){$\times$}}
\put(458,514){\makebox(0,0){$\times$}}
\put(458,511){\makebox(0,0){$\times$}}
\put(459,504){\makebox(0,0){$\times$}}
\put(460,501){\makebox(0,0){$\times$}}
\put(460,497){\makebox(0,0){$\times$}}
\put(461,494){\makebox(0,0){$\times$}}
\put(461,491){\makebox(0,0){$\times$}}
\put(462,488){\makebox(0,0){$\times$}}
\put(462,484){\makebox(0,0){$\times$}}
\put(463,481){\makebox(0,0){$\times$}}
\put(463,478){\makebox(0,0){$\times$}}
\put(464,474){\makebox(0,0){$\times$}}
\put(464,471){\makebox(0,0){$\times$}}
\put(465,468){\makebox(0,0){$\times$}}
\put(465,465){\makebox(0,0){$\times$}}
\put(466,462){\makebox(0,0){$\times$}}
\put(466,459){\makebox(0,0){$\times$}}
\put(467,456){\makebox(0,0){$\times$}}
\put(467,453){\makebox(0,0){$\times$}}
\put(468,449){\makebox(0,0){$\times$}}
\put(468,446){\makebox(0,0){$\times$}}
\put(469,444){\makebox(0,0){$\times$}}
\put(469,440){\makebox(0,0){$\times$}}
\put(470,438){\makebox(0,0){$\times$}}
\put(470,435){\makebox(0,0){$\times$}}
\put(471,432){\makebox(0,0){$\times$}}
\put(471,429){\makebox(0,0){$\times$}}
\put(472,426){\makebox(0,0){$\times$}}
\put(472,423){\makebox(0,0){$\times$}}
\put(473,421){\makebox(0,0){$\times$}}
\put(474,418){\makebox(0,0){$\times$}}
\put(474,415){\makebox(0,0){$\times$}}
\put(475,413){\makebox(0,0){$\times$}}
\put(475,410){\makebox(0,0){$\times$}}
\put(476,407){\makebox(0,0){$\times$}}
\put(476,405){\makebox(0,0){$\times$}}
\put(477,403){\makebox(0,0){$\times$}}
\put(477,400){\makebox(0,0){$\times$}}
\put(478,398){\makebox(0,0){$\times$}}
\put(478,396){\makebox(0,0){$\times$}}
\put(479,393){\makebox(0,0){$\times$}}
\put(479,391){\makebox(0,0){$\times$}}
\put(480,389){\makebox(0,0){$\times$}}
\put(480,387){\makebox(0,0){$\times$}}
\put(481,385){\makebox(0,0){$\times$}}
\put(481,383){\makebox(0,0){$\times$}}
\put(482,381){\makebox(0,0){$\times$}}
\put(482,379){\makebox(0,0){$\times$}}
\put(483,377){\makebox(0,0){$\times$}}
\put(483,376){\makebox(0,0){$\times$}}
\put(484,374){\makebox(0,0){$\times$}}
\put(484,372){\makebox(0,0){$\times$}}
\put(485,370){\makebox(0,0){$\times$}}
\put(485,369){\makebox(0,0){$\times$}}
\put(486,367){\makebox(0,0){$\times$}}
\put(486,366){\makebox(0,0){$\times$}}
\put(487,365){\makebox(0,0){$\times$}}
\put(487,363){\makebox(0,0){$\times$}}
\put(488,362){\makebox(0,0){$\times$}}
\put(488,361){\makebox(0,0){$\times$}}
\put(489,360){\makebox(0,0){$\times$}}
\put(489,358){\makebox(0,0){$\times$}}
\put(490,357){\makebox(0,0){$\times$}}
\put(491,356){\makebox(0,0){$\times$}}
\put(491,355){\makebox(0,0){$\times$}}
\put(492,355){\makebox(0,0){$\times$}}
\put(492,354){\makebox(0,0){$\times$}}
\put(493,353){\makebox(0,0){$\times$}}
\put(493,352){\makebox(0,0){$\times$}}
\put(494,351){\makebox(0,0){$\times$}}
\put(494,351){\makebox(0,0){$\times$}}
\put(495,350){\makebox(0,0){$\times$}}
\put(495,350){\makebox(0,0){$\times$}}
\put(496,349){\makebox(0,0){$\times$}}
\put(496,349){\makebox(0,0){$\times$}}
\put(497,349){\makebox(0,0){$\times$}}
\put(497,349){\makebox(0,0){$\times$}}
\put(498,348){\makebox(0,0){$\times$}}
\put(498,348){\makebox(0,0){$\times$}}
\put(499,348){\makebox(0,0){$\times$}}
\put(499,348){\makebox(0,0){$\times$}}
\put(500,348){\makebox(0,0){$\times$}}
\put(500,348){\makebox(0,0){$\times$}}
\put(501,349){\makebox(0,0){$\times$}}
\put(501,349){\makebox(0,0){$\times$}}
\put(502,349){\makebox(0,0){$\times$}}
\put(502,350){\makebox(0,0){$\times$}}
\put(503,350){\makebox(0,0){$\times$}}
\put(503,351){\makebox(0,0){$\times$}}
\put(504,351){\makebox(0,0){$\times$}}
\put(504,352){\makebox(0,0){$\times$}}
\put(505,353){\makebox(0,0){$\times$}}
\put(505,353){\makebox(0,0){$\times$}}
\put(506,354){\makebox(0,0){$\times$}}
\put(506,355){\makebox(0,0){$\times$}}
\put(507,356){\makebox(0,0){$\times$}}
\put(508,357){\makebox(0,0){$\times$}}
\put(508,358){\makebox(0,0){$\times$}}
\put(509,359){\makebox(0,0){$\times$}}
\put(509,360){\makebox(0,0){$\times$}}
\put(510,362){\makebox(0,0){$\times$}}
\put(510,362){\makebox(0,0){$\times$}}
\put(511,364){\makebox(0,0){$\times$}}
\put(511,365){\makebox(0,0){$\times$}}
\put(512,366){\makebox(0,0){$\times$}}
\put(512,368){\makebox(0,0){$\times$}}
\put(513,369){\makebox(0,0){$\times$}}
\put(513,370){\makebox(0,0){$\times$}}
\put(514,372){\makebox(0,0){$\times$}}
\put(514,374){\makebox(0,0){$\times$}}
\put(515,376){\makebox(0,0){$\times$}}
\put(515,377){\makebox(0,0){$\times$}}
\put(516,379){\makebox(0,0){$\times$}}
\put(516,381){\makebox(0,0){$\times$}}
\put(517,382){\makebox(0,0){$\times$}}
\put(517,384){\makebox(0,0){$\times$}}
\put(518,386){\makebox(0,0){$\times$}}
\put(518,388){\makebox(0,0){$\times$}}
\put(519,390){\makebox(0,0){$\times$}}
\put(519,391){\makebox(0,0){$\times$}}
\put(520,393){\makebox(0,0){$\times$}}
\put(520,396){\makebox(0,0){$\times$}}
\put(521,398){\makebox(0,0){$\times$}}
\put(521,400){\makebox(0,0){$\times$}}
\put(522,402){\makebox(0,0){$\times$}}
\put(522,404){\makebox(0,0){$\times$}}
\put(523,406){\makebox(0,0){$\times$}}
\put(523,409){\makebox(0,0){$\times$}}
\put(524,411){\makebox(0,0){$\times$}}
\put(525,413){\makebox(0,0){$\times$}}
\put(525,415){\makebox(0,0){$\times$}}
\put(526,418){\makebox(0,0){$\times$}}
\put(526,420){\makebox(0,0){$\times$}}
\put(527,422){\makebox(0,0){$\times$}}
\put(527,425){\makebox(0,0){$\times$}}
\put(528,427){\makebox(0,0){$\times$}}
\put(528,429){\makebox(0,0){$\times$}}
\put(529,432){\makebox(0,0){$\times$}}
\put(529,434){\makebox(0,0){$\times$}}
\put(530,437){\makebox(0,0){$\times$}}
\put(530,439){\makebox(0,0){$\times$}}
\put(531,442){\makebox(0,0){$\times$}}
\put(531,444){\makebox(0,0){$\times$}}
\put(532,447){\makebox(0,0){$\times$}}
\put(532,449){\makebox(0,0){$\times$}}
\put(533,452){\makebox(0,0){$\times$}}
\put(533,454){\makebox(0,0){$\times$}}
\put(534,457){\makebox(0,0){$\times$}}
\put(534,460){\makebox(0,0){$\times$}}
\put(535,462){\makebox(0,0){$\times$}}
\put(535,465){\makebox(0,0){$\times$}}
\put(536,467){\makebox(0,0){$\times$}}
\put(536,470){\makebox(0,0){$\times$}}
\put(537,473){\makebox(0,0){$\times$}}
\put(537,475){\makebox(0,0){$\times$}}
\put(538,478){\makebox(0,0){$\times$}}
\put(538,481){\makebox(0,0){$\times$}}
\put(539,484){\makebox(0,0){$\times$}}
\put(539,486){\makebox(0,0){$\times$}}
\put(540,489){\makebox(0,0){$\times$}}
\put(540,491){\makebox(0,0){$\times$}}
\put(541,494){\makebox(0,0){$\times$}}
\put(542,497){\makebox(0,0){$\times$}}
\put(542,500){\makebox(0,0){$\times$}}
\put(543,502){\makebox(0,0){$\times$}}
\put(543,505){\makebox(0,0){$\times$}}
\put(544,508){\makebox(0,0){$\times$}}
\put(545,513){\makebox(0,0){$\times$}}
\put(545,516){\makebox(0,0){$\times$}}
\put(546,518){\makebox(0,0){$\times$}}
\put(546,521){\makebox(0,0){$\times$}}
\put(547,524){\makebox(0,0){$\times$}}
\put(547,526){\makebox(0,0){$\times$}}
\put(548,529){\makebox(0,0){$\times$}}
\put(548,531){\makebox(0,0){$\times$}}
\put(549,534){\makebox(0,0){$\times$}}
\put(549,537){\makebox(0,0){$\times$}}
\put(550,539){\makebox(0,0){$\times$}}
\put(550,542){\makebox(0,0){$\times$}}
\put(551,544){\makebox(0,0){$\times$}}
\put(551,547){\makebox(0,0){$\times$}}
\put(552,549){\makebox(0,0){$\times$}}
\put(552,551){\makebox(0,0){$\times$}}
\put(553,554){\makebox(0,0){$\times$}}
\put(553,556){\makebox(0,0){$\times$}}
\put(554,559){\makebox(0,0){$\times$}}
\put(554,561){\makebox(0,0){$\times$}}
\put(555,563){\makebox(0,0){$\times$}}
\put(555,566){\makebox(0,0){$\times$}}
\put(556,568){\makebox(0,0){$\times$}}
\put(556,570){\makebox(0,0){$\times$}}
\put(557,572){\makebox(0,0){$\times$}}
\put(557,575){\makebox(0,0){$\times$}}
\put(558,577){\makebox(0,0){$\times$}}
\put(559,579){\makebox(0,0){$\times$}}
\put(559,581){\makebox(0,0){$\times$}}
\put(560,583){\makebox(0,0){$\times$}}
\put(560,585){\makebox(0,0){$\times$}}
\put(561,587){\makebox(0,0){$\times$}}
\put(561,589){\makebox(0,0){$\times$}}
\put(562,591){\makebox(0,0){$\times$}}
\put(562,593){\makebox(0,0){$\times$}}
\put(563,595){\makebox(0,0){$\times$}}
\put(563,596){\makebox(0,0){$\times$}}
\put(564,598){\makebox(0,0){$\times$}}
\put(564,600){\makebox(0,0){$\times$}}
\put(565,602){\makebox(0,0){$\times$}}
\put(565,603){\makebox(0,0){$\times$}}
\put(566,605){\makebox(0,0){$\times$}}
\put(566,607){\makebox(0,0){$\times$}}
\put(567,608){\makebox(0,0){$\times$}}
\put(567,610){\makebox(0,0){$\times$}}
\put(568,611){\makebox(0,0){$\times$}}
\put(568,613){\makebox(0,0){$\times$}}
\put(569,614){\makebox(0,0){$\times$}}
\put(569,615){\makebox(0,0){$\times$}}
\put(570,617){\makebox(0,0){$\times$}}
\put(570,618){\makebox(0,0){$\times$}}
\put(571,619){\makebox(0,0){$\times$}}
\put(571,620){\makebox(0,0){$\times$}}
\put(572,622){\makebox(0,0){$\times$}}
\put(572,623){\makebox(0,0){$\times$}}
\put(573,624){\makebox(0,0){$\times$}}
\put(573,625){\makebox(0,0){$\times$}}
\put(574,626){\makebox(0,0){$\times$}}
\put(574,627){\makebox(0,0){$\times$}}
\put(575,627){\makebox(0,0){$\times$}}
\put(576,629){\makebox(0,0){$\times$}}
\put(576,629){\makebox(0,0){$\times$}}
\put(577,630){\makebox(0,0){$\times$}}
\put(577,631){\makebox(0,0){$\times$}}
\put(578,631){\makebox(0,0){$\times$}}
\put(578,632){\makebox(0,0){$\times$}}
\put(579,633){\makebox(0,0){$\times$}}
\put(579,633){\makebox(0,0){$\times$}}
\put(580,633){\makebox(0,0){$\times$}}
\put(580,634){\makebox(0,0){$\times$}}
\put(581,634){\makebox(0,0){$\times$}}
\put(581,634){\makebox(0,0){$\times$}}
\put(582,635){\makebox(0,0){$\times$}}
\put(582,635){\makebox(0,0){$\times$}}
\put(583,635){\makebox(0,0){$\times$}}
\put(583,635){\makebox(0,0){$\times$}}
\put(584,635){\makebox(0,0){$\times$}}
\put(584,635){\makebox(0,0){$\times$}}
\put(585,635){\makebox(0,0){$\times$}}
\put(585,635){\makebox(0,0){$\times$}}
\put(586,635){\makebox(0,0){$\times$}}
\put(586,635){\makebox(0,0){$\times$}}
\put(587,635){\makebox(0,0){$\times$}}
\put(587,635){\makebox(0,0){$\times$}}
\put(588,634){\makebox(0,0){$\times$}}
\put(588,634){\makebox(0,0){$\times$}}
\put(589,634){\makebox(0,0){$\times$}}
\put(589,633){\makebox(0,0){$\times$}}
\put(590,633){\makebox(0,0){$\times$}}
\put(590,632){\makebox(0,0){$\times$}}
\put(591,632){\makebox(0,0){$\times$}}
\put(591,631){\makebox(0,0){$\times$}}
\put(592,631){\makebox(0,0){$\times$}}
\put(593,630){\makebox(0,0){$\times$}}
\put(593,629){\makebox(0,0){$\times$}}
\put(594,628){\makebox(0,0){$\times$}}
\put(594,627){\makebox(0,0){$\times$}}
\put(595,626){\makebox(0,0){$\times$}}
\put(595,626){\makebox(0,0){$\times$}}
\put(596,625){\makebox(0,0){$\times$}}
\put(596,624){\makebox(0,0){$\times$}}
\put(597,622){\makebox(0,0){$\times$}}
\put(597,622){\makebox(0,0){$\times$}}
\put(598,620){\makebox(0,0){$\times$}}
\put(598,619){\makebox(0,0){$\times$}}
\put(599,618){\makebox(0,0){$\times$}}
\put(599,617){\makebox(0,0){$\times$}}
\put(600,615){\makebox(0,0){$\times$}}
\put(600,614){\makebox(0,0){$\times$}}
\put(601,613){\makebox(0,0){$\times$}}
\put(601,612){\makebox(0,0){$\times$}}
\put(602,610){\makebox(0,0){$\times$}}
\put(602,608){\makebox(0,0){$\times$}}
\put(603,607){\makebox(0,0){$\times$}}
\put(603,605){\makebox(0,0){$\times$}}
\put(604,604){\makebox(0,0){$\times$}}
\put(604,602){\makebox(0,0){$\times$}}
\put(605,601){\makebox(0,0){$\times$}}
\put(605,599){\makebox(0,0){$\times$}}
\put(606,597){\makebox(0,0){$\times$}}
\put(606,596){\makebox(0,0){$\times$}}
\put(607,594){\makebox(0,0){$\times$}}
\put(607,592){\makebox(0,0){$\times$}}
\put(608,590){\makebox(0,0){$\times$}}
\put(608,589){\makebox(0,0){$\times$}}
\put(609,587){\makebox(0,0){$\times$}}
\put(610,585){\makebox(0,0){$\times$}}
\put(610,583){\makebox(0,0){$\times$}}
\put(611,581){\makebox(0,0){$\times$}}
\put(611,579){\makebox(0,0){$\times$}}
\put(612,577){\makebox(0,0){$\times$}}
\put(612,575){\makebox(0,0){$\times$}}
\put(613,573){\makebox(0,0){$\times$}}
\put(613,571){\makebox(0,0){$\times$}}
\put(614,569){\makebox(0,0){$\times$}}
\put(614,567){\makebox(0,0){$\times$}}
\put(615,564){\makebox(0,0){$\times$}}
\put(615,563){\makebox(0,0){$\times$}}
\put(616,560){\makebox(0,0){$\times$}}
\put(616,558){\makebox(0,0){$\times$}}
\put(617,556){\makebox(0,0){$\times$}}
\put(617,554){\makebox(0,0){$\times$}}
\put(618,552){\makebox(0,0){$\times$}}
\put(618,549){\makebox(0,0){$\times$}}
\put(619,547){\makebox(0,0){$\times$}}
\put(619,545){\makebox(0,0){$\times$}}
\put(620,543){\makebox(0,0){$\times$}}
\put(620,540){\makebox(0,0){$\times$}}
\put(621,538){\makebox(0,0){$\times$}}
\put(621,536){\makebox(0,0){$\times$}}
\put(622,533){\makebox(0,0){$\times$}}
\put(622,531){\makebox(0,0){$\times$}}
\put(623,529){\makebox(0,0){$\times$}}
\put(623,527){\makebox(0,0){$\times$}}
\put(624,524){\makebox(0,0){$\times$}}
\put(624,522){\makebox(0,0){$\times$}}
\put(625,520){\makebox(0,0){$\times$}}
\put(625,517){\makebox(0,0){$\times$}}
\put(626,515){\makebox(0,0){$\times$}}
\put(627,513){\makebox(0,0){$\times$}}
\put(627,511){\makebox(0,0){$\times$}}
\put(628,509){\makebox(0,0){$\times$}}
\put(628,506){\makebox(0,0){$\times$}}
\put(629,504){\makebox(0,0){$\times$}}
\put(629,502){\makebox(0,0){$\times$}}
\put(630,499){\makebox(0,0){$\times$}}
\put(630,497){\makebox(0,0){$\times$}}
\put(631,495){\makebox(0,0){$\times$}}
\put(631,493){\makebox(0,0){$\times$}}
\put(632,490){\makebox(0,0){$\times$}}
\put(632,488){\makebox(0,0){$\times$}}
\put(633,486){\makebox(0,0){$\times$}}
\put(633,484){\makebox(0,0){$\times$}}
\put(634,481){\makebox(0,0){$\times$}}
\put(634,479){\makebox(0,0){$\times$}}
\put(635,477){\makebox(0,0){$\times$}}
\put(635,475){\makebox(0,0){$\times$}}
\put(636,473){\makebox(0,0){$\times$}}
\put(636,470){\makebox(0,0){$\times$}}
\put(637,468){\makebox(0,0){$\times$}}
\put(637,467){\makebox(0,0){$\times$}}
\put(638,464){\makebox(0,0){$\times$}}
\put(638,462){\makebox(0,0){$\times$}}
\put(639,460){\makebox(0,0){$\times$}}
\put(639,458){\makebox(0,0){$\times$}}
\put(640,456){\makebox(0,0){$\times$}}
\put(640,454){\makebox(0,0){$\times$}}
\put(641,453){\makebox(0,0){$\times$}}
\put(641,451){\makebox(0,0){$\times$}}
\put(642,449){\makebox(0,0){$\times$}}
\put(643,447){\makebox(0,0){$\times$}}
\put(643,445){\makebox(0,0){$\times$}}
\put(644,443){\makebox(0,0){$\times$}}
\put(644,441){\makebox(0,0){$\times$}}
\put(645,440){\makebox(0,0){$\times$}}
\put(645,438){\makebox(0,0){$\times$}}
\put(646,436){\makebox(0,0){$\times$}}
\put(646,435){\makebox(0,0){$\times$}}
\put(647,433){\makebox(0,0){$\times$}}
\put(647,431){\makebox(0,0){$\times$}}
\put(648,430){\makebox(0,0){$\times$}}
\put(648,428){\makebox(0,0){$\times$}}
\put(649,427){\makebox(0,0){$\times$}}
\put(649,425){\makebox(0,0){$\times$}}
\put(650,424){\makebox(0,0){$\times$}}
\put(650,423){\makebox(0,0){$\times$}}
\put(651,421){\makebox(0,0){$\times$}}
\put(651,420){\makebox(0,0){$\times$}}
\put(652,418){\makebox(0,0){$\times$}}
\put(652,417){\makebox(0,0){$\times$}}
\put(653,416){\makebox(0,0){$\times$}}
\put(653,415){\makebox(0,0){$\times$}}
\put(654,414){\makebox(0,0){$\times$}}
\put(654,413){\makebox(0,0){$\times$}}
\put(655,412){\makebox(0,0){$\times$}}
\put(655,411){\makebox(0,0){$\times$}}
\put(656,410){\makebox(0,0){$\times$}}
\put(656,409){\makebox(0,0){$\times$}}
\put(657,408){\makebox(0,0){$\times$}}
\put(657,407){\makebox(0,0){$\times$}}
\put(658,406){\makebox(0,0){$\times$}}
\put(658,405){\makebox(0,0){$\times$}}
\put(659,405){\makebox(0,0){$\times$}}
\put(660,404){\makebox(0,0){$\times$}}
\put(660,404){\makebox(0,0){$\times$}}
\put(661,403){\makebox(0,0){$\times$}}
\put(661,402){\makebox(0,0){$\times$}}
\put(662,402){\makebox(0,0){$\times$}}
\put(662,401){\makebox(0,0){$\times$}}
\put(663,401){\makebox(0,0){$\times$}}
\put(663,400){\makebox(0,0){$\times$}}
\put(664,400){\makebox(0,0){$\times$}}
\put(664,400){\makebox(0,0){$\times$}}
\put(665,400){\makebox(0,0){$\times$}}
\put(665,399){\makebox(0,0){$\times$}}
\put(666,399){\makebox(0,0){$\times$}}
\put(666,399){\makebox(0,0){$\times$}}
\put(667,399){\makebox(0,0){$\times$}}
\put(667,399){\makebox(0,0){$\times$}}
\put(668,399){\makebox(0,0){$\times$}}
\put(668,399){\makebox(0,0){$\times$}}
\put(669,399){\makebox(0,0){$\times$}}
\put(669,399){\makebox(0,0){$\times$}}
\put(670,399){\makebox(0,0){$\times$}}
\put(670,399){\makebox(0,0){$\times$}}
\put(671,399){\makebox(0,0){$\times$}}
\put(671,399){\makebox(0,0){$\times$}}
\put(672,400){\makebox(0,0){$\times$}}
\put(672,400){\makebox(0,0){$\times$}}
\put(673,400){\makebox(0,0){$\times$}}
\put(673,401){\makebox(0,0){$\times$}}
\put(674,401){\makebox(0,0){$\times$}}
\put(674,402){\makebox(0,0){$\times$}}
\put(675,402){\makebox(0,0){$\times$}}
\put(675,403){\makebox(0,0){$\times$}}
\put(676,403){\makebox(0,0){$\times$}}
\put(677,404){\makebox(0,0){$\times$}}
\put(677,404){\makebox(0,0){$\times$}}
\put(678,405){\makebox(0,0){$\times$}}
\put(678,406){\makebox(0,0){$\times$}}
\put(679,407){\makebox(0,0){$\times$}}
\put(679,407){\makebox(0,0){$\times$}}
\put(680,408){\makebox(0,0){$\times$}}
\put(680,409){\makebox(0,0){$\times$}}
\put(681,410){\makebox(0,0){$\times$}}
\put(681,411){\makebox(0,0){$\times$}}
\put(682,412){\makebox(0,0){$\times$}}
\put(682,412){\makebox(0,0){$\times$}}
\put(683,413){\makebox(0,0){$\times$}}
\put(683,414){\makebox(0,0){$\times$}}
\put(684,416){\makebox(0,0){$\times$}}
\put(684,417){\makebox(0,0){$\times$}}
\put(685,418){\makebox(0,0){$\times$}}
\put(685,419){\makebox(0,0){$\times$}}
\put(686,420){\makebox(0,0){$\times$}}
\put(686,421){\makebox(0,0){$\times$}}
\put(687,423){\makebox(0,0){$\times$}}
\put(687,424){\makebox(0,0){$\times$}}
\put(688,425){\makebox(0,0){$\times$}}
\put(688,426){\makebox(0,0){$\times$}}
\put(689,428){\makebox(0,0){$\times$}}
\put(689,429){\makebox(0,0){$\times$}}
\put(690,430){\makebox(0,0){$\times$}}
\put(690,432){\makebox(0,0){$\times$}}
\put(691,433){\makebox(0,0){$\times$}}
\put(691,435){\makebox(0,0){$\times$}}
\put(692,436){\makebox(0,0){$\times$}}
\put(692,438){\makebox(0,0){$\times$}}
\put(693,439){\makebox(0,0){$\times$}}
\put(694,441){\makebox(0,0){$\times$}}
\put(694,442){\makebox(0,0){$\times$}}
\put(695,444){\makebox(0,0){$\times$}}
\put(695,446){\makebox(0,0){$\times$}}
\put(696,447){\makebox(0,0){$\times$}}
\put(696,449){\makebox(0,0){$\times$}}
\put(697,451){\makebox(0,0){$\times$}}
\put(697,452){\makebox(0,0){$\times$}}
\put(698,454){\makebox(0,0){$\times$}}
\put(698,456){\makebox(0,0){$\times$}}
\put(699,457){\makebox(0,0){$\times$}}
\put(699,459){\makebox(0,0){$\times$}}
\put(700,461){\makebox(0,0){$\times$}}
\put(700,462){\makebox(0,0){$\times$}}
\put(701,464){\makebox(0,0){$\times$}}
\put(701,466){\makebox(0,0){$\times$}}
\put(702,468){\makebox(0,0){$\times$}}
\put(702,469){\makebox(0,0){$\times$}}
\put(703,471){\makebox(0,0){$\times$}}
\put(703,473){\makebox(0,0){$\times$}}
\put(704,475){\makebox(0,0){$\times$}}
\put(704,477){\makebox(0,0){$\times$}}
\put(705,479){\makebox(0,0){$\times$}}
\put(705,480){\makebox(0,0){$\times$}}
\put(706,482){\makebox(0,0){$\times$}}
\put(706,484){\makebox(0,0){$\times$}}
\put(707,486){\makebox(0,0){$\times$}}
\put(707,488){\makebox(0,0){$\times$}}
\put(708,489){\makebox(0,0){$\times$}}
\put(708,491){\makebox(0,0){$\times$}}
\put(709,493){\makebox(0,0){$\times$}}
\put(709,495){\makebox(0,0){$\times$}}
\put(710,497){\makebox(0,0){$\times$}}
\put(711,499){\makebox(0,0){$\times$}}
\put(711,501){\makebox(0,0){$\times$}}
\put(712,502){\makebox(0,0){$\times$}}
\put(712,504){\makebox(0,0){$\times$}}
\put(713,506){\makebox(0,0){$\times$}}
\put(713,508){\makebox(0,0){$\times$}}
\put(714,510){\makebox(0,0){$\times$}}
\put(714,512){\makebox(0,0){$\times$}}
\put(715,514){\makebox(0,0){$\times$}}
\put(715,515){\makebox(0,0){$\times$}}
\put(716,517){\makebox(0,0){$\times$}}
\put(716,519){\makebox(0,0){$\times$}}
\put(717,521){\makebox(0,0){$\times$}}
\put(717,523){\makebox(0,0){$\times$}}
\put(718,524){\makebox(0,0){$\times$}}
\put(718,526){\makebox(0,0){$\times$}}
\put(719,528){\makebox(0,0){$\times$}}
\put(719,530){\makebox(0,0){$\times$}}
\put(720,531){\makebox(0,0){$\times$}}
\put(720,533){\makebox(0,0){$\times$}}
\put(721,535){\makebox(0,0){$\times$}}
\put(721,536){\makebox(0,0){$\times$}}
\put(722,538){\makebox(0,0){$\times$}}
\put(722,540){\makebox(0,0){$\times$}}
\put(723,541){\makebox(0,0){$\times$}}
\put(723,543){\makebox(0,0){$\times$}}
\put(724,544){\makebox(0,0){$\times$}}
\put(724,546){\makebox(0,0){$\times$}}
\put(725,548){\makebox(0,0){$\times$}}
\put(725,549){\makebox(0,0){$\times$}}
\put(726,550){\makebox(0,0){$\times$}}
\put(726,552){\makebox(0,0){$\times$}}
\put(727,554){\makebox(0,0){$\times$}}
\put(728,555){\makebox(0,0){$\times$}}
\put(728,557){\makebox(0,0){$\times$}}
\put(729,558){\makebox(0,0){$\times$}}
\put(729,559){\makebox(0,0){$\times$}}
\put(730,561){\makebox(0,0){$\times$}}
\put(730,562){\makebox(0,0){$\times$}}
\put(731,564){\makebox(0,0){$\times$}}
\put(731,565){\makebox(0,0){$\times$}}
\put(732,566){\makebox(0,0){$\times$}}
\put(732,568){\makebox(0,0){$\times$}}
\put(733,569){\makebox(0,0){$\times$}}
\put(733,570){\makebox(0,0){$\times$}}
\put(734,571){\makebox(0,0){$\times$}}
\put(734,572){\makebox(0,0){$\times$}}
\put(735,573){\makebox(0,0){$\times$}}
\put(735,575){\makebox(0,0){$\times$}}
\put(736,576){\makebox(0,0){$\times$}}
\put(736,577){\makebox(0,0){$\times$}}
\put(737,578){\makebox(0,0){$\times$}}
\put(737,579){\makebox(0,0){$\times$}}
\put(738,580){\makebox(0,0){$\times$}}
\put(738,581){\makebox(0,0){$\times$}}
\put(739,582){\makebox(0,0){$\times$}}
\put(739,582){\makebox(0,0){$\times$}}
\put(740,583){\makebox(0,0){$\times$}}
\put(740,584){\makebox(0,0){$\times$}}
\put(741,585){\makebox(0,0){$\times$}}
\put(741,585){\makebox(0,0){$\times$}}
\put(742,586){\makebox(0,0){$\times$}}
\put(742,587){\makebox(0,0){$\times$}}
\put(743,588){\makebox(0,0){$\times$}}
\put(743,588){\makebox(0,0){$\times$}}
\put(744,589){\makebox(0,0){$\times$}}
\put(745,589){\makebox(0,0){$\times$}}
\put(745,590){\makebox(0,0){$\times$}}
\put(746,591){\makebox(0,0){$\times$}}
\put(746,591){\makebox(0,0){$\times$}}
\put(747,591){\makebox(0,0){$\times$}}
\put(747,592){\makebox(0,0){$\times$}}
\put(748,592){\makebox(0,0){$\times$}}
\put(748,592){\makebox(0,0){$\times$}}
\put(749,593){\makebox(0,0){$\times$}}
\put(749,593){\makebox(0,0){$\times$}}
\put(750,593){\makebox(0,0){$\times$}}
\put(750,594){\makebox(0,0){$\times$}}
\put(751,594){\makebox(0,0){$\times$}}
\put(751,594){\makebox(0,0){$\times$}}
\put(752,594){\makebox(0,0){$\times$}}
\put(752,594){\makebox(0,0){$\times$}}
\put(753,594){\makebox(0,0){$\times$}}
\put(753,594){\makebox(0,0){$\times$}}
\put(754,594){\makebox(0,0){$\times$}}
\put(754,594){\makebox(0,0){$\times$}}
\put(755,594){\makebox(0,0){$\times$}}
\put(755,594){\makebox(0,0){$\times$}}
\put(756,594){\makebox(0,0){$\times$}}
\put(756,594){\makebox(0,0){$\times$}}
\put(757,593){\makebox(0,0){$\times$}}
\put(757,593){\makebox(0,0){$\times$}}
\put(758,593){\makebox(0,0){$\times$}}
\put(758,592){\makebox(0,0){$\times$}}
\put(759,592){\makebox(0,0){$\times$}}
\put(759,592){\makebox(0,0){$\times$}}
\put(760,591){\makebox(0,0){$\times$}}
\put(760,591){\makebox(0,0){$\times$}}
\put(761,591){\makebox(0,0){$\times$}}
\put(762,590){\makebox(0,0){$\times$}}
\put(762,589){\makebox(0,0){$\times$}}
\put(763,589){\makebox(0,0){$\times$}}
\put(763,588){\makebox(0,0){$\times$}}
\put(764,588){\makebox(0,0){$\times$}}
\put(764,587){\makebox(0,0){$\times$}}
\put(765,586){\makebox(0,0){$\times$}}
\put(765,586){\makebox(0,0){$\times$}}
\put(766,585){\makebox(0,0){$\times$}}
\put(766,584){\makebox(0,0){$\times$}}
\put(767,584){\makebox(0,0){$\times$}}
\put(767,583){\makebox(0,0){$\times$}}
\put(768,582){\makebox(0,0){$\times$}}
\put(768,581){\makebox(0,0){$\times$}}
\put(769,580){\makebox(0,0){$\times$}}
\put(769,579){\makebox(0,0){$\times$}}
\put(770,578){\makebox(0,0){$\times$}}
\put(770,577){\makebox(0,0){$\times$}}
\put(771,576){\makebox(0,0){$\times$}}
\put(771,575){\makebox(0,0){$\times$}}
\put(772,574){\makebox(0,0){$\times$}}
\put(772,573){\makebox(0,0){$\times$}}
\put(773,572){\makebox(0,0){$\times$}}
\put(773,571){\makebox(0,0){$\times$}}
\put(774,570){\makebox(0,0){$\times$}}
\put(774,569){\makebox(0,0){$\times$}}
\put(775,568){\makebox(0,0){$\times$}}
\put(775,566){\makebox(0,0){$\times$}}
\put(776,565){\makebox(0,0){$\times$}}
\put(776,564){\makebox(0,0){$\times$}}
\put(777,563){\makebox(0,0){$\times$}}
\put(777,561){\makebox(0,0){$\times$}}
\put(778,560){\makebox(0,0){$\times$}}
\put(779,559){\makebox(0,0){$\times$}}
\put(779,557){\makebox(0,0){$\times$}}
\put(780,556){\makebox(0,0){$\times$}}
\put(780,555){\makebox(0,0){$\times$}}
\put(781,554){\makebox(0,0){$\times$}}
\put(781,552){\makebox(0,0){$\times$}}
\put(782,550){\makebox(0,0){$\times$}}
\put(782,549){\makebox(0,0){$\times$}}
\put(783,548){\makebox(0,0){$\times$}}
\put(783,547){\makebox(0,0){$\times$}}
\put(784,545){\makebox(0,0){$\times$}}
\put(784,544){\makebox(0,0){$\times$}}
\put(785,542){\makebox(0,0){$\times$}}
\put(785,541){\makebox(0,0){$\times$}}
\put(786,539){\makebox(0,0){$\times$}}
\put(786,538){\makebox(0,0){$\times$}}
\put(787,536){\makebox(0,0){$\times$}}
\put(787,535){\makebox(0,0){$\times$}}
\put(788,533){\makebox(0,0){$\times$}}
\put(788,532){\makebox(0,0){$\times$}}
\put(789,530){\makebox(0,0){$\times$}}
\put(789,529){\makebox(0,0){$\times$}}
\put(790,527){\makebox(0,0){$\times$}}
\put(790,526){\makebox(0,0){$\times$}}
\put(791,524){\makebox(0,0){$\times$}}
\put(791,523){\makebox(0,0){$\times$}}
\put(792,521){\makebox(0,0){$\times$}}
\put(792,519){\makebox(0,0){$\times$}}
\put(793,518){\makebox(0,0){$\times$}}
\put(793,516){\makebox(0,0){$\times$}}
\put(794,515){\makebox(0,0){$\times$}}
\put(794,513){\makebox(0,0){$\times$}}
\put(795,512){\makebox(0,0){$\times$}}
\put(796,510){\makebox(0,0){$\times$}}
\put(796,509){\makebox(0,0){$\times$}}
\put(797,505){\makebox(0,0){$\times$}}
\put(798,504){\makebox(0,0){$\times$}}
\put(798,502){\makebox(0,0){$\times$}}
\put(799,501){\makebox(0,0){$\times$}}
\put(799,500){\makebox(0,0){$\times$}}
\put(800,498){\makebox(0,0){$\times$}}
\put(800,496){\makebox(0,0){$\times$}}
\put(801,495){\makebox(0,0){$\times$}}
\put(801,494){\makebox(0,0){$\times$}}
\put(802,492){\makebox(0,0){$\times$}}
\put(802,491){\makebox(0,0){$\times$}}
\put(803,489){\makebox(0,0){$\times$}}
\put(803,488){\makebox(0,0){$\times$}}
\put(804,486){\makebox(0,0){$\times$}}
\put(804,485){\makebox(0,0){$\times$}}
\put(805,484){\makebox(0,0){$\times$}}
\put(805,482){\makebox(0,0){$\times$}}
\put(806,481){\makebox(0,0){$\times$}}
\put(806,479){\makebox(0,0){$\times$}}
\put(807,478){\makebox(0,0){$\times$}}
\put(807,477){\makebox(0,0){$\times$}}
\put(808,475){\makebox(0,0){$\times$}}
\put(808,474){\makebox(0,0){$\times$}}
\put(809,473){\makebox(0,0){$\times$}}
\put(809,472){\makebox(0,0){$\times$}}
\put(810,470){\makebox(0,0){$\times$}}
\put(810,469){\makebox(0,0){$\times$}}
\put(811,468){\makebox(0,0){$\times$}}
\put(811,467){\makebox(0,0){$\times$}}
\put(812,466){\makebox(0,0){$\times$}}
\put(813,465){\makebox(0,0){$\times$}}
\put(813,463){\makebox(0,0){$\times$}}
\put(814,462){\makebox(0,0){$\times$}}
\put(814,461){\makebox(0,0){$\times$}}
\put(815,460){\makebox(0,0){$\times$}}
\put(815,459){\makebox(0,0){$\times$}}
\put(816,458){\makebox(0,0){$\times$}}
\put(816,457){\makebox(0,0){$\times$}}
\put(817,456){\makebox(0,0){$\times$}}
\put(817,455){\makebox(0,0){$\times$}}
\put(818,454){\makebox(0,0){$\times$}}
\put(818,453){\makebox(0,0){$\times$}}
\put(819,452){\makebox(0,0){$\times$}}
\put(819,451){\makebox(0,0){$\times$}}
\put(820,450){\makebox(0,0){$\times$}}
\put(820,449){\makebox(0,0){$\times$}}
\put(821,449){\makebox(0,0){$\times$}}
\put(821,448){\makebox(0,0){$\times$}}
\put(822,447){\makebox(0,0){$\times$}}
\put(822,446){\makebox(0,0){$\times$}}
\put(823,446){\makebox(0,0){$\times$}}
\put(823,445){\makebox(0,0){$\times$}}
\put(824,444){\makebox(0,0){$\times$}}
\put(824,444){\makebox(0,0){$\times$}}
\put(825,443){\makebox(0,0){$\times$}}
\put(825,442){\makebox(0,0){$\times$}}
\put(826,442){\makebox(0,0){$\times$}}
\put(826,442){\makebox(0,0){$\times$}}
\put(827,441){\makebox(0,0){$\times$}}
\put(827,440){\makebox(0,0){$\times$}}
\put(828,440){\makebox(0,0){$\times$}}
\put(828,440){\makebox(0,0){$\times$}}
\put(829,439){\makebox(0,0){$\times$}}
\put(830,439){\makebox(0,0){$\times$}}
\put(830,439){\makebox(0,0){$\times$}}
\put(831,439){\makebox(0,0){$\times$}}
\put(831,438){\makebox(0,0){$\times$}}
\put(832,438){\makebox(0,0){$\times$}}
\put(832,438){\makebox(0,0){$\times$}}
\put(833,437){\makebox(0,0){$\times$}}
\put(833,437){\makebox(0,0){$\times$}}
\put(834,437){\makebox(0,0){$\times$}}
\put(834,437){\makebox(0,0){$\times$}}
\put(835,437){\makebox(0,0){$\times$}}
\put(835,437){\makebox(0,0){$\times$}}
\put(836,437){\makebox(0,0){$\times$}}
\put(836,437){\makebox(0,0){$\times$}}
\put(837,437){\makebox(0,0){$\times$}}
\put(837,437){\makebox(0,0){$\times$}}
\put(838,437){\makebox(0,0){$\times$}}
\put(838,437){\makebox(0,0){$\times$}}
\put(839,437){\makebox(0,0){$\times$}}
\put(839,437){\makebox(0,0){$\times$}}
\put(840,438){\makebox(0,0){$\times$}}
\put(840,438){\makebox(0,0){$\times$}}
\put(841,438){\makebox(0,0){$\times$}}
\put(841,438){\makebox(0,0){$\times$}}
\put(842,439){\makebox(0,0){$\times$}}
\put(842,439){\makebox(0,0){$\times$}}
\put(843,439){\makebox(0,0){$\times$}}
\put(843,440){\makebox(0,0){$\times$}}
\put(844,440){\makebox(0,0){$\times$}}
\put(844,440){\makebox(0,0){$\times$}}
\put(845,441){\makebox(0,0){$\times$}}
\put(845,441){\makebox(0,0){$\times$}}
\put(846,442){\makebox(0,0){$\times$}}
\put(847,442){\makebox(0,0){$\times$}}
\put(847,443){\makebox(0,0){$\times$}}
\put(848,443){\makebox(0,0){$\times$}}
\put(848,444){\makebox(0,0){$\times$}}
\put(849,444){\makebox(0,0){$\times$}}
\put(849,445){\makebox(0,0){$\times$}}
\put(850,446){\makebox(0,0){$\times$}}
\put(850,446){\makebox(0,0){$\times$}}
\put(851,447){\makebox(0,0){$\times$}}
\put(851,447){\makebox(0,0){$\times$}}
\put(852,448){\makebox(0,0){$\times$}}
\put(852,449){\makebox(0,0){$\times$}}
\put(853,450){\makebox(0,0){$\times$}}
\put(853,451){\makebox(0,0){$\times$}}
\put(854,451){\makebox(0,0){$\times$}}
\put(854,452){\makebox(0,0){$\times$}}
\put(855,453){\makebox(0,0){$\times$}}
\put(855,454){\makebox(0,0){$\times$}}
\put(856,454){\makebox(0,0){$\times$}}
\put(856,455){\makebox(0,0){$\times$}}
\put(857,456){\makebox(0,0){$\times$}}
\put(857,457){\makebox(0,0){$\times$}}
\put(858,458){\makebox(0,0){$\times$}}
\put(858,459){\makebox(0,0){$\times$}}
\put(859,460){\makebox(0,0){$\times$}}
\put(859,461){\makebox(0,0){$\times$}}
\put(860,461){\makebox(0,0){$\times$}}
\put(860,463){\makebox(0,0){$\times$}}
\put(861,463){\makebox(0,0){$\times$}}
\put(861,465){\makebox(0,0){$\times$}}
\put(862,466){\makebox(0,0){$\times$}}
\put(862,467){\makebox(0,0){$\times$}}
\put(863,468){\makebox(0,0){$\times$}}
\put(864,469){\makebox(0,0){$\times$}}
\put(864,470){\makebox(0,0){$\times$}}
\put(865,471){\makebox(0,0){$\times$}}
\put(865,472){\makebox(0,0){$\times$}}
\put(866,473){\makebox(0,0){$\times$}}
\put(866,474){\makebox(0,0){$\times$}}
\put(867,475){\makebox(0,0){$\times$}}
\put(867,476){\makebox(0,0){$\times$}}
\put(868,477){\makebox(0,0){$\times$}}
\put(868,479){\makebox(0,0){$\times$}}
\put(869,480){\makebox(0,0){$\times$}}
\put(869,481){\makebox(0,0){$\times$}}
\put(870,482){\makebox(0,0){$\times$}}
\put(870,483){\makebox(0,0){$\times$}}
\put(871,484){\makebox(0,0){$\times$}}
\put(871,485){\makebox(0,0){$\times$}}
\put(872,486){\makebox(0,0){$\times$}}
\put(872,488){\makebox(0,0){$\times$}}
\put(873,489){\makebox(0,0){$\times$}}
\put(873,490){\makebox(0,0){$\times$}}
\put(874,491){\makebox(0,0){$\times$}}
\put(874,492){\makebox(0,0){$\times$}}
\put(875,493){\makebox(0,0){$\times$}}
\put(875,495){\makebox(0,0){$\times$}}
\put(876,496){\makebox(0,0){$\times$}}
\put(876,497){\makebox(0,0){$\times$}}
\put(877,498){\makebox(0,0){$\times$}}
\put(877,499){\makebox(0,0){$\times$}}
\put(878,500){\makebox(0,0){$\times$}}
\put(878,502){\makebox(0,0){$\times$}}
\put(879,503){\makebox(0,0){$\times$}}
\put(879,504){\makebox(0,0){$\times$}}
\put(880,505){\makebox(0,0){$\times$}}
\put(881,506){\makebox(0,0){$\times$}}
\put(882,509){\makebox(0,0){$\times$}}
\put(882,510){\makebox(0,0){$\times$}}
\put(883,511){\makebox(0,0){$\times$}}
\put(883,512){\makebox(0,0){$\times$}}
\put(884,513){\makebox(0,0){$\times$}}
\put(884,514){\makebox(0,0){$\times$}}
\put(885,516){\makebox(0,0){$\times$}}
\put(885,517){\makebox(0,0){$\times$}}
\put(886,518){\makebox(0,0){$\times$}}
\put(886,519){\makebox(0,0){$\times$}}
\put(887,520){\makebox(0,0){$\times$}}
\put(887,521){\makebox(0,0){$\times$}}
\put(888,522){\makebox(0,0){$\times$}}
\put(888,523){\makebox(0,0){$\times$}}
\put(889,524){\makebox(0,0){$\times$}}
\put(889,525){\makebox(0,0){$\times$}}
\put(890,526){\makebox(0,0){$\times$}}
\put(890,527){\makebox(0,0){$\times$}}
\put(891,528){\makebox(0,0){$\times$}}
\put(891,530){\makebox(0,0){$\times$}}
\put(892,530){\makebox(0,0){$\times$}}
\put(892,531){\makebox(0,0){$\times$}}
\put(893,532){\makebox(0,0){$\times$}}
\put(893,533){\makebox(0,0){$\times$}}
\put(894,534){\makebox(0,0){$\times$}}
\put(894,535){\makebox(0,0){$\times$}}
\put(895,536){\makebox(0,0){$\times$}}
\put(895,537){\makebox(0,0){$\times$}}
\put(896,538){\makebox(0,0){$\times$}}
\put(896,539){\makebox(0,0){$\times$}}
\put(897,540){\makebox(0,0){$\times$}}
\put(898,540){\makebox(0,0){$\times$}}
\put(898,541){\makebox(0,0){$\times$}}
\put(899,542){\makebox(0,0){$\times$}}
\put(899,543){\makebox(0,0){$\times$}}
\put(900,544){\makebox(0,0){$\times$}}
\put(900,544){\makebox(0,0){$\times$}}
\put(901,545){\makebox(0,0){$\times$}}
\put(901,546){\makebox(0,0){$\times$}}
\put(902,547){\makebox(0,0){$\times$}}
\put(902,547){\makebox(0,0){$\times$}}
\put(903,548){\makebox(0,0){$\times$}}
\put(903,549){\makebox(0,0){$\times$}}
\put(904,549){\makebox(0,0){$\times$}}
\put(904,550){\makebox(0,0){$\times$}}
\put(905,550){\makebox(0,0){$\times$}}
\put(905,551){\makebox(0,0){$\times$}}
\put(906,552){\makebox(0,0){$\times$}}
\put(906,552){\makebox(0,0){$\times$}}
\put(907,553){\makebox(0,0){$\times$}}
\put(907,553){\makebox(0,0){$\times$}}
\put(908,554){\makebox(0,0){$\times$}}
\put(908,554){\makebox(0,0){$\times$}}
\put(909,555){\makebox(0,0){$\times$}}
\put(909,555){\makebox(0,0){$\times$}}
\put(910,556){\makebox(0,0){$\times$}}
\put(910,556){\makebox(0,0){$\times$}}
\put(911,556){\makebox(0,0){$\times$}}
\put(911,557){\makebox(0,0){$\times$}}
\put(912,557){\makebox(0,0){$\times$}}
\put(912,557){\makebox(0,0){$\times$}}
\put(913,557){\makebox(0,0){$\times$}}
\put(913,558){\makebox(0,0){$\times$}}
\put(914,558){\makebox(0,0){$\times$}}
\put(915,558){\makebox(0,0){$\times$}}
\put(915,559){\makebox(0,0){$\times$}}
\put(916,559){\makebox(0,0){$\times$}}
\put(916,559){\makebox(0,0){$\times$}}
\put(917,559){\makebox(0,0){$\times$}}
\put(917,559){\makebox(0,0){$\times$}}
\put(918,559){\makebox(0,0){$\times$}}
\put(918,559){\makebox(0,0){$\times$}}
\put(919,559){\makebox(0,0){$\times$}}
\put(919,559){\makebox(0,0){$\times$}}
\put(920,559){\makebox(0,0){$\times$}}
\put(920,559){\makebox(0,0){$\times$}}
\put(921,559){\makebox(0,0){$\times$}}
\put(921,559){\makebox(0,0){$\times$}}
\put(922,559){\makebox(0,0){$\times$}}
\put(922,559){\makebox(0,0){$\times$}}
\put(923,559){\makebox(0,0){$\times$}}
\put(923,559){\makebox(0,0){$\times$}}
\put(924,559){\makebox(0,0){$\times$}}
\put(924,559){\makebox(0,0){$\times$}}
\put(925,559){\makebox(0,0){$\times$}}
\put(925,559){\makebox(0,0){$\times$}}
\put(926,558){\makebox(0,0){$\times$}}
\put(926,558){\makebox(0,0){$\times$}}
\put(927,558){\makebox(0,0){$\times$}}
\put(927,557){\makebox(0,0){$\times$}}
\put(928,557){\makebox(0,0){$\times$}}
\put(928,557){\makebox(0,0){$\times$}}
\put(929,557){\makebox(0,0){$\times$}}
\put(929,556){\makebox(0,0){$\times$}}
\put(930,556){\makebox(0,0){$\times$}}
\put(930,556){\makebox(0,0){$\times$}}
\put(931,555){\makebox(0,0){$\times$}}
\put(932,555){\makebox(0,0){$\times$}}
\put(932,554){\makebox(0,0){$\times$}}
\put(933,554){\makebox(0,0){$\times$}}
\put(933,554){\makebox(0,0){$\times$}}
\put(934,553){\makebox(0,0){$\times$}}
\put(934,553){\makebox(0,0){$\times$}}
\put(935,552){\makebox(0,0){$\times$}}
\put(935,552){\makebox(0,0){$\times$}}
\put(936,551){\makebox(0,0){$\times$}}
\put(936,550){\makebox(0,0){$\times$}}
\put(937,550){\makebox(0,0){$\times$}}
\put(937,549){\makebox(0,0){$\times$}}
\put(938,549){\makebox(0,0){$\times$}}
\put(938,549){\makebox(0,0){$\times$}}
\put(939,548){\makebox(0,0){$\times$}}
\put(939,547){\makebox(0,0){$\times$}}
\put(940,547){\makebox(0,0){$\times$}}
\put(940,546){\makebox(0,0){$\times$}}
\put(941,545){\makebox(0,0){$\times$}}
\put(941,545){\makebox(0,0){$\times$}}
\put(942,544){\makebox(0,0){$\times$}}
\put(942,543){\makebox(0,0){$\times$}}
\put(943,543){\makebox(0,0){$\times$}}
\put(943,542){\makebox(0,0){$\times$}}
\put(944,541){\makebox(0,0){$\times$}}
\put(944,540){\makebox(0,0){$\times$}}
\put(945,540){\makebox(0,0){$\times$}}
\put(945,539){\makebox(0,0){$\times$}}
\put(946,538){\makebox(0,0){$\times$}}
\put(946,538){\makebox(0,0){$\times$}}
\put(947,537){\makebox(0,0){$\times$}}
\put(947,536){\makebox(0,0){$\times$}}
\put(948,535){\makebox(0,0){$\times$}}
\put(949,535){\makebox(0,0){$\times$}}
\put(949,533){\makebox(0,0){$\times$}}
\put(950,533){\makebox(0,0){$\times$}}
\put(950,532){\makebox(0,0){$\times$}}
\put(951,531){\makebox(0,0){$\times$}}
\put(951,530){\makebox(0,0){$\times$}}
\put(952,530){\makebox(0,0){$\times$}}
\put(952,529){\makebox(0,0){$\times$}}
\put(953,528){\makebox(0,0){$\times$}}
\put(953,527){\makebox(0,0){$\times$}}
\put(954,526){\makebox(0,0){$\times$}}
\put(954,525){\makebox(0,0){$\times$}}
\put(955,524){\makebox(0,0){$\times$}}
\put(955,524){\makebox(0,0){$\times$}}
\put(956,523){\makebox(0,0){$\times$}}
\put(956,522){\makebox(0,0){$\times$}}
\put(957,521){\makebox(0,0){$\times$}}
\put(957,520){\makebox(0,0){$\times$}}
\put(958,519){\makebox(0,0){$\times$}}
\put(958,518){\makebox(0,0){$\times$}}
\put(959,517){\makebox(0,0){$\times$}}
\put(959,517){\makebox(0,0){$\times$}}
\put(960,516){\makebox(0,0){$\times$}}
\put(960,515){\makebox(0,0){$\times$}}
\put(961,514){\makebox(0,0){$\times$}}
\put(961,513){\makebox(0,0){$\times$}}
\put(962,512){\makebox(0,0){$\times$}}
\put(962,511){\makebox(0,0){$\times$}}
\put(963,510){\makebox(0,0){$\times$}}
\put(964,509){\makebox(0,0){$\times$}}
\put(965,508){\makebox(0,0){$\times$}}
\put(966,506){\makebox(0,0){$\times$}}
\put(966,505){\makebox(0,0){$\times$}}
\put(967,504){\makebox(0,0){$\times$}}
\put(967,503){\makebox(0,0){$\times$}}
\put(968,503){\makebox(0,0){$\times$}}
\put(968,502){\makebox(0,0){$\times$}}
\put(969,501){\makebox(0,0){$\times$}}
\put(969,500){\makebox(0,0){$\times$}}
\put(970,499){\makebox(0,0){$\times$}}
\put(970,498){\makebox(0,0){$\times$}}
\put(971,498){\makebox(0,0){$\times$}}
\put(971,497){\makebox(0,0){$\times$}}
\put(972,496){\makebox(0,0){$\times$}}
\put(972,495){\makebox(0,0){$\times$}}
\put(973,494){\makebox(0,0){$\times$}}
\put(973,494){\makebox(0,0){$\times$}}
\put(974,493){\makebox(0,0){$\times$}}
\put(974,492){\makebox(0,0){$\times$}}
\put(975,491){\makebox(0,0){$\times$}}
\put(975,491){\makebox(0,0){$\times$}}
\put(976,490){\makebox(0,0){$\times$}}
\put(976,489){\makebox(0,0){$\times$}}
\put(977,488){\makebox(0,0){$\times$}}
\put(977,488){\makebox(0,0){$\times$}}
\put(978,487){\makebox(0,0){$\times$}}
\put(978,486){\makebox(0,0){$\times$}}
\put(979,485){\makebox(0,0){$\times$}}
\put(979,485){\makebox(0,0){$\times$}}
\put(980,484){\makebox(0,0){$\times$}}
\put(980,483){\makebox(0,0){$\times$}}
\put(981,483){\makebox(0,0){$\times$}}
\put(982,482){\makebox(0,0){$\times$}}
\put(982,481){\makebox(0,0){$\times$}}
\put(983,481){\makebox(0,0){$\times$}}
\put(983,480){\makebox(0,0){$\times$}}
\put(984,480){\makebox(0,0){$\times$}}
\put(984,479){\makebox(0,0){$\times$}}
\put(985,479){\makebox(0,0){$\times$}}
\put(985,478){\makebox(0,0){$\times$}}
\put(986,477){\makebox(0,0){$\times$}}
\put(986,477){\makebox(0,0){$\times$}}
\put(987,476){\makebox(0,0){$\times$}}
\put(987,476){\makebox(0,0){$\times$}}
\put(988,475){\makebox(0,0){$\times$}}
\put(988,475){\makebox(0,0){$\times$}}
\put(989,474){\makebox(0,0){$\times$}}
\put(989,474){\makebox(0,0){$\times$}}
\put(990,474){\makebox(0,0){$\times$}}
\put(990,473){\makebox(0,0){$\times$}}
\put(991,473){\makebox(0,0){$\times$}}
\put(991,472){\makebox(0,0){$\times$}}
\put(992,472){\makebox(0,0){$\times$}}
\put(992,472){\makebox(0,0){$\times$}}
\put(993,471){\makebox(0,0){$\times$}}
\put(993,471){\makebox(0,0){$\times$}}
\put(994,470){\makebox(0,0){$\times$}}
\put(994,470){\makebox(0,0){$\times$}}
\put(995,470){\makebox(0,0){$\times$}}
\put(995,470){\makebox(0,0){$\times$}}
\put(996,469){\makebox(0,0){$\times$}}
\put(996,469){\makebox(0,0){$\times$}}
\put(997,469){\makebox(0,0){$\times$}}
\put(997,469){\makebox(0,0){$\times$}}
\put(998,468){\makebox(0,0){$\times$}}
\put(999,468){\makebox(0,0){$\times$}}
\put(999,468){\makebox(0,0){$\times$}}
\put(1000,468){\makebox(0,0){$\times$}}
\put(1000,468){\makebox(0,0){$\times$}}
\put(1001,468){\makebox(0,0){$\times$}}
\put(1001,468){\makebox(0,0){$\times$}}
\put(1002,468){\makebox(0,0){$\times$}}
\put(1002,468){\makebox(0,0){$\times$}}
\put(1003,468){\makebox(0,0){$\times$}}
\put(1003,468){\makebox(0,0){$\times$}}
\put(1004,467){\makebox(0,0){$\times$}}
\put(1004,467){\makebox(0,0){$\times$}}
\put(1005,467){\makebox(0,0){$\times$}}
\put(1005,467){\makebox(0,0){$\times$}}
\put(1006,467){\makebox(0,0){$\times$}}
\put(1006,467){\makebox(0,0){$\times$}}
\put(1007,467){\makebox(0,0){$\times$}}
\put(1007,467){\makebox(0,0){$\times$}}
\put(1008,468){\makebox(0,0){$\times$}}
\put(1008,468){\makebox(0,0){$\times$}}
\put(1009,468){\makebox(0,0){$\times$}}
\put(1009,468){\makebox(0,0){$\times$}}
\put(1010,468){\makebox(0,0){$\times$}}
\put(1010,468){\makebox(0,0){$\times$}}
\put(1011,468){\makebox(0,0){$\times$}}
\put(1011,468){\makebox(0,0){$\times$}}
\put(1012,469){\makebox(0,0){$\times$}}
\put(1012,469){\makebox(0,0){$\times$}}
\put(1013,469){\makebox(0,0){$\times$}}
\put(1013,470){\makebox(0,0){$\times$}}
\put(1014,470){\makebox(0,0){$\times$}}
\put(1014,470){\makebox(0,0){$\times$}}
\put(1015,470){\makebox(0,0){$\times$}}
\put(1016,471){\makebox(0,0){$\times$}}
\put(1016,471){\makebox(0,0){$\times$}}
\put(1017,471){\makebox(0,0){$\times$}}
\put(1017,472){\makebox(0,0){$\times$}}
\put(1018,472){\makebox(0,0){$\times$}}
\put(1018,472){\makebox(0,0){$\times$}}
\put(1019,473){\makebox(0,0){$\times$}}
\put(1019,473){\makebox(0,0){$\times$}}
\put(1020,474){\makebox(0,0){$\times$}}
\put(1020,474){\makebox(0,0){$\times$}}
\put(1021,474){\makebox(0,0){$\times$}}
\put(1021,475){\makebox(0,0){$\times$}}
\put(1022,475){\makebox(0,0){$\times$}}
\put(1022,475){\makebox(0,0){$\times$}}
\put(1023,476){\makebox(0,0){$\times$}}
\put(1023,477){\makebox(0,0){$\times$}}
\put(1024,477){\makebox(0,0){$\times$}}
\put(1024,477){\makebox(0,0){$\times$}}
\put(1025,478){\makebox(0,0){$\times$}}
\put(1025,479){\makebox(0,0){$\times$}}
\put(1026,479){\makebox(0,0){$\times$}}
\put(1026,480){\makebox(0,0){$\times$}}
\put(1027,480){\makebox(0,0){$\times$}}
\put(1027,481){\makebox(0,0){$\times$}}
\put(1028,481){\makebox(0,0){$\times$}}
\put(1028,482){\makebox(0,0){$\times$}}
\put(1029,482){\makebox(0,0){$\times$}}
\put(1029,483){\makebox(0,0){$\times$}}
\put(1030,484){\makebox(0,0){$\times$}}
\put(1030,484){\makebox(0,0){$\times$}}
\put(1031,485){\makebox(0,0){$\times$}}
\put(1031,486){\makebox(0,0){$\times$}}
\put(1032,486){\makebox(0,0){$\times$}}
\put(1033,487){\makebox(0,0){$\times$}}
\put(1033,488){\makebox(0,0){$\times$}}
\put(1034,488){\makebox(0,0){$\times$}}
\put(1034,489){\makebox(0,0){$\times$}}
\put(1035,489){\makebox(0,0){$\times$}}
\put(1035,490){\makebox(0,0){$\times$}}
\put(1036,491){\makebox(0,0){$\times$}}
\put(1036,491){\makebox(0,0){$\times$}}
\put(1037,492){\makebox(0,0){$\times$}}
\put(1037,493){\makebox(0,0){$\times$}}
\put(1038,493){\makebox(0,0){$\times$}}
\put(1038,494){\makebox(0,0){$\times$}}
\put(1039,495){\makebox(0,0){$\times$}}
\put(1039,495){\makebox(0,0){$\times$}}
\put(1040,496){\makebox(0,0){$\times$}}
\put(1040,497){\makebox(0,0){$\times$}}
\put(1041,498){\makebox(0,0){$\times$}}
\put(1041,498){\makebox(0,0){$\times$}}
\put(1042,499){\makebox(0,0){$\times$}}
\put(1042,500){\makebox(0,0){$\times$}}
\put(1043,500){\makebox(0,0){$\times$}}
\put(1043,501){\makebox(0,0){$\times$}}
\put(1044,502){\makebox(0,0){$\times$}}
\put(1044,502){\makebox(0,0){$\times$}}
\put(1045,503){\makebox(0,0){$\times$}}
\put(1045,504){\makebox(0,0){$\times$}}
\put(1046,505){\makebox(0,0){$\times$}}
\put(1046,505){\makebox(0,0){$\times$}}
\put(1047,506){\makebox(0,0){$\times$}}
\put(1047,507){\makebox(0,0){$\times$}}
\put(1048,508){\makebox(0,0){$\times$}}
\put(1049,509){\makebox(0,0){$\times$}}
\put(1050,510){\makebox(0,0){$\times$}}
\put(1050,510){\makebox(0,0){$\times$}}
\put(1051,511){\makebox(0,0){$\times$}}
\put(1051,512){\makebox(0,0){$\times$}}
\put(1052,512){\makebox(0,0){$\times$}}
\put(1052,513){\makebox(0,0){$\times$}}
\put(1053,514){\makebox(0,0){$\times$}}
\put(1053,514){\makebox(0,0){$\times$}}
\put(1054,515){\makebox(0,0){$\times$}}
\put(1054,516){\makebox(0,0){$\times$}}
\put(1055,516){\makebox(0,0){$\times$}}
\put(1055,517){\makebox(0,0){$\times$}}
\put(1056,517){\makebox(0,0){$\times$}}
\put(1056,518){\makebox(0,0){$\times$}}
\put(1057,519){\makebox(0,0){$\times$}}
\put(1057,519){\makebox(0,0){$\times$}}
\put(1058,520){\makebox(0,0){$\times$}}
\put(1058,521){\makebox(0,0){$\times$}}
\put(1059,521){\makebox(0,0){$\times$}}
\put(1059,522){\makebox(0,0){$\times$}}
\put(1060,522){\makebox(0,0){$\times$}}
\put(1060,523){\makebox(0,0){$\times$}}
\put(1061,523){\makebox(0,0){$\times$}}
\put(1061,524){\makebox(0,0){$\times$}}
\put(1062,524){\makebox(0,0){$\times$}}
\put(1062,525){\makebox(0,0){$\times$}}
\put(1063,526){\makebox(0,0){$\times$}}
\put(1063,526){\makebox(0,0){$\times$}}
\put(1064,526){\makebox(0,0){$\times$}}
\put(1064,527){\makebox(0,0){$\times$}}
\put(1065,528){\makebox(0,0){$\times$}}
\put(1065,528){\makebox(0,0){$\times$}}
\put(1066,528){\makebox(0,0){$\times$}}
\put(1067,529){\makebox(0,0){$\times$}}
\put(1067,529){\makebox(0,0){$\times$}}
\put(1068,530){\makebox(0,0){$\times$}}
\put(1068,530){\makebox(0,0){$\times$}}
\put(1069,530){\makebox(0,0){$\times$}}
\put(1069,531){\makebox(0,0){$\times$}}
\put(1070,531){\makebox(0,0){$\times$}}
\put(1070,531){\makebox(0,0){$\times$}}
\put(1071,532){\makebox(0,0){$\times$}}
\put(1071,532){\makebox(0,0){$\times$}}
\put(1072,533){\makebox(0,0){$\times$}}
\put(1072,533){\makebox(0,0){$\times$}}
\put(1073,533){\makebox(0,0){$\times$}}
\put(1073,533){\makebox(0,0){$\times$}}
\put(1074,533){\makebox(0,0){$\times$}}
\put(1074,534){\makebox(0,0){$\times$}}
\put(1075,534){\makebox(0,0){$\times$}}
\put(1075,535){\makebox(0,0){$\times$}}
\put(1076,535){\makebox(0,0){$\times$}}
\put(1076,535){\makebox(0,0){$\times$}}
\put(1077,535){\makebox(0,0){$\times$}}
\put(1077,535){\makebox(0,0){$\times$}}
\put(1078,535){\makebox(0,0){$\times$}}
\put(1078,536){\makebox(0,0){$\times$}}
\put(1079,536){\makebox(0,0){$\times$}}
\put(1079,536){\makebox(0,0){$\times$}}
\put(1080,536){\makebox(0,0){$\times$}}
\put(1080,536){\makebox(0,0){$\times$}}
\put(1081,536){\makebox(0,0){$\times$}}
\put(1081,537){\makebox(0,0){$\times$}}
\put(1082,537){\makebox(0,0){$\times$}}
\put(1082,537){\makebox(0,0){$\times$}}
\put(1083,537){\makebox(0,0){$\times$}}
\put(1084,537){\makebox(0,0){$\times$}}
\put(1084,537){\makebox(0,0){$\times$}}
\put(1085,537){\makebox(0,0){$\times$}}
\put(1085,537){\makebox(0,0){$\times$}}
\put(1086,537){\makebox(0,0){$\times$}}
\put(1086,537){\makebox(0,0){$\times$}}
\put(1087,537){\makebox(0,0){$\times$}}
\put(1087,537){\makebox(0,0){$\times$}}
\put(1088,537){\makebox(0,0){$\times$}}
\put(1088,537){\makebox(0,0){$\times$}}
\put(1089,537){\makebox(0,0){$\times$}}
\put(1089,537){\makebox(0,0){$\times$}}
\put(1090,537){\makebox(0,0){$\times$}}
\put(1090,537){\makebox(0,0){$\times$}}
\put(1091,537){\makebox(0,0){$\times$}}
\put(1091,537){\makebox(0,0){$\times$}}
\put(1092,536){\makebox(0,0){$\times$}}
\put(1092,536){\makebox(0,0){$\times$}}
\put(1093,536){\makebox(0,0){$\times$}}
\put(1093,536){\makebox(0,0){$\times$}}
\put(1094,536){\makebox(0,0){$\times$}}
\put(1094,536){\makebox(0,0){$\times$}}
\put(1095,535){\makebox(0,0){$\times$}}
\put(1095,535){\makebox(0,0){$\times$}}
\put(1096,535){\makebox(0,0){$\times$}}
\put(1096,535){\makebox(0,0){$\times$}}
\put(1097,535){\makebox(0,0){$\times$}}
\put(1097,534){\makebox(0,0){$\times$}}
\put(1098,534){\makebox(0,0){$\times$}}
\put(1098,534){\makebox(0,0){$\times$}}
\put(1099,533){\makebox(0,0){$\times$}}
\put(1099,533){\makebox(0,0){$\times$}}
\put(1100,533){\makebox(0,0){$\times$}}
\put(1101,533){\makebox(0,0){$\times$}}
\put(1101,532){\makebox(0,0){$\times$}}
\put(1102,532){\makebox(0,0){$\times$}}
\put(1102,532){\makebox(0,0){$\times$}}
\put(1103,531){\makebox(0,0){$\times$}}
\put(1103,531){\makebox(0,0){$\times$}}
\put(1104,531){\makebox(0,0){$\times$}}
\put(1104,530){\makebox(0,0){$\times$}}
\put(1105,530){\makebox(0,0){$\times$}}
\put(1105,530){\makebox(0,0){$\times$}}
\put(1106,530){\makebox(0,0){$\times$}}
\put(1106,529){\makebox(0,0){$\times$}}
\put(1107,529){\makebox(0,0){$\times$}}
\put(1107,528){\makebox(0,0){$\times$}}
\put(1108,528){\makebox(0,0){$\times$}}
\put(1108,528){\makebox(0,0){$\times$}}
\put(1109,527){\makebox(0,0){$\times$}}
\put(1109,527){\makebox(0,0){$\times$}}
\put(1110,526){\makebox(0,0){$\times$}}
\put(1110,526){\makebox(0,0){$\times$}}
\put(1111,526){\makebox(0,0){$\times$}}
\put(1111,525){\makebox(0,0){$\times$}}
\put(1112,524){\makebox(0,0){$\times$}}
\put(1112,524){\makebox(0,0){$\times$}}
\put(1113,524){\makebox(0,0){$\times$}}
\put(1113,523){\makebox(0,0){$\times$}}
\put(1114,523){\makebox(0,0){$\times$}}
\put(1114,523){\makebox(0,0){$\times$}}
\put(1115,522){\makebox(0,0){$\times$}}
\put(1115,521){\makebox(0,0){$\times$}}
\put(1116,521){\makebox(0,0){$\times$}}
\put(1116,521){\makebox(0,0){$\times$}}
\put(1117,520){\makebox(0,0){$\times$}}
\put(1118,519){\makebox(0,0){$\times$}}
\put(1118,519){\makebox(0,0){$\times$}}
\put(1119,519){\makebox(0,0){$\times$}}
\put(1119,518){\makebox(0,0){$\times$}}
\put(1120,518){\makebox(0,0){$\times$}}
\put(1120,517){\makebox(0,0){$\times$}}
\put(1121,517){\makebox(0,0){$\times$}}
\put(1121,516){\makebox(0,0){$\times$}}
\put(1122,516){\makebox(0,0){$\times$}}
\put(1122,515){\makebox(0,0){$\times$}}
\put(1123,515){\makebox(0,0){$\times$}}
\put(1123,514){\makebox(0,0){$\times$}}
\put(1124,514){\makebox(0,0){$\times$}}
\put(1124,513){\makebox(0,0){$\times$}}
\put(1125,513){\makebox(0,0){$\times$}}
\put(1125,512){\makebox(0,0){$\times$}}
\put(1126,512){\makebox(0,0){$\times$}}
\put(1126,511){\makebox(0,0){$\times$}}
\put(1127,511){\makebox(0,0){$\times$}}
\put(1127,510){\makebox(0,0){$\times$}}
\put(1128,510){\makebox(0,0){$\times$}}
\put(1128,509){\makebox(0,0){$\times$}}
\put(1129,509){\makebox(0,0){$\times$}}
\put(1129,509){\makebox(0,0){$\times$}}
\put(1130,508){\makebox(0,0){$\times$}}
\put(1131,507){\makebox(0,0){$\times$}}
\put(1132,505){\makebox(0,0){$\times$}}
\put(1133,505){\makebox(0,0){$\times$}}
\put(1133,505){\makebox(0,0){$\times$}}
\put(1134,504){\makebox(0,0){$\times$}}
\put(1135,503){\makebox(0,0){$\times$}}
\put(1135,503){\makebox(0,0){$\times$}}
\put(1136,503){\makebox(0,0){$\times$}}
\put(1136,502){\makebox(0,0){$\times$}}
\put(1137,502){\makebox(0,0){$\times$}}
\put(1137,502){\makebox(0,0){$\times$}}
\put(1138,501){\makebox(0,0){$\times$}}
\put(1138,500){\makebox(0,0){$\times$}}
\put(1139,500){\makebox(0,0){$\times$}}
\put(1139,500){\makebox(0,0){$\times$}}
\put(1140,499){\makebox(0,0){$\times$}}
\put(1140,499){\makebox(0,0){$\times$}}
\put(1141,498){\makebox(0,0){$\times$}}
\put(1141,498){\makebox(0,0){$\times$}}
\put(1142,498){\makebox(0,0){$\times$}}
\put(1142,497){\makebox(0,0){$\times$}}
\put(1143,497){\makebox(0,0){$\times$}}
\put(1143,496){\makebox(0,0){$\times$}}
\put(1144,496){\makebox(0,0){$\times$}}
\put(1144,496){\makebox(0,0){$\times$}}
\put(1145,495){\makebox(0,0){$\times$}}
\put(1145,495){\makebox(0,0){$\times$}}
\put(1146,495){\makebox(0,0){$\times$}}
\put(1146,494){\makebox(0,0){$\times$}}
\put(1147,494){\makebox(0,0){$\times$}}
\put(1147,494){\makebox(0,0){$\times$}}
\put(1148,493){\makebox(0,0){$\times$}}
\put(1148,493){\makebox(0,0){$\times$}}
\put(1149,493){\makebox(0,0){$\times$}}
\put(1149,493){\makebox(0,0){$\times$}}
\put(1150,492){\makebox(0,0){$\times$}}
\put(1150,492){\makebox(0,0){$\times$}}
\put(1151,491){\makebox(0,0){$\times$}}
\put(1152,491){\makebox(0,0){$\times$}}
\put(1152,491){\makebox(0,0){$\times$}}
\put(1153,491){\makebox(0,0){$\times$}}
\put(1153,491){\makebox(0,0){$\times$}}
\put(1154,490){\makebox(0,0){$\times$}}
\put(1154,490){\makebox(0,0){$\times$}}
\put(1155,490){\makebox(0,0){$\times$}}
\put(1155,489){\makebox(0,0){$\times$}}
\put(1156,489){\makebox(0,0){$\times$}}
\put(1156,489){\makebox(0,0){$\times$}}
\put(1157,489){\makebox(0,0){$\times$}}
\put(1157,489){\makebox(0,0){$\times$}}
\put(1158,489){\makebox(0,0){$\times$}}
\put(1158,489){\makebox(0,0){$\times$}}
\put(1159,488){\makebox(0,0){$\times$}}
\put(1159,488){\makebox(0,0){$\times$}}
\put(1160,488){\makebox(0,0){$\times$}}
\put(1160,488){\makebox(0,0){$\times$}}
\put(1161,488){\makebox(0,0){$\times$}}
\put(1161,488){\makebox(0,0){$\times$}}
\put(1162,488){\makebox(0,0){$\times$}}
\put(1162,488){\makebox(0,0){$\times$}}
\put(1163,488){\makebox(0,0){$\times$}}
\put(1163,488){\makebox(0,0){$\times$}}
\put(1164,488){\makebox(0,0){$\times$}}
\put(1164,488){\makebox(0,0){$\times$}}
\put(1165,488){\makebox(0,0){$\times$}}
\put(1165,488){\makebox(0,0){$\times$}}
\put(1166,488){\makebox(0,0){$\times$}}
\put(1166,488){\makebox(0,0){$\times$}}
\put(1167,488){\makebox(0,0){$\times$}}
\put(1167,488){\makebox(0,0){$\times$}}
\put(1168,488){\makebox(0,0){$\times$}}
\put(1169,488){\makebox(0,0){$\times$}}
\put(1169,488){\makebox(0,0){$\times$}}
\put(1170,488){\makebox(0,0){$\times$}}
\put(1170,488){\makebox(0,0){$\times$}}
\put(1171,488){\makebox(0,0){$\times$}}
\put(1171,488){\makebox(0,0){$\times$}}
\put(1172,488){\makebox(0,0){$\times$}}
\put(1172,488){\makebox(0,0){$\times$}}
\put(1173,488){\makebox(0,0){$\times$}}
\put(1173,488){\makebox(0,0){$\times$}}
\put(1174,488){\makebox(0,0){$\times$}}
\put(1174,488){\makebox(0,0){$\times$}}
\put(1175,488){\makebox(0,0){$\times$}}
\put(1175,488){\makebox(0,0){$\times$}}
\put(1176,488){\makebox(0,0){$\times$}}
\put(1176,488){\makebox(0,0){$\times$}}
\put(1177,488){\makebox(0,0){$\times$}}
\put(1177,489){\makebox(0,0){$\times$}}
\put(1178,489){\makebox(0,0){$\times$}}
\put(1178,489){\makebox(0,0){$\times$}}
\put(1179,489){\makebox(0,0){$\times$}}
\put(1179,489){\makebox(0,0){$\times$}}
\put(1180,489){\makebox(0,0){$\times$}}
\put(1180,489){\makebox(0,0){$\times$}}
\put(1181,490){\makebox(0,0){$\times$}}
\put(1181,490){\makebox(0,0){$\times$}}
\put(1182,490){\makebox(0,0){$\times$}}
\put(1182,490){\makebox(0,0){$\times$}}
\put(1183,491){\makebox(0,0){$\times$}}
\put(1183,491){\makebox(0,0){$\times$}}
\put(1184,491){\makebox(0,0){$\times$}}
\put(1184,491){\makebox(0,0){$\times$}}
\put(1185,491){\makebox(0,0){$\times$}}
\put(1186,492){\makebox(0,0){$\times$}}
\put(1186,492){\makebox(0,0){$\times$}}
\put(1187,492){\makebox(0,0){$\times$}}
\put(1187,493){\makebox(0,0){$\times$}}
\put(1188,493){\makebox(0,0){$\times$}}
\put(1188,493){\makebox(0,0){$\times$}}
\put(1189,493){\makebox(0,0){$\times$}}
\put(1189,493){\makebox(0,0){$\times$}}
\put(1190,494){\makebox(0,0){$\times$}}
\put(1190,494){\makebox(0,0){$\times$}}
\put(1191,494){\makebox(0,0){$\times$}}
\put(1191,495){\makebox(0,0){$\times$}}
\put(1192,495){\makebox(0,0){$\times$}}
\put(1192,495){\makebox(0,0){$\times$}}
\put(1193,495){\makebox(0,0){$\times$}}
\put(1193,496){\makebox(0,0){$\times$}}
\put(1194,496){\makebox(0,0){$\times$}}
\put(1194,496){\makebox(0,0){$\times$}}
\put(1195,496){\makebox(0,0){$\times$}}
\put(1195,497){\makebox(0,0){$\times$}}
\put(1196,497){\makebox(0,0){$\times$}}
\put(1196,498){\makebox(0,0){$\times$}}
\put(1197,498){\makebox(0,0){$\times$}}
\put(1197,498){\makebox(0,0){$\times$}}
\put(1198,498){\makebox(0,0){$\times$}}
\put(1198,499){\makebox(0,0){$\times$}}
\put(1199,499){\makebox(0,0){$\times$}}
\put(1199,499){\makebox(0,0){$\times$}}
\put(1200,500){\makebox(0,0){$\times$}}
\put(1200,500){\makebox(0,0){$\times$}}
\put(1201,500){\makebox(0,0){$\times$}}
\put(1201,501){\makebox(0,0){$\times$}}
\put(1202,501){\makebox(0,0){$\times$}}
\put(1203,501){\makebox(0,0){$\times$}}
\put(1203,502){\makebox(0,0){$\times$}}
\put(1204,502){\makebox(0,0){$\times$}}
\put(1204,502){\makebox(0,0){$\times$}}
\put(1205,503){\makebox(0,0){$\times$}}
\put(1205,503){\makebox(0,0){$\times$}}
\put(1206,503){\makebox(0,0){$\times$}}
\put(1206,503){\makebox(0,0){$\times$}}
\put(1207,504){\makebox(0,0){$\times$}}
\put(1207,504){\makebox(0,0){$\times$}}
\put(1208,505){\makebox(0,0){$\times$}}
\put(1208,505){\makebox(0,0){$\times$}}
\put(1209,505){\makebox(0,0){$\times$}}
\put(1209,505){\makebox(0,0){$\times$}}
\put(1210,506){\makebox(0,0){$\times$}}
\put(1211,507){\makebox(0,0){$\times$}}
\put(1213,508){\makebox(0,0){$\times$}}
\put(1213,508){\makebox(0,0){$\times$}}
\put(1214,509){\makebox(0,0){$\times$}}
\put(1214,509){\makebox(0,0){$\times$}}
\put(1215,509){\makebox(0,0){$\times$}}
\put(1215,509){\makebox(0,0){$\times$}}
\put(1216,510){\makebox(0,0){$\times$}}
\put(1216,510){\makebox(0,0){$\times$}}
\put(1217,510){\makebox(0,0){$\times$}}
\put(1218,511){\makebox(0,0){$\times$}}
\put(1218,511){\makebox(0,0){$\times$}}
\put(1219,511){\makebox(0,0){$\times$}}
\put(1220,512){\makebox(0,0){$\times$}}
\put(1220,512){\makebox(0,0){$\times$}}
\put(1221,512){\makebox(0,0){$\times$}}
\put(1221,512){\makebox(0,0){$\times$}}
\put(1222,512){\makebox(0,0){$\times$}}
\put(1222,513){\makebox(0,0){$\times$}}
\put(1223,513){\makebox(0,0){$\times$}}
\put(1223,513){\makebox(0,0){$\times$}}
\put(1224,513){\makebox(0,0){$\times$}}
\put(1224,514){\makebox(0,0){$\times$}}
\put(1225,514){\makebox(0,0){$\times$}}
\put(1225,514){\makebox(0,0){$\times$}}
\put(1226,514){\makebox(0,0){$\times$}}
\put(1226,514){\makebox(0,0){$\times$}}
\put(1227,514){\makebox(0,0){$\times$}}
\put(1227,515){\makebox(0,0){$\times$}}
\put(1228,515){\makebox(0,0){$\times$}}
\put(1228,515){\makebox(0,0){$\times$}}
\put(1229,515){\makebox(0,0){$\times$}}
\put(1229,515){\makebox(0,0){$\times$}}
\put(1230,516){\makebox(0,0){$\times$}}
\put(1230,516){\makebox(0,0){$\times$}}
\put(1231,516){\makebox(0,0){$\times$}}
\put(1231,516){\makebox(0,0){$\times$}}
\put(1232,516){\makebox(0,0){$\times$}}
\put(1232,516){\makebox(0,0){$\times$}}
\put(1233,516){\makebox(0,0){$\times$}}
\put(1233,516){\makebox(0,0){$\times$}}
\put(1234,516){\makebox(0,0){$\times$}}
\put(1234,516){\makebox(0,0){$\times$}}
\put(1235,517){\makebox(0,0){$\times$}}
\put(1235,517){\makebox(0,0){$\times$}}
\put(1236,517){\makebox(0,0){$\times$}}
\put(1237,517){\makebox(0,0){$\times$}}
\put(1237,517){\makebox(0,0){$\times$}}
\put(1238,517){\makebox(0,0){$\times$}}
\put(1238,517){\makebox(0,0){$\times$}}
\put(1239,517){\makebox(0,0){$\times$}}
\put(1239,517){\makebox(0,0){$\times$}}
\put(1240,517){\makebox(0,0){$\times$}}
\put(1240,517){\makebox(0,0){$\times$}}
\put(1241,517){\makebox(0,0){$\times$}}
\put(1241,517){\makebox(0,0){$\times$}}
\put(1242,517){\makebox(0,0){$\times$}}
\put(1242,517){\makebox(0,0){$\times$}}
\put(1243,517){\makebox(0,0){$\times$}}
\put(1243,517){\makebox(0,0){$\times$}}
\put(1244,517){\makebox(0,0){$\times$}}
\put(1244,517){\makebox(0,0){$\times$}}
\put(1245,517){\makebox(0,0){$\times$}}
\put(1245,517){\makebox(0,0){$\times$}}
\put(1246,517){\makebox(0,0){$\times$}}
\put(1246,517){\makebox(0,0){$\times$}}
\put(1247,517){\makebox(0,0){$\times$}}
\put(1247,517){\makebox(0,0){$\times$}}
\put(1248,517){\makebox(0,0){$\times$}}
\put(1248,517){\makebox(0,0){$\times$}}
\put(1249,517){\makebox(0,0){$\times$}}
\put(1249,517){\makebox(0,0){$\times$}}
\put(1250,517){\makebox(0,0){$\times$}}
\put(1250,517){\makebox(0,0){$\times$}}
\put(1251,517){\makebox(0,0){$\times$}}
\put(1251,517){\makebox(0,0){$\times$}}
\put(1252,517){\makebox(0,0){$\times$}}
\put(1252,517){\makebox(0,0){$\times$}}
\put(1253,517){\makebox(0,0){$\times$}}
\put(1254,517){\makebox(0,0){$\times$}}
\put(1254,517){\makebox(0,0){$\times$}}
\put(1255,517){\makebox(0,0){$\times$}}
\put(1255,517){\makebox(0,0){$\times$}}
\put(1256,516){\makebox(0,0){$\times$}}
\put(1256,516){\makebox(0,0){$\times$}}
\put(1257,516){\makebox(0,0){$\times$}}
\put(1257,516){\makebox(0,0){$\times$}}
\put(1258,516){\makebox(0,0){$\times$}}
\put(1258,516){\makebox(0,0){$\times$}}
\put(1259,516){\makebox(0,0){$\times$}}
\put(1259,516){\makebox(0,0){$\times$}}
\put(1260,516){\makebox(0,0){$\times$}}
\put(1260,516){\makebox(0,0){$\times$}}
\put(1261,515){\makebox(0,0){$\times$}}
\put(1261,515){\makebox(0,0){$\times$}}
\put(1262,515){\makebox(0,0){$\times$}}
\put(1262,515){\makebox(0,0){$\times$}}
\put(1263,515){\makebox(0,0){$\times$}}
\put(1263,515){\makebox(0,0){$\times$}}
\put(1264,514){\makebox(0,0){$\times$}}
\put(1264,514){\makebox(0,0){$\times$}}
\put(1265,514){\makebox(0,0){$\times$}}
\put(1265,514){\makebox(0,0){$\times$}}
\put(1266,514){\makebox(0,0){$\times$}}
\put(1266,514){\makebox(0,0){$\times$}}
\put(1267,514){\makebox(0,0){$\times$}}
\put(1267,513){\makebox(0,0){$\times$}}
\put(1268,513){\makebox(0,0){$\times$}}
\put(1268,513){\makebox(0,0){$\times$}}
\put(1269,513){\makebox(0,0){$\times$}}
\put(1269,513){\makebox(0,0){$\times$}}
\put(1270,512){\makebox(0,0){$\times$}}
\put(1271,512){\makebox(0,0){$\times$}}
\put(1271,512){\makebox(0,0){$\times$}}
\put(1272,512){\makebox(0,0){$\times$}}
\put(1272,512){\makebox(0,0){$\times$}}
\put(1273,512){\makebox(0,0){$\times$}}
\put(1273,511){\makebox(0,0){$\times$}}
\put(1274,511){\makebox(0,0){$\times$}}
\put(1274,511){\makebox(0,0){$\times$}}
\put(1275,511){\makebox(0,0){$\times$}}
\put(1275,511){\makebox(0,0){$\times$}}
\put(1277,510){\makebox(0,0){$\times$}}
\put(1277,510){\makebox(0,0){$\times$}}
\put(1278,510){\makebox(0,0){$\times$}}
\put(1278,510){\makebox(0,0){$\times$}}
\put(1279,509){\makebox(0,0){$\times$}}
\put(1279,509){\makebox(0,0){$\times$}}
\put(1280,509){\makebox(0,0){$\times$}}
\put(1280,509){\makebox(0,0){$\times$}}
\put(1281,509){\makebox(0,0){$\times$}}
\put(1281,509){\makebox(0,0){$\times$}}
\put(1282,509){\makebox(0,0){$\times$}}
\put(1282,508){\makebox(0,0){$\times$}}
\put(1283,508){\makebox(0,0){$\times$}}
\put(1283,508){\makebox(0,0){$\times$}}
\put(1284,508){\makebox(0,0){$\times$}}
\put(1284,508){\makebox(0,0){$\times$}}
\put(1285,508){\makebox(0,0){$\times$}}
\put(1288,507){\makebox(0,0){$\times$}}
\put(1289,507){\makebox(0,0){$\times$}}
\put(1289,507){\makebox(0,0){$\times$}}
\put(1290,507){\makebox(0,0){$\times$}}
\put(1290,506){\makebox(0,0){$\times$}}
\put(1291,506){\makebox(0,0){$\times$}}
\put(1291,506){\makebox(0,0){$\times$}}
\put(1293,505){\makebox(0,0){$\times$}}
\put(1294,505){\makebox(0,0){$\times$}}
\put(1294,505){\makebox(0,0){$\times$}}
\put(1295,505){\makebox(0,0){$\times$}}
\put(1295,505){\makebox(0,0){$\times$}}
\put(1296,505){\makebox(0,0){$\times$}}
\put(1296,505){\makebox(0,0){$\times$}}
\put(1297,505){\makebox(0,0){$\times$}}
\put(1297,505){\makebox(0,0){$\times$}}
\put(1298,505){\makebox(0,0){$\times$}}
\put(1298,505){\makebox(0,0){$\times$}}
\put(1299,505){\makebox(0,0){$\times$}}
\put(1299,505){\makebox(0,0){$\times$}}
\put(1300,505){\makebox(0,0){$\times$}}
\put(1300,505){\makebox(0,0){$\times$}}
\put(1301,505){\makebox(0,0){$\times$}}
\put(1301,504){\makebox(0,0){$\times$}}
\put(1302,504){\makebox(0,0){$\times$}}
\put(1302,504){\makebox(0,0){$\times$}}
\put(1303,504){\makebox(0,0){$\times$}}
\put(1304,504){\makebox(0,0){$\times$}}
\put(1304,504){\makebox(0,0){$\times$}}
\put(1305,504){\makebox(0,0){$\times$}}
\put(1305,504){\makebox(0,0){$\times$}}
\put(1306,504){\makebox(0,0){$\times$}}
\put(1306,504){\makebox(0,0){$\times$}}
\put(1307,504){\makebox(0,0){$\times$}}
\put(1307,504){\makebox(0,0){$\times$}}
\put(1308,504){\makebox(0,0){$\times$}}
\put(1308,504){\makebox(0,0){$\times$}}
\put(1309,504){\makebox(0,0){$\times$}}
\put(1309,504){\makebox(0,0){$\times$}}
\put(1310,504){\makebox(0,0){$\times$}}
\put(1310,504){\makebox(0,0){$\times$}}
\put(1311,504){\makebox(0,0){$\times$}}
\put(1311,504){\makebox(0,0){$\times$}}
\put(1312,504){\makebox(0,0){$\times$}}
\put(1312,504){\makebox(0,0){$\times$}}
\put(1313,504){\makebox(0,0){$\times$}}
\put(1313,504){\makebox(0,0){$\times$}}
\put(1314,504){\makebox(0,0){$\times$}}
\put(1314,504){\makebox(0,0){$\times$}}
\put(1315,504){\makebox(0,0){$\times$}}
\put(1315,504){\makebox(0,0){$\times$}}
\put(1316,504){\makebox(0,0){$\times$}}
\put(1316,504){\makebox(0,0){$\times$}}
\put(1317,504){\makebox(0,0){$\times$}}
\put(1317,504){\makebox(0,0){$\times$}}
\put(1318,504){\makebox(0,0){$\times$}}
\put(1318,504){\makebox(0,0){$\times$}}
\put(1319,504){\makebox(0,0){$\times$}}
\put(1319,505){\makebox(0,0){$\times$}}
\put(1320,505){\makebox(0,0){$\times$}}
\put(1321,505){\makebox(0,0){$\times$}}
\put(1321,505){\makebox(0,0){$\times$}}
\put(1322,505){\makebox(0,0){$\times$}}
\put(1322,505){\makebox(0,0){$\times$}}
\put(1323,505){\makebox(0,0){$\times$}}
\put(1323,505){\makebox(0,0){$\times$}}
\put(1324,505){\makebox(0,0){$\times$}}
\put(1324,505){\makebox(0,0){$\times$}}
\put(1325,505){\makebox(0,0){$\times$}}
\put(1325,505){\makebox(0,0){$\times$}}
\put(1326,505){\makebox(0,0){$\times$}}
\put(1326,505){\makebox(0,0){$\times$}}
\put(1327,505){\makebox(0,0){$\times$}}
\put(1327,505){\makebox(0,0){$\times$}}
\put(1328,505){\makebox(0,0){$\times$}}
\put(1328,505){\makebox(0,0){$\times$}}
\put(1329,505){\makebox(0,0){$\times$}}
\put(1329,505){\makebox(0,0){$\times$}}
\put(1330,505){\makebox(0,0){$\times$}}
\put(1330,505){\makebox(0,0){$\times$}}
\put(1334,506){\makebox(0,0){$\times$}}
\put(1334,506){\makebox(0,0){$\times$}}
\put(1335,506){\makebox(0,0){$\times$}}
\put(1335,506){\makebox(0,0){$\times$}}
\put(1336,506){\makebox(0,0){$\times$}}
\put(1336,506){\makebox(0,0){$\times$}}
\put(1337,506){\makebox(0,0){$\times$}}
\put(1349,778){\makebox(0,0){$\times$}}
\put(151.0,131.0){\rule[-0.200pt]{0.400pt}{175.375pt}}
\put(151.0,131.0){\rule[-0.200pt]{310.279pt}{0.400pt}}
\put(1439.0,131.0){\rule[-0.200pt]{0.400pt}{175.375pt}}
\put(151.0,859.0){\rule[-0.200pt]{310.279pt}{0.400pt}}
\end{picture}

\caption{Závislosť polohy $x$ v čase $t$, preložené funkciou $x= \(1\cdot10^7\pm6.7\cdot10^5\)e^{-\(1.26\pm0.01\)t} sin\(\(18.88\pm0.01\)t +\( 15.2\pm0.07\)\) $}  \label{G_1}
\end{figure}



\begin{figure}
% GNUPLOT: LaTeX picture
\setlength{\unitlength}{0.240900pt}
\ifx\plotpoint\undefined\newsavebox{\plotpoint}\fi
\begin{picture}(1500,900)(0,0)
\sbox{\plotpoint}{\rule[-0.200pt]{0.400pt}{0.400pt}}%
\put(151.0,131.0){\rule[-0.200pt]{4.818pt}{0.400pt}}
\put(131,131){\makebox(0,0)[r]{-60}}
\put(1419.0,131.0){\rule[-0.200pt]{4.818pt}{0.400pt}}
\put(151.0,204.0){\rule[-0.200pt]{4.818pt}{0.400pt}}
\put(131,204){\makebox(0,0)[r]{-50}}
\put(1419.0,204.0){\rule[-0.200pt]{4.818pt}{0.400pt}}
\put(151.0,277.0){\rule[-0.200pt]{4.818pt}{0.400pt}}
\put(131,277){\makebox(0,0)[r]{-40}}
\put(1419.0,277.0){\rule[-0.200pt]{4.818pt}{0.400pt}}
\put(151.0,349.0){\rule[-0.200pt]{4.818pt}{0.400pt}}
\put(131,349){\makebox(0,0)[r]{-30}}
\put(1419.0,349.0){\rule[-0.200pt]{4.818pt}{0.400pt}}
\put(151.0,422.0){\rule[-0.200pt]{4.818pt}{0.400pt}}
\put(131,422){\makebox(0,0)[r]{-20}}
\put(1419.0,422.0){\rule[-0.200pt]{4.818pt}{0.400pt}}
\put(151.0,495.0){\rule[-0.200pt]{4.818pt}{0.400pt}}
\put(131,495){\makebox(0,0)[r]{-10}}
\put(1419.0,495.0){\rule[-0.200pt]{4.818pt}{0.400pt}}
\put(151.0,568.0){\rule[-0.200pt]{4.818pt}{0.400pt}}
\put(131,568){\makebox(0,0)[r]{ 0}}
\put(1419.0,568.0){\rule[-0.200pt]{4.818pt}{0.400pt}}
\put(151.0,641.0){\rule[-0.200pt]{4.818pt}{0.400pt}}
\put(131,641){\makebox(0,0)[r]{ 10}}
\put(1419.0,641.0){\rule[-0.200pt]{4.818pt}{0.400pt}}
\put(151.0,713.0){\rule[-0.200pt]{4.818pt}{0.400pt}}
\put(131,713){\makebox(0,0)[r]{ 20}}
\put(1419.0,713.0){\rule[-0.200pt]{4.818pt}{0.400pt}}
\put(151.0,786.0){\rule[-0.200pt]{4.818pt}{0.400pt}}
\put(131,786){\makebox(0,0)[r]{ 30}}
\put(1419.0,786.0){\rule[-0.200pt]{4.818pt}{0.400pt}}
\put(151.0,859.0){\rule[-0.200pt]{4.818pt}{0.400pt}}
\put(131,859){\makebox(0,0)[r]{ 40}}
\put(1419.0,859.0){\rule[-0.200pt]{4.818pt}{0.400pt}}
\put(151.0,131.0){\rule[-0.200pt]{0.400pt}{4.818pt}}
\put(151,90){\makebox(0,0){ 16.5}}
\put(151.0,839.0){\rule[-0.200pt]{0.400pt}{4.818pt}}
\put(335.0,131.0){\rule[-0.200pt]{0.400pt}{4.818pt}}
\put(335,90){\makebox(0,0){ 17}}
\put(335.0,839.0){\rule[-0.200pt]{0.400pt}{4.818pt}}
\put(519.0,131.0){\rule[-0.200pt]{0.400pt}{4.818pt}}
\put(519,90){\makebox(0,0){ 17.5}}
\put(519.0,839.0){\rule[-0.200pt]{0.400pt}{4.818pt}}
\put(703.0,131.0){\rule[-0.200pt]{0.400pt}{4.818pt}}
\put(703,90){\makebox(0,0){ 18}}
\put(703.0,839.0){\rule[-0.200pt]{0.400pt}{4.818pt}}
\put(887.0,131.0){\rule[-0.200pt]{0.400pt}{4.818pt}}
\put(887,90){\makebox(0,0){ 18.5}}
\put(887.0,839.0){\rule[-0.200pt]{0.400pt}{4.818pt}}
\put(1071.0,131.0){\rule[-0.200pt]{0.400pt}{4.818pt}}
\put(1071,90){\makebox(0,0){ 19}}
\put(1071.0,839.0){\rule[-0.200pt]{0.400pt}{4.818pt}}
\put(1255.0,131.0){\rule[-0.200pt]{0.400pt}{4.818pt}}
\put(1255,90){\makebox(0,0){ 19.5}}
\put(1255.0,839.0){\rule[-0.200pt]{0.400pt}{4.818pt}}
\put(1439.0,131.0){\rule[-0.200pt]{0.400pt}{4.818pt}}
\put(1439,90){\makebox(0,0){ 20}}
\put(1439.0,839.0){\rule[-0.200pt]{0.400pt}{4.818pt}}
\put(151.0,131.0){\rule[-0.200pt]{0.400pt}{175.375pt}}
\put(151.0,131.0){\rule[-0.200pt]{310.279pt}{0.400pt}}
\put(1439.0,131.0){\rule[-0.200pt]{0.400pt}{175.375pt}}
\put(151.0,859.0){\rule[-0.200pt]{310.279pt}{0.400pt}}
\put(30,495){\makebox(0,0){\popi{x}{mm}}}
\put(795,29){\makebox(0,0){\popi{t}{s}}}
\put(1279,819){\makebox(0,0)[r]{$x= f(t) $}}
\put(1299.0,819.0){\rule[-0.200pt]{24.090pt}{0.400pt}}
\put(269,465){\usebox{\plotpoint}}
\multiput(269.58,441.06)(0.492,-7.303){21}{\rule{0.119pt}{5.767pt}}
\multiput(268.17,453.03)(12.000,-158.031){2}{\rule{0.400pt}{2.883pt}}
\multiput(281.58,282.81)(0.492,-3.656){19}{\rule{0.118pt}{2.936pt}}
\multiput(280.17,288.91)(11.000,-71.905){2}{\rule{0.400pt}{1.468pt}}
\multiput(292.58,217.00)(0.492,1.315){21}{\rule{0.119pt}{1.133pt}}
\multiput(291.17,217.00)(12.000,28.648){2}{\rule{0.400pt}{0.567pt}}
\multiput(304.58,248.00)(0.492,5.278){21}{\rule{0.119pt}{4.200pt}}
\multiput(303.17,248.00)(12.000,114.283){2}{\rule{0.400pt}{2.100pt}}
\multiput(316.58,371.00)(0.492,7.131){21}{\rule{0.119pt}{5.633pt}}
\multiput(315.17,371.00)(12.000,154.308){2}{\rule{0.400pt}{2.817pt}}
\multiput(328.58,537.00)(0.492,6.657){21}{\rule{0.119pt}{5.267pt}}
\multiput(327.17,537.00)(12.000,144.069){2}{\rule{0.400pt}{2.633pt}}
\multiput(340.58,692.00)(0.492,4.317){19}{\rule{0.118pt}{3.445pt}}
\multiput(339.17,692.00)(11.000,84.849){2}{\rule{0.400pt}{1.723pt}}
\multiput(351.00,784.59)(0.874,0.485){11}{\rule{0.786pt}{0.117pt}}
\multiput(351.00,783.17)(10.369,7.000){2}{\rule{0.393pt}{0.400pt}}
\multiput(363.58,780.21)(0.492,-3.210){21}{\rule{0.119pt}{2.600pt}}
\multiput(362.17,785.60)(12.000,-69.604){2}{\rule{0.400pt}{1.300pt}}
\multiput(375.58,698.15)(0.492,-5.408){21}{\rule{0.119pt}{4.300pt}}
\multiput(374.17,707.08)(12.000,-117.075){2}{\rule{0.400pt}{2.150pt}}
\multiput(387.58,571.18)(0.492,-5.709){21}{\rule{0.119pt}{4.533pt}}
\multiput(386.17,580.59)(12.000,-123.591){2}{\rule{0.400pt}{2.267pt}}
\multiput(399.58,443.30)(0.492,-4.115){21}{\rule{0.119pt}{3.300pt}}
\multiput(398.17,450.15)(12.000,-89.151){2}{\rule{0.400pt}{1.650pt}}
\multiput(411.58,355.91)(0.492,-1.439){19}{\rule{0.118pt}{1.227pt}}
\multiput(410.17,358.45)(11.000,-28.453){2}{\rule{0.400pt}{0.614pt}}
\multiput(422.58,330.00)(0.492,1.616){21}{\rule{0.119pt}{1.367pt}}
\multiput(421.17,330.00)(12.000,35.163){2}{\rule{0.400pt}{0.683pt}}
\multiput(434.58,368.00)(0.492,3.857){21}{\rule{0.119pt}{3.100pt}}
\multiput(433.17,368.00)(12.000,83.566){2}{\rule{0.400pt}{1.550pt}}
\multiput(446.58,458.00)(0.492,4.675){21}{\rule{0.119pt}{3.733pt}}
\multiput(445.17,458.00)(12.000,101.251){2}{\rule{0.400pt}{1.867pt}}
\multiput(458.58,567.00)(0.492,3.943){21}{\rule{0.119pt}{3.167pt}}
\multiput(457.17,567.00)(12.000,85.427){2}{\rule{0.400pt}{1.583pt}}
\multiput(470.58,659.00)(0.492,2.100){19}{\rule{0.118pt}{1.736pt}}
\multiput(469.17,659.00)(11.000,41.396){2}{\rule{0.400pt}{0.868pt}}
\multiput(481.00,702.92)(0.543,-0.492){19}{\rule{0.536pt}{0.118pt}}
\multiput(481.00,703.17)(10.887,-11.000){2}{\rule{0.268pt}{0.400pt}}
\multiput(493.58,684.28)(0.492,-2.564){21}{\rule{0.119pt}{2.100pt}}
\multiput(492.17,688.64)(12.000,-55.641){2}{\rule{0.400pt}{1.050pt}}
\multiput(505.58,620.69)(0.492,-3.684){21}{\rule{0.119pt}{2.967pt}}
\multiput(504.17,626.84)(12.000,-79.843){2}{\rule{0.400pt}{1.483pt}}
\multiput(517.58,535.38)(0.492,-3.469){21}{\rule{0.119pt}{2.800pt}}
\multiput(516.17,541.19)(12.000,-75.188){2}{\rule{0.400pt}{1.400pt}}
\multiput(529.58,458.39)(0.492,-2.219){21}{\rule{0.119pt}{1.833pt}}
\multiput(528.17,462.19)(12.000,-48.195){2}{\rule{0.400pt}{0.917pt}}
\multiput(541.00,412.93)(0.798,-0.485){11}{\rule{0.729pt}{0.117pt}}
\multiput(541.00,413.17)(9.488,-7.000){2}{\rule{0.364pt}{0.400pt}}
\multiput(552.58,407.00)(0.492,1.487){21}{\rule{0.119pt}{1.267pt}}
\multiput(551.17,407.00)(12.000,32.371){2}{\rule{0.400pt}{0.633pt}}
\multiput(564.58,442.00)(0.492,2.736){21}{\rule{0.119pt}{2.233pt}}
\multiput(563.17,442.00)(12.000,59.365){2}{\rule{0.400pt}{1.117pt}}
\multiput(576.58,506.00)(0.492,2.952){21}{\rule{0.119pt}{2.400pt}}
\multiput(575.17,506.00)(12.000,64.019){2}{\rule{0.400pt}{1.200pt}}
\multiput(588.58,575.00)(0.492,2.219){21}{\rule{0.119pt}{1.833pt}}
\multiput(587.17,575.00)(12.000,48.195){2}{\rule{0.400pt}{0.917pt}}
\multiput(600.58,627.00)(0.492,0.920){19}{\rule{0.118pt}{0.827pt}}
\multiput(599.17,627.00)(11.000,18.283){2}{\rule{0.400pt}{0.414pt}}
\multiput(611.58,644.23)(0.492,-0.712){21}{\rule{0.119pt}{0.667pt}}
\multiput(610.17,645.62)(12.000,-15.616){2}{\rule{0.400pt}{0.333pt}}
\multiput(623.58,623.36)(0.492,-1.918){21}{\rule{0.119pt}{1.600pt}}
\multiput(622.17,626.68)(12.000,-41.679){2}{\rule{0.400pt}{0.800pt}}
\multiput(635.58,576.84)(0.492,-2.392){21}{\rule{0.119pt}{1.967pt}}
\multiput(634.17,580.92)(12.000,-51.918){2}{\rule{0.400pt}{0.983pt}}
\multiput(647.58,521.80)(0.492,-2.090){21}{\rule{0.119pt}{1.733pt}}
\multiput(646.17,525.40)(12.000,-45.402){2}{\rule{0.400pt}{0.867pt}}
\multiput(659.58,475.99)(0.492,-1.099){21}{\rule{0.119pt}{0.967pt}}
\multiput(658.17,477.99)(12.000,-23.994){2}{\rule{0.400pt}{0.483pt}}
\multiput(671.00,454.61)(2.248,0.447){3}{\rule{1.567pt}{0.108pt}}
\multiput(671.00,453.17)(7.748,3.000){2}{\rule{0.783pt}{0.400pt}}
\multiput(682.58,457.00)(0.492,1.229){21}{\rule{0.119pt}{1.067pt}}
\multiput(681.17,457.00)(12.000,26.786){2}{\rule{0.400pt}{0.533pt}}
\multiput(694.58,486.00)(0.492,1.875){21}{\rule{0.119pt}{1.567pt}}
\multiput(693.17,486.00)(12.000,40.748){2}{\rule{0.400pt}{0.783pt}}
\multiput(706.58,530.00)(0.492,1.832){21}{\rule{0.119pt}{1.533pt}}
\multiput(705.17,530.00)(12.000,39.817){2}{\rule{0.400pt}{0.767pt}}
\multiput(718.58,573.00)(0.492,1.229){21}{\rule{0.119pt}{1.067pt}}
\multiput(717.17,573.00)(12.000,26.786){2}{\rule{0.400pt}{0.533pt}}
\multiput(730.00,602.59)(0.943,0.482){9}{\rule{0.833pt}{0.116pt}}
\multiput(730.00,601.17)(9.270,6.000){2}{\rule{0.417pt}{0.400pt}}
\multiput(741.58,605.23)(0.492,-0.712){21}{\rule{0.119pt}{0.667pt}}
\multiput(740.17,606.62)(12.000,-15.616){2}{\rule{0.400pt}{0.333pt}}
\multiput(753.58,586.16)(0.492,-1.358){21}{\rule{0.119pt}{1.167pt}}
\multiput(752.17,588.58)(12.000,-29.579){2}{\rule{0.400pt}{0.583pt}}
\multiput(765.58,553.60)(0.492,-1.530){21}{\rule{0.119pt}{1.300pt}}
\multiput(764.17,556.30)(12.000,-33.302){2}{\rule{0.400pt}{0.650pt}}
\multiput(777.58,518.71)(0.492,-1.186){21}{\rule{0.119pt}{1.033pt}}
\multiput(776.17,520.86)(12.000,-25.855){2}{\rule{0.400pt}{0.517pt}}
\multiput(789.00,493.92)(0.496,-0.492){21}{\rule{0.500pt}{0.119pt}}
\multiput(789.00,494.17)(10.962,-12.000){2}{\rule{0.250pt}{0.400pt}}
\multiput(801.00,483.59)(0.798,0.485){11}{\rule{0.729pt}{0.117pt}}
\multiput(801.00,482.17)(9.488,7.000){2}{\rule{0.364pt}{0.400pt}}
\multiput(812.58,490.00)(0.492,0.927){21}{\rule{0.119pt}{0.833pt}}
\multiput(811.17,490.00)(12.000,20.270){2}{\rule{0.400pt}{0.417pt}}
\multiput(824.58,512.00)(0.492,1.229){21}{\rule{0.119pt}{1.067pt}}
\multiput(823.17,512.00)(12.000,26.786){2}{\rule{0.400pt}{0.533pt}}
\multiput(836.58,541.00)(0.492,1.099){21}{\rule{0.119pt}{0.967pt}}
\multiput(835.17,541.00)(12.000,23.994){2}{\rule{0.400pt}{0.483pt}}
\multiput(848.58,567.00)(0.492,0.625){21}{\rule{0.119pt}{0.600pt}}
\multiput(847.17,567.00)(12.000,13.755){2}{\rule{0.400pt}{0.300pt}}
\multiput(871.58,579.51)(0.492,-0.625){21}{\rule{0.119pt}{0.600pt}}
\multiput(870.17,580.75)(12.000,-13.755){2}{\rule{0.400pt}{0.300pt}}
\multiput(883.58,563.54)(0.492,-0.927){21}{\rule{0.119pt}{0.833pt}}
\multiput(882.17,565.27)(12.000,-20.270){2}{\rule{0.400pt}{0.417pt}}
\multiput(895.58,541.54)(0.492,-0.927){21}{\rule{0.119pt}{0.833pt}}
\multiput(894.17,543.27)(12.000,-20.270){2}{\rule{0.400pt}{0.417pt}}
\multiput(907.58,520.37)(0.492,-0.669){21}{\rule{0.119pt}{0.633pt}}
\multiput(906.17,521.69)(12.000,-14.685){2}{\rule{0.400pt}{0.317pt}}
\multiput(919.00,505.94)(1.651,-0.468){5}{\rule{1.300pt}{0.113pt}}
\multiput(919.00,506.17)(9.302,-4.000){2}{\rule{0.650pt}{0.400pt}}
\multiput(931.00,503.59)(0.798,0.485){11}{\rule{0.729pt}{0.117pt}}
\multiput(931.00,502.17)(9.488,7.000){2}{\rule{0.364pt}{0.400pt}}
\multiput(942.58,510.00)(0.492,0.712){21}{\rule{0.119pt}{0.667pt}}
\multiput(941.17,510.00)(12.000,15.616){2}{\rule{0.400pt}{0.333pt}}
\multiput(954.58,527.00)(0.492,0.755){21}{\rule{0.119pt}{0.700pt}}
\multiput(953.17,527.00)(12.000,16.547){2}{\rule{0.400pt}{0.350pt}}
\multiput(966.58,545.00)(0.492,0.669){21}{\rule{0.119pt}{0.633pt}}
\multiput(965.17,545.00)(12.000,14.685){2}{\rule{0.400pt}{0.317pt}}
\multiput(978.00,561.59)(1.033,0.482){9}{\rule{0.900pt}{0.116pt}}
\multiput(978.00,560.17)(10.132,6.000){2}{\rule{0.450pt}{0.400pt}}
\put(990,565.17){\rule{2.300pt}{0.400pt}}
\multiput(990.00,566.17)(6.226,-2.000){2}{\rule{1.150pt}{0.400pt}}
\multiput(1001.00,563.92)(0.543,-0.492){19}{\rule{0.536pt}{0.118pt}}
\multiput(1001.00,564.17)(10.887,-11.000){2}{\rule{0.268pt}{0.400pt}}
\multiput(1013.58,551.51)(0.492,-0.625){21}{\rule{0.119pt}{0.600pt}}
\multiput(1012.17,552.75)(12.000,-13.755){2}{\rule{0.400pt}{0.300pt}}
\multiput(1025.58,536.65)(0.492,-0.582){21}{\rule{0.119pt}{0.567pt}}
\multiput(1024.17,537.82)(12.000,-12.824){2}{\rule{0.400pt}{0.283pt}}
\multiput(1037.00,523.93)(0.758,-0.488){13}{\rule{0.700pt}{0.117pt}}
\multiput(1037.00,524.17)(10.547,-8.000){2}{\rule{0.350pt}{0.400pt}}
\put(1049,515.67){\rule{2.650pt}{0.400pt}}
\multiput(1049.00,516.17)(5.500,-1.000){2}{\rule{1.325pt}{0.400pt}}
\multiput(1060.00,516.59)(0.874,0.485){11}{\rule{0.786pt}{0.117pt}}
\multiput(1060.00,515.17)(10.369,7.000){2}{\rule{0.393pt}{0.400pt}}
\multiput(1072.00,523.58)(0.543,0.492){19}{\rule{0.536pt}{0.118pt}}
\multiput(1072.00,522.17)(10.887,11.000){2}{\rule{0.268pt}{0.400pt}}
\multiput(1084.00,534.58)(0.496,0.492){21}{\rule{0.500pt}{0.119pt}}
\multiput(1084.00,533.17)(10.962,12.000){2}{\rule{0.250pt}{0.400pt}}
\multiput(1096.00,546.59)(0.669,0.489){15}{\rule{0.633pt}{0.118pt}}
\multiput(1096.00,545.17)(10.685,9.000){2}{\rule{0.317pt}{0.400pt}}
\put(1108,555.17){\rule{2.500pt}{0.400pt}}
\multiput(1108.00,554.17)(6.811,2.000){2}{\rule{1.250pt}{0.400pt}}
\multiput(1120.00,555.95)(2.248,-0.447){3}{\rule{1.567pt}{0.108pt}}
\multiput(1120.00,556.17)(7.748,-3.000){2}{\rule{0.783pt}{0.400pt}}
\multiput(1131.00,552.93)(0.758,-0.488){13}{\rule{0.700pt}{0.117pt}}
\multiput(1131.00,553.17)(10.547,-8.000){2}{\rule{0.350pt}{0.400pt}}
\multiput(1143.00,544.92)(0.600,-0.491){17}{\rule{0.580pt}{0.118pt}}
\multiput(1143.00,545.17)(10.796,-10.000){2}{\rule{0.290pt}{0.400pt}}
\multiput(1155.00,534.93)(0.758,-0.488){13}{\rule{0.700pt}{0.117pt}}
\multiput(1155.00,535.17)(10.547,-8.000){2}{\rule{0.350pt}{0.400pt}}
\multiput(1167.00,526.94)(1.651,-0.468){5}{\rule{1.300pt}{0.113pt}}
\multiput(1167.00,527.17)(9.302,-4.000){2}{\rule{0.650pt}{0.400pt}}
\put(1179,523.67){\rule{2.650pt}{0.400pt}}
\multiput(1179.00,523.17)(5.500,1.000){2}{\rule{1.325pt}{0.400pt}}
\multiput(1190.00,525.59)(1.267,0.477){7}{\rule{1.060pt}{0.115pt}}
\multiput(1190.00,524.17)(9.800,5.000){2}{\rule{0.530pt}{0.400pt}}
\multiput(1202.00,530.59)(0.758,0.488){13}{\rule{0.700pt}{0.117pt}}
\multiput(1202.00,529.17)(10.547,8.000){2}{\rule{0.350pt}{0.400pt}}
\multiput(1214.00,538.59)(0.874,0.485){11}{\rule{0.786pt}{0.117pt}}
\multiput(1214.00,537.17)(10.369,7.000){2}{\rule{0.393pt}{0.400pt}}
\multiput(1226.00,545.59)(1.267,0.477){7}{\rule{1.060pt}{0.115pt}}
\multiput(1226.00,544.17)(9.800,5.000){2}{\rule{0.530pt}{0.400pt}}
\put(860.0,582.0){\rule[-0.200pt]{2.650pt}{0.400pt}}
\multiput(1250.00,548.95)(2.248,-0.447){3}{\rule{1.567pt}{0.108pt}}
\multiput(1250.00,549.17)(7.748,-3.000){2}{\rule{0.783pt}{0.400pt}}
\multiput(1261.00,545.93)(1.267,-0.477){7}{\rule{1.060pt}{0.115pt}}
\multiput(1261.00,546.17)(9.800,-5.000){2}{\rule{0.530pt}{0.400pt}}
\multiput(1273.00,540.93)(0.874,-0.485){11}{\rule{0.786pt}{0.117pt}}
\multiput(1273.00,541.17)(10.369,-7.000){2}{\rule{0.393pt}{0.400pt}}
\multiput(1285.00,533.94)(1.651,-0.468){5}{\rule{1.300pt}{0.113pt}}
\multiput(1285.00,534.17)(9.302,-4.000){2}{\rule{0.650pt}{0.400pt}}
\put(1297,529.17){\rule{2.500pt}{0.400pt}}
\multiput(1297.00,530.17)(6.811,-2.000){2}{\rule{1.250pt}{0.400pt}}
\put(1309,529.17){\rule{2.300pt}{0.400pt}}
\multiput(1309.00,528.17)(6.226,2.000){2}{\rule{1.150pt}{0.400pt}}
\multiput(1320.00,531.60)(1.651,0.468){5}{\rule{1.300pt}{0.113pt}}
\multiput(1320.00,530.17)(9.302,4.000){2}{\rule{0.650pt}{0.400pt}}
\multiput(1332.00,535.59)(1.267,0.477){7}{\rule{1.060pt}{0.115pt}}
\multiput(1332.00,534.17)(9.800,5.000){2}{\rule{0.530pt}{0.400pt}}
\multiput(1344.00,540.60)(1.651,0.468){5}{\rule{1.300pt}{0.113pt}}
\multiput(1344.00,539.17)(9.302,4.000){2}{\rule{0.650pt}{0.400pt}}
\put(1356,544.17){\rule{2.500pt}{0.400pt}}
\multiput(1356.00,543.17)(6.811,2.000){2}{\rule{1.250pt}{0.400pt}}
\put(1238.0,550.0){\rule[-0.200pt]{2.891pt}{0.400pt}}
\multiput(1380.00,544.95)(2.248,-0.447){3}{\rule{1.567pt}{0.108pt}}
\multiput(1380.00,545.17)(7.748,-3.000){2}{\rule{0.783pt}{0.400pt}}
\multiput(1391.00,541.94)(1.651,-0.468){5}{\rule{1.300pt}{0.113pt}}
\multiput(1391.00,542.17)(9.302,-4.000){2}{\rule{0.650pt}{0.400pt}}
\multiput(1403.00,537.95)(2.472,-0.447){3}{\rule{1.700pt}{0.108pt}}
\multiput(1403.00,538.17)(8.472,-3.000){2}{\rule{0.850pt}{0.400pt}}
\multiput(1415.00,534.95)(2.472,-0.447){3}{\rule{1.700pt}{0.108pt}}
\multiput(1415.00,535.17)(8.472,-3.000){2}{\rule{0.850pt}{0.400pt}}
\put(1368.0,546.0){\rule[-0.200pt]{2.891pt}{0.400pt}}
\put(1427.0,533.0){\rule[-0.200pt]{2.891pt}{0.400pt}}
\put(1279,778){\makebox(0,0)[r]{namerané dáta}}
\put(269,189){\makebox(0,0){$\times$}}
\put(269,190){\makebox(0,0){$\times$}}
\put(269,191){\makebox(0,0){$\times$}}
\put(270,192){\makebox(0,0){$\times$}}
\put(270,192){\makebox(0,0){$\times$}}
\put(271,193){\makebox(0,0){$\times$}}
\put(271,194){\makebox(0,0){$\times$}}
\put(271,195){\makebox(0,0){$\times$}}
\put(272,196){\makebox(0,0){$\times$}}
\put(272,197){\makebox(0,0){$\times$}}
\put(272,197){\makebox(0,0){$\times$}}
\put(273,198){\makebox(0,0){$\times$}}
\put(273,199){\makebox(0,0){$\times$}}
\put(274,200){\makebox(0,0){$\times$}}
\put(274,201){\makebox(0,0){$\times$}}
\put(274,201){\makebox(0,0){$\times$}}
\put(275,202){\makebox(0,0){$\times$}}
\put(275,203){\makebox(0,0){$\times$}}
\put(275,204){\makebox(0,0){$\times$}}
\put(276,205){\makebox(0,0){$\times$}}
\put(276,206){\makebox(0,0){$\times$}}
\put(276,207){\makebox(0,0){$\times$}}
\put(277,207){\makebox(0,0){$\times$}}
\put(277,208){\makebox(0,0){$\times$}}
\put(278,209){\makebox(0,0){$\times$}}
\put(278,210){\makebox(0,0){$\times$}}
\put(278,211){\makebox(0,0){$\times$}}
\put(279,212){\makebox(0,0){$\times$}}
\put(279,212){\makebox(0,0){$\times$}}
\put(279,213){\makebox(0,0){$\times$}}
\put(280,214){\makebox(0,0){$\times$}}
\put(280,215){\makebox(0,0){$\times$}}
\put(281,216){\makebox(0,0){$\times$}}
\put(281,216){\makebox(0,0){$\times$}}
\put(281,217){\makebox(0,0){$\times$}}
\put(282,218){\makebox(0,0){$\times$}}
\put(282,219){\makebox(0,0){$\times$}}
\put(282,220){\makebox(0,0){$\times$}}
\put(283,221){\makebox(0,0){$\times$}}
\put(283,222){\makebox(0,0){$\times$}}
\put(283,222){\makebox(0,0){$\times$}}
\put(284,223){\makebox(0,0){$\times$}}
\put(284,224){\makebox(0,0){$\times$}}
\put(285,225){\makebox(0,0){$\times$}}
\put(285,226){\makebox(0,0){$\times$}}
\put(285,227){\makebox(0,0){$\times$}}
\put(286,228){\makebox(0,0){$\times$}}
\put(286,228){\makebox(0,0){$\times$}}
\put(286,229){\makebox(0,0){$\times$}}
\put(287,230){\makebox(0,0){$\times$}}
\put(287,231){\makebox(0,0){$\times$}}
\put(288,232){\makebox(0,0){$\times$}}
\put(288,233){\makebox(0,0){$\times$}}
\put(288,234){\makebox(0,0){$\times$}}
\put(289,235){\makebox(0,0){$\times$}}
\put(289,236){\makebox(0,0){$\times$}}
\put(289,237){\makebox(0,0){$\times$}}
\put(290,238){\makebox(0,0){$\times$}}
\put(290,239){\makebox(0,0){$\times$}}
\put(290,240){\makebox(0,0){$\times$}}
\put(291,241){\makebox(0,0){$\times$}}
\put(291,242){\makebox(0,0){$\times$}}
\put(292,243){\makebox(0,0){$\times$}}
\put(292,244){\makebox(0,0){$\times$}}
\put(292,245){\makebox(0,0){$\times$}}
\put(293,246){\makebox(0,0){$\times$}}
\put(293,247){\makebox(0,0){$\times$}}
\put(293,248){\makebox(0,0){$\times$}}
\put(294,250){\makebox(0,0){$\times$}}
\put(294,251){\makebox(0,0){$\times$}}
\put(295,252){\makebox(0,0){$\times$}}
\put(295,253){\makebox(0,0){$\times$}}
\put(295,255){\makebox(0,0){$\times$}}
\put(296,256){\makebox(0,0){$\times$}}
\put(296,257){\makebox(0,0){$\times$}}
\put(296,259){\makebox(0,0){$\times$}}
\put(297,260){\makebox(0,0){$\times$}}
\put(297,261){\makebox(0,0){$\times$}}
\put(297,263){\makebox(0,0){$\times$}}
\put(298,264){\makebox(0,0){$\times$}}
\put(298,266){\makebox(0,0){$\times$}}
\put(299,267){\makebox(0,0){$\times$}}
\put(299,269){\makebox(0,0){$\times$}}
\put(299,271){\makebox(0,0){$\times$}}
\put(300,272){\makebox(0,0){$\times$}}
\put(300,274){\makebox(0,0){$\times$}}
\put(300,276){\makebox(0,0){$\times$}}
\put(301,277){\makebox(0,0){$\times$}}
\put(301,279){\makebox(0,0){$\times$}}
\put(302,281){\makebox(0,0){$\times$}}
\put(302,283){\makebox(0,0){$\times$}}
\put(302,285){\makebox(0,0){$\times$}}
\put(303,287){\makebox(0,0){$\times$}}
\put(303,289){\makebox(0,0){$\times$}}
\put(303,291){\makebox(0,0){$\times$}}
\put(304,293){\makebox(0,0){$\times$}}
\put(304,295){\makebox(0,0){$\times$}}
\put(304,298){\makebox(0,0){$\times$}}
\put(305,300){\makebox(0,0){$\times$}}
\put(305,302){\makebox(0,0){$\times$}}
\put(306,305){\makebox(0,0){$\times$}}
\put(306,307){\makebox(0,0){$\times$}}
\put(306,309){\makebox(0,0){$\times$}}
\put(307,312){\makebox(0,0){$\times$}}
\put(307,314){\makebox(0,0){$\times$}}
\put(307,317){\makebox(0,0){$\times$}}
\put(308,319){\makebox(0,0){$\times$}}
\put(308,322){\makebox(0,0){$\times$}}
\put(309,325){\makebox(0,0){$\times$}}
\put(309,327){\makebox(0,0){$\times$}}
\put(309,330){\makebox(0,0){$\times$}}
\put(310,333){\makebox(0,0){$\times$}}
\put(310,336){\makebox(0,0){$\times$}}
\put(310,339){\makebox(0,0){$\times$}}
\put(311,342){\makebox(0,0){$\times$}}
\put(311,345){\makebox(0,0){$\times$}}
\put(311,348){\makebox(0,0){$\times$}}
\put(312,351){\makebox(0,0){$\times$}}
\put(312,355){\makebox(0,0){$\times$}}
\put(313,358){\makebox(0,0){$\times$}}
\put(313,362){\makebox(0,0){$\times$}}
\put(313,365){\makebox(0,0){$\times$}}
\put(314,369){\makebox(0,0){$\times$}}
\put(314,372){\makebox(0,0){$\times$}}
\put(314,376){\makebox(0,0){$\times$}}
\put(315,379){\makebox(0,0){$\times$}}
\put(315,383){\makebox(0,0){$\times$}}
\put(315,387){\makebox(0,0){$\times$}}
\put(316,391){\makebox(0,0){$\times$}}
\put(316,394){\makebox(0,0){$\times$}}
\put(317,399){\makebox(0,0){$\times$}}
\put(317,402){\makebox(0,0){$\times$}}
\put(317,406){\makebox(0,0){$\times$}}
\put(318,410){\makebox(0,0){$\times$}}
\put(318,414){\makebox(0,0){$\times$}}
\put(318,418){\makebox(0,0){$\times$}}
\put(319,422){\makebox(0,0){$\times$}}
\put(319,427){\makebox(0,0){$\times$}}
\put(320,431){\makebox(0,0){$\times$}}
\put(320,435){\makebox(0,0){$\times$}}
\put(320,439){\makebox(0,0){$\times$}}
\put(321,444){\makebox(0,0){$\times$}}
\put(321,448){\makebox(0,0){$\times$}}
\put(321,452){\makebox(0,0){$\times$}}
\put(322,456){\makebox(0,0){$\times$}}
\put(322,461){\makebox(0,0){$\times$}}
\put(322,465){\makebox(0,0){$\times$}}
\put(323,470){\makebox(0,0){$\times$}}
\put(323,474){\makebox(0,0){$\times$}}
\put(324,479){\makebox(0,0){$\times$}}
\put(324,483){\makebox(0,0){$\times$}}
\put(324,487){\makebox(0,0){$\times$}}
\put(325,492){\makebox(0,0){$\times$}}
\put(325,496){\makebox(0,0){$\times$}}
\put(325,501){\makebox(0,0){$\times$}}
\put(326,505){\makebox(0,0){$\times$}}
\put(326,510){\makebox(0,0){$\times$}}
\put(327,514){\makebox(0,0){$\times$}}
\put(327,518){\makebox(0,0){$\times$}}
\put(327,523){\makebox(0,0){$\times$}}
\put(328,527){\makebox(0,0){$\times$}}
\put(328,532){\makebox(0,0){$\times$}}
\put(328,536){\makebox(0,0){$\times$}}
\put(329,541){\makebox(0,0){$\times$}}
\put(329,545){\makebox(0,0){$\times$}}
\put(329,550){\makebox(0,0){$\times$}}
\put(330,554){\makebox(0,0){$\times$}}
\put(330,558){\makebox(0,0){$\times$}}
\put(331,563){\makebox(0,0){$\times$}}
\put(331,567){\makebox(0,0){$\times$}}
\put(331,572){\makebox(0,0){$\times$}}
\put(332,576){\makebox(0,0){$\times$}}
\put(332,580){\makebox(0,0){$\times$}}
\put(332,585){\makebox(0,0){$\times$}}
\put(333,589){\makebox(0,0){$\times$}}
\put(333,593){\makebox(0,0){$\times$}}
\put(334,597){\makebox(0,0){$\times$}}
\put(334,602){\makebox(0,0){$\times$}}
\put(334,606){\makebox(0,0){$\times$}}
\put(335,610){\makebox(0,0){$\times$}}
\put(335,614){\makebox(0,0){$\times$}}
\put(335,618){\makebox(0,0){$\times$}}
\put(336,622){\makebox(0,0){$\times$}}
\put(336,626){\makebox(0,0){$\times$}}
\put(336,630){\makebox(0,0){$\times$}}
\put(337,634){\makebox(0,0){$\times$}}
\put(337,638){\makebox(0,0){$\times$}}
\put(338,642){\makebox(0,0){$\times$}}
\put(338,646){\makebox(0,0){$\times$}}
\put(338,650){\makebox(0,0){$\times$}}
\put(339,654){\makebox(0,0){$\times$}}
\put(339,657){\makebox(0,0){$\times$}}
\put(339,661){\makebox(0,0){$\times$}}
\put(340,664){\makebox(0,0){$\times$}}
\put(340,668){\makebox(0,0){$\times$}}
\put(341,671){\makebox(0,0){$\times$}}
\put(341,675){\makebox(0,0){$\times$}}
\put(341,678){\makebox(0,0){$\times$}}
\put(342,682){\makebox(0,0){$\times$}}
\put(342,685){\makebox(0,0){$\times$}}
\put(342,688){\makebox(0,0){$\times$}}
\put(343,692){\makebox(0,0){$\times$}}
\put(343,695){\makebox(0,0){$\times$}}
\put(343,698){\makebox(0,0){$\times$}}
\put(344,701){\makebox(0,0){$\times$}}
\put(344,704){\makebox(0,0){$\times$}}
\put(345,707){\makebox(0,0){$\times$}}
\put(345,710){\makebox(0,0){$\times$}}
\put(345,712){\makebox(0,0){$\times$}}
\put(346,715){\makebox(0,0){$\times$}}
\put(346,718){\makebox(0,0){$\times$}}
\put(346,720){\makebox(0,0){$\times$}}
\put(347,723){\makebox(0,0){$\times$}}
\put(347,725){\makebox(0,0){$\times$}}
\put(348,728){\makebox(0,0){$\times$}}
\put(348,730){\makebox(0,0){$\times$}}
\put(348,733){\makebox(0,0){$\times$}}
\put(349,735){\makebox(0,0){$\times$}}
\put(349,737){\makebox(0,0){$\times$}}
\put(349,740){\makebox(0,0){$\times$}}
\put(350,742){\makebox(0,0){$\times$}}
\put(350,744){\makebox(0,0){$\times$}}
\put(350,747){\makebox(0,0){$\times$}}
\put(351,749){\makebox(0,0){$\times$}}
\put(351,751){\makebox(0,0){$\times$}}
\put(352,753){\makebox(0,0){$\times$}}
\put(352,755){\makebox(0,0){$\times$}}
\put(352,757){\makebox(0,0){$\times$}}
\put(353,759){\makebox(0,0){$\times$}}
\put(353,761){\makebox(0,0){$\times$}}
\put(353,763){\makebox(0,0){$\times$}}
\put(354,765){\makebox(0,0){$\times$}}
\put(354,766){\makebox(0,0){$\times$}}
\put(355,768){\makebox(0,0){$\times$}}
\put(355,769){\makebox(0,0){$\times$}}
\put(355,771){\makebox(0,0){$\times$}}
\put(356,772){\makebox(0,0){$\times$}}
\put(356,772){\makebox(0,0){$\times$}}
\put(356,774){\makebox(0,0){$\times$}}
\put(357,774){\makebox(0,0){$\times$}}
\put(357,775){\makebox(0,0){$\times$}}
\put(357,775){\makebox(0,0){$\times$}}
\put(358,776){\makebox(0,0){$\times$}}
\put(358,777){\makebox(0,0){$\times$}}
\put(359,777){\makebox(0,0){$\times$}}
\put(359,778){\makebox(0,0){$\times$}}
\put(359,778){\makebox(0,0){$\times$}}
\put(360,779){\makebox(0,0){$\times$}}
\put(360,779){\makebox(0,0){$\times$}}
\put(360,780){\makebox(0,0){$\times$}}
\put(361,780){\makebox(0,0){$\times$}}
\put(361,780){\makebox(0,0){$\times$}}
\put(361,780){\makebox(0,0){$\times$}}
\put(362,781){\makebox(0,0){$\times$}}
\put(362,781){\makebox(0,0){$\times$}}
\put(363,781){\makebox(0,0){$\times$}}
\put(363,781){\makebox(0,0){$\times$}}
\put(363,781){\makebox(0,0){$\times$}}
\put(364,781){\makebox(0,0){$\times$}}
\put(364,780){\makebox(0,0){$\times$}}
\put(364,780){\makebox(0,0){$\times$}}
\put(365,780){\makebox(0,0){$\times$}}
\put(365,780){\makebox(0,0){$\times$}}
\put(366,779){\makebox(0,0){$\times$}}
\put(366,779){\makebox(0,0){$\times$}}
\put(366,778){\makebox(0,0){$\times$}}
\put(367,778){\makebox(0,0){$\times$}}
\put(367,777){\makebox(0,0){$\times$}}
\put(367,777){\makebox(0,0){$\times$}}
\put(368,776){\makebox(0,0){$\times$}}
\put(368,775){\makebox(0,0){$\times$}}
\put(368,774){\makebox(0,0){$\times$}}
\put(369,774){\makebox(0,0){$\times$}}
\put(369,773){\makebox(0,0){$\times$}}
\put(370,772){\makebox(0,0){$\times$}}
\put(370,771){\makebox(0,0){$\times$}}
\put(370,770){\makebox(0,0){$\times$}}
\put(371,769){\makebox(0,0){$\times$}}
\put(371,768){\makebox(0,0){$\times$}}
\put(371,766){\makebox(0,0){$\times$}}
\put(372,765){\makebox(0,0){$\times$}}
\put(372,764){\makebox(0,0){$\times$}}
\put(373,762){\makebox(0,0){$\times$}}
\put(373,760){\makebox(0,0){$\times$}}
\put(373,759){\makebox(0,0){$\times$}}
\put(374,757){\makebox(0,0){$\times$}}
\put(374,755){\makebox(0,0){$\times$}}
\put(374,753){\makebox(0,0){$\times$}}
\put(375,752){\makebox(0,0){$\times$}}
\put(375,750){\makebox(0,0){$\times$}}
\put(375,747){\makebox(0,0){$\times$}}
\put(376,746){\makebox(0,0){$\times$}}
\put(376,743){\makebox(0,0){$\times$}}
\put(377,741){\makebox(0,0){$\times$}}
\put(377,739){\makebox(0,0){$\times$}}
\put(377,737){\makebox(0,0){$\times$}}
\put(378,735){\makebox(0,0){$\times$}}
\put(378,732){\makebox(0,0){$\times$}}
\put(378,730){\makebox(0,0){$\times$}}
\put(379,727){\makebox(0,0){$\times$}}
\put(379,725){\makebox(0,0){$\times$}}
\put(380,722){\makebox(0,0){$\times$}}
\put(380,720){\makebox(0,0){$\times$}}
\put(380,717){\makebox(0,0){$\times$}}
\put(381,714){\makebox(0,0){$\times$}}
\put(381,711){\makebox(0,0){$\times$}}
\put(381,709){\makebox(0,0){$\times$}}
\put(382,706){\makebox(0,0){$\times$}}
\put(382,703){\makebox(0,0){$\times$}}
\put(382,700){\makebox(0,0){$\times$}}
\put(383,697){\makebox(0,0){$\times$}}
\put(383,694){\makebox(0,0){$\times$}}
\put(384,691){\makebox(0,0){$\times$}}
\put(384,687){\makebox(0,0){$\times$}}
\put(384,684){\makebox(0,0){$\times$}}
\put(385,681){\makebox(0,0){$\times$}}
\put(385,677){\makebox(0,0){$\times$}}
\put(385,674){\makebox(0,0){$\times$}}
\put(386,671){\makebox(0,0){$\times$}}
\put(386,667){\makebox(0,0){$\times$}}
\put(387,664){\makebox(0,0){$\times$}}
\put(387,660){\makebox(0,0){$\times$}}
\put(387,657){\makebox(0,0){$\times$}}
\put(388,653){\makebox(0,0){$\times$}}
\put(388,650){\makebox(0,0){$\times$}}
\put(388,646){\makebox(0,0){$\times$}}
\put(389,642){\makebox(0,0){$\times$}}
\put(389,639){\makebox(0,0){$\times$}}
\put(389,635){\makebox(0,0){$\times$}}
\put(390,631){\makebox(0,0){$\times$}}
\put(390,628){\makebox(0,0){$\times$}}
\put(391,624){\makebox(0,0){$\times$}}
\put(391,620){\makebox(0,0){$\times$}}
\put(391,616){\makebox(0,0){$\times$}}
\put(392,612){\makebox(0,0){$\times$}}
\put(392,609){\makebox(0,0){$\times$}}
\put(392,605){\makebox(0,0){$\times$}}
\put(393,601){\makebox(0,0){$\times$}}
\put(393,597){\makebox(0,0){$\times$}}
\put(394,593){\makebox(0,0){$\times$}}
\put(394,589){\makebox(0,0){$\times$}}
\put(394,585){\makebox(0,0){$\times$}}
\put(395,581){\makebox(0,0){$\times$}}
\put(395,578){\makebox(0,0){$\times$}}
\put(395,574){\makebox(0,0){$\times$}}
\put(396,570){\makebox(0,0){$\times$}}
\put(396,566){\makebox(0,0){$\times$}}
\put(396,562){\makebox(0,0){$\times$}}
\put(397,558){\makebox(0,0){$\times$}}
\put(397,554){\makebox(0,0){$\times$}}
\put(398,550){\makebox(0,0){$\times$}}
\put(398,546){\makebox(0,0){$\times$}}
\put(399,538){\makebox(0,0){$\times$}}
\put(399,534){\makebox(0,0){$\times$}}
\put(399,531){\makebox(0,0){$\times$}}
\put(400,527){\makebox(0,0){$\times$}}
\put(400,523){\makebox(0,0){$\times$}}
\put(401,519){\makebox(0,0){$\times$}}
\put(401,515){\makebox(0,0){$\times$}}
\put(401,511){\makebox(0,0){$\times$}}
\put(402,507){\makebox(0,0){$\times$}}
\put(402,504){\makebox(0,0){$\times$}}
\put(402,500){\makebox(0,0){$\times$}}
\put(403,496){\makebox(0,0){$\times$}}
\put(403,492){\makebox(0,0){$\times$}}
\put(403,489){\makebox(0,0){$\times$}}
\put(404,485){\makebox(0,0){$\times$}}
\put(404,481){\makebox(0,0){$\times$}}
\put(405,478){\makebox(0,0){$\times$}}
\put(405,474){\makebox(0,0){$\times$}}
\put(405,471){\makebox(0,0){$\times$}}
\put(406,467){\makebox(0,0){$\times$}}
\put(406,464){\makebox(0,0){$\times$}}
\put(406,460){\makebox(0,0){$\times$}}
\put(407,457){\makebox(0,0){$\times$}}
\put(407,453){\makebox(0,0){$\times$}}
\put(407,450){\makebox(0,0){$\times$}}
\put(408,447){\makebox(0,0){$\times$}}
\put(408,443){\makebox(0,0){$\times$}}
\put(409,440){\makebox(0,0){$\times$}}
\put(409,437){\makebox(0,0){$\times$}}
\put(409,434){\makebox(0,0){$\times$}}
\put(410,431){\makebox(0,0){$\times$}}
\put(410,428){\makebox(0,0){$\times$}}
\put(410,425){\makebox(0,0){$\times$}}
\put(411,422){\makebox(0,0){$\times$}}
\put(411,419){\makebox(0,0){$\times$}}
\put(412,416){\makebox(0,0){$\times$}}
\put(412,413){\makebox(0,0){$\times$}}
\put(412,410){\makebox(0,0){$\times$}}
\put(413,408){\makebox(0,0){$\times$}}
\put(413,405){\makebox(0,0){$\times$}}
\put(413,402){\makebox(0,0){$\times$}}
\put(414,400){\makebox(0,0){$\times$}}
\put(414,397){\makebox(0,0){$\times$}}
\put(414,395){\makebox(0,0){$\times$}}
\put(415,393){\makebox(0,0){$\times$}}
\put(415,390){\makebox(0,0){$\times$}}
\put(416,388){\makebox(0,0){$\times$}}
\put(416,386){\makebox(0,0){$\times$}}
\put(416,384){\makebox(0,0){$\times$}}
\put(417,382){\makebox(0,0){$\times$}}
\put(417,380){\makebox(0,0){$\times$}}
\put(417,378){\makebox(0,0){$\times$}}
\put(418,376){\makebox(0,0){$\times$}}
\put(418,374){\makebox(0,0){$\times$}}
\put(419,373){\makebox(0,0){$\times$}}
\put(419,371){\makebox(0,0){$\times$}}
\put(419,369){\makebox(0,0){$\times$}}
\put(420,368){\makebox(0,0){$\times$}}
\put(420,366){\makebox(0,0){$\times$}}
\put(420,365){\makebox(0,0){$\times$}}
\put(421,364){\makebox(0,0){$\times$}}
\put(421,363){\makebox(0,0){$\times$}}
\put(421,361){\makebox(0,0){$\times$}}
\put(422,360){\makebox(0,0){$\times$}}
\put(422,359){\makebox(0,0){$\times$}}
\put(423,359){\makebox(0,0){$\times$}}
\put(423,358){\makebox(0,0){$\times$}}
\put(423,357){\makebox(0,0){$\times$}}
\put(424,356){\makebox(0,0){$\times$}}
\put(424,356){\makebox(0,0){$\times$}}
\put(424,355){\makebox(0,0){$\times$}}
\put(425,354){\makebox(0,0){$\times$}}
\put(425,354){\makebox(0,0){$\times$}}
\put(426,354){\makebox(0,0){$\times$}}
\put(426,353){\makebox(0,0){$\times$}}
\put(426,353){\makebox(0,0){$\times$}}
\put(427,353){\makebox(0,0){$\times$}}
\put(427,353){\makebox(0,0){$\times$}}
\put(427,353){\makebox(0,0){$\times$}}
\put(428,353){\makebox(0,0){$\times$}}
\put(428,353){\makebox(0,0){$\times$}}
\put(428,353){\makebox(0,0){$\times$}}
\put(429,353){\makebox(0,0){$\times$}}
\put(429,354){\makebox(0,0){$\times$}}
\put(430,354){\makebox(0,0){$\times$}}
\put(430,354){\makebox(0,0){$\times$}}
\put(430,355){\makebox(0,0){$\times$}}
\put(431,356){\makebox(0,0){$\times$}}
\put(431,356){\makebox(0,0){$\times$}}
\put(431,357){\makebox(0,0){$\times$}}
\put(432,358){\makebox(0,0){$\times$}}
\put(432,359){\makebox(0,0){$\times$}}
\put(433,360){\makebox(0,0){$\times$}}
\put(433,360){\makebox(0,0){$\times$}}
\put(433,361){\makebox(0,0){$\times$}}
\put(434,363){\makebox(0,0){$\times$}}
\put(434,364){\makebox(0,0){$\times$}}
\put(434,365){\makebox(0,0){$\times$}}
\put(435,366){\makebox(0,0){$\times$}}
\put(435,368){\makebox(0,0){$\times$}}
\put(435,369){\makebox(0,0){$\times$}}
\put(436,371){\makebox(0,0){$\times$}}
\put(436,372){\makebox(0,0){$\times$}}
\put(437,374){\makebox(0,0){$\times$}}
\put(437,376){\makebox(0,0){$\times$}}
\put(437,377){\makebox(0,0){$\times$}}
\put(438,379){\makebox(0,0){$\times$}}
\put(438,381){\makebox(0,0){$\times$}}
\put(438,383){\makebox(0,0){$\times$}}
\put(439,385){\makebox(0,0){$\times$}}
\put(439,387){\makebox(0,0){$\times$}}
\put(440,389){\makebox(0,0){$\times$}}
\put(440,391){\makebox(0,0){$\times$}}
\put(440,393){\makebox(0,0){$\times$}}
\put(441,396){\makebox(0,0){$\times$}}
\put(441,398){\makebox(0,0){$\times$}}
\put(441,400){\makebox(0,0){$\times$}}
\put(442,402){\makebox(0,0){$\times$}}
\put(442,405){\makebox(0,0){$\times$}}
\put(442,407){\makebox(0,0){$\times$}}
\put(443,410){\makebox(0,0){$\times$}}
\put(443,412){\makebox(0,0){$\times$}}
\put(444,415){\makebox(0,0){$\times$}}
\put(444,418){\makebox(0,0){$\times$}}
\put(444,420){\makebox(0,0){$\times$}}
\put(445,423){\makebox(0,0){$\times$}}
\put(445,426){\makebox(0,0){$\times$}}
\put(445,428){\makebox(0,0){$\times$}}
\put(446,431){\makebox(0,0){$\times$}}
\put(446,434){\makebox(0,0){$\times$}}
\put(447,437){\makebox(0,0){$\times$}}
\put(447,440){\makebox(0,0){$\times$}}
\put(447,443){\makebox(0,0){$\times$}}
\put(448,446){\makebox(0,0){$\times$}}
\put(448,449){\makebox(0,0){$\times$}}
\put(448,452){\makebox(0,0){$\times$}}
\put(449,455){\makebox(0,0){$\times$}}
\put(449,458){\makebox(0,0){$\times$}}
\put(449,462){\makebox(0,0){$\times$}}
\put(450,465){\makebox(0,0){$\times$}}
\put(450,468){\makebox(0,0){$\times$}}
\put(451,471){\makebox(0,0){$\times$}}
\put(451,474){\makebox(0,0){$\times$}}
\put(451,478){\makebox(0,0){$\times$}}
\put(452,481){\makebox(0,0){$\times$}}
\put(452,485){\makebox(0,0){$\times$}}
\put(452,488){\makebox(0,0){$\times$}}
\put(453,491){\makebox(0,0){$\times$}}
\put(453,495){\makebox(0,0){$\times$}}
\put(453,498){\makebox(0,0){$\times$}}
\put(454,501){\makebox(0,0){$\times$}}
\put(454,505){\makebox(0,0){$\times$}}
\put(455,508){\makebox(0,0){$\times$}}
\put(455,511){\makebox(0,0){$\times$}}
\put(455,515){\makebox(0,0){$\times$}}
\put(456,518){\makebox(0,0){$\times$}}
\put(456,522){\makebox(0,0){$\times$}}
\put(456,525){\makebox(0,0){$\times$}}
\put(457,529){\makebox(0,0){$\times$}}
\put(457,532){\makebox(0,0){$\times$}}
\put(458,535){\makebox(0,0){$\times$}}
\put(458,539){\makebox(0,0){$\times$}}
\put(459,546){\makebox(0,0){$\times$}}
\put(459,549){\makebox(0,0){$\times$}}
\put(459,553){\makebox(0,0){$\times$}}
\put(460,556){\makebox(0,0){$\times$}}
\put(460,559){\makebox(0,0){$\times$}}
\put(460,563){\makebox(0,0){$\times$}}
\put(461,566){\makebox(0,0){$\times$}}
\put(461,569){\makebox(0,0){$\times$}}
\put(462,573){\makebox(0,0){$\times$}}
\put(462,576){\makebox(0,0){$\times$}}
\put(462,580){\makebox(0,0){$\times$}}
\put(463,583){\makebox(0,0){$\times$}}
\put(463,586){\makebox(0,0){$\times$}}
\put(463,590){\makebox(0,0){$\times$}}
\put(464,593){\makebox(0,0){$\times$}}
\put(464,596){\makebox(0,0){$\times$}}
\put(465,599){\makebox(0,0){$\times$}}
\put(465,602){\makebox(0,0){$\times$}}
\put(465,606){\makebox(0,0){$\times$}}
\put(466,609){\makebox(0,0){$\times$}}
\put(466,612){\makebox(0,0){$\times$}}
\put(466,615){\makebox(0,0){$\times$}}
\put(467,618){\makebox(0,0){$\times$}}
\put(467,621){\makebox(0,0){$\times$}}
\put(467,624){\makebox(0,0){$\times$}}
\put(468,627){\makebox(0,0){$\times$}}
\put(468,630){\makebox(0,0){$\times$}}
\put(469,633){\makebox(0,0){$\times$}}
\put(469,636){\makebox(0,0){$\times$}}
\put(469,639){\makebox(0,0){$\times$}}
\put(470,641){\makebox(0,0){$\times$}}
\put(470,644){\makebox(0,0){$\times$}}
\put(470,647){\makebox(0,0){$\times$}}
\put(471,649){\makebox(0,0){$\times$}}
\put(471,652){\makebox(0,0){$\times$}}
\put(472,655){\makebox(0,0){$\times$}}
\put(472,657){\makebox(0,0){$\times$}}
\put(472,660){\makebox(0,0){$\times$}}
\put(473,663){\makebox(0,0){$\times$}}
\put(473,665){\makebox(0,0){$\times$}}
\put(473,667){\makebox(0,0){$\times$}}
\put(474,670){\makebox(0,0){$\times$}}
\put(474,672){\makebox(0,0){$\times$}}
\put(474,674){\makebox(0,0){$\times$}}
\put(475,676){\makebox(0,0){$\times$}}
\put(475,679){\makebox(0,0){$\times$}}
\put(476,681){\makebox(0,0){$\times$}}
\put(476,683){\makebox(0,0){$\times$}}
\put(476,685){\makebox(0,0){$\times$}}
\put(477,687){\makebox(0,0){$\times$}}
\put(477,689){\makebox(0,0){$\times$}}
\put(477,691){\makebox(0,0){$\times$}}
\put(478,693){\makebox(0,0){$\times$}}
\put(478,695){\makebox(0,0){$\times$}}
\put(479,697){\makebox(0,0){$\times$}}
\put(479,698){\makebox(0,0){$\times$}}
\put(479,700){\makebox(0,0){$\times$}}
\put(480,701){\makebox(0,0){$\times$}}
\put(480,703){\makebox(0,0){$\times$}}
\put(480,704){\makebox(0,0){$\times$}}
\put(481,706){\makebox(0,0){$\times$}}
\put(481,707){\makebox(0,0){$\times$}}
\put(481,708){\makebox(0,0){$\times$}}
\put(482,710){\makebox(0,0){$\times$}}
\put(482,711){\makebox(0,0){$\times$}}
\put(483,712){\makebox(0,0){$\times$}}
\put(483,713){\makebox(0,0){$\times$}}
\put(483,714){\makebox(0,0){$\times$}}
\put(484,715){\makebox(0,0){$\times$}}
\put(484,716){\makebox(0,0){$\times$}}
\put(484,717){\makebox(0,0){$\times$}}
\put(485,717){\makebox(0,0){$\times$}}
\put(485,718){\makebox(0,0){$\times$}}
\put(486,719){\makebox(0,0){$\times$}}
\put(486,719){\makebox(0,0){$\times$}}
\put(486,720){\makebox(0,0){$\times$}}
\put(487,720){\makebox(0,0){$\times$}}
\put(487,721){\makebox(0,0){$\times$}}
\put(487,721){\makebox(0,0){$\times$}}
\put(488,722){\makebox(0,0){$\times$}}
\put(488,722){\makebox(0,0){$\times$}}
\put(488,722){\makebox(0,0){$\times$}}
\put(489,722){\makebox(0,0){$\times$}}
\put(489,722){\makebox(0,0){$\times$}}
\put(490,722){\makebox(0,0){$\times$}}
\put(490,722){\makebox(0,0){$\times$}}
\put(490,722){\makebox(0,0){$\times$}}
\put(491,722){\makebox(0,0){$\times$}}
\put(491,722){\makebox(0,0){$\times$}}
\put(491,722){\makebox(0,0){$\times$}}
\put(492,721){\makebox(0,0){$\times$}}
\put(492,721){\makebox(0,0){$\times$}}
\put(493,720){\makebox(0,0){$\times$}}
\put(493,720){\makebox(0,0){$\times$}}
\put(493,720){\makebox(0,0){$\times$}}
\put(494,719){\makebox(0,0){$\times$}}
\put(494,718){\makebox(0,0){$\times$}}
\put(494,718){\makebox(0,0){$\times$}}
\put(495,717){\makebox(0,0){$\times$}}
\put(495,716){\makebox(0,0){$\times$}}
\put(495,715){\makebox(0,0){$\times$}}
\put(496,714){\makebox(0,0){$\times$}}
\put(496,713){\makebox(0,0){$\times$}}
\put(497,712){\makebox(0,0){$\times$}}
\put(497,711){\makebox(0,0){$\times$}}
\put(497,710){\makebox(0,0){$\times$}}
\put(498,709){\makebox(0,0){$\times$}}
\put(498,708){\makebox(0,0){$\times$}}
\put(498,706){\makebox(0,0){$\times$}}
\put(499,705){\makebox(0,0){$\times$}}
\put(499,704){\makebox(0,0){$\times$}}
\put(499,702){\makebox(0,0){$\times$}}
\put(500,701){\makebox(0,0){$\times$}}
\put(500,699){\makebox(0,0){$\times$}}
\put(501,698){\makebox(0,0){$\times$}}
\put(501,696){\makebox(0,0){$\times$}}
\put(501,694){\makebox(0,0){$\times$}}
\put(502,693){\makebox(0,0){$\times$}}
\put(502,691){\makebox(0,0){$\times$}}
\put(502,689){\makebox(0,0){$\times$}}
\put(503,687){\makebox(0,0){$\times$}}
\put(503,685){\makebox(0,0){$\times$}}
\put(504,683){\makebox(0,0){$\times$}}
\put(504,682){\makebox(0,0){$\times$}}
\put(504,679){\makebox(0,0){$\times$}}
\put(505,677){\makebox(0,0){$\times$}}
\put(505,675){\makebox(0,0){$\times$}}
\put(505,673){\makebox(0,0){$\times$}}
\put(506,671){\makebox(0,0){$\times$}}
\put(506,669){\makebox(0,0){$\times$}}
\put(506,667){\makebox(0,0){$\times$}}
\put(507,664){\makebox(0,0){$\times$}}
\put(507,662){\makebox(0,0){$\times$}}
\put(508,660){\makebox(0,0){$\times$}}
\put(508,657){\makebox(0,0){$\times$}}
\put(508,655){\makebox(0,0){$\times$}}
\put(509,652){\makebox(0,0){$\times$}}
\put(509,650){\makebox(0,0){$\times$}}
\put(509,647){\makebox(0,0){$\times$}}
\put(510,645){\makebox(0,0){$\times$}}
\put(510,642){\makebox(0,0){$\times$}}
\put(511,640){\makebox(0,0){$\times$}}
\put(511,637){\makebox(0,0){$\times$}}
\put(511,634){\makebox(0,0){$\times$}}
\put(512,632){\makebox(0,0){$\times$}}
\put(512,629){\makebox(0,0){$\times$}}
\put(512,626){\makebox(0,0){$\times$}}
\put(513,623){\makebox(0,0){$\times$}}
\put(513,621){\makebox(0,0){$\times$}}
\put(513,618){\makebox(0,0){$\times$}}
\put(514,615){\makebox(0,0){$\times$}}
\put(514,612){\makebox(0,0){$\times$}}
\put(515,609){\makebox(0,0){$\times$}}
\put(515,606){\makebox(0,0){$\times$}}
\put(515,603){\makebox(0,0){$\times$}}
\put(516,600){\makebox(0,0){$\times$}}
\put(516,597){\makebox(0,0){$\times$}}
\put(516,595){\makebox(0,0){$\times$}}
\put(517,592){\makebox(0,0){$\times$}}
\put(517,589){\makebox(0,0){$\times$}}
\put(518,586){\makebox(0,0){$\times$}}
\put(518,583){\makebox(0,0){$\times$}}
\put(518,580){\makebox(0,0){$\times$}}
\put(519,577){\makebox(0,0){$\times$}}
\put(519,574){\makebox(0,0){$\times$}}
\put(519,571){\makebox(0,0){$\times$}}
\put(520,568){\makebox(0,0){$\times$}}
\put(520,565){\makebox(0,0){$\times$}}
\put(520,562){\makebox(0,0){$\times$}}
\put(521,559){\makebox(0,0){$\times$}}
\put(521,556){\makebox(0,0){$\times$}}
\put(522,553){\makebox(0,0){$\times$}}
\put(522,550){\makebox(0,0){$\times$}}
\put(522,547){\makebox(0,0){$\times$}}
\put(523,544){\makebox(0,0){$\times$}}
\put(523,541){\makebox(0,0){$\times$}}
\put(523,538){\makebox(0,0){$\times$}}
\put(524,535){\makebox(0,0){$\times$}}
\put(524,532){\makebox(0,0){$\times$}}
\put(525,529){\makebox(0,0){$\times$}}
\put(525,526){\makebox(0,0){$\times$}}
\put(525,524){\makebox(0,0){$\times$}}
\put(526,521){\makebox(0,0){$\times$}}
\put(526,518){\makebox(0,0){$\times$}}
\put(526,515){\makebox(0,0){$\times$}}
\put(527,512){\makebox(0,0){$\times$}}
\put(527,509){\makebox(0,0){$\times$}}
\put(527,507){\makebox(0,0){$\times$}}
\put(528,504){\makebox(0,0){$\times$}}
\put(528,501){\makebox(0,0){$\times$}}
\put(529,498){\makebox(0,0){$\times$}}
\put(529,495){\makebox(0,0){$\times$}}
\put(529,493){\makebox(0,0){$\times$}}
\put(530,490){\makebox(0,0){$\times$}}
\put(530,487){\makebox(0,0){$\times$}}
\put(530,485){\makebox(0,0){$\times$}}
\put(531,482){\makebox(0,0){$\times$}}
\put(531,480){\makebox(0,0){$\times$}}
\put(532,477){\makebox(0,0){$\times$}}
\put(532,474){\makebox(0,0){$\times$}}
\put(532,472){\makebox(0,0){$\times$}}
\put(533,470){\makebox(0,0){$\times$}}
\put(533,467){\makebox(0,0){$\times$}}
\put(533,465){\makebox(0,0){$\times$}}
\put(534,462){\makebox(0,0){$\times$}}
\put(534,460){\makebox(0,0){$\times$}}
\put(534,458){\makebox(0,0){$\times$}}
\put(535,456){\makebox(0,0){$\times$}}
\put(535,453){\makebox(0,0){$\times$}}
\put(536,451){\makebox(0,0){$\times$}}
\put(536,449){\makebox(0,0){$\times$}}
\put(536,447){\makebox(0,0){$\times$}}
\put(537,445){\makebox(0,0){$\times$}}
\put(537,443){\makebox(0,0){$\times$}}
\put(537,441){\makebox(0,0){$\times$}}
\put(538,439){\makebox(0,0){$\times$}}
\put(538,437){\makebox(0,0){$\times$}}
\put(539,435){\makebox(0,0){$\times$}}
\put(539,434){\makebox(0,0){$\times$}}
\put(539,432){\makebox(0,0){$\times$}}
\put(540,430){\makebox(0,0){$\times$}}
\put(540,428){\makebox(0,0){$\times$}}
\put(540,427){\makebox(0,0){$\times$}}
\put(541,425){\makebox(0,0){$\times$}}
\put(541,424){\makebox(0,0){$\times$}}
\put(541,422){\makebox(0,0){$\times$}}
\put(542,421){\makebox(0,0){$\times$}}
\put(542,419){\makebox(0,0){$\times$}}
\put(543,418){\makebox(0,0){$\times$}}
\put(543,417){\makebox(0,0){$\times$}}
\put(543,416){\makebox(0,0){$\times$}}
\put(544,415){\makebox(0,0){$\times$}}
\put(544,413){\makebox(0,0){$\times$}}
\put(544,412){\makebox(0,0){$\times$}}
\put(545,411){\makebox(0,0){$\times$}}
\put(545,410){\makebox(0,0){$\times$}}
\put(545,409){\makebox(0,0){$\times$}}
\put(546,409){\makebox(0,0){$\times$}}
\put(546,408){\makebox(0,0){$\times$}}
\put(547,407){\makebox(0,0){$\times$}}
\put(547,406){\makebox(0,0){$\times$}}
\put(547,406){\makebox(0,0){$\times$}}
\put(548,405){\makebox(0,0){$\times$}}
\put(548,405){\makebox(0,0){$\times$}}
\put(548,404){\makebox(0,0){$\times$}}
\put(549,404){\makebox(0,0){$\times$}}
\put(549,403){\makebox(0,0){$\times$}}
\put(550,403){\makebox(0,0){$\times$}}
\put(550,403){\makebox(0,0){$\times$}}
\put(550,403){\makebox(0,0){$\times$}}
\put(551,402){\makebox(0,0){$\times$}}
\put(551,402){\makebox(0,0){$\times$}}
\put(551,402){\makebox(0,0){$\times$}}
\put(552,402){\makebox(0,0){$\times$}}
\put(552,402){\makebox(0,0){$\times$}}
\put(552,402){\makebox(0,0){$\times$}}
\put(553,402){\makebox(0,0){$\times$}}
\put(553,403){\makebox(0,0){$\times$}}
\put(554,403){\makebox(0,0){$\times$}}
\put(554,403){\makebox(0,0){$\times$}}
\put(554,404){\makebox(0,0){$\times$}}
\put(555,404){\makebox(0,0){$\times$}}
\put(555,405){\makebox(0,0){$\times$}}
\put(555,405){\makebox(0,0){$\times$}}
\put(556,406){\makebox(0,0){$\times$}}
\put(556,406){\makebox(0,0){$\times$}}
\put(557,407){\makebox(0,0){$\times$}}
\put(557,407){\makebox(0,0){$\times$}}
\put(557,408){\makebox(0,0){$\times$}}
\put(558,409){\makebox(0,0){$\times$}}
\put(558,410){\makebox(0,0){$\times$}}
\put(558,411){\makebox(0,0){$\times$}}
\put(559,412){\makebox(0,0){$\times$}}
\put(559,413){\makebox(0,0){$\times$}}
\put(559,414){\makebox(0,0){$\times$}}
\put(560,415){\makebox(0,0){$\times$}}
\put(560,416){\makebox(0,0){$\times$}}
\put(561,417){\makebox(0,0){$\times$}}
\put(561,418){\makebox(0,0){$\times$}}
\put(561,420){\makebox(0,0){$\times$}}
\put(562,421){\makebox(0,0){$\times$}}
\put(562,422){\makebox(0,0){$\times$}}
\put(562,424){\makebox(0,0){$\times$}}
\put(563,425){\makebox(0,0){$\times$}}
\put(563,427){\makebox(0,0){$\times$}}
\put(564,428){\makebox(0,0){$\times$}}
\put(564,430){\makebox(0,0){$\times$}}
\put(564,431){\makebox(0,0){$\times$}}
\put(565,433){\makebox(0,0){$\times$}}
\put(565,435){\makebox(0,0){$\times$}}
\put(565,436){\makebox(0,0){$\times$}}
\put(566,438){\makebox(0,0){$\times$}}
\put(566,440){\makebox(0,0){$\times$}}
\put(566,442){\makebox(0,0){$\times$}}
\put(567,444){\makebox(0,0){$\times$}}
\put(567,446){\makebox(0,0){$\times$}}
\put(568,448){\makebox(0,0){$\times$}}
\put(568,450){\makebox(0,0){$\times$}}
\put(568,452){\makebox(0,0){$\times$}}
\put(569,454){\makebox(0,0){$\times$}}
\put(569,456){\makebox(0,0){$\times$}}
\put(569,458){\makebox(0,0){$\times$}}
\put(570,460){\makebox(0,0){$\times$}}
\put(570,462){\makebox(0,0){$\times$}}
\put(571,464){\makebox(0,0){$\times$}}
\put(571,467){\makebox(0,0){$\times$}}
\put(571,469){\makebox(0,0){$\times$}}
\put(572,471){\makebox(0,0){$\times$}}
\put(572,473){\makebox(0,0){$\times$}}
\put(572,476){\makebox(0,0){$\times$}}
\put(573,478){\makebox(0,0){$\times$}}
\put(573,480){\makebox(0,0){$\times$}}
\put(573,483){\makebox(0,0){$\times$}}
\put(574,485){\makebox(0,0){$\times$}}
\put(574,488){\makebox(0,0){$\times$}}
\put(575,490){\makebox(0,0){$\times$}}
\put(575,492){\makebox(0,0){$\times$}}
\put(575,495){\makebox(0,0){$\times$}}
\put(576,497){\makebox(0,0){$\times$}}
\put(576,500){\makebox(0,0){$\times$}}
\put(576,502){\makebox(0,0){$\times$}}
\put(577,505){\makebox(0,0){$\times$}}
\put(577,507){\makebox(0,0){$\times$}}
\put(578,510){\makebox(0,0){$\times$}}
\put(578,512){\makebox(0,0){$\times$}}
\put(578,515){\makebox(0,0){$\times$}}
\put(579,517){\makebox(0,0){$\times$}}
\put(579,520){\makebox(0,0){$\times$}}
\put(579,522){\makebox(0,0){$\times$}}
\put(580,525){\makebox(0,0){$\times$}}
\put(580,528){\makebox(0,0){$\times$}}
\put(580,530){\makebox(0,0){$\times$}}
\put(581,533){\makebox(0,0){$\times$}}
\put(581,535){\makebox(0,0){$\times$}}
\put(582,538){\makebox(0,0){$\times$}}
\put(582,541){\makebox(0,0){$\times$}}
\put(582,543){\makebox(0,0){$\times$}}
\put(583,546){\makebox(0,0){$\times$}}
\put(583,548){\makebox(0,0){$\times$}}
\put(583,551){\makebox(0,0){$\times$}}
\put(584,554){\makebox(0,0){$\times$}}
\put(584,556){\makebox(0,0){$\times$}}
\put(585,559){\makebox(0,0){$\times$}}
\put(585,561){\makebox(0,0){$\times$}}
\put(585,564){\makebox(0,0){$\times$}}
\put(586,566){\makebox(0,0){$\times$}}
\put(586,569){\makebox(0,0){$\times$}}
\put(586,571){\makebox(0,0){$\times$}}
\put(587,574){\makebox(0,0){$\times$}}
\put(587,576){\makebox(0,0){$\times$}}
\put(587,579){\makebox(0,0){$\times$}}
\put(588,581){\makebox(0,0){$\times$}}
\put(588,584){\makebox(0,0){$\times$}}
\put(589,586){\makebox(0,0){$\times$}}
\put(589,588){\makebox(0,0){$\times$}}
\put(589,591){\makebox(0,0){$\times$}}
\put(590,593){\makebox(0,0){$\times$}}
\put(590,596){\makebox(0,0){$\times$}}
\put(590,598){\makebox(0,0){$\times$}}
\put(591,600){\makebox(0,0){$\times$}}
\put(591,602){\makebox(0,0){$\times$}}
\put(591,605){\makebox(0,0){$\times$}}
\put(592,607){\makebox(0,0){$\times$}}
\put(592,609){\makebox(0,0){$\times$}}
\put(593,611){\makebox(0,0){$\times$}}
\put(593,614){\makebox(0,0){$\times$}}
\put(593,616){\makebox(0,0){$\times$}}
\put(594,618){\makebox(0,0){$\times$}}
\put(594,620){\makebox(0,0){$\times$}}
\put(594,622){\makebox(0,0){$\times$}}
\put(595,624){\makebox(0,0){$\times$}}
\put(595,626){\makebox(0,0){$\times$}}
\put(596,628){\makebox(0,0){$\times$}}
\put(596,630){\makebox(0,0){$\times$}}
\put(596,632){\makebox(0,0){$\times$}}
\put(597,634){\makebox(0,0){$\times$}}
\put(597,636){\makebox(0,0){$\times$}}
\put(597,637){\makebox(0,0){$\times$}}
\put(598,639){\makebox(0,0){$\times$}}
\put(598,641){\makebox(0,0){$\times$}}
\put(598,643){\makebox(0,0){$\times$}}
\put(599,644){\makebox(0,0){$\times$}}
\put(599,646){\makebox(0,0){$\times$}}
\put(600,648){\makebox(0,0){$\times$}}
\put(600,649){\makebox(0,0){$\times$}}
\put(600,651){\makebox(0,0){$\times$}}
\put(601,652){\makebox(0,0){$\times$}}
\put(601,654){\makebox(0,0){$\times$}}
\put(601,655){\makebox(0,0){$\times$}}
\put(602,656){\makebox(0,0){$\times$}}
\put(602,658){\makebox(0,0){$\times$}}
\put(603,659){\makebox(0,0){$\times$}}
\put(603,660){\makebox(0,0){$\times$}}
\put(603,661){\makebox(0,0){$\times$}}
\put(604,663){\makebox(0,0){$\times$}}
\put(604,664){\makebox(0,0){$\times$}}
\put(604,665){\makebox(0,0){$\times$}}
\put(605,666){\makebox(0,0){$\times$}}
\put(605,667){\makebox(0,0){$\times$}}
\put(605,668){\makebox(0,0){$\times$}}
\put(606,668){\makebox(0,0){$\times$}}
\put(606,669){\makebox(0,0){$\times$}}
\put(607,670){\makebox(0,0){$\times$}}
\put(607,671){\makebox(0,0){$\times$}}
\put(607,672){\makebox(0,0){$\times$}}
\put(608,673){\makebox(0,0){$\times$}}
\put(608,673){\makebox(0,0){$\times$}}
\put(608,674){\makebox(0,0){$\times$}}
\put(609,674){\makebox(0,0){$\times$}}
\put(609,675){\makebox(0,0){$\times$}}
\put(610,676){\makebox(0,0){$\times$}}
\put(610,676){\makebox(0,0){$\times$}}
\put(610,676){\makebox(0,0){$\times$}}
\put(611,677){\makebox(0,0){$\times$}}
\put(611,677){\makebox(0,0){$\times$}}
\put(611,677){\makebox(0,0){$\times$}}
\put(612,677){\makebox(0,0){$\times$}}
\put(612,677){\makebox(0,0){$\times$}}
\put(612,677){\makebox(0,0){$\times$}}
\put(613,677){\makebox(0,0){$\times$}}
\put(613,677){\makebox(0,0){$\times$}}
\put(614,677){\makebox(0,0){$\times$}}
\put(614,677){\makebox(0,0){$\times$}}
\put(614,677){\makebox(0,0){$\times$}}
\put(615,677){\makebox(0,0){$\times$}}
\put(615,677){\makebox(0,0){$\times$}}
\put(615,677){\makebox(0,0){$\times$}}
\put(616,676){\makebox(0,0){$\times$}}
\put(616,676){\makebox(0,0){$\times$}}
\put(617,676){\makebox(0,0){$\times$}}
\put(617,675){\makebox(0,0){$\times$}}
\put(617,675){\makebox(0,0){$\times$}}
\put(618,674){\makebox(0,0){$\times$}}
\put(618,674){\makebox(0,0){$\times$}}
\put(618,673){\makebox(0,0){$\times$}}
\put(619,673){\makebox(0,0){$\times$}}
\put(619,672){\makebox(0,0){$\times$}}
\put(619,671){\makebox(0,0){$\times$}}
\put(620,670){\makebox(0,0){$\times$}}
\put(620,669){\makebox(0,0){$\times$}}
\put(621,668){\makebox(0,0){$\times$}}
\put(621,668){\makebox(0,0){$\times$}}
\put(621,667){\makebox(0,0){$\times$}}
\put(622,666){\makebox(0,0){$\times$}}
\put(622,665){\makebox(0,0){$\times$}}
\put(622,664){\makebox(0,0){$\times$}}
\put(623,663){\makebox(0,0){$\times$}}
\put(623,661){\makebox(0,0){$\times$}}
\put(624,660){\makebox(0,0){$\times$}}
\put(624,659){\makebox(0,0){$\times$}}
\put(624,658){\makebox(0,0){$\times$}}
\put(625,657){\makebox(0,0){$\times$}}
\put(625,655){\makebox(0,0){$\times$}}
\put(625,654){\makebox(0,0){$\times$}}
\put(626,652){\makebox(0,0){$\times$}}
\put(626,651){\makebox(0,0){$\times$}}
\put(626,649){\makebox(0,0){$\times$}}
\put(627,648){\makebox(0,0){$\times$}}
\put(627,646){\makebox(0,0){$\times$}}
\put(628,645){\makebox(0,0){$\times$}}
\put(628,643){\makebox(0,0){$\times$}}
\put(628,642){\makebox(0,0){$\times$}}
\put(629,640){\makebox(0,0){$\times$}}
\put(629,639){\makebox(0,0){$\times$}}
\put(629,637){\makebox(0,0){$\times$}}
\put(630,635){\makebox(0,0){$\times$}}
\put(630,633){\makebox(0,0){$\times$}}
\put(631,631){\makebox(0,0){$\times$}}
\put(631,630){\makebox(0,0){$\times$}}
\put(631,628){\makebox(0,0){$\times$}}
\put(632,626){\makebox(0,0){$\times$}}
\put(632,624){\makebox(0,0){$\times$}}
\put(632,622){\makebox(0,0){$\times$}}
\put(633,620){\makebox(0,0){$\times$}}
\put(633,618){\makebox(0,0){$\times$}}
\put(633,616){\makebox(0,0){$\times$}}
\put(634,614){\makebox(0,0){$\times$}}
\put(634,612){\makebox(0,0){$\times$}}
\put(635,610){\makebox(0,0){$\times$}}
\put(635,608){\makebox(0,0){$\times$}}
\put(635,606){\makebox(0,0){$\times$}}
\put(636,604){\makebox(0,0){$\times$}}
\put(636,602){\makebox(0,0){$\times$}}
\put(636,600){\makebox(0,0){$\times$}}
\put(637,597){\makebox(0,0){$\times$}}
\put(637,595){\makebox(0,0){$\times$}}
\put(637,593){\makebox(0,0){$\times$}}
\put(638,591){\makebox(0,0){$\times$}}
\put(638,589){\makebox(0,0){$\times$}}
\put(639,587){\makebox(0,0){$\times$}}
\put(639,584){\makebox(0,0){$\times$}}
\put(639,582){\makebox(0,0){$\times$}}
\put(640,580){\makebox(0,0){$\times$}}
\put(640,578){\makebox(0,0){$\times$}}
\put(640,576){\makebox(0,0){$\times$}}
\put(641,574){\makebox(0,0){$\times$}}
\put(641,571){\makebox(0,0){$\times$}}
\put(642,569){\makebox(0,0){$\times$}}
\put(642,567){\makebox(0,0){$\times$}}
\put(642,565){\makebox(0,0){$\times$}}
\put(643,562){\makebox(0,0){$\times$}}
\put(643,560){\makebox(0,0){$\times$}}
\put(643,558){\makebox(0,0){$\times$}}
\put(644,556){\makebox(0,0){$\times$}}
\put(644,553){\makebox(0,0){$\times$}}
\put(644,551){\makebox(0,0){$\times$}}
\put(645,549){\makebox(0,0){$\times$}}
\put(645,547){\makebox(0,0){$\times$}}
\put(646,544){\makebox(0,0){$\times$}}
\put(647,538){\makebox(0,0){$\times$}}
\put(647,535){\makebox(0,0){$\times$}}
\put(647,533){\makebox(0,0){$\times$}}
\put(648,531){\makebox(0,0){$\times$}}
\put(648,529){\makebox(0,0){$\times$}}
\put(649,527){\makebox(0,0){$\times$}}
\put(649,525){\makebox(0,0){$\times$}}
\put(649,523){\makebox(0,0){$\times$}}
\put(650,520){\makebox(0,0){$\times$}}
\put(650,518){\makebox(0,0){$\times$}}
\put(650,516){\makebox(0,0){$\times$}}
\put(651,514){\makebox(0,0){$\times$}}
\put(651,512){\makebox(0,0){$\times$}}
\put(651,510){\makebox(0,0){$\times$}}
\put(652,508){\makebox(0,0){$\times$}}
\put(652,506){\makebox(0,0){$\times$}}
\put(653,504){\makebox(0,0){$\times$}}
\put(653,502){\makebox(0,0){$\times$}}
\put(653,500){\makebox(0,0){$\times$}}
\put(654,498){\makebox(0,0){$\times$}}
\put(654,496){\makebox(0,0){$\times$}}
\put(654,495){\makebox(0,0){$\times$}}
\put(655,493){\makebox(0,0){$\times$}}
\put(655,491){\makebox(0,0){$\times$}}
\put(656,489){\makebox(0,0){$\times$}}
\put(656,487){\makebox(0,0){$\times$}}
\put(656,485){\makebox(0,0){$\times$}}
\put(657,484){\makebox(0,0){$\times$}}
\put(657,482){\makebox(0,0){$\times$}}
\put(657,480){\makebox(0,0){$\times$}}
\put(658,479){\makebox(0,0){$\times$}}
\put(658,477){\makebox(0,0){$\times$}}
\put(658,476){\makebox(0,0){$\times$}}
\put(659,474){\makebox(0,0){$\times$}}
\put(659,472){\makebox(0,0){$\times$}}
\put(660,471){\makebox(0,0){$\times$}}
\put(660,469){\makebox(0,0){$\times$}}
\put(660,468){\makebox(0,0){$\times$}}
\put(661,467){\makebox(0,0){$\times$}}
\put(661,465){\makebox(0,0){$\times$}}
\put(661,464){\makebox(0,0){$\times$}}
\put(662,462){\makebox(0,0){$\times$}}
\put(662,461){\makebox(0,0){$\times$}}
\put(663,460){\makebox(0,0){$\times$}}
\put(663,459){\makebox(0,0){$\times$}}
\put(663,458){\makebox(0,0){$\times$}}
\put(664,457){\makebox(0,0){$\times$}}
\put(664,456){\makebox(0,0){$\times$}}
\put(664,455){\makebox(0,0){$\times$}}
\put(665,453){\makebox(0,0){$\times$}}
\put(665,453){\makebox(0,0){$\times$}}
\put(665,452){\makebox(0,0){$\times$}}
\put(666,451){\makebox(0,0){$\times$}}
\put(666,450){\makebox(0,0){$\times$}}
\put(667,449){\makebox(0,0){$\times$}}
\put(667,448){\makebox(0,0){$\times$}}
\put(667,448){\makebox(0,0){$\times$}}
\put(668,447){\makebox(0,0){$\times$}}
\put(668,446){\makebox(0,0){$\times$}}
\put(668,446){\makebox(0,0){$\times$}}
\put(669,445){\makebox(0,0){$\times$}}
\put(669,445){\makebox(0,0){$\times$}}
\put(670,444){\makebox(0,0){$\times$}}
\put(670,443){\makebox(0,0){$\times$}}
\put(670,443){\makebox(0,0){$\times$}}
\put(671,443){\makebox(0,0){$\times$}}
\put(671,442){\makebox(0,0){$\times$}}
\put(671,442){\makebox(0,0){$\times$}}
\put(672,442){\makebox(0,0){$\times$}}
\put(672,442){\makebox(0,0){$\times$}}
\put(672,442){\makebox(0,0){$\times$}}
\put(673,441){\makebox(0,0){$\times$}}
\put(673,441){\makebox(0,0){$\times$}}
\put(674,441){\makebox(0,0){$\times$}}
\put(674,441){\makebox(0,0){$\times$}}
\put(674,441){\makebox(0,0){$\times$}}
\put(675,441){\makebox(0,0){$\times$}}
\put(675,441){\makebox(0,0){$\times$}}
\put(675,442){\makebox(0,0){$\times$}}
\put(676,442){\makebox(0,0){$\times$}}
\put(676,442){\makebox(0,0){$\times$}}
\put(677,442){\makebox(0,0){$\times$}}
\put(677,442){\makebox(0,0){$\times$}}
\put(677,443){\makebox(0,0){$\times$}}
\put(678,443){\makebox(0,0){$\times$}}
\put(678,444){\makebox(0,0){$\times$}}
\put(678,444){\makebox(0,0){$\times$}}
\put(679,445){\makebox(0,0){$\times$}}
\put(679,445){\makebox(0,0){$\times$}}
\put(679,446){\makebox(0,0){$\times$}}
\put(680,446){\makebox(0,0){$\times$}}
\put(680,447){\makebox(0,0){$\times$}}
\put(681,448){\makebox(0,0){$\times$}}
\put(681,448){\makebox(0,0){$\times$}}
\put(681,449){\makebox(0,0){$\times$}}
\put(682,450){\makebox(0,0){$\times$}}
\put(682,451){\makebox(0,0){$\times$}}
\put(682,452){\makebox(0,0){$\times$}}
\put(683,452){\makebox(0,0){$\times$}}
\put(683,453){\makebox(0,0){$\times$}}
\put(683,454){\makebox(0,0){$\times$}}
\put(684,455){\makebox(0,0){$\times$}}
\put(684,456){\makebox(0,0){$\times$}}
\put(685,457){\makebox(0,0){$\times$}}
\put(685,458){\makebox(0,0){$\times$}}
\put(685,459){\makebox(0,0){$\times$}}
\put(686,461){\makebox(0,0){$\times$}}
\put(686,462){\makebox(0,0){$\times$}}
\put(686,463){\makebox(0,0){$\times$}}
\put(687,464){\makebox(0,0){$\times$}}
\put(687,465){\makebox(0,0){$\times$}}
\put(688,467){\makebox(0,0){$\times$}}
\put(688,468){\makebox(0,0){$\times$}}
\put(688,469){\makebox(0,0){$\times$}}
\put(689,471){\makebox(0,0){$\times$}}
\put(689,472){\makebox(0,0){$\times$}}
\put(689,473){\makebox(0,0){$\times$}}
\put(690,475){\makebox(0,0){$\times$}}
\put(690,476){\makebox(0,0){$\times$}}
\put(690,478){\makebox(0,0){$\times$}}
\put(691,479){\makebox(0,0){$\times$}}
\put(691,481){\makebox(0,0){$\times$}}
\put(692,482){\makebox(0,0){$\times$}}
\put(692,484){\makebox(0,0){$\times$}}
\put(692,485){\makebox(0,0){$\times$}}
\put(693,487){\makebox(0,0){$\times$}}
\put(693,489){\makebox(0,0){$\times$}}
\put(693,490){\makebox(0,0){$\times$}}
\put(694,492){\makebox(0,0){$\times$}}
\put(694,494){\makebox(0,0){$\times$}}
\put(695,495){\makebox(0,0){$\times$}}
\put(695,497){\makebox(0,0){$\times$}}
\put(695,499){\makebox(0,0){$\times$}}
\put(696,500){\makebox(0,0){$\times$}}
\put(696,502){\makebox(0,0){$\times$}}
\put(696,504){\makebox(0,0){$\times$}}
\put(697,505){\makebox(0,0){$\times$}}
\put(697,507){\makebox(0,0){$\times$}}
\put(697,509){\makebox(0,0){$\times$}}
\put(698,511){\makebox(0,0){$\times$}}
\put(698,513){\makebox(0,0){$\times$}}
\put(699,514){\makebox(0,0){$\times$}}
\put(699,516){\makebox(0,0){$\times$}}
\put(699,518){\makebox(0,0){$\times$}}
\put(700,520){\makebox(0,0){$\times$}}
\put(700,522){\makebox(0,0){$\times$}}
\put(700,524){\makebox(0,0){$\times$}}
\put(701,525){\makebox(0,0){$\times$}}
\put(701,527){\makebox(0,0){$\times$}}
\put(702,529){\makebox(0,0){$\times$}}
\put(702,531){\makebox(0,0){$\times$}}
\put(702,533){\makebox(0,0){$\times$}}
\put(703,535){\makebox(0,0){$\times$}}
\put(703,537){\makebox(0,0){$\times$}}
\put(703,538){\makebox(0,0){$\times$}}
\put(704,540){\makebox(0,0){$\times$}}
\put(704,544){\makebox(0,0){$\times$}}
\put(705,546){\makebox(0,0){$\times$}}
\put(705,548){\makebox(0,0){$\times$}}
\put(706,550){\makebox(0,0){$\times$}}
\put(706,551){\makebox(0,0){$\times$}}
\put(706,553){\makebox(0,0){$\times$}}
\put(707,555){\makebox(0,0){$\times$}}
\put(707,557){\makebox(0,0){$\times$}}
\put(707,559){\makebox(0,0){$\times$}}
\put(708,561){\makebox(0,0){$\times$}}
\put(708,563){\makebox(0,0){$\times$}}
\put(709,565){\makebox(0,0){$\times$}}
\put(709,566){\makebox(0,0){$\times$}}
\put(709,568){\makebox(0,0){$\times$}}
\put(710,570){\makebox(0,0){$\times$}}
\put(710,572){\makebox(0,0){$\times$}}
\put(710,573){\makebox(0,0){$\times$}}
\put(711,575){\makebox(0,0){$\times$}}
\put(711,577){\makebox(0,0){$\times$}}
\put(711,578){\makebox(0,0){$\times$}}
\put(712,580){\makebox(0,0){$\times$}}
\put(712,582){\makebox(0,0){$\times$}}
\put(713,584){\makebox(0,0){$\times$}}
\put(713,585){\makebox(0,0){$\times$}}
\put(713,587){\makebox(0,0){$\times$}}
\put(714,588){\makebox(0,0){$\times$}}
\put(714,590){\makebox(0,0){$\times$}}
\put(714,592){\makebox(0,0){$\times$}}
\put(715,593){\makebox(0,0){$\times$}}
\put(715,595){\makebox(0,0){$\times$}}
\put(716,596){\makebox(0,0){$\times$}}
\put(716,598){\makebox(0,0){$\times$}}
\put(716,599){\makebox(0,0){$\times$}}
\put(717,601){\makebox(0,0){$\times$}}
\put(717,602){\makebox(0,0){$\times$}}
\put(717,604){\makebox(0,0){$\times$}}
\put(718,605){\makebox(0,0){$\times$}}
\put(718,606){\makebox(0,0){$\times$}}
\put(718,608){\makebox(0,0){$\times$}}
\put(719,609){\makebox(0,0){$\times$}}
\put(719,611){\makebox(0,0){$\times$}}
\put(720,612){\makebox(0,0){$\times$}}
\put(720,613){\makebox(0,0){$\times$}}
\put(720,614){\makebox(0,0){$\times$}}
\put(721,615){\makebox(0,0){$\times$}}
\put(721,617){\makebox(0,0){$\times$}}
\put(721,618){\makebox(0,0){$\times$}}
\put(722,619){\makebox(0,0){$\times$}}
\put(722,620){\makebox(0,0){$\times$}}
\put(723,621){\makebox(0,0){$\times$}}
\put(723,622){\makebox(0,0){$\times$}}
\put(723,623){\makebox(0,0){$\times$}}
\put(724,624){\makebox(0,0){$\times$}}
\put(724,625){\makebox(0,0){$\times$}}
\put(724,626){\makebox(0,0){$\times$}}
\put(725,627){\makebox(0,0){$\times$}}
\put(725,628){\makebox(0,0){$\times$}}
\put(725,629){\makebox(0,0){$\times$}}
\put(726,630){\makebox(0,0){$\times$}}
\put(726,630){\makebox(0,0){$\times$}}
\put(727,631){\makebox(0,0){$\times$}}
\put(727,632){\makebox(0,0){$\times$}}
\put(727,633){\makebox(0,0){$\times$}}
\put(728,633){\makebox(0,0){$\times$}}
\put(728,634){\makebox(0,0){$\times$}}
\put(728,634){\makebox(0,0){$\times$}}
\put(729,635){\makebox(0,0){$\times$}}
\put(729,636){\makebox(0,0){$\times$}}
\put(729,636){\makebox(0,0){$\times$}}
\put(730,636){\makebox(0,0){$\times$}}
\put(730,637){\makebox(0,0){$\times$}}
\put(731,637){\makebox(0,0){$\times$}}
\put(731,638){\makebox(0,0){$\times$}}
\put(731,638){\makebox(0,0){$\times$}}
\put(732,638){\makebox(0,0){$\times$}}
\put(732,639){\makebox(0,0){$\times$}}
\put(732,639){\makebox(0,0){$\times$}}
\put(733,639){\makebox(0,0){$\times$}}
\put(733,639){\makebox(0,0){$\times$}}
\put(734,639){\makebox(0,0){$\times$}}
\put(734,639){\makebox(0,0){$\times$}}
\put(734,639){\makebox(0,0){$\times$}}
\put(735,639){\makebox(0,0){$\times$}}
\put(735,639){\makebox(0,0){$\times$}}
\put(735,639){\makebox(0,0){$\times$}}
\put(736,639){\makebox(0,0){$\times$}}
\put(736,639){\makebox(0,0){$\times$}}
\put(736,639){\makebox(0,0){$\times$}}
\put(737,639){\makebox(0,0){$\times$}}
\put(737,639){\makebox(0,0){$\times$}}
\put(738,639){\makebox(0,0){$\times$}}
\put(738,638){\makebox(0,0){$\times$}}
\put(738,638){\makebox(0,0){$\times$}}
\put(739,638){\makebox(0,0){$\times$}}
\put(739,637){\makebox(0,0){$\times$}}
\put(739,637){\makebox(0,0){$\times$}}
\put(740,637){\makebox(0,0){$\times$}}
\put(740,636){\makebox(0,0){$\times$}}
\put(741,636){\makebox(0,0){$\times$}}
\put(741,635){\makebox(0,0){$\times$}}
\put(741,634){\makebox(0,0){$\times$}}
\put(742,634){\makebox(0,0){$\times$}}
\put(742,633){\makebox(0,0){$\times$}}
\put(742,633){\makebox(0,0){$\times$}}
\put(743,632){\makebox(0,0){$\times$}}
\put(743,631){\makebox(0,0){$\times$}}
\put(743,631){\makebox(0,0){$\times$}}
\put(744,630){\makebox(0,0){$\times$}}
\put(744,629){\makebox(0,0){$\times$}}
\put(745,628){\makebox(0,0){$\times$}}
\put(745,628){\makebox(0,0){$\times$}}
\put(745,627){\makebox(0,0){$\times$}}
\put(746,626){\makebox(0,0){$\times$}}
\put(746,625){\makebox(0,0){$\times$}}
\put(746,624){\makebox(0,0){$\times$}}
\put(747,623){\makebox(0,0){$\times$}}
\put(747,622){\makebox(0,0){$\times$}}
\put(748,621){\makebox(0,0){$\times$}}
\put(748,620){\makebox(0,0){$\times$}}
\put(748,619){\makebox(0,0){$\times$}}
\put(749,618){\makebox(0,0){$\times$}}
\put(749,617){\makebox(0,0){$\times$}}
\put(749,616){\makebox(0,0){$\times$}}
\put(750,615){\makebox(0,0){$\times$}}
\put(750,614){\makebox(0,0){$\times$}}
\put(750,612){\makebox(0,0){$\times$}}
\put(751,611){\makebox(0,0){$\times$}}
\put(751,610){\makebox(0,0){$\times$}}
\put(752,609){\makebox(0,0){$\times$}}
\put(752,607){\makebox(0,0){$\times$}}
\put(752,606){\makebox(0,0){$\times$}}
\put(753,605){\makebox(0,0){$\times$}}
\put(753,603){\makebox(0,0){$\times$}}
\put(753,602){\makebox(0,0){$\times$}}
\put(754,601){\makebox(0,0){$\times$}}
\put(754,599){\makebox(0,0){$\times$}}
\put(755,598){\makebox(0,0){$\times$}}
\put(755,596){\makebox(0,0){$\times$}}
\put(755,595){\makebox(0,0){$\times$}}
\put(756,594){\makebox(0,0){$\times$}}
\put(756,592){\makebox(0,0){$\times$}}
\put(756,591){\makebox(0,0){$\times$}}
\put(757,589){\makebox(0,0){$\times$}}
\put(757,588){\makebox(0,0){$\times$}}
\put(757,586){\makebox(0,0){$\times$}}
\put(758,585){\makebox(0,0){$\times$}}
\put(758,583){\makebox(0,0){$\times$}}
\put(759,582){\makebox(0,0){$\times$}}
\put(759,580){\makebox(0,0){$\times$}}
\put(759,579){\makebox(0,0){$\times$}}
\put(760,577){\makebox(0,0){$\times$}}
\put(760,576){\makebox(0,0){$\times$}}
\put(760,574){\makebox(0,0){$\times$}}
\put(761,572){\makebox(0,0){$\times$}}
\put(761,571){\makebox(0,0){$\times$}}
\put(762,569){\makebox(0,0){$\times$}}
\put(762,568){\makebox(0,0){$\times$}}
\put(762,566){\makebox(0,0){$\times$}}
\put(763,565){\makebox(0,0){$\times$}}
\put(763,563){\makebox(0,0){$\times$}}
\put(763,562){\makebox(0,0){$\times$}}
\put(764,560){\makebox(0,0){$\times$}}
\put(764,558){\makebox(0,0){$\times$}}
\put(764,557){\makebox(0,0){$\times$}}
\put(765,555){\makebox(0,0){$\times$}}
\put(765,554){\makebox(0,0){$\times$}}
\put(766,552){\makebox(0,0){$\times$}}
\put(766,550){\makebox(0,0){$\times$}}
\put(766,549){\makebox(0,0){$\times$}}
\put(767,547){\makebox(0,0){$\times$}}
\put(767,545){\makebox(0,0){$\times$}}
\put(767,544){\makebox(0,0){$\times$}}
\put(768,542){\makebox(0,0){$\times$}}
\put(768,541){\makebox(0,0){$\times$}}
\put(769,539){\makebox(0,0){$\times$}}
\put(769,538){\makebox(0,0){$\times$}}
\put(769,536){\makebox(0,0){$\times$}}
\put(770,535){\makebox(0,0){$\times$}}
\put(770,533){\makebox(0,0){$\times$}}
\put(770,532){\makebox(0,0){$\times$}}
\put(771,530){\makebox(0,0){$\times$}}
\put(771,529){\makebox(0,0){$\times$}}
\put(771,527){\makebox(0,0){$\times$}}
\put(772,526){\makebox(0,0){$\times$}}
\put(772,524){\makebox(0,0){$\times$}}
\put(773,523){\makebox(0,0){$\times$}}
\put(773,521){\makebox(0,0){$\times$}}
\put(773,520){\makebox(0,0){$\times$}}
\put(774,519){\makebox(0,0){$\times$}}
\put(774,517){\makebox(0,0){$\times$}}
\put(774,516){\makebox(0,0){$\times$}}
\put(775,514){\makebox(0,0){$\times$}}
\put(775,513){\makebox(0,0){$\times$}}
\put(775,511){\makebox(0,0){$\times$}}
\put(776,510){\makebox(0,0){$\times$}}
\put(776,509){\makebox(0,0){$\times$}}
\put(777,508){\makebox(0,0){$\times$}}
\put(777,506){\makebox(0,0){$\times$}}
\put(777,505){\makebox(0,0){$\times$}}
\put(778,504){\makebox(0,0){$\times$}}
\put(778,503){\makebox(0,0){$\times$}}
\put(778,501){\makebox(0,0){$\times$}}
\put(779,500){\makebox(0,0){$\times$}}
\put(779,499){\makebox(0,0){$\times$}}
\put(780,498){\makebox(0,0){$\times$}}
\put(780,497){\makebox(0,0){$\times$}}
\put(780,496){\makebox(0,0){$\times$}}
\put(781,495){\makebox(0,0){$\times$}}
\put(781,494){\makebox(0,0){$\times$}}
\put(781,493){\makebox(0,0){$\times$}}
\put(782,492){\makebox(0,0){$\times$}}
\put(782,491){\makebox(0,0){$\times$}}
\put(782,490){\makebox(0,0){$\times$}}
\put(783,489){\makebox(0,0){$\times$}}
\put(783,488){\makebox(0,0){$\times$}}
\put(784,487){\makebox(0,0){$\times$}}
\put(784,486){\makebox(0,0){$\times$}}
\put(784,485){\makebox(0,0){$\times$}}
\put(785,484){\makebox(0,0){$\times$}}
\put(785,484){\makebox(0,0){$\times$}}
\put(785,483){\makebox(0,0){$\times$}}
\put(786,482){\makebox(0,0){$\times$}}
\put(786,482){\makebox(0,0){$\times$}}
\put(787,481){\makebox(0,0){$\times$}}
\put(787,480){\makebox(0,0){$\times$}}
\put(787,479){\makebox(0,0){$\times$}}
\put(788,479){\makebox(0,0){$\times$}}
\put(788,478){\makebox(0,0){$\times$}}
\put(788,478){\makebox(0,0){$\times$}}
\put(789,477){\makebox(0,0){$\times$}}
\put(789,477){\makebox(0,0){$\times$}}
\put(789,476){\makebox(0,0){$\times$}}
\put(790,476){\makebox(0,0){$\times$}}
\put(790,476){\makebox(0,0){$\times$}}
\put(791,475){\makebox(0,0){$\times$}}
\put(791,475){\makebox(0,0){$\times$}}
\put(791,474){\makebox(0,0){$\times$}}
\put(792,474){\makebox(0,0){$\times$}}
\put(792,474){\makebox(0,0){$\times$}}
\put(792,474){\makebox(0,0){$\times$}}
\put(793,473){\makebox(0,0){$\times$}}
\put(793,473){\makebox(0,0){$\times$}}
\put(794,473){\makebox(0,0){$\times$}}
\put(794,473){\makebox(0,0){$\times$}}
\put(794,473){\makebox(0,0){$\times$}}
\put(795,473){\makebox(0,0){$\times$}}
\put(795,473){\makebox(0,0){$\times$}}
\put(795,473){\makebox(0,0){$\times$}}
\put(796,473){\makebox(0,0){$\times$}}
\put(796,473){\makebox(0,0){$\times$}}
\put(796,473){\makebox(0,0){$\times$}}
\put(797,473){\makebox(0,0){$\times$}}
\put(797,473){\makebox(0,0){$\times$}}
\put(798,473){\makebox(0,0){$\times$}}
\put(798,473){\makebox(0,0){$\times$}}
\put(798,474){\makebox(0,0){$\times$}}
\put(799,474){\makebox(0,0){$\times$}}
\put(799,474){\makebox(0,0){$\times$}}
\put(799,474){\makebox(0,0){$\times$}}
\put(800,475){\makebox(0,0){$\times$}}
\put(800,475){\makebox(0,0){$\times$}}
\put(801,476){\makebox(0,0){$\times$}}
\put(801,476){\makebox(0,0){$\times$}}
\put(801,476){\makebox(0,0){$\times$}}
\put(802,477){\makebox(0,0){$\times$}}
\put(802,477){\makebox(0,0){$\times$}}
\put(802,477){\makebox(0,0){$\times$}}
\put(803,478){\makebox(0,0){$\times$}}
\put(803,479){\makebox(0,0){$\times$}}
\put(803,479){\makebox(0,0){$\times$}}
\put(804,480){\makebox(0,0){$\times$}}
\put(804,480){\makebox(0,0){$\times$}}
\put(805,481){\makebox(0,0){$\times$}}
\put(805,482){\makebox(0,0){$\times$}}
\put(805,482){\makebox(0,0){$\times$}}
\put(806,483){\makebox(0,0){$\times$}}
\put(806,484){\makebox(0,0){$\times$}}
\put(806,485){\makebox(0,0){$\times$}}
\put(807,485){\makebox(0,0){$\times$}}
\put(807,486){\makebox(0,0){$\times$}}
\put(808,487){\makebox(0,0){$\times$}}
\put(808,488){\makebox(0,0){$\times$}}
\put(808,489){\makebox(0,0){$\times$}}
\put(809,489){\makebox(0,0){$\times$}}
\put(809,490){\makebox(0,0){$\times$}}
\put(809,491){\makebox(0,0){$\times$}}
\put(810,492){\makebox(0,0){$\times$}}
\put(810,493){\makebox(0,0){$\times$}}
\put(810,494){\makebox(0,0){$\times$}}
\put(811,495){\makebox(0,0){$\times$}}
\put(811,496){\makebox(0,0){$\times$}}
\put(812,497){\makebox(0,0){$\times$}}
\put(812,498){\makebox(0,0){$\times$}}
\put(812,499){\makebox(0,0){$\times$}}
\put(813,500){\makebox(0,0){$\times$}}
\put(813,501){\makebox(0,0){$\times$}}
\put(813,502){\makebox(0,0){$\times$}}
\put(814,504){\makebox(0,0){$\times$}}
\put(814,505){\makebox(0,0){$\times$}}
\put(815,506){\makebox(0,0){$\times$}}
\put(815,507){\makebox(0,0){$\times$}}
\put(815,508){\makebox(0,0){$\times$}}
\put(816,509){\makebox(0,0){$\times$}}
\put(816,510){\makebox(0,0){$\times$}}
\put(816,511){\makebox(0,0){$\times$}}
\put(817,513){\makebox(0,0){$\times$}}
\put(817,514){\makebox(0,0){$\times$}}
\put(817,515){\makebox(0,0){$\times$}}
\put(818,516){\makebox(0,0){$\times$}}
\put(818,518){\makebox(0,0){$\times$}}
\put(819,519){\makebox(0,0){$\times$}}
\put(819,520){\makebox(0,0){$\times$}}
\put(819,522){\makebox(0,0){$\times$}}
\put(820,523){\makebox(0,0){$\times$}}
\put(820,524){\makebox(0,0){$\times$}}
\put(820,525){\makebox(0,0){$\times$}}
\put(821,526){\makebox(0,0){$\times$}}
\put(821,528){\makebox(0,0){$\times$}}
\put(821,529){\makebox(0,0){$\times$}}
\put(822,531){\makebox(0,0){$\times$}}
\put(822,532){\makebox(0,0){$\times$}}
\put(823,533){\makebox(0,0){$\times$}}
\put(823,534){\makebox(0,0){$\times$}}
\put(823,536){\makebox(0,0){$\times$}}
\put(824,537){\makebox(0,0){$\times$}}
\put(824,538){\makebox(0,0){$\times$}}
\put(825,541){\makebox(0,0){$\times$}}
\put(825,542){\makebox(0,0){$\times$}}
\put(826,544){\makebox(0,0){$\times$}}
\put(826,545){\makebox(0,0){$\times$}}
\put(826,546){\makebox(0,0){$\times$}}
\put(827,548){\makebox(0,0){$\times$}}
\put(827,549){\makebox(0,0){$\times$}}
\put(827,550){\makebox(0,0){$\times$}}
\put(828,552){\makebox(0,0){$\times$}}
\put(828,553){\makebox(0,0){$\times$}}
\put(828,554){\makebox(0,0){$\times$}}
\put(829,556){\makebox(0,0){$\times$}}
\put(829,557){\makebox(0,0){$\times$}}
\put(830,558){\makebox(0,0){$\times$}}
\put(830,560){\makebox(0,0){$\times$}}
\put(830,561){\makebox(0,0){$\times$}}
\put(831,562){\makebox(0,0){$\times$}}
\put(831,563){\makebox(0,0){$\times$}}
\put(831,565){\makebox(0,0){$\times$}}
\put(832,566){\makebox(0,0){$\times$}}
\put(832,567){\makebox(0,0){$\times$}}
\put(833,569){\makebox(0,0){$\times$}}
\put(833,570){\makebox(0,0){$\times$}}
\put(833,571){\makebox(0,0){$\times$}}
\put(834,572){\makebox(0,0){$\times$}}
\put(834,574){\makebox(0,0){$\times$}}
\put(834,575){\makebox(0,0){$\times$}}
\put(835,576){\makebox(0,0){$\times$}}
\put(835,577){\makebox(0,0){$\times$}}
\put(835,578){\makebox(0,0){$\times$}}
\put(836,579){\makebox(0,0){$\times$}}
\put(836,580){\makebox(0,0){$\times$}}
\put(837,581){\makebox(0,0){$\times$}}
\put(837,582){\makebox(0,0){$\times$}}
\put(837,584){\makebox(0,0){$\times$}}
\put(838,585){\makebox(0,0){$\times$}}
\put(838,586){\makebox(0,0){$\times$}}
\put(838,587){\makebox(0,0){$\times$}}
\put(839,588){\makebox(0,0){$\times$}}
\put(839,589){\makebox(0,0){$\times$}}
\put(840,590){\makebox(0,0){$\times$}}
\put(840,591){\makebox(0,0){$\times$}}
\put(840,592){\makebox(0,0){$\times$}}
\put(841,593){\makebox(0,0){$\times$}}
\put(841,594){\makebox(0,0){$\times$}}
\put(841,594){\makebox(0,0){$\times$}}
\put(842,595){\makebox(0,0){$\times$}}
\put(842,596){\makebox(0,0){$\times$}}
\put(842,597){\makebox(0,0){$\times$}}
\put(843,598){\makebox(0,0){$\times$}}
\put(843,599){\makebox(0,0){$\times$}}
\put(844,599){\makebox(0,0){$\times$}}
\put(844,600){\makebox(0,0){$\times$}}
\put(844,601){\makebox(0,0){$\times$}}
\put(845,602){\makebox(0,0){$\times$}}
\put(845,602){\makebox(0,0){$\times$}}
\put(845,603){\makebox(0,0){$\times$}}
\put(846,603){\makebox(0,0){$\times$}}
\put(846,604){\makebox(0,0){$\times$}}
\put(847,605){\makebox(0,0){$\times$}}
\put(847,605){\makebox(0,0){$\times$}}
\put(847,606){\makebox(0,0){$\times$}}
\put(848,606){\makebox(0,0){$\times$}}
\put(848,607){\makebox(0,0){$\times$}}
\put(848,607){\makebox(0,0){$\times$}}
\put(849,608){\makebox(0,0){$\times$}}
\put(849,608){\makebox(0,0){$\times$}}
\put(849,609){\makebox(0,0){$\times$}}
\put(850,609){\makebox(0,0){$\times$}}
\put(850,609){\makebox(0,0){$\times$}}
\put(851,610){\makebox(0,0){$\times$}}
\put(851,610){\makebox(0,0){$\times$}}
\put(851,611){\makebox(0,0){$\times$}}
\put(852,611){\makebox(0,0){$\times$}}
\put(852,611){\makebox(0,0){$\times$}}
\put(852,611){\makebox(0,0){$\times$}}
\put(853,611){\makebox(0,0){$\times$}}
\put(853,612){\makebox(0,0){$\times$}}
\put(854,612){\makebox(0,0){$\times$}}
\put(854,612){\makebox(0,0){$\times$}}
\put(854,612){\makebox(0,0){$\times$}}
\put(855,612){\makebox(0,0){$\times$}}
\put(855,612){\makebox(0,0){$\times$}}
\put(855,612){\makebox(0,0){$\times$}}
\put(856,612){\makebox(0,0){$\times$}}
\put(856,612){\makebox(0,0){$\times$}}
\put(856,612){\makebox(0,0){$\times$}}
\put(857,612){\makebox(0,0){$\times$}}
\put(857,612){\makebox(0,0){$\times$}}
\put(858,612){\makebox(0,0){$\times$}}
\put(858,612){\makebox(0,0){$\times$}}
\put(858,612){\makebox(0,0){$\times$}}
\put(859,611){\makebox(0,0){$\times$}}
\put(859,611){\makebox(0,0){$\times$}}
\put(859,611){\makebox(0,0){$\times$}}
\put(860,611){\makebox(0,0){$\times$}}
\put(860,611){\makebox(0,0){$\times$}}
\put(861,610){\makebox(0,0){$\times$}}
\put(861,610){\makebox(0,0){$\times$}}
\put(861,609){\makebox(0,0){$\times$}}
\put(862,609){\makebox(0,0){$\times$}}
\put(862,609){\makebox(0,0){$\times$}}
\put(862,608){\makebox(0,0){$\times$}}
\put(863,608){\makebox(0,0){$\times$}}
\put(863,608){\makebox(0,0){$\times$}}
\put(863,607){\makebox(0,0){$\times$}}
\put(864,606){\makebox(0,0){$\times$}}
\put(864,606){\makebox(0,0){$\times$}}
\put(865,606){\makebox(0,0){$\times$}}
\put(865,605){\makebox(0,0){$\times$}}
\put(865,605){\makebox(0,0){$\times$}}
\put(866,604){\makebox(0,0){$\times$}}
\put(866,603){\makebox(0,0){$\times$}}
\put(866,603){\makebox(0,0){$\times$}}
\put(867,602){\makebox(0,0){$\times$}}
\put(867,602){\makebox(0,0){$\times$}}
\put(867,601){\makebox(0,0){$\times$}}
\put(868,600){\makebox(0,0){$\times$}}
\put(868,599){\makebox(0,0){$\times$}}
\put(869,599){\makebox(0,0){$\times$}}
\put(869,598){\makebox(0,0){$\times$}}
\put(869,597){\makebox(0,0){$\times$}}
\put(870,596){\makebox(0,0){$\times$}}
\put(870,596){\makebox(0,0){$\times$}}
\put(870,595){\makebox(0,0){$\times$}}
\put(871,594){\makebox(0,0){$\times$}}
\put(871,593){\makebox(0,0){$\times$}}
\put(872,593){\makebox(0,0){$\times$}}
\put(872,592){\makebox(0,0){$\times$}}
\put(872,591){\makebox(0,0){$\times$}}
\put(873,590){\makebox(0,0){$\times$}}
\put(873,589){\makebox(0,0){$\times$}}
\put(873,588){\makebox(0,0){$\times$}}
\put(874,587){\makebox(0,0){$\times$}}
\put(874,586){\makebox(0,0){$\times$}}
\put(874,585){\makebox(0,0){$\times$}}
\put(875,584){\makebox(0,0){$\times$}}
\put(875,583){\makebox(0,0){$\times$}}
\put(876,582){\makebox(0,0){$\times$}}
\put(876,581){\makebox(0,0){$\times$}}
\put(876,580){\makebox(0,0){$\times$}}
\put(877,579){\makebox(0,0){$\times$}}
\put(877,578){\makebox(0,0){$\times$}}
\put(877,577){\makebox(0,0){$\times$}}
\put(878,576){\makebox(0,0){$\times$}}
\put(878,575){\makebox(0,0){$\times$}}
\put(879,574){\makebox(0,0){$\times$}}
\put(879,573){\makebox(0,0){$\times$}}
\put(879,572){\makebox(0,0){$\times$}}
\put(880,571){\makebox(0,0){$\times$}}
\put(880,570){\makebox(0,0){$\times$}}
\put(880,569){\makebox(0,0){$\times$}}
\put(881,568){\makebox(0,0){$\times$}}
\put(881,566){\makebox(0,0){$\times$}}
\put(881,565){\makebox(0,0){$\times$}}
\put(882,564){\makebox(0,0){$\times$}}
\put(882,563){\makebox(0,0){$\times$}}
\put(883,562){\makebox(0,0){$\times$}}
\put(883,561){\makebox(0,0){$\times$}}
\put(883,560){\makebox(0,0){$\times$}}
\put(884,559){\makebox(0,0){$\times$}}
\put(884,557){\makebox(0,0){$\times$}}
\put(884,556){\makebox(0,0){$\times$}}
\put(885,555){\makebox(0,0){$\times$}}
\put(885,554){\makebox(0,0){$\times$}}
\put(886,553){\makebox(0,0){$\times$}}
\put(886,552){\makebox(0,0){$\times$}}
\put(886,551){\makebox(0,0){$\times$}}
\put(887,550){\makebox(0,0){$\times$}}
\put(887,548){\makebox(0,0){$\times$}}
\put(887,547){\makebox(0,0){$\times$}}
\put(888,546){\makebox(0,0){$\times$}}
\put(888,545){\makebox(0,0){$\times$}}
\put(888,544){\makebox(0,0){$\times$}}
\put(889,543){\makebox(0,0){$\times$}}
\put(889,542){\makebox(0,0){$\times$}}
\put(890,541){\makebox(0,0){$\times$}}
\put(890,539){\makebox(0,0){$\times$}}
\put(891,538){\makebox(0,0){$\times$}}
\put(891,537){\makebox(0,0){$\times$}}
\put(891,535){\makebox(0,0){$\times$}}
\put(892,534){\makebox(0,0){$\times$}}
\put(892,533){\makebox(0,0){$\times$}}
\put(893,532){\makebox(0,0){$\times$}}
\put(893,531){\makebox(0,0){$\times$}}
\put(893,530){\makebox(0,0){$\times$}}
\put(894,529){\makebox(0,0){$\times$}}
\put(894,528){\makebox(0,0){$\times$}}
\put(894,527){\makebox(0,0){$\times$}}
\put(895,526){\makebox(0,0){$\times$}}
\put(895,525){\makebox(0,0){$\times$}}
\put(895,525){\makebox(0,0){$\times$}}
\put(896,523){\makebox(0,0){$\times$}}
\put(896,523){\makebox(0,0){$\times$}}
\put(897,522){\makebox(0,0){$\times$}}
\put(897,521){\makebox(0,0){$\times$}}
\put(897,520){\makebox(0,0){$\times$}}
\put(898,519){\makebox(0,0){$\times$}}
\put(898,518){\makebox(0,0){$\times$}}
\put(898,517){\makebox(0,0){$\times$}}
\put(899,516){\makebox(0,0){$\times$}}
\put(899,516){\makebox(0,0){$\times$}}
\put(900,515){\makebox(0,0){$\times$}}
\put(900,514){\makebox(0,0){$\times$}}
\put(900,513){\makebox(0,0){$\times$}}
\put(901,512){\makebox(0,0){$\times$}}
\put(901,512){\makebox(0,0){$\times$}}
\put(901,511){\makebox(0,0){$\times$}}
\put(902,510){\makebox(0,0){$\times$}}
\put(902,510){\makebox(0,0){$\times$}}
\put(902,509){\makebox(0,0){$\times$}}
\put(903,508){\makebox(0,0){$\times$}}
\put(903,507){\makebox(0,0){$\times$}}
\put(904,507){\makebox(0,0){$\times$}}
\put(904,506){\makebox(0,0){$\times$}}
\put(904,505){\makebox(0,0){$\times$}}
\put(905,505){\makebox(0,0){$\times$}}
\put(905,504){\makebox(0,0){$\times$}}
\put(905,504){\makebox(0,0){$\times$}}
\put(906,503){\makebox(0,0){$\times$}}
\put(906,503){\makebox(0,0){$\times$}}
\put(907,502){\makebox(0,0){$\times$}}
\put(907,502){\makebox(0,0){$\times$}}
\put(907,501){\makebox(0,0){$\times$}}
\put(908,501){\makebox(0,0){$\times$}}
\put(908,501){\makebox(0,0){$\times$}}
\put(908,500){\makebox(0,0){$\times$}}
\put(909,500){\makebox(0,0){$\times$}}
\put(909,499){\makebox(0,0){$\times$}}
\put(909,499){\makebox(0,0){$\times$}}
\put(910,499){\makebox(0,0){$\times$}}
\put(910,498){\makebox(0,0){$\times$}}
\put(911,498){\makebox(0,0){$\times$}}
\put(911,498){\makebox(0,0){$\times$}}
\put(911,497){\makebox(0,0){$\times$}}
\put(912,497){\makebox(0,0){$\times$}}
\put(912,497){\makebox(0,0){$\times$}}
\put(912,497){\makebox(0,0){$\times$}}
\put(913,496){\makebox(0,0){$\times$}}
\put(913,496){\makebox(0,0){$\times$}}
\put(913,496){\makebox(0,0){$\times$}}
\put(914,496){\makebox(0,0){$\times$}}
\put(914,496){\makebox(0,0){$\times$}}
\put(915,496){\makebox(0,0){$\times$}}
\put(915,496){\makebox(0,0){$\times$}}
\put(915,496){\makebox(0,0){$\times$}}
\put(916,496){\makebox(0,0){$\times$}}
\put(916,496){\makebox(0,0){$\times$}}
\put(916,496){\makebox(0,0){$\times$}}
\put(917,496){\makebox(0,0){$\times$}}
\put(917,496){\makebox(0,0){$\times$}}
\put(918,496){\makebox(0,0){$\times$}}
\put(918,496){\makebox(0,0){$\times$}}
\put(918,496){\makebox(0,0){$\times$}}
\put(919,496){\makebox(0,0){$\times$}}
\put(919,496){\makebox(0,0){$\times$}}
\put(919,497){\makebox(0,0){$\times$}}
\put(920,497){\makebox(0,0){$\times$}}
\put(920,497){\makebox(0,0){$\times$}}
\put(920,497){\makebox(0,0){$\times$}}
\put(921,498){\makebox(0,0){$\times$}}
\put(921,498){\makebox(0,0){$\times$}}
\put(922,498){\makebox(0,0){$\times$}}
\put(922,498){\makebox(0,0){$\times$}}
\put(922,499){\makebox(0,0){$\times$}}
\put(923,499){\makebox(0,0){$\times$}}
\put(923,499){\makebox(0,0){$\times$}}
\put(923,500){\makebox(0,0){$\times$}}
\put(924,500){\makebox(0,0){$\times$}}
\put(924,501){\makebox(0,0){$\times$}}
\put(925,501){\makebox(0,0){$\times$}}
\put(925,501){\makebox(0,0){$\times$}}
\put(925,502){\makebox(0,0){$\times$}}
\put(926,502){\makebox(0,0){$\times$}}
\put(926,503){\makebox(0,0){$\times$}}
\put(926,504){\makebox(0,0){$\times$}}
\put(927,504){\makebox(0,0){$\times$}}
\put(927,505){\makebox(0,0){$\times$}}
\put(927,505){\makebox(0,0){$\times$}}
\put(928,506){\makebox(0,0){$\times$}}
\put(928,506){\makebox(0,0){$\times$}}
\put(929,507){\makebox(0,0){$\times$}}
\put(929,508){\makebox(0,0){$\times$}}
\put(929,508){\makebox(0,0){$\times$}}
\put(930,509){\makebox(0,0){$\times$}}
\put(930,510){\makebox(0,0){$\times$}}
\put(930,510){\makebox(0,0){$\times$}}
\put(931,511){\makebox(0,0){$\times$}}
\put(931,511){\makebox(0,0){$\times$}}
\put(932,512){\makebox(0,0){$\times$}}
\put(932,513){\makebox(0,0){$\times$}}
\put(932,514){\makebox(0,0){$\times$}}
\put(933,514){\makebox(0,0){$\times$}}
\put(933,515){\makebox(0,0){$\times$}}
\put(933,516){\makebox(0,0){$\times$}}
\put(934,517){\makebox(0,0){$\times$}}
\put(934,518){\makebox(0,0){$\times$}}
\put(934,519){\makebox(0,0){$\times$}}
\put(935,519){\makebox(0,0){$\times$}}
\put(935,520){\makebox(0,0){$\times$}}
\put(936,521){\makebox(0,0){$\times$}}
\put(936,522){\makebox(0,0){$\times$}}
\put(936,523){\makebox(0,0){$\times$}}
\put(937,523){\makebox(0,0){$\times$}}
\put(937,524){\makebox(0,0){$\times$}}
\put(937,525){\makebox(0,0){$\times$}}
\put(938,526){\makebox(0,0){$\times$}}
\put(938,527){\makebox(0,0){$\times$}}
\put(939,528){\makebox(0,0){$\times$}}
\put(939,529){\makebox(0,0){$\times$}}
\put(939,529){\makebox(0,0){$\times$}}
\put(940,531){\makebox(0,0){$\times$}}
\put(940,531){\makebox(0,0){$\times$}}
\put(940,532){\makebox(0,0){$\times$}}
\put(941,533){\makebox(0,0){$\times$}}
\put(941,534){\makebox(0,0){$\times$}}
\put(941,535){\makebox(0,0){$\times$}}
\put(942,536){\makebox(0,0){$\times$}}
\put(942,537){\makebox(0,0){$\times$}}
\put(943,538){\makebox(0,0){$\times$}}
\put(943,539){\makebox(0,0){$\times$}}
\put(943,539){\makebox(0,0){$\times$}}
\put(944,541){\makebox(0,0){$\times$}}
\put(944,541){\makebox(0,0){$\times$}}
\put(945,543){\makebox(0,0){$\times$}}
\put(945,544){\makebox(0,0){$\times$}}
\put(946,545){\makebox(0,0){$\times$}}
\put(946,546){\makebox(0,0){$\times$}}
\put(946,547){\makebox(0,0){$\times$}}
\put(947,548){\makebox(0,0){$\times$}}
\put(947,548){\makebox(0,0){$\times$}}
\put(947,550){\makebox(0,0){$\times$}}
\put(948,550){\makebox(0,0){$\times$}}
\put(948,551){\makebox(0,0){$\times$}}
\put(948,552){\makebox(0,0){$\times$}}
\put(949,553){\makebox(0,0){$\times$}}
\put(949,554){\makebox(0,0){$\times$}}
\put(950,555){\makebox(0,0){$\times$}}
\put(950,556){\makebox(0,0){$\times$}}
\put(950,557){\makebox(0,0){$\times$}}
\put(951,557){\makebox(0,0){$\times$}}
\put(951,558){\makebox(0,0){$\times$}}
\put(951,559){\makebox(0,0){$\times$}}
\put(952,560){\makebox(0,0){$\times$}}
\put(952,561){\makebox(0,0){$\times$}}
\put(953,562){\makebox(0,0){$\times$}}
\put(953,563){\makebox(0,0){$\times$}}
\put(953,563){\makebox(0,0){$\times$}}
\put(954,564){\makebox(0,0){$\times$}}
\put(954,565){\makebox(0,0){$\times$}}
\put(954,566){\makebox(0,0){$\times$}}
\put(955,567){\makebox(0,0){$\times$}}
\put(955,568){\makebox(0,0){$\times$}}
\put(955,568){\makebox(0,0){$\times$}}
\put(956,569){\makebox(0,0){$\times$}}
\put(956,570){\makebox(0,0){$\times$}}
\put(957,571){\makebox(0,0){$\times$}}
\put(957,571){\makebox(0,0){$\times$}}
\put(957,572){\makebox(0,0){$\times$}}
\put(958,573){\makebox(0,0){$\times$}}
\put(958,574){\makebox(0,0){$\times$}}
\put(958,574){\makebox(0,0){$\times$}}
\put(959,575){\makebox(0,0){$\times$}}
\put(959,576){\makebox(0,0){$\times$}}
\put(959,576){\makebox(0,0){$\times$}}
\put(960,577){\makebox(0,0){$\times$}}
\put(960,578){\makebox(0,0){$\times$}}
\put(961,578){\makebox(0,0){$\times$}}
\put(961,579){\makebox(0,0){$\times$}}
\put(961,580){\makebox(0,0){$\times$}}
\put(962,580){\makebox(0,0){$\times$}}
\put(962,581){\makebox(0,0){$\times$}}
\put(962,581){\makebox(0,0){$\times$}}
\put(963,582){\makebox(0,0){$\times$}}
\put(963,582){\makebox(0,0){$\times$}}
\put(964,583){\makebox(0,0){$\times$}}
\put(964,583){\makebox(0,0){$\times$}}
\put(964,584){\makebox(0,0){$\times$}}
\put(965,584){\makebox(0,0){$\times$}}
\put(965,585){\makebox(0,0){$\times$}}
\put(965,585){\makebox(0,0){$\times$}}
\put(966,585){\makebox(0,0){$\times$}}
\put(966,586){\makebox(0,0){$\times$}}
\put(966,586){\makebox(0,0){$\times$}}
\put(967,587){\makebox(0,0){$\times$}}
\put(967,587){\makebox(0,0){$\times$}}
\put(968,587){\makebox(0,0){$\times$}}
\put(968,588){\makebox(0,0){$\times$}}
\put(968,588){\makebox(0,0){$\times$}}
\put(969,588){\makebox(0,0){$\times$}}
\put(969,589){\makebox(0,0){$\times$}}
\put(969,589){\makebox(0,0){$\times$}}
\put(970,589){\makebox(0,0){$\times$}}
\put(970,590){\makebox(0,0){$\times$}}
\put(971,590){\makebox(0,0){$\times$}}
\put(971,590){\makebox(0,0){$\times$}}
\put(971,590){\makebox(0,0){$\times$}}
\put(972,590){\makebox(0,0){$\times$}}
\put(972,590){\makebox(0,0){$\times$}}
\put(972,591){\makebox(0,0){$\times$}}
\put(973,591){\makebox(0,0){$\times$}}
\put(973,591){\makebox(0,0){$\times$}}
\put(973,591){\makebox(0,0){$\times$}}
\put(974,591){\makebox(0,0){$\times$}}
\put(974,591){\makebox(0,0){$\times$}}
\put(975,591){\makebox(0,0){$\times$}}
\put(975,591){\makebox(0,0){$\times$}}
\put(975,591){\makebox(0,0){$\times$}}
\put(976,591){\makebox(0,0){$\times$}}
\put(976,591){\makebox(0,0){$\times$}}
\put(976,591){\makebox(0,0){$\times$}}
\put(977,591){\makebox(0,0){$\times$}}
\put(977,591){\makebox(0,0){$\times$}}
\put(978,591){\makebox(0,0){$\times$}}
\put(978,591){\makebox(0,0){$\times$}}
\put(978,590){\makebox(0,0){$\times$}}
\put(979,590){\makebox(0,0){$\times$}}
\put(979,590){\makebox(0,0){$\times$}}
\put(979,590){\makebox(0,0){$\times$}}
\put(980,590){\makebox(0,0){$\times$}}
\put(980,590){\makebox(0,0){$\times$}}
\put(980,589){\makebox(0,0){$\times$}}
\put(981,589){\makebox(0,0){$\times$}}
\put(981,589){\makebox(0,0){$\times$}}
\put(982,588){\makebox(0,0){$\times$}}
\put(982,588){\makebox(0,0){$\times$}}
\put(982,588){\makebox(0,0){$\times$}}
\put(983,588){\makebox(0,0){$\times$}}
\put(983,587){\makebox(0,0){$\times$}}
\put(983,587){\makebox(0,0){$\times$}}
\put(984,587){\makebox(0,0){$\times$}}
\put(984,586){\makebox(0,0){$\times$}}
\put(985,586){\makebox(0,0){$\times$}}
\put(985,585){\makebox(0,0){$\times$}}
\put(985,585){\makebox(0,0){$\times$}}
\put(986,585){\makebox(0,0){$\times$}}
\put(986,584){\makebox(0,0){$\times$}}
\put(986,584){\makebox(0,0){$\times$}}
\put(987,584){\makebox(0,0){$\times$}}
\put(987,583){\makebox(0,0){$\times$}}
\put(987,582){\makebox(0,0){$\times$}}
\put(988,582){\makebox(0,0){$\times$}}
\put(988,582){\makebox(0,0){$\times$}}
\put(989,581){\makebox(0,0){$\times$}}
\put(989,581){\makebox(0,0){$\times$}}
\put(989,580){\makebox(0,0){$\times$}}
\put(990,580){\makebox(0,0){$\times$}}
\put(990,579){\makebox(0,0){$\times$}}
\put(990,579){\makebox(0,0){$\times$}}
\put(991,578){\makebox(0,0){$\times$}}
\put(991,577){\makebox(0,0){$\times$}}
\put(992,577){\makebox(0,0){$\times$}}
\put(992,576){\makebox(0,0){$\times$}}
\put(992,576){\makebox(0,0){$\times$}}
\put(993,575){\makebox(0,0){$\times$}}
\put(993,575){\makebox(0,0){$\times$}}
\put(993,574){\makebox(0,0){$\times$}}
\put(994,573){\makebox(0,0){$\times$}}
\put(994,573){\makebox(0,0){$\times$}}
\put(994,572){\makebox(0,0){$\times$}}
\put(995,571){\makebox(0,0){$\times$}}
\put(995,571){\makebox(0,0){$\times$}}
\put(996,570){\makebox(0,0){$\times$}}
\put(996,569){\makebox(0,0){$\times$}}
\put(996,569){\makebox(0,0){$\times$}}
\put(997,568){\makebox(0,0){$\times$}}
\put(997,568){\makebox(0,0){$\times$}}
\put(997,567){\makebox(0,0){$\times$}}
\put(998,566){\makebox(0,0){$\times$}}
\put(998,566){\makebox(0,0){$\times$}}
\put(999,565){\makebox(0,0){$\times$}}
\put(999,564){\makebox(0,0){$\times$}}
\put(999,563){\makebox(0,0){$\times$}}
\put(1000,563){\makebox(0,0){$\times$}}
\put(1000,562){\makebox(0,0){$\times$}}
\put(1000,561){\makebox(0,0){$\times$}}
\put(1001,560){\makebox(0,0){$\times$}}
\put(1001,560){\makebox(0,0){$\times$}}
\put(1001,559){\makebox(0,0){$\times$}}
\put(1002,559){\makebox(0,0){$\times$}}
\put(1002,558){\makebox(0,0){$\times$}}
\put(1003,557){\makebox(0,0){$\times$}}
\put(1003,556){\makebox(0,0){$\times$}}
\put(1003,556){\makebox(0,0){$\times$}}
\put(1004,555){\makebox(0,0){$\times$}}
\put(1004,554){\makebox(0,0){$\times$}}
\put(1004,553){\makebox(0,0){$\times$}}
\put(1005,553){\makebox(0,0){$\times$}}
\put(1005,552){\makebox(0,0){$\times$}}
\put(1005,551){\makebox(0,0){$\times$}}
\put(1006,550){\makebox(0,0){$\times$}}
\put(1006,550){\makebox(0,0){$\times$}}
\put(1007,549){\makebox(0,0){$\times$}}
\put(1007,548){\makebox(0,0){$\times$}}
\put(1007,548){\makebox(0,0){$\times$}}
\put(1008,547){\makebox(0,0){$\times$}}
\put(1008,546){\makebox(0,0){$\times$}}
\put(1008,545){\makebox(0,0){$\times$}}
\put(1009,545){\makebox(0,0){$\times$}}
\put(1009,544){\makebox(0,0){$\times$}}
\put(1010,543){\makebox(0,0){$\times$}}
\put(1010,543){\makebox(0,0){$\times$}}
\put(1010,542){\makebox(0,0){$\times$}}
\put(1011,541){\makebox(0,0){$\times$}}
\put(1011,541){\makebox(0,0){$\times$}}
\put(1012,539){\makebox(0,0){$\times$}}
\put(1012,538){\makebox(0,0){$\times$}}
\put(1012,538){\makebox(0,0){$\times$}}
\put(1013,537){\makebox(0,0){$\times$}}
\put(1013,537){\makebox(0,0){$\times$}}
\put(1014,536){\makebox(0,0){$\times$}}
\put(1014,535){\makebox(0,0){$\times$}}
\put(1014,535){\makebox(0,0){$\times$}}
\put(1015,534){\makebox(0,0){$\times$}}
\put(1015,534){\makebox(0,0){$\times$}}
\put(1015,533){\makebox(0,0){$\times$}}
\put(1016,532){\makebox(0,0){$\times$}}
\put(1016,532){\makebox(0,0){$\times$}}
\put(1017,531){\makebox(0,0){$\times$}}
\put(1017,531){\makebox(0,0){$\times$}}
\put(1017,530){\makebox(0,0){$\times$}}
\put(1018,529){\makebox(0,0){$\times$}}
\put(1018,529){\makebox(0,0){$\times$}}
\put(1018,528){\makebox(0,0){$\times$}}
\put(1019,528){\makebox(0,0){$\times$}}
\put(1019,527){\makebox(0,0){$\times$}}
\put(1019,527){\makebox(0,0){$\times$}}
\put(1020,526){\makebox(0,0){$\times$}}
\put(1020,526){\makebox(0,0){$\times$}}
\put(1021,525){\makebox(0,0){$\times$}}
\put(1021,525){\makebox(0,0){$\times$}}
\put(1021,524){\makebox(0,0){$\times$}}
\put(1022,524){\makebox(0,0){$\times$}}
\put(1022,523){\makebox(0,0){$\times$}}
\put(1022,523){\makebox(0,0){$\times$}}
\put(1023,523){\makebox(0,0){$\times$}}
\put(1023,522){\makebox(0,0){$\times$}}
\put(1024,522){\makebox(0,0){$\times$}}
\put(1024,522){\makebox(0,0){$\times$}}
\put(1024,521){\makebox(0,0){$\times$}}
\put(1025,521){\makebox(0,0){$\times$}}
\put(1025,520){\makebox(0,0){$\times$}}
\put(1025,520){\makebox(0,0){$\times$}}
\put(1026,520){\makebox(0,0){$\times$}}
\put(1026,520){\makebox(0,0){$\times$}}
\put(1026,519){\makebox(0,0){$\times$}}
\put(1027,519){\makebox(0,0){$\times$}}
\put(1027,519){\makebox(0,0){$\times$}}
\put(1028,519){\makebox(0,0){$\times$}}
\put(1028,518){\makebox(0,0){$\times$}}
\put(1028,518){\makebox(0,0){$\times$}}
\put(1029,518){\makebox(0,0){$\times$}}
\put(1029,518){\makebox(0,0){$\times$}}
\put(1029,518){\makebox(0,0){$\times$}}
\put(1030,517){\makebox(0,0){$\times$}}
\put(1030,517){\makebox(0,0){$\times$}}
\put(1031,517){\makebox(0,0){$\times$}}
\put(1031,517){\makebox(0,0){$\times$}}
\put(1031,517){\makebox(0,0){$\times$}}
\put(1032,517){\makebox(0,0){$\times$}}
\put(1032,517){\makebox(0,0){$\times$}}
\put(1032,517){\makebox(0,0){$\times$}}
\put(1033,517){\makebox(0,0){$\times$}}
\put(1033,517){\makebox(0,0){$\times$}}
\put(1033,517){\makebox(0,0){$\times$}}
\put(1034,517){\makebox(0,0){$\times$}}
\put(1034,517){\makebox(0,0){$\times$}}
\put(1035,517){\makebox(0,0){$\times$}}
\put(1035,517){\makebox(0,0){$\times$}}
\put(1035,517){\makebox(0,0){$\times$}}
\put(1036,517){\makebox(0,0){$\times$}}
\put(1036,517){\makebox(0,0){$\times$}}
\put(1036,517){\makebox(0,0){$\times$}}
\put(1037,517){\makebox(0,0){$\times$}}
\put(1037,517){\makebox(0,0){$\times$}}
\put(1038,517){\makebox(0,0){$\times$}}
\put(1038,517){\makebox(0,0){$\times$}}
\put(1038,517){\makebox(0,0){$\times$}}
\put(1039,517){\makebox(0,0){$\times$}}
\put(1039,517){\makebox(0,0){$\times$}}
\put(1039,517){\makebox(0,0){$\times$}}
\put(1040,518){\makebox(0,0){$\times$}}
\put(1040,518){\makebox(0,0){$\times$}}
\put(1040,518){\makebox(0,0){$\times$}}
\put(1041,518){\makebox(0,0){$\times$}}
\put(1041,519){\makebox(0,0){$\times$}}
\put(1042,519){\makebox(0,0){$\times$}}
\put(1042,519){\makebox(0,0){$\times$}}
\put(1042,519){\makebox(0,0){$\times$}}
\put(1043,519){\makebox(0,0){$\times$}}
\put(1043,520){\makebox(0,0){$\times$}}
\put(1043,520){\makebox(0,0){$\times$}}
\put(1044,520){\makebox(0,0){$\times$}}
\put(1044,520){\makebox(0,0){$\times$}}
\put(1045,521){\makebox(0,0){$\times$}}
\put(1045,521){\makebox(0,0){$\times$}}
\put(1045,521){\makebox(0,0){$\times$}}
\put(1046,522){\makebox(0,0){$\times$}}
\put(1046,522){\makebox(0,0){$\times$}}
\put(1046,522){\makebox(0,0){$\times$}}
\put(1047,523){\makebox(0,0){$\times$}}
\put(1047,523){\makebox(0,0){$\times$}}
\put(1047,523){\makebox(0,0){$\times$}}
\put(1048,524){\makebox(0,0){$\times$}}
\put(1048,524){\makebox(0,0){$\times$}}
\put(1049,525){\makebox(0,0){$\times$}}
\put(1049,525){\makebox(0,0){$\times$}}
\put(1049,525){\makebox(0,0){$\times$}}
\put(1050,526){\makebox(0,0){$\times$}}
\put(1050,526){\makebox(0,0){$\times$}}
\put(1050,527){\makebox(0,0){$\times$}}
\put(1051,527){\makebox(0,0){$\times$}}
\put(1051,528){\makebox(0,0){$\times$}}
\put(1051,528){\makebox(0,0){$\times$}}
\put(1052,529){\makebox(0,0){$\times$}}
\put(1052,529){\makebox(0,0){$\times$}}
\put(1053,529){\makebox(0,0){$\times$}}
\put(1053,530){\makebox(0,0){$\times$}}
\put(1053,531){\makebox(0,0){$\times$}}
\put(1054,531){\makebox(0,0){$\times$}}
\put(1054,531){\makebox(0,0){$\times$}}
\put(1054,532){\makebox(0,0){$\times$}}
\put(1055,532){\makebox(0,0){$\times$}}
\put(1055,533){\makebox(0,0){$\times$}}
\put(1056,534){\makebox(0,0){$\times$}}
\put(1056,534){\makebox(0,0){$\times$}}
\put(1056,535){\makebox(0,0){$\times$}}
\put(1057,535){\makebox(0,0){$\times$}}
\put(1057,536){\makebox(0,0){$\times$}}
\put(1057,536){\makebox(0,0){$\times$}}
\put(1058,537){\makebox(0,0){$\times$}}
\put(1058,537){\makebox(0,0){$\times$}}
\put(1058,538){\makebox(0,0){$\times$}}
\put(1059,538){\makebox(0,0){$\times$}}
\put(1060,539){\makebox(0,0){$\times$}}
\put(1060,541){\makebox(0,0){$\times$}}
\put(1061,541){\makebox(0,0){$\times$}}
\put(1061,542){\makebox(0,0){$\times$}}
\put(1062,543){\makebox(0,0){$\times$}}
\put(1062,543){\makebox(0,0){$\times$}}
\put(1063,544){\makebox(0,0){$\times$}}
\put(1063,545){\makebox(0,0){$\times$}}
\put(1063,545){\makebox(0,0){$\times$}}
\put(1064,546){\makebox(0,0){$\times$}}
\put(1064,546){\makebox(0,0){$\times$}}
\put(1064,547){\makebox(0,0){$\times$}}
\put(1065,547){\makebox(0,0){$\times$}}
\put(1065,548){\makebox(0,0){$\times$}}
\put(1065,548){\makebox(0,0){$\times$}}
\put(1066,549){\makebox(0,0){$\times$}}
\put(1066,550){\makebox(0,0){$\times$}}
\put(1067,550){\makebox(0,0){$\times$}}
\put(1067,551){\makebox(0,0){$\times$}}
\put(1067,551){\makebox(0,0){$\times$}}
\put(1068,552){\makebox(0,0){$\times$}}
\put(1068,552){\makebox(0,0){$\times$}}
\put(1068,553){\makebox(0,0){$\times$}}
\put(1069,553){\makebox(0,0){$\times$}}
\put(1069,554){\makebox(0,0){$\times$}}
\put(1070,554){\makebox(0,0){$\times$}}
\put(1070,555){\makebox(0,0){$\times$}}
\put(1070,555){\makebox(0,0){$\times$}}
\put(1071,556){\makebox(0,0){$\times$}}
\put(1071,556){\makebox(0,0){$\times$}}
\put(1071,557){\makebox(0,0){$\times$}}
\put(1072,557){\makebox(0,0){$\times$}}
\put(1072,558){\makebox(0,0){$\times$}}
\put(1072,558){\makebox(0,0){$\times$}}
\put(1073,559){\makebox(0,0){$\times$}}
\put(1073,559){\makebox(0,0){$\times$}}
\put(1074,560){\makebox(0,0){$\times$}}
\put(1074,560){\makebox(0,0){$\times$}}
\put(1074,560){\makebox(0,0){$\times$}}
\put(1075,561){\makebox(0,0){$\times$}}
\put(1075,562){\makebox(0,0){$\times$}}
\put(1075,562){\makebox(0,0){$\times$}}
\put(1076,562){\makebox(0,0){$\times$}}
\put(1076,563){\makebox(0,0){$\times$}}
\put(1077,563){\makebox(0,0){$\times$}}
\put(1077,563){\makebox(0,0){$\times$}}
\put(1077,564){\makebox(0,0){$\times$}}
\put(1078,564){\makebox(0,0){$\times$}}
\put(1078,565){\makebox(0,0){$\times$}}
\put(1078,565){\makebox(0,0){$\times$}}
\put(1079,565){\makebox(0,0){$\times$}}
\put(1079,566){\makebox(0,0){$\times$}}
\put(1079,566){\makebox(0,0){$\times$}}
\put(1080,566){\makebox(0,0){$\times$}}
\put(1080,567){\makebox(0,0){$\times$}}
\put(1081,567){\makebox(0,0){$\times$}}
\put(1081,568){\makebox(0,0){$\times$}}
\put(1081,568){\makebox(0,0){$\times$}}
\put(1082,568){\makebox(0,0){$\times$}}
\put(1082,569){\makebox(0,0){$\times$}}
\put(1082,569){\makebox(0,0){$\times$}}
\put(1083,569){\makebox(0,0){$\times$}}
\put(1083,569){\makebox(0,0){$\times$}}
\put(1084,570){\makebox(0,0){$\times$}}
\put(1084,570){\makebox(0,0){$\times$}}
\put(1084,570){\makebox(0,0){$\times$}}
\put(1085,570){\makebox(0,0){$\times$}}
\put(1085,571){\makebox(0,0){$\times$}}
\put(1085,571){\makebox(0,0){$\times$}}
\put(1086,571){\makebox(0,0){$\times$}}
\put(1086,571){\makebox(0,0){$\times$}}
\put(1086,571){\makebox(0,0){$\times$}}
\put(1087,572){\makebox(0,0){$\times$}}
\put(1087,572){\makebox(0,0){$\times$}}
\put(1088,572){\makebox(0,0){$\times$}}
\put(1088,572){\makebox(0,0){$\times$}}
\put(1088,572){\makebox(0,0){$\times$}}
\put(1089,572){\makebox(0,0){$\times$}}
\put(1089,572){\makebox(0,0){$\times$}}
\put(1089,573){\makebox(0,0){$\times$}}
\put(1090,573){\makebox(0,0){$\times$}}
\put(1090,573){\makebox(0,0){$\times$}}
\put(1091,573){\makebox(0,0){$\times$}}
\put(1091,573){\makebox(0,0){$\times$}}
\put(1091,573){\makebox(0,0){$\times$}}
\put(1092,573){\makebox(0,0){$\times$}}
\put(1092,573){\makebox(0,0){$\times$}}
\put(1092,573){\makebox(0,0){$\times$}}
\put(1093,573){\makebox(0,0){$\times$}}
\put(1093,573){\makebox(0,0){$\times$}}
\put(1093,573){\makebox(0,0){$\times$}}
\put(1094,573){\makebox(0,0){$\times$}}
\put(1094,573){\makebox(0,0){$\times$}}
\put(1095,573){\makebox(0,0){$\times$}}
\put(1095,573){\makebox(0,0){$\times$}}
\put(1095,573){\makebox(0,0){$\times$}}
\put(1096,573){\makebox(0,0){$\times$}}
\put(1096,573){\makebox(0,0){$\times$}}
\put(1096,573){\makebox(0,0){$\times$}}
\put(1097,573){\makebox(0,0){$\times$}}
\put(1097,573){\makebox(0,0){$\times$}}
\put(1097,573){\makebox(0,0){$\times$}}
\put(1098,573){\makebox(0,0){$\times$}}
\put(1098,573){\makebox(0,0){$\times$}}
\put(1099,572){\makebox(0,0){$\times$}}
\put(1099,572){\makebox(0,0){$\times$}}
\put(1099,572){\makebox(0,0){$\times$}}
\put(1100,572){\makebox(0,0){$\times$}}
\put(1100,572){\makebox(0,0){$\times$}}
\put(1100,572){\makebox(0,0){$\times$}}
\put(1101,572){\makebox(0,0){$\times$}}
\put(1101,571){\makebox(0,0){$\times$}}
\put(1102,571){\makebox(0,0){$\times$}}
\put(1102,571){\makebox(0,0){$\times$}}
\put(1102,571){\makebox(0,0){$\times$}}
\put(1103,571){\makebox(0,0){$\times$}}
\put(1103,570){\makebox(0,0){$\times$}}
\put(1103,570){\makebox(0,0){$\times$}}
\put(1104,570){\makebox(0,0){$\times$}}
\put(1104,570){\makebox(0,0){$\times$}}
\put(1104,569){\makebox(0,0){$\times$}}
\put(1105,569){\makebox(0,0){$\times$}}
\put(1105,569){\makebox(0,0){$\times$}}
\put(1106,569){\makebox(0,0){$\times$}}
\put(1106,568){\makebox(0,0){$\times$}}
\put(1106,568){\makebox(0,0){$\times$}}
\put(1107,568){\makebox(0,0){$\times$}}
\put(1107,567){\makebox(0,0){$\times$}}
\put(1107,567){\makebox(0,0){$\times$}}
\put(1108,567){\makebox(0,0){$\times$}}
\put(1108,566){\makebox(0,0){$\times$}}
\put(1109,566){\makebox(0,0){$\times$}}
\put(1109,566){\makebox(0,0){$\times$}}
\put(1109,565){\makebox(0,0){$\times$}}
\put(1110,565){\makebox(0,0){$\times$}}
\put(1110,565){\makebox(0,0){$\times$}}
\put(1110,564){\makebox(0,0){$\times$}}
\put(1111,564){\makebox(0,0){$\times$}}
\put(1111,564){\makebox(0,0){$\times$}}
\put(1111,563){\makebox(0,0){$\times$}}
\put(1112,563){\makebox(0,0){$\times$}}
\put(1112,563){\makebox(0,0){$\times$}}
\put(1113,562){\makebox(0,0){$\times$}}
\put(1113,562){\makebox(0,0){$\times$}}
\put(1113,562){\makebox(0,0){$\times$}}
\put(1114,561){\makebox(0,0){$\times$}}
\put(1114,561){\makebox(0,0){$\times$}}
\put(1114,560){\makebox(0,0){$\times$}}
\put(1115,560){\makebox(0,0){$\times$}}
\put(1115,559){\makebox(0,0){$\times$}}
\put(1116,559){\makebox(0,0){$\times$}}
\put(1116,559){\makebox(0,0){$\times$}}
\put(1116,558){\makebox(0,0){$\times$}}
\put(1117,558){\makebox(0,0){$\times$}}
\put(1117,557){\makebox(0,0){$\times$}}
\put(1117,557){\makebox(0,0){$\times$}}
\put(1118,557){\makebox(0,0){$\times$}}
\put(1118,556){\makebox(0,0){$\times$}}
\put(1118,556){\makebox(0,0){$\times$}}
\put(1119,556){\makebox(0,0){$\times$}}
\put(1119,555){\makebox(0,0){$\times$}}
\put(1120,554){\makebox(0,0){$\times$}}
\put(1120,554){\makebox(0,0){$\times$}}
\put(1120,554){\makebox(0,0){$\times$}}
\put(1121,553){\makebox(0,0){$\times$}}
\put(1121,553){\makebox(0,0){$\times$}}
\put(1121,553){\makebox(0,0){$\times$}}
\put(1122,552){\makebox(0,0){$\times$}}
\put(1122,552){\makebox(0,0){$\times$}}
\put(1123,551){\makebox(0,0){$\times$}}
\put(1123,551){\makebox(0,0){$\times$}}
\put(1123,550){\makebox(0,0){$\times$}}
\put(1124,550){\makebox(0,0){$\times$}}
\put(1124,550){\makebox(0,0){$\times$}}
\put(1124,549){\makebox(0,0){$\times$}}
\put(1125,549){\makebox(0,0){$\times$}}
\put(1125,548){\makebox(0,0){$\times$}}
\put(1125,548){\makebox(0,0){$\times$}}
\put(1126,547){\makebox(0,0){$\times$}}
\put(1126,547){\makebox(0,0){$\times$}}
\put(1127,547){\makebox(0,0){$\times$}}
\put(1127,546){\makebox(0,0){$\times$}}
\put(1127,546){\makebox(0,0){$\times$}}
\put(1128,545){\makebox(0,0){$\times$}}
\put(1128,545){\makebox(0,0){$\times$}}
\put(1128,544){\makebox(0,0){$\times$}}
\put(1129,544){\makebox(0,0){$\times$}}
\put(1129,544){\makebox(0,0){$\times$}}
\put(1130,543){\makebox(0,0){$\times$}}
\put(1130,543){\makebox(0,0){$\times$}}
\put(1130,542){\makebox(0,0){$\times$}}
\put(1131,542){\makebox(0,0){$\times$}}
\put(1131,541){\makebox(0,0){$\times$}}
\put(1132,541){\makebox(0,0){$\times$}}
\put(1132,541){\makebox(0,0){$\times$}}
\put(1132,540){\makebox(0,0){$\times$}}
\put(1133,539){\makebox(0,0){$\times$}}
\put(1134,539){\makebox(0,0){$\times$}}
\put(1134,539){\makebox(0,0){$\times$}}
\put(1134,538){\makebox(0,0){$\times$}}
\put(1135,538){\makebox(0,0){$\times$}}
\put(1135,538){\makebox(0,0){$\times$}}
\put(1135,537){\makebox(0,0){$\times$}}
\put(1136,537){\makebox(0,0){$\times$}}
\put(1136,537){\makebox(0,0){$\times$}}
\put(1137,537){\makebox(0,0){$\times$}}
\put(1137,536){\makebox(0,0){$\times$}}
\put(1137,536){\makebox(0,0){$\times$}}
\put(1138,535){\makebox(0,0){$\times$}}
\put(1138,535){\makebox(0,0){$\times$}}
\put(1138,535){\makebox(0,0){$\times$}}
\put(1139,535){\makebox(0,0){$\times$}}
\put(1139,534){\makebox(0,0){$\times$}}
\put(1139,534){\makebox(0,0){$\times$}}
\put(1140,534){\makebox(0,0){$\times$}}
\put(1140,534){\makebox(0,0){$\times$}}
\put(1141,533){\makebox(0,0){$\times$}}
\put(1141,533){\makebox(0,0){$\times$}}
\put(1141,533){\makebox(0,0){$\times$}}
\put(1142,533){\makebox(0,0){$\times$}}
\put(1142,532){\makebox(0,0){$\times$}}
\put(1142,532){\makebox(0,0){$\times$}}
\put(1143,532){\makebox(0,0){$\times$}}
\put(1143,532){\makebox(0,0){$\times$}}
\put(1143,532){\makebox(0,0){$\times$}}
\put(1144,532){\makebox(0,0){$\times$}}
\put(1144,531){\makebox(0,0){$\times$}}
\put(1145,531){\makebox(0,0){$\times$}}
\put(1145,531){\makebox(0,0){$\times$}}
\put(1145,531){\makebox(0,0){$\times$}}
\put(1146,531){\makebox(0,0){$\times$}}
\put(1146,531){\makebox(0,0){$\times$}}
\put(1146,531){\makebox(0,0){$\times$}}
\put(1147,531){\makebox(0,0){$\times$}}
\put(1147,531){\makebox(0,0){$\times$}}
\put(1148,530){\makebox(0,0){$\times$}}
\put(1148,530){\makebox(0,0){$\times$}}
\put(1148,530){\makebox(0,0){$\times$}}
\put(1149,530){\makebox(0,0){$\times$}}
\put(1149,530){\makebox(0,0){$\times$}}
\put(1149,530){\makebox(0,0){$\times$}}
\put(1150,530){\makebox(0,0){$\times$}}
\put(1150,530){\makebox(0,0){$\times$}}
\put(1150,530){\makebox(0,0){$\times$}}
\put(1151,530){\makebox(0,0){$\times$}}
\put(1151,530){\makebox(0,0){$\times$}}
\put(1152,530){\makebox(0,0){$\times$}}
\put(1152,530){\makebox(0,0){$\times$}}
\put(1152,530){\makebox(0,0){$\times$}}
\put(1153,530){\makebox(0,0){$\times$}}
\put(1153,530){\makebox(0,0){$\times$}}
\put(1153,530){\makebox(0,0){$\times$}}
\put(1154,530){\makebox(0,0){$\times$}}
\put(1154,530){\makebox(0,0){$\times$}}
\put(1155,530){\makebox(0,0){$\times$}}
\put(1155,530){\makebox(0,0){$\times$}}
\put(1155,530){\makebox(0,0){$\times$}}
\put(1156,530){\makebox(0,0){$\times$}}
\put(1156,530){\makebox(0,0){$\times$}}
\put(1156,531){\makebox(0,0){$\times$}}
\put(1157,531){\makebox(0,0){$\times$}}
\put(1157,531){\makebox(0,0){$\times$}}
\put(1157,531){\makebox(0,0){$\times$}}
\put(1158,531){\makebox(0,0){$\times$}}
\put(1158,531){\makebox(0,0){$\times$}}
\put(1159,531){\makebox(0,0){$\times$}}
\put(1159,531){\makebox(0,0){$\times$}}
\put(1159,531){\makebox(0,0){$\times$}}
\put(1160,532){\makebox(0,0){$\times$}}
\put(1160,532){\makebox(0,0){$\times$}}
\put(1160,532){\makebox(0,0){$\times$}}
\put(1161,532){\makebox(0,0){$\times$}}
\put(1161,532){\makebox(0,0){$\times$}}
\put(1162,532){\makebox(0,0){$\times$}}
\put(1162,532){\makebox(0,0){$\times$}}
\put(1162,533){\makebox(0,0){$\times$}}
\put(1163,533){\makebox(0,0){$\times$}}
\put(1163,533){\makebox(0,0){$\times$}}
\put(1163,533){\makebox(0,0){$\times$}}
\put(1164,534){\makebox(0,0){$\times$}}
\put(1164,534){\makebox(0,0){$\times$}}
\put(1164,534){\makebox(0,0){$\times$}}
\put(1165,534){\makebox(0,0){$\times$}}
\put(1165,534){\makebox(0,0){$\times$}}
\put(1166,535){\makebox(0,0){$\times$}}
\put(1166,535){\makebox(0,0){$\times$}}
\put(1166,535){\makebox(0,0){$\times$}}
\put(1167,535){\makebox(0,0){$\times$}}
\put(1167,535){\makebox(0,0){$\times$}}
\put(1167,536){\makebox(0,0){$\times$}}
\put(1168,536){\makebox(0,0){$\times$}}
\put(1168,536){\makebox(0,0){$\times$}}
\put(1169,537){\makebox(0,0){$\times$}}
\put(1169,537){\makebox(0,0){$\times$}}
\put(1169,537){\makebox(0,0){$\times$}}
\put(1170,537){\makebox(0,0){$\times$}}
\put(1170,538){\makebox(0,0){$\times$}}
\put(1170,538){\makebox(0,0){$\times$}}
\put(1171,538){\makebox(0,0){$\times$}}
\put(1171,538){\makebox(0,0){$\times$}}
\put(1171,539){\makebox(0,0){$\times$}}
\put(1172,539){\makebox(0,0){$\times$}}
\put(1173,539){\makebox(0,0){$\times$}}
\put(1174,540){\makebox(0,0){$\times$}}
\put(1174,541){\makebox(0,0){$\times$}}
\put(1174,541){\makebox(0,0){$\times$}}
\put(1175,541){\makebox(0,0){$\times$}}
\put(1175,542){\makebox(0,0){$\times$}}
\put(1176,542){\makebox(0,0){$\times$}}
\put(1176,542){\makebox(0,0){$\times$}}
\put(1177,543){\makebox(0,0){$\times$}}
\put(1177,543){\makebox(0,0){$\times$}}
\put(1177,544){\makebox(0,0){$\times$}}
\put(1178,544){\makebox(0,0){$\times$}}
\put(1178,544){\makebox(0,0){$\times$}}
\put(1178,544){\makebox(0,0){$\times$}}
\put(1179,545){\makebox(0,0){$\times$}}
\put(1179,545){\makebox(0,0){$\times$}}
\put(1180,545){\makebox(0,0){$\times$}}
\put(1180,546){\makebox(0,0){$\times$}}
\put(1180,546){\makebox(0,0){$\times$}}
\put(1181,546){\makebox(0,0){$\times$}}
\put(1181,547){\makebox(0,0){$\times$}}
\put(1181,547){\makebox(0,0){$\times$}}
\put(1182,547){\makebox(0,0){$\times$}}
\put(1182,548){\makebox(0,0){$\times$}}
\put(1183,548){\makebox(0,0){$\times$}}
\put(1183,548){\makebox(0,0){$\times$}}
\put(1183,548){\makebox(0,0){$\times$}}
\put(1184,549){\makebox(0,0){$\times$}}
\put(1184,549){\makebox(0,0){$\times$}}
\put(1184,550){\makebox(0,0){$\times$}}
\put(1185,550){\makebox(0,0){$\times$}}
\put(1185,550){\makebox(0,0){$\times$}}
\put(1185,550){\makebox(0,0){$\times$}}
\put(1186,551){\makebox(0,0){$\times$}}
\put(1186,551){\makebox(0,0){$\times$}}
\put(1187,551){\makebox(0,0){$\times$}}
\put(1187,552){\makebox(0,0){$\times$}}
\put(1187,552){\makebox(0,0){$\times$}}
\put(1188,552){\makebox(0,0){$\times$}}
\put(1188,553){\makebox(0,0){$\times$}}
\put(1188,553){\makebox(0,0){$\times$}}
\put(1189,553){\makebox(0,0){$\times$}}
\put(1189,553){\makebox(0,0){$\times$}}
\put(1189,554){\makebox(0,0){$\times$}}
\put(1190,554){\makebox(0,0){$\times$}}
\put(1190,554){\makebox(0,0){$\times$}}
\put(1191,554){\makebox(0,0){$\times$}}
\put(1191,555){\makebox(0,0){$\times$}}
\put(1191,555){\makebox(0,0){$\times$}}
\put(1192,555){\makebox(0,0){$\times$}}
\put(1192,556){\makebox(0,0){$\times$}}
\put(1192,556){\makebox(0,0){$\times$}}
\put(1193,556){\makebox(0,0){$\times$}}
\put(1193,556){\makebox(0,0){$\times$}}
\put(1194,556){\makebox(0,0){$\times$}}
\put(1194,557){\makebox(0,0){$\times$}}
\put(1194,557){\makebox(0,0){$\times$}}
\put(1195,557){\makebox(0,0){$\times$}}
\put(1195,557){\makebox(0,0){$\times$}}
\put(1195,558){\makebox(0,0){$\times$}}
\put(1196,558){\makebox(0,0){$\times$}}
\put(1196,558){\makebox(0,0){$\times$}}
\put(1196,558){\makebox(0,0){$\times$}}
\put(1197,558){\makebox(0,0){$\times$}}
\put(1197,559){\makebox(0,0){$\times$}}
\put(1198,559){\makebox(0,0){$\times$}}
\put(1198,559){\makebox(0,0){$\times$}}
\put(1198,559){\makebox(0,0){$\times$}}
\put(1199,559){\makebox(0,0){$\times$}}
\put(1199,559){\makebox(0,0){$\times$}}
\put(1199,560){\makebox(0,0){$\times$}}
\put(1200,560){\makebox(0,0){$\times$}}
\put(1200,560){\makebox(0,0){$\times$}}
\put(1201,560){\makebox(0,0){$\times$}}
\put(1201,560){\makebox(0,0){$\times$}}
\put(1201,560){\makebox(0,0){$\times$}}
\put(1202,560){\makebox(0,0){$\times$}}
\put(1202,561){\makebox(0,0){$\times$}}
\put(1202,561){\makebox(0,0){$\times$}}
\put(1203,561){\makebox(0,0){$\times$}}
\put(1203,561){\makebox(0,0){$\times$}}
\put(1203,561){\makebox(0,0){$\times$}}
\put(1204,561){\makebox(0,0){$\times$}}
\put(1204,561){\makebox(0,0){$\times$}}
\put(1205,561){\makebox(0,0){$\times$}}
\put(1205,562){\makebox(0,0){$\times$}}
\put(1205,562){\makebox(0,0){$\times$}}
\put(1206,562){\makebox(0,0){$\times$}}
\put(1206,562){\makebox(0,0){$\times$}}
\put(1206,562){\makebox(0,0){$\times$}}
\put(1207,562){\makebox(0,0){$\times$}}
\put(1207,562){\makebox(0,0){$\times$}}
\put(1208,562){\makebox(0,0){$\times$}}
\put(1208,562){\makebox(0,0){$\times$}}
\put(1208,562){\makebox(0,0){$\times$}}
\put(1209,562){\makebox(0,0){$\times$}}
\put(1209,562){\makebox(0,0){$\times$}}
\put(1209,562){\makebox(0,0){$\times$}}
\put(1210,562){\makebox(0,0){$\times$}}
\put(1210,562){\makebox(0,0){$\times$}}
\put(1210,562){\makebox(0,0){$\times$}}
\put(1211,562){\makebox(0,0){$\times$}}
\put(1211,562){\makebox(0,0){$\times$}}
\put(1212,562){\makebox(0,0){$\times$}}
\put(1212,562){\makebox(0,0){$\times$}}
\put(1212,562){\makebox(0,0){$\times$}}
\put(1213,562){\makebox(0,0){$\times$}}
\put(1213,562){\makebox(0,0){$\times$}}
\put(1213,562){\makebox(0,0){$\times$}}
\put(1214,562){\makebox(0,0){$\times$}}
\put(1214,562){\makebox(0,0){$\times$}}
\put(1215,562){\makebox(0,0){$\times$}}
\put(1215,562){\makebox(0,0){$\times$}}
\put(1215,562){\makebox(0,0){$\times$}}
\put(1216,562){\makebox(0,0){$\times$}}
\put(1216,562){\makebox(0,0){$\times$}}
\put(1216,562){\makebox(0,0){$\times$}}
\put(1217,562){\makebox(0,0){$\times$}}
\put(1217,561){\makebox(0,0){$\times$}}
\put(1217,561){\makebox(0,0){$\times$}}
\put(1218,561){\makebox(0,0){$\times$}}
\put(1218,561){\makebox(0,0){$\times$}}
\put(1219,561){\makebox(0,0){$\times$}}
\put(1219,561){\makebox(0,0){$\times$}}
\put(1219,561){\makebox(0,0){$\times$}}
\put(1220,560){\makebox(0,0){$\times$}}
\put(1220,560){\makebox(0,0){$\times$}}
\put(1220,560){\makebox(0,0){$\times$}}
\put(1221,560){\makebox(0,0){$\times$}}
\put(1221,560){\makebox(0,0){$\times$}}
\put(1222,560){\makebox(0,0){$\times$}}
\put(1222,560){\makebox(0,0){$\times$}}
\put(1222,560){\makebox(0,0){$\times$}}
\put(1223,559){\makebox(0,0){$\times$}}
\put(1223,559){\makebox(0,0){$\times$}}
\put(1223,559){\makebox(0,0){$\times$}}
\put(1224,559){\makebox(0,0){$\times$}}
\put(1224,559){\makebox(0,0){$\times$}}
\put(1224,559){\makebox(0,0){$\times$}}
\put(1225,558){\makebox(0,0){$\times$}}
\put(1225,558){\makebox(0,0){$\times$}}
\put(1226,558){\makebox(0,0){$\times$}}
\put(1226,558){\makebox(0,0){$\times$}}
\put(1226,558){\makebox(0,0){$\times$}}
\put(1227,557){\makebox(0,0){$\times$}}
\put(1227,557){\makebox(0,0){$\times$}}
\put(1227,557){\makebox(0,0){$\times$}}
\put(1228,557){\makebox(0,0){$\times$}}
\put(1228,557){\makebox(0,0){$\times$}}
\put(1229,556){\makebox(0,0){$\times$}}
\put(1229,556){\makebox(0,0){$\times$}}
\put(1229,556){\makebox(0,0){$\times$}}
\put(1230,556){\makebox(0,0){$\times$}}
\put(1230,556){\makebox(0,0){$\times$}}
\put(1230,555){\makebox(0,0){$\times$}}
\put(1231,555){\makebox(0,0){$\times$}}
\put(1231,555){\makebox(0,0){$\times$}}
\put(1231,555){\makebox(0,0){$\times$}}
\put(1232,554){\makebox(0,0){$\times$}}
\put(1232,554){\makebox(0,0){$\times$}}
\put(1233,554){\makebox(0,0){$\times$}}
\put(1233,554){\makebox(0,0){$\times$}}
\put(1233,554){\makebox(0,0){$\times$}}
\put(1234,553){\makebox(0,0){$\times$}}
\put(1234,553){\makebox(0,0){$\times$}}
\put(1234,553){\makebox(0,0){$\times$}}
\put(1235,553){\makebox(0,0){$\times$}}
\put(1235,552){\makebox(0,0){$\times$}}
\put(1235,552){\makebox(0,0){$\times$}}
\put(1236,552){\makebox(0,0){$\times$}}
\put(1236,552){\makebox(0,0){$\times$}}
\put(1237,551){\makebox(0,0){$\times$}}
\put(1237,551){\makebox(0,0){$\times$}}
\put(1237,551){\makebox(0,0){$\times$}}
\put(1238,551){\makebox(0,0){$\times$}}
\put(1238,551){\makebox(0,0){$\times$}}
\put(1238,550){\makebox(0,0){$\times$}}
\put(1239,550){\makebox(0,0){$\times$}}
\put(1239,550){\makebox(0,0){$\times$}}
\put(1240,550){\makebox(0,0){$\times$}}
\put(1240,550){\makebox(0,0){$\times$}}
\put(1240,549){\makebox(0,0){$\times$}}
\put(1241,549){\makebox(0,0){$\times$}}
\put(1241,549){\makebox(0,0){$\times$}}
\put(1241,548){\makebox(0,0){$\times$}}
\put(1242,548){\makebox(0,0){$\times$}}
\put(1242,548){\makebox(0,0){$\times$}}
\put(1242,548){\makebox(0,0){$\times$}}
\put(1243,548){\makebox(0,0){$\times$}}
\put(1243,548){\makebox(0,0){$\times$}}
\put(1244,547){\makebox(0,0){$\times$}}
\put(1244,547){\makebox(0,0){$\times$}}
\put(1244,547){\makebox(0,0){$\times$}}
\put(1245,547){\makebox(0,0){$\times$}}
\put(1245,547){\makebox(0,0){$\times$}}
\put(1245,546){\makebox(0,0){$\times$}}
\put(1246,546){\makebox(0,0){$\times$}}
\put(1246,546){\makebox(0,0){$\times$}}
\put(1247,546){\makebox(0,0){$\times$}}
\put(1247,545){\makebox(0,0){$\times$}}
\put(1247,545){\makebox(0,0){$\times$}}
\put(1248,545){\makebox(0,0){$\times$}}
\put(1248,545){\makebox(0,0){$\times$}}
\put(1248,545){\makebox(0,0){$\times$}}
\put(1249,545){\makebox(0,0){$\times$}}
\put(1249,544){\makebox(0,0){$\times$}}
\put(1249,544){\makebox(0,0){$\times$}}
\put(1250,544){\makebox(0,0){$\times$}}
\put(1250,544){\makebox(0,0){$\times$}}
\put(1251,544){\makebox(0,0){$\times$}}
\put(1251,544){\makebox(0,0){$\times$}}
\put(1251,544){\makebox(0,0){$\times$}}
\put(1252,543){\makebox(0,0){$\times$}}
\put(1252,543){\makebox(0,0){$\times$}}
\put(1252,543){\makebox(0,0){$\times$}}
\put(1253,543){\makebox(0,0){$\times$}}
\put(1253,543){\makebox(0,0){$\times$}}
\put(1254,542){\makebox(0,0){$\times$}}
\put(1254,542){\makebox(0,0){$\times$}}
\put(1255,542){\makebox(0,0){$\times$}}
\put(1255,542){\makebox(0,0){$\times$}}
\put(1256,542){\makebox(0,0){$\times$}}
\put(1256,542){\makebox(0,0){$\times$}}
\put(1256,541){\makebox(0,0){$\times$}}
\put(1257,541){\makebox(0,0){$\times$}}
\put(1257,541){\makebox(0,0){$\times$}}
\put(1258,541){\makebox(0,0){$\times$}}
\put(1258,541){\makebox(0,0){$\times$}}
\put(1258,541){\makebox(0,0){$\times$}}
\put(1259,541){\makebox(0,0){$\times$}}
\put(1259,541){\makebox(0,0){$\times$}}
\put(1259,541){\makebox(0,0){$\times$}}
\put(1260,541){\makebox(0,0){$\times$}}
\put(1260,541){\makebox(0,0){$\times$}}
\put(1261,541){\makebox(0,0){$\times$}}
\put(1261,541){\makebox(0,0){$\times$}}
\put(1261,540){\makebox(0,0){$\times$}}
\put(1262,540){\makebox(0,0){$\times$}}
\put(1262,540){\makebox(0,0){$\times$}}
\put(1262,540){\makebox(0,0){$\times$}}
\put(1263,540){\makebox(0,0){$\times$}}
\put(1263,540){\makebox(0,0){$\times$}}
\put(1271,540){\makebox(0,0){$\times$}}
\put(1271,540){\makebox(0,0){$\times$}}
\put(1272,540){\makebox(0,0){$\times$}}
\put(1272,540){\makebox(0,0){$\times$}}
\put(1272,540){\makebox(0,0){$\times$}}
\put(1273,541){\makebox(0,0){$\times$}}
\put(1273,541){\makebox(0,0){$\times$}}
\put(1273,541){\makebox(0,0){$\times$}}
\put(1274,541){\makebox(0,0){$\times$}}
\put(1274,541){\makebox(0,0){$\times$}}
\put(1275,541){\makebox(0,0){$\times$}}
\put(1275,541){\makebox(0,0){$\times$}}
\put(1275,541){\makebox(0,0){$\times$}}
\put(1276,541){\makebox(0,0){$\times$}}
\put(1276,541){\makebox(0,0){$\times$}}
\put(1276,541){\makebox(0,0){$\times$}}
\put(1277,541){\makebox(0,0){$\times$}}
\put(1277,541){\makebox(0,0){$\times$}}
\put(1277,542){\makebox(0,0){$\times$}}
\put(1278,542){\makebox(0,0){$\times$}}
\put(1278,542){\makebox(0,0){$\times$}}
\put(1279,542){\makebox(0,0){$\times$}}
\put(1279,542){\makebox(0,0){$\times$}}
\put(1279,542){\makebox(0,0){$\times$}}
\put(1280,542){\makebox(0,0){$\times$}}
\put(1281,542){\makebox(0,0){$\times$}}
\put(1281,542){\makebox(0,0){$\times$}}
\put(1281,543){\makebox(0,0){$\times$}}
\put(1282,543){\makebox(0,0){$\times$}}
\put(1282,543){\makebox(0,0){$\times$}}
\put(1283,543){\makebox(0,0){$\times$}}
\put(1283,543){\makebox(0,0){$\times$}}
\put(1283,543){\makebox(0,0){$\times$}}
\put(1284,543){\makebox(0,0){$\times$}}
\put(1284,544){\makebox(0,0){$\times$}}
\put(1284,544){\makebox(0,0){$\times$}}
\put(1285,544){\makebox(0,0){$\times$}}
\put(1285,544){\makebox(0,0){$\times$}}
\put(1286,544){\makebox(0,0){$\times$}}
\put(1286,544){\makebox(0,0){$\times$}}
\put(1286,544){\makebox(0,0){$\times$}}
\put(1287,544){\makebox(0,0){$\times$}}
\put(1287,545){\makebox(0,0){$\times$}}
\put(1287,545){\makebox(0,0){$\times$}}
\put(1288,545){\makebox(0,0){$\times$}}
\put(1288,545){\makebox(0,0){$\times$}}
\put(1288,545){\makebox(0,0){$\times$}}
\put(1289,545){\makebox(0,0){$\times$}}
\put(1289,545){\makebox(0,0){$\times$}}
\put(1290,545){\makebox(0,0){$\times$}}
\put(1290,545){\makebox(0,0){$\times$}}
\put(1290,546){\makebox(0,0){$\times$}}
\put(1291,546){\makebox(0,0){$\times$}}
\put(1291,546){\makebox(0,0){$\times$}}
\put(1291,546){\makebox(0,0){$\times$}}
\put(1292,546){\makebox(0,0){$\times$}}
\put(1292,546){\makebox(0,0){$\times$}}
\put(1293,546){\makebox(0,0){$\times$}}
\put(1293,547){\makebox(0,0){$\times$}}
\put(1293,547){\makebox(0,0){$\times$}}
\put(1294,547){\makebox(0,0){$\times$}}
\put(1294,547){\makebox(0,0){$\times$}}
\put(1294,547){\makebox(0,0){$\times$}}
\put(1295,547){\makebox(0,0){$\times$}}
\put(1295,547){\makebox(0,0){$\times$}}
\put(1295,547){\makebox(0,0){$\times$}}
\put(1296,548){\makebox(0,0){$\times$}}
\put(1296,548){\makebox(0,0){$\times$}}
\put(1297,548){\makebox(0,0){$\times$}}
\put(1297,548){\makebox(0,0){$\times$}}
\put(1297,548){\makebox(0,0){$\times$}}
\put(1298,548){\makebox(0,0){$\times$}}
\put(1298,548){\makebox(0,0){$\times$}}
\put(1298,548){\makebox(0,0){$\times$}}
\put(1299,548){\makebox(0,0){$\times$}}
\put(1299,549){\makebox(0,0){$\times$}}
\put(1300,549){\makebox(0,0){$\times$}}
\put(1300,549){\makebox(0,0){$\times$}}
\put(1300,549){\makebox(0,0){$\times$}}
\put(1301,549){\makebox(0,0){$\times$}}
\put(1301,549){\makebox(0,0){$\times$}}
\put(1301,549){\makebox(0,0){$\times$}}
\put(1302,550){\makebox(0,0){$\times$}}
\put(1302,550){\makebox(0,0){$\times$}}
\put(1302,550){\makebox(0,0){$\times$}}
\put(1303,550){\makebox(0,0){$\times$}}
\put(1303,550){\makebox(0,0){$\times$}}
\put(1304,550){\makebox(0,0){$\times$}}
\put(1304,550){\makebox(0,0){$\times$}}
\put(1304,550){\makebox(0,0){$\times$}}
\put(1305,550){\makebox(0,0){$\times$}}
\put(1305,550){\makebox(0,0){$\times$}}
\put(1305,550){\makebox(0,0){$\times$}}
\put(1306,551){\makebox(0,0){$\times$}}
\put(1306,551){\makebox(0,0){$\times$}}
\put(1307,551){\makebox(0,0){$\times$}}
\put(1307,551){\makebox(0,0){$\times$}}
\put(1307,551){\makebox(0,0){$\times$}}
\put(1308,551){\makebox(0,0){$\times$}}
\put(1308,551){\makebox(0,0){$\times$}}
\put(1308,551){\makebox(0,0){$\times$}}
\put(1309,551){\makebox(0,0){$\times$}}
\put(1309,551){\makebox(0,0){$\times$}}
\put(1309,551){\makebox(0,0){$\times$}}
\put(1310,551){\makebox(0,0){$\times$}}
\put(1310,551){\makebox(0,0){$\times$}}
\put(1311,552){\makebox(0,0){$\times$}}
\put(1311,552){\makebox(0,0){$\times$}}
\put(1311,552){\makebox(0,0){$\times$}}
\put(1312,552){\makebox(0,0){$\times$}}
\put(1312,552){\makebox(0,0){$\times$}}
\put(1312,552){\makebox(0,0){$\times$}}
\put(1313,552){\makebox(0,0){$\times$}}
\put(1313,552){\makebox(0,0){$\times$}}
\put(1314,552){\makebox(0,0){$\times$}}
\put(1314,552){\makebox(0,0){$\times$}}
\put(1314,552){\makebox(0,0){$\times$}}
\put(1315,552){\makebox(0,0){$\times$}}
\put(1315,552){\makebox(0,0){$\times$}}
\put(1315,552){\makebox(0,0){$\times$}}
\put(1316,552){\makebox(0,0){$\times$}}
\put(1316,552){\makebox(0,0){$\times$}}
\put(1316,552){\makebox(0,0){$\times$}}
\put(1317,553){\makebox(0,0){$\times$}}
\put(1317,553){\makebox(0,0){$\times$}}
\put(1318,553){\makebox(0,0){$\times$}}
\put(1318,553){\makebox(0,0){$\times$}}
\put(1318,553){\makebox(0,0){$\times$}}
\put(1319,553){\makebox(0,0){$\times$}}
\put(1319,553){\makebox(0,0){$\times$}}
\put(1319,553){\makebox(0,0){$\times$}}
\put(1320,553){\makebox(0,0){$\times$}}
\put(1320,553){\makebox(0,0){$\times$}}
\put(1321,553){\makebox(0,0){$\times$}}
\put(1321,553){\makebox(0,0){$\times$}}
\put(1321,553){\makebox(0,0){$\times$}}
\put(1322,553){\makebox(0,0){$\times$}}
\put(1322,553){\makebox(0,0){$\times$}}
\put(1322,553){\makebox(0,0){$\times$}}
\put(1323,553){\makebox(0,0){$\times$}}
\put(1323,553){\makebox(0,0){$\times$}}
\put(1323,553){\makebox(0,0){$\times$}}
\put(1324,553){\makebox(0,0){$\times$}}
\put(1324,553){\makebox(0,0){$\times$}}
\put(1325,553){\makebox(0,0){$\times$}}
\put(1325,553){\makebox(0,0){$\times$}}
\put(1325,553){\makebox(0,0){$\times$}}
\put(1326,553){\makebox(0,0){$\times$}}
\put(1326,553){\makebox(0,0){$\times$}}
\put(1326,553){\makebox(0,0){$\times$}}
\put(1327,552){\makebox(0,0){$\times$}}
\put(1327,552){\makebox(0,0){$\times$}}
\put(1327,552){\makebox(0,0){$\times$}}
\put(1328,552){\makebox(0,0){$\times$}}
\put(1328,552){\makebox(0,0){$\times$}}
\put(1329,552){\makebox(0,0){$\times$}}
\put(1329,552){\makebox(0,0){$\times$}}
\put(1329,552){\makebox(0,0){$\times$}}
\put(1330,552){\makebox(0,0){$\times$}}
\put(1330,552){\makebox(0,0){$\times$}}
\put(1330,552){\makebox(0,0){$\times$}}
\put(1331,552){\makebox(0,0){$\times$}}
\put(1331,552){\makebox(0,0){$\times$}}
\put(1332,552){\makebox(0,0){$\times$}}
\put(1332,552){\makebox(0,0){$\times$}}
\put(1332,552){\makebox(0,0){$\times$}}
\put(1333,552){\makebox(0,0){$\times$}}
\put(1333,552){\makebox(0,0){$\times$}}
\put(1333,551){\makebox(0,0){$\times$}}
\put(1334,551){\makebox(0,0){$\times$}}
\put(1334,551){\makebox(0,0){$\times$}}
\put(1334,551){\makebox(0,0){$\times$}}
\put(1335,551){\makebox(0,0){$\times$}}
\put(1335,551){\makebox(0,0){$\times$}}
\put(1336,551){\makebox(0,0){$\times$}}
\put(1336,551){\makebox(0,0){$\times$}}
\put(1336,551){\makebox(0,0){$\times$}}
\put(1337,551){\makebox(0,0){$\times$}}
\put(1337,551){\makebox(0,0){$\times$}}
\put(1337,551){\makebox(0,0){$\times$}}
\put(1338,551){\makebox(0,0){$\times$}}
\put(1338,551){\makebox(0,0){$\times$}}
\put(1339,551){\makebox(0,0){$\times$}}
\put(1339,551){\makebox(0,0){$\times$}}
\put(1339,551){\makebox(0,0){$\times$}}
\put(1340,551){\makebox(0,0){$\times$}}
\put(1340,551){\makebox(0,0){$\times$}}
\put(1340,551){\makebox(0,0){$\times$}}
\put(1341,551){\makebox(0,0){$\times$}}
\put(1341,550){\makebox(0,0){$\times$}}
\put(1341,550){\makebox(0,0){$\times$}}
\put(1342,550){\makebox(0,0){$\times$}}
\put(1342,550){\makebox(0,0){$\times$}}
\put(1343,550){\makebox(0,0){$\times$}}
\put(1343,550){\makebox(0,0){$\times$}}
\put(1343,550){\makebox(0,0){$\times$}}
\put(1344,550){\makebox(0,0){$\times$}}
\put(1344,550){\makebox(0,0){$\times$}}
\put(1344,550){\makebox(0,0){$\times$}}
\put(1345,550){\makebox(0,0){$\times$}}
\put(1345,550){\makebox(0,0){$\times$}}
\put(1346,550){\makebox(0,0){$\times$}}
\put(1346,550){\makebox(0,0){$\times$}}
\put(1346,550){\makebox(0,0){$\times$}}
\put(1347,550){\makebox(0,0){$\times$}}
\put(1347,550){\makebox(0,0){$\times$}}
\put(1347,550){\makebox(0,0){$\times$}}
\put(1348,550){\makebox(0,0){$\times$}}
\put(1348,549){\makebox(0,0){$\times$}}
\put(1348,549){\makebox(0,0){$\times$}}
\put(1349,549){\makebox(0,0){$\times$}}
\put(1349,549){\makebox(0,0){$\times$}}
\put(1350,549){\makebox(0,0){$\times$}}
\put(1350,549){\makebox(0,0){$\times$}}
\put(1350,549){\makebox(0,0){$\times$}}
\put(1351,549){\makebox(0,0){$\times$}}
\put(1351,549){\makebox(0,0){$\times$}}
\put(1351,549){\makebox(0,0){$\times$}}
\put(1352,549){\makebox(0,0){$\times$}}
\put(1352,549){\makebox(0,0){$\times$}}
\put(1353,549){\makebox(0,0){$\times$}}
\put(1353,549){\makebox(0,0){$\times$}}
\put(1353,549){\makebox(0,0){$\times$}}
\put(1354,549){\makebox(0,0){$\times$}}
\put(1354,548){\makebox(0,0){$\times$}}
\put(1354,548){\makebox(0,0){$\times$}}
\put(1355,548){\makebox(0,0){$\times$}}
\put(1355,548){\makebox(0,0){$\times$}}
\put(1355,548){\makebox(0,0){$\times$}}
\put(1356,548){\makebox(0,0){$\times$}}
\put(1356,548){\makebox(0,0){$\times$}}
\put(1357,548){\makebox(0,0){$\times$}}
\put(1357,548){\makebox(0,0){$\times$}}
\put(1357,548){\makebox(0,0){$\times$}}
\put(1358,548){\makebox(0,0){$\times$}}
\put(1358,548){\makebox(0,0){$\times$}}
\put(1358,548){\makebox(0,0){$\times$}}
\put(1359,548){\makebox(0,0){$\times$}}
\put(1359,548){\makebox(0,0){$\times$}}
\put(1360,548){\makebox(0,0){$\times$}}
\put(1360,548){\makebox(0,0){$\times$}}
\put(1360,548){\makebox(0,0){$\times$}}
\put(1361,548){\makebox(0,0){$\times$}}
\put(1361,548){\makebox(0,0){$\times$}}
\put(1361,548){\makebox(0,0){$\times$}}
\put(1362,548){\makebox(0,0){$\times$}}
\put(1362,548){\makebox(0,0){$\times$}}
\put(1362,548){\makebox(0,0){$\times$}}
\put(1363,548){\makebox(0,0){$\times$}}
\put(1363,548){\makebox(0,0){$\times$}}
\put(1364,548){\makebox(0,0){$\times$}}
\put(1364,548){\makebox(0,0){$\times$}}
\put(1364,548){\makebox(0,0){$\times$}}
\put(1365,548){\makebox(0,0){$\times$}}
\put(1365,548){\makebox(0,0){$\times$}}
\put(1365,548){\makebox(0,0){$\times$}}
\put(1366,548){\makebox(0,0){$\times$}}
\put(1366,548){\makebox(0,0){$\times$}}
\put(1367,548){\makebox(0,0){$\times$}}
\put(1367,548){\makebox(0,0){$\times$}}
\put(1367,548){\makebox(0,0){$\times$}}
\put(1368,548){\makebox(0,0){$\times$}}
\put(1368,548){\makebox(0,0){$\times$}}
\put(1368,547){\makebox(0,0){$\times$}}
\put(1369,547){\makebox(0,0){$\times$}}
\put(1369,547){\makebox(0,0){$\times$}}
\put(1369,547){\makebox(0,0){$\times$}}
\put(1370,547){\makebox(0,0){$\times$}}
\put(1370,547){\makebox(0,0){$\times$}}
\put(1371,547){\makebox(0,0){$\times$}}
\put(1371,547){\makebox(0,0){$\times$}}
\put(1371,547){\makebox(0,0){$\times$}}
\put(1372,547){\makebox(0,0){$\times$}}
\put(1372,547){\makebox(0,0){$\times$}}
\put(1372,547){\makebox(0,0){$\times$}}
\put(1373,547){\makebox(0,0){$\times$}}
\put(1373,547){\makebox(0,0){$\times$}}
\put(1373,547){\makebox(0,0){$\times$}}
\put(1374,547){\makebox(0,0){$\times$}}
\put(1374,547){\makebox(0,0){$\times$}}
\put(1375,547){\makebox(0,0){$\times$}}
\put(1375,547){\makebox(0,0){$\times$}}
\put(1375,547){\makebox(0,0){$\times$}}
\put(1376,547){\makebox(0,0){$\times$}}
\put(1376,547){\makebox(0,0){$\times$}}
\put(1376,547){\makebox(0,0){$\times$}}
\put(1377,547){\makebox(0,0){$\times$}}
\put(1377,547){\makebox(0,0){$\times$}}
\put(1378,547){\makebox(0,0){$\times$}}
\put(1378,547){\makebox(0,0){$\times$}}
\put(1378,547){\makebox(0,0){$\times$}}
\put(1379,547){\makebox(0,0){$\times$}}
\put(1379,547){\makebox(0,0){$\times$}}
\put(1379,547){\makebox(0,0){$\times$}}
\put(1380,547){\makebox(0,0){$\times$}}
\put(1380,547){\makebox(0,0){$\times$}}
\put(1380,547){\makebox(0,0){$\times$}}
\put(1381,547){\makebox(0,0){$\times$}}
\put(1381,547){\makebox(0,0){$\times$}}
\put(1382,547){\makebox(0,0){$\times$}}
\put(1382,547){\makebox(0,0){$\times$}}
\put(1382,547){\makebox(0,0){$\times$}}
\put(1383,547){\makebox(0,0){$\times$}}
\put(1383,547){\makebox(0,0){$\times$}}
\put(1383,547){\makebox(0,0){$\times$}}
\put(1384,547){\makebox(0,0){$\times$}}
\put(1384,548){\makebox(0,0){$\times$}}
\put(1385,548){\makebox(0,0){$\times$}}
\put(1385,548){\makebox(0,0){$\times$}}
\put(1385,548){\makebox(0,0){$\times$}}
\put(1386,548){\makebox(0,0){$\times$}}
\put(1386,548){\makebox(0,0){$\times$}}
\put(1386,548){\makebox(0,0){$\times$}}
\put(1387,548){\makebox(0,0){$\times$}}
\put(1387,548){\makebox(0,0){$\times$}}
\put(1387,548){\makebox(0,0){$\times$}}
\put(1388,548){\makebox(0,0){$\times$}}
\put(1388,548){\makebox(0,0){$\times$}}
\put(1389,548){\makebox(0,0){$\times$}}
\put(1389,548){\makebox(0,0){$\times$}}
\put(1389,548){\makebox(0,0){$\times$}}
\put(1390,548){\makebox(0,0){$\times$}}
\put(1390,548){\makebox(0,0){$\times$}}
\put(1390,548){\makebox(0,0){$\times$}}
\put(1391,548){\makebox(0,0){$\times$}}
\put(1391,548){\makebox(0,0){$\times$}}
\put(1392,548){\makebox(0,0){$\times$}}
\put(1392,548){\makebox(0,0){$\times$}}
\put(1392,548){\makebox(0,0){$\times$}}
\put(1393,548){\makebox(0,0){$\times$}}
\put(1393,548){\makebox(0,0){$\times$}}
\put(1393,548){\makebox(0,0){$\times$}}
\put(1394,548){\makebox(0,0){$\times$}}
\put(1394,548){\makebox(0,0){$\times$}}
\put(1394,548){\makebox(0,0){$\times$}}
\put(1395,548){\makebox(0,0){$\times$}}
\put(1395,548){\makebox(0,0){$\times$}}
\put(1396,548){\makebox(0,0){$\times$}}
\put(1396,548){\makebox(0,0){$\times$}}
\put(1396,548){\makebox(0,0){$\times$}}
\put(1397,548){\makebox(0,0){$\times$}}
\put(1397,548){\makebox(0,0){$\times$}}
\put(1397,548){\makebox(0,0){$\times$}}
\put(1398,548){\makebox(0,0){$\times$}}
\put(1398,548){\makebox(0,0){$\times$}}
\put(1399,548){\makebox(0,0){$\times$}}
\put(1399,548){\makebox(0,0){$\times$}}
\put(1399,548){\makebox(0,0){$\times$}}
\put(1400,548){\makebox(0,0){$\times$}}
\put(1400,548){\makebox(0,0){$\times$}}
\put(1400,548){\makebox(0,0){$\times$}}
\put(1401,548){\makebox(0,0){$\times$}}
\put(1401,549){\makebox(0,0){$\times$}}
\put(1401,549){\makebox(0,0){$\times$}}
\put(1402,549){\makebox(0,0){$\times$}}
\put(1402,549){\makebox(0,0){$\times$}}
\put(1403,549){\makebox(0,0){$\times$}}
\put(1403,549){\makebox(0,0){$\times$}}
\put(1403,549){\makebox(0,0){$\times$}}
\put(1404,549){\makebox(0,0){$\times$}}
\put(1404,549){\makebox(0,0){$\times$}}
\put(1404,549){\makebox(0,0){$\times$}}
\put(1405,549){\makebox(0,0){$\times$}}
\put(1405,549){\makebox(0,0){$\times$}}
\put(1406,549){\makebox(0,0){$\times$}}
\put(1406,549){\makebox(0,0){$\times$}}
\put(1406,549){\makebox(0,0){$\times$}}
\put(1407,549){\makebox(0,0){$\times$}}
\put(1407,549){\makebox(0,0){$\times$}}
\put(1407,549){\makebox(0,0){$\times$}}
\put(1408,549){\makebox(0,0){$\times$}}
\put(1408,549){\makebox(0,0){$\times$}}
\put(1408,549){\makebox(0,0){$\times$}}
\put(1409,549){\makebox(0,0){$\times$}}
\put(1409,549){\makebox(0,0){$\times$}}
\put(1410,549){\makebox(0,0){$\times$}}
\put(1410,549){\makebox(0,0){$\times$}}
\put(1410,549){\makebox(0,0){$\times$}}
\put(1411,549){\makebox(0,0){$\times$}}
\put(1411,549){\makebox(0,0){$\times$}}
\put(1411,549){\makebox(0,0){$\times$}}
\put(1412,549){\makebox(0,0){$\times$}}
\put(1412,549){\makebox(0,0){$\times$}}
\put(1413,549){\makebox(0,0){$\times$}}
\put(1413,549){\makebox(0,0){$\times$}}
\put(1413,549){\makebox(0,0){$\times$}}
\put(1414,549){\makebox(0,0){$\times$}}
\put(1414,549){\makebox(0,0){$\times$}}
\put(1414,549){\makebox(0,0){$\times$}}
\put(1415,549){\makebox(0,0){$\times$}}
\put(1415,549){\makebox(0,0){$\times$}}
\put(1415,549){\makebox(0,0){$\times$}}
\put(1416,549){\makebox(0,0){$\times$}}
\put(1416,549){\makebox(0,0){$\times$}}
\put(1417,549){\makebox(0,0){$\times$}}
\put(1417,549){\makebox(0,0){$\times$}}
\put(1417,549){\makebox(0,0){$\times$}}
\put(1418,549){\makebox(0,0){$\times$}}
\put(1418,549){\makebox(0,0){$\times$}}
\put(1418,549){\makebox(0,0){$\times$}}
\put(1419,549){\makebox(0,0){$\times$}}
\put(1419,549){\makebox(0,0){$\times$}}
\put(1419,549){\makebox(0,0){$\times$}}
\put(1420,549){\makebox(0,0){$\times$}}
\put(1420,549){\makebox(0,0){$\times$}}
\put(1421,549){\makebox(0,0){$\times$}}
\put(1421,549){\makebox(0,0){$\times$}}
\put(1421,549){\makebox(0,0){$\times$}}
\put(1422,549){\makebox(0,0){$\times$}}
\put(1422,549){\makebox(0,0){$\times$}}
\put(1422,549){\makebox(0,0){$\times$}}
\put(1423,549){\makebox(0,0){$\times$}}
\put(1423,549){\makebox(0,0){$\times$}}
\put(1424,549){\makebox(0,0){$\times$}}
\put(1424,549){\makebox(0,0){$\times$}}
\put(1424,549){\makebox(0,0){$\times$}}
\put(1425,549){\makebox(0,0){$\times$}}
\put(1425,549){\makebox(0,0){$\times$}}
\put(1425,549){\makebox(0,0){$\times$}}
\put(1426,549){\makebox(0,0){$\times$}}
\put(1426,549){\makebox(0,0){$\times$}}
\put(1426,549){\makebox(0,0){$\times$}}
\put(1427,549){\makebox(0,0){$\times$}}
\put(1427,549){\makebox(0,0){$\times$}}
\put(1428,549){\makebox(0,0){$\times$}}
\put(1428,549){\makebox(0,0){$\times$}}
\put(1428,549){\makebox(0,0){$\times$}}
\put(1429,549){\makebox(0,0){$\times$}}
\put(1429,549){\makebox(0,0){$\times$}}
\put(1429,549){\makebox(0,0){$\times$}}
\put(1430,549){\makebox(0,0){$\times$}}
\put(1430,549){\makebox(0,0){$\times$}}
\put(1431,549){\makebox(0,0){$\times$}}
\put(1431,549){\makebox(0,0){$\times$}}
\put(1431,549){\makebox(0,0){$\times$}}
\put(1432,549){\makebox(0,0){$\times$}}
\put(1432,549){\makebox(0,0){$\times$}}
\put(1432,549){\makebox(0,0){$\times$}}
\put(1433,549){\makebox(0,0){$\times$}}
\put(1433,549){\makebox(0,0){$\times$}}
\put(1433,549){\makebox(0,0){$\times$}}
\put(1434,549){\makebox(0,0){$\times$}}
\put(1434,549){\makebox(0,0){$\times$}}
\put(1435,549){\makebox(0,0){$\times$}}
\put(1435,549){\makebox(0,0){$\times$}}
\put(1435,549){\makebox(0,0){$\times$}}
\put(1436,549){\makebox(0,0){$\times$}}
\put(1436,549){\makebox(0,0){$\times$}}
\put(1436,549){\makebox(0,0){$\times$}}
\put(1437,549){\makebox(0,0){$\times$}}
\put(1437,549){\makebox(0,0){$\times$}}
\put(1438,549){\makebox(0,0){$\times$}}
\put(1438,549){\makebox(0,0){$\times$}}
\put(1438,549){\makebox(0,0){$\times$}}
\put(1439,549){\makebox(0,0){$\times$}}
\put(1349,778){\makebox(0,0){$\times$}}
\put(151.0,131.0){\rule[-0.200pt]{0.400pt}{175.375pt}}
\put(151.0,131.0){\rule[-0.200pt]{310.279pt}{0.400pt}}
\put(1439.0,131.0){\rule[-0.200pt]{0.400pt}{175.375pt}}
\put(151.0,859.0){\rule[-0.200pt]{310.279pt}{0.400pt}}
\end{picture}

\caption{Závislosť polohy $x$ v čase $t$, preložené funkciou $x= \(1\cdot10^11\pm2.8\cdot10^10\)e^{-\(1.28\pm0.02\)t} sin\(\(18.24\pm0.02\)t +\( 34.23\pm0.29\)\) $}  \label{G_2}
\end{figure}




\begin{figure}
% GNUPLOT: LaTeX picture
\setlength{\unitlength}{0.240900pt}
\ifx\plotpoint\undefined\newsavebox{\plotpoint}\fi
\begin{picture}(1500,900)(0,0)
\sbox{\plotpoint}{\rule[-0.200pt]{0.400pt}{0.400pt}}%
\put(151.0,131.0){\rule[-0.200pt]{4.818pt}{0.400pt}}
\put(131,131){\makebox(0,0)[r]{-26}}
\put(1419.0,131.0){\rule[-0.200pt]{4.818pt}{0.400pt}}
\put(151.0,204.0){\rule[-0.200pt]{4.818pt}{0.400pt}}
\put(131,204){\makebox(0,0)[r]{-24}}
\put(1419.0,204.0){\rule[-0.200pt]{4.818pt}{0.400pt}}
\put(151.0,277.0){\rule[-0.200pt]{4.818pt}{0.400pt}}
\put(131,277){\makebox(0,0)[r]{-22}}
\put(1419.0,277.0){\rule[-0.200pt]{4.818pt}{0.400pt}}
\put(151.0,349.0){\rule[-0.200pt]{4.818pt}{0.400pt}}
\put(131,349){\makebox(0,0)[r]{-20}}
\put(1419.0,349.0){\rule[-0.200pt]{4.818pt}{0.400pt}}
\put(151.0,422.0){\rule[-0.200pt]{4.818pt}{0.400pt}}
\put(131,422){\makebox(0,0)[r]{-18}}
\put(1419.0,422.0){\rule[-0.200pt]{4.818pt}{0.400pt}}
\put(151.0,495.0){\rule[-0.200pt]{4.818pt}{0.400pt}}
\put(131,495){\makebox(0,0)[r]{-16}}
\put(1419.0,495.0){\rule[-0.200pt]{4.818pt}{0.400pt}}
\put(151.0,568.0){\rule[-0.200pt]{4.818pt}{0.400pt}}
\put(131,568){\makebox(0,0)[r]{-14}}
\put(1419.0,568.0){\rule[-0.200pt]{4.818pt}{0.400pt}}
\put(151.0,641.0){\rule[-0.200pt]{4.818pt}{0.400pt}}
\put(131,641){\makebox(0,0)[r]{-12}}
\put(1419.0,641.0){\rule[-0.200pt]{4.818pt}{0.400pt}}
\put(151.0,713.0){\rule[-0.200pt]{4.818pt}{0.400pt}}
\put(131,713){\makebox(0,0)[r]{-10}}
\put(1419.0,713.0){\rule[-0.200pt]{4.818pt}{0.400pt}}
\put(151.0,786.0){\rule[-0.200pt]{4.818pt}{0.400pt}}
\put(131,786){\makebox(0,0)[r]{-8}}
\put(1419.0,786.0){\rule[-0.200pt]{4.818pt}{0.400pt}}
\put(151.0,859.0){\rule[-0.200pt]{4.818pt}{0.400pt}}
\put(131,859){\makebox(0,0)[r]{-6}}
\put(1419.0,859.0){\rule[-0.200pt]{4.818pt}{0.400pt}}
\put(151.0,131.0){\rule[-0.200pt]{0.400pt}{4.818pt}}
\put(151,90){\makebox(0,0){ 9.6}}
\put(151.0,839.0){\rule[-0.200pt]{0.400pt}{4.818pt}}
\put(294.0,131.0){\rule[-0.200pt]{0.400pt}{4.818pt}}
\put(294,90){\makebox(0,0){ 9.8}}
\put(294.0,839.0){\rule[-0.200pt]{0.400pt}{4.818pt}}
\put(437.0,131.0){\rule[-0.200pt]{0.400pt}{4.818pt}}
\put(437,90){\makebox(0,0){ 10}}
\put(437.0,839.0){\rule[-0.200pt]{0.400pt}{4.818pt}}
\put(580.0,131.0){\rule[-0.200pt]{0.400pt}{4.818pt}}
\put(580,90){\makebox(0,0){ 10.2}}
\put(580.0,839.0){\rule[-0.200pt]{0.400pt}{4.818pt}}
\put(723.0,131.0){\rule[-0.200pt]{0.400pt}{4.818pt}}
\put(723,90){\makebox(0,0){ 10.4}}
\put(723.0,839.0){\rule[-0.200pt]{0.400pt}{4.818pt}}
\put(867.0,131.0){\rule[-0.200pt]{0.400pt}{4.818pt}}
\put(867,90){\makebox(0,0){ 10.6}}
\put(867.0,839.0){\rule[-0.200pt]{0.400pt}{4.818pt}}
\put(1010.0,131.0){\rule[-0.200pt]{0.400pt}{4.818pt}}
\put(1010,90){\makebox(0,0){ 10.8}}
\put(1010.0,839.0){\rule[-0.200pt]{0.400pt}{4.818pt}}
\put(1153.0,131.0){\rule[-0.200pt]{0.400pt}{4.818pt}}
\put(1153,90){\makebox(0,0){ 11}}
\put(1153.0,839.0){\rule[-0.200pt]{0.400pt}{4.818pt}}
\put(1296.0,131.0){\rule[-0.200pt]{0.400pt}{4.818pt}}
\put(1296,90){\makebox(0,0){ 11.2}}
\put(1296.0,839.0){\rule[-0.200pt]{0.400pt}{4.818pt}}
\put(1439.0,131.0){\rule[-0.200pt]{0.400pt}{4.818pt}}
\put(1439,90){\makebox(0,0){ 11.4}}
\put(1439.0,839.0){\rule[-0.200pt]{0.400pt}{4.818pt}}
\put(151.0,131.0){\rule[-0.200pt]{0.400pt}{175.375pt}}
\put(151.0,131.0){\rule[-0.200pt]{310.279pt}{0.400pt}}
\put(1439.0,131.0){\rule[-0.200pt]{0.400pt}{175.375pt}}
\put(151.0,859.0){\rule[-0.200pt]{310.279pt}{0.400pt}}
\put(30,495){\makebox(0,0){\popi{x}{mm}}}
\put(795,29){\makebox(0,0){\popi{t}{s}}}
\put(1279,213){\makebox(0,0)[r]{$x= f(t) $}}
\put(1299.0,213.0){\rule[-0.200pt]{24.090pt}{0.400pt}}
\put(202,131){\usebox{\plotpoint}}
\multiput(202.58,131.00)(0.492,0.539){21}{\rule{0.119pt}{0.533pt}}
\multiput(201.17,131.00)(12.000,11.893){2}{\rule{0.400pt}{0.267pt}}
\multiput(214.58,144.00)(0.493,1.726){23}{\rule{0.119pt}{1.454pt}}
\multiput(213.17,144.00)(13.000,40.982){2}{\rule{0.400pt}{0.727pt}}
\multiput(227.58,188.00)(0.492,2.952){21}{\rule{0.119pt}{2.400pt}}
\multiput(226.17,188.00)(12.000,64.019){2}{\rule{0.400pt}{1.200pt}}
\multiput(239.58,257.00)(0.492,3.727){21}{\rule{0.119pt}{3.000pt}}
\multiput(238.17,257.00)(12.000,80.773){2}{\rule{0.400pt}{1.500pt}}
\multiput(251.58,344.00)(0.493,3.906){23}{\rule{0.119pt}{3.146pt}}
\multiput(250.17,344.00)(13.000,92.470){2}{\rule{0.400pt}{1.573pt}}
\multiput(264.58,443.00)(0.492,4.330){21}{\rule{0.119pt}{3.467pt}}
\multiput(263.17,443.00)(12.000,93.805){2}{\rule{0.400pt}{1.733pt}}
\multiput(276.58,544.00)(0.493,3.748){23}{\rule{0.119pt}{3.023pt}}
\multiput(275.17,544.00)(13.000,88.725){2}{\rule{0.400pt}{1.512pt}}
\multiput(289.58,639.00)(0.492,3.512){21}{\rule{0.119pt}{2.833pt}}
\multiput(288.17,639.00)(12.000,76.119){2}{\rule{0.400pt}{1.417pt}}
\multiput(301.58,721.00)(0.492,2.693){21}{\rule{0.119pt}{2.200pt}}
\multiput(300.17,721.00)(12.000,58.434){2}{\rule{0.400pt}{1.100pt}}
\multiput(313.58,784.00)(0.493,1.527){23}{\rule{0.119pt}{1.300pt}}
\multiput(312.17,784.00)(13.000,36.302){2}{\rule{0.400pt}{0.650pt}}
\multiput(326.58,823.00)(0.492,0.582){21}{\rule{0.119pt}{0.567pt}}
\multiput(325.17,823.00)(12.000,12.824){2}{\rule{0.400pt}{0.283pt}}
\multiput(338.00,835.92)(0.590,-0.492){19}{\rule{0.573pt}{0.118pt}}
\multiput(338.00,836.17)(11.811,-11.000){2}{\rule{0.286pt}{0.400pt}}
\multiput(351.58,820.88)(0.492,-1.444){21}{\rule{0.119pt}{1.233pt}}
\multiput(350.17,823.44)(12.000,-31.440){2}{\rule{0.400pt}{0.617pt}}
\multiput(363.58,784.11)(0.492,-2.306){21}{\rule{0.119pt}{1.900pt}}
\multiput(362.17,788.06)(12.000,-50.056){2}{\rule{0.400pt}{0.950pt}}
\multiput(375.58,729.03)(0.493,-2.638){23}{\rule{0.119pt}{2.162pt}}
\multiput(374.17,733.51)(13.000,-62.514){2}{\rule{0.400pt}{1.081pt}}
\multiput(388.58,660.07)(0.492,-3.253){21}{\rule{0.119pt}{2.633pt}}
\multiput(387.17,665.53)(12.000,-70.534){2}{\rule{0.400pt}{1.317pt}}
\multiput(400.58,584.75)(0.493,-3.034){23}{\rule{0.119pt}{2.469pt}}
\multiput(399.17,589.88)(13.000,-71.875){2}{\rule{0.400pt}{1.235pt}}
\multiput(413.58,507.62)(0.492,-3.081){21}{\rule{0.119pt}{2.500pt}}
\multiput(412.17,512.81)(12.000,-66.811){2}{\rule{0.400pt}{1.250pt}}
\multiput(425.58,437.01)(0.492,-2.650){21}{\rule{0.119pt}{2.167pt}}
\multiput(424.17,441.50)(12.000,-57.503){2}{\rule{0.400pt}{1.083pt}}
\multiput(437.58,377.58)(0.493,-1.845){23}{\rule{0.119pt}{1.546pt}}
\multiput(436.17,380.79)(13.000,-43.791){2}{\rule{0.400pt}{0.773pt}}
\multiput(450.58,332.57)(0.492,-1.229){21}{\rule{0.119pt}{1.067pt}}
\multiput(449.17,334.79)(12.000,-26.786){2}{\rule{0.400pt}{0.533pt}}
\multiput(462.00,306.92)(0.652,-0.491){17}{\rule{0.620pt}{0.118pt}}
\multiput(462.00,307.17)(11.713,-10.000){2}{\rule{0.310pt}{0.400pt}}
\multiput(475.00,298.59)(0.669,0.489){15}{\rule{0.633pt}{0.118pt}}
\multiput(475.00,297.17)(10.685,9.000){2}{\rule{0.317pt}{0.400pt}}
\multiput(487.58,307.00)(0.492,1.142){21}{\rule{0.119pt}{1.000pt}}
\multiput(486.17,307.00)(12.000,24.924){2}{\rule{0.400pt}{0.500pt}}
\multiput(499.58,334.00)(0.493,1.607){23}{\rule{0.119pt}{1.362pt}}
\multiput(498.17,334.00)(13.000,38.174){2}{\rule{0.400pt}{0.681pt}}
\multiput(512.58,375.00)(0.492,2.219){21}{\rule{0.119pt}{1.833pt}}
\multiput(511.17,375.00)(12.000,48.195){2}{\rule{0.400pt}{0.917pt}}
\multiput(524.58,427.00)(0.492,2.478){21}{\rule{0.119pt}{2.033pt}}
\multiput(523.17,427.00)(12.000,53.780){2}{\rule{0.400pt}{1.017pt}}
\multiput(536.58,485.00)(0.493,2.320){23}{\rule{0.119pt}{1.915pt}}
\multiput(535.17,485.00)(13.000,55.025){2}{\rule{0.400pt}{0.958pt}}
\multiput(549.58,544.00)(0.492,2.349){21}{\rule{0.119pt}{1.933pt}}
\multiput(548.17,544.00)(12.000,50.987){2}{\rule{0.400pt}{0.967pt}}
\multiput(561.58,599.00)(0.493,1.845){23}{\rule{0.119pt}{1.546pt}}
\multiput(560.17,599.00)(13.000,43.791){2}{\rule{0.400pt}{0.773pt}}
\multiput(574.58,646.00)(0.492,1.487){21}{\rule{0.119pt}{1.267pt}}
\multiput(573.17,646.00)(12.000,32.371){2}{\rule{0.400pt}{0.633pt}}
\multiput(586.58,681.00)(0.492,0.927){21}{\rule{0.119pt}{0.833pt}}
\multiput(585.17,681.00)(12.000,20.270){2}{\rule{0.400pt}{0.417pt}}
\multiput(598.00,703.59)(0.950,0.485){11}{\rule{0.843pt}{0.117pt}}
\multiput(598.00,702.17)(11.251,7.000){2}{\rule{0.421pt}{0.400pt}}
\multiput(611.00,708.93)(0.758,-0.488){13}{\rule{0.700pt}{0.117pt}}
\multiput(611.00,709.17)(10.547,-8.000){2}{\rule{0.350pt}{0.400pt}}
\multiput(623.58,698.90)(0.493,-0.814){23}{\rule{0.119pt}{0.746pt}}
\multiput(622.17,700.45)(13.000,-19.451){2}{\rule{0.400pt}{0.373pt}}
\multiput(636.58,676.16)(0.492,-1.358){21}{\rule{0.119pt}{1.167pt}}
\multiput(635.17,678.58)(12.000,-29.579){2}{\rule{0.400pt}{0.583pt}}
\multiput(648.58,643.05)(0.492,-1.703){21}{\rule{0.119pt}{1.433pt}}
\multiput(647.17,646.03)(12.000,-37.025){2}{\rule{0.400pt}{0.717pt}}
\multiput(660.58,602.84)(0.493,-1.765){23}{\rule{0.119pt}{1.485pt}}
\multiput(659.17,605.92)(13.000,-41.919){2}{\rule{0.400pt}{0.742pt}}
\multiput(673.58,557.50)(0.492,-1.875){21}{\rule{0.119pt}{1.567pt}}
\multiput(672.17,560.75)(12.000,-40.748){2}{\rule{0.400pt}{0.783pt}}
\multiput(685.58,514.22)(0.493,-1.646){23}{\rule{0.119pt}{1.392pt}}
\multiput(684.17,517.11)(13.000,-39.110){2}{\rule{0.400pt}{0.696pt}}
\multiput(698.58,472.60)(0.492,-1.530){21}{\rule{0.119pt}{1.300pt}}
\multiput(697.17,475.30)(12.000,-33.302){2}{\rule{0.400pt}{0.650pt}}
\multiput(710.58,437.99)(0.492,-1.099){21}{\rule{0.119pt}{0.967pt}}
\multiput(709.17,439.99)(12.000,-23.994){2}{\rule{0.400pt}{0.483pt}}
\multiput(722.58,413.54)(0.493,-0.616){23}{\rule{0.119pt}{0.592pt}}
\multiput(721.17,414.77)(13.000,-14.771){2}{\rule{0.400pt}{0.296pt}}
\multiput(735.00,398.93)(1.267,-0.477){7}{\rule{1.060pt}{0.115pt}}
\multiput(735.00,399.17)(9.800,-5.000){2}{\rule{0.530pt}{0.400pt}}
\multiput(747.00,395.59)(1.123,0.482){9}{\rule{0.967pt}{0.116pt}}
\multiput(747.00,394.17)(10.994,6.000){2}{\rule{0.483pt}{0.400pt}}
\multiput(760.58,401.00)(0.492,0.712){21}{\rule{0.119pt}{0.667pt}}
\multiput(759.17,401.00)(12.000,15.616){2}{\rule{0.400pt}{0.333pt}}
\multiput(772.58,418.00)(0.492,1.056){21}{\rule{0.119pt}{0.933pt}}
\multiput(771.17,418.00)(12.000,23.063){2}{\rule{0.400pt}{0.467pt}}
\multiput(784.58,443.00)(0.493,1.210){23}{\rule{0.119pt}{1.054pt}}
\multiput(783.17,443.00)(13.000,28.813){2}{\rule{0.400pt}{0.527pt}}
\multiput(797.58,474.00)(0.492,1.444){21}{\rule{0.119pt}{1.233pt}}
\multiput(796.17,474.00)(12.000,31.440){2}{\rule{0.400pt}{0.617pt}}
\multiput(809.58,508.00)(0.493,1.329){23}{\rule{0.119pt}{1.146pt}}
\multiput(808.17,508.00)(13.000,31.621){2}{\rule{0.400pt}{0.573pt}}
\multiput(822.58,542.00)(0.492,1.358){21}{\rule{0.119pt}{1.167pt}}
\multiput(821.17,542.00)(12.000,29.579){2}{\rule{0.400pt}{0.583pt}}
\multiput(834.58,574.00)(0.492,1.142){21}{\rule{0.119pt}{1.000pt}}
\multiput(833.17,574.00)(12.000,24.924){2}{\rule{0.400pt}{0.500pt}}
\multiput(846.58,601.00)(0.493,0.774){23}{\rule{0.119pt}{0.715pt}}
\multiput(845.17,601.00)(13.000,18.515){2}{\rule{0.400pt}{0.358pt}}
\multiput(859.00,621.58)(0.543,0.492){19}{\rule{0.536pt}{0.118pt}}
\multiput(859.00,620.17)(10.887,11.000){2}{\rule{0.268pt}{0.400pt}}
\multiput(871.00,632.60)(1.797,0.468){5}{\rule{1.400pt}{0.113pt}}
\multiput(871.00,631.17)(10.094,4.000){2}{\rule{0.700pt}{0.400pt}}
\multiput(884.00,634.93)(1.033,-0.482){9}{\rule{0.900pt}{0.116pt}}
\multiput(884.00,635.17)(10.132,-6.000){2}{\rule{0.450pt}{0.400pt}}
\multiput(896.58,627.79)(0.492,-0.539){21}{\rule{0.119pt}{0.533pt}}
\multiput(895.17,628.89)(12.000,-11.893){2}{\rule{0.400pt}{0.267pt}}
\multiput(908.58,614.16)(0.493,-0.734){23}{\rule{0.119pt}{0.685pt}}
\multiput(907.17,615.58)(13.000,-17.579){2}{\rule{0.400pt}{0.342pt}}
\multiput(921.58,594.26)(0.492,-1.013){21}{\rule{0.119pt}{0.900pt}}
\multiput(920.17,596.13)(12.000,-22.132){2}{\rule{0.400pt}{0.450pt}}
\multiput(933.58,570.26)(0.493,-1.012){23}{\rule{0.119pt}{0.900pt}}
\multiput(932.17,572.13)(13.000,-24.132){2}{\rule{0.400pt}{0.450pt}}
\multiput(946.58,543.99)(0.492,-1.099){21}{\rule{0.119pt}{0.967pt}}
\multiput(945.17,545.99)(12.000,-23.994){2}{\rule{0.400pt}{0.483pt}}
\multiput(958.58,518.26)(0.492,-1.013){21}{\rule{0.119pt}{0.900pt}}
\multiput(957.17,520.13)(12.000,-22.132){2}{\rule{0.400pt}{0.450pt}}
\multiput(970.58,495.03)(0.493,-0.774){23}{\rule{0.119pt}{0.715pt}}
\multiput(969.17,496.52)(13.000,-18.515){2}{\rule{0.400pt}{0.358pt}}
\multiput(983.58,475.51)(0.492,-0.625){21}{\rule{0.119pt}{0.600pt}}
\multiput(982.17,476.75)(12.000,-13.755){2}{\rule{0.400pt}{0.300pt}}
\multiput(995.00,461.93)(0.728,-0.489){15}{\rule{0.678pt}{0.118pt}}
\multiput(995.00,462.17)(11.593,-9.000){2}{\rule{0.339pt}{0.400pt}}
\put(1008,452.17){\rule{2.500pt}{0.400pt}}
\multiput(1008.00,453.17)(6.811,-2.000){2}{\rule{1.250pt}{0.400pt}}
\multiput(1020.00,452.60)(1.651,0.468){5}{\rule{1.300pt}{0.113pt}}
\multiput(1020.00,451.17)(9.302,4.000){2}{\rule{0.650pt}{0.400pt}}
\multiput(1032.00,456.58)(0.652,0.491){17}{\rule{0.620pt}{0.118pt}}
\multiput(1032.00,455.17)(11.713,10.000){2}{\rule{0.310pt}{0.400pt}}
\multiput(1045.58,466.00)(0.492,0.625){21}{\rule{0.119pt}{0.600pt}}
\multiput(1044.17,466.00)(12.000,13.755){2}{\rule{0.400pt}{0.300pt}}
\multiput(1057.58,481.00)(0.493,0.734){23}{\rule{0.119pt}{0.685pt}}
\multiput(1056.17,481.00)(13.000,17.579){2}{\rule{0.400pt}{0.342pt}}
\multiput(1070.58,500.00)(0.492,0.798){21}{\rule{0.119pt}{0.733pt}}
\multiput(1069.17,500.00)(12.000,17.478){2}{\rule{0.400pt}{0.367pt}}
\multiput(1082.58,519.00)(0.492,0.841){21}{\rule{0.119pt}{0.767pt}}
\multiput(1081.17,519.00)(12.000,18.409){2}{\rule{0.400pt}{0.383pt}}
\multiput(1094.58,539.00)(0.493,0.734){23}{\rule{0.119pt}{0.685pt}}
\multiput(1093.17,539.00)(13.000,17.579){2}{\rule{0.400pt}{0.342pt}}
\multiput(1107.58,558.00)(0.492,0.625){21}{\rule{0.119pt}{0.600pt}}
\multiput(1106.17,558.00)(12.000,13.755){2}{\rule{0.400pt}{0.300pt}}
\multiput(1119.00,573.58)(0.543,0.492){19}{\rule{0.536pt}{0.118pt}}
\multiput(1119.00,572.17)(10.887,11.000){2}{\rule{0.268pt}{0.400pt}}
\multiput(1131.00,584.59)(0.950,0.485){11}{\rule{0.843pt}{0.117pt}}
\multiput(1131.00,583.17)(11.251,7.000){2}{\rule{0.421pt}{0.400pt}}
\put(1144,590.67){\rule{2.891pt}{0.400pt}}
\multiput(1144.00,590.17)(6.000,1.000){2}{\rule{1.445pt}{0.400pt}}
\multiput(1156.00,590.95)(2.695,-0.447){3}{\rule{1.833pt}{0.108pt}}
\multiput(1156.00,591.17)(9.195,-3.000){2}{\rule{0.917pt}{0.400pt}}
\multiput(1169.00,587.93)(0.758,-0.488){13}{\rule{0.700pt}{0.117pt}}
\multiput(1169.00,588.17)(10.547,-8.000){2}{\rule{0.350pt}{0.400pt}}
\multiput(1181.00,579.92)(0.496,-0.492){21}{\rule{0.500pt}{0.119pt}}
\multiput(1181.00,580.17)(10.962,-12.000){2}{\rule{0.250pt}{0.400pt}}
\multiput(1193.58,566.80)(0.493,-0.536){23}{\rule{0.119pt}{0.531pt}}
\multiput(1192.17,567.90)(13.000,-12.898){2}{\rule{0.400pt}{0.265pt}}
\multiput(1206.58,552.51)(0.492,-0.625){21}{\rule{0.119pt}{0.600pt}}
\multiput(1205.17,553.75)(12.000,-13.755){2}{\rule{0.400pt}{0.300pt}}
\multiput(1218.58,537.67)(0.493,-0.576){23}{\rule{0.119pt}{0.562pt}}
\multiput(1217.17,538.83)(13.000,-13.834){2}{\rule{0.400pt}{0.281pt}}
\multiput(1231.58,522.65)(0.492,-0.582){21}{\rule{0.119pt}{0.567pt}}
\multiput(1230.17,523.82)(12.000,-12.824){2}{\rule{0.400pt}{0.283pt}}
\multiput(1243.00,509.92)(0.496,-0.492){21}{\rule{0.500pt}{0.119pt}}
\multiput(1243.00,510.17)(10.962,-12.000){2}{\rule{0.250pt}{0.400pt}}
\multiput(1255.00,497.93)(0.824,-0.488){13}{\rule{0.750pt}{0.117pt}}
\multiput(1255.00,498.17)(11.443,-8.000){2}{\rule{0.375pt}{0.400pt}}
\multiput(1268.00,489.93)(1.267,-0.477){7}{\rule{1.060pt}{0.115pt}}
\multiput(1268.00,490.17)(9.800,-5.000){2}{\rule{0.530pt}{0.400pt}}
\put(1280,484.67){\rule{3.132pt}{0.400pt}}
\multiput(1280.00,485.17)(6.500,-1.000){2}{\rule{1.566pt}{0.400pt}}
\multiput(1293.00,485.61)(2.472,0.447){3}{\rule{1.700pt}{0.108pt}}
\multiput(1293.00,484.17)(8.472,3.000){2}{\rule{0.850pt}{0.400pt}}
\multiput(1305.00,488.59)(1.033,0.482){9}{\rule{0.900pt}{0.116pt}}
\multiput(1305.00,487.17)(10.132,6.000){2}{\rule{0.450pt}{0.400pt}}
\multiput(1317.00,494.59)(0.728,0.489){15}{\rule{0.678pt}{0.118pt}}
\multiput(1317.00,493.17)(11.593,9.000){2}{\rule{0.339pt}{0.400pt}}
\multiput(1330.00,503.58)(0.543,0.492){19}{\rule{0.536pt}{0.118pt}}
\multiput(1330.00,502.17)(10.887,11.000){2}{\rule{0.268pt}{0.400pt}}
\multiput(1342.00,514.58)(0.539,0.492){21}{\rule{0.533pt}{0.119pt}}
\multiput(1342.00,513.17)(11.893,12.000){2}{\rule{0.267pt}{0.400pt}}
\multiput(1355.00,526.58)(0.543,0.492){19}{\rule{0.536pt}{0.118pt}}
\multiput(1355.00,525.17)(10.887,11.000){2}{\rule{0.268pt}{0.400pt}}
\multiput(1367.00,537.58)(0.543,0.492){19}{\rule{0.536pt}{0.118pt}}
\multiput(1367.00,536.17)(10.887,11.000){2}{\rule{0.268pt}{0.400pt}}
\multiput(1379.00,548.59)(0.824,0.488){13}{\rule{0.750pt}{0.117pt}}
\multiput(1379.00,547.17)(11.443,8.000){2}{\rule{0.375pt}{0.400pt}}
\multiput(1392.00,556.59)(0.874,0.485){11}{\rule{0.786pt}{0.117pt}}
\multiput(1392.00,555.17)(10.369,7.000){2}{\rule{0.393pt}{0.400pt}}
\multiput(1404.00,563.61)(2.695,0.447){3}{\rule{1.833pt}{0.108pt}}
\multiput(1404.00,562.17)(9.195,3.000){2}{\rule{0.917pt}{0.400pt}}
\put(1417,565.67){\rule{2.891pt}{0.400pt}}
\multiput(1417.00,565.17)(6.000,1.000){2}{\rule{1.445pt}{0.400pt}}
\put(1279,172){\makebox(0,0)[r]{namerané dáta}}
\put(202,163){\makebox(0,0){$\times$}}
\put(203,163){\makebox(0,0){$\times$}}
\put(203,164){\makebox(0,0){$\times$}}
\put(204,166){\makebox(0,0){$\times$}}
\put(205,166){\makebox(0,0){$\times$}}
\put(205,167){\makebox(0,0){$\times$}}
\put(206,168){\makebox(0,0){$\times$}}
\put(207,170){\makebox(0,0){$\times$}}
\put(208,170){\makebox(0,0){$\times$}}
\put(208,171){\makebox(0,0){$\times$}}
\put(209,172){\makebox(0,0){$\times$}}
\put(210,174){\makebox(0,0){$\times$}}
\put(210,175){\makebox(0,0){$\times$}}
\put(211,177){\makebox(0,0){$\times$}}
\put(212,178){\makebox(0,0){$\times$}}
\put(213,181){\makebox(0,0){$\times$}}
\put(213,182){\makebox(0,0){$\times$}}
\put(214,183){\makebox(0,0){$\times$}}
\put(215,185){\makebox(0,0){$\times$}}
\put(215,186){\makebox(0,0){$\times$}}
\put(216,189){\makebox(0,0){$\times$}}
\put(217,190){\makebox(0,0){$\times$}}
\put(218,193){\makebox(0,0){$\times$}}
\put(218,194){\makebox(0,0){$\times$}}
\put(219,197){\makebox(0,0){$\times$}}
\put(220,198){\makebox(0,0){$\times$}}
\put(220,201){\makebox(0,0){$\times$}}
\put(221,202){\makebox(0,0){$\times$}}
\put(222,205){\makebox(0,0){$\times$}}
\put(223,208){\makebox(0,0){$\times$}}
\put(223,211){\makebox(0,0){$\times$}}
\put(224,213){\makebox(0,0){$\times$}}
\put(225,216){\makebox(0,0){$\times$}}
\put(225,219){\makebox(0,0){$\times$}}
\put(226,221){\makebox(0,0){$\times$}}
\put(227,224){\makebox(0,0){$\times$}}
\put(228,227){\makebox(0,0){$\times$}}
\put(228,230){\makebox(0,0){$\times$}}
\put(229,234){\makebox(0,0){$\times$}}
\put(230,236){\makebox(0,0){$\times$}}
\put(230,239){\makebox(0,0){$\times$}}
\put(231,243){\makebox(0,0){$\times$}}
\put(232,246){\makebox(0,0){$\times$}}
\put(233,250){\makebox(0,0){$\times$}}
\put(233,254){\makebox(0,0){$\times$}}
\put(234,257){\makebox(0,0){$\times$}}
\put(235,261){\makebox(0,0){$\times$}}
\put(235,265){\makebox(0,0){$\times$}}
\put(236,269){\makebox(0,0){$\times$}}
\put(237,273){\makebox(0,0){$\times$}}
\put(238,277){\makebox(0,0){$\times$}}
\put(238,281){\makebox(0,0){$\times$}}
\put(239,285){\makebox(0,0){$\times$}}
\put(240,289){\makebox(0,0){$\times$}}
\put(240,294){\makebox(0,0){$\times$}}
\put(241,298){\makebox(0,0){$\times$}}
\put(242,302){\makebox(0,0){$\times$}}
\put(243,307){\makebox(0,0){$\times$}}
\put(243,311){\makebox(0,0){$\times$}}
\put(244,315){\makebox(0,0){$\times$}}
\put(245,321){\makebox(0,0){$\times$}}
\put(245,325){\makebox(0,0){$\times$}}
\put(246,330){\makebox(0,0){$\times$}}
\put(247,334){\makebox(0,0){$\times$}}
\put(248,340){\makebox(0,0){$\times$}}
\put(248,344){\makebox(0,0){$\times$}}
\put(249,349){\makebox(0,0){$\times$}}
\put(250,355){\makebox(0,0){$\times$}}
\put(250,359){\makebox(0,0){$\times$}}
\put(251,364){\makebox(0,0){$\times$}}
\put(252,370){\makebox(0,0){$\times$}}
\put(253,374){\makebox(0,0){$\times$}}
\put(253,379){\makebox(0,0){$\times$}}
\put(254,385){\makebox(0,0){$\times$}}
\put(255,390){\makebox(0,0){$\times$}}
\put(255,394){\makebox(0,0){$\times$}}
\put(256,400){\makebox(0,0){$\times$}}
\put(257,405){\makebox(0,0){$\times$}}
\put(258,411){\makebox(0,0){$\times$}}
\put(258,416){\makebox(0,0){$\times$}}
\put(259,421){\makebox(0,0){$\times$}}
\put(260,427){\makebox(0,0){$\times$}}
\put(260,431){\makebox(0,0){$\times$}}
\put(261,436){\makebox(0,0){$\times$}}
\put(262,442){\makebox(0,0){$\times$}}
\put(263,447){\makebox(0,0){$\times$}}
\put(263,453){\makebox(0,0){$\times$}}
\put(264,460){\makebox(0,0){$\times$}}
\put(265,465){\makebox(0,0){$\times$}}
\put(265,470){\makebox(0,0){$\times$}}
\put(266,476){\makebox(0,0){$\times$}}
\put(267,481){\makebox(0,0){$\times$}}
\put(268,487){\makebox(0,0){$\times$}}
\put(268,492){\makebox(0,0){$\times$}}
\put(269,498){\makebox(0,0){$\times$}}
\put(270,503){\makebox(0,0){$\times$}}
\put(270,509){\makebox(0,0){$\times$}}
\put(271,514){\makebox(0,0){$\times$}}
\put(272,521){\makebox(0,0){$\times$}}
\put(273,526){\makebox(0,0){$\times$}}
\put(273,532){\makebox(0,0){$\times$}}
\put(274,537){\makebox(0,0){$\times$}}
\put(275,543){\makebox(0,0){$\times$}}
\put(276,548){\makebox(0,0){$\times$}}
\put(277,559){\makebox(0,0){$\times$}}
\put(278,564){\makebox(0,0){$\times$}}
\put(278,570){\makebox(0,0){$\times$}}
\put(279,575){\makebox(0,0){$\times$}}
\put(280,581){\makebox(0,0){$\times$}}
\put(281,586){\makebox(0,0){$\times$}}
\put(281,592){\makebox(0,0){$\times$}}
\put(282,597){\makebox(0,0){$\times$}}
\put(283,602){\makebox(0,0){$\times$}}
\put(283,608){\makebox(0,0){$\times$}}
\put(284,613){\makebox(0,0){$\times$}}
\put(285,619){\makebox(0,0){$\times$}}
\put(286,624){\makebox(0,0){$\times$}}
\put(286,630){\makebox(0,0){$\times$}}
\put(287,635){\makebox(0,0){$\times$}}
\put(288,639){\makebox(0,0){$\times$}}
\put(288,645){\makebox(0,0){$\times$}}
\put(289,650){\makebox(0,0){$\times$}}
\put(290,656){\makebox(0,0){$\times$}}
\put(291,660){\makebox(0,0){$\times$}}
\put(291,665){\makebox(0,0){$\times$}}
\put(292,671){\makebox(0,0){$\times$}}
\put(293,676){\makebox(0,0){$\times$}}
\put(293,680){\makebox(0,0){$\times$}}
\put(294,685){\makebox(0,0){$\times$}}
\put(295,690){\makebox(0,0){$\times$}}
\put(296,695){\makebox(0,0){$\times$}}
\put(296,699){\makebox(0,0){$\times$}}
\put(297,705){\makebox(0,0){$\times$}}
\put(298,709){\makebox(0,0){$\times$}}
\put(298,713){\makebox(0,0){$\times$}}
\put(299,718){\makebox(0,0){$\times$}}
\put(300,722){\makebox(0,0){$\times$}}
\put(301,726){\makebox(0,0){$\times$}}
\put(301,730){\makebox(0,0){$\times$}}
\put(302,736){\makebox(0,0){$\times$}}
\put(303,740){\makebox(0,0){$\times$}}
\put(303,744){\makebox(0,0){$\times$}}
\put(304,748){\makebox(0,0){$\times$}}
\put(305,752){\makebox(0,0){$\times$}}
\put(306,756){\makebox(0,0){$\times$}}
\put(306,760){\makebox(0,0){$\times$}}
\put(307,763){\makebox(0,0){$\times$}}
\put(308,767){\makebox(0,0){$\times$}}
\put(308,771){\makebox(0,0){$\times$}}
\put(309,775){\makebox(0,0){$\times$}}
\put(310,778){\makebox(0,0){$\times$}}
\put(311,782){\makebox(0,0){$\times$}}
\put(311,786){\makebox(0,0){$\times$}}
\put(312,789){\makebox(0,0){$\times$}}
\put(313,793){\makebox(0,0){$\times$}}
\put(313,796){\makebox(0,0){$\times$}}
\put(314,798){\makebox(0,0){$\times$}}
\put(315,801){\makebox(0,0){$\times$}}
\put(316,805){\makebox(0,0){$\times$}}
\put(316,808){\makebox(0,0){$\times$}}
\put(317,811){\makebox(0,0){$\times$}}
\put(318,813){\makebox(0,0){$\times$}}
\put(318,816){\makebox(0,0){$\times$}}
\put(319,819){\makebox(0,0){$\times$}}
\put(320,822){\makebox(0,0){$\times$}}
\put(321,824){\makebox(0,0){$\times$}}
\put(321,826){\makebox(0,0){$\times$}}
\put(322,828){\makebox(0,0){$\times$}}
\put(323,831){\makebox(0,0){$\times$}}
\put(323,832){\makebox(0,0){$\times$}}
\put(324,835){\makebox(0,0){$\times$}}
\put(325,837){\makebox(0,0){$\times$}}
\put(326,838){\makebox(0,0){$\times$}}
\put(326,841){\makebox(0,0){$\times$}}
\put(327,842){\makebox(0,0){$\times$}}
\put(328,843){\makebox(0,0){$\times$}}
\put(328,845){\makebox(0,0){$\times$}}
\put(329,846){\makebox(0,0){$\times$}}
\put(330,847){\makebox(0,0){$\times$}}
\put(331,849){\makebox(0,0){$\times$}}
\put(331,850){\makebox(0,0){$\times$}}
\put(332,852){\makebox(0,0){$\times$}}
\put(333,853){\makebox(0,0){$\times$}}
\put(333,853){\makebox(0,0){$\times$}}
\put(334,854){\makebox(0,0){$\times$}}
\put(335,854){\makebox(0,0){$\times$}}
\put(336,856){\makebox(0,0){$\times$}}
\put(336,856){\makebox(0,0){$\times$}}
\put(337,856){\makebox(0,0){$\times$}}
\put(338,857){\makebox(0,0){$\times$}}
\put(338,857){\makebox(0,0){$\times$}}
\put(339,857){\makebox(0,0){$\times$}}
\put(340,857){\makebox(0,0){$\times$}}
\put(341,857){\makebox(0,0){$\times$}}
\put(341,857){\makebox(0,0){$\times$}}
\put(342,857){\makebox(0,0){$\times$}}
\put(343,857){\makebox(0,0){$\times$}}
\put(343,857){\makebox(0,0){$\times$}}
\put(344,857){\makebox(0,0){$\times$}}
\put(345,857){\makebox(0,0){$\times$}}
\put(346,856){\makebox(0,0){$\times$}}
\put(346,856){\makebox(0,0){$\times$}}
\put(347,854){\makebox(0,0){$\times$}}
\put(348,854){\makebox(0,0){$\times$}}
\put(348,853){\makebox(0,0){$\times$}}
\put(349,853){\makebox(0,0){$\times$}}
\put(350,852){\makebox(0,0){$\times$}}
\put(351,850){\makebox(0,0){$\times$}}
\put(351,849){\makebox(0,0){$\times$}}
\put(352,847){\makebox(0,0){$\times$}}
\put(353,846){\makebox(0,0){$\times$}}
\put(354,845){\makebox(0,0){$\times$}}
\put(354,843){\makebox(0,0){$\times$}}
\put(355,842){\makebox(0,0){$\times$}}
\put(356,841){\makebox(0,0){$\times$}}
\put(356,839){\makebox(0,0){$\times$}}
\put(357,837){\makebox(0,0){$\times$}}
\put(358,835){\makebox(0,0){$\times$}}
\put(359,834){\makebox(0,0){$\times$}}
\put(359,831){\makebox(0,0){$\times$}}
\put(360,830){\makebox(0,0){$\times$}}
\put(361,827){\makebox(0,0){$\times$}}
\put(361,826){\makebox(0,0){$\times$}}
\put(362,823){\makebox(0,0){$\times$}}
\put(363,820){\makebox(0,0){$\times$}}
\put(364,817){\makebox(0,0){$\times$}}
\put(364,816){\makebox(0,0){$\times$}}
\put(365,813){\makebox(0,0){$\times$}}
\put(366,811){\makebox(0,0){$\times$}}
\put(366,808){\makebox(0,0){$\times$}}
\put(367,805){\makebox(0,0){$\times$}}
\put(368,803){\makebox(0,0){$\times$}}
\put(369,800){\makebox(0,0){$\times$}}
\put(369,797){\makebox(0,0){$\times$}}
\put(370,794){\makebox(0,0){$\times$}}
\put(371,792){\makebox(0,0){$\times$}}
\put(371,788){\makebox(0,0){$\times$}}
\put(372,785){\makebox(0,0){$\times$}}
\put(373,782){\makebox(0,0){$\times$}}
\put(374,778){\makebox(0,0){$\times$}}
\put(374,775){\makebox(0,0){$\times$}}
\put(375,773){\makebox(0,0){$\times$}}
\put(376,769){\makebox(0,0){$\times$}}
\put(376,766){\makebox(0,0){$\times$}}
\put(377,762){\makebox(0,0){$\times$}}
\put(378,759){\makebox(0,0){$\times$}}
\put(379,755){\makebox(0,0){$\times$}}
\put(379,751){\makebox(0,0){$\times$}}
\put(380,748){\makebox(0,0){$\times$}}
\put(381,744){\makebox(0,0){$\times$}}
\put(381,740){\makebox(0,0){$\times$}}
\put(382,736){\makebox(0,0){$\times$}}
\put(383,733){\makebox(0,0){$\times$}}
\put(384,729){\makebox(0,0){$\times$}}
\put(384,725){\makebox(0,0){$\times$}}
\put(385,721){\makebox(0,0){$\times$}}
\put(386,717){\makebox(0,0){$\times$}}
\put(386,713){\makebox(0,0){$\times$}}
\put(387,709){\makebox(0,0){$\times$}}
\put(388,705){\makebox(0,0){$\times$}}
\put(389,700){\makebox(0,0){$\times$}}
\put(389,696){\makebox(0,0){$\times$}}
\put(390,692){\makebox(0,0){$\times$}}
\put(391,688){\makebox(0,0){$\times$}}
\put(391,684){\makebox(0,0){$\times$}}
\put(392,679){\makebox(0,0){$\times$}}
\put(393,675){\makebox(0,0){$\times$}}
\put(394,671){\makebox(0,0){$\times$}}
\put(394,666){\makebox(0,0){$\times$}}
\put(395,662){\makebox(0,0){$\times$}}
\put(396,657){\makebox(0,0){$\times$}}
\put(396,653){\makebox(0,0){$\times$}}
\put(397,649){\makebox(0,0){$\times$}}
\put(398,643){\makebox(0,0){$\times$}}
\put(399,639){\makebox(0,0){$\times$}}
\put(399,635){\makebox(0,0){$\times$}}
\put(400,631){\makebox(0,0){$\times$}}
\put(401,626){\makebox(0,0){$\times$}}
\put(401,622){\makebox(0,0){$\times$}}
\put(402,617){\makebox(0,0){$\times$}}
\put(403,612){\makebox(0,0){$\times$}}
\put(404,608){\makebox(0,0){$\times$}}
\put(404,604){\makebox(0,0){$\times$}}
\put(405,598){\makebox(0,0){$\times$}}
\put(406,594){\makebox(0,0){$\times$}}
\put(406,589){\makebox(0,0){$\times$}}
\put(407,585){\makebox(0,0){$\times$}}
\put(408,581){\makebox(0,0){$\times$}}
\put(409,577){\makebox(0,0){$\times$}}
\put(409,571){\makebox(0,0){$\times$}}
\put(410,567){\makebox(0,0){$\times$}}
\put(411,563){\makebox(0,0){$\times$}}
\put(411,558){\makebox(0,0){$\times$}}
\put(414,544){\makebox(0,0){$\times$}}
\put(414,540){\makebox(0,0){$\times$}}
\put(415,536){\makebox(0,0){$\times$}}
\put(416,532){\makebox(0,0){$\times$}}
\put(416,526){\makebox(0,0){$\times$}}
\put(417,522){\makebox(0,0){$\times$}}
\put(418,518){\makebox(0,0){$\times$}}
\put(419,514){\makebox(0,0){$\times$}}
\put(419,510){\makebox(0,0){$\times$}}
\put(420,505){\makebox(0,0){$\times$}}
\put(421,500){\makebox(0,0){$\times$}}
\put(421,496){\makebox(0,0){$\times$}}
\put(422,492){\makebox(0,0){$\times$}}
\put(423,488){\makebox(0,0){$\times$}}
\put(424,484){\makebox(0,0){$\times$}}
\put(424,480){\makebox(0,0){$\times$}}
\put(425,476){\makebox(0,0){$\times$}}
\put(426,472){\makebox(0,0){$\times$}}
\put(426,468){\makebox(0,0){$\times$}}
\put(427,464){\makebox(0,0){$\times$}}
\put(428,460){\makebox(0,0){$\times$}}
\put(429,456){\makebox(0,0){$\times$}}
\put(429,451){\makebox(0,0){$\times$}}
\put(430,447){\makebox(0,0){$\times$}}
\put(431,443){\makebox(0,0){$\times$}}
\put(431,439){\makebox(0,0){$\times$}}
\put(432,435){\makebox(0,0){$\times$}}
\put(433,432){\makebox(0,0){$\times$}}
\put(434,428){\makebox(0,0){$\times$}}
\put(434,424){\makebox(0,0){$\times$}}
\put(435,421){\makebox(0,0){$\times$}}
\put(436,417){\makebox(0,0){$\times$}}
\put(437,413){\makebox(0,0){$\times$}}
\put(437,411){\makebox(0,0){$\times$}}
\put(438,407){\makebox(0,0){$\times$}}
\put(439,404){\makebox(0,0){$\times$}}
\put(439,400){\makebox(0,0){$\times$}}
\put(440,397){\makebox(0,0){$\times$}}
\put(441,393){\makebox(0,0){$\times$}}
\put(442,390){\makebox(0,0){$\times$}}
\put(442,387){\makebox(0,0){$\times$}}
\put(443,385){\makebox(0,0){$\times$}}
\put(444,381){\makebox(0,0){$\times$}}
\put(444,378){\makebox(0,0){$\times$}}
\put(445,375){\makebox(0,0){$\times$}}
\put(446,373){\makebox(0,0){$\times$}}
\put(447,370){\makebox(0,0){$\times$}}
\put(447,367){\makebox(0,0){$\times$}}
\put(448,364){\makebox(0,0){$\times$}}
\put(449,362){\makebox(0,0){$\times$}}
\put(449,359){\makebox(0,0){$\times$}}
\put(450,356){\makebox(0,0){$\times$}}
\put(451,353){\makebox(0,0){$\times$}}
\put(452,351){\makebox(0,0){$\times$}}
\put(452,348){\makebox(0,0){$\times$}}
\put(453,347){\makebox(0,0){$\times$}}
\put(454,344){\makebox(0,0){$\times$}}
\put(454,343){\makebox(0,0){$\times$}}
\put(455,340){\makebox(0,0){$\times$}}
\put(456,337){\makebox(0,0){$\times$}}
\put(457,336){\makebox(0,0){$\times$}}
\put(457,334){\makebox(0,0){$\times$}}
\put(458,332){\makebox(0,0){$\times$}}
\put(459,330){\makebox(0,0){$\times$}}
\put(459,329){\makebox(0,0){$\times$}}
\put(460,328){\makebox(0,0){$\times$}}
\put(461,326){\makebox(0,0){$\times$}}
\put(462,325){\makebox(0,0){$\times$}}
\put(462,322){\makebox(0,0){$\times$}}
\put(463,321){\makebox(0,0){$\times$}}
\put(464,321){\makebox(0,0){$\times$}}
\put(464,319){\makebox(0,0){$\times$}}
\put(465,318){\makebox(0,0){$\times$}}
\put(466,317){\makebox(0,0){$\times$}}
\put(467,315){\makebox(0,0){$\times$}}
\put(467,315){\makebox(0,0){$\times$}}
\put(468,314){\makebox(0,0){$\times$}}
\put(469,314){\makebox(0,0){$\times$}}
\put(469,313){\makebox(0,0){$\times$}}
\put(470,313){\makebox(0,0){$\times$}}
\put(471,311){\makebox(0,0){$\times$}}
\put(472,311){\makebox(0,0){$\times$}}
\put(472,311){\makebox(0,0){$\times$}}
\put(473,310){\makebox(0,0){$\times$}}
\put(474,310){\makebox(0,0){$\times$}}
\put(474,310){\makebox(0,0){$\times$}}
\put(475,310){\makebox(0,0){$\times$}}
\put(476,310){\makebox(0,0){$\times$}}
\put(477,310){\makebox(0,0){$\times$}}
\put(477,310){\makebox(0,0){$\times$}}
\put(478,310){\makebox(0,0){$\times$}}
\put(479,310){\makebox(0,0){$\times$}}
\put(479,310){\makebox(0,0){$\times$}}
\put(480,310){\makebox(0,0){$\times$}}
\put(481,311){\makebox(0,0){$\times$}}
\put(482,311){\makebox(0,0){$\times$}}
\put(482,311){\makebox(0,0){$\times$}}
\put(483,313){\makebox(0,0){$\times$}}
\put(484,314){\makebox(0,0){$\times$}}
\put(484,314){\makebox(0,0){$\times$}}
\put(485,315){\makebox(0,0){$\times$}}
\put(486,315){\makebox(0,0){$\times$}}
\put(487,317){\makebox(0,0){$\times$}}
\put(487,318){\makebox(0,0){$\times$}}
\put(488,319){\makebox(0,0){$\times$}}
\put(489,319){\makebox(0,0){$\times$}}
\put(489,321){\makebox(0,0){$\times$}}
\put(490,322){\makebox(0,0){$\times$}}
\put(491,324){\makebox(0,0){$\times$}}
\put(492,325){\makebox(0,0){$\times$}}
\put(492,326){\makebox(0,0){$\times$}}
\put(493,328){\makebox(0,0){$\times$}}
\put(494,329){\makebox(0,0){$\times$}}
\put(494,332){\makebox(0,0){$\times$}}
\put(495,333){\makebox(0,0){$\times$}}
\put(496,334){\makebox(0,0){$\times$}}
\put(497,336){\makebox(0,0){$\times$}}
\put(497,338){\makebox(0,0){$\times$}}
\put(498,340){\makebox(0,0){$\times$}}
\put(499,343){\makebox(0,0){$\times$}}
\put(499,344){\makebox(0,0){$\times$}}
\put(500,345){\makebox(0,0){$\times$}}
\put(501,348){\makebox(0,0){$\times$}}
\put(502,349){\makebox(0,0){$\times$}}
\put(502,352){\makebox(0,0){$\times$}}
\put(503,355){\makebox(0,0){$\times$}}
\put(504,356){\makebox(0,0){$\times$}}
\put(504,359){\makebox(0,0){$\times$}}
\put(505,362){\makebox(0,0){$\times$}}
\put(506,364){\makebox(0,0){$\times$}}
\put(507,366){\makebox(0,0){$\times$}}
\put(507,368){\makebox(0,0){$\times$}}
\put(508,371){\makebox(0,0){$\times$}}
\put(509,374){\makebox(0,0){$\times$}}
\put(509,377){\makebox(0,0){$\times$}}
\put(510,379){\makebox(0,0){$\times$}}
\put(511,382){\makebox(0,0){$\times$}}
\put(512,385){\makebox(0,0){$\times$}}
\put(512,387){\makebox(0,0){$\times$}}
\put(513,390){\makebox(0,0){$\times$}}
\put(514,393){\makebox(0,0){$\times$}}
\put(515,396){\makebox(0,0){$\times$}}
\put(515,398){\makebox(0,0){$\times$}}
\put(516,402){\makebox(0,0){$\times$}}
\put(517,405){\makebox(0,0){$\times$}}
\put(517,408){\makebox(0,0){$\times$}}
\put(518,411){\makebox(0,0){$\times$}}
\put(519,413){\makebox(0,0){$\times$}}
\put(520,417){\makebox(0,0){$\times$}}
\put(520,420){\makebox(0,0){$\times$}}
\put(521,423){\makebox(0,0){$\times$}}
\put(522,427){\makebox(0,0){$\times$}}
\put(522,430){\makebox(0,0){$\times$}}
\put(523,434){\makebox(0,0){$\times$}}
\put(524,436){\makebox(0,0){$\times$}}
\put(525,441){\makebox(0,0){$\times$}}
\put(525,443){\makebox(0,0){$\times$}}
\put(526,446){\makebox(0,0){$\times$}}
\put(527,450){\makebox(0,0){$\times$}}
\put(527,453){\makebox(0,0){$\times$}}
\put(528,457){\makebox(0,0){$\times$}}
\put(529,461){\makebox(0,0){$\times$}}
\put(530,464){\makebox(0,0){$\times$}}
\put(530,468){\makebox(0,0){$\times$}}
\put(531,470){\makebox(0,0){$\times$}}
\put(532,475){\makebox(0,0){$\times$}}
\put(532,477){\makebox(0,0){$\times$}}
\put(533,481){\makebox(0,0){$\times$}}
\put(534,485){\makebox(0,0){$\times$}}
\put(535,488){\makebox(0,0){$\times$}}
\put(535,492){\makebox(0,0){$\times$}}
\put(536,495){\makebox(0,0){$\times$}}
\put(537,499){\makebox(0,0){$\times$}}
\put(537,503){\makebox(0,0){$\times$}}
\put(538,506){\makebox(0,0){$\times$}}
\put(539,510){\makebox(0,0){$\times$}}
\put(540,514){\makebox(0,0){$\times$}}
\put(540,517){\makebox(0,0){$\times$}}
\put(541,521){\makebox(0,0){$\times$}}
\put(542,525){\makebox(0,0){$\times$}}
\put(542,528){\makebox(0,0){$\times$}}
\put(543,532){\makebox(0,0){$\times$}}
\put(544,536){\makebox(0,0){$\times$}}
\put(545,539){\makebox(0,0){$\times$}}
\put(545,543){\makebox(0,0){$\times$}}
\put(548,556){\makebox(0,0){$\times$}}
\put(549,560){\makebox(0,0){$\times$}}
\put(550,564){\makebox(0,0){$\times$}}
\put(550,567){\makebox(0,0){$\times$}}
\put(551,571){\makebox(0,0){$\times$}}
\put(552,575){\makebox(0,0){$\times$}}
\put(552,578){\makebox(0,0){$\times$}}
\put(553,582){\makebox(0,0){$\times$}}
\put(554,585){\makebox(0,0){$\times$}}
\put(555,589){\makebox(0,0){$\times$}}
\put(555,592){\makebox(0,0){$\times$}}
\put(556,596){\makebox(0,0){$\times$}}
\put(557,598){\makebox(0,0){$\times$}}
\put(557,602){\makebox(0,0){$\times$}}
\put(558,605){\makebox(0,0){$\times$}}
\put(559,609){\makebox(0,0){$\times$}}
\put(560,612){\makebox(0,0){$\times$}}
\put(560,616){\makebox(0,0){$\times$}}
\put(561,619){\makebox(0,0){$\times$}}
\put(562,622){\makebox(0,0){$\times$}}
\put(562,626){\makebox(0,0){$\times$}}
\put(563,628){\makebox(0,0){$\times$}}
\put(564,631){\makebox(0,0){$\times$}}
\put(565,635){\makebox(0,0){$\times$}}
\put(565,638){\makebox(0,0){$\times$}}
\put(566,641){\makebox(0,0){$\times$}}
\put(567,643){\makebox(0,0){$\times$}}
\put(567,647){\makebox(0,0){$\times$}}
\put(568,650){\makebox(0,0){$\times$}}
\put(569,653){\makebox(0,0){$\times$}}
\put(570,656){\makebox(0,0){$\times$}}
\put(570,658){\makebox(0,0){$\times$}}
\put(571,661){\makebox(0,0){$\times$}}
\put(572,664){\makebox(0,0){$\times$}}
\put(572,668){\makebox(0,0){$\times$}}
\put(573,671){\makebox(0,0){$\times$}}
\put(574,673){\makebox(0,0){$\times$}}
\put(575,675){\makebox(0,0){$\times$}}
\put(575,677){\makebox(0,0){$\times$}}
\put(576,680){\makebox(0,0){$\times$}}
\put(577,683){\makebox(0,0){$\times$}}
\put(577,685){\makebox(0,0){$\times$}}
\put(578,688){\makebox(0,0){$\times$}}
\put(579,691){\makebox(0,0){$\times$}}
\put(580,692){\makebox(0,0){$\times$}}
\put(580,695){\makebox(0,0){$\times$}}
\put(581,698){\makebox(0,0){$\times$}}
\put(582,699){\makebox(0,0){$\times$}}
\put(582,702){\makebox(0,0){$\times$}}
\put(583,703){\makebox(0,0){$\times$}}
\put(584,706){\makebox(0,0){$\times$}}
\put(585,707){\makebox(0,0){$\times$}}
\put(585,710){\makebox(0,0){$\times$}}
\put(586,711){\makebox(0,0){$\times$}}
\put(587,714){\makebox(0,0){$\times$}}
\put(587,715){\makebox(0,0){$\times$}}
\put(588,717){\makebox(0,0){$\times$}}
\put(589,720){\makebox(0,0){$\times$}}
\put(590,721){\makebox(0,0){$\times$}}
\put(590,722){\makebox(0,0){$\times$}}
\put(591,724){\makebox(0,0){$\times$}}
\put(592,725){\makebox(0,0){$\times$}}
\put(592,728){\makebox(0,0){$\times$}}
\put(593,729){\makebox(0,0){$\times$}}
\put(594,730){\makebox(0,0){$\times$}}
\put(595,730){\makebox(0,0){$\times$}}
\put(595,733){\makebox(0,0){$\times$}}
\put(596,733){\makebox(0,0){$\times$}}
\put(597,734){\makebox(0,0){$\times$}}
\put(598,736){\makebox(0,0){$\times$}}
\put(598,737){\makebox(0,0){$\times$}}
\put(599,739){\makebox(0,0){$\times$}}
\put(600,739){\makebox(0,0){$\times$}}
\put(600,740){\makebox(0,0){$\times$}}
\put(601,741){\makebox(0,0){$\times$}}
\put(602,741){\makebox(0,0){$\times$}}
\put(603,743){\makebox(0,0){$\times$}}
\put(603,743){\makebox(0,0){$\times$}}
\put(604,744){\makebox(0,0){$\times$}}
\put(605,744){\makebox(0,0){$\times$}}
\put(605,744){\makebox(0,0){$\times$}}
\put(606,745){\makebox(0,0){$\times$}}
\put(607,745){\makebox(0,0){$\times$}}
\put(608,745){\makebox(0,0){$\times$}}
\put(608,745){\makebox(0,0){$\times$}}
\put(609,747){\makebox(0,0){$\times$}}
\put(610,747){\makebox(0,0){$\times$}}
\put(610,747){\makebox(0,0){$\times$}}
\put(611,747){\makebox(0,0){$\times$}}
\put(612,747){\makebox(0,0){$\times$}}
\put(613,747){\makebox(0,0){$\times$}}
\put(613,747){\makebox(0,0){$\times$}}
\put(614,747){\makebox(0,0){$\times$}}
\put(615,747){\makebox(0,0){$\times$}}
\put(615,747){\makebox(0,0){$\times$}}
\put(616,747){\makebox(0,0){$\times$}}
\put(617,745){\makebox(0,0){$\times$}}
\put(618,745){\makebox(0,0){$\times$}}
\put(618,745){\makebox(0,0){$\times$}}
\put(619,744){\makebox(0,0){$\times$}}
\put(620,744){\makebox(0,0){$\times$}}
\put(620,743){\makebox(0,0){$\times$}}
\put(621,743){\makebox(0,0){$\times$}}
\put(622,741){\makebox(0,0){$\times$}}
\put(623,741){\makebox(0,0){$\times$}}
\put(623,740){\makebox(0,0){$\times$}}
\put(624,740){\makebox(0,0){$\times$}}
\put(625,739){\makebox(0,0){$\times$}}
\put(625,737){\makebox(0,0){$\times$}}
\put(626,736){\makebox(0,0){$\times$}}
\put(627,736){\makebox(0,0){$\times$}}
\put(628,734){\makebox(0,0){$\times$}}
\put(628,733){\makebox(0,0){$\times$}}
\put(629,732){\makebox(0,0){$\times$}}
\put(630,730){\makebox(0,0){$\times$}}
\put(630,729){\makebox(0,0){$\times$}}
\put(631,728){\makebox(0,0){$\times$}}
\put(632,726){\makebox(0,0){$\times$}}
\put(633,725){\makebox(0,0){$\times$}}
\put(633,724){\makebox(0,0){$\times$}}
\put(634,722){\makebox(0,0){$\times$}}
\put(635,721){\makebox(0,0){$\times$}}
\put(635,718){\makebox(0,0){$\times$}}
\put(636,717){\makebox(0,0){$\times$}}
\put(637,715){\makebox(0,0){$\times$}}
\put(638,714){\makebox(0,0){$\times$}}
\put(638,711){\makebox(0,0){$\times$}}
\put(639,710){\makebox(0,0){$\times$}}
\put(640,709){\makebox(0,0){$\times$}}
\put(640,706){\makebox(0,0){$\times$}}
\put(641,705){\makebox(0,0){$\times$}}
\put(642,702){\makebox(0,0){$\times$}}
\put(643,700){\makebox(0,0){$\times$}}
\put(643,698){\makebox(0,0){$\times$}}
\put(644,696){\makebox(0,0){$\times$}}
\put(645,694){\makebox(0,0){$\times$}}
\put(645,692){\makebox(0,0){$\times$}}
\put(646,690){\makebox(0,0){$\times$}}
\put(647,687){\makebox(0,0){$\times$}}
\put(648,685){\makebox(0,0){$\times$}}
\put(648,683){\makebox(0,0){$\times$}}
\put(649,680){\makebox(0,0){$\times$}}
\put(650,679){\makebox(0,0){$\times$}}
\put(650,676){\makebox(0,0){$\times$}}
\put(651,673){\makebox(0,0){$\times$}}
\put(652,671){\makebox(0,0){$\times$}}
\put(653,669){\makebox(0,0){$\times$}}
\put(653,666){\makebox(0,0){$\times$}}
\put(654,664){\makebox(0,0){$\times$}}
\put(655,661){\makebox(0,0){$\times$}}
\put(655,658){\makebox(0,0){$\times$}}
\put(656,656){\makebox(0,0){$\times$}}
\put(657,653){\makebox(0,0){$\times$}}
\put(658,650){\makebox(0,0){$\times$}}
\put(658,647){\makebox(0,0){$\times$}}
\put(659,646){\makebox(0,0){$\times$}}
\put(660,642){\makebox(0,0){$\times$}}
\put(660,641){\makebox(0,0){$\times$}}
\put(661,636){\makebox(0,0){$\times$}}
\put(662,635){\makebox(0,0){$\times$}}
\put(663,631){\makebox(0,0){$\times$}}
\put(663,628){\makebox(0,0){$\times$}}
\put(664,626){\makebox(0,0){$\times$}}
\put(665,623){\makebox(0,0){$\times$}}
\put(665,620){\makebox(0,0){$\times$}}
\put(666,617){\makebox(0,0){$\times$}}
\put(667,615){\makebox(0,0){$\times$}}
\put(668,612){\makebox(0,0){$\times$}}
\put(668,609){\makebox(0,0){$\times$}}
\put(669,607){\makebox(0,0){$\times$}}
\put(670,604){\makebox(0,0){$\times$}}
\put(670,600){\makebox(0,0){$\times$}}
\put(671,598){\makebox(0,0){$\times$}}
\put(672,594){\makebox(0,0){$\times$}}
\put(673,592){\makebox(0,0){$\times$}}
\put(673,589){\makebox(0,0){$\times$}}
\put(674,586){\makebox(0,0){$\times$}}
\put(675,583){\makebox(0,0){$\times$}}
\put(676,581){\makebox(0,0){$\times$}}
\put(676,577){\makebox(0,0){$\times$}}
\put(677,574){\makebox(0,0){$\times$}}
\put(678,571){\makebox(0,0){$\times$}}
\put(678,568){\makebox(0,0){$\times$}}
\put(679,566){\makebox(0,0){$\times$}}
\put(680,563){\makebox(0,0){$\times$}}
\put(681,560){\makebox(0,0){$\times$}}
\put(681,556){\makebox(0,0){$\times$}}
\put(683,548){\makebox(0,0){$\times$}}
\put(684,545){\makebox(0,0){$\times$}}
\put(685,543){\makebox(0,0){$\times$}}
\put(686,540){\makebox(0,0){$\times$}}
\put(686,537){\makebox(0,0){$\times$}}
\put(687,534){\makebox(0,0){$\times$}}
\put(688,532){\makebox(0,0){$\times$}}
\put(688,528){\makebox(0,0){$\times$}}
\put(689,525){\makebox(0,0){$\times$}}
\put(690,522){\makebox(0,0){$\times$}}
\put(691,519){\makebox(0,0){$\times$}}
\put(691,517){\makebox(0,0){$\times$}}
\put(692,514){\makebox(0,0){$\times$}}
\put(693,511){\makebox(0,0){$\times$}}
\put(693,509){\makebox(0,0){$\times$}}
\put(694,506){\makebox(0,0){$\times$}}
\put(695,503){\makebox(0,0){$\times$}}
\put(696,500){\makebox(0,0){$\times$}}
\put(696,499){\makebox(0,0){$\times$}}
\put(697,496){\makebox(0,0){$\times$}}
\put(698,494){\makebox(0,0){$\times$}}
\put(698,491){\makebox(0,0){$\times$}}
\put(699,488){\makebox(0,0){$\times$}}
\put(700,485){\makebox(0,0){$\times$}}
\put(701,483){\makebox(0,0){$\times$}}
\put(701,480){\makebox(0,0){$\times$}}
\put(702,479){\makebox(0,0){$\times$}}
\put(703,476){\makebox(0,0){$\times$}}
\put(703,473){\makebox(0,0){$\times$}}
\put(704,470){\makebox(0,0){$\times$}}
\put(705,469){\makebox(0,0){$\times$}}
\put(706,466){\makebox(0,0){$\times$}}
\put(706,464){\makebox(0,0){$\times$}}
\put(707,462){\makebox(0,0){$\times$}}
\put(708,460){\makebox(0,0){$\times$}}
\put(708,457){\makebox(0,0){$\times$}}
\put(709,456){\makebox(0,0){$\times$}}
\put(710,453){\makebox(0,0){$\times$}}
\put(711,451){\makebox(0,0){$\times$}}
\put(711,449){\makebox(0,0){$\times$}}
\put(712,447){\makebox(0,0){$\times$}}
\put(713,445){\makebox(0,0){$\times$}}
\put(713,443){\makebox(0,0){$\times$}}
\put(714,442){\makebox(0,0){$\times$}}
\put(715,439){\makebox(0,0){$\times$}}
\put(716,438){\makebox(0,0){$\times$}}
\put(716,435){\makebox(0,0){$\times$}}
\put(717,434){\makebox(0,0){$\times$}}
\put(718,432){\makebox(0,0){$\times$}}
\put(718,431){\makebox(0,0){$\times$}}
\put(719,430){\makebox(0,0){$\times$}}
\put(720,427){\makebox(0,0){$\times$}}
\put(721,426){\makebox(0,0){$\times$}}
\put(721,424){\makebox(0,0){$\times$}}
\put(722,423){\makebox(0,0){$\times$}}
\put(723,421){\makebox(0,0){$\times$}}
\put(723,420){\makebox(0,0){$\times$}}
\put(724,419){\makebox(0,0){$\times$}}
\put(725,417){\makebox(0,0){$\times$}}
\put(726,416){\makebox(0,0){$\times$}}
\put(726,415){\makebox(0,0){$\times$}}
\put(727,413){\makebox(0,0){$\times$}}
\put(728,412){\makebox(0,0){$\times$}}
\put(728,412){\makebox(0,0){$\times$}}
\put(729,411){\makebox(0,0){$\times$}}
\put(730,409){\makebox(0,0){$\times$}}
\put(731,408){\makebox(0,0){$\times$}}
\put(731,408){\makebox(0,0){$\times$}}
\put(732,407){\makebox(0,0){$\times$}}
\put(733,405){\makebox(0,0){$\times$}}
\put(733,405){\makebox(0,0){$\times$}}
\put(734,404){\makebox(0,0){$\times$}}
\put(735,404){\makebox(0,0){$\times$}}
\put(736,402){\makebox(0,0){$\times$}}
\put(736,402){\makebox(0,0){$\times$}}
\put(737,401){\makebox(0,0){$\times$}}
\put(738,401){\makebox(0,0){$\times$}}
\put(738,401){\makebox(0,0){$\times$}}
\put(739,400){\makebox(0,0){$\times$}}
\put(740,400){\makebox(0,0){$\times$}}
\put(741,400){\makebox(0,0){$\times$}}
\put(741,398){\makebox(0,0){$\times$}}
\put(742,398){\makebox(0,0){$\times$}}
\put(743,398){\makebox(0,0){$\times$}}
\put(743,398){\makebox(0,0){$\times$}}
\put(744,398){\makebox(0,0){$\times$}}
\put(745,398){\makebox(0,0){$\times$}}
\put(746,398){\makebox(0,0){$\times$}}
\put(746,398){\makebox(0,0){$\times$}}
\put(747,398){\makebox(0,0){$\times$}}
\put(748,398){\makebox(0,0){$\times$}}
\put(748,398){\makebox(0,0){$\times$}}
\put(749,398){\makebox(0,0){$\times$}}
\put(750,398){\makebox(0,0){$\times$}}
\put(751,398){\makebox(0,0){$\times$}}
\put(751,398){\makebox(0,0){$\times$}}
\put(752,400){\makebox(0,0){$\times$}}
\put(753,400){\makebox(0,0){$\times$}}
\put(753,400){\makebox(0,0){$\times$}}
\put(754,401){\makebox(0,0){$\times$}}
\put(755,401){\makebox(0,0){$\times$}}
\put(756,402){\makebox(0,0){$\times$}}
\put(756,402){\makebox(0,0){$\times$}}
\put(757,402){\makebox(0,0){$\times$}}
\put(758,404){\makebox(0,0){$\times$}}
\put(759,404){\makebox(0,0){$\times$}}
\put(759,405){\makebox(0,0){$\times$}}
\put(760,407){\makebox(0,0){$\times$}}
\put(761,407){\makebox(0,0){$\times$}}
\put(761,408){\makebox(0,0){$\times$}}
\put(762,409){\makebox(0,0){$\times$}}
\put(763,409){\makebox(0,0){$\times$}}
\put(764,411){\makebox(0,0){$\times$}}
\put(764,412){\makebox(0,0){$\times$}}
\put(765,413){\makebox(0,0){$\times$}}
\put(766,413){\makebox(0,0){$\times$}}
\put(766,415){\makebox(0,0){$\times$}}
\put(767,416){\makebox(0,0){$\times$}}
\put(768,417){\makebox(0,0){$\times$}}
\put(769,419){\makebox(0,0){$\times$}}
\put(769,420){\makebox(0,0){$\times$}}
\put(770,421){\makebox(0,0){$\times$}}
\put(771,423){\makebox(0,0){$\times$}}
\put(771,424){\makebox(0,0){$\times$}}
\put(772,426){\makebox(0,0){$\times$}}
\put(773,427){\makebox(0,0){$\times$}}
\put(774,428){\makebox(0,0){$\times$}}
\put(774,430){\makebox(0,0){$\times$}}
\put(775,431){\makebox(0,0){$\times$}}
\put(776,432){\makebox(0,0){$\times$}}
\put(776,434){\makebox(0,0){$\times$}}
\put(777,436){\makebox(0,0){$\times$}}
\put(778,438){\makebox(0,0){$\times$}}
\put(779,439){\makebox(0,0){$\times$}}
\put(779,441){\makebox(0,0){$\times$}}
\put(780,442){\makebox(0,0){$\times$}}
\put(781,445){\makebox(0,0){$\times$}}
\put(781,446){\makebox(0,0){$\times$}}
\put(782,447){\makebox(0,0){$\times$}}
\put(783,450){\makebox(0,0){$\times$}}
\put(784,451){\makebox(0,0){$\times$}}
\put(784,453){\makebox(0,0){$\times$}}
\put(785,456){\makebox(0,0){$\times$}}
\put(786,457){\makebox(0,0){$\times$}}
\put(786,460){\makebox(0,0){$\times$}}
\put(787,461){\makebox(0,0){$\times$}}
\put(788,462){\makebox(0,0){$\times$}}
\put(789,465){\makebox(0,0){$\times$}}
\put(789,466){\makebox(0,0){$\times$}}
\put(790,469){\makebox(0,0){$\times$}}
\put(791,470){\makebox(0,0){$\times$}}
\put(791,473){\makebox(0,0){$\times$}}
\put(792,475){\makebox(0,0){$\times$}}
\put(793,477){\makebox(0,0){$\times$}}
\put(794,479){\makebox(0,0){$\times$}}
\put(794,481){\makebox(0,0){$\times$}}
\put(795,484){\makebox(0,0){$\times$}}
\put(796,485){\makebox(0,0){$\times$}}
\put(796,488){\makebox(0,0){$\times$}}
\put(797,490){\makebox(0,0){$\times$}}
\put(798,492){\makebox(0,0){$\times$}}
\put(799,494){\makebox(0,0){$\times$}}
\put(799,496){\makebox(0,0){$\times$}}
\put(800,499){\makebox(0,0){$\times$}}
\put(801,500){\makebox(0,0){$\times$}}
\put(801,503){\makebox(0,0){$\times$}}
\put(802,505){\makebox(0,0){$\times$}}
\put(803,507){\makebox(0,0){$\times$}}
\put(804,510){\makebox(0,0){$\times$}}
\put(804,511){\makebox(0,0){$\times$}}
\put(805,514){\makebox(0,0){$\times$}}
\put(806,517){\makebox(0,0){$\times$}}
\put(806,518){\makebox(0,0){$\times$}}
\put(807,521){\makebox(0,0){$\times$}}
\put(808,522){\makebox(0,0){$\times$}}
\put(809,525){\makebox(0,0){$\times$}}
\put(809,528){\makebox(0,0){$\times$}}
\put(810,529){\makebox(0,0){$\times$}}
\put(811,532){\makebox(0,0){$\times$}}
\put(811,534){\makebox(0,0){$\times$}}
\put(812,536){\makebox(0,0){$\times$}}
\put(813,539){\makebox(0,0){$\times$}}
\put(814,540){\makebox(0,0){$\times$}}
\put(814,543){\makebox(0,0){$\times$}}
\put(815,545){\makebox(0,0){$\times$}}
\put(819,556){\makebox(0,0){$\times$}}
\put(819,559){\makebox(0,0){$\times$}}
\put(820,560){\makebox(0,0){$\times$}}
\put(821,563){\makebox(0,0){$\times$}}
\put(821,564){\makebox(0,0){$\times$}}
\put(822,567){\makebox(0,0){$\times$}}
\put(823,568){\makebox(0,0){$\times$}}
\put(824,571){\makebox(0,0){$\times$}}
\put(824,573){\makebox(0,0){$\times$}}
\put(825,575){\makebox(0,0){$\times$}}
\put(826,578){\makebox(0,0){$\times$}}
\put(826,579){\makebox(0,0){$\times$}}
\put(827,582){\makebox(0,0){$\times$}}
\put(828,583){\makebox(0,0){$\times$}}
\put(829,586){\makebox(0,0){$\times$}}
\put(829,587){\makebox(0,0){$\times$}}
\put(830,590){\makebox(0,0){$\times$}}
\put(831,592){\makebox(0,0){$\times$}}
\put(831,594){\makebox(0,0){$\times$}}
\put(832,596){\makebox(0,0){$\times$}}
\put(833,598){\makebox(0,0){$\times$}}
\put(834,600){\makebox(0,0){$\times$}}
\put(834,601){\makebox(0,0){$\times$}}
\put(835,604){\makebox(0,0){$\times$}}
\put(836,605){\makebox(0,0){$\times$}}
\put(837,608){\makebox(0,0){$\times$}}
\put(837,609){\makebox(0,0){$\times$}}
\put(838,611){\makebox(0,0){$\times$}}
\put(839,613){\makebox(0,0){$\times$}}
\put(839,615){\makebox(0,0){$\times$}}
\put(840,616){\makebox(0,0){$\times$}}
\put(841,617){\makebox(0,0){$\times$}}
\put(842,620){\makebox(0,0){$\times$}}
\put(842,622){\makebox(0,0){$\times$}}
\put(843,623){\makebox(0,0){$\times$}}
\put(844,624){\makebox(0,0){$\times$}}
\put(844,626){\makebox(0,0){$\times$}}
\put(845,628){\makebox(0,0){$\times$}}
\put(846,630){\makebox(0,0){$\times$}}
\put(847,631){\makebox(0,0){$\times$}}
\put(847,632){\makebox(0,0){$\times$}}
\put(848,634){\makebox(0,0){$\times$}}
\put(849,635){\makebox(0,0){$\times$}}
\put(849,636){\makebox(0,0){$\times$}}
\put(850,638){\makebox(0,0){$\times$}}
\put(851,639){\makebox(0,0){$\times$}}
\put(852,641){\makebox(0,0){$\times$}}
\put(852,642){\makebox(0,0){$\times$}}
\put(853,643){\makebox(0,0){$\times$}}
\put(854,645){\makebox(0,0){$\times$}}
\put(854,646){\makebox(0,0){$\times$}}
\put(855,647){\makebox(0,0){$\times$}}
\put(856,649){\makebox(0,0){$\times$}}
\put(857,649){\makebox(0,0){$\times$}}
\put(857,650){\makebox(0,0){$\times$}}
\put(858,651){\makebox(0,0){$\times$}}
\put(859,653){\makebox(0,0){$\times$}}
\put(859,653){\makebox(0,0){$\times$}}
\put(860,654){\makebox(0,0){$\times$}}
\put(861,656){\makebox(0,0){$\times$}}
\put(862,656){\makebox(0,0){$\times$}}
\put(862,657){\makebox(0,0){$\times$}}
\put(863,658){\makebox(0,0){$\times$}}
\put(864,658){\makebox(0,0){$\times$}}
\put(864,660){\makebox(0,0){$\times$}}
\put(865,660){\makebox(0,0){$\times$}}
\put(866,661){\makebox(0,0){$\times$}}
\put(867,661){\makebox(0,0){$\times$}}
\put(867,662){\makebox(0,0){$\times$}}
\put(868,662){\makebox(0,0){$\times$}}
\put(869,662){\makebox(0,0){$\times$}}
\put(869,664){\makebox(0,0){$\times$}}
\put(870,664){\makebox(0,0){$\times$}}
\put(871,664){\makebox(0,0){$\times$}}
\put(872,665){\makebox(0,0){$\times$}}
\put(872,665){\makebox(0,0){$\times$}}
\put(873,665){\makebox(0,0){$\times$}}
\put(874,665){\makebox(0,0){$\times$}}
\put(874,665){\makebox(0,0){$\times$}}
\put(875,666){\makebox(0,0){$\times$}}
\put(876,666){\makebox(0,0){$\times$}}
\put(877,666){\makebox(0,0){$\times$}}
\put(877,666){\makebox(0,0){$\times$}}
\put(878,666){\makebox(0,0){$\times$}}
\put(879,666){\makebox(0,0){$\times$}}
\put(879,666){\makebox(0,0){$\times$}}
\put(880,666){\makebox(0,0){$\times$}}
\put(881,666){\makebox(0,0){$\times$}}
\put(882,666){\makebox(0,0){$\times$}}
\put(882,666){\makebox(0,0){$\times$}}
\put(883,666){\makebox(0,0){$\times$}}
\put(884,666){\makebox(0,0){$\times$}}
\put(884,666){\makebox(0,0){$\times$}}
\put(885,666){\makebox(0,0){$\times$}}
\put(886,665){\makebox(0,0){$\times$}}
\put(887,665){\makebox(0,0){$\times$}}
\put(887,665){\makebox(0,0){$\times$}}
\put(888,665){\makebox(0,0){$\times$}}
\put(889,664){\makebox(0,0){$\times$}}
\put(889,664){\makebox(0,0){$\times$}}
\put(890,664){\makebox(0,0){$\times$}}
\put(891,662){\makebox(0,0){$\times$}}
\put(892,662){\makebox(0,0){$\times$}}
\put(892,662){\makebox(0,0){$\times$}}
\put(893,661){\makebox(0,0){$\times$}}
\put(894,661){\makebox(0,0){$\times$}}
\put(894,660){\makebox(0,0){$\times$}}
\put(895,660){\makebox(0,0){$\times$}}
\put(896,658){\makebox(0,0){$\times$}}
\put(897,658){\makebox(0,0){$\times$}}
\put(897,657){\makebox(0,0){$\times$}}
\put(898,656){\makebox(0,0){$\times$}}
\put(899,656){\makebox(0,0){$\times$}}
\put(899,654){\makebox(0,0){$\times$}}
\put(900,654){\makebox(0,0){$\times$}}
\put(901,653){\makebox(0,0){$\times$}}
\put(902,651){\makebox(0,0){$\times$}}
\put(902,651){\makebox(0,0){$\times$}}
\put(903,650){\makebox(0,0){$\times$}}
\put(904,649){\makebox(0,0){$\times$}}
\put(904,647){\makebox(0,0){$\times$}}
\put(905,647){\makebox(0,0){$\times$}}
\put(906,646){\makebox(0,0){$\times$}}
\put(907,645){\makebox(0,0){$\times$}}
\put(907,643){\makebox(0,0){$\times$}}
\put(908,642){\makebox(0,0){$\times$}}
\put(909,642){\makebox(0,0){$\times$}}
\put(909,641){\makebox(0,0){$\times$}}
\put(910,639){\makebox(0,0){$\times$}}
\put(911,638){\makebox(0,0){$\times$}}
\put(912,636){\makebox(0,0){$\times$}}
\put(912,635){\makebox(0,0){$\times$}}
\put(913,634){\makebox(0,0){$\times$}}
\put(914,632){\makebox(0,0){$\times$}}
\put(914,631){\makebox(0,0){$\times$}}
\put(915,631){\makebox(0,0){$\times$}}
\put(916,630){\makebox(0,0){$\times$}}
\put(917,627){\makebox(0,0){$\times$}}
\put(917,626){\makebox(0,0){$\times$}}
\put(918,626){\makebox(0,0){$\times$}}
\put(919,624){\makebox(0,0){$\times$}}
\put(920,622){\makebox(0,0){$\times$}}
\put(920,620){\makebox(0,0){$\times$}}
\put(921,619){\makebox(0,0){$\times$}}
\put(922,617){\makebox(0,0){$\times$}}
\put(922,616){\makebox(0,0){$\times$}}
\put(923,615){\makebox(0,0){$\times$}}
\put(924,613){\makebox(0,0){$\times$}}
\put(925,612){\makebox(0,0){$\times$}}
\put(925,611){\makebox(0,0){$\times$}}
\put(926,609){\makebox(0,0){$\times$}}
\put(927,608){\makebox(0,0){$\times$}}
\put(927,607){\makebox(0,0){$\times$}}
\put(928,604){\makebox(0,0){$\times$}}
\put(929,602){\makebox(0,0){$\times$}}
\put(930,601){\makebox(0,0){$\times$}}
\put(930,600){\makebox(0,0){$\times$}}
\put(931,598){\makebox(0,0){$\times$}}
\put(932,597){\makebox(0,0){$\times$}}
\put(932,596){\makebox(0,0){$\times$}}
\put(933,593){\makebox(0,0){$\times$}}
\put(934,592){\makebox(0,0){$\times$}}
\put(935,590){\makebox(0,0){$\times$}}
\put(935,589){\makebox(0,0){$\times$}}
\put(936,587){\makebox(0,0){$\times$}}
\put(937,586){\makebox(0,0){$\times$}}
\put(937,583){\makebox(0,0){$\times$}}
\put(938,582){\makebox(0,0){$\times$}}
\put(939,581){\makebox(0,0){$\times$}}
\put(940,579){\makebox(0,0){$\times$}}
\put(940,578){\makebox(0,0){$\times$}}
\put(941,577){\makebox(0,0){$\times$}}
\put(942,574){\makebox(0,0){$\times$}}
\put(942,573){\makebox(0,0){$\times$}}
\put(943,571){\makebox(0,0){$\times$}}
\put(944,570){\makebox(0,0){$\times$}}
\put(945,568){\makebox(0,0){$\times$}}
\put(945,567){\makebox(0,0){$\times$}}
\put(946,564){\makebox(0,0){$\times$}}
\put(947,563){\makebox(0,0){$\times$}}
\put(947,562){\makebox(0,0){$\times$}}
\put(948,560){\makebox(0,0){$\times$}}
\put(949,559){\makebox(0,0){$\times$}}
\put(950,556){\makebox(0,0){$\times$}}
\put(954,548){\makebox(0,0){$\times$}}
\put(955,544){\makebox(0,0){$\times$}}
\put(956,543){\makebox(0,0){$\times$}}
\put(957,541){\makebox(0,0){$\times$}}
\put(957,540){\makebox(0,0){$\times$}}
\put(958,539){\makebox(0,0){$\times$}}
\put(959,537){\makebox(0,0){$\times$}}
\put(960,536){\makebox(0,0){$\times$}}
\put(960,534){\makebox(0,0){$\times$}}
\put(961,533){\makebox(0,0){$\times$}}
\put(962,532){\makebox(0,0){$\times$}}
\put(962,530){\makebox(0,0){$\times$}}
\put(963,529){\makebox(0,0){$\times$}}
\put(964,528){\makebox(0,0){$\times$}}
\put(965,526){\makebox(0,0){$\times$}}
\put(965,525){\makebox(0,0){$\times$}}
\put(966,524){\makebox(0,0){$\times$}}
\put(967,522){\makebox(0,0){$\times$}}
\put(967,521){\makebox(0,0){$\times$}}
\put(968,519){\makebox(0,0){$\times$}}
\put(969,518){\makebox(0,0){$\times$}}
\put(970,517){\makebox(0,0){$\times$}}
\put(970,515){\makebox(0,0){$\times$}}
\put(971,514){\makebox(0,0){$\times$}}
\put(972,513){\makebox(0,0){$\times$}}
\put(972,511){\makebox(0,0){$\times$}}
\put(973,511){\makebox(0,0){$\times$}}
\put(974,510){\makebox(0,0){$\times$}}
\put(975,509){\makebox(0,0){$\times$}}
\put(975,507){\makebox(0,0){$\times$}}
\put(976,506){\makebox(0,0){$\times$}}
\put(977,505){\makebox(0,0){$\times$}}
\put(977,505){\makebox(0,0){$\times$}}
\put(978,503){\makebox(0,0){$\times$}}
\put(979,502){\makebox(0,0){$\times$}}
\put(980,500){\makebox(0,0){$\times$}}
\put(980,500){\makebox(0,0){$\times$}}
\put(981,499){\makebox(0,0){$\times$}}
\put(982,498){\makebox(0,0){$\times$}}
\put(982,496){\makebox(0,0){$\times$}}
\put(983,496){\makebox(0,0){$\times$}}
\put(984,495){\makebox(0,0){$\times$}}
\put(985,494){\makebox(0,0){$\times$}}
\put(985,494){\makebox(0,0){$\times$}}
\put(986,492){\makebox(0,0){$\times$}}
\put(987,492){\makebox(0,0){$\times$}}
\put(987,491){\makebox(0,0){$\times$}}
\put(988,490){\makebox(0,0){$\times$}}
\put(989,490){\makebox(0,0){$\times$}}
\put(990,490){\makebox(0,0){$\times$}}
\put(990,488){\makebox(0,0){$\times$}}
\put(991,487){\makebox(0,0){$\times$}}
\put(992,487){\makebox(0,0){$\times$}}
\put(992,485){\makebox(0,0){$\times$}}
\put(993,485){\makebox(0,0){$\times$}}
\put(994,485){\makebox(0,0){$\times$}}
\put(995,484){\makebox(0,0){$\times$}}
\put(995,484){\makebox(0,0){$\times$}}
\put(996,484){\makebox(0,0){$\times$}}
\put(997,483){\makebox(0,0){$\times$}}
\put(998,483){\makebox(0,0){$\times$}}
\put(998,483){\makebox(0,0){$\times$}}
\put(999,481){\makebox(0,0){$\times$}}
\put(1000,481){\makebox(0,0){$\times$}}
\put(1000,481){\makebox(0,0){$\times$}}
\put(1001,480){\makebox(0,0){$\times$}}
\put(1002,480){\makebox(0,0){$\times$}}
\put(1003,480){\makebox(0,0){$\times$}}
\put(1003,480){\makebox(0,0){$\times$}}
\put(1004,480){\makebox(0,0){$\times$}}
\put(1005,479){\makebox(0,0){$\times$}}
\put(1005,479){\makebox(0,0){$\times$}}
\put(1006,479){\makebox(0,0){$\times$}}
\put(1007,479){\makebox(0,0){$\times$}}
\put(1008,479){\makebox(0,0){$\times$}}
\put(1008,479){\makebox(0,0){$\times$}}
\put(1009,479){\makebox(0,0){$\times$}}
\put(1010,479){\makebox(0,0){$\times$}}
\put(1010,479){\makebox(0,0){$\times$}}
\put(1011,479){\makebox(0,0){$\times$}}
\put(1012,479){\makebox(0,0){$\times$}}
\put(1013,479){\makebox(0,0){$\times$}}
\put(1013,479){\makebox(0,0){$\times$}}
\put(1014,479){\makebox(0,0){$\times$}}
\put(1015,479){\makebox(0,0){$\times$}}
\put(1015,479){\makebox(0,0){$\times$}}
\put(1016,479){\makebox(0,0){$\times$}}
\put(1017,479){\makebox(0,0){$\times$}}
\put(1018,479){\makebox(0,0){$\times$}}
\put(1018,480){\makebox(0,0){$\times$}}
\put(1019,480){\makebox(0,0){$\times$}}
\put(1020,480){\makebox(0,0){$\times$}}
\put(1020,480){\makebox(0,0){$\times$}}
\put(1021,480){\makebox(0,0){$\times$}}
\put(1022,481){\makebox(0,0){$\times$}}
\put(1023,481){\makebox(0,0){$\times$}}
\put(1023,481){\makebox(0,0){$\times$}}
\put(1024,483){\makebox(0,0){$\times$}}
\put(1025,483){\makebox(0,0){$\times$}}
\put(1025,483){\makebox(0,0){$\times$}}
\put(1026,484){\makebox(0,0){$\times$}}
\put(1027,484){\makebox(0,0){$\times$}}
\put(1028,484){\makebox(0,0){$\times$}}
\put(1028,485){\makebox(0,0){$\times$}}
\put(1029,485){\makebox(0,0){$\times$}}
\put(1030,485){\makebox(0,0){$\times$}}
\put(1030,487){\makebox(0,0){$\times$}}
\put(1031,487){\makebox(0,0){$\times$}}
\put(1032,488){\makebox(0,0){$\times$}}
\put(1033,488){\makebox(0,0){$\times$}}
\put(1033,490){\makebox(0,0){$\times$}}
\put(1034,490){\makebox(0,0){$\times$}}
\put(1035,490){\makebox(0,0){$\times$}}
\put(1035,491){\makebox(0,0){$\times$}}
\put(1036,491){\makebox(0,0){$\times$}}
\put(1037,492){\makebox(0,0){$\times$}}
\put(1038,492){\makebox(0,0){$\times$}}
\put(1038,494){\makebox(0,0){$\times$}}
\put(1039,494){\makebox(0,0){$\times$}}
\put(1040,495){\makebox(0,0){$\times$}}
\put(1040,495){\makebox(0,0){$\times$}}
\put(1041,496){\makebox(0,0){$\times$}}
\put(1042,496){\makebox(0,0){$\times$}}
\put(1043,498){\makebox(0,0){$\times$}}
\put(1043,499){\makebox(0,0){$\times$}}
\put(1044,499){\makebox(0,0){$\times$}}
\put(1045,500){\makebox(0,0){$\times$}}
\put(1045,500){\makebox(0,0){$\times$}}
\put(1046,502){\makebox(0,0){$\times$}}
\put(1047,503){\makebox(0,0){$\times$}}
\put(1048,503){\makebox(0,0){$\times$}}
\put(1048,505){\makebox(0,0){$\times$}}
\put(1049,506){\makebox(0,0){$\times$}}
\put(1050,506){\makebox(0,0){$\times$}}
\put(1050,507){\makebox(0,0){$\times$}}
\put(1051,507){\makebox(0,0){$\times$}}
\put(1052,509){\makebox(0,0){$\times$}}
\put(1053,510){\makebox(0,0){$\times$}}
\put(1053,510){\makebox(0,0){$\times$}}
\put(1054,511){\makebox(0,0){$\times$}}
\put(1055,513){\makebox(0,0){$\times$}}
\put(1055,513){\makebox(0,0){$\times$}}
\put(1056,514){\makebox(0,0){$\times$}}
\put(1057,515){\makebox(0,0){$\times$}}
\put(1058,517){\makebox(0,0){$\times$}}
\put(1058,517){\makebox(0,0){$\times$}}
\put(1059,518){\makebox(0,0){$\times$}}
\put(1060,518){\makebox(0,0){$\times$}}
\put(1060,519){\makebox(0,0){$\times$}}
\put(1061,521){\makebox(0,0){$\times$}}
\put(1062,522){\makebox(0,0){$\times$}}
\put(1063,522){\makebox(0,0){$\times$}}
\put(1063,524){\makebox(0,0){$\times$}}
\put(1064,525){\makebox(0,0){$\times$}}
\put(1065,526){\makebox(0,0){$\times$}}
\put(1065,526){\makebox(0,0){$\times$}}
\put(1066,528){\makebox(0,0){$\times$}}
\put(1067,529){\makebox(0,0){$\times$}}
\put(1068,529){\makebox(0,0){$\times$}}
\put(1068,530){\makebox(0,0){$\times$}}
\put(1069,532){\makebox(0,0){$\times$}}
\put(1070,533){\makebox(0,0){$\times$}}
\put(1070,533){\makebox(0,0){$\times$}}
\put(1071,534){\makebox(0,0){$\times$}}
\put(1072,536){\makebox(0,0){$\times$}}
\put(1073,536){\makebox(0,0){$\times$}}
\put(1073,537){\makebox(0,0){$\times$}}
\put(1074,539){\makebox(0,0){$\times$}}
\put(1075,540){\makebox(0,0){$\times$}}
\put(1075,540){\makebox(0,0){$\times$}}
\put(1076,541){\makebox(0,0){$\times$}}
\put(1077,543){\makebox(0,0){$\times$}}
\put(1078,544){\makebox(0,0){$\times$}}
\put(1078,544){\makebox(0,0){$\times$}}
\put(1079,545){\makebox(0,0){$\times$}}
\put(1081,548){\makebox(0,0){$\times$}}
\put(1087,556){\makebox(0,0){$\times$}}
\put(1088,556){\makebox(0,0){$\times$}}
\put(1088,558){\makebox(0,0){$\times$}}
\put(1089,559){\makebox(0,0){$\times$}}
\put(1090,559){\makebox(0,0){$\times$}}
\put(1091,560){\makebox(0,0){$\times$}}
\put(1091,562){\makebox(0,0){$\times$}}
\put(1092,562){\makebox(0,0){$\times$}}
\put(1093,563){\makebox(0,0){$\times$}}
\put(1093,564){\makebox(0,0){$\times$}}
\put(1094,564){\makebox(0,0){$\times$}}
\put(1095,566){\makebox(0,0){$\times$}}
\put(1096,566){\makebox(0,0){$\times$}}
\put(1096,567){\makebox(0,0){$\times$}}
\put(1097,567){\makebox(0,0){$\times$}}
\put(1098,568){\makebox(0,0){$\times$}}
\put(1098,570){\makebox(0,0){$\times$}}
\put(1099,570){\makebox(0,0){$\times$}}
\put(1100,571){\makebox(0,0){$\times$}}
\put(1101,571){\makebox(0,0){$\times$}}
\put(1101,573){\makebox(0,0){$\times$}}
\put(1102,574){\makebox(0,0){$\times$}}
\put(1103,574){\makebox(0,0){$\times$}}
\put(1103,575){\makebox(0,0){$\times$}}
\put(1104,575){\makebox(0,0){$\times$}}
\put(1105,577){\makebox(0,0){$\times$}}
\put(1106,577){\makebox(0,0){$\times$}}
\put(1106,578){\makebox(0,0){$\times$}}
\put(1107,578){\makebox(0,0){$\times$}}
\put(1108,579){\makebox(0,0){$\times$}}
\put(1108,579){\makebox(0,0){$\times$}}
\put(1109,581){\makebox(0,0){$\times$}}
\put(1110,581){\makebox(0,0){$\times$}}
\put(1111,582){\makebox(0,0){$\times$}}
\put(1111,582){\makebox(0,0){$\times$}}
\put(1112,582){\makebox(0,0){$\times$}}
\put(1113,583){\makebox(0,0){$\times$}}
\put(1113,583){\makebox(0,0){$\times$}}
\put(1114,585){\makebox(0,0){$\times$}}
\put(1115,585){\makebox(0,0){$\times$}}
\put(1116,586){\makebox(0,0){$\times$}}
\put(1116,586){\makebox(0,0){$\times$}}
\put(1117,586){\makebox(0,0){$\times$}}
\put(1118,587){\makebox(0,0){$\times$}}
\put(1118,587){\makebox(0,0){$\times$}}
\put(1119,587){\makebox(0,0){$\times$}}
\put(1120,589){\makebox(0,0){$\times$}}
\put(1121,589){\makebox(0,0){$\times$}}
\put(1121,589){\makebox(0,0){$\times$}}
\put(1122,589){\makebox(0,0){$\times$}}
\put(1123,590){\makebox(0,0){$\times$}}
\put(1123,590){\makebox(0,0){$\times$}}
\put(1124,590){\makebox(0,0){$\times$}}
\put(1125,592){\makebox(0,0){$\times$}}
\put(1126,592){\makebox(0,0){$\times$}}
\put(1126,592){\makebox(0,0){$\times$}}
\put(1127,592){\makebox(0,0){$\times$}}
\put(1128,592){\makebox(0,0){$\times$}}
\put(1128,593){\makebox(0,0){$\times$}}
\put(1129,593){\makebox(0,0){$\times$}}
\put(1130,593){\makebox(0,0){$\times$}}
\put(1131,593){\makebox(0,0){$\times$}}
\put(1131,593){\makebox(0,0){$\times$}}
\put(1132,593){\makebox(0,0){$\times$}}
\put(1133,593){\makebox(0,0){$\times$}}
\put(1133,594){\makebox(0,0){$\times$}}
\put(1134,594){\makebox(0,0){$\times$}}
\put(1135,594){\makebox(0,0){$\times$}}
\put(1136,594){\makebox(0,0){$\times$}}
\put(1136,594){\makebox(0,0){$\times$}}
\put(1137,594){\makebox(0,0){$\times$}}
\put(1138,594){\makebox(0,0){$\times$}}
\put(1138,594){\makebox(0,0){$\times$}}
\put(1139,594){\makebox(0,0){$\times$}}
\put(1140,594){\makebox(0,0){$\times$}}
\put(1141,594){\makebox(0,0){$\times$}}
\put(1141,594){\makebox(0,0){$\times$}}
\put(1142,594){\makebox(0,0){$\times$}}
\put(1143,594){\makebox(0,0){$\times$}}
\put(1143,594){\makebox(0,0){$\times$}}
\put(1144,594){\makebox(0,0){$\times$}}
\put(1145,594){\makebox(0,0){$\times$}}
\put(1146,594){\makebox(0,0){$\times$}}
\put(1146,594){\makebox(0,0){$\times$}}
\put(1147,594){\makebox(0,0){$\times$}}
\put(1148,594){\makebox(0,0){$\times$}}
\put(1148,594){\makebox(0,0){$\times$}}
\put(1149,594){\makebox(0,0){$\times$}}
\put(1150,594){\makebox(0,0){$\times$}}
\put(1151,593){\makebox(0,0){$\times$}}
\put(1151,593){\makebox(0,0){$\times$}}
\put(1152,593){\makebox(0,0){$\times$}}
\put(1153,593){\makebox(0,0){$\times$}}
\put(1153,593){\makebox(0,0){$\times$}}
\put(1154,593){\makebox(0,0){$\times$}}
\put(1155,593){\makebox(0,0){$\times$}}
\put(1156,592){\makebox(0,0){$\times$}}
\put(1156,592){\makebox(0,0){$\times$}}
\put(1157,592){\makebox(0,0){$\times$}}
\put(1158,592){\makebox(0,0){$\times$}}
\put(1159,590){\makebox(0,0){$\times$}}
\put(1159,590){\makebox(0,0){$\times$}}
\put(1160,590){\makebox(0,0){$\times$}}
\put(1161,590){\makebox(0,0){$\times$}}
\put(1161,589){\makebox(0,0){$\times$}}
\put(1162,589){\makebox(0,0){$\times$}}
\put(1163,589){\makebox(0,0){$\times$}}
\put(1164,589){\makebox(0,0){$\times$}}
\put(1164,587){\makebox(0,0){$\times$}}
\put(1165,587){\makebox(0,0){$\times$}}
\put(1166,587){\makebox(0,0){$\times$}}
\put(1166,587){\makebox(0,0){$\times$}}
\put(1167,586){\makebox(0,0){$\times$}}
\put(1168,586){\makebox(0,0){$\times$}}
\put(1169,586){\makebox(0,0){$\times$}}
\put(1169,585){\makebox(0,0){$\times$}}
\put(1170,585){\makebox(0,0){$\times$}}
\put(1171,585){\makebox(0,0){$\times$}}
\put(1171,583){\makebox(0,0){$\times$}}
\put(1172,583){\makebox(0,0){$\times$}}
\put(1173,583){\makebox(0,0){$\times$}}
\put(1174,582){\makebox(0,0){$\times$}}
\put(1174,582){\makebox(0,0){$\times$}}
\put(1175,582){\makebox(0,0){$\times$}}
\put(1176,581){\makebox(0,0){$\times$}}
\put(1176,581){\makebox(0,0){$\times$}}
\put(1177,581){\makebox(0,0){$\times$}}
\put(1178,579){\makebox(0,0){$\times$}}
\put(1179,579){\makebox(0,0){$\times$}}
\put(1179,578){\makebox(0,0){$\times$}}
\put(1180,578){\makebox(0,0){$\times$}}
\put(1181,578){\makebox(0,0){$\times$}}
\put(1181,577){\makebox(0,0){$\times$}}
\put(1182,577){\makebox(0,0){$\times$}}
\put(1183,577){\makebox(0,0){$\times$}}
\put(1184,575){\makebox(0,0){$\times$}}
\put(1184,575){\makebox(0,0){$\times$}}
\put(1185,574){\makebox(0,0){$\times$}}
\put(1186,574){\makebox(0,0){$\times$}}
\put(1186,574){\makebox(0,0){$\times$}}
\put(1187,573){\makebox(0,0){$\times$}}
\put(1188,573){\makebox(0,0){$\times$}}
\put(1189,571){\makebox(0,0){$\times$}}
\put(1189,571){\makebox(0,0){$\times$}}
\put(1190,571){\makebox(0,0){$\times$}}
\put(1191,570){\makebox(0,0){$\times$}}
\put(1191,570){\makebox(0,0){$\times$}}
\put(1192,568){\makebox(0,0){$\times$}}
\put(1193,568){\makebox(0,0){$\times$}}
\put(1194,567){\makebox(0,0){$\times$}}
\put(1194,567){\makebox(0,0){$\times$}}
\put(1195,567){\makebox(0,0){$\times$}}
\put(1196,566){\makebox(0,0){$\times$}}
\put(1196,566){\makebox(0,0){$\times$}}
\put(1197,566){\makebox(0,0){$\times$}}
\put(1198,564){\makebox(0,0){$\times$}}
\put(1199,564){\makebox(0,0){$\times$}}
\put(1199,563){\makebox(0,0){$\times$}}
\put(1200,563){\makebox(0,0){$\times$}}
\put(1204,560){\makebox(0,0){$\times$}}
\put(1204,559){\makebox(0,0){$\times$}}
\put(1205,559){\makebox(0,0){$\times$}}
\put(1206,559){\makebox(0,0){$\times$}}
\put(1206,558){\makebox(0,0){$\times$}}
\put(1207,558){\makebox(0,0){$\times$}}
\put(1208,556){\makebox(0,0){$\times$}}
\put(1209,556){\makebox(0,0){$\times$}}
\put(1219,548){\makebox(0,0){$\times$}}
\put(1220,548){\makebox(0,0){$\times$}}
\put(1223,545){\makebox(0,0){$\times$}}
\put(1224,545){\makebox(0,0){$\times$}}
\put(1224,545){\makebox(0,0){$\times$}}
\put(1225,544){\makebox(0,0){$\times$}}
\put(1226,544){\makebox(0,0){$\times$}}
\put(1226,544){\makebox(0,0){$\times$}}
\put(1227,543){\makebox(0,0){$\times$}}
\put(1228,543){\makebox(0,0){$\times$}}
\put(1229,543){\makebox(0,0){$\times$}}
\put(1229,543){\makebox(0,0){$\times$}}
\put(1230,541){\makebox(0,0){$\times$}}
\put(1231,541){\makebox(0,0){$\times$}}
\put(1231,541){\makebox(0,0){$\times$}}
\put(1232,540){\makebox(0,0){$\times$}}
\put(1233,540){\makebox(0,0){$\times$}}
\put(1234,540){\makebox(0,0){$\times$}}
\put(1234,540){\makebox(0,0){$\times$}}
\put(1235,539){\makebox(0,0){$\times$}}
\put(1236,539){\makebox(0,0){$\times$}}
\put(1236,539){\makebox(0,0){$\times$}}
\put(1237,539){\makebox(0,0){$\times$}}
\put(1238,537){\makebox(0,0){$\times$}}
\put(1239,537){\makebox(0,0){$\times$}}
\put(1239,537){\makebox(0,0){$\times$}}
\put(1240,537){\makebox(0,0){$\times$}}
\put(1241,536){\makebox(0,0){$\times$}}
\put(1242,536){\makebox(0,0){$\times$}}
\put(1242,536){\makebox(0,0){$\times$}}
\put(1243,536){\makebox(0,0){$\times$}}
\put(1244,536){\makebox(0,0){$\times$}}
\put(1244,534){\makebox(0,0){$\times$}}
\put(1245,534){\makebox(0,0){$\times$}}
\put(1246,534){\makebox(0,0){$\times$}}
\put(1247,534){\makebox(0,0){$\times$}}
\put(1247,534){\makebox(0,0){$\times$}}
\put(1248,534){\makebox(0,0){$\times$}}
\put(1249,533){\makebox(0,0){$\times$}}
\put(1249,533){\makebox(0,0){$\times$}}
\put(1250,533){\makebox(0,0){$\times$}}
\put(1251,533){\makebox(0,0){$\times$}}
\put(1252,533){\makebox(0,0){$\times$}}
\put(1252,533){\makebox(0,0){$\times$}}
\put(1253,533){\makebox(0,0){$\times$}}
\put(1254,533){\makebox(0,0){$\times$}}
\put(1254,533){\makebox(0,0){$\times$}}
\put(1255,533){\makebox(0,0){$\times$}}
\put(1256,533){\makebox(0,0){$\times$}}
\put(1257,532){\makebox(0,0){$\times$}}
\put(1257,532){\makebox(0,0){$\times$}}
\put(1258,532){\makebox(0,0){$\times$}}
\put(1259,532){\makebox(0,0){$\times$}}
\put(1259,532){\makebox(0,0){$\times$}}
\put(1260,532){\makebox(0,0){$\times$}}
\put(1261,532){\makebox(0,0){$\times$}}
\put(1262,532){\makebox(0,0){$\times$}}
\put(1262,532){\makebox(0,0){$\times$}}
\put(1263,532){\makebox(0,0){$\times$}}
\put(1264,532){\makebox(0,0){$\times$}}
\put(1264,532){\makebox(0,0){$\times$}}
\put(1265,532){\makebox(0,0){$\times$}}
\put(1266,532){\makebox(0,0){$\times$}}
\put(1267,532){\makebox(0,0){$\times$}}
\put(1267,532){\makebox(0,0){$\times$}}
\put(1268,532){\makebox(0,0){$\times$}}
\put(1269,532){\makebox(0,0){$\times$}}
\put(1269,532){\makebox(0,0){$\times$}}
\put(1270,532){\makebox(0,0){$\times$}}
\put(1271,532){\makebox(0,0){$\times$}}
\put(1272,532){\makebox(0,0){$\times$}}
\put(1272,532){\makebox(0,0){$\times$}}
\put(1273,532){\makebox(0,0){$\times$}}
\put(1274,532){\makebox(0,0){$\times$}}
\put(1274,532){\makebox(0,0){$\times$}}
\put(1275,532){\makebox(0,0){$\times$}}
\put(1276,532){\makebox(0,0){$\times$}}
\put(1277,533){\makebox(0,0){$\times$}}
\put(1277,533){\makebox(0,0){$\times$}}
\put(1278,533){\makebox(0,0){$\times$}}
\put(1279,533){\makebox(0,0){$\times$}}
\put(1279,533){\makebox(0,0){$\times$}}
\put(1280,533){\makebox(0,0){$\times$}}
\put(1281,533){\makebox(0,0){$\times$}}
\put(1282,533){\makebox(0,0){$\times$}}
\put(1282,533){\makebox(0,0){$\times$}}
\put(1283,533){\makebox(0,0){$\times$}}
\put(1284,534){\makebox(0,0){$\times$}}
\put(1284,534){\makebox(0,0){$\times$}}
\put(1285,534){\makebox(0,0){$\times$}}
\put(1286,534){\makebox(0,0){$\times$}}
\put(1287,534){\makebox(0,0){$\times$}}
\put(1287,534){\makebox(0,0){$\times$}}
\put(1288,534){\makebox(0,0){$\times$}}
\put(1289,534){\makebox(0,0){$\times$}}
\put(1289,536){\makebox(0,0){$\times$}}
\put(1290,536){\makebox(0,0){$\times$}}
\put(1291,536){\makebox(0,0){$\times$}}
\put(1292,536){\makebox(0,0){$\times$}}
\put(1292,536){\makebox(0,0){$\times$}}
\put(1293,536){\makebox(0,0){$\times$}}
\put(1294,537){\makebox(0,0){$\times$}}
\put(1294,537){\makebox(0,0){$\times$}}
\put(1295,537){\makebox(0,0){$\times$}}
\put(1296,537){\makebox(0,0){$\times$}}
\put(1297,537){\makebox(0,0){$\times$}}
\put(1297,539){\makebox(0,0){$\times$}}
\put(1298,539){\makebox(0,0){$\times$}}
\put(1299,539){\makebox(0,0){$\times$}}
\put(1299,539){\makebox(0,0){$\times$}}
\put(1300,539){\makebox(0,0){$\times$}}
\put(1301,539){\makebox(0,0){$\times$}}
\put(1302,539){\makebox(0,0){$\times$}}
\put(1302,540){\makebox(0,0){$\times$}}
\put(1303,540){\makebox(0,0){$\times$}}
\put(1304,540){\makebox(0,0){$\times$}}
\put(1304,540){\makebox(0,0){$\times$}}
\put(1305,540){\makebox(0,0){$\times$}}
\put(1306,541){\makebox(0,0){$\times$}}
\put(1307,541){\makebox(0,0){$\times$}}
\put(1307,541){\makebox(0,0){$\times$}}
\put(1308,541){\makebox(0,0){$\times$}}
\put(1309,541){\makebox(0,0){$\times$}}
\put(1309,543){\makebox(0,0){$\times$}}
\put(1310,543){\makebox(0,0){$\times$}}
\put(1311,543){\makebox(0,0){$\times$}}
\put(1312,543){\makebox(0,0){$\times$}}
\put(1312,543){\makebox(0,0){$\times$}}
\put(1313,544){\makebox(0,0){$\times$}}
\put(1314,544){\makebox(0,0){$\times$}}
\put(1314,544){\makebox(0,0){$\times$}}
\put(1315,544){\makebox(0,0){$\times$}}
\put(1316,544){\makebox(0,0){$\times$}}
\put(1317,544){\makebox(0,0){$\times$}}
\put(1317,545){\makebox(0,0){$\times$}}
\put(1318,545){\makebox(0,0){$\times$}}
\put(1319,545){\makebox(0,0){$\times$}}
\put(1320,545){\makebox(0,0){$\times$}}
\put(1320,545){\makebox(0,0){$\times$}}
\put(1321,545){\makebox(0,0){$\times$}}
\put(1325,548){\makebox(0,0){$\times$}}
\put(1325,548){\makebox(0,0){$\times$}}
\put(1326,548){\makebox(0,0){$\times$}}
\put(1327,548){\makebox(0,0){$\times$}}
\put(1327,548){\makebox(0,0){$\times$}}
\put(1328,548){\makebox(0,0){$\times$}}
\put(1352,556){\makebox(0,0){$\times$}}
\put(1352,556){\makebox(0,0){$\times$}}
\put(1353,556){\makebox(0,0){$\times$}}
\put(1354,556){\makebox(0,0){$\times$}}
\put(1355,556){\makebox(0,0){$\times$}}
\put(1355,556){\makebox(0,0){$\times$}}
\put(1356,556){\makebox(0,0){$\times$}}
\put(1357,556){\makebox(0,0){$\times$}}
\put(1357,556){\makebox(0,0){$\times$}}
\put(1358,556){\makebox(0,0){$\times$}}
\put(1359,558){\makebox(0,0){$\times$}}
\put(1360,558){\makebox(0,0){$\times$}}
\put(1360,558){\makebox(0,0){$\times$}}
\put(1361,558){\makebox(0,0){$\times$}}
\put(1362,558){\makebox(0,0){$\times$}}
\put(1362,558){\makebox(0,0){$\times$}}
\put(1363,558){\makebox(0,0){$\times$}}
\put(1364,558){\makebox(0,0){$\times$}}
\put(1365,558){\makebox(0,0){$\times$}}
\put(1365,558){\makebox(0,0){$\times$}}
\put(1366,558){\makebox(0,0){$\times$}}
\put(1367,558){\makebox(0,0){$\times$}}
\put(1367,558){\makebox(0,0){$\times$}}
\put(1368,559){\makebox(0,0){$\times$}}
\put(1369,559){\makebox(0,0){$\times$}}
\put(1370,559){\makebox(0,0){$\times$}}
\put(1370,559){\makebox(0,0){$\times$}}
\put(1371,559){\makebox(0,0){$\times$}}
\put(1372,559){\makebox(0,0){$\times$}}
\put(1372,559){\makebox(0,0){$\times$}}
\put(1373,559){\makebox(0,0){$\times$}}
\put(1374,559){\makebox(0,0){$\times$}}
\put(1375,559){\makebox(0,0){$\times$}}
\put(1375,559){\makebox(0,0){$\times$}}
\put(1376,559){\makebox(0,0){$\times$}}
\put(1377,559){\makebox(0,0){$\times$}}
\put(1377,559){\makebox(0,0){$\times$}}
\put(1378,559){\makebox(0,0){$\times$}}
\put(1379,559){\makebox(0,0){$\times$}}
\put(1380,559){\makebox(0,0){$\times$}}
\put(1380,559){\makebox(0,0){$\times$}}
\put(1381,559){\makebox(0,0){$\times$}}
\put(1382,559){\makebox(0,0){$\times$}}
\put(1382,559){\makebox(0,0){$\times$}}
\put(1383,559){\makebox(0,0){$\times$}}
\put(1384,559){\makebox(0,0){$\times$}}
\put(1385,559){\makebox(0,0){$\times$}}
\put(1385,559){\makebox(0,0){$\times$}}
\put(1386,559){\makebox(0,0){$\times$}}
\put(1387,559){\makebox(0,0){$\times$}}
\put(1387,559){\makebox(0,0){$\times$}}
\put(1388,559){\makebox(0,0){$\times$}}
\put(1389,559){\makebox(0,0){$\times$}}
\put(1390,559){\makebox(0,0){$\times$}}
\put(1390,559){\makebox(0,0){$\times$}}
\put(1391,559){\makebox(0,0){$\times$}}
\put(1392,559){\makebox(0,0){$\times$}}
\put(1392,559){\makebox(0,0){$\times$}}
\put(1393,559){\makebox(0,0){$\times$}}
\put(1394,559){\makebox(0,0){$\times$}}
\put(1395,559){\makebox(0,0){$\times$}}
\put(1395,559){\makebox(0,0){$\times$}}
\put(1396,559){\makebox(0,0){$\times$}}
\put(1397,559){\makebox(0,0){$\times$}}
\put(1397,559){\makebox(0,0){$\times$}}
\put(1398,559){\makebox(0,0){$\times$}}
\put(1399,559){\makebox(0,0){$\times$}}
\put(1400,558){\makebox(0,0){$\times$}}
\put(1400,558){\makebox(0,0){$\times$}}
\put(1401,558){\makebox(0,0){$\times$}}
\put(1402,558){\makebox(0,0){$\times$}}
\put(1403,558){\makebox(0,0){$\times$}}
\put(1403,558){\makebox(0,0){$\times$}}
\put(1404,558){\makebox(0,0){$\times$}}
\put(1405,558){\makebox(0,0){$\times$}}
\put(1405,558){\makebox(0,0){$\times$}}
\put(1406,558){\makebox(0,0){$\times$}}
\put(1407,558){\makebox(0,0){$\times$}}
\put(1408,558){\makebox(0,0){$\times$}}
\put(1408,558){\makebox(0,0){$\times$}}
\put(1409,558){\makebox(0,0){$\times$}}
\put(1410,558){\makebox(0,0){$\times$}}
\put(1410,558){\makebox(0,0){$\times$}}
\put(1411,558){\makebox(0,0){$\times$}}
\put(1412,558){\makebox(0,0){$\times$}}
\put(1413,556){\makebox(0,0){$\times$}}
\put(1413,556){\makebox(0,0){$\times$}}
\put(1414,556){\makebox(0,0){$\times$}}
\put(1415,556){\makebox(0,0){$\times$}}
\put(1415,556){\makebox(0,0){$\times$}}
\put(1416,556){\makebox(0,0){$\times$}}
\put(1417,556){\makebox(0,0){$\times$}}
\put(1418,556){\makebox(0,0){$\times$}}
\put(1418,556){\makebox(0,0){$\times$}}
\put(1419,556){\makebox(0,0){$\times$}}
\put(1420,556){\makebox(0,0){$\times$}}
\put(1420,556){\makebox(0,0){$\times$}}
\put(1421,556){\makebox(0,0){$\times$}}
\put(1422,556){\makebox(0,0){$\times$}}
\put(1423,556){\makebox(0,0){$\times$}}
\put(1423,556){\makebox(0,0){$\times$}}
\put(1424,556){\makebox(0,0){$\times$}}
\put(1425,556){\makebox(0,0){$\times$}}
\put(1425,556){\makebox(0,0){$\times$}}
\put(1426,556){\makebox(0,0){$\times$}}
\put(1427,556){\makebox(0,0){$\times$}}
\put(1428,556){\makebox(0,0){$\times$}}
\put(1428,556){\makebox(0,0){$\times$}}
\put(1429,556){\makebox(0,0){$\times$}}
\put(1349,172){\makebox(0,0){$\times$}}
\put(151.0,131.0){\rule[-0.200pt]{0.400pt}{175.375pt}}
\put(151.0,131.0){\rule[-0.200pt]{310.279pt}{0.400pt}}
\put(1439.0,131.0){\rule[-0.200pt]{0.400pt}{175.375pt}}
\put(151.0,859.0){\rule[-0.200pt]{310.279pt}{0.400pt}}
\end{picture}

\caption{Závislosť polohy $x$ v čase $t$, preložené funkciou $x= \(1\cdot10^7\pm1.5\cdot10^6\)e^{-\(1.14\pm0.02\)t} sin\(\(16.56\pm0.02\)t +\( 37.7\pm0.12\)\) $}  \label{G_3}
\end{figure}




\begin{figure}
% GNUPLOT: LaTeX picture
\setlength{\unitlength}{0.240900pt}
\ifx\plotpoint\undefined\newsavebox{\plotpoint}\fi
\begin{picture}(1500,900)(0,0)
\sbox{\plotpoint}{\rule[-0.200pt]{0.400pt}{0.400pt}}%
\put(151.0,131.0){\rule[-0.200pt]{4.818pt}{0.400pt}}
\put(131,131){\makebox(0,0)[r]{-40}}
\put(1419.0,131.0){\rule[-0.200pt]{4.818pt}{0.400pt}}
\put(151.0,212.0){\rule[-0.200pt]{4.818pt}{0.400pt}}
\put(131,212){\makebox(0,0)[r]{-35}}
\put(1419.0,212.0){\rule[-0.200pt]{4.818pt}{0.400pt}}
\put(151.0,293.0){\rule[-0.200pt]{4.818pt}{0.400pt}}
\put(131,293){\makebox(0,0)[r]{-30}}
\put(1419.0,293.0){\rule[-0.200pt]{4.818pt}{0.400pt}}
\put(151.0,374.0){\rule[-0.200pt]{4.818pt}{0.400pt}}
\put(131,374){\makebox(0,0)[r]{-25}}
\put(1419.0,374.0){\rule[-0.200pt]{4.818pt}{0.400pt}}
\put(151.0,455.0){\rule[-0.200pt]{4.818pt}{0.400pt}}
\put(131,455){\makebox(0,0)[r]{-20}}
\put(1419.0,455.0){\rule[-0.200pt]{4.818pt}{0.400pt}}
\put(151.0,535.0){\rule[-0.200pt]{4.818pt}{0.400pt}}
\put(131,535){\makebox(0,0)[r]{-15}}
\put(1419.0,535.0){\rule[-0.200pt]{4.818pt}{0.400pt}}
\put(151.0,616.0){\rule[-0.200pt]{4.818pt}{0.400pt}}
\put(131,616){\makebox(0,0)[r]{-10}}
\put(1419.0,616.0){\rule[-0.200pt]{4.818pt}{0.400pt}}
\put(151.0,697.0){\rule[-0.200pt]{4.818pt}{0.400pt}}
\put(131,697){\makebox(0,0)[r]{-5}}
\put(1419.0,697.0){\rule[-0.200pt]{4.818pt}{0.400pt}}
\put(151.0,778.0){\rule[-0.200pt]{4.818pt}{0.400pt}}
\put(131,778){\makebox(0,0)[r]{ 0}}
\put(1419.0,778.0){\rule[-0.200pt]{4.818pt}{0.400pt}}
\put(151.0,859.0){\rule[-0.200pt]{4.818pt}{0.400pt}}
\put(131,859){\makebox(0,0)[r]{ 5}}
\put(1419.0,859.0){\rule[-0.200pt]{4.818pt}{0.400pt}}
\put(151.0,131.0){\rule[-0.200pt]{0.400pt}{4.818pt}}
\put(151,90){\makebox(0,0){ 14}}
\put(151.0,839.0){\rule[-0.200pt]{0.400pt}{4.818pt}}
\put(409.0,131.0){\rule[-0.200pt]{0.400pt}{4.818pt}}
\put(409,90){\makebox(0,0){ 14.5}}
\put(409.0,839.0){\rule[-0.200pt]{0.400pt}{4.818pt}}
\put(666.0,131.0){\rule[-0.200pt]{0.400pt}{4.818pt}}
\put(666,90){\makebox(0,0){ 15}}
\put(666.0,839.0){\rule[-0.200pt]{0.400pt}{4.818pt}}
\put(924.0,131.0){\rule[-0.200pt]{0.400pt}{4.818pt}}
\put(924,90){\makebox(0,0){ 15.5}}
\put(924.0,839.0){\rule[-0.200pt]{0.400pt}{4.818pt}}
\put(1181.0,131.0){\rule[-0.200pt]{0.400pt}{4.818pt}}
\put(1181,90){\makebox(0,0){ 16}}
\put(1181.0,839.0){\rule[-0.200pt]{0.400pt}{4.818pt}}
\put(1439.0,131.0){\rule[-0.200pt]{0.400pt}{4.818pt}}
\put(1439,90){\makebox(0,0){ 16.5}}
\put(1439.0,839.0){\rule[-0.200pt]{0.400pt}{4.818pt}}
\put(151.0,131.0){\rule[-0.200pt]{0.400pt}{175.375pt}}
\put(151.0,131.0){\rule[-0.200pt]{310.279pt}{0.400pt}}
\put(1439.0,131.0){\rule[-0.200pt]{0.400pt}{175.375pt}}
\put(151.0,859.0){\rule[-0.200pt]{310.279pt}{0.400pt}}
\put(30,495){\makebox(0,0){\popi{x}{mm}}}
\put(795,29){\makebox(0,0){\popi{t}{s}}}
\put(1279,213){\makebox(0,0)[r]{$x= f(t) $}}
\put(1299.0,213.0){\rule[-0.200pt]{24.090pt}{0.400pt}}
\put(321,175){\usebox{\plotpoint}}
\multiput(321.58,175.00)(0.491,3.362){17}{\rule{0.118pt}{2.700pt}}
\multiput(320.17,175.00)(10.000,59.396){2}{\rule{0.400pt}{1.350pt}}
\multiput(331.58,240.00)(0.492,4.505){19}{\rule{0.118pt}{3.591pt}}
\multiput(330.17,240.00)(11.000,88.547){2}{\rule{0.400pt}{1.795pt}}
\multiput(342.58,336.00)(0.492,5.260){19}{\rule{0.118pt}{4.173pt}}
\multiput(341.17,336.00)(11.000,103.339){2}{\rule{0.400pt}{2.086pt}}
\multiput(353.58,448.00)(0.492,5.401){19}{\rule{0.118pt}{4.282pt}}
\multiput(352.17,448.00)(11.000,106.113){2}{\rule{0.400pt}{2.141pt}}
\multiput(364.58,563.00)(0.492,4.930){19}{\rule{0.118pt}{3.918pt}}
\multiput(363.17,563.00)(11.000,96.868){2}{\rule{0.400pt}{1.959pt}}
\multiput(375.58,668.00)(0.492,3.892){19}{\rule{0.118pt}{3.118pt}}
\multiput(374.17,668.00)(11.000,76.528){2}{\rule{0.400pt}{1.559pt}}
\multiput(386.58,751.00)(0.492,2.430){19}{\rule{0.118pt}{1.991pt}}
\multiput(385.17,751.00)(11.000,47.868){2}{\rule{0.400pt}{0.995pt}}
\multiput(397.58,803.00)(0.492,0.826){19}{\rule{0.118pt}{0.755pt}}
\multiput(396.17,803.00)(11.000,16.434){2}{\rule{0.400pt}{0.377pt}}
\multiput(408.58,818.17)(0.492,-0.732){19}{\rule{0.118pt}{0.682pt}}
\multiput(407.17,819.58)(11.000,-14.585){2}{\rule{0.400pt}{0.341pt}}
\multiput(419.58,797.49)(0.492,-2.194){19}{\rule{0.118pt}{1.809pt}}
\multiput(418.17,801.25)(11.000,-43.245){2}{\rule{0.400pt}{0.905pt}}
\multiput(430.58,747.17)(0.492,-3.232){19}{\rule{0.118pt}{2.609pt}}
\multiput(429.17,752.58)(11.000,-63.585){2}{\rule{0.400pt}{1.305pt}}
\multiput(441.58,676.06)(0.492,-3.892){19}{\rule{0.118pt}{3.118pt}}
\multiput(440.17,682.53)(11.000,-76.528){2}{\rule{0.400pt}{1.559pt}}
\multiput(452.58,592.60)(0.492,-4.034){19}{\rule{0.118pt}{3.227pt}}
\multiput(451.17,599.30)(11.000,-79.302){2}{\rule{0.400pt}{1.614pt}}
\multiput(463.58,507.81)(0.492,-3.656){19}{\rule{0.118pt}{2.936pt}}
\multiput(462.17,513.91)(11.000,-71.905){2}{\rule{0.400pt}{1.468pt}}
\multiput(474.58,432.08)(0.492,-2.949){19}{\rule{0.118pt}{2.391pt}}
\multiput(473.17,437.04)(11.000,-58.038){2}{\rule{0.400pt}{1.195pt}}
\multiput(485.58,372.55)(0.492,-1.864){19}{\rule{0.118pt}{1.555pt}}
\multiput(484.17,375.77)(11.000,-36.773){2}{\rule{0.400pt}{0.777pt}}
\multiput(496.58,336.32)(0.492,-0.684){19}{\rule{0.118pt}{0.645pt}}
\multiput(495.17,337.66)(11.000,-13.660){2}{\rule{0.400pt}{0.323pt}}
\multiput(507.00,324.58)(0.547,0.491){17}{\rule{0.540pt}{0.118pt}}
\multiput(507.00,323.17)(9.879,10.000){2}{\rule{0.270pt}{0.400pt}}
\multiput(518.58,334.00)(0.492,1.534){19}{\rule{0.118pt}{1.300pt}}
\multiput(517.17,334.00)(11.000,30.302){2}{\rule{0.400pt}{0.650pt}}
\multiput(529.58,367.00)(0.492,2.383){19}{\rule{0.118pt}{1.955pt}}
\multiput(528.17,367.00)(11.000,46.943){2}{\rule{0.400pt}{0.977pt}}
\multiput(540.58,418.00)(0.492,2.854){19}{\rule{0.118pt}{2.318pt}}
\multiput(539.17,418.00)(11.000,56.188){2}{\rule{0.400pt}{1.159pt}}
\multiput(551.58,479.00)(0.491,3.310){17}{\rule{0.118pt}{2.660pt}}
\multiput(550.17,479.00)(10.000,58.479){2}{\rule{0.400pt}{1.330pt}}
\multiput(561.58,543.00)(0.492,2.760){19}{\rule{0.118pt}{2.245pt}}
\multiput(560.17,543.00)(11.000,54.339){2}{\rule{0.400pt}{1.123pt}}
\multiput(572.58,602.00)(0.492,2.194){19}{\rule{0.118pt}{1.809pt}}
\multiput(571.17,602.00)(11.000,43.245){2}{\rule{0.400pt}{0.905pt}}
\multiput(583.58,649.00)(0.492,1.439){19}{\rule{0.118pt}{1.227pt}}
\multiput(582.17,649.00)(11.000,28.453){2}{\rule{0.400pt}{0.614pt}}
\multiput(594.58,680.00)(0.492,0.590){19}{\rule{0.118pt}{0.573pt}}
\multiput(593.17,680.00)(11.000,11.811){2}{\rule{0.400pt}{0.286pt}}
\multiput(605.00,691.93)(0.798,-0.485){11}{\rule{0.729pt}{0.117pt}}
\multiput(605.00,692.17)(9.488,-7.000){2}{\rule{0.364pt}{0.400pt}}
\multiput(616.58,681.96)(0.492,-1.109){19}{\rule{0.118pt}{0.973pt}}
\multiput(615.17,683.98)(11.000,-21.981){2}{\rule{0.400pt}{0.486pt}}
\multiput(627.58,656.00)(0.492,-1.722){19}{\rule{0.118pt}{1.445pt}}
\multiput(626.17,659.00)(11.000,-34.000){2}{\rule{0.400pt}{0.723pt}}
\multiput(638.58,617.79)(0.492,-2.100){19}{\rule{0.118pt}{1.736pt}}
\multiput(637.17,621.40)(11.000,-41.396){2}{\rule{0.400pt}{0.868pt}}
\multiput(649.58,572.49)(0.492,-2.194){19}{\rule{0.118pt}{1.809pt}}
\multiput(648.17,576.25)(11.000,-43.245){2}{\rule{0.400pt}{0.905pt}}
\multiput(660.58,525.94)(0.492,-2.052){19}{\rule{0.118pt}{1.700pt}}
\multiput(659.17,529.47)(11.000,-40.472){2}{\rule{0.400pt}{0.850pt}}
\multiput(671.58,483.15)(0.492,-1.675){19}{\rule{0.118pt}{1.409pt}}
\multiput(670.17,486.08)(11.000,-33.075){2}{\rule{0.400pt}{0.705pt}}
\multiput(682.58,448.96)(0.492,-1.109){19}{\rule{0.118pt}{0.973pt}}
\multiput(681.17,450.98)(11.000,-21.981){2}{\rule{0.400pt}{0.486pt}}
\multiput(693.00,427.92)(0.547,-0.491){17}{\rule{0.540pt}{0.118pt}}
\multiput(693.00,428.17)(9.879,-10.000){2}{\rule{0.270pt}{0.400pt}}
\multiput(704.00,419.60)(1.505,0.468){5}{\rule{1.200pt}{0.113pt}}
\multiput(704.00,418.17)(8.509,4.000){2}{\rule{0.600pt}{0.400pt}}
\multiput(715.58,423.00)(0.492,0.779){19}{\rule{0.118pt}{0.718pt}}
\multiput(714.17,423.00)(11.000,15.509){2}{\rule{0.400pt}{0.359pt}}
\multiput(726.58,440.00)(0.492,1.251){19}{\rule{0.118pt}{1.082pt}}
\multiput(725.17,440.00)(11.000,24.755){2}{\rule{0.400pt}{0.541pt}}
\multiput(737.58,467.00)(0.492,1.534){19}{\rule{0.118pt}{1.300pt}}
\multiput(736.17,467.00)(11.000,30.302){2}{\rule{0.400pt}{0.650pt}}
\multiput(748.58,500.00)(0.492,1.628){19}{\rule{0.118pt}{1.373pt}}
\multiput(747.17,500.00)(11.000,32.151){2}{\rule{0.400pt}{0.686pt}}
\multiput(759.58,535.00)(0.492,1.534){19}{\rule{0.118pt}{1.300pt}}
\multiput(758.17,535.00)(11.000,30.302){2}{\rule{0.400pt}{0.650pt}}
\multiput(770.58,568.00)(0.492,1.251){19}{\rule{0.118pt}{1.082pt}}
\multiput(769.17,568.00)(11.000,24.755){2}{\rule{0.400pt}{0.541pt}}
\multiput(781.58,595.00)(0.492,0.873){19}{\rule{0.118pt}{0.791pt}}
\multiput(780.17,595.00)(11.000,17.358){2}{\rule{0.400pt}{0.395pt}}
\multiput(792.00,614.59)(0.626,0.488){13}{\rule{0.600pt}{0.117pt}}
\multiput(792.00,613.17)(8.755,8.000){2}{\rule{0.300pt}{0.400pt}}
\multiput(802.00,620.95)(2.248,-0.447){3}{\rule{1.567pt}{0.108pt}}
\multiput(802.00,621.17)(7.748,-3.000){2}{\rule{0.783pt}{0.400pt}}
\multiput(813.58,616.77)(0.492,-0.543){19}{\rule{0.118pt}{0.536pt}}
\multiput(812.17,617.89)(11.000,-10.887){2}{\rule{0.400pt}{0.268pt}}
\multiput(824.58,603.72)(0.492,-0.873){19}{\rule{0.118pt}{0.791pt}}
\multiput(823.17,605.36)(11.000,-17.358){2}{\rule{0.400pt}{0.395pt}}
\multiput(835.58,583.81)(0.492,-1.156){19}{\rule{0.118pt}{1.009pt}}
\multiput(834.17,585.91)(11.000,-22.906){2}{\rule{0.400pt}{0.505pt}}
\multiput(846.58,558.66)(0.492,-1.203){19}{\rule{0.118pt}{1.045pt}}
\multiput(845.17,560.83)(11.000,-23.830){2}{\rule{0.400pt}{0.523pt}}
\multiput(857.58,532.96)(0.492,-1.109){19}{\rule{0.118pt}{0.973pt}}
\multiput(856.17,534.98)(11.000,-21.981){2}{\rule{0.400pt}{0.486pt}}
\multiput(868.58,509.41)(0.492,-0.967){19}{\rule{0.118pt}{0.864pt}}
\multiput(867.17,511.21)(11.000,-19.207){2}{\rule{0.400pt}{0.432pt}}
\multiput(879.58,489.47)(0.492,-0.637){19}{\rule{0.118pt}{0.609pt}}
\multiput(878.17,490.74)(11.000,-12.736){2}{\rule{0.400pt}{0.305pt}}
\multiput(890.00,476.93)(0.943,-0.482){9}{\rule{0.833pt}{0.116pt}}
\multiput(890.00,477.17)(9.270,-6.000){2}{\rule{0.417pt}{0.400pt}}
\put(901,471.67){\rule{2.650pt}{0.400pt}}
\multiput(901.00,471.17)(5.500,1.000){2}{\rule{1.325pt}{0.400pt}}
\multiput(912.00,473.59)(0.692,0.488){13}{\rule{0.650pt}{0.117pt}}
\multiput(912.00,472.17)(9.651,8.000){2}{\rule{0.325pt}{0.400pt}}
\multiput(923.58,481.00)(0.492,0.684){19}{\rule{0.118pt}{0.645pt}}
\multiput(922.17,481.00)(11.000,13.660){2}{\rule{0.400pt}{0.323pt}}
\multiput(934.58,496.00)(0.492,0.826){19}{\rule{0.118pt}{0.755pt}}
\multiput(933.17,496.00)(11.000,16.434){2}{\rule{0.400pt}{0.377pt}}
\multiput(945.58,514.00)(0.492,0.873){19}{\rule{0.118pt}{0.791pt}}
\multiput(944.17,514.00)(11.000,17.358){2}{\rule{0.400pt}{0.395pt}}
\multiput(956.58,533.00)(0.492,0.826){19}{\rule{0.118pt}{0.755pt}}
\multiput(955.17,533.00)(11.000,16.434){2}{\rule{0.400pt}{0.377pt}}
\multiput(967.58,551.00)(0.492,0.732){19}{\rule{0.118pt}{0.682pt}}
\multiput(966.17,551.00)(11.000,14.585){2}{\rule{0.400pt}{0.341pt}}
\multiput(978.00,567.58)(0.496,0.492){19}{\rule{0.500pt}{0.118pt}}
\multiput(978.00,566.17)(9.962,11.000){2}{\rule{0.250pt}{0.400pt}}
\multiput(989.00,578.59)(1.155,0.477){7}{\rule{0.980pt}{0.115pt}}
\multiput(989.00,577.17)(8.966,5.000){2}{\rule{0.490pt}{0.400pt}}
\put(1000,581.67){\rule{2.650pt}{0.400pt}}
\multiput(1000.00,582.17)(5.500,-1.000){2}{\rule{1.325pt}{0.400pt}}
\multiput(1011.00,580.93)(0.943,-0.482){9}{\rule{0.833pt}{0.116pt}}
\multiput(1011.00,581.17)(9.270,-6.000){2}{\rule{0.417pt}{0.400pt}}
\multiput(1022.00,574.92)(0.547,-0.491){17}{\rule{0.540pt}{0.118pt}}
\multiput(1022.00,575.17)(9.879,-10.000){2}{\rule{0.270pt}{0.400pt}}
\multiput(1033.58,563.26)(0.491,-0.704){17}{\rule{0.118pt}{0.660pt}}
\multiput(1032.17,564.63)(10.000,-12.630){2}{\rule{0.400pt}{0.330pt}}
\multiput(1043.58,549.47)(0.492,-0.637){19}{\rule{0.118pt}{0.609pt}}
\multiput(1042.17,550.74)(11.000,-12.736){2}{\rule{0.400pt}{0.305pt}}
\multiput(1054.58,535.47)(0.492,-0.637){19}{\rule{0.118pt}{0.609pt}}
\multiput(1053.17,536.74)(11.000,-12.736){2}{\rule{0.400pt}{0.305pt}}
\multiput(1065.00,522.92)(0.496,-0.492){19}{\rule{0.500pt}{0.118pt}}
\multiput(1065.00,523.17)(9.962,-11.000){2}{\rule{0.250pt}{0.400pt}}
\multiput(1076.00,511.93)(0.692,-0.488){13}{\rule{0.650pt}{0.117pt}}
\multiput(1076.00,512.17)(9.651,-8.000){2}{\rule{0.325pt}{0.400pt}}
\multiput(1087.00,503.93)(1.155,-0.477){7}{\rule{0.980pt}{0.115pt}}
\multiput(1087.00,504.17)(8.966,-5.000){2}{\rule{0.490pt}{0.400pt}}
\put(1098,499.67){\rule{2.650pt}{0.400pt}}
\multiput(1098.00,499.17)(5.500,1.000){2}{\rule{1.325pt}{0.400pt}}
\multiput(1109.00,501.60)(1.505,0.468){5}{\rule{1.200pt}{0.113pt}}
\multiput(1109.00,500.17)(8.509,4.000){2}{\rule{0.600pt}{0.400pt}}
\multiput(1120.00,505.59)(0.798,0.485){11}{\rule{0.729pt}{0.117pt}}
\multiput(1120.00,504.17)(9.488,7.000){2}{\rule{0.364pt}{0.400pt}}
\multiput(1131.00,512.58)(0.547,0.491){17}{\rule{0.540pt}{0.118pt}}
\multiput(1131.00,511.17)(9.879,10.000){2}{\rule{0.270pt}{0.400pt}}
\multiput(1142.00,522.58)(0.496,0.492){19}{\rule{0.500pt}{0.118pt}}
\multiput(1142.00,521.17)(9.962,11.000){2}{\rule{0.250pt}{0.400pt}}
\multiput(1153.00,533.58)(0.547,0.491){17}{\rule{0.540pt}{0.118pt}}
\multiput(1153.00,532.17)(9.879,10.000){2}{\rule{0.270pt}{0.400pt}}
\multiput(1164.00,543.59)(0.611,0.489){15}{\rule{0.589pt}{0.118pt}}
\multiput(1164.00,542.17)(9.778,9.000){2}{\rule{0.294pt}{0.400pt}}
\multiput(1175.00,552.59)(0.943,0.482){9}{\rule{0.833pt}{0.116pt}}
\multiput(1175.00,551.17)(9.270,6.000){2}{\rule{0.417pt}{0.400pt}}
\multiput(1186.00,558.61)(2.248,0.447){3}{\rule{1.567pt}{0.108pt}}
\multiput(1186.00,557.17)(7.748,3.000){2}{\rule{0.783pt}{0.400pt}}
\multiput(1208.00,559.95)(2.248,-0.447){3}{\rule{1.567pt}{0.108pt}}
\multiput(1208.00,560.17)(7.748,-3.000){2}{\rule{0.783pt}{0.400pt}}
\multiput(1219.00,556.93)(1.155,-0.477){7}{\rule{0.980pt}{0.115pt}}
\multiput(1219.00,557.17)(8.966,-5.000){2}{\rule{0.490pt}{0.400pt}}
\multiput(1230.00,551.93)(0.798,-0.485){11}{\rule{0.729pt}{0.117pt}}
\multiput(1230.00,552.17)(9.488,-7.000){2}{\rule{0.364pt}{0.400pt}}
\multiput(1241.00,544.93)(0.692,-0.488){13}{\rule{0.650pt}{0.117pt}}
\multiput(1241.00,545.17)(9.651,-8.000){2}{\rule{0.325pt}{0.400pt}}
\multiput(1252.00,536.93)(0.692,-0.488){13}{\rule{0.650pt}{0.117pt}}
\multiput(1252.00,537.17)(9.651,-8.000){2}{\rule{0.325pt}{0.400pt}}
\multiput(1263.00,528.93)(0.943,-0.482){9}{\rule{0.833pt}{0.116pt}}
\multiput(1263.00,529.17)(9.270,-6.000){2}{\rule{0.417pt}{0.400pt}}
\multiput(1274.00,522.93)(1.044,-0.477){7}{\rule{0.900pt}{0.115pt}}
\multiput(1274.00,523.17)(8.132,-5.000){2}{\rule{0.450pt}{0.400pt}}
\multiput(1284.00,517.95)(2.248,-0.447){3}{\rule{1.567pt}{0.108pt}}
\multiput(1284.00,518.17)(7.748,-3.000){2}{\rule{0.783pt}{0.400pt}}
\put(1197.0,561.0){\rule[-0.200pt]{2.650pt}{0.400pt}}
\put(1306,516.17){\rule{2.300pt}{0.400pt}}
\multiput(1306.00,515.17)(6.226,2.000){2}{\rule{1.150pt}{0.400pt}}
\multiput(1317.00,518.60)(1.505,0.468){5}{\rule{1.200pt}{0.113pt}}
\multiput(1317.00,517.17)(8.509,4.000){2}{\rule{0.600pt}{0.400pt}}
\multiput(1328.00,522.59)(1.155,0.477){7}{\rule{0.980pt}{0.115pt}}
\multiput(1328.00,521.17)(8.966,5.000){2}{\rule{0.490pt}{0.400pt}}
\multiput(1339.00,527.59)(0.943,0.482){9}{\rule{0.833pt}{0.116pt}}
\multiput(1339.00,526.17)(9.270,6.000){2}{\rule{0.417pt}{0.400pt}}
\multiput(1350.00,533.59)(0.943,0.482){9}{\rule{0.833pt}{0.116pt}}
\multiput(1350.00,532.17)(9.270,6.000){2}{\rule{0.417pt}{0.400pt}}
\multiput(1361.00,539.59)(1.155,0.477){7}{\rule{0.980pt}{0.115pt}}
\multiput(1361.00,538.17)(8.966,5.000){2}{\rule{0.490pt}{0.400pt}}
\multiput(1372.00,544.60)(1.505,0.468){5}{\rule{1.200pt}{0.113pt}}
\multiput(1372.00,543.17)(8.509,4.000){2}{\rule{0.600pt}{0.400pt}}
\put(1383,548.17){\rule{2.300pt}{0.400pt}}
\multiput(1383.00,547.17)(6.226,2.000){2}{\rule{1.150pt}{0.400pt}}
\put(1295.0,516.0){\rule[-0.200pt]{2.650pt}{0.400pt}}
\put(1394.0,550.0){\rule[-0.200pt]{2.650pt}{0.400pt}}
\put(1279,172){\makebox(0,0)[r]{namerané dáta}}
\put(321,242){\makebox(0,0){$\times$}}
\put(321,243){\makebox(0,0){$\times$}}
\put(322,245){\makebox(0,0){$\times$}}
\put(322,247){\makebox(0,0){$\times$}}
\put(323,250){\makebox(0,0){$\times$}}
\put(323,251){\makebox(0,0){$\times$}}
\put(324,254){\makebox(0,0){$\times$}}
\put(324,256){\makebox(0,0){$\times$}}
\put(325,258){\makebox(0,0){$\times$}}
\put(325,260){\makebox(0,0){$\times$}}
\put(326,263){\makebox(0,0){$\times$}}
\put(326,265){\makebox(0,0){$\times$}}
\put(327,268){\makebox(0,0){$\times$}}
\put(327,271){\makebox(0,0){$\times$}}
\put(328,273){\makebox(0,0){$\times$}}
\put(328,276){\makebox(0,0){$\times$}}
\put(329,279){\makebox(0,0){$\times$}}
\put(329,282){\makebox(0,0){$\times$}}
\put(330,285){\makebox(0,0){$\times$}}
\put(330,287){\makebox(0,0){$\times$}}
\put(331,290){\makebox(0,0){$\times$}}
\put(331,293){\makebox(0,0){$\times$}}
\put(332,297){\makebox(0,0){$\times$}}
\put(332,300){\makebox(0,0){$\times$}}
\put(333,303){\makebox(0,0){$\times$}}
\put(333,306){\makebox(0,0){$\times$}}
\put(334,309){\makebox(0,0){$\times$}}
\put(334,313){\makebox(0,0){$\times$}}
\put(335,316){\makebox(0,0){$\times$}}
\put(335,320){\makebox(0,0){$\times$}}
\put(336,323){\makebox(0,0){$\times$}}
\put(336,327){\makebox(0,0){$\times$}}
\put(337,331){\makebox(0,0){$\times$}}
\put(338,334){\makebox(0,0){$\times$}}
\put(338,338){\makebox(0,0){$\times$}}
\put(339,341){\makebox(0,0){$\times$}}
\put(339,346){\makebox(0,0){$\times$}}
\put(340,349){\makebox(0,0){$\times$}}
\put(340,353){\makebox(0,0){$\times$}}
\put(341,357){\makebox(0,0){$\times$}}
\put(341,361){\makebox(0,0){$\times$}}
\put(342,365){\makebox(0,0){$\times$}}
\put(342,369){\makebox(0,0){$\times$}}
\put(343,373){\makebox(0,0){$\times$}}
\put(343,377){\makebox(0,0){$\times$}}
\put(344,381){\makebox(0,0){$\times$}}
\put(344,386){\makebox(0,0){$\times$}}
\put(345,390){\makebox(0,0){$\times$}}
\put(345,394){\makebox(0,0){$\times$}}
\put(346,398){\makebox(0,0){$\times$}}
\put(346,403){\makebox(0,0){$\times$}}
\put(347,407){\makebox(0,0){$\times$}}
\put(347,412){\makebox(0,0){$\times$}}
\put(348,416){\makebox(0,0){$\times$}}
\put(348,420){\makebox(0,0){$\times$}}
\put(349,425){\makebox(0,0){$\times$}}
\put(349,429){\makebox(0,0){$\times$}}
\put(350,434){\makebox(0,0){$\times$}}
\put(350,438){\makebox(0,0){$\times$}}
\put(351,443){\makebox(0,0){$\times$}}
\put(351,447){\makebox(0,0){$\times$}}
\put(352,452){\makebox(0,0){$\times$}}
\put(352,456){\makebox(0,0){$\times$}}
\put(353,461){\makebox(0,0){$\times$}}
\put(353,465){\makebox(0,0){$\times$}}
\put(354,470){\makebox(0,0){$\times$}}
\put(355,475){\makebox(0,0){$\times$}}
\put(355,479){\makebox(0,0){$\times$}}
\put(356,484){\makebox(0,0){$\times$}}
\put(356,488){\makebox(0,0){$\times$}}
\put(357,493){\makebox(0,0){$\times$}}
\put(357,497){\makebox(0,0){$\times$}}
\put(358,502){\makebox(0,0){$\times$}}
\put(358,507){\makebox(0,0){$\times$}}
\put(359,512){\makebox(0,0){$\times$}}
\put(359,516){\makebox(0,0){$\times$}}
\put(360,521){\makebox(0,0){$\times$}}
\put(360,526){\makebox(0,0){$\times$}}
\put(361,530){\makebox(0,0){$\times$}}
\put(361,535){\makebox(0,0){$\times$}}
\put(362,540){\makebox(0,0){$\times$}}
\put(363,549){\makebox(0,0){$\times$}}
\put(363,553){\makebox(0,0){$\times$}}
\put(364,558){\makebox(0,0){$\times$}}
\put(364,563){\makebox(0,0){$\times$}}
\put(365,567){\makebox(0,0){$\times$}}
\put(365,572){\makebox(0,0){$\times$}}
\put(366,576){\makebox(0,0){$\times$}}
\put(366,581){\makebox(0,0){$\times$}}
\put(367,585){\makebox(0,0){$\times$}}
\put(367,589){\makebox(0,0){$\times$}}
\put(368,594){\makebox(0,0){$\times$}}
\put(368,598){\makebox(0,0){$\times$}}
\put(369,603){\makebox(0,0){$\times$}}
\put(369,608){\makebox(0,0){$\times$}}
\put(370,612){\makebox(0,0){$\times$}}
\put(370,616){\makebox(0,0){$\times$}}
\put(371,620){\makebox(0,0){$\times$}}
\put(372,624){\makebox(0,0){$\times$}}
\put(372,629){\makebox(0,0){$\times$}}
\put(373,633){\makebox(0,0){$\times$}}
\put(373,637){\makebox(0,0){$\times$}}
\put(374,641){\makebox(0,0){$\times$}}
\put(374,646){\makebox(0,0){$\times$}}
\put(375,650){\makebox(0,0){$\times$}}
\put(375,654){\makebox(0,0){$\times$}}
\put(376,658){\makebox(0,0){$\times$}}
\put(376,662){\makebox(0,0){$\times$}}
\put(377,666){\makebox(0,0){$\times$}}
\put(377,670){\makebox(0,0){$\times$}}
\put(378,673){\makebox(0,0){$\times$}}
\put(378,677){\makebox(0,0){$\times$}}
\put(379,681){\makebox(0,0){$\times$}}
\put(379,685){\makebox(0,0){$\times$}}
\put(380,689){\makebox(0,0){$\times$}}
\put(380,692){\makebox(0,0){$\times$}}
\put(381,696){\makebox(0,0){$\times$}}
\put(381,699){\makebox(0,0){$\times$}}
\put(382,703){\makebox(0,0){$\times$}}
\put(382,707){\makebox(0,0){$\times$}}
\put(383,710){\makebox(0,0){$\times$}}
\put(383,713){\makebox(0,0){$\times$}}
\put(384,717){\makebox(0,0){$\times$}}
\put(384,720){\makebox(0,0){$\times$}}
\put(385,724){\makebox(0,0){$\times$}}
\put(385,727){\makebox(0,0){$\times$}}
\put(386,730){\makebox(0,0){$\times$}}
\put(386,733){\makebox(0,0){$\times$}}
\put(387,736){\makebox(0,0){$\times$}}
\put(387,739){\makebox(0,0){$\times$}}
\put(388,742){\makebox(0,0){$\times$}}
\put(389,745){\makebox(0,0){$\times$}}
\put(389,748){\makebox(0,0){$\times$}}
\put(390,750){\makebox(0,0){$\times$}}
\put(390,753){\makebox(0,0){$\times$}}
\put(391,756){\makebox(0,0){$\times$}}
\put(391,759){\makebox(0,0){$\times$}}
\put(392,761){\makebox(0,0){$\times$}}
\put(392,764){\makebox(0,0){$\times$}}
\put(393,766){\makebox(0,0){$\times$}}
\put(393,768){\makebox(0,0){$\times$}}
\put(394,771){\makebox(0,0){$\times$}}
\put(394,773){\makebox(0,0){$\times$}}
\put(395,775){\makebox(0,0){$\times$}}
\put(395,777){\makebox(0,0){$\times$}}
\put(396,779){\makebox(0,0){$\times$}}
\put(396,781){\makebox(0,0){$\times$}}
\put(397,783){\makebox(0,0){$\times$}}
\put(397,785){\makebox(0,0){$\times$}}
\put(398,787){\makebox(0,0){$\times$}}
\put(398,788){\makebox(0,0){$\times$}}
\put(399,790){\makebox(0,0){$\times$}}
\put(399,791){\makebox(0,0){$\times$}}
\put(400,793){\makebox(0,0){$\times$}}
\put(400,794){\makebox(0,0){$\times$}}
\put(401,796){\makebox(0,0){$\times$}}
\put(401,797){\makebox(0,0){$\times$}}
\put(402,798){\makebox(0,0){$\times$}}
\put(402,799){\makebox(0,0){$\times$}}
\put(403,800){\makebox(0,0){$\times$}}
\put(403,801){\makebox(0,0){$\times$}}
\put(404,802){\makebox(0,0){$\times$}}
\put(404,803){\makebox(0,0){$\times$}}
\put(405,804){\makebox(0,0){$\times$}}
\put(406,805){\makebox(0,0){$\times$}}
\put(406,805){\makebox(0,0){$\times$}}
\put(407,805){\makebox(0,0){$\times$}}
\put(407,806){\makebox(0,0){$\times$}}
\put(408,807){\makebox(0,0){$\times$}}
\put(408,807){\makebox(0,0){$\times$}}
\put(409,807){\makebox(0,0){$\times$}}
\put(409,807){\makebox(0,0){$\times$}}
\put(410,808){\makebox(0,0){$\times$}}
\put(410,808){\makebox(0,0){$\times$}}
\put(411,808){\makebox(0,0){$\times$}}
\put(411,808){\makebox(0,0){$\times$}}
\put(412,808){\makebox(0,0){$\times$}}
\put(412,807){\makebox(0,0){$\times$}}
\put(413,807){\makebox(0,0){$\times$}}
\put(413,807){\makebox(0,0){$\times$}}
\put(414,807){\makebox(0,0){$\times$}}
\put(414,806){\makebox(0,0){$\times$}}
\put(415,805){\makebox(0,0){$\times$}}
\put(415,805){\makebox(0,0){$\times$}}
\put(416,804){\makebox(0,0){$\times$}}
\put(416,803){\makebox(0,0){$\times$}}
\put(417,802){\makebox(0,0){$\times$}}
\put(417,802){\makebox(0,0){$\times$}}
\put(418,800){\makebox(0,0){$\times$}}
\put(418,799){\makebox(0,0){$\times$}}
\put(419,799){\makebox(0,0){$\times$}}
\put(419,797){\makebox(0,0){$\times$}}
\put(420,796){\makebox(0,0){$\times$}}
\put(420,795){\makebox(0,0){$\times$}}
\put(421,793){\makebox(0,0){$\times$}}
\put(421,792){\makebox(0,0){$\times$}}
\put(422,791){\makebox(0,0){$\times$}}
\put(423,789){\makebox(0,0){$\times$}}
\put(423,788){\makebox(0,0){$\times$}}
\put(424,786){\makebox(0,0){$\times$}}
\put(424,784){\makebox(0,0){$\times$}}
\put(425,782){\makebox(0,0){$\times$}}
\put(425,781){\makebox(0,0){$\times$}}
\put(426,779){\makebox(0,0){$\times$}}
\put(426,777){\makebox(0,0){$\times$}}
\put(427,775){\makebox(0,0){$\times$}}
\put(427,773){\makebox(0,0){$\times$}}
\put(428,771){\makebox(0,0){$\times$}}
\put(428,768){\makebox(0,0){$\times$}}
\put(429,766){\makebox(0,0){$\times$}}
\put(429,764){\makebox(0,0){$\times$}}
\put(430,762){\makebox(0,0){$\times$}}
\put(430,759){\makebox(0,0){$\times$}}
\put(431,757){\makebox(0,0){$\times$}}
\put(431,755){\makebox(0,0){$\times$}}
\put(432,752){\makebox(0,0){$\times$}}
\put(432,750){\makebox(0,0){$\times$}}
\put(433,747){\makebox(0,0){$\times$}}
\put(433,744){\makebox(0,0){$\times$}}
\put(434,742){\makebox(0,0){$\times$}}
\put(434,739){\makebox(0,0){$\times$}}
\put(435,736){\makebox(0,0){$\times$}}
\put(435,733){\makebox(0,0){$\times$}}
\put(436,730){\makebox(0,0){$\times$}}
\put(436,727){\makebox(0,0){$\times$}}
\put(437,725){\makebox(0,0){$\times$}}
\put(437,722){\makebox(0,0){$\times$}}
\put(438,718){\makebox(0,0){$\times$}}
\put(438,715){\makebox(0,0){$\times$}}
\put(439,712){\makebox(0,0){$\times$}}
\put(440,709){\makebox(0,0){$\times$}}
\put(440,706){\makebox(0,0){$\times$}}
\put(441,703){\makebox(0,0){$\times$}}
\put(441,699){\makebox(0,0){$\times$}}
\put(442,696){\makebox(0,0){$\times$}}
\put(442,693){\makebox(0,0){$\times$}}
\put(443,689){\makebox(0,0){$\times$}}
\put(443,686){\makebox(0,0){$\times$}}
\put(444,683){\makebox(0,0){$\times$}}
\put(444,679){\makebox(0,0){$\times$}}
\put(445,676){\makebox(0,0){$\times$}}
\put(445,672){\makebox(0,0){$\times$}}
\put(446,669){\makebox(0,0){$\times$}}
\put(446,665){\makebox(0,0){$\times$}}
\put(447,661){\makebox(0,0){$\times$}}
\put(447,658){\makebox(0,0){$\times$}}
\put(448,655){\makebox(0,0){$\times$}}
\put(448,651){\makebox(0,0){$\times$}}
\put(449,647){\makebox(0,0){$\times$}}
\put(449,643){\makebox(0,0){$\times$}}
\put(450,640){\makebox(0,0){$\times$}}
\put(450,636){\makebox(0,0){$\times$}}
\put(451,632){\makebox(0,0){$\times$}}
\put(451,629){\makebox(0,0){$\times$}}
\put(452,625){\makebox(0,0){$\times$}}
\put(452,621){\makebox(0,0){$\times$}}
\put(453,617){\makebox(0,0){$\times$}}
\put(453,614){\makebox(0,0){$\times$}}
\put(454,609){\makebox(0,0){$\times$}}
\put(454,606){\makebox(0,0){$\times$}}
\put(455,602){\makebox(0,0){$\times$}}
\put(455,598){\makebox(0,0){$\times$}}
\put(456,594){\makebox(0,0){$\times$}}
\put(457,590){\makebox(0,0){$\times$}}
\put(457,586){\makebox(0,0){$\times$}}
\put(458,583){\makebox(0,0){$\times$}}
\put(458,579){\makebox(0,0){$\times$}}
\put(459,575){\makebox(0,0){$\times$}}
\put(459,571){\makebox(0,0){$\times$}}
\put(460,567){\makebox(0,0){$\times$}}
\put(460,563){\makebox(0,0){$\times$}}
\put(461,559){\makebox(0,0){$\times$}}
\put(461,556){\makebox(0,0){$\times$}}
\put(462,551){\makebox(0,0){$\times$}}
\put(462,548){\makebox(0,0){$\times$}}
\put(463,540){\makebox(0,0){$\times$}}
\put(464,536){\makebox(0,0){$\times$}}
\put(464,532){\makebox(0,0){$\times$}}
\put(465,528){\makebox(0,0){$\times$}}
\put(465,524){\makebox(0,0){$\times$}}
\put(466,520){\makebox(0,0){$\times$}}
\put(466,517){\makebox(0,0){$\times$}}
\put(467,513){\makebox(0,0){$\times$}}
\put(467,509){\makebox(0,0){$\times$}}
\put(468,505){\makebox(0,0){$\times$}}
\put(468,502){\makebox(0,0){$\times$}}
\put(469,498){\makebox(0,0){$\times$}}
\put(469,494){\makebox(0,0){$\times$}}
\put(470,490){\makebox(0,0){$\times$}}
\put(470,487){\makebox(0,0){$\times$}}
\put(471,483){\makebox(0,0){$\times$}}
\put(471,479){\makebox(0,0){$\times$}}
\put(472,476){\makebox(0,0){$\times$}}
\put(472,472){\makebox(0,0){$\times$}}
\put(473,469){\makebox(0,0){$\times$}}
\put(474,465){\makebox(0,0){$\times$}}
\put(474,462){\makebox(0,0){$\times$}}
\put(475,458){\makebox(0,0){$\times$}}
\put(475,455){\makebox(0,0){$\times$}}
\put(476,452){\makebox(0,0){$\times$}}
\put(476,448){\makebox(0,0){$\times$}}
\put(477,444){\makebox(0,0){$\times$}}
\put(477,441){\makebox(0,0){$\times$}}
\put(478,436){\makebox(0,0){$\times$}}
\put(478,433){\makebox(0,0){$\times$}}
\put(479,430){\makebox(0,0){$\times$}}
\put(479,427){\makebox(0,0){$\times$}}
\put(480,424){\makebox(0,0){$\times$}}
\put(480,421){\makebox(0,0){$\times$}}
\put(481,418){\makebox(0,0){$\times$}}
\put(481,415){\makebox(0,0){$\times$}}
\put(482,412){\makebox(0,0){$\times$}}
\put(482,409){\makebox(0,0){$\times$}}
\put(483,406){\makebox(0,0){$\times$}}
\put(483,403){\makebox(0,0){$\times$}}
\put(484,400){\makebox(0,0){$\times$}}
\put(484,397){\makebox(0,0){$\times$}}
\put(485,395){\makebox(0,0){$\times$}}
\put(485,392){\makebox(0,0){$\times$}}
\put(486,389){\makebox(0,0){$\times$}}
\put(486,387){\makebox(0,0){$\times$}}
\put(487,384){\makebox(0,0){$\times$}}
\put(487,381){\makebox(0,0){$\times$}}
\put(488,379){\makebox(0,0){$\times$}}
\put(488,377){\makebox(0,0){$\times$}}
\put(489,374){\makebox(0,0){$\times$}}
\put(489,372){\makebox(0,0){$\times$}}
\put(490,370){\makebox(0,0){$\times$}}
\put(491,367){\makebox(0,0){$\times$}}
\put(491,365){\makebox(0,0){$\times$}}
\put(492,363){\makebox(0,0){$\times$}}
\put(492,361){\makebox(0,0){$\times$}}
\put(493,359){\makebox(0,0){$\times$}}
\put(493,357){\makebox(0,0){$\times$}}
\put(494,355){\makebox(0,0){$\times$}}
\put(494,354){\makebox(0,0){$\times$}}
\put(495,352){\makebox(0,0){$\times$}}
\put(495,350){\makebox(0,0){$\times$}}
\put(496,348){\makebox(0,0){$\times$}}
\put(496,347){\makebox(0,0){$\times$}}
\put(497,345){\makebox(0,0){$\times$}}
\put(497,343){\makebox(0,0){$\times$}}
\put(498,342){\makebox(0,0){$\times$}}
\put(498,341){\makebox(0,0){$\times$}}
\put(499,340){\makebox(0,0){$\times$}}
\put(499,338){\makebox(0,0){$\times$}}
\put(500,337){\makebox(0,0){$\times$}}
\put(500,336){\makebox(0,0){$\times$}}
\put(501,335){\makebox(0,0){$\times$}}
\put(501,334){\makebox(0,0){$\times$}}
\put(502,333){\makebox(0,0){$\times$}}
\put(502,332){\makebox(0,0){$\times$}}
\put(503,331){\makebox(0,0){$\times$}}
\put(503,330){\makebox(0,0){$\times$}}
\put(504,329){\makebox(0,0){$\times$}}
\put(504,329){\makebox(0,0){$\times$}}
\put(505,328){\makebox(0,0){$\times$}}
\put(505,328){\makebox(0,0){$\times$}}
\put(506,327){\makebox(0,0){$\times$}}
\put(506,326){\makebox(0,0){$\times$}}
\put(507,326){\makebox(0,0){$\times$}}
\put(508,326){\makebox(0,0){$\times$}}
\put(508,326){\makebox(0,0){$\times$}}
\put(509,326){\makebox(0,0){$\times$}}
\put(509,326){\makebox(0,0){$\times$}}
\put(510,325){\makebox(0,0){$\times$}}
\put(510,325){\makebox(0,0){$\times$}}
\put(511,325){\makebox(0,0){$\times$}}
\put(511,325){\makebox(0,0){$\times$}}
\put(512,325){\makebox(0,0){$\times$}}
\put(512,326){\makebox(0,0){$\times$}}
\put(513,326){\makebox(0,0){$\times$}}
\put(513,326){\makebox(0,0){$\times$}}
\put(514,326){\makebox(0,0){$\times$}}
\put(514,327){\makebox(0,0){$\times$}}
\put(515,328){\makebox(0,0){$\times$}}
\put(515,328){\makebox(0,0){$\times$}}
\put(516,329){\makebox(0,0){$\times$}}
\put(516,329){\makebox(0,0){$\times$}}
\put(517,330){\makebox(0,0){$\times$}}
\put(517,331){\makebox(0,0){$\times$}}
\put(518,332){\makebox(0,0){$\times$}}
\put(518,332){\makebox(0,0){$\times$}}
\put(519,334){\makebox(0,0){$\times$}}
\put(519,335){\makebox(0,0){$\times$}}
\put(520,335){\makebox(0,0){$\times$}}
\put(520,337){\makebox(0,0){$\times$}}
\put(521,338){\makebox(0,0){$\times$}}
\put(521,339){\makebox(0,0){$\times$}}
\put(522,340){\makebox(0,0){$\times$}}
\put(522,341){\makebox(0,0){$\times$}}
\put(523,343){\makebox(0,0){$\times$}}
\put(523,344){\makebox(0,0){$\times$}}
\put(524,346){\makebox(0,0){$\times$}}
\put(525,347){\makebox(0,0){$\times$}}
\put(525,349){\makebox(0,0){$\times$}}
\put(526,351){\makebox(0,0){$\times$}}
\put(526,352){\makebox(0,0){$\times$}}
\put(527,354){\makebox(0,0){$\times$}}
\put(527,356){\makebox(0,0){$\times$}}
\put(528,358){\makebox(0,0){$\times$}}
\put(528,360){\makebox(0,0){$\times$}}
\put(529,361){\makebox(0,0){$\times$}}
\put(529,363){\makebox(0,0){$\times$}}
\put(530,365){\makebox(0,0){$\times$}}
\put(530,367){\makebox(0,0){$\times$}}
\put(531,369){\makebox(0,0){$\times$}}
\put(531,372){\makebox(0,0){$\times$}}
\put(532,373){\makebox(0,0){$\times$}}
\put(532,376){\makebox(0,0){$\times$}}
\put(533,378){\makebox(0,0){$\times$}}
\put(533,380){\makebox(0,0){$\times$}}
\put(534,383){\makebox(0,0){$\times$}}
\put(534,385){\makebox(0,0){$\times$}}
\put(535,387){\makebox(0,0){$\times$}}
\put(535,390){\makebox(0,0){$\times$}}
\put(536,392){\makebox(0,0){$\times$}}
\put(536,395){\makebox(0,0){$\times$}}
\put(537,397){\makebox(0,0){$\times$}}
\put(537,400){\makebox(0,0){$\times$}}
\put(538,403){\makebox(0,0){$\times$}}
\put(538,405){\makebox(0,0){$\times$}}
\put(539,407){\makebox(0,0){$\times$}}
\put(539,410){\makebox(0,0){$\times$}}
\put(540,413){\makebox(0,0){$\times$}}
\put(540,416){\makebox(0,0){$\times$}}
\put(541,418){\makebox(0,0){$\times$}}
\put(542,421){\makebox(0,0){$\times$}}
\put(542,424){\makebox(0,0){$\times$}}
\put(543,427){\makebox(0,0){$\times$}}
\put(543,430){\makebox(0,0){$\times$}}
\put(544,433){\makebox(0,0){$\times$}}
\put(544,436){\makebox(0,0){$\times$}}
\put(545,439){\makebox(0,0){$\times$}}
\put(545,442){\makebox(0,0){$\times$}}
\put(546,444){\makebox(0,0){$\times$}}
\put(546,447){\makebox(0,0){$\times$}}
\put(547,450){\makebox(0,0){$\times$}}
\put(547,453){\makebox(0,0){$\times$}}
\put(548,456){\makebox(0,0){$\times$}}
\put(548,459){\makebox(0,0){$\times$}}
\put(549,463){\makebox(0,0){$\times$}}
\put(549,466){\makebox(0,0){$\times$}}
\put(550,469){\makebox(0,0){$\times$}}
\put(550,472){\makebox(0,0){$\times$}}
\put(551,475){\makebox(0,0){$\times$}}
\put(551,478){\makebox(0,0){$\times$}}
\put(552,482){\makebox(0,0){$\times$}}
\put(552,485){\makebox(0,0){$\times$}}
\put(553,488){\makebox(0,0){$\times$}}
\put(553,491){\makebox(0,0){$\times$}}
\put(554,494){\makebox(0,0){$\times$}}
\put(554,497){\makebox(0,0){$\times$}}
\put(555,501){\makebox(0,0){$\times$}}
\put(555,504){\makebox(0,0){$\times$}}
\put(556,508){\makebox(0,0){$\times$}}
\put(556,511){\makebox(0,0){$\times$}}
\put(557,514){\makebox(0,0){$\times$}}
\put(557,518){\makebox(0,0){$\times$}}
\put(558,521){\makebox(0,0){$\times$}}
\put(559,524){\makebox(0,0){$\times$}}
\put(559,528){\makebox(0,0){$\times$}}
\put(560,531){\makebox(0,0){$\times$}}
\put(560,534){\makebox(0,0){$\times$}}
\put(561,537){\makebox(0,0){$\times$}}
\put(561,540){\makebox(0,0){$\times$}}
\put(562,547){\makebox(0,0){$\times$}}
\put(563,550){\makebox(0,0){$\times$}}
\put(563,553){\makebox(0,0){$\times$}}
\put(564,556){\makebox(0,0){$\times$}}
\put(564,560){\makebox(0,0){$\times$}}
\put(565,563){\makebox(0,0){$\times$}}
\put(565,566){\makebox(0,0){$\times$}}
\put(566,569){\makebox(0,0){$\times$}}
\put(566,572){\makebox(0,0){$\times$}}
\put(567,576){\makebox(0,0){$\times$}}
\put(567,579){\makebox(0,0){$\times$}}
\put(568,582){\makebox(0,0){$\times$}}
\put(568,585){\makebox(0,0){$\times$}}
\put(569,588){\makebox(0,0){$\times$}}
\put(569,591){\makebox(0,0){$\times$}}
\put(570,594){\makebox(0,0){$\times$}}
\put(570,597){\makebox(0,0){$\times$}}
\put(571,600){\makebox(0,0){$\times$}}
\put(571,603){\makebox(0,0){$\times$}}
\put(572,606){\makebox(0,0){$\times$}}
\put(572,609){\makebox(0,0){$\times$}}
\put(573,612){\makebox(0,0){$\times$}}
\put(573,615){\makebox(0,0){$\times$}}
\put(574,618){\makebox(0,0){$\times$}}
\put(574,620){\makebox(0,0){$\times$}}
\put(575,623){\makebox(0,0){$\times$}}
\put(576,626){\makebox(0,0){$\times$}}
\put(576,629){\makebox(0,0){$\times$}}
\put(577,631){\makebox(0,0){$\times$}}
\put(577,634){\makebox(0,0){$\times$}}
\put(578,637){\makebox(0,0){$\times$}}
\put(578,640){\makebox(0,0){$\times$}}
\put(579,642){\makebox(0,0){$\times$}}
\put(579,644){\makebox(0,0){$\times$}}
\put(580,647){\makebox(0,0){$\times$}}
\put(580,649){\makebox(0,0){$\times$}}
\put(581,652){\makebox(0,0){$\times$}}
\put(581,654){\makebox(0,0){$\times$}}
\put(582,657){\makebox(0,0){$\times$}}
\put(582,659){\makebox(0,0){$\times$}}
\put(583,661){\makebox(0,0){$\times$}}
\put(583,664){\makebox(0,0){$\times$}}
\put(584,666){\makebox(0,0){$\times$}}
\put(584,668){\makebox(0,0){$\times$}}
\put(585,670){\makebox(0,0){$\times$}}
\put(585,672){\makebox(0,0){$\times$}}
\put(586,675){\makebox(0,0){$\times$}}
\put(586,676){\makebox(0,0){$\times$}}
\put(587,679){\makebox(0,0){$\times$}}
\put(587,681){\makebox(0,0){$\times$}}
\put(588,683){\makebox(0,0){$\times$}}
\put(588,684){\makebox(0,0){$\times$}}
\put(589,686){\makebox(0,0){$\times$}}
\put(589,688){\makebox(0,0){$\times$}}
\put(590,690){\makebox(0,0){$\times$}}
\put(590,692){\makebox(0,0){$\times$}}
\put(591,693){\makebox(0,0){$\times$}}
\put(591,695){\makebox(0,0){$\times$}}
\put(592,696){\makebox(0,0){$\times$}}
\put(593,698){\makebox(0,0){$\times$}}
\put(593,700){\makebox(0,0){$\times$}}
\put(594,701){\makebox(0,0){$\times$}}
\put(594,703){\makebox(0,0){$\times$}}
\put(595,704){\makebox(0,0){$\times$}}
\put(595,706){\makebox(0,0){$\times$}}
\put(596,707){\makebox(0,0){$\times$}}
\put(596,708){\makebox(0,0){$\times$}}
\put(597,709){\makebox(0,0){$\times$}}
\put(597,710){\makebox(0,0){$\times$}}
\put(598,712){\makebox(0,0){$\times$}}
\put(598,713){\makebox(0,0){$\times$}}
\put(599,713){\makebox(0,0){$\times$}}
\put(599,715){\makebox(0,0){$\times$}}
\put(600,715){\makebox(0,0){$\times$}}
\put(600,716){\makebox(0,0){$\times$}}
\put(601,717){\makebox(0,0){$\times$}}
\put(601,718){\makebox(0,0){$\times$}}
\put(602,718){\makebox(0,0){$\times$}}
\put(602,719){\makebox(0,0){$\times$}}
\put(603,720){\makebox(0,0){$\times$}}
\put(603,720){\makebox(0,0){$\times$}}
\put(604,721){\makebox(0,0){$\times$}}
\put(604,721){\makebox(0,0){$\times$}}
\put(605,722){\makebox(0,0){$\times$}}
\put(605,722){\makebox(0,0){$\times$}}
\put(606,722){\makebox(0,0){$\times$}}
\put(606,722){\makebox(0,0){$\times$}}
\put(607,722){\makebox(0,0){$\times$}}
\put(607,723){\makebox(0,0){$\times$}}
\put(608,723){\makebox(0,0){$\times$}}
\put(608,723){\makebox(0,0){$\times$}}
\put(609,723){\makebox(0,0){$\times$}}
\put(610,723){\makebox(0,0){$\times$}}
\put(610,723){\makebox(0,0){$\times$}}
\put(611,723){\makebox(0,0){$\times$}}
\put(611,723){\makebox(0,0){$\times$}}
\put(612,722){\makebox(0,0){$\times$}}
\put(612,722){\makebox(0,0){$\times$}}
\put(613,722){\makebox(0,0){$\times$}}
\put(613,722){\makebox(0,0){$\times$}}
\put(614,721){\makebox(0,0){$\times$}}
\put(614,721){\makebox(0,0){$\times$}}
\put(615,720){\makebox(0,0){$\times$}}
\put(615,720){\makebox(0,0){$\times$}}
\put(616,719){\makebox(0,0){$\times$}}
\put(616,718){\makebox(0,0){$\times$}}
\put(617,718){\makebox(0,0){$\times$}}
\put(617,717){\makebox(0,0){$\times$}}
\put(618,716){\makebox(0,0){$\times$}}
\put(618,715){\makebox(0,0){$\times$}}
\put(619,715){\makebox(0,0){$\times$}}
\put(619,713){\makebox(0,0){$\times$}}
\put(620,713){\makebox(0,0){$\times$}}
\put(620,712){\makebox(0,0){$\times$}}
\put(621,711){\makebox(0,0){$\times$}}
\put(621,710){\makebox(0,0){$\times$}}
\put(622,709){\makebox(0,0){$\times$}}
\put(622,707){\makebox(0,0){$\times$}}
\put(623,706){\makebox(0,0){$\times$}}
\put(623,705){\makebox(0,0){$\times$}}
\put(624,704){\makebox(0,0){$\times$}}
\put(624,703){\makebox(0,0){$\times$}}
\put(625,701){\makebox(0,0){$\times$}}
\put(625,699){\makebox(0,0){$\times$}}
\put(626,698){\makebox(0,0){$\times$}}
\put(627,696){\makebox(0,0){$\times$}}
\put(627,695){\makebox(0,0){$\times$}}
\put(628,693){\makebox(0,0){$\times$}}
\put(628,692){\makebox(0,0){$\times$}}
\put(629,690){\makebox(0,0){$\times$}}
\put(629,689){\makebox(0,0){$\times$}}
\put(630,687){\makebox(0,0){$\times$}}
\put(630,686){\makebox(0,0){$\times$}}
\put(631,684){\makebox(0,0){$\times$}}
\put(631,682){\makebox(0,0){$\times$}}
\put(632,680){\makebox(0,0){$\times$}}
\put(632,678){\makebox(0,0){$\times$}}
\put(633,676){\makebox(0,0){$\times$}}
\put(633,674){\makebox(0,0){$\times$}}
\put(634,672){\makebox(0,0){$\times$}}
\put(634,670){\makebox(0,0){$\times$}}
\put(635,669){\makebox(0,0){$\times$}}
\put(635,667){\makebox(0,0){$\times$}}
\put(636,664){\makebox(0,0){$\times$}}
\put(636,663){\makebox(0,0){$\times$}}
\put(637,660){\makebox(0,0){$\times$}}
\put(637,658){\makebox(0,0){$\times$}}
\put(638,656){\makebox(0,0){$\times$}}
\put(638,654){\makebox(0,0){$\times$}}
\put(639,652){\makebox(0,0){$\times$}}
\put(639,649){\makebox(0,0){$\times$}}
\put(640,647){\makebox(0,0){$\times$}}
\put(640,645){\makebox(0,0){$\times$}}
\put(641,643){\makebox(0,0){$\times$}}
\put(641,640){\makebox(0,0){$\times$}}
\put(642,638){\makebox(0,0){$\times$}}
\put(643,635){\makebox(0,0){$\times$}}
\put(643,633){\makebox(0,0){$\times$}}
\put(644,631){\makebox(0,0){$\times$}}
\put(644,628){\makebox(0,0){$\times$}}
\put(645,626){\makebox(0,0){$\times$}}
\put(645,623){\makebox(0,0){$\times$}}
\put(646,621){\makebox(0,0){$\times$}}
\put(646,618){\makebox(0,0){$\times$}}
\put(647,616){\makebox(0,0){$\times$}}
\put(647,614){\makebox(0,0){$\times$}}
\put(648,611){\makebox(0,0){$\times$}}
\put(648,609){\makebox(0,0){$\times$}}
\put(649,606){\makebox(0,0){$\times$}}
\put(649,603){\makebox(0,0){$\times$}}
\put(650,601){\makebox(0,0){$\times$}}
\put(650,598){\makebox(0,0){$\times$}}
\put(651,595){\makebox(0,0){$\times$}}
\put(651,593){\makebox(0,0){$\times$}}
\put(652,591){\makebox(0,0){$\times$}}
\put(652,588){\makebox(0,0){$\times$}}
\put(653,585){\makebox(0,0){$\times$}}
\put(653,583){\makebox(0,0){$\times$}}
\put(654,580){\makebox(0,0){$\times$}}
\put(654,577){\makebox(0,0){$\times$}}
\put(655,575){\makebox(0,0){$\times$}}
\put(655,572){\makebox(0,0){$\times$}}
\put(656,569){\makebox(0,0){$\times$}}
\put(656,567){\makebox(0,0){$\times$}}
\put(657,565){\makebox(0,0){$\times$}}
\put(657,562){\makebox(0,0){$\times$}}
\put(658,559){\makebox(0,0){$\times$}}
\put(658,556){\makebox(0,0){$\times$}}
\put(659,554){\makebox(0,0){$\times$}}
\put(660,551){\makebox(0,0){$\times$}}
\put(660,549){\makebox(0,0){$\times$}}
\put(662,541){\makebox(0,0){$\times$}}
\put(662,538){\makebox(0,0){$\times$}}
\put(663,536){\makebox(0,0){$\times$}}
\put(663,533){\makebox(0,0){$\times$}}
\put(664,531){\makebox(0,0){$\times$}}
\put(664,528){\makebox(0,0){$\times$}}
\put(665,525){\makebox(0,0){$\times$}}
\put(665,523){\makebox(0,0){$\times$}}
\put(666,520){\makebox(0,0){$\times$}}
\put(666,518){\makebox(0,0){$\times$}}
\put(667,516){\makebox(0,0){$\times$}}
\put(667,513){\makebox(0,0){$\times$}}
\put(668,510){\makebox(0,0){$\times$}}
\put(668,508){\makebox(0,0){$\times$}}
\put(669,505){\makebox(0,0){$\times$}}
\put(669,503){\makebox(0,0){$\times$}}
\put(670,501){\makebox(0,0){$\times$}}
\put(670,498){\makebox(0,0){$\times$}}
\put(671,496){\makebox(0,0){$\times$}}
\put(671,494){\makebox(0,0){$\times$}}
\put(672,491){\makebox(0,0){$\times$}}
\put(672,489){\makebox(0,0){$\times$}}
\put(673,487){\makebox(0,0){$\times$}}
\put(673,484){\makebox(0,0){$\times$}}
\put(674,482){\makebox(0,0){$\times$}}
\put(674,480){\makebox(0,0){$\times$}}
\put(675,478){\makebox(0,0){$\times$}}
\put(675,476){\makebox(0,0){$\times$}}
\put(676,473){\makebox(0,0){$\times$}}
\put(677,471){\makebox(0,0){$\times$}}
\put(677,469){\makebox(0,0){$\times$}}
\put(678,467){\makebox(0,0){$\times$}}
\put(678,465){\makebox(0,0){$\times$}}
\put(679,463){\makebox(0,0){$\times$}}
\put(679,461){\makebox(0,0){$\times$}}
\put(680,459){\makebox(0,0){$\times$}}
\put(680,456){\makebox(0,0){$\times$}}
\put(681,455){\makebox(0,0){$\times$}}
\put(681,452){\makebox(0,0){$\times$}}
\put(682,450){\makebox(0,0){$\times$}}
\put(682,448){\makebox(0,0){$\times$}}
\put(683,447){\makebox(0,0){$\times$}}
\put(683,445){\makebox(0,0){$\times$}}
\put(684,443){\makebox(0,0){$\times$}}
\put(684,441){\makebox(0,0){$\times$}}
\put(685,439){\makebox(0,0){$\times$}}
\put(685,438){\makebox(0,0){$\times$}}
\put(686,436){\makebox(0,0){$\times$}}
\put(686,435){\makebox(0,0){$\times$}}
\put(687,433){\makebox(0,0){$\times$}}
\put(687,432){\makebox(0,0){$\times$}}
\put(688,430){\makebox(0,0){$\times$}}
\put(688,429){\makebox(0,0){$\times$}}
\put(689,427){\makebox(0,0){$\times$}}
\put(689,426){\makebox(0,0){$\times$}}
\put(690,425){\makebox(0,0){$\times$}}
\put(690,424){\makebox(0,0){$\times$}}
\put(691,422){\makebox(0,0){$\times$}}
\put(691,421){\makebox(0,0){$\times$}}
\put(692,420){\makebox(0,0){$\times$}}
\put(692,419){\makebox(0,0){$\times$}}
\put(693,418){\makebox(0,0){$\times$}}
\put(694,416){\makebox(0,0){$\times$}}
\put(694,416){\makebox(0,0){$\times$}}
\put(695,415){\makebox(0,0){$\times$}}
\put(695,413){\makebox(0,0){$\times$}}
\put(696,413){\makebox(0,0){$\times$}}
\put(696,412){\makebox(0,0){$\times$}}
\put(697,411){\makebox(0,0){$\times$}}
\put(697,410){\makebox(0,0){$\times$}}
\put(698,410){\makebox(0,0){$\times$}}
\put(698,409){\makebox(0,0){$\times$}}
\put(699,408){\makebox(0,0){$\times$}}
\put(699,407){\makebox(0,0){$\times$}}
\put(700,407){\makebox(0,0){$\times$}}
\put(700,406){\makebox(0,0){$\times$}}
\put(701,406){\makebox(0,0){$\times$}}
\put(701,406){\makebox(0,0){$\times$}}
\put(702,405){\makebox(0,0){$\times$}}
\put(702,405){\makebox(0,0){$\times$}}
\put(703,404){\makebox(0,0){$\times$}}
\put(703,404){\makebox(0,0){$\times$}}
\put(704,404){\makebox(0,0){$\times$}}
\put(704,404){\makebox(0,0){$\times$}}
\put(705,404){\makebox(0,0){$\times$}}
\put(705,403){\makebox(0,0){$\times$}}
\put(706,403){\makebox(0,0){$\times$}}
\put(706,403){\makebox(0,0){$\times$}}
\put(707,403){\makebox(0,0){$\times$}}
\put(707,403){\makebox(0,0){$\times$}}
\put(708,403){\makebox(0,0){$\times$}}
\put(708,403){\makebox(0,0){$\times$}}
\put(709,403){\makebox(0,0){$\times$}}
\put(709,403){\makebox(0,0){$\times$}}
\put(710,404){\makebox(0,0){$\times$}}
\put(711,404){\makebox(0,0){$\times$}}
\put(711,404){\makebox(0,0){$\times$}}
\put(712,404){\makebox(0,0){$\times$}}
\put(712,405){\makebox(0,0){$\times$}}
\put(713,406){\makebox(0,0){$\times$}}
\put(713,406){\makebox(0,0){$\times$}}
\put(714,406){\makebox(0,0){$\times$}}
\put(714,407){\makebox(0,0){$\times$}}
\put(715,407){\makebox(0,0){$\times$}}
\put(715,408){\makebox(0,0){$\times$}}
\put(716,409){\makebox(0,0){$\times$}}
\put(716,409){\makebox(0,0){$\times$}}
\put(717,410){\makebox(0,0){$\times$}}
\put(717,410){\makebox(0,0){$\times$}}
\put(718,412){\makebox(0,0){$\times$}}
\put(718,412){\makebox(0,0){$\times$}}
\put(719,413){\makebox(0,0){$\times$}}
\put(719,414){\makebox(0,0){$\times$}}
\put(720,415){\makebox(0,0){$\times$}}
\put(720,416){\makebox(0,0){$\times$}}
\put(721,417){\makebox(0,0){$\times$}}
\put(721,418){\makebox(0,0){$\times$}}
\put(722,419){\makebox(0,0){$\times$}}
\put(722,420){\makebox(0,0){$\times$}}
\put(723,421){\makebox(0,0){$\times$}}
\put(723,422){\makebox(0,0){$\times$}}
\put(724,423){\makebox(0,0){$\times$}}
\put(724,424){\makebox(0,0){$\times$}}
\put(725,426){\makebox(0,0){$\times$}}
\put(725,427){\makebox(0,0){$\times$}}
\put(726,428){\makebox(0,0){$\times$}}
\put(726,429){\makebox(0,0){$\times$}}
\put(727,430){\makebox(0,0){$\times$}}
\put(728,432){\makebox(0,0){$\times$}}
\put(728,433){\makebox(0,0){$\times$}}
\put(729,435){\makebox(0,0){$\times$}}
\put(729,436){\makebox(0,0){$\times$}}
\put(730,438){\makebox(0,0){$\times$}}
\put(730,439){\makebox(0,0){$\times$}}
\put(731,441){\makebox(0,0){$\times$}}
\put(731,442){\makebox(0,0){$\times$}}
\put(732,444){\makebox(0,0){$\times$}}
\put(732,445){\makebox(0,0){$\times$}}
\put(733,447){\makebox(0,0){$\times$}}
\put(733,448){\makebox(0,0){$\times$}}
\put(734,450){\makebox(0,0){$\times$}}
\put(734,452){\makebox(0,0){$\times$}}
\put(735,453){\makebox(0,0){$\times$}}
\put(735,455){\makebox(0,0){$\times$}}
\put(736,457){\makebox(0,0){$\times$}}
\put(736,459){\makebox(0,0){$\times$}}
\put(737,460){\makebox(0,0){$\times$}}
\put(737,462){\makebox(0,0){$\times$}}
\put(738,464){\makebox(0,0){$\times$}}
\put(738,465){\makebox(0,0){$\times$}}
\put(739,467){\makebox(0,0){$\times$}}
\put(739,469){\makebox(0,0){$\times$}}
\put(740,471){\makebox(0,0){$\times$}}
\put(740,473){\makebox(0,0){$\times$}}
\put(741,475){\makebox(0,0){$\times$}}
\put(741,477){\makebox(0,0){$\times$}}
\put(742,479){\makebox(0,0){$\times$}}
\put(742,481){\makebox(0,0){$\times$}}
\put(743,483){\makebox(0,0){$\times$}}
\put(743,485){\makebox(0,0){$\times$}}
\put(744,487){\makebox(0,0){$\times$}}
\put(745,489){\makebox(0,0){$\times$}}
\put(745,491){\makebox(0,0){$\times$}}
\put(746,493){\makebox(0,0){$\times$}}
\put(746,495){\makebox(0,0){$\times$}}
\put(747,497){\makebox(0,0){$\times$}}
\put(747,499){\makebox(0,0){$\times$}}
\put(748,501){\makebox(0,0){$\times$}}
\put(748,503){\makebox(0,0){$\times$}}
\put(749,505){\makebox(0,0){$\times$}}
\put(749,507){\makebox(0,0){$\times$}}
\put(750,510){\makebox(0,0){$\times$}}
\put(750,512){\makebox(0,0){$\times$}}
\put(751,514){\makebox(0,0){$\times$}}
\put(751,516){\makebox(0,0){$\times$}}
\put(752,518){\makebox(0,0){$\times$}}
\put(752,520){\makebox(0,0){$\times$}}
\put(753,522){\makebox(0,0){$\times$}}
\put(753,524){\makebox(0,0){$\times$}}
\put(754,527){\makebox(0,0){$\times$}}
\put(754,528){\makebox(0,0){$\times$}}
\put(755,531){\makebox(0,0){$\times$}}
\put(755,533){\makebox(0,0){$\times$}}
\put(756,535){\makebox(0,0){$\times$}}
\put(756,537){\makebox(0,0){$\times$}}
\put(757,539){\makebox(0,0){$\times$}}
\put(757,541){\makebox(0,0){$\times$}}
\put(758,543){\makebox(0,0){$\times$}}
\put(759,547){\makebox(0,0){$\times$}}
\put(759,549){\makebox(0,0){$\times$}}
\put(760,551){\makebox(0,0){$\times$}}
\put(760,553){\makebox(0,0){$\times$}}
\put(761,556){\makebox(0,0){$\times$}}
\put(762,557){\makebox(0,0){$\times$}}
\put(762,560){\makebox(0,0){$\times$}}
\put(763,562){\makebox(0,0){$\times$}}
\put(763,563){\makebox(0,0){$\times$}}
\put(764,566){\makebox(0,0){$\times$}}
\put(764,568){\makebox(0,0){$\times$}}
\put(765,569){\makebox(0,0){$\times$}}
\put(765,572){\makebox(0,0){$\times$}}
\put(766,574){\makebox(0,0){$\times$}}
\put(766,576){\makebox(0,0){$\times$}}
\put(767,577){\makebox(0,0){$\times$}}
\put(767,580){\makebox(0,0){$\times$}}
\put(768,582){\makebox(0,0){$\times$}}
\put(768,583){\makebox(0,0){$\times$}}
\put(769,585){\makebox(0,0){$\times$}}
\put(769,587){\makebox(0,0){$\times$}}
\put(770,589){\makebox(0,0){$\times$}}
\put(770,591){\makebox(0,0){$\times$}}
\put(771,592){\makebox(0,0){$\times$}}
\put(771,594){\makebox(0,0){$\times$}}
\put(772,596){\makebox(0,0){$\times$}}
\put(772,598){\makebox(0,0){$\times$}}
\put(773,600){\makebox(0,0){$\times$}}
\put(773,602){\makebox(0,0){$\times$}}
\put(774,603){\makebox(0,0){$\times$}}
\put(774,605){\makebox(0,0){$\times$}}
\put(775,606){\makebox(0,0){$\times$}}
\put(775,608){\makebox(0,0){$\times$}}
\put(776,610){\makebox(0,0){$\times$}}
\put(776,612){\makebox(0,0){$\times$}}
\put(777,613){\makebox(0,0){$\times$}}
\put(777,615){\makebox(0,0){$\times$}}
\put(778,616){\makebox(0,0){$\times$}}
\put(779,618){\makebox(0,0){$\times$}}
\put(779,619){\makebox(0,0){$\times$}}
\put(780,621){\makebox(0,0){$\times$}}
\put(780,622){\makebox(0,0){$\times$}}
\put(781,624){\makebox(0,0){$\times$}}
\put(781,625){\makebox(0,0){$\times$}}
\put(782,626){\makebox(0,0){$\times$}}
\put(782,628){\makebox(0,0){$\times$}}
\put(783,629){\makebox(0,0){$\times$}}
\put(783,631){\makebox(0,0){$\times$}}
\put(784,632){\makebox(0,0){$\times$}}
\put(784,633){\makebox(0,0){$\times$}}
\put(785,635){\makebox(0,0){$\times$}}
\put(785,635){\makebox(0,0){$\times$}}
\put(786,637){\makebox(0,0){$\times$}}
\put(786,638){\makebox(0,0){$\times$}}
\put(787,639){\makebox(0,0){$\times$}}
\put(787,640){\makebox(0,0){$\times$}}
\put(788,641){\makebox(0,0){$\times$}}
\put(788,643){\makebox(0,0){$\times$}}
\put(789,643){\makebox(0,0){$\times$}}
\put(789,644){\makebox(0,0){$\times$}}
\put(790,645){\makebox(0,0){$\times$}}
\put(790,646){\makebox(0,0){$\times$}}
\put(791,647){\makebox(0,0){$\times$}}
\put(791,648){\makebox(0,0){$\times$}}
\put(792,649){\makebox(0,0){$\times$}}
\put(792,650){\makebox(0,0){$\times$}}
\put(793,651){\makebox(0,0){$\times$}}
\put(793,651){\makebox(0,0){$\times$}}
\put(794,652){\makebox(0,0){$\times$}}
\put(794,652){\makebox(0,0){$\times$}}
\put(795,653){\makebox(0,0){$\times$}}
\put(796,654){\makebox(0,0){$\times$}}
\put(796,655){\makebox(0,0){$\times$}}
\put(797,655){\makebox(0,0){$\times$}}
\put(797,655){\makebox(0,0){$\times$}}
\put(798,656){\makebox(0,0){$\times$}}
\put(798,657){\makebox(0,0){$\times$}}
\put(799,657){\makebox(0,0){$\times$}}
\put(799,657){\makebox(0,0){$\times$}}
\put(800,658){\makebox(0,0){$\times$}}
\put(800,658){\makebox(0,0){$\times$}}
\put(801,658){\makebox(0,0){$\times$}}
\put(801,658){\makebox(0,0){$\times$}}
\put(802,659){\makebox(0,0){$\times$}}
\put(802,659){\makebox(0,0){$\times$}}
\put(803,659){\makebox(0,0){$\times$}}
\put(803,660){\makebox(0,0){$\times$}}
\put(804,660){\makebox(0,0){$\times$}}
\put(804,660){\makebox(0,0){$\times$}}
\put(805,660){\makebox(0,0){$\times$}}
\put(805,660){\makebox(0,0){$\times$}}
\put(806,660){\makebox(0,0){$\times$}}
\put(806,660){\makebox(0,0){$\times$}}
\put(807,660){\makebox(0,0){$\times$}}
\put(807,660){\makebox(0,0){$\times$}}
\put(808,660){\makebox(0,0){$\times$}}
\put(808,660){\makebox(0,0){$\times$}}
\put(809,659){\makebox(0,0){$\times$}}
\put(809,659){\makebox(0,0){$\times$}}
\put(810,658){\makebox(0,0){$\times$}}
\put(810,658){\makebox(0,0){$\times$}}
\put(811,658){\makebox(0,0){$\times$}}
\put(811,658){\makebox(0,0){$\times$}}
\put(812,657){\makebox(0,0){$\times$}}
\put(813,657){\makebox(0,0){$\times$}}
\put(813,657){\makebox(0,0){$\times$}}
\put(814,656){\makebox(0,0){$\times$}}
\put(814,655){\makebox(0,0){$\times$}}
\put(815,655){\makebox(0,0){$\times$}}
\put(815,655){\makebox(0,0){$\times$}}
\put(816,654){\makebox(0,0){$\times$}}
\put(816,653){\makebox(0,0){$\times$}}
\put(817,652){\makebox(0,0){$\times$}}
\put(817,652){\makebox(0,0){$\times$}}
\put(818,651){\makebox(0,0){$\times$}}
\put(818,651){\makebox(0,0){$\times$}}
\put(819,650){\makebox(0,0){$\times$}}
\put(819,649){\makebox(0,0){$\times$}}
\put(820,648){\makebox(0,0){$\times$}}
\put(820,647){\makebox(0,0){$\times$}}
\put(821,647){\makebox(0,0){$\times$}}
\put(821,646){\makebox(0,0){$\times$}}
\put(822,645){\makebox(0,0){$\times$}}
\put(822,644){\makebox(0,0){$\times$}}
\put(823,643){\makebox(0,0){$\times$}}
\put(823,642){\makebox(0,0){$\times$}}
\put(824,641){\makebox(0,0){$\times$}}
\put(824,640){\makebox(0,0){$\times$}}
\put(825,639){\makebox(0,0){$\times$}}
\put(825,638){\makebox(0,0){$\times$}}
\put(826,637){\makebox(0,0){$\times$}}
\put(826,635){\makebox(0,0){$\times$}}
\put(827,635){\makebox(0,0){$\times$}}
\put(827,634){\makebox(0,0){$\times$}}
\put(828,632){\makebox(0,0){$\times$}}
\put(828,631){\makebox(0,0){$\times$}}
\put(829,630){\makebox(0,0){$\times$}}
\put(830,629){\makebox(0,0){$\times$}}
\put(830,628){\makebox(0,0){$\times$}}
\put(831,626){\makebox(0,0){$\times$}}
\put(831,624){\makebox(0,0){$\times$}}
\put(832,623){\makebox(0,0){$\times$}}
\put(832,622){\makebox(0,0){$\times$}}
\put(833,621){\makebox(0,0){$\times$}}
\put(833,619){\makebox(0,0){$\times$}}
\put(834,618){\makebox(0,0){$\times$}}
\put(834,617){\makebox(0,0){$\times$}}
\put(835,615){\makebox(0,0){$\times$}}
\put(835,614){\makebox(0,0){$\times$}}
\put(836,612){\makebox(0,0){$\times$}}
\put(836,611){\makebox(0,0){$\times$}}
\put(837,609){\makebox(0,0){$\times$}}
\put(837,608){\makebox(0,0){$\times$}}
\put(838,606){\makebox(0,0){$\times$}}
\put(838,605){\makebox(0,0){$\times$}}
\put(839,603){\makebox(0,0){$\times$}}
\put(839,602){\makebox(0,0){$\times$}}
\put(840,600){\makebox(0,0){$\times$}}
\put(840,598){\makebox(0,0){$\times$}}
\put(841,597){\makebox(0,0){$\times$}}
\put(841,595){\makebox(0,0){$\times$}}
\put(842,594){\makebox(0,0){$\times$}}
\put(842,592){\makebox(0,0){$\times$}}
\put(843,590){\makebox(0,0){$\times$}}
\put(843,589){\makebox(0,0){$\times$}}
\put(844,587){\makebox(0,0){$\times$}}
\put(844,585){\makebox(0,0){$\times$}}
\put(845,584){\makebox(0,0){$\times$}}
\put(845,582){\makebox(0,0){$\times$}}
\put(846,580){\makebox(0,0){$\times$}}
\put(847,579){\makebox(0,0){$\times$}}
\put(847,577){\makebox(0,0){$\times$}}
\put(848,575){\makebox(0,0){$\times$}}
\put(848,574){\makebox(0,0){$\times$}}
\put(849,572){\makebox(0,0){$\times$}}
\put(849,570){\makebox(0,0){$\times$}}
\put(850,568){\makebox(0,0){$\times$}}
\put(850,566){\makebox(0,0){$\times$}}
\put(851,565){\makebox(0,0){$\times$}}
\put(851,563){\makebox(0,0){$\times$}}
\put(852,562){\makebox(0,0){$\times$}}
\put(852,560){\makebox(0,0){$\times$}}
\put(853,558){\makebox(0,0){$\times$}}
\put(853,556){\makebox(0,0){$\times$}}
\put(854,555){\makebox(0,0){$\times$}}
\put(854,553){\makebox(0,0){$\times$}}
\put(855,551){\makebox(0,0){$\times$}}
\put(855,549){\makebox(0,0){$\times$}}
\put(856,548){\makebox(0,0){$\times$}}
\put(857,543){\makebox(0,0){$\times$}}
\put(858,541){\makebox(0,0){$\times$}}
\put(858,539){\makebox(0,0){$\times$}}
\put(859,538){\makebox(0,0){$\times$}}
\put(859,536){\makebox(0,0){$\times$}}
\put(860,534){\makebox(0,0){$\times$}}
\put(860,533){\makebox(0,0){$\times$}}
\put(861,531){\makebox(0,0){$\times$}}
\put(861,530){\makebox(0,0){$\times$}}
\put(862,528){\makebox(0,0){$\times$}}
\put(862,527){\makebox(0,0){$\times$}}
\put(863,525){\makebox(0,0){$\times$}}
\put(864,523){\makebox(0,0){$\times$}}
\put(864,522){\makebox(0,0){$\times$}}
\put(865,520){\makebox(0,0){$\times$}}
\put(865,519){\makebox(0,0){$\times$}}
\put(866,517){\makebox(0,0){$\times$}}
\put(866,516){\makebox(0,0){$\times$}}
\put(867,514){\makebox(0,0){$\times$}}
\put(867,513){\makebox(0,0){$\times$}}
\put(868,511){\makebox(0,0){$\times$}}
\put(868,510){\makebox(0,0){$\times$}}
\put(869,508){\makebox(0,0){$\times$}}
\put(869,507){\makebox(0,0){$\times$}}
\put(870,505){\makebox(0,0){$\times$}}
\put(870,504){\makebox(0,0){$\times$}}
\put(871,502){\makebox(0,0){$\times$}}
\put(871,501){\makebox(0,0){$\times$}}
\put(872,499){\makebox(0,0){$\times$}}
\put(872,498){\makebox(0,0){$\times$}}
\put(873,497){\makebox(0,0){$\times$}}
\put(873,495){\makebox(0,0){$\times$}}
\put(874,494){\makebox(0,0){$\times$}}
\put(874,493){\makebox(0,0){$\times$}}
\put(875,491){\makebox(0,0){$\times$}}
\put(875,490){\makebox(0,0){$\times$}}
\put(876,489){\makebox(0,0){$\times$}}
\put(876,488){\makebox(0,0){$\times$}}
\put(877,487){\makebox(0,0){$\times$}}
\put(877,485){\makebox(0,0){$\times$}}
\put(878,484){\makebox(0,0){$\times$}}
\put(878,483){\makebox(0,0){$\times$}}
\put(879,482){\makebox(0,0){$\times$}}
\put(879,481){\makebox(0,0){$\times$}}
\put(880,479){\makebox(0,0){$\times$}}
\put(881,478){\makebox(0,0){$\times$}}
\put(881,478){\makebox(0,0){$\times$}}
\put(882,476){\makebox(0,0){$\times$}}
\put(882,475){\makebox(0,0){$\times$}}
\put(883,475){\makebox(0,0){$\times$}}
\put(883,473){\makebox(0,0){$\times$}}
\put(884,473){\makebox(0,0){$\times$}}
\put(884,471){\makebox(0,0){$\times$}}
\put(885,471){\makebox(0,0){$\times$}}
\put(885,470){\makebox(0,0){$\times$}}
\put(886,468){\makebox(0,0){$\times$}}
\put(886,468){\makebox(0,0){$\times$}}
\put(887,467){\makebox(0,0){$\times$}}
\put(887,467){\makebox(0,0){$\times$}}
\put(888,465){\makebox(0,0){$\times$}}
\put(888,465){\makebox(0,0){$\times$}}
\put(889,464){\makebox(0,0){$\times$}}
\put(889,464){\makebox(0,0){$\times$}}
\put(890,463){\makebox(0,0){$\times$}}
\put(890,462){\makebox(0,0){$\times$}}
\put(891,462){\makebox(0,0){$\times$}}
\put(891,461){\makebox(0,0){$\times$}}
\put(892,461){\makebox(0,0){$\times$}}
\put(892,460){\makebox(0,0){$\times$}}
\put(893,459){\makebox(0,0){$\times$}}
\put(893,459){\makebox(0,0){$\times$}}
\put(894,459){\makebox(0,0){$\times$}}
\put(894,458){\makebox(0,0){$\times$}}
\put(895,458){\makebox(0,0){$\times$}}
\put(895,457){\makebox(0,0){$\times$}}
\put(896,457){\makebox(0,0){$\times$}}
\put(896,456){\makebox(0,0){$\times$}}
\put(897,456){\makebox(0,0){$\times$}}
\put(898,456){\makebox(0,0){$\times$}}
\put(898,456){\makebox(0,0){$\times$}}
\put(899,456){\makebox(0,0){$\times$}}
\put(899,455){\makebox(0,0){$\times$}}
\put(900,455){\makebox(0,0){$\times$}}
\put(900,455){\makebox(0,0){$\times$}}
\put(901,455){\makebox(0,0){$\times$}}
\put(901,455){\makebox(0,0){$\times$}}
\put(902,455){\makebox(0,0){$\times$}}
\put(902,455){\makebox(0,0){$\times$}}
\put(903,455){\makebox(0,0){$\times$}}
\put(903,455){\makebox(0,0){$\times$}}
\put(904,455){\makebox(0,0){$\times$}}
\put(904,455){\makebox(0,0){$\times$}}
\put(905,455){\makebox(0,0){$\times$}}
\put(905,455){\makebox(0,0){$\times$}}
\put(906,455){\makebox(0,0){$\times$}}
\put(906,455){\makebox(0,0){$\times$}}
\put(907,455){\makebox(0,0){$\times$}}
\put(907,456){\makebox(0,0){$\times$}}
\put(908,456){\makebox(0,0){$\times$}}
\put(908,456){\makebox(0,0){$\times$}}
\put(909,456){\makebox(0,0){$\times$}}
\put(909,457){\makebox(0,0){$\times$}}
\put(910,457){\makebox(0,0){$\times$}}
\put(910,458){\makebox(0,0){$\times$}}
\put(911,458){\makebox(0,0){$\times$}}
\put(911,458){\makebox(0,0){$\times$}}
\put(912,459){\makebox(0,0){$\times$}}
\put(912,459){\makebox(0,0){$\times$}}
\put(913,459){\makebox(0,0){$\times$}}
\put(913,460){\makebox(0,0){$\times$}}
\put(914,461){\makebox(0,0){$\times$}}
\put(915,461){\makebox(0,0){$\times$}}
\put(915,462){\makebox(0,0){$\times$}}
\put(916,462){\makebox(0,0){$\times$}}
\put(916,463){\makebox(0,0){$\times$}}
\put(917,464){\makebox(0,0){$\times$}}
\put(917,464){\makebox(0,0){$\times$}}
\put(918,465){\makebox(0,0){$\times$}}
\put(918,465){\makebox(0,0){$\times$}}
\put(919,466){\makebox(0,0){$\times$}}
\put(919,467){\makebox(0,0){$\times$}}
\put(920,468){\makebox(0,0){$\times$}}
\put(920,468){\makebox(0,0){$\times$}}
\put(921,469){\makebox(0,0){$\times$}}
\put(921,470){\makebox(0,0){$\times$}}
\put(922,471){\makebox(0,0){$\times$}}
\put(922,471){\makebox(0,0){$\times$}}
\put(923,473){\makebox(0,0){$\times$}}
\put(923,473){\makebox(0,0){$\times$}}
\put(924,475){\makebox(0,0){$\times$}}
\put(924,475){\makebox(0,0){$\times$}}
\put(925,476){\makebox(0,0){$\times$}}
\put(925,477){\makebox(0,0){$\times$}}
\put(926,478){\makebox(0,0){$\times$}}
\put(926,479){\makebox(0,0){$\times$}}
\put(927,480){\makebox(0,0){$\times$}}
\put(927,481){\makebox(0,0){$\times$}}
\put(928,482){\makebox(0,0){$\times$}}
\put(928,483){\makebox(0,0){$\times$}}
\put(929,484){\makebox(0,0){$\times$}}
\put(929,485){\makebox(0,0){$\times$}}
\put(930,486){\makebox(0,0){$\times$}}
\put(930,487){\makebox(0,0){$\times$}}
\put(931,488){\makebox(0,0){$\times$}}
\put(932,490){\makebox(0,0){$\times$}}
\put(932,490){\makebox(0,0){$\times$}}
\put(933,491){\makebox(0,0){$\times$}}
\put(933,493){\makebox(0,0){$\times$}}
\put(934,494){\makebox(0,0){$\times$}}
\put(934,495){\makebox(0,0){$\times$}}
\put(935,496){\makebox(0,0){$\times$}}
\put(935,497){\makebox(0,0){$\times$}}
\put(936,499){\makebox(0,0){$\times$}}
\put(936,500){\makebox(0,0){$\times$}}
\put(937,501){\makebox(0,0){$\times$}}
\put(937,502){\makebox(0,0){$\times$}}
\put(938,504){\makebox(0,0){$\times$}}
\put(938,505){\makebox(0,0){$\times$}}
\put(939,506){\makebox(0,0){$\times$}}
\put(939,507){\makebox(0,0){$\times$}}
\put(940,508){\makebox(0,0){$\times$}}
\put(940,510){\makebox(0,0){$\times$}}
\put(941,511){\makebox(0,0){$\times$}}
\put(941,512){\makebox(0,0){$\times$}}
\put(942,513){\makebox(0,0){$\times$}}
\put(942,515){\makebox(0,0){$\times$}}
\put(943,516){\makebox(0,0){$\times$}}
\put(943,517){\makebox(0,0){$\times$}}
\put(944,519){\makebox(0,0){$\times$}}
\put(944,520){\makebox(0,0){$\times$}}
\put(945,521){\makebox(0,0){$\times$}}
\put(945,522){\makebox(0,0){$\times$}}
\put(946,524){\makebox(0,0){$\times$}}
\put(946,525){\makebox(0,0){$\times$}}
\put(947,527){\makebox(0,0){$\times$}}
\put(947,528){\makebox(0,0){$\times$}}
\put(948,529){\makebox(0,0){$\times$}}
\put(949,530){\makebox(0,0){$\times$}}
\put(949,532){\makebox(0,0){$\times$}}
\put(950,533){\makebox(0,0){$\times$}}
\put(950,534){\makebox(0,0){$\times$}}
\put(951,536){\makebox(0,0){$\times$}}
\put(951,537){\makebox(0,0){$\times$}}
\put(952,538){\makebox(0,0){$\times$}}
\put(952,539){\makebox(0,0){$\times$}}
\put(953,541){\makebox(0,0){$\times$}}
\put(955,547){\makebox(0,0){$\times$}}
\put(956,548){\makebox(0,0){$\times$}}
\put(956,550){\makebox(0,0){$\times$}}
\put(957,551){\makebox(0,0){$\times$}}
\put(957,553){\makebox(0,0){$\times$}}
\put(958,554){\makebox(0,0){$\times$}}
\put(958,555){\makebox(0,0){$\times$}}
\put(959,556){\makebox(0,0){$\times$}}
\put(959,557){\makebox(0,0){$\times$}}
\put(960,559){\makebox(0,0){$\times$}}
\put(960,560){\makebox(0,0){$\times$}}
\put(961,561){\makebox(0,0){$\times$}}
\put(961,562){\makebox(0,0){$\times$}}
\put(962,563){\makebox(0,0){$\times$}}
\put(962,565){\makebox(0,0){$\times$}}
\put(963,566){\makebox(0,0){$\times$}}
\put(963,567){\makebox(0,0){$\times$}}
\put(964,568){\makebox(0,0){$\times$}}
\put(965,569){\makebox(0,0){$\times$}}
\put(965,571){\makebox(0,0){$\times$}}
\put(966,572){\makebox(0,0){$\times$}}
\put(966,573){\makebox(0,0){$\times$}}
\put(967,574){\makebox(0,0){$\times$}}
\put(967,576){\makebox(0,0){$\times$}}
\put(968,576){\makebox(0,0){$\times$}}
\put(968,577){\makebox(0,0){$\times$}}
\put(969,579){\makebox(0,0){$\times$}}
\put(969,580){\makebox(0,0){$\times$}}
\put(970,581){\makebox(0,0){$\times$}}
\put(970,582){\makebox(0,0){$\times$}}
\put(971,583){\makebox(0,0){$\times$}}
\put(971,584){\makebox(0,0){$\times$}}
\put(972,585){\makebox(0,0){$\times$}}
\put(972,586){\makebox(0,0){$\times$}}
\put(973,587){\makebox(0,0){$\times$}}
\put(973,588){\makebox(0,0){$\times$}}
\put(974,589){\makebox(0,0){$\times$}}
\put(974,589){\makebox(0,0){$\times$}}
\put(975,591){\makebox(0,0){$\times$}}
\put(975,591){\makebox(0,0){$\times$}}
\put(976,592){\makebox(0,0){$\times$}}
\put(976,593){\makebox(0,0){$\times$}}
\put(977,594){\makebox(0,0){$\times$}}
\put(977,595){\makebox(0,0){$\times$}}
\put(978,596){\makebox(0,0){$\times$}}
\put(978,597){\makebox(0,0){$\times$}}
\put(979,597){\makebox(0,0){$\times$}}
\put(979,598){\makebox(0,0){$\times$}}
\put(980,599){\makebox(0,0){$\times$}}
\put(980,600){\makebox(0,0){$\times$}}
\put(981,600){\makebox(0,0){$\times$}}
\put(982,601){\makebox(0,0){$\times$}}
\put(982,602){\makebox(0,0){$\times$}}
\put(983,602){\makebox(0,0){$\times$}}
\put(983,603){\makebox(0,0){$\times$}}
\put(984,604){\makebox(0,0){$\times$}}
\put(984,605){\makebox(0,0){$\times$}}
\put(985,605){\makebox(0,0){$\times$}}
\put(985,606){\makebox(0,0){$\times$}}
\put(986,606){\makebox(0,0){$\times$}}
\put(986,606){\makebox(0,0){$\times$}}
\put(987,607){\makebox(0,0){$\times$}}
\put(987,608){\makebox(0,0){$\times$}}
\put(988,608){\makebox(0,0){$\times$}}
\put(988,609){\makebox(0,0){$\times$}}
\put(989,609){\makebox(0,0){$\times$}}
\put(989,609){\makebox(0,0){$\times$}}
\put(990,609){\makebox(0,0){$\times$}}
\put(990,610){\makebox(0,0){$\times$}}
\put(991,611){\makebox(0,0){$\times$}}
\put(991,611){\makebox(0,0){$\times$}}
\put(992,611){\makebox(0,0){$\times$}}
\put(992,611){\makebox(0,0){$\times$}}
\put(993,611){\makebox(0,0){$\times$}}
\put(993,612){\makebox(0,0){$\times$}}
\put(994,612){\makebox(0,0){$\times$}}
\put(994,612){\makebox(0,0){$\times$}}
\put(995,612){\makebox(0,0){$\times$}}
\put(995,612){\makebox(0,0){$\times$}}
\put(996,612){\makebox(0,0){$\times$}}
\put(996,612){\makebox(0,0){$\times$}}
\put(997,613){\makebox(0,0){$\times$}}
\put(997,613){\makebox(0,0){$\times$}}
\put(998,613){\makebox(0,0){$\times$}}
\put(999,613){\makebox(0,0){$\times$}}
\put(999,613){\makebox(0,0){$\times$}}
\put(1000,613){\makebox(0,0){$\times$}}
\put(1000,613){\makebox(0,0){$\times$}}
\put(1001,613){\makebox(0,0){$\times$}}
\put(1001,613){\makebox(0,0){$\times$}}
\put(1002,613){\makebox(0,0){$\times$}}
\put(1002,613){\makebox(0,0){$\times$}}
\put(1003,612){\makebox(0,0){$\times$}}
\put(1003,612){\makebox(0,0){$\times$}}
\put(1004,612){\makebox(0,0){$\times$}}
\put(1004,612){\makebox(0,0){$\times$}}
\put(1005,612){\makebox(0,0){$\times$}}
\put(1005,612){\makebox(0,0){$\times$}}
\put(1006,611){\makebox(0,0){$\times$}}
\put(1006,611){\makebox(0,0){$\times$}}
\put(1007,611){\makebox(0,0){$\times$}}
\put(1007,611){\makebox(0,0){$\times$}}
\put(1008,610){\makebox(0,0){$\times$}}
\put(1008,610){\makebox(0,0){$\times$}}
\put(1009,609){\makebox(0,0){$\times$}}
\put(1009,609){\makebox(0,0){$\times$}}
\put(1010,609){\makebox(0,0){$\times$}}
\put(1010,608){\makebox(0,0){$\times$}}
\put(1011,608){\makebox(0,0){$\times$}}
\put(1011,608){\makebox(0,0){$\times$}}
\put(1012,607){\makebox(0,0){$\times$}}
\put(1012,606){\makebox(0,0){$\times$}}
\put(1013,606){\makebox(0,0){$\times$}}
\put(1013,606){\makebox(0,0){$\times$}}
\put(1014,605){\makebox(0,0){$\times$}}
\put(1014,605){\makebox(0,0){$\times$}}
\put(1015,604){\makebox(0,0){$\times$}}
\put(1016,603){\makebox(0,0){$\times$}}
\put(1016,603){\makebox(0,0){$\times$}}
\put(1017,602){\makebox(0,0){$\times$}}
\put(1017,602){\makebox(0,0){$\times$}}
\put(1018,601){\makebox(0,0){$\times$}}
\put(1018,600){\makebox(0,0){$\times$}}
\put(1019,600){\makebox(0,0){$\times$}}
\put(1019,599){\makebox(0,0){$\times$}}
\put(1020,598){\makebox(0,0){$\times$}}
\put(1020,598){\makebox(0,0){$\times$}}
\put(1021,597){\makebox(0,0){$\times$}}
\put(1021,597){\makebox(0,0){$\times$}}
\put(1022,595){\makebox(0,0){$\times$}}
\put(1022,595){\makebox(0,0){$\times$}}
\put(1023,594){\makebox(0,0){$\times$}}
\put(1023,594){\makebox(0,0){$\times$}}
\put(1024,592){\makebox(0,0){$\times$}}
\put(1024,592){\makebox(0,0){$\times$}}
\put(1025,591){\makebox(0,0){$\times$}}
\put(1025,591){\makebox(0,0){$\times$}}
\put(1026,589){\makebox(0,0){$\times$}}
\put(1026,589){\makebox(0,0){$\times$}}
\put(1027,588){\makebox(0,0){$\times$}}
\put(1027,587){\makebox(0,0){$\times$}}
\put(1028,586){\makebox(0,0){$\times$}}
\put(1028,585){\makebox(0,0){$\times$}}
\put(1029,585){\makebox(0,0){$\times$}}
\put(1029,584){\makebox(0,0){$\times$}}
\put(1030,583){\makebox(0,0){$\times$}}
\put(1030,582){\makebox(0,0){$\times$}}
\put(1031,581){\makebox(0,0){$\times$}}
\put(1031,580){\makebox(0,0){$\times$}}
\put(1032,580){\makebox(0,0){$\times$}}
\put(1033,579){\makebox(0,0){$\times$}}
\put(1033,577){\makebox(0,0){$\times$}}
\put(1034,577){\makebox(0,0){$\times$}}
\put(1034,576){\makebox(0,0){$\times$}}
\put(1035,575){\makebox(0,0){$\times$}}
\put(1035,574){\makebox(0,0){$\times$}}
\put(1036,573){\makebox(0,0){$\times$}}
\put(1036,572){\makebox(0,0){$\times$}}
\put(1037,571){\makebox(0,0){$\times$}}
\put(1037,570){\makebox(0,0){$\times$}}
\put(1038,569){\makebox(0,0){$\times$}}
\put(1038,568){\makebox(0,0){$\times$}}
\put(1039,568){\makebox(0,0){$\times$}}
\put(1039,566){\makebox(0,0){$\times$}}
\put(1040,565){\makebox(0,0){$\times$}}
\put(1040,565){\makebox(0,0){$\times$}}
\put(1041,563){\makebox(0,0){$\times$}}
\put(1041,563){\makebox(0,0){$\times$}}
\put(1042,562){\makebox(0,0){$\times$}}
\put(1042,560){\makebox(0,0){$\times$}}
\put(1043,560){\makebox(0,0){$\times$}}
\put(1043,559){\makebox(0,0){$\times$}}
\put(1044,558){\makebox(0,0){$\times$}}
\put(1044,557){\makebox(0,0){$\times$}}
\put(1045,556){\makebox(0,0){$\times$}}
\put(1045,555){\makebox(0,0){$\times$}}
\put(1046,554){\makebox(0,0){$\times$}}
\put(1046,553){\makebox(0,0){$\times$}}
\put(1047,552){\makebox(0,0){$\times$}}
\put(1047,551){\makebox(0,0){$\times$}}
\put(1048,550){\makebox(0,0){$\times$}}
\put(1049,548){\makebox(0,0){$\times$}}
\put(1050,547){\makebox(0,0){$\times$}}
\put(1050,546){\makebox(0,0){$\times$}}
\put(1052,543){\makebox(0,0){$\times$}}
\put(1053,542){\makebox(0,0){$\times$}}
\put(1053,541){\makebox(0,0){$\times$}}
\put(1054,540){\makebox(0,0){$\times$}}
\put(1054,539){\makebox(0,0){$\times$}}
\put(1055,538){\makebox(0,0){$\times$}}
\put(1055,537){\makebox(0,0){$\times$}}
\put(1056,536){\makebox(0,0){$\times$}}
\put(1056,536){\makebox(0,0){$\times$}}
\put(1057,534){\makebox(0,0){$\times$}}
\put(1057,534){\makebox(0,0){$\times$}}
\put(1058,533){\makebox(0,0){$\times$}}
\put(1058,532){\makebox(0,0){$\times$}}
\put(1059,531){\makebox(0,0){$\times$}}
\put(1059,530){\makebox(0,0){$\times$}}
\put(1060,530){\makebox(0,0){$\times$}}
\put(1060,529){\makebox(0,0){$\times$}}
\put(1061,528){\makebox(0,0){$\times$}}
\put(1061,527){\makebox(0,0){$\times$}}
\put(1062,527){\makebox(0,0){$\times$}}
\put(1062,525){\makebox(0,0){$\times$}}
\put(1063,525){\makebox(0,0){$\times$}}
\put(1063,524){\makebox(0,0){$\times$}}
\put(1064,523){\makebox(0,0){$\times$}}
\put(1064,522){\makebox(0,0){$\times$}}
\put(1065,522){\makebox(0,0){$\times$}}
\put(1065,521){\makebox(0,0){$\times$}}
\put(1066,520){\makebox(0,0){$\times$}}
\put(1067,519){\makebox(0,0){$\times$}}
\put(1067,519){\makebox(0,0){$\times$}}
\put(1068,518){\makebox(0,0){$\times$}}
\put(1068,517){\makebox(0,0){$\times$}}
\put(1069,517){\makebox(0,0){$\times$}}
\put(1069,516){\makebox(0,0){$\times$}}
\put(1070,516){\makebox(0,0){$\times$}}
\put(1070,515){\makebox(0,0){$\times$}}
\put(1071,514){\makebox(0,0){$\times$}}
\put(1071,514){\makebox(0,0){$\times$}}
\put(1072,513){\makebox(0,0){$\times$}}
\put(1072,513){\makebox(0,0){$\times$}}
\put(1073,512){\makebox(0,0){$\times$}}
\put(1073,511){\makebox(0,0){$\times$}}
\put(1074,511){\makebox(0,0){$\times$}}
\put(1074,510){\makebox(0,0){$\times$}}
\put(1075,510){\makebox(0,0){$\times$}}
\put(1075,509){\makebox(0,0){$\times$}}
\put(1076,509){\makebox(0,0){$\times$}}
\put(1076,508){\makebox(0,0){$\times$}}
\put(1077,508){\makebox(0,0){$\times$}}
\put(1077,507){\makebox(0,0){$\times$}}
\put(1078,507){\makebox(0,0){$\times$}}
\put(1078,507){\makebox(0,0){$\times$}}
\put(1079,506){\makebox(0,0){$\times$}}
\put(1079,505){\makebox(0,0){$\times$}}
\put(1080,505){\makebox(0,0){$\times$}}
\put(1080,505){\makebox(0,0){$\times$}}
\put(1081,505){\makebox(0,0){$\times$}}
\put(1081,504){\makebox(0,0){$\times$}}
\put(1082,504){\makebox(0,0){$\times$}}
\put(1082,504){\makebox(0,0){$\times$}}
\put(1083,504){\makebox(0,0){$\times$}}
\put(1084,503){\makebox(0,0){$\times$}}
\put(1084,503){\makebox(0,0){$\times$}}
\put(1085,502){\makebox(0,0){$\times$}}
\put(1085,502){\makebox(0,0){$\times$}}
\put(1086,502){\makebox(0,0){$\times$}}
\put(1086,502){\makebox(0,0){$\times$}}
\put(1087,502){\makebox(0,0){$\times$}}
\put(1087,502){\makebox(0,0){$\times$}}
\put(1088,502){\makebox(0,0){$\times$}}
\put(1088,501){\makebox(0,0){$\times$}}
\put(1089,501){\makebox(0,0){$\times$}}
\put(1089,501){\makebox(0,0){$\times$}}
\put(1090,501){\makebox(0,0){$\times$}}
\put(1090,501){\makebox(0,0){$\times$}}
\put(1091,501){\makebox(0,0){$\times$}}
\put(1091,501){\makebox(0,0){$\times$}}
\put(1092,501){\makebox(0,0){$\times$}}
\put(1092,501){\makebox(0,0){$\times$}}
\put(1093,501){\makebox(0,0){$\times$}}
\put(1093,501){\makebox(0,0){$\times$}}
\put(1094,501){\makebox(0,0){$\times$}}
\put(1094,501){\makebox(0,0){$\times$}}
\put(1095,501){\makebox(0,0){$\times$}}
\put(1095,501){\makebox(0,0){$\times$}}
\put(1096,501){\makebox(0,0){$\times$}}
\put(1096,501){\makebox(0,0){$\times$}}
\put(1097,501){\makebox(0,0){$\times$}}
\put(1097,501){\makebox(0,0){$\times$}}
\put(1098,501){\makebox(0,0){$\times$}}
\put(1098,501){\makebox(0,0){$\times$}}
\put(1099,501){\makebox(0,0){$\times$}}
\put(1099,502){\makebox(0,0){$\times$}}
\put(1100,502){\makebox(0,0){$\times$}}
\put(1101,502){\makebox(0,0){$\times$}}
\put(1101,502){\makebox(0,0){$\times$}}
\put(1102,502){\makebox(0,0){$\times$}}
\put(1102,502){\makebox(0,0){$\times$}}
\put(1103,503){\makebox(0,0){$\times$}}
\put(1103,503){\makebox(0,0){$\times$}}
\put(1104,504){\makebox(0,0){$\times$}}
\put(1104,504){\makebox(0,0){$\times$}}
\put(1105,504){\makebox(0,0){$\times$}}
\put(1105,504){\makebox(0,0){$\times$}}
\put(1106,505){\makebox(0,0){$\times$}}
\put(1106,505){\makebox(0,0){$\times$}}
\put(1107,505){\makebox(0,0){$\times$}}
\put(1107,505){\makebox(0,0){$\times$}}
\put(1108,506){\makebox(0,0){$\times$}}
\put(1108,507){\makebox(0,0){$\times$}}
\put(1109,507){\makebox(0,0){$\times$}}
\put(1109,507){\makebox(0,0){$\times$}}
\put(1110,507){\makebox(0,0){$\times$}}
\put(1110,508){\makebox(0,0){$\times$}}
\put(1111,508){\makebox(0,0){$\times$}}
\put(1111,508){\makebox(0,0){$\times$}}
\put(1112,509){\makebox(0,0){$\times$}}
\put(1112,510){\makebox(0,0){$\times$}}
\put(1113,510){\makebox(0,0){$\times$}}
\put(1113,510){\makebox(0,0){$\times$}}
\put(1114,511){\makebox(0,0){$\times$}}
\put(1114,511){\makebox(0,0){$\times$}}
\put(1115,512){\makebox(0,0){$\times$}}
\put(1115,512){\makebox(0,0){$\times$}}
\put(1116,513){\makebox(0,0){$\times$}}
\put(1116,513){\makebox(0,0){$\times$}}
\put(1117,514){\makebox(0,0){$\times$}}
\put(1118,514){\makebox(0,0){$\times$}}
\put(1118,514){\makebox(0,0){$\times$}}
\put(1119,515){\makebox(0,0){$\times$}}
\put(1119,516){\makebox(0,0){$\times$}}
\put(1120,516){\makebox(0,0){$\times$}}
\put(1120,517){\makebox(0,0){$\times$}}
\put(1121,517){\makebox(0,0){$\times$}}
\put(1121,517){\makebox(0,0){$\times$}}
\put(1122,518){\makebox(0,0){$\times$}}
\put(1122,519){\makebox(0,0){$\times$}}
\put(1123,519){\makebox(0,0){$\times$}}
\put(1123,520){\makebox(0,0){$\times$}}
\put(1124,520){\makebox(0,0){$\times$}}
\put(1124,521){\makebox(0,0){$\times$}}
\put(1125,522){\makebox(0,0){$\times$}}
\put(1125,522){\makebox(0,0){$\times$}}
\put(1126,522){\makebox(0,0){$\times$}}
\put(1126,523){\makebox(0,0){$\times$}}
\put(1127,523){\makebox(0,0){$\times$}}
\put(1127,524){\makebox(0,0){$\times$}}
\put(1128,525){\makebox(0,0){$\times$}}
\put(1128,525){\makebox(0,0){$\times$}}
\put(1129,526){\makebox(0,0){$\times$}}
\put(1129,527){\makebox(0,0){$\times$}}
\put(1130,527){\makebox(0,0){$\times$}}
\put(1130,528){\makebox(0,0){$\times$}}
\put(1131,528){\makebox(0,0){$\times$}}
\put(1131,529){\makebox(0,0){$\times$}}
\put(1132,530){\makebox(0,0){$\times$}}
\put(1132,530){\makebox(0,0){$\times$}}
\put(1133,531){\makebox(0,0){$\times$}}
\put(1133,531){\makebox(0,0){$\times$}}
\put(1134,532){\makebox(0,0){$\times$}}
\put(1135,533){\makebox(0,0){$\times$}}
\put(1135,533){\makebox(0,0){$\times$}}
\put(1136,534){\makebox(0,0){$\times$}}
\put(1136,534){\makebox(0,0){$\times$}}
\put(1137,535){\makebox(0,0){$\times$}}
\put(1137,536){\makebox(0,0){$\times$}}
\put(1138,536){\makebox(0,0){$\times$}}
\put(1138,537){\makebox(0,0){$\times$}}
\put(1139,537){\makebox(0,0){$\times$}}
\put(1139,538){\makebox(0,0){$\times$}}
\put(1140,539){\makebox(0,0){$\times$}}
\put(1140,539){\makebox(0,0){$\times$}}
\put(1141,539){\makebox(0,0){$\times$}}
\put(1141,540){\makebox(0,0){$\times$}}
\put(1142,540){\makebox(0,0){$\times$}}
\put(1142,541){\makebox(0,0){$\times$}}
\put(1143,542){\makebox(0,0){$\times$}}
\put(1144,543){\makebox(0,0){$\times$}}
\put(1147,546){\makebox(0,0){$\times$}}
\put(1147,547){\makebox(0,0){$\times$}}
\put(1148,548){\makebox(0,0){$\times$}}
\put(1148,548){\makebox(0,0){$\times$}}
\put(1149,548){\makebox(0,0){$\times$}}
\put(1150,549){\makebox(0,0){$\times$}}
\put(1150,550){\makebox(0,0){$\times$}}
\put(1151,551){\makebox(0,0){$\times$}}
\put(1152,551){\makebox(0,0){$\times$}}
\put(1152,552){\makebox(0,0){$\times$}}
\put(1153,553){\makebox(0,0){$\times$}}
\put(1153,553){\makebox(0,0){$\times$}}
\put(1154,553){\makebox(0,0){$\times$}}
\put(1154,554){\makebox(0,0){$\times$}}
\put(1155,554){\makebox(0,0){$\times$}}
\put(1155,555){\makebox(0,0){$\times$}}
\put(1156,556){\makebox(0,0){$\times$}}
\put(1156,556){\makebox(0,0){$\times$}}
\put(1157,556){\makebox(0,0){$\times$}}
\put(1157,557){\makebox(0,0){$\times$}}
\put(1158,557){\makebox(0,0){$\times$}}
\put(1158,558){\makebox(0,0){$\times$}}
\put(1159,558){\makebox(0,0){$\times$}}
\put(1159,559){\makebox(0,0){$\times$}}
\put(1160,559){\makebox(0,0){$\times$}}
\put(1160,559){\makebox(0,0){$\times$}}
\put(1161,560){\makebox(0,0){$\times$}}
\put(1161,560){\makebox(0,0){$\times$}}
\put(1162,560){\makebox(0,0){$\times$}}
\put(1162,561){\makebox(0,0){$\times$}}
\put(1163,562){\makebox(0,0){$\times$}}
\put(1163,562){\makebox(0,0){$\times$}}
\put(1164,562){\makebox(0,0){$\times$}}
\put(1164,563){\makebox(0,0){$\times$}}
\put(1165,563){\makebox(0,0){$\times$}}
\put(1165,563){\makebox(0,0){$\times$}}
\put(1166,563){\makebox(0,0){$\times$}}
\put(1166,564){\makebox(0,0){$\times$}}
\put(1167,565){\makebox(0,0){$\times$}}
\put(1167,565){\makebox(0,0){$\times$}}
\put(1168,565){\makebox(0,0){$\times$}}
\put(1169,565){\makebox(0,0){$\times$}}
\put(1169,566){\makebox(0,0){$\times$}}
\put(1170,566){\makebox(0,0){$\times$}}
\put(1170,566){\makebox(0,0){$\times$}}
\put(1171,566){\makebox(0,0){$\times$}}
\put(1171,566){\makebox(0,0){$\times$}}
\put(1172,567){\makebox(0,0){$\times$}}
\put(1172,567){\makebox(0,0){$\times$}}
\put(1173,568){\makebox(0,0){$\times$}}
\put(1173,568){\makebox(0,0){$\times$}}
\put(1174,568){\makebox(0,0){$\times$}}
\put(1174,568){\makebox(0,0){$\times$}}
\put(1175,568){\makebox(0,0){$\times$}}
\put(1175,568){\makebox(0,0){$\times$}}
\put(1176,569){\makebox(0,0){$\times$}}
\put(1176,569){\makebox(0,0){$\times$}}
\put(1177,569){\makebox(0,0){$\times$}}
\put(1177,569){\makebox(0,0){$\times$}}
\put(1178,569){\makebox(0,0){$\times$}}
\put(1178,569){\makebox(0,0){$\times$}}
\put(1179,570){\makebox(0,0){$\times$}}
\put(1179,570){\makebox(0,0){$\times$}}
\put(1180,570){\makebox(0,0){$\times$}}
\put(1180,570){\makebox(0,0){$\times$}}
\put(1181,570){\makebox(0,0){$\times$}}
\put(1181,570){\makebox(0,0){$\times$}}
\put(1182,570){\makebox(0,0){$\times$}}
\put(1182,571){\makebox(0,0){$\times$}}
\put(1183,571){\makebox(0,0){$\times$}}
\put(1183,571){\makebox(0,0){$\times$}}
\put(1184,571){\makebox(0,0){$\times$}}
\put(1184,571){\makebox(0,0){$\times$}}
\put(1185,571){\makebox(0,0){$\times$}}
\put(1186,571){\makebox(0,0){$\times$}}
\put(1186,571){\makebox(0,0){$\times$}}
\put(1187,571){\makebox(0,0){$\times$}}
\put(1187,571){\makebox(0,0){$\times$}}
\put(1188,571){\makebox(0,0){$\times$}}
\put(1188,571){\makebox(0,0){$\times$}}
\put(1189,571){\makebox(0,0){$\times$}}
\put(1189,571){\makebox(0,0){$\times$}}
\put(1190,571){\makebox(0,0){$\times$}}
\put(1190,571){\makebox(0,0){$\times$}}
\put(1191,571){\makebox(0,0){$\times$}}
\put(1191,571){\makebox(0,0){$\times$}}
\put(1192,571){\makebox(0,0){$\times$}}
\put(1192,571){\makebox(0,0){$\times$}}
\put(1193,571){\makebox(0,0){$\times$}}
\put(1193,571){\makebox(0,0){$\times$}}
\put(1194,570){\makebox(0,0){$\times$}}
\put(1194,570){\makebox(0,0){$\times$}}
\put(1195,570){\makebox(0,0){$\times$}}
\put(1195,570){\makebox(0,0){$\times$}}
\put(1196,570){\makebox(0,0){$\times$}}
\put(1196,569){\makebox(0,0){$\times$}}
\put(1197,569){\makebox(0,0){$\times$}}
\put(1197,569){\makebox(0,0){$\times$}}
\put(1198,569){\makebox(0,0){$\times$}}
\put(1198,569){\makebox(0,0){$\times$}}
\put(1199,569){\makebox(0,0){$\times$}}
\put(1199,569){\makebox(0,0){$\times$}}
\put(1200,568){\makebox(0,0){$\times$}}
\put(1200,568){\makebox(0,0){$\times$}}
\put(1201,568){\makebox(0,0){$\times$}}
\put(1201,568){\makebox(0,0){$\times$}}
\put(1202,568){\makebox(0,0){$\times$}}
\put(1203,568){\makebox(0,0){$\times$}}
\put(1203,567){\makebox(0,0){$\times$}}
\put(1204,567){\makebox(0,0){$\times$}}
\put(1204,567){\makebox(0,0){$\times$}}
\put(1205,566){\makebox(0,0){$\times$}}
\put(1205,566){\makebox(0,0){$\times$}}
\put(1206,566){\makebox(0,0){$\times$}}
\put(1206,566){\makebox(0,0){$\times$}}
\put(1207,566){\makebox(0,0){$\times$}}
\put(1207,565){\makebox(0,0){$\times$}}
\put(1208,565){\makebox(0,0){$\times$}}
\put(1208,565){\makebox(0,0){$\times$}}
\put(1209,565){\makebox(0,0){$\times$}}
\put(1209,564){\makebox(0,0){$\times$}}
\put(1210,564){\makebox(0,0){$\times$}}
\put(1210,563){\makebox(0,0){$\times$}}
\put(1211,563){\makebox(0,0){$\times$}}
\put(1211,563){\makebox(0,0){$\times$}}
\put(1212,563){\makebox(0,0){$\times$}}
\put(1212,562){\makebox(0,0){$\times$}}
\put(1213,562){\makebox(0,0){$\times$}}
\put(1213,562){\makebox(0,0){$\times$}}
\put(1214,562){\makebox(0,0){$\times$}}
\put(1214,561){\makebox(0,0){$\times$}}
\put(1215,561){\makebox(0,0){$\times$}}
\put(1215,560){\makebox(0,0){$\times$}}
\put(1216,560){\makebox(0,0){$\times$}}
\put(1216,560){\makebox(0,0){$\times$}}
\put(1217,560){\makebox(0,0){$\times$}}
\put(1217,559){\makebox(0,0){$\times$}}
\put(1218,559){\makebox(0,0){$\times$}}
\put(1218,559){\makebox(0,0){$\times$}}
\put(1219,558){\makebox(0,0){$\times$}}
\put(1220,558){\makebox(0,0){$\times$}}
\put(1220,557){\makebox(0,0){$\times$}}
\put(1221,557){\makebox(0,0){$\times$}}
\put(1221,557){\makebox(0,0){$\times$}}
\put(1222,556){\makebox(0,0){$\times$}}
\put(1222,556){\makebox(0,0){$\times$}}
\put(1223,556){\makebox(0,0){$\times$}}
\put(1223,556){\makebox(0,0){$\times$}}
\put(1224,555){\makebox(0,0){$\times$}}
\put(1224,555){\makebox(0,0){$\times$}}
\put(1225,554){\makebox(0,0){$\times$}}
\put(1225,554){\makebox(0,0){$\times$}}
\put(1226,554){\makebox(0,0){$\times$}}
\put(1226,553){\makebox(0,0){$\times$}}
\put(1227,553){\makebox(0,0){$\times$}}
\put(1227,553){\makebox(0,0){$\times$}}
\put(1228,552){\makebox(0,0){$\times$}}
\put(1228,552){\makebox(0,0){$\times$}}
\put(1229,551){\makebox(0,0){$\times$}}
\put(1229,551){\makebox(0,0){$\times$}}
\put(1230,551){\makebox(0,0){$\times$}}
\put(1230,550){\makebox(0,0){$\times$}}
\put(1231,550){\makebox(0,0){$\times$}}
\put(1231,549){\makebox(0,0){$\times$}}
\put(1234,547){\makebox(0,0){$\times$}}
\put(1235,547){\makebox(0,0){$\times$}}
\put(1235,546){\makebox(0,0){$\times$}}
\put(1236,546){\makebox(0,0){$\times$}}
\put(1241,543){\makebox(0,0){$\times$}}
\put(1241,543){\makebox(0,0){$\times$}}
\put(1243,542){\makebox(0,0){$\times$}}
\put(1243,542){\makebox(0,0){$\times$}}
\put(1244,541){\makebox(0,0){$\times$}}
\put(1244,541){\makebox(0,0){$\times$}}
\put(1245,540){\makebox(0,0){$\times$}}
\put(1245,540){\makebox(0,0){$\times$}}
\put(1246,540){\makebox(0,0){$\times$}}
\put(1246,539){\makebox(0,0){$\times$}}
\put(1247,539){\makebox(0,0){$\times$}}
\put(1247,539){\makebox(0,0){$\times$}}
\put(1248,539){\makebox(0,0){$\times$}}
\put(1248,538){\makebox(0,0){$\times$}}
\put(1249,538){\makebox(0,0){$\times$}}
\put(1249,537){\makebox(0,0){$\times$}}
\put(1250,537){\makebox(0,0){$\times$}}
\put(1250,537){\makebox(0,0){$\times$}}
\put(1251,537){\makebox(0,0){$\times$}}
\put(1251,536){\makebox(0,0){$\times$}}
\put(1252,536){\makebox(0,0){$\times$}}
\put(1252,536){\makebox(0,0){$\times$}}
\put(1253,536){\makebox(0,0){$\times$}}
\put(1254,536){\makebox(0,0){$\times$}}
\put(1254,535){\makebox(0,0){$\times$}}
\put(1255,535){\makebox(0,0){$\times$}}
\put(1255,534){\makebox(0,0){$\times$}}
\put(1256,534){\makebox(0,0){$\times$}}
\put(1256,534){\makebox(0,0){$\times$}}
\put(1257,534){\makebox(0,0){$\times$}}
\put(1257,534){\makebox(0,0){$\times$}}
\put(1258,534){\makebox(0,0){$\times$}}
\put(1258,533){\makebox(0,0){$\times$}}
\put(1259,533){\makebox(0,0){$\times$}}
\put(1259,533){\makebox(0,0){$\times$}}
\put(1260,533){\makebox(0,0){$\times$}}
\put(1260,533){\makebox(0,0){$\times$}}
\put(1261,532){\makebox(0,0){$\times$}}
\put(1261,532){\makebox(0,0){$\times$}}
\put(1262,532){\makebox(0,0){$\times$}}
\put(1262,532){\makebox(0,0){$\times$}}
\put(1263,531){\makebox(0,0){$\times$}}
\put(1263,531){\makebox(0,0){$\times$}}
\put(1264,531){\makebox(0,0){$\times$}}
\put(1264,531){\makebox(0,0){$\times$}}
\put(1265,531){\makebox(0,0){$\times$}}
\put(1265,531){\makebox(0,0){$\times$}}
\put(1266,531){\makebox(0,0){$\times$}}
\put(1266,531){\makebox(0,0){$\times$}}
\put(1267,531){\makebox(0,0){$\times$}}
\put(1267,531){\makebox(0,0){$\times$}}
\put(1268,530){\makebox(0,0){$\times$}}
\put(1268,530){\makebox(0,0){$\times$}}
\put(1269,530){\makebox(0,0){$\times$}}
\put(1269,530){\makebox(0,0){$\times$}}
\put(1270,530){\makebox(0,0){$\times$}}
\put(1271,530){\makebox(0,0){$\times$}}
\put(1271,530){\makebox(0,0){$\times$}}
\put(1272,530){\makebox(0,0){$\times$}}
\put(1272,530){\makebox(0,0){$\times$}}
\put(1273,530){\makebox(0,0){$\times$}}
\put(1273,530){\makebox(0,0){$\times$}}
\put(1274,530){\makebox(0,0){$\times$}}
\put(1274,530){\makebox(0,0){$\times$}}
\put(1275,530){\makebox(0,0){$\times$}}
\put(1275,530){\makebox(0,0){$\times$}}
\put(1276,530){\makebox(0,0){$\times$}}
\put(1276,530){\makebox(0,0){$\times$}}
\put(1277,530){\makebox(0,0){$\times$}}
\put(1277,530){\makebox(0,0){$\times$}}
\put(1278,530){\makebox(0,0){$\times$}}
\put(1278,530){\makebox(0,0){$\times$}}
\put(1279,530){\makebox(0,0){$\times$}}
\put(1279,530){\makebox(0,0){$\times$}}
\put(1280,530){\makebox(0,0){$\times$}}
\put(1280,530){\makebox(0,0){$\times$}}
\put(1281,530){\makebox(0,0){$\times$}}
\put(1281,530){\makebox(0,0){$\times$}}
\put(1282,530){\makebox(0,0){$\times$}}
\put(1282,530){\makebox(0,0){$\times$}}
\put(1283,530){\makebox(0,0){$\times$}}
\put(1283,530){\makebox(0,0){$\times$}}
\put(1284,530){\makebox(0,0){$\times$}}
\put(1284,530){\makebox(0,0){$\times$}}
\put(1285,530){\makebox(0,0){$\times$}}
\put(1285,530){\makebox(0,0){$\times$}}
\put(1286,530){\makebox(0,0){$\times$}}
\put(1287,530){\makebox(0,0){$\times$}}
\put(1287,530){\makebox(0,0){$\times$}}
\put(1288,530){\makebox(0,0){$\times$}}
\put(1288,530){\makebox(0,0){$\times$}}
\put(1289,530){\makebox(0,0){$\times$}}
\put(1289,531){\makebox(0,0){$\times$}}
\put(1290,531){\makebox(0,0){$\times$}}
\put(1290,531){\makebox(0,0){$\times$}}
\put(1291,531){\makebox(0,0){$\times$}}
\put(1291,531){\makebox(0,0){$\times$}}
\put(1292,531){\makebox(0,0){$\times$}}
\put(1292,531){\makebox(0,0){$\times$}}
\put(1293,531){\makebox(0,0){$\times$}}
\put(1293,531){\makebox(0,0){$\times$}}
\put(1294,531){\makebox(0,0){$\times$}}
\put(1294,532){\makebox(0,0){$\times$}}
\put(1295,532){\makebox(0,0){$\times$}}
\put(1295,532){\makebox(0,0){$\times$}}
\put(1296,532){\makebox(0,0){$\times$}}
\put(1296,532){\makebox(0,0){$\times$}}
\put(1297,533){\makebox(0,0){$\times$}}
\put(1297,533){\makebox(0,0){$\times$}}
\put(1298,533){\makebox(0,0){$\times$}}
\put(1298,533){\makebox(0,0){$\times$}}
\put(1299,533){\makebox(0,0){$\times$}}
\put(1299,533){\makebox(0,0){$\times$}}
\put(1300,533){\makebox(0,0){$\times$}}
\put(1300,533){\makebox(0,0){$\times$}}
\put(1301,534){\makebox(0,0){$\times$}}
\put(1301,534){\makebox(0,0){$\times$}}
\put(1302,534){\makebox(0,0){$\times$}}
\put(1302,534){\makebox(0,0){$\times$}}
\put(1303,534){\makebox(0,0){$\times$}}
\put(1304,534){\makebox(0,0){$\times$}}
\put(1304,534){\makebox(0,0){$\times$}}
\put(1305,534){\makebox(0,0){$\times$}}
\put(1305,535){\makebox(0,0){$\times$}}
\put(1306,535){\makebox(0,0){$\times$}}
\put(1306,535){\makebox(0,0){$\times$}}
\put(1307,536){\makebox(0,0){$\times$}}
\put(1307,536){\makebox(0,0){$\times$}}
\put(1308,536){\makebox(0,0){$\times$}}
\put(1308,536){\makebox(0,0){$\times$}}
\put(1309,536){\makebox(0,0){$\times$}}
\put(1309,536){\makebox(0,0){$\times$}}
\put(1310,536){\makebox(0,0){$\times$}}
\put(1310,537){\makebox(0,0){$\times$}}
\put(1311,537){\makebox(0,0){$\times$}}
\put(1311,537){\makebox(0,0){$\times$}}
\put(1312,537){\makebox(0,0){$\times$}}
\put(1312,537){\makebox(0,0){$\times$}}
\put(1313,537){\makebox(0,0){$\times$}}
\put(1313,537){\makebox(0,0){$\times$}}
\put(1314,538){\makebox(0,0){$\times$}}
\put(1314,538){\makebox(0,0){$\times$}}
\put(1315,538){\makebox(0,0){$\times$}}
\put(1315,539){\makebox(0,0){$\times$}}
\put(1316,539){\makebox(0,0){$\times$}}
\put(1316,539){\makebox(0,0){$\times$}}
\put(1317,539){\makebox(0,0){$\times$}}
\put(1317,539){\makebox(0,0){$\times$}}
\put(1318,539){\makebox(0,0){$\times$}}
\put(1318,539){\makebox(0,0){$\times$}}
\put(1319,540){\makebox(0,0){$\times$}}
\put(1319,540){\makebox(0,0){$\times$}}
\put(1320,540){\makebox(0,0){$\times$}}
\put(1321,540){\makebox(0,0){$\times$}}
\put(1321,540){\makebox(0,0){$\times$}}
\put(1322,540){\makebox(0,0){$\times$}}
\put(1322,540){\makebox(0,0){$\times$}}
\put(1323,541){\makebox(0,0){$\times$}}
\put(1323,541){\makebox(0,0){$\times$}}
\put(1324,541){\makebox(0,0){$\times$}}
\put(1324,541){\makebox(0,0){$\times$}}
\put(1325,542){\makebox(0,0){$\times$}}
\put(1325,542){\makebox(0,0){$\times$}}
\put(1326,542){\makebox(0,0){$\times$}}
\put(1328,543){\makebox(0,0){$\times$}}
\put(1328,543){\makebox(0,0){$\times$}}
\put(1329,543){\makebox(0,0){$\times$}}
\put(1329,543){\makebox(0,0){$\times$}}
\put(1341,546){\makebox(0,0){$\times$}}
\put(1342,546){\makebox(0,0){$\times$}}
\put(1342,546){\makebox(0,0){$\times$}}
\put(1343,546){\makebox(0,0){$\times$}}
\put(1343,546){\makebox(0,0){$\times$}}
\put(1344,547){\makebox(0,0){$\times$}}
\put(1344,547){\makebox(0,0){$\times$}}
\put(1345,547){\makebox(0,0){$\times$}}
\put(1345,547){\makebox(0,0){$\times$}}
\put(1346,547){\makebox(0,0){$\times$}}
\put(1346,548){\makebox(0,0){$\times$}}
\put(1347,548){\makebox(0,0){$\times$}}
\put(1347,548){\makebox(0,0){$\times$}}
\put(1348,548){\makebox(0,0){$\times$}}
\put(1348,548){\makebox(0,0){$\times$}}
\put(1349,548){\makebox(0,0){$\times$}}
\put(1363,549){\makebox(0,0){$\times$}}
\put(1363,549){\makebox(0,0){$\times$}}
\put(1364,549){\makebox(0,0){$\times$}}
\put(1364,549){\makebox(0,0){$\times$}}
\put(1365,549){\makebox(0,0){$\times$}}
\put(1365,549){\makebox(0,0){$\times$}}
\put(1366,549){\makebox(0,0){$\times$}}
\put(1366,549){\makebox(0,0){$\times$}}
\put(1367,549){\makebox(0,0){$\times$}}
\put(1367,549){\makebox(0,0){$\times$}}
\put(1368,549){\makebox(0,0){$\times$}}
\put(1368,549){\makebox(0,0){$\times$}}
\put(1369,549){\makebox(0,0){$\times$}}
\put(1369,549){\makebox(0,0){$\times$}}
\put(1370,549){\makebox(0,0){$\times$}}
\put(1370,549){\makebox(0,0){$\times$}}
\put(1371,549){\makebox(0,0){$\times$}}
\put(1372,549){\makebox(0,0){$\times$}}
\put(1372,549){\makebox(0,0){$\times$}}
\put(1373,549){\makebox(0,0){$\times$}}
\put(1373,549){\makebox(0,0){$\times$}}
\put(1374,549){\makebox(0,0){$\times$}}
\put(1374,549){\makebox(0,0){$\times$}}
\put(1375,549){\makebox(0,0){$\times$}}
\put(1375,549){\makebox(0,0){$\times$}}
\put(1376,549){\makebox(0,0){$\times$}}
\put(1376,549){\makebox(0,0){$\times$}}
\put(1377,549){\makebox(0,0){$\times$}}
\put(1377,549){\makebox(0,0){$\times$}}
\put(1392,548){\makebox(0,0){$\times$}}
\put(1392,548){\makebox(0,0){$\times$}}
\put(1393,548){\makebox(0,0){$\times$}}
\put(1393,548){\makebox(0,0){$\times$}}
\put(1394,548){\makebox(0,0){$\times$}}
\put(1394,548){\makebox(0,0){$\times$}}
\put(1395,548){\makebox(0,0){$\times$}}
\put(1395,548){\makebox(0,0){$\times$}}
\put(1396,548){\makebox(0,0){$\times$}}
\put(1396,547){\makebox(0,0){$\times$}}
\put(1397,547){\makebox(0,0){$\times$}}
\put(1397,547){\makebox(0,0){$\times$}}
\put(1398,547){\makebox(0,0){$\times$}}
\put(1398,547){\makebox(0,0){$\times$}}
\put(1399,547){\makebox(0,0){$\times$}}
\put(1399,547){\makebox(0,0){$\times$}}
\put(1400,547){\makebox(0,0){$\times$}}
\put(1400,546){\makebox(0,0){$\times$}}
\put(1401,546){\makebox(0,0){$\times$}}
\put(1401,546){\makebox(0,0){$\times$}}
\put(1402,546){\makebox(0,0){$\times$}}
\put(1402,546){\makebox(0,0){$\times$}}
\put(1403,546){\makebox(0,0){$\times$}}
\put(1403,546){\makebox(0,0){$\times$}}
\put(1404,546){\makebox(0,0){$\times$}}
\put(1404,546){\makebox(0,0){$\times$}}
\put(1405,546){\makebox(0,0){$\times$}}
\put(1349,172){\makebox(0,0){$\times$}}
\put(151.0,131.0){\rule[-0.200pt]{0.400pt}{175.375pt}}
\put(151.0,131.0){\rule[-0.200pt]{310.279pt}{0.400pt}}
\put(1439.0,131.0){\rule[-0.200pt]{0.400pt}{175.375pt}}
\put(151.0,859.0){\rule[-0.200pt]{310.279pt}{0.400pt}}
\end{picture}

\caption{Závislosť polohy $x$ v čase $t$, preložené funkciou $x= \(1\cdot10^11\pm1.97\cdot10^10\)e^{-\(1.55\pm0.01\)t} sin\(\(16.30\pm0.01\)t +\( 58.8\pm0.19\)\) $}  \label{G_4}
\end{figure}






\subsection{Nútené a tlmené kmity}

V grafe Obr. \ref{G_6} je vynesená kalibračná krivka otáčiek motoru na vstupnom napätí. Z toho bola získaný vzťah
\eq{
\gamma = "\(2.1U-0.35\) rad \cdot s^{-1} /V"\,, \lbl{R_V1}
}
kde $U$ ja úbytok napätie na motorčeku.

\begin{figure}
% GNUPLOT: LaTeX picture
\setlength{\unitlength}{0.240900pt}
\ifx\plotpoint\undefined\newsavebox{\plotpoint}\fi
\begin{picture}(1500,900)(0,0)
\sbox{\plotpoint}{\rule[-0.200pt]{0.400pt}{0.400pt}}%
\put(151.0,131.0){\rule[-0.200pt]{4.818pt}{0.400pt}}
\put(131,131){\makebox(0,0)[r]{ 2}}
\put(1419.0,131.0){\rule[-0.200pt]{4.818pt}{0.400pt}}
\put(151.0,212.0){\rule[-0.200pt]{4.818pt}{0.400pt}}
\put(131,212){\makebox(0,0)[r]{ 4}}
\put(1419.0,212.0){\rule[-0.200pt]{4.818pt}{0.400pt}}
\put(151.0,293.0){\rule[-0.200pt]{4.818pt}{0.400pt}}
\put(131,293){\makebox(0,0)[r]{ 6}}
\put(1419.0,293.0){\rule[-0.200pt]{4.818pt}{0.400pt}}
\put(151.0,374.0){\rule[-0.200pt]{4.818pt}{0.400pt}}
\put(131,374){\makebox(0,0)[r]{ 8}}
\put(1419.0,374.0){\rule[-0.200pt]{4.818pt}{0.400pt}}
\put(151.0,455.0){\rule[-0.200pt]{4.818pt}{0.400pt}}
\put(131,455){\makebox(0,0)[r]{ 10}}
\put(1419.0,455.0){\rule[-0.200pt]{4.818pt}{0.400pt}}
\put(151.0,535.0){\rule[-0.200pt]{4.818pt}{0.400pt}}
\put(131,535){\makebox(0,0)[r]{ 12}}
\put(1419.0,535.0){\rule[-0.200pt]{4.818pt}{0.400pt}}
\put(151.0,616.0){\rule[-0.200pt]{4.818pt}{0.400pt}}
\put(131,616){\makebox(0,0)[r]{ 14}}
\put(1419.0,616.0){\rule[-0.200pt]{4.818pt}{0.400pt}}
\put(151.0,697.0){\rule[-0.200pt]{4.818pt}{0.400pt}}
\put(131,697){\makebox(0,0)[r]{ 16}}
\put(1419.0,697.0){\rule[-0.200pt]{4.818pt}{0.400pt}}
\put(151.0,778.0){\rule[-0.200pt]{4.818pt}{0.400pt}}
\put(131,778){\makebox(0,0)[r]{ 18}}
\put(1419.0,778.0){\rule[-0.200pt]{4.818pt}{0.400pt}}
\put(151.0,859.0){\rule[-0.200pt]{4.818pt}{0.400pt}}
\put(131,859){\makebox(0,0)[r]{ 20}}
\put(1419.0,859.0){\rule[-0.200pt]{4.818pt}{0.400pt}}
\put(151.0,131.0){\rule[-0.200pt]{0.400pt}{4.818pt}}
\put(151,90){\makebox(0,0){ 2}}
\put(151.0,839.0){\rule[-0.200pt]{0.400pt}{4.818pt}}
\put(335.0,131.0){\rule[-0.200pt]{0.400pt}{4.818pt}}
\put(335,90){\makebox(0,0){ 3}}
\put(335.0,839.0){\rule[-0.200pt]{0.400pt}{4.818pt}}
\put(519.0,131.0){\rule[-0.200pt]{0.400pt}{4.818pt}}
\put(519,90){\makebox(0,0){ 4}}
\put(519.0,839.0){\rule[-0.200pt]{0.400pt}{4.818pt}}
\put(703.0,131.0){\rule[-0.200pt]{0.400pt}{4.818pt}}
\put(703,90){\makebox(0,0){ 5}}
\put(703.0,839.0){\rule[-0.200pt]{0.400pt}{4.818pt}}
\put(887.0,131.0){\rule[-0.200pt]{0.400pt}{4.818pt}}
\put(887,90){\makebox(0,0){ 6}}
\put(887.0,839.0){\rule[-0.200pt]{0.400pt}{4.818pt}}
\put(1071.0,131.0){\rule[-0.200pt]{0.400pt}{4.818pt}}
\put(1071,90){\makebox(0,0){ 7}}
\put(1071.0,839.0){\rule[-0.200pt]{0.400pt}{4.818pt}}
\put(1255.0,131.0){\rule[-0.200pt]{0.400pt}{4.818pt}}
\put(1255,90){\makebox(0,0){ 8}}
\put(1255.0,839.0){\rule[-0.200pt]{0.400pt}{4.818pt}}
\put(1439.0,131.0){\rule[-0.200pt]{0.400pt}{4.818pt}}
\put(1439,90){\makebox(0,0){ 9}}
\put(1439.0,839.0){\rule[-0.200pt]{0.400pt}{4.818pt}}
\put(151.0,131.0){\rule[-0.200pt]{0.400pt}{175.375pt}}
\put(151.0,131.0){\rule[-0.200pt]{310.279pt}{0.400pt}}
\put(1439.0,131.0){\rule[-0.200pt]{0.400pt}{175.375pt}}
\put(151.0,859.0){\rule[-0.200pt]{310.279pt}{0.400pt}}
\put(30,495){\makebox(0,0){\popi{\gamma}{Hz}}}
\put(795,29){\makebox(0,0){\popi{U}{V}}}
\put(1279,213){\makebox(0,0)[r]{linarny fit $\gamma = "\(2.1U-0.35V\) rad \cdot s^{-1} \cdot V"$}}
\put(1299.0,213.0){\rule[-0.200pt]{24.090pt}{0.400pt}}
\put(151,205){\usebox{\plotpoint}}
\multiput(151.00,205.59)(1.123,0.482){9}{\rule{0.967pt}{0.116pt}}
\multiput(151.00,204.17)(10.994,6.000){2}{\rule{0.483pt}{0.400pt}}
\multiput(164.00,211.59)(1.123,0.482){9}{\rule{0.967pt}{0.116pt}}
\multiput(164.00,210.17)(10.994,6.000){2}{\rule{0.483pt}{0.400pt}}
\multiput(177.00,217.59)(1.123,0.482){9}{\rule{0.967pt}{0.116pt}}
\multiput(177.00,216.17)(10.994,6.000){2}{\rule{0.483pt}{0.400pt}}
\multiput(190.00,223.59)(1.123,0.482){9}{\rule{0.967pt}{0.116pt}}
\multiput(190.00,222.17)(10.994,6.000){2}{\rule{0.483pt}{0.400pt}}
\multiput(203.00,229.59)(1.123,0.482){9}{\rule{0.967pt}{0.116pt}}
\multiput(203.00,228.17)(10.994,6.000){2}{\rule{0.483pt}{0.400pt}}
\multiput(216.00,235.59)(1.123,0.482){9}{\rule{0.967pt}{0.116pt}}
\multiput(216.00,234.17)(10.994,6.000){2}{\rule{0.483pt}{0.400pt}}
\multiput(229.00,241.59)(1.123,0.482){9}{\rule{0.967pt}{0.116pt}}
\multiput(229.00,240.17)(10.994,6.000){2}{\rule{0.483pt}{0.400pt}}
\multiput(242.00,247.59)(1.123,0.482){9}{\rule{0.967pt}{0.116pt}}
\multiput(242.00,246.17)(10.994,6.000){2}{\rule{0.483pt}{0.400pt}}
\multiput(255.00,253.59)(1.123,0.482){9}{\rule{0.967pt}{0.116pt}}
\multiput(255.00,252.17)(10.994,6.000){2}{\rule{0.483pt}{0.400pt}}
\multiput(268.00,259.59)(1.123,0.482){9}{\rule{0.967pt}{0.116pt}}
\multiput(268.00,258.17)(10.994,6.000){2}{\rule{0.483pt}{0.400pt}}
\multiput(281.00,265.59)(1.123,0.482){9}{\rule{0.967pt}{0.116pt}}
\multiput(281.00,264.17)(10.994,6.000){2}{\rule{0.483pt}{0.400pt}}
\multiput(294.00,271.59)(1.123,0.482){9}{\rule{0.967pt}{0.116pt}}
\multiput(294.00,270.17)(10.994,6.000){2}{\rule{0.483pt}{0.400pt}}
\multiput(307.00,277.59)(1.123,0.482){9}{\rule{0.967pt}{0.116pt}}
\multiput(307.00,276.17)(10.994,6.000){2}{\rule{0.483pt}{0.400pt}}
\multiput(320.00,283.59)(1.123,0.482){9}{\rule{0.967pt}{0.116pt}}
\multiput(320.00,282.17)(10.994,6.000){2}{\rule{0.483pt}{0.400pt}}
\multiput(333.00,289.59)(1.123,0.482){9}{\rule{0.967pt}{0.116pt}}
\multiput(333.00,288.17)(10.994,6.000){2}{\rule{0.483pt}{0.400pt}}
\multiput(346.00,295.59)(1.123,0.482){9}{\rule{0.967pt}{0.116pt}}
\multiput(346.00,294.17)(10.994,6.000){2}{\rule{0.483pt}{0.400pt}}
\multiput(359.00,301.59)(1.123,0.482){9}{\rule{0.967pt}{0.116pt}}
\multiput(359.00,300.17)(10.994,6.000){2}{\rule{0.483pt}{0.400pt}}
\multiput(372.00,307.59)(1.123,0.482){9}{\rule{0.967pt}{0.116pt}}
\multiput(372.00,306.17)(10.994,6.000){2}{\rule{0.483pt}{0.400pt}}
\multiput(385.00,313.59)(1.123,0.482){9}{\rule{0.967pt}{0.116pt}}
\multiput(385.00,312.17)(10.994,6.000){2}{\rule{0.483pt}{0.400pt}}
\multiput(398.00,319.59)(1.123,0.482){9}{\rule{0.967pt}{0.116pt}}
\multiput(398.00,318.17)(10.994,6.000){2}{\rule{0.483pt}{0.400pt}}
\multiput(411.00,325.59)(1.123,0.482){9}{\rule{0.967pt}{0.116pt}}
\multiput(411.00,324.17)(10.994,6.000){2}{\rule{0.483pt}{0.400pt}}
\multiput(424.00,331.59)(1.123,0.482){9}{\rule{0.967pt}{0.116pt}}
\multiput(424.00,330.17)(10.994,6.000){2}{\rule{0.483pt}{0.400pt}}
\multiput(437.00,337.59)(1.123,0.482){9}{\rule{0.967pt}{0.116pt}}
\multiput(437.00,336.17)(10.994,6.000){2}{\rule{0.483pt}{0.400pt}}
\multiput(450.00,343.59)(1.123,0.482){9}{\rule{0.967pt}{0.116pt}}
\multiput(450.00,342.17)(10.994,6.000){2}{\rule{0.483pt}{0.400pt}}
\multiput(463.00,349.59)(1.123,0.482){9}{\rule{0.967pt}{0.116pt}}
\multiput(463.00,348.17)(10.994,6.000){2}{\rule{0.483pt}{0.400pt}}
\multiput(476.00,355.59)(1.123,0.482){9}{\rule{0.967pt}{0.116pt}}
\multiput(476.00,354.17)(10.994,6.000){2}{\rule{0.483pt}{0.400pt}}
\multiput(489.00,361.59)(1.123,0.482){9}{\rule{0.967pt}{0.116pt}}
\multiput(489.00,360.17)(10.994,6.000){2}{\rule{0.483pt}{0.400pt}}
\multiput(502.00,367.59)(1.123,0.482){9}{\rule{0.967pt}{0.116pt}}
\multiput(502.00,366.17)(10.994,6.000){2}{\rule{0.483pt}{0.400pt}}
\multiput(515.00,373.59)(1.123,0.482){9}{\rule{0.967pt}{0.116pt}}
\multiput(515.00,372.17)(10.994,6.000){2}{\rule{0.483pt}{0.400pt}}
\multiput(528.00,379.59)(1.123,0.482){9}{\rule{0.967pt}{0.116pt}}
\multiput(528.00,378.17)(10.994,6.000){2}{\rule{0.483pt}{0.400pt}}
\multiput(541.00,385.59)(1.123,0.482){9}{\rule{0.967pt}{0.116pt}}
\multiput(541.00,384.17)(10.994,6.000){2}{\rule{0.483pt}{0.400pt}}
\multiput(554.00,391.59)(1.123,0.482){9}{\rule{0.967pt}{0.116pt}}
\multiput(554.00,390.17)(10.994,6.000){2}{\rule{0.483pt}{0.400pt}}
\multiput(567.00,397.59)(1.123,0.482){9}{\rule{0.967pt}{0.116pt}}
\multiput(567.00,396.17)(10.994,6.000){2}{\rule{0.483pt}{0.400pt}}
\multiput(580.00,403.59)(1.123,0.482){9}{\rule{0.967pt}{0.116pt}}
\multiput(580.00,402.17)(10.994,6.000){2}{\rule{0.483pt}{0.400pt}}
\multiput(593.00,409.59)(1.123,0.482){9}{\rule{0.967pt}{0.116pt}}
\multiput(593.00,408.17)(10.994,6.000){2}{\rule{0.483pt}{0.400pt}}
\multiput(606.00,415.59)(1.123,0.482){9}{\rule{0.967pt}{0.116pt}}
\multiput(606.00,414.17)(10.994,6.000){2}{\rule{0.483pt}{0.400pt}}
\multiput(619.00,421.59)(1.123,0.482){9}{\rule{0.967pt}{0.116pt}}
\multiput(619.00,420.17)(10.994,6.000){2}{\rule{0.483pt}{0.400pt}}
\multiput(632.00,427.59)(1.123,0.482){9}{\rule{0.967pt}{0.116pt}}
\multiput(632.00,426.17)(10.994,6.000){2}{\rule{0.483pt}{0.400pt}}
\multiput(645.00,433.59)(1.123,0.482){9}{\rule{0.967pt}{0.116pt}}
\multiput(645.00,432.17)(10.994,6.000){2}{\rule{0.483pt}{0.400pt}}
\multiput(658.00,439.59)(1.123,0.482){9}{\rule{0.967pt}{0.116pt}}
\multiput(658.00,438.17)(10.994,6.000){2}{\rule{0.483pt}{0.400pt}}
\multiput(671.00,445.59)(1.123,0.482){9}{\rule{0.967pt}{0.116pt}}
\multiput(671.00,444.17)(10.994,6.000){2}{\rule{0.483pt}{0.400pt}}
\multiput(684.00,451.59)(1.123,0.482){9}{\rule{0.967pt}{0.116pt}}
\multiput(684.00,450.17)(10.994,6.000){2}{\rule{0.483pt}{0.400pt}}
\multiput(697.00,457.59)(1.123,0.482){9}{\rule{0.967pt}{0.116pt}}
\multiput(697.00,456.17)(10.994,6.000){2}{\rule{0.483pt}{0.400pt}}
\multiput(710.00,463.59)(1.123,0.482){9}{\rule{0.967pt}{0.116pt}}
\multiput(710.00,462.17)(10.994,6.000){2}{\rule{0.483pt}{0.400pt}}
\multiput(723.00,469.59)(1.123,0.482){9}{\rule{0.967pt}{0.116pt}}
\multiput(723.00,468.17)(10.994,6.000){2}{\rule{0.483pt}{0.400pt}}
\multiput(736.00,475.59)(1.123,0.482){9}{\rule{0.967pt}{0.116pt}}
\multiput(736.00,474.17)(10.994,6.000){2}{\rule{0.483pt}{0.400pt}}
\multiput(749.00,481.59)(1.123,0.482){9}{\rule{0.967pt}{0.116pt}}
\multiput(749.00,480.17)(10.994,6.000){2}{\rule{0.483pt}{0.400pt}}
\multiput(762.00,487.59)(1.123,0.482){9}{\rule{0.967pt}{0.116pt}}
\multiput(762.00,486.17)(10.994,6.000){2}{\rule{0.483pt}{0.400pt}}
\multiput(775.00,493.59)(1.123,0.482){9}{\rule{0.967pt}{0.116pt}}
\multiput(775.00,492.17)(10.994,6.000){2}{\rule{0.483pt}{0.400pt}}
\multiput(788.00,499.59)(1.214,0.482){9}{\rule{1.033pt}{0.116pt}}
\multiput(788.00,498.17)(11.855,6.000){2}{\rule{0.517pt}{0.400pt}}
\multiput(802.00,505.59)(1.123,0.482){9}{\rule{0.967pt}{0.116pt}}
\multiput(802.00,504.17)(10.994,6.000){2}{\rule{0.483pt}{0.400pt}}
\multiput(815.00,511.59)(1.123,0.482){9}{\rule{0.967pt}{0.116pt}}
\multiput(815.00,510.17)(10.994,6.000){2}{\rule{0.483pt}{0.400pt}}
\multiput(828.00,517.59)(1.123,0.482){9}{\rule{0.967pt}{0.116pt}}
\multiput(828.00,516.17)(10.994,6.000){2}{\rule{0.483pt}{0.400pt}}
\multiput(841.00,523.59)(1.123,0.482){9}{\rule{0.967pt}{0.116pt}}
\multiput(841.00,522.17)(10.994,6.000){2}{\rule{0.483pt}{0.400pt}}
\multiput(854.00,529.59)(1.123,0.482){9}{\rule{0.967pt}{0.116pt}}
\multiput(854.00,528.17)(10.994,6.000){2}{\rule{0.483pt}{0.400pt}}
\multiput(867.00,535.59)(1.123,0.482){9}{\rule{0.967pt}{0.116pt}}
\multiput(867.00,534.17)(10.994,6.000){2}{\rule{0.483pt}{0.400pt}}
\multiput(880.00,541.59)(1.123,0.482){9}{\rule{0.967pt}{0.116pt}}
\multiput(880.00,540.17)(10.994,6.000){2}{\rule{0.483pt}{0.400pt}}
\multiput(893.00,547.59)(1.123,0.482){9}{\rule{0.967pt}{0.116pt}}
\multiput(893.00,546.17)(10.994,6.000){2}{\rule{0.483pt}{0.400pt}}
\multiput(906.00,553.59)(1.123,0.482){9}{\rule{0.967pt}{0.116pt}}
\multiput(906.00,552.17)(10.994,6.000){2}{\rule{0.483pt}{0.400pt}}
\multiput(919.00,559.59)(1.123,0.482){9}{\rule{0.967pt}{0.116pt}}
\multiput(919.00,558.17)(10.994,6.000){2}{\rule{0.483pt}{0.400pt}}
\multiput(932.00,565.59)(1.123,0.482){9}{\rule{0.967pt}{0.116pt}}
\multiput(932.00,564.17)(10.994,6.000){2}{\rule{0.483pt}{0.400pt}}
\multiput(945.00,571.59)(1.123,0.482){9}{\rule{0.967pt}{0.116pt}}
\multiput(945.00,570.17)(10.994,6.000){2}{\rule{0.483pt}{0.400pt}}
\multiput(958.00,577.59)(1.123,0.482){9}{\rule{0.967pt}{0.116pt}}
\multiput(958.00,576.17)(10.994,6.000){2}{\rule{0.483pt}{0.400pt}}
\multiput(971.00,583.59)(1.123,0.482){9}{\rule{0.967pt}{0.116pt}}
\multiput(971.00,582.17)(10.994,6.000){2}{\rule{0.483pt}{0.400pt}}
\multiput(984.00,589.59)(1.123,0.482){9}{\rule{0.967pt}{0.116pt}}
\multiput(984.00,588.17)(10.994,6.000){2}{\rule{0.483pt}{0.400pt}}
\multiput(997.00,595.59)(1.123,0.482){9}{\rule{0.967pt}{0.116pt}}
\multiput(997.00,594.17)(10.994,6.000){2}{\rule{0.483pt}{0.400pt}}
\multiput(1010.00,601.59)(1.123,0.482){9}{\rule{0.967pt}{0.116pt}}
\multiput(1010.00,600.17)(10.994,6.000){2}{\rule{0.483pt}{0.400pt}}
\multiput(1023.00,607.59)(1.123,0.482){9}{\rule{0.967pt}{0.116pt}}
\multiput(1023.00,606.17)(10.994,6.000){2}{\rule{0.483pt}{0.400pt}}
\multiput(1036.00,613.59)(1.123,0.482){9}{\rule{0.967pt}{0.116pt}}
\multiput(1036.00,612.17)(10.994,6.000){2}{\rule{0.483pt}{0.400pt}}
\multiput(1049.00,619.59)(1.123,0.482){9}{\rule{0.967pt}{0.116pt}}
\multiput(1049.00,618.17)(10.994,6.000){2}{\rule{0.483pt}{0.400pt}}
\multiput(1062.00,625.59)(1.123,0.482){9}{\rule{0.967pt}{0.116pt}}
\multiput(1062.00,624.17)(10.994,6.000){2}{\rule{0.483pt}{0.400pt}}
\multiput(1075.00,631.59)(1.123,0.482){9}{\rule{0.967pt}{0.116pt}}
\multiput(1075.00,630.17)(10.994,6.000){2}{\rule{0.483pt}{0.400pt}}
\multiput(1088.00,637.59)(1.123,0.482){9}{\rule{0.967pt}{0.116pt}}
\multiput(1088.00,636.17)(10.994,6.000){2}{\rule{0.483pt}{0.400pt}}
\multiput(1101.00,643.59)(1.123,0.482){9}{\rule{0.967pt}{0.116pt}}
\multiput(1101.00,642.17)(10.994,6.000){2}{\rule{0.483pt}{0.400pt}}
\multiput(1114.00,649.59)(1.123,0.482){9}{\rule{0.967pt}{0.116pt}}
\multiput(1114.00,648.17)(10.994,6.000){2}{\rule{0.483pt}{0.400pt}}
\multiput(1127.00,655.59)(1.123,0.482){9}{\rule{0.967pt}{0.116pt}}
\multiput(1127.00,654.17)(10.994,6.000){2}{\rule{0.483pt}{0.400pt}}
\multiput(1140.00,661.59)(1.123,0.482){9}{\rule{0.967pt}{0.116pt}}
\multiput(1140.00,660.17)(10.994,6.000){2}{\rule{0.483pt}{0.400pt}}
\multiput(1153.00,667.59)(1.123,0.482){9}{\rule{0.967pt}{0.116pt}}
\multiput(1153.00,666.17)(10.994,6.000){2}{\rule{0.483pt}{0.400pt}}
\multiput(1166.00,673.59)(1.378,0.477){7}{\rule{1.140pt}{0.115pt}}
\multiput(1166.00,672.17)(10.634,5.000){2}{\rule{0.570pt}{0.400pt}}
\multiput(1179.00,678.59)(1.123,0.482){9}{\rule{0.967pt}{0.116pt}}
\multiput(1179.00,677.17)(10.994,6.000){2}{\rule{0.483pt}{0.400pt}}
\multiput(1192.00,684.59)(1.123,0.482){9}{\rule{0.967pt}{0.116pt}}
\multiput(1192.00,683.17)(10.994,6.000){2}{\rule{0.483pt}{0.400pt}}
\multiput(1205.00,690.59)(1.123,0.482){9}{\rule{0.967pt}{0.116pt}}
\multiput(1205.00,689.17)(10.994,6.000){2}{\rule{0.483pt}{0.400pt}}
\multiput(1218.00,696.59)(1.123,0.482){9}{\rule{0.967pt}{0.116pt}}
\multiput(1218.00,695.17)(10.994,6.000){2}{\rule{0.483pt}{0.400pt}}
\multiput(1231.00,702.59)(1.123,0.482){9}{\rule{0.967pt}{0.116pt}}
\multiput(1231.00,701.17)(10.994,6.000){2}{\rule{0.483pt}{0.400pt}}
\multiput(1244.00,708.59)(1.123,0.482){9}{\rule{0.967pt}{0.116pt}}
\multiput(1244.00,707.17)(10.994,6.000){2}{\rule{0.483pt}{0.400pt}}
\multiput(1257.00,714.59)(1.123,0.482){9}{\rule{0.967pt}{0.116pt}}
\multiput(1257.00,713.17)(10.994,6.000){2}{\rule{0.483pt}{0.400pt}}
\multiput(1270.00,720.59)(1.123,0.482){9}{\rule{0.967pt}{0.116pt}}
\multiput(1270.00,719.17)(10.994,6.000){2}{\rule{0.483pt}{0.400pt}}
\multiput(1283.00,726.59)(1.123,0.482){9}{\rule{0.967pt}{0.116pt}}
\multiput(1283.00,725.17)(10.994,6.000){2}{\rule{0.483pt}{0.400pt}}
\multiput(1296.00,732.59)(1.123,0.482){9}{\rule{0.967pt}{0.116pt}}
\multiput(1296.00,731.17)(10.994,6.000){2}{\rule{0.483pt}{0.400pt}}
\multiput(1309.00,738.59)(1.123,0.482){9}{\rule{0.967pt}{0.116pt}}
\multiput(1309.00,737.17)(10.994,6.000){2}{\rule{0.483pt}{0.400pt}}
\multiput(1322.00,744.59)(1.123,0.482){9}{\rule{0.967pt}{0.116pt}}
\multiput(1322.00,743.17)(10.994,6.000){2}{\rule{0.483pt}{0.400pt}}
\multiput(1335.00,750.59)(1.123,0.482){9}{\rule{0.967pt}{0.116pt}}
\multiput(1335.00,749.17)(10.994,6.000){2}{\rule{0.483pt}{0.400pt}}
\multiput(1348.00,756.59)(1.123,0.482){9}{\rule{0.967pt}{0.116pt}}
\multiput(1348.00,755.17)(10.994,6.000){2}{\rule{0.483pt}{0.400pt}}
\multiput(1361.00,762.59)(1.123,0.482){9}{\rule{0.967pt}{0.116pt}}
\multiput(1361.00,761.17)(10.994,6.000){2}{\rule{0.483pt}{0.400pt}}
\multiput(1374.00,768.59)(1.123,0.482){9}{\rule{0.967pt}{0.116pt}}
\multiput(1374.00,767.17)(10.994,6.000){2}{\rule{0.483pt}{0.400pt}}
\multiput(1387.00,774.59)(1.123,0.482){9}{\rule{0.967pt}{0.116pt}}
\multiput(1387.00,773.17)(10.994,6.000){2}{\rule{0.483pt}{0.400pt}}
\multiput(1400.00,780.59)(1.123,0.482){9}{\rule{0.967pt}{0.116pt}}
\multiput(1400.00,779.17)(10.994,6.000){2}{\rule{0.483pt}{0.400pt}}
\multiput(1413.00,786.59)(1.123,0.482){9}{\rule{0.967pt}{0.116pt}}
\multiput(1413.00,785.17)(10.994,6.000){2}{\rule{0.483pt}{0.400pt}}
\multiput(1426.00,792.59)(1.123,0.482){9}{\rule{0.967pt}{0.116pt}}
\multiput(1426.00,791.17)(10.994,6.000){2}{\rule{0.483pt}{0.400pt}}
\put(1279,172){\makebox(0,0)[r]{namerané hodnoty}}
\put(887,500){\makebox(0,0){$\times$}}
\put(1071,637){\makebox(0,0){$\times$}}
\put(1255,719){\makebox(0,0){$\times$}}
\put(703,463){\makebox(0,0){$\times$}}
\put(519,375){\makebox(0,0){$\times$}}
\put(335,295){\makebox(0,0){$\times$}}
\put(151,215){\makebox(0,0){$\times$}}
\put(1439,811){\makebox(0,0){$\times$}}
\put(1349,172){\makebox(0,0){$\times$}}
\put(151.0,131.0){\rule[-0.200pt]{0.400pt}{175.375pt}}
\put(151.0,131.0){\rule[-0.200pt]{310.279pt}{0.400pt}}
\put(1439.0,131.0){\rule[-0.200pt]{0.400pt}{175.375pt}}
\put(151.0,859.0){\rule[-0.200pt]{310.279pt}{0.400pt}}
\end{picture}

\caption{Kalibračná krivka k prevodu napätia $U$ na otáčky motorčeka $\gamma$}  \label{G_6}
\end{figure}


\begin{figure}
% GNUPLOT: LaTeX picture
\setlength{\unitlength}{0.240900pt}
\ifx\plotpoint\undefined\newsavebox{\plotpoint}\fi
\begin{picture}(1500,900)(0,0)
\sbox{\plotpoint}{\rule[-0.200pt]{0.400pt}{0.400pt}}%
\put(151.0,131.0){\rule[-0.200pt]{4.818pt}{0.400pt}}
\put(131,131){\makebox(0,0)[r]{ 0}}
\put(1419.0,131.0){\rule[-0.200pt]{4.818pt}{0.400pt}}
\put(151.0,277.0){\rule[-0.200pt]{4.818pt}{0.400pt}}
\put(131,277){\makebox(0,0)[r]{ 5}}
\put(1419.0,277.0){\rule[-0.200pt]{4.818pt}{0.400pt}}
\put(151.0,422.0){\rule[-0.200pt]{4.818pt}{0.400pt}}
\put(131,422){\makebox(0,0)[r]{ 10}}
\put(1419.0,422.0){\rule[-0.200pt]{4.818pt}{0.400pt}}
\put(151.0,568.0){\rule[-0.200pt]{4.818pt}{0.400pt}}
\put(131,568){\makebox(0,0)[r]{ 15}}
\put(1419.0,568.0){\rule[-0.200pt]{4.818pt}{0.400pt}}
\put(151.0,713.0){\rule[-0.200pt]{4.818pt}{0.400pt}}
\put(131,713){\makebox(0,0)[r]{ 20}}
\put(1419.0,713.0){\rule[-0.200pt]{4.818pt}{0.400pt}}
\put(151.0,859.0){\rule[-0.200pt]{4.818pt}{0.400pt}}
\put(131,859){\makebox(0,0)[r]{ 25}}
\put(1419.0,859.0){\rule[-0.200pt]{4.818pt}{0.400pt}}
\put(151.0,131.0){\rule[-0.200pt]{0.400pt}{4.818pt}}
\put(151,90){\makebox(0,0){ 0}}
\put(151.0,839.0){\rule[-0.200pt]{0.400pt}{4.818pt}}
\put(312.0,131.0){\rule[-0.200pt]{0.400pt}{4.818pt}}
\put(312,90){\makebox(0,0){ 5}}
\put(312.0,839.0){\rule[-0.200pt]{0.400pt}{4.818pt}}
\put(473.0,131.0){\rule[-0.200pt]{0.400pt}{4.818pt}}
\put(473,90){\makebox(0,0){ 10}}
\put(473.0,839.0){\rule[-0.200pt]{0.400pt}{4.818pt}}
\put(634.0,131.0){\rule[-0.200pt]{0.400pt}{4.818pt}}
\put(634,90){\makebox(0,0){ 15}}
\put(634.0,839.0){\rule[-0.200pt]{0.400pt}{4.818pt}}
\put(795.0,131.0){\rule[-0.200pt]{0.400pt}{4.818pt}}
\put(795,90){\makebox(0,0){ 20}}
\put(795.0,839.0){\rule[-0.200pt]{0.400pt}{4.818pt}}
\put(956.0,131.0){\rule[-0.200pt]{0.400pt}{4.818pt}}
\put(956,90){\makebox(0,0){ 25}}
\put(956.0,839.0){\rule[-0.200pt]{0.400pt}{4.818pt}}
\put(1117.0,131.0){\rule[-0.200pt]{0.400pt}{4.818pt}}
\put(1117,90){\makebox(0,0){ 30}}
\put(1117.0,839.0){\rule[-0.200pt]{0.400pt}{4.818pt}}
\put(1278.0,131.0){\rule[-0.200pt]{0.400pt}{4.818pt}}
\put(1278,90){\makebox(0,0){ 35}}
\put(1278.0,839.0){\rule[-0.200pt]{0.400pt}{4.818pt}}
\put(1439.0,131.0){\rule[-0.200pt]{0.400pt}{4.818pt}}
\put(1439,90){\makebox(0,0){ 40}}
\put(1439.0,839.0){\rule[-0.200pt]{0.400pt}{4.818pt}}
\put(151.0,131.0){\rule[-0.200pt]{0.400pt}{175.375pt}}
\put(151.0,131.0){\rule[-0.200pt]{310.279pt}{0.400pt}}
\put(1439.0,131.0){\rule[-0.200pt]{0.400pt}{175.375pt}}
\put(151.0,859.0){\rule[-0.200pt]{310.279pt}{0.400pt}}
\put(30,495){\makebox(0,0){\popi{B}{mm}}}
\put(795,29){\makebox(0,0){\popi{\gamma}{Hz}}}
\put(1279,819){\makebox(0,0)[r]{dáta pre veľké závažie}}
\put(814,214){\makebox(0,0){$+$}}
\put(1286,226){\makebox(0,0){$+$}}
\put(1354,650){\makebox(0,0){$+$}}
\put(1421,761){\makebox(0,0){$+$}}
\put(207,179){\makebox(0,0){$+$}}
\put(275,180){\makebox(0,0){$+$}}
\put(342,182){\makebox(0,0){$+$}}
\put(409,184){\makebox(0,0){$+$}}
\put(477,204){\makebox(0,0){$+$}}
\put(612,189){\makebox(0,0){$+$}}
\put(747,183){\makebox(0,0){$+$}}
\put(1349,819){\makebox(0,0){$+$}}
\put(1279,778){\makebox(0,0)[r]{dáta pre stredné závažie}}
\put(814,270){\makebox(0,0){$\times$}}
\put(983,631){\makebox(0,0){$\times$}}
\put(949,650){\makebox(0,0){$\times$}}
\put(1016,415){\makebox(0,0){$\times$}}
\put(1084,292){\makebox(0,0){$\times$}}
\put(207,181){\makebox(0,0){$\times$}}
\put(342,178){\makebox(0,0){$\times$}}
\put(477,184){\makebox(0,0){$\times$}}
\put(612,205){\makebox(0,0){$\times$}}
\put(1349,778){\makebox(0,0){$\times$}}
\sbox{\plotpoint}{\rule[-0.400pt]{0.800pt}{0.800pt}}%
\sbox{\plotpoint}{\rule[-0.200pt]{0.400pt}{0.400pt}}%
\put(1279,737){\makebox(0,0)[r]{fit $B_{\(\gamma\)} = \frac{1.55}{4.16\sqrt{\(12.23^2 - \gamma^2\)^2+4\cdot 0.5^2 \gamma^2 } }$}}
\sbox{\plotpoint}{\rule[-0.400pt]{0.800pt}{0.800pt}}%
\put(1299.0,737.0){\rule[-0.400pt]{24.090pt}{0.800pt}}
\put(207,178){\usebox{\plotpoint}}
\put(232,176.84){\rule{2.891pt}{0.800pt}}
\multiput(232.00,176.34)(6.000,1.000){2}{\rule{1.445pt}{0.800pt}}
\put(207.0,178.0){\rule[-0.400pt]{6.022pt}{0.800pt}}
\put(293,177.84){\rule{2.891pt}{0.800pt}}
\multiput(293.00,177.34)(6.000,1.000){2}{\rule{1.445pt}{0.800pt}}
\put(244.0,179.0){\rule[-0.400pt]{11.804pt}{0.800pt}}
\put(330,178.84){\rule{2.891pt}{0.800pt}}
\multiput(330.00,178.34)(6.000,1.000){2}{\rule{1.445pt}{0.800pt}}
\put(305.0,180.0){\rule[-0.400pt]{6.022pt}{0.800pt}}
\put(367,179.84){\rule{2.891pt}{0.800pt}}
\multiput(367.00,179.34)(6.000,1.000){2}{\rule{1.445pt}{0.800pt}}
\put(342.0,181.0){\rule[-0.400pt]{6.022pt}{0.800pt}}
\put(391,180.84){\rule{2.891pt}{0.800pt}}
\multiput(391.00,180.34)(6.000,1.000){2}{\rule{1.445pt}{0.800pt}}
\put(379.0,182.0){\rule[-0.400pt]{2.891pt}{0.800pt}}
\put(416,181.84){\rule{2.891pt}{0.800pt}}
\multiput(416.00,181.34)(6.000,1.000){2}{\rule{1.445pt}{0.800pt}}
\put(428,182.84){\rule{2.891pt}{0.800pt}}
\multiput(428.00,182.34)(6.000,1.000){2}{\rule{1.445pt}{0.800pt}}
\put(403.0,183.0){\rule[-0.400pt]{3.132pt}{0.800pt}}
\put(452,183.84){\rule{3.132pt}{0.800pt}}
\multiput(452.00,183.34)(6.500,1.000){2}{\rule{1.566pt}{0.800pt}}
\put(465,184.84){\rule{2.891pt}{0.800pt}}
\multiput(465.00,184.34)(6.000,1.000){2}{\rule{1.445pt}{0.800pt}}
\put(477,185.84){\rule{2.891pt}{0.800pt}}
\multiput(477.00,185.34)(6.000,1.000){2}{\rule{1.445pt}{0.800pt}}
\put(489,186.84){\rule{2.891pt}{0.800pt}}
\multiput(489.00,186.34)(6.000,1.000){2}{\rule{1.445pt}{0.800pt}}
\put(440.0,185.0){\rule[-0.400pt]{2.891pt}{0.800pt}}
\put(514,187.84){\rule{2.891pt}{0.800pt}}
\multiput(514.00,187.34)(6.000,1.000){2}{\rule{1.445pt}{0.800pt}}
\put(526,189.34){\rule{2.891pt}{0.800pt}}
\multiput(526.00,188.34)(6.000,2.000){2}{\rule{1.445pt}{0.800pt}}
\put(538,190.84){\rule{2.891pt}{0.800pt}}
\multiput(538.00,190.34)(6.000,1.000){2}{\rule{1.445pt}{0.800pt}}
\put(550,191.84){\rule{3.132pt}{0.800pt}}
\multiput(550.00,191.34)(6.500,1.000){2}{\rule{1.566pt}{0.800pt}}
\put(563,192.84){\rule{2.891pt}{0.800pt}}
\multiput(563.00,192.34)(6.000,1.000){2}{\rule{1.445pt}{0.800pt}}
\put(575,194.34){\rule{2.891pt}{0.800pt}}
\multiput(575.00,193.34)(6.000,2.000){2}{\rule{1.445pt}{0.800pt}}
\put(587,195.84){\rule{3.132pt}{0.800pt}}
\multiput(587.00,195.34)(6.500,1.000){2}{\rule{1.566pt}{0.800pt}}
\put(600,197.34){\rule{2.891pt}{0.800pt}}
\multiput(600.00,196.34)(6.000,2.000){2}{\rule{1.445pt}{0.800pt}}
\put(612,199.34){\rule{2.891pt}{0.800pt}}
\multiput(612.00,198.34)(6.000,2.000){2}{\rule{1.445pt}{0.800pt}}
\put(624,201.34){\rule{2.891pt}{0.800pt}}
\multiput(624.00,200.34)(6.000,2.000){2}{\rule{1.445pt}{0.800pt}}
\put(636,203.34){\rule{3.132pt}{0.800pt}}
\multiput(636.00,202.34)(6.500,2.000){2}{\rule{1.566pt}{0.800pt}}
\put(649,205.34){\rule{2.891pt}{0.800pt}}
\multiput(649.00,204.34)(6.000,2.000){2}{\rule{1.445pt}{0.800pt}}
\put(661,207.34){\rule{2.891pt}{0.800pt}}
\multiput(661.00,206.34)(6.000,2.000){2}{\rule{1.445pt}{0.800pt}}
\put(673,209.84){\rule{2.891pt}{0.800pt}}
\multiput(673.00,208.34)(6.000,3.000){2}{\rule{1.445pt}{0.800pt}}
\put(685,212.84){\rule{3.132pt}{0.800pt}}
\multiput(685.00,211.34)(6.500,3.000){2}{\rule{1.566pt}{0.800pt}}
\put(698,215.84){\rule{2.891pt}{0.800pt}}
\multiput(698.00,214.34)(6.000,3.000){2}{\rule{1.445pt}{0.800pt}}
\put(710,219.34){\rule{2.600pt}{0.800pt}}
\multiput(710.00,217.34)(6.604,4.000){2}{\rule{1.300pt}{0.800pt}}
\put(722,223.34){\rule{2.600pt}{0.800pt}}
\multiput(722.00,221.34)(6.604,4.000){2}{\rule{1.300pt}{0.800pt}}
\put(734,227.34){\rule{2.800pt}{0.800pt}}
\multiput(734.00,225.34)(7.188,4.000){2}{\rule{1.400pt}{0.800pt}}
\multiput(747.00,232.38)(1.600,0.560){3}{\rule{2.120pt}{0.135pt}}
\multiput(747.00,229.34)(7.600,5.000){2}{\rule{1.060pt}{0.800pt}}
\multiput(759.00,237.39)(1.132,0.536){5}{\rule{1.800pt}{0.129pt}}
\multiput(759.00,234.34)(8.264,6.000){2}{\rule{0.900pt}{0.800pt}}
\multiput(771.00,243.39)(1.132,0.536){5}{\rule{1.800pt}{0.129pt}}
\multiput(771.00,240.34)(8.264,6.000){2}{\rule{0.900pt}{0.800pt}}
\multiput(783.00,249.40)(1.000,0.526){7}{\rule{1.686pt}{0.127pt}}
\multiput(783.00,246.34)(9.501,7.000){2}{\rule{0.843pt}{0.800pt}}
\multiput(796.00,256.40)(0.774,0.520){9}{\rule{1.400pt}{0.125pt}}
\multiput(796.00,253.34)(9.094,8.000){2}{\rule{0.700pt}{0.800pt}}
\multiput(808.00,264.40)(0.599,0.514){13}{\rule{1.160pt}{0.124pt}}
\multiput(808.00,261.34)(9.592,10.000){2}{\rule{0.580pt}{0.800pt}}
\multiput(820.00,274.40)(0.539,0.512){15}{\rule{1.073pt}{0.123pt}}
\multiput(820.00,271.34)(9.774,11.000){2}{\rule{0.536pt}{0.800pt}}
\multiput(832.00,285.41)(0.492,0.509){19}{\rule{1.000pt}{0.123pt}}
\multiput(832.00,282.34)(10.924,13.000){2}{\rule{0.500pt}{0.800pt}}
\multiput(846.41,297.00)(0.511,0.671){17}{\rule{0.123pt}{1.267pt}}
\multiput(843.34,297.00)(12.000,13.371){2}{\rule{0.800pt}{0.633pt}}
\multiput(858.41,313.00)(0.511,0.807){17}{\rule{0.123pt}{1.467pt}}
\multiput(855.34,313.00)(12.000,15.956){2}{\rule{0.800pt}{0.733pt}}
\multiput(870.41,332.00)(0.509,0.947){19}{\rule{0.123pt}{1.677pt}}
\multiput(867.34,332.00)(13.000,20.519){2}{\rule{0.800pt}{0.838pt}}
\multiput(883.41,356.00)(0.511,1.304){17}{\rule{0.123pt}{2.200pt}}
\multiput(880.34,356.00)(12.000,25.434){2}{\rule{0.800pt}{1.100pt}}
\multiput(895.41,386.00)(0.511,1.711){17}{\rule{0.123pt}{2.800pt}}
\multiput(892.34,386.00)(12.000,33.188){2}{\rule{0.800pt}{1.400pt}}
\multiput(907.41,425.00)(0.511,2.208){17}{\rule{0.123pt}{3.533pt}}
\multiput(904.34,425.00)(12.000,42.666){2}{\rule{0.800pt}{1.767pt}}
\multiput(919.41,475.00)(0.509,2.643){19}{\rule{0.123pt}{4.200pt}}
\multiput(916.34,475.00)(13.000,56.283){2}{\rule{0.800pt}{2.100pt}}
\multiput(932.41,540.00)(0.511,3.384){17}{\rule{0.123pt}{5.267pt}}
\multiput(929.34,540.00)(12.000,65.069){2}{\rule{0.800pt}{2.633pt}}
\multiput(944.41,616.00)(0.511,2.932){17}{\rule{0.123pt}{4.600pt}}
\multiput(941.34,616.00)(12.000,56.452){2}{\rule{0.800pt}{2.300pt}}
\multiput(956.41,682.00)(0.511,0.626){17}{\rule{0.123pt}{1.200pt}}
\multiput(953.34,682.00)(12.000,12.509){2}{\rule{0.800pt}{0.600pt}}
\multiput(968.41,682.89)(0.509,-2.105){19}{\rule{0.123pt}{3.400pt}}
\multiput(965.34,689.94)(13.000,-44.943){2}{\rule{0.800pt}{1.700pt}}
\multiput(981.41,621.75)(0.511,-3.610){17}{\rule{0.123pt}{5.600pt}}
\multiput(978.34,633.38)(12.000,-69.377){2}{\rule{0.800pt}{2.800pt}}
\multiput(993.41,542.69)(0.511,-3.293){17}{\rule{0.123pt}{5.133pt}}
\multiput(990.34,553.35)(12.000,-63.346){2}{\rule{0.800pt}{2.567pt}}
\multiput(1005.41,472.57)(0.511,-2.660){17}{\rule{0.123pt}{4.200pt}}
\multiput(1002.34,481.28)(12.000,-51.283){2}{\rule{0.800pt}{2.100pt}}
\multiput(1017.41,417.67)(0.509,-1.815){19}{\rule{0.123pt}{2.969pt}}
\multiput(1014.34,423.84)(13.000,-38.837){2}{\rule{0.800pt}{1.485pt}}
\multiput(1030.41,374.48)(0.511,-1.530){17}{\rule{0.123pt}{2.533pt}}
\multiput(1027.34,379.74)(12.000,-29.742){2}{\rule{0.800pt}{1.267pt}}
\multiput(1042.41,341.42)(0.511,-1.214){17}{\rule{0.123pt}{2.067pt}}
\multiput(1039.34,345.71)(12.000,-23.711){2}{\rule{0.800pt}{1.033pt}}
\multiput(1054.41,315.08)(0.511,-0.943){17}{\rule{0.123pt}{1.667pt}}
\multiput(1051.34,318.54)(12.000,-18.541){2}{\rule{0.800pt}{0.833pt}}
\multiput(1066.41,294.57)(0.509,-0.698){19}{\rule{0.123pt}{1.308pt}}
\multiput(1063.34,297.29)(13.000,-15.286){2}{\rule{0.800pt}{0.654pt}}
\multiput(1079.41,277.30)(0.511,-0.581){17}{\rule{0.123pt}{1.133pt}}
\multiput(1076.34,279.65)(12.000,-11.648){2}{\rule{0.800pt}{0.567pt}}
\multiput(1091.41,263.57)(0.511,-0.536){17}{\rule{0.123pt}{1.067pt}}
\multiput(1088.34,265.79)(12.000,-10.786){2}{\rule{0.800pt}{0.533pt}}
\multiput(1102.00,253.08)(0.599,-0.514){13}{\rule{1.160pt}{0.124pt}}
\multiput(1102.00,253.34)(9.592,-10.000){2}{\rule{0.580pt}{0.800pt}}
\multiput(1114.00,243.08)(0.737,-0.516){11}{\rule{1.356pt}{0.124pt}}
\multiput(1114.00,243.34)(10.186,-9.000){2}{\rule{0.678pt}{0.800pt}}
\multiput(1127.00,234.08)(0.774,-0.520){9}{\rule{1.400pt}{0.125pt}}
\multiput(1127.00,234.34)(9.094,-8.000){2}{\rule{0.700pt}{0.800pt}}
\multiput(1139.00,226.08)(0.913,-0.526){7}{\rule{1.571pt}{0.127pt}}
\multiput(1139.00,226.34)(8.738,-7.000){2}{\rule{0.786pt}{0.800pt}}
\multiput(1151.00,219.07)(1.244,-0.536){5}{\rule{1.933pt}{0.129pt}}
\multiput(1151.00,219.34)(8.987,-6.000){2}{\rule{0.967pt}{0.800pt}}
\multiput(1164.00,213.06)(1.600,-0.560){3}{\rule{2.120pt}{0.135pt}}
\multiput(1164.00,213.34)(7.600,-5.000){2}{\rule{1.060pt}{0.800pt}}
\multiput(1176.00,208.06)(1.600,-0.560){3}{\rule{2.120pt}{0.135pt}}
\multiput(1176.00,208.34)(7.600,-5.000){2}{\rule{1.060pt}{0.800pt}}
\put(1188,201.34){\rule{2.600pt}{0.800pt}}
\multiput(1188.00,203.34)(6.604,-4.000){2}{\rule{1.300pt}{0.800pt}}
\put(1200,197.34){\rule{2.800pt}{0.800pt}}
\multiput(1200.00,199.34)(7.188,-4.000){2}{\rule{1.400pt}{0.800pt}}
\put(1213,193.84){\rule{2.891pt}{0.800pt}}
\multiput(1213.00,195.34)(6.000,-3.000){2}{\rule{1.445pt}{0.800pt}}
\put(1225,190.84){\rule{2.891pt}{0.800pt}}
\multiput(1225.00,192.34)(6.000,-3.000){2}{\rule{1.445pt}{0.800pt}}
\put(1237,187.84){\rule{2.891pt}{0.800pt}}
\multiput(1237.00,189.34)(6.000,-3.000){2}{\rule{1.445pt}{0.800pt}}
\put(1249,184.84){\rule{3.132pt}{0.800pt}}
\multiput(1249.00,186.34)(6.500,-3.000){2}{\rule{1.566pt}{0.800pt}}
\put(1262,182.34){\rule{2.891pt}{0.800pt}}
\multiput(1262.00,183.34)(6.000,-2.000){2}{\rule{1.445pt}{0.800pt}}
\put(1274,179.84){\rule{2.891pt}{0.800pt}}
\multiput(1274.00,181.34)(6.000,-3.000){2}{\rule{1.445pt}{0.800pt}}
\put(1286,177.34){\rule{2.891pt}{0.800pt}}
\multiput(1286.00,178.34)(6.000,-2.000){2}{\rule{1.445pt}{0.800pt}}
\put(1298,175.34){\rule{3.132pt}{0.800pt}}
\multiput(1298.00,176.34)(6.500,-2.000){2}{\rule{1.566pt}{0.800pt}}
\put(1311,173.34){\rule{2.891pt}{0.800pt}}
\multiput(1311.00,174.34)(6.000,-2.000){2}{\rule{1.445pt}{0.800pt}}
\put(1323,171.84){\rule{2.891pt}{0.800pt}}
\multiput(1323.00,172.34)(6.000,-1.000){2}{\rule{1.445pt}{0.800pt}}
\put(1335,170.34){\rule{2.891pt}{0.800pt}}
\multiput(1335.00,171.34)(6.000,-2.000){2}{\rule{1.445pt}{0.800pt}}
\put(1347,168.84){\rule{3.132pt}{0.800pt}}
\multiput(1347.00,169.34)(6.500,-1.000){2}{\rule{1.566pt}{0.800pt}}
\put(1360,167.34){\rule{2.891pt}{0.800pt}}
\multiput(1360.00,168.34)(6.000,-2.000){2}{\rule{1.445pt}{0.800pt}}
\put(1372,165.84){\rule{2.891pt}{0.800pt}}
\multiput(1372.00,166.34)(6.000,-1.000){2}{\rule{1.445pt}{0.800pt}}
\put(1384,164.84){\rule{2.891pt}{0.800pt}}
\multiput(1384.00,165.34)(6.000,-1.000){2}{\rule{1.445pt}{0.800pt}}
\put(1396,163.84){\rule{3.132pt}{0.800pt}}
\multiput(1396.00,164.34)(6.500,-1.000){2}{\rule{1.566pt}{0.800pt}}
\put(1409,162.84){\rule{2.891pt}{0.800pt}}
\multiput(1409.00,163.34)(6.000,-1.000){2}{\rule{1.445pt}{0.800pt}}
\put(501.0,189.0){\rule[-0.400pt]{3.132pt}{0.800pt}}
\sbox{\plotpoint}{\rule[-0.200pt]{0.400pt}{0.400pt}}%
\put(151.0,131.0){\rule[-0.200pt]{0.400pt}{175.375pt}}
\put(151.0,131.0){\rule[-0.200pt]{310.279pt}{0.400pt}}
\put(1439.0,131.0){\rule[-0.200pt]{0.400pt}{175.375pt}}
\put(151.0,859.0){\rule[-0.200pt]{310.279pt}{0.400pt}}
\end{picture}

\caption{Závislosť amplitúdy $B$ od frekvencia budiacej sily $\gamma$,
pre stredné a veľké závažie preložená, a 
pre stredné závažie funkciou $B_{\(\gamma\)} = \frac{1.55}{4.16\sqrt{\(12.23^2 - \gamma^2\)^2+4\cdot 0.5^2 \gamma^2 } }$}  \label{G_5}
\end{figure}


V grafe Obr. \ref{G_5} bola vynesená závislosť maximálnej amplitúdy $B$ na frekvencií budiacej sily $\gamma$, pre veľké závažie preložená závislosťou \ref{R_4}.

Frekvencia $\gamma$ bola vypočítaná pomocou kalibračného vzťahu \ref{R_V1} z napätia.
Maximum nastalo pre $\gamma ="25.28 Hz"$.




\section{Pohlovo kyvadlo}

Pre pohlové kyvadlo bez tlmenia boli namerané dáta vynesené do grafov napr. Obr. \ref{G_PB_1}.
Pre ostatné merania sa postupovalo rovnako a fitnuté parametre boli vynesené do tabuľky Tab. \ref{T_PB_1}

\begin{table}[h]

\begin{center}
\begin{tabular}{| c | c | c |}
\hline
\popi{\gamma}{Hz} & \popi{\omega}{rad\cdot s^{-1}} &  \popi{\omega_0}{rad\cdot s^{-1}} \\
\hline
$0.10\pm0.02$ & $3.70\pm0.2$ & $3.70\pm0.2$ \\
$0.04\pm0.01$ & $3.68\pm0.2$ & $3.68\pm0.2$ \\
$0.05\pm0.01$ & $3.68\pm0.2$ & $3.68\pm0.2$ \\
$0.03\pm0.01$ & $3.70\pm0.2$ & $3.70\pm0.2$ \\
$0.03\pm0.01$ & $3.70\pm0.2$ & $3.70\pm0.2$ \\
\hline

\end{tabular}
\caption{Namerané hodnoty výchylky $x$, na hmotnosti závažia $m$, vypočítaná tuhosť pružiny $k$ podľa vzťahu \ref{R_1} a vlastná uhlová frekvencia $\omega_0$ podľa vzťahu \ref{R_2} } \label{T_PB_1}
\end{center}
\end{table}

A z nich spočítaná podľa vzťahu \ref{V_PB_1} a vzťahu \ref{SCH_1} hodnota 
\eq{
\omega_0 = "\(3.69\pm0.01\) rad\cdot s^{-1}" \,.
}




\begin{figure}
% GNUPLOT: LaTeX picture
\setlength{\unitlength}{0.240900pt}
\ifx\plotpoint\undefined\newsavebox{\plotpoint}\fi
\sbox{\plotpoint}{\rule[-0.200pt]{0.400pt}{0.400pt}}%
\begin{picture}(1500,900)(0,0)
\sbox{\plotpoint}{\rule[-0.200pt]{0.400pt}{0.400pt}}%
\put(130.0,82.0){\rule[-0.200pt]{4.818pt}{0.400pt}}
\put(110,82){\makebox(0,0)[r]{-0.6}}
\put(1419.0,82.0){\rule[-0.200pt]{4.818pt}{0.400pt}}
\put(130.0,211.0){\rule[-0.200pt]{4.818pt}{0.400pt}}
\put(110,211){\makebox(0,0)[r]{-0.4}}
\put(1419.0,211.0){\rule[-0.200pt]{4.818pt}{0.400pt}}
\put(130.0,341.0){\rule[-0.200pt]{4.818pt}{0.400pt}}
\put(110,341){\makebox(0,0)[r]{-0.2}}
\put(1419.0,341.0){\rule[-0.200pt]{4.818pt}{0.400pt}}
\put(130.0,471.0){\rule[-0.200pt]{4.818pt}{0.400pt}}
\put(110,471){\makebox(0,0)[r]{ 0}}
\put(1419.0,471.0){\rule[-0.200pt]{4.818pt}{0.400pt}}
\put(130.0,600.0){\rule[-0.200pt]{4.818pt}{0.400pt}}
\put(110,600){\makebox(0,0)[r]{ 0.2}}
\put(1419.0,600.0){\rule[-0.200pt]{4.818pt}{0.400pt}}
\put(130.0,729.0){\rule[-0.200pt]{4.818pt}{0.400pt}}
\put(110,729){\makebox(0,0)[r]{ 0.4}}
\put(1419.0,729.0){\rule[-0.200pt]{4.818pt}{0.400pt}}
\put(130.0,859.0){\rule[-0.200pt]{4.818pt}{0.400pt}}
\put(110,859){\makebox(0,0)[r]{ 0.6}}
\put(1419.0,859.0){\rule[-0.200pt]{4.818pt}{0.400pt}}
\put(130.0,82.0){\rule[-0.200pt]{0.400pt}{4.818pt}}
\put(130,41){\makebox(0,0){ 5}}
\put(130.0,839.0){\rule[-0.200pt]{0.400pt}{4.818pt}}
\put(294.0,82.0){\rule[-0.200pt]{0.400pt}{4.818pt}}
\put(294,41){\makebox(0,0){ 5.5}}
\put(294.0,839.0){\rule[-0.200pt]{0.400pt}{4.818pt}}
\put(457.0,82.0){\rule[-0.200pt]{0.400pt}{4.818pt}}
\put(457,41){\makebox(0,0){ 6}}
\put(457.0,839.0){\rule[-0.200pt]{0.400pt}{4.818pt}}
\put(621.0,82.0){\rule[-0.200pt]{0.400pt}{4.818pt}}
\put(621,41){\makebox(0,0){ 6.5}}
\put(621.0,839.0){\rule[-0.200pt]{0.400pt}{4.818pt}}
\put(785.0,82.0){\rule[-0.200pt]{0.400pt}{4.818pt}}
\put(785,41){\makebox(0,0){ 7}}
\put(785.0,839.0){\rule[-0.200pt]{0.400pt}{4.818pt}}
\put(948.0,82.0){\rule[-0.200pt]{0.400pt}{4.818pt}}
\put(948,41){\makebox(0,0){ 7.5}}
\put(948.0,839.0){\rule[-0.200pt]{0.400pt}{4.818pt}}
\put(1112.0,82.0){\rule[-0.200pt]{0.400pt}{4.818pt}}
\put(1112,41){\makebox(0,0){ 8}}
\put(1112.0,839.0){\rule[-0.200pt]{0.400pt}{4.818pt}}
\put(1275.0,82.0){\rule[-0.200pt]{0.400pt}{4.818pt}}
\put(1275,41){\makebox(0,0){ 8.5}}
\put(1275.0,839.0){\rule[-0.200pt]{0.400pt}{4.818pt}}
\put(1439.0,82.0){\rule[-0.200pt]{0.400pt}{4.818pt}}
\put(1439,41){\makebox(0,0){ 9}}
\put(1439.0,839.0){\rule[-0.200pt]{0.400pt}{4.818pt}}
\put(130.0,82.0){\rule[-0.200pt]{0.400pt}{187.179pt}}
\put(130.0,82.0){\rule[-0.200pt]{315.338pt}{0.400pt}}
\put(1439.0,82.0){\rule[-0.200pt]{0.400pt}{187.179pt}}
\put(130.0,859.0){\rule[-0.200pt]{315.338pt}{0.400pt}}
\put(1279,819){\makebox(0,0)[r]{$x = 2.4\cdot e^{-0.033 t}\cdot sin \(3.6t-3.4\)$}}
\put(1299.0,819.0){\rule[-0.200pt]{24.090pt}{0.400pt}}
\put(130,128){\usebox{\plotpoint}}
\multiput(130.58,128.00)(0.493,0.774){23}{\rule{0.119pt}{0.715pt}}
\multiput(129.17,128.00)(13.000,18.515){2}{\rule{0.400pt}{0.358pt}}
\multiput(143.58,148.00)(0.493,1.052){23}{\rule{0.119pt}{0.931pt}}
\multiput(142.17,148.00)(13.000,25.068){2}{\rule{0.400pt}{0.465pt}}
\multiput(156.58,175.00)(0.494,1.195){25}{\rule{0.119pt}{1.043pt}}
\multiput(155.17,175.00)(14.000,30.835){2}{\rule{0.400pt}{0.521pt}}
\multiput(170.58,208.00)(0.493,1.488){23}{\rule{0.119pt}{1.269pt}}
\multiput(169.17,208.00)(13.000,35.366){2}{\rule{0.400pt}{0.635pt}}
\multiput(183.58,246.00)(0.493,1.686){23}{\rule{0.119pt}{1.423pt}}
\multiput(182.17,246.00)(13.000,40.046){2}{\rule{0.400pt}{0.712pt}}
\multiput(196.58,289.00)(0.493,1.845){23}{\rule{0.119pt}{1.546pt}}
\multiput(195.17,289.00)(13.000,43.791){2}{\rule{0.400pt}{0.773pt}}
\multiput(209.58,336.00)(0.494,1.819){25}{\rule{0.119pt}{1.529pt}}
\multiput(208.17,336.00)(14.000,46.827){2}{\rule{0.400pt}{0.764pt}}
\multiput(223.58,386.00)(0.493,2.003){23}{\rule{0.119pt}{1.669pt}}
\multiput(222.17,386.00)(13.000,47.535){2}{\rule{0.400pt}{0.835pt}}
\multiput(236.58,437.00)(0.493,2.003){23}{\rule{0.119pt}{1.669pt}}
\multiput(235.17,437.00)(13.000,47.535){2}{\rule{0.400pt}{0.835pt}}
\multiput(249.58,488.00)(0.493,1.964){23}{\rule{0.119pt}{1.638pt}}
\multiput(248.17,488.00)(13.000,46.599){2}{\rule{0.400pt}{0.819pt}}
\multiput(262.58,538.00)(0.493,1.924){23}{\rule{0.119pt}{1.608pt}}
\multiput(261.17,538.00)(13.000,45.663){2}{\rule{0.400pt}{0.804pt}}
\multiput(275.58,587.00)(0.494,1.635){25}{\rule{0.119pt}{1.386pt}}
\multiput(274.17,587.00)(14.000,42.124){2}{\rule{0.400pt}{0.693pt}}
\multiput(289.58,632.00)(0.493,1.646){23}{\rule{0.119pt}{1.392pt}}
\multiput(288.17,632.00)(13.000,39.110){2}{\rule{0.400pt}{0.696pt}}
\multiput(302.58,674.00)(0.493,1.448){23}{\rule{0.119pt}{1.238pt}}
\multiput(301.17,674.00)(13.000,34.430){2}{\rule{0.400pt}{0.619pt}}
\multiput(315.58,711.00)(0.493,1.210){23}{\rule{0.119pt}{1.054pt}}
\multiput(314.17,711.00)(13.000,28.813){2}{\rule{0.400pt}{0.527pt}}
\multiput(328.58,742.00)(0.494,0.864){25}{\rule{0.119pt}{0.786pt}}
\multiput(327.17,742.00)(14.000,22.369){2}{\rule{0.400pt}{0.393pt}}
\multiput(342.58,766.00)(0.493,0.695){23}{\rule{0.119pt}{0.654pt}}
\multiput(341.17,766.00)(13.000,16.643){2}{\rule{0.400pt}{0.327pt}}
\multiput(355.00,784.58)(0.590,0.492){19}{\rule{0.573pt}{0.118pt}}
\multiput(355.00,783.17)(11.811,11.000){2}{\rule{0.286pt}{0.400pt}}
\multiput(368.00,795.60)(1.797,0.468){5}{\rule{1.400pt}{0.113pt}}
\multiput(368.00,794.17)(10.094,4.000){2}{\rule{0.700pt}{0.400pt}}
\multiput(381.00,797.94)(1.797,-0.468){5}{\rule{1.400pt}{0.113pt}}
\multiput(381.00,798.17)(10.094,-4.000){2}{\rule{0.700pt}{0.400pt}}
\multiput(394.00,793.92)(0.637,-0.492){19}{\rule{0.609pt}{0.118pt}}
\multiput(394.00,794.17)(12.736,-11.000){2}{\rule{0.305pt}{0.400pt}}
\multiput(408.58,781.29)(0.493,-0.695){23}{\rule{0.119pt}{0.654pt}}
\multiput(407.17,782.64)(13.000,-16.643){2}{\rule{0.400pt}{0.327pt}}
\multiput(421.58,762.39)(0.493,-0.972){23}{\rule{0.119pt}{0.869pt}}
\multiput(420.17,764.20)(13.000,-23.196){2}{\rule{0.400pt}{0.435pt}}
\multiput(434.58,736.75)(0.493,-1.171){23}{\rule{0.119pt}{1.023pt}}
\multiput(433.17,738.88)(13.000,-27.877){2}{\rule{0.400pt}{0.512pt}}
\multiput(447.58,706.43)(0.494,-1.268){25}{\rule{0.119pt}{1.100pt}}
\multiput(446.17,708.72)(14.000,-32.717){2}{\rule{0.400pt}{0.550pt}}
\multiput(461.58,670.60)(0.493,-1.527){23}{\rule{0.119pt}{1.300pt}}
\multiput(460.17,673.30)(13.000,-36.302){2}{\rule{0.400pt}{0.650pt}}
\multiput(474.58,631.09)(0.493,-1.686){23}{\rule{0.119pt}{1.423pt}}
\multiput(473.17,634.05)(13.000,-40.046){2}{\rule{0.400pt}{0.712pt}}
\multiput(487.58,587.71)(0.493,-1.805){23}{\rule{0.119pt}{1.515pt}}
\multiput(486.17,590.85)(13.000,-42.855){2}{\rule{0.400pt}{0.758pt}}
\multiput(500.58,541.71)(0.493,-1.805){23}{\rule{0.119pt}{1.515pt}}
\multiput(499.17,544.85)(13.000,-42.855){2}{\rule{0.400pt}{0.758pt}}
\multiput(513.58,496.01)(0.494,-1.709){25}{\rule{0.119pt}{1.443pt}}
\multiput(512.17,499.01)(14.000,-44.005){2}{\rule{0.400pt}{0.721pt}}
\multiput(527.58,448.71)(0.493,-1.805){23}{\rule{0.119pt}{1.515pt}}
\multiput(526.17,451.85)(13.000,-42.855){2}{\rule{0.400pt}{0.758pt}}
\multiput(540.58,402.84)(0.493,-1.765){23}{\rule{0.119pt}{1.485pt}}
\multiput(539.17,405.92)(13.000,-41.919){2}{\rule{0.400pt}{0.742pt}}
\multiput(553.58,358.35)(0.493,-1.607){23}{\rule{0.119pt}{1.362pt}}
\multiput(552.17,361.17)(13.000,-38.174){2}{\rule{0.400pt}{0.681pt}}
\multiput(566.58,318.08)(0.494,-1.378){25}{\rule{0.119pt}{1.186pt}}
\multiput(565.17,320.54)(14.000,-35.539){2}{\rule{0.400pt}{0.593pt}}
\multiput(580.58,280.24)(0.493,-1.329){23}{\rule{0.119pt}{1.146pt}}
\multiput(579.17,282.62)(13.000,-31.621){2}{\rule{0.400pt}{0.573pt}}
\multiput(593.58,247.01)(0.493,-1.091){23}{\rule{0.119pt}{0.962pt}}
\multiput(592.17,249.00)(13.000,-26.004){2}{\rule{0.400pt}{0.481pt}}
\multiput(606.58,219.65)(0.493,-0.893){23}{\rule{0.119pt}{0.808pt}}
\multiput(605.17,221.32)(13.000,-21.324){2}{\rule{0.400pt}{0.404pt}}
\multiput(619.58,197.54)(0.493,-0.616){23}{\rule{0.119pt}{0.592pt}}
\multiput(618.17,198.77)(13.000,-14.771){2}{\rule{0.400pt}{0.296pt}}
\multiput(632.00,182.92)(0.704,-0.491){17}{\rule{0.660pt}{0.118pt}}
\multiput(632.00,183.17)(12.630,-10.000){2}{\rule{0.330pt}{0.400pt}}
\multiput(646.00,172.94)(1.797,-0.468){5}{\rule{1.400pt}{0.113pt}}
\multiput(646.00,173.17)(10.094,-4.000){2}{\rule{0.700pt}{0.400pt}}
\multiput(659.00,170.60)(1.797,0.468){5}{\rule{1.400pt}{0.113pt}}
\multiput(659.00,169.17)(10.094,4.000){2}{\rule{0.700pt}{0.400pt}}
\multiput(672.00,174.58)(0.652,0.491){17}{\rule{0.620pt}{0.118pt}}
\multiput(672.00,173.17)(11.713,10.000){2}{\rule{0.310pt}{0.400pt}}
\multiput(685.58,184.00)(0.494,0.570){25}{\rule{0.119pt}{0.557pt}}
\multiput(684.17,184.00)(14.000,14.844){2}{\rule{0.400pt}{0.279pt}}
\multiput(699.58,200.00)(0.493,0.853){23}{\rule{0.119pt}{0.777pt}}
\multiput(698.17,200.00)(13.000,20.387){2}{\rule{0.400pt}{0.388pt}}
\multiput(712.58,222.00)(0.493,1.091){23}{\rule{0.119pt}{0.962pt}}
\multiput(711.17,222.00)(13.000,26.004){2}{\rule{0.400pt}{0.481pt}}
\multiput(725.58,250.00)(0.493,1.250){23}{\rule{0.119pt}{1.085pt}}
\multiput(724.17,250.00)(13.000,29.749){2}{\rule{0.400pt}{0.542pt}}
\multiput(738.58,282.00)(0.493,1.408){23}{\rule{0.119pt}{1.208pt}}
\multiput(737.17,282.00)(13.000,33.493){2}{\rule{0.400pt}{0.604pt}}
\multiput(751.58,318.00)(0.494,1.452){25}{\rule{0.119pt}{1.243pt}}
\multiput(750.17,318.00)(14.000,37.420){2}{\rule{0.400pt}{0.621pt}}
\multiput(765.58,358.00)(0.493,1.607){23}{\rule{0.119pt}{1.362pt}}
\multiput(764.17,358.00)(13.000,38.174){2}{\rule{0.400pt}{0.681pt}}
\multiput(778.58,399.00)(0.493,1.686){23}{\rule{0.119pt}{1.423pt}}
\multiput(777.17,399.00)(13.000,40.046){2}{\rule{0.400pt}{0.712pt}}
\multiput(791.58,442.00)(0.493,1.686){23}{\rule{0.119pt}{1.423pt}}
\multiput(790.17,442.00)(13.000,40.046){2}{\rule{0.400pt}{0.712pt}}
\multiput(804.58,485.00)(0.494,1.525){25}{\rule{0.119pt}{1.300pt}}
\multiput(803.17,485.00)(14.000,39.302){2}{\rule{0.400pt}{0.650pt}}
\multiput(818.58,527.00)(0.493,1.567){23}{\rule{0.119pt}{1.331pt}}
\multiput(817.17,527.00)(13.000,37.238){2}{\rule{0.400pt}{0.665pt}}
\multiput(831.58,567.00)(0.493,1.527){23}{\rule{0.119pt}{1.300pt}}
\multiput(830.17,567.00)(13.000,36.302){2}{\rule{0.400pt}{0.650pt}}
\multiput(844.58,606.00)(0.493,1.329){23}{\rule{0.119pt}{1.146pt}}
\multiput(843.17,606.00)(13.000,31.621){2}{\rule{0.400pt}{0.573pt}}
\multiput(857.58,640.00)(0.493,1.210){23}{\rule{0.119pt}{1.054pt}}
\multiput(856.17,640.00)(13.000,28.813){2}{\rule{0.400pt}{0.527pt}}
\multiput(870.58,671.00)(0.494,0.938){25}{\rule{0.119pt}{0.843pt}}
\multiput(869.17,671.00)(14.000,24.251){2}{\rule{0.400pt}{0.421pt}}
\multiput(884.58,697.00)(0.493,0.814){23}{\rule{0.119pt}{0.746pt}}
\multiput(883.17,697.00)(13.000,19.451){2}{\rule{0.400pt}{0.373pt}}
\multiput(897.58,718.00)(0.493,0.576){23}{\rule{0.119pt}{0.562pt}}
\multiput(896.17,718.00)(13.000,13.834){2}{\rule{0.400pt}{0.281pt}}
\multiput(910.00,733.59)(0.728,0.489){15}{\rule{0.678pt}{0.118pt}}
\multiput(910.00,732.17)(11.593,9.000){2}{\rule{0.339pt}{0.400pt}}
\multiput(923.00,742.61)(2.918,0.447){3}{\rule{1.967pt}{0.108pt}}
\multiput(923.00,741.17)(9.918,3.000){2}{\rule{0.983pt}{0.400pt}}
\multiput(937.00,743.95)(2.695,-0.447){3}{\rule{1.833pt}{0.108pt}}
\multiput(937.00,744.17)(9.195,-3.000){2}{\rule{0.917pt}{0.400pt}}
\multiput(950.00,740.93)(0.728,-0.489){15}{\rule{0.678pt}{0.118pt}}
\multiput(950.00,741.17)(11.593,-9.000){2}{\rule{0.339pt}{0.400pt}}
\multiput(963.58,730.67)(0.493,-0.576){23}{\rule{0.119pt}{0.562pt}}
\multiput(962.17,731.83)(13.000,-13.834){2}{\rule{0.400pt}{0.281pt}}
\multiput(976.58,715.03)(0.493,-0.774){23}{\rule{0.119pt}{0.715pt}}
\multiput(975.17,716.52)(13.000,-18.515){2}{\rule{0.400pt}{0.358pt}}
\multiput(989.58,694.50)(0.494,-0.938){25}{\rule{0.119pt}{0.843pt}}
\multiput(988.17,696.25)(14.000,-24.251){2}{\rule{0.400pt}{0.421pt}}
\multiput(1003.58,667.88)(0.493,-1.131){23}{\rule{0.119pt}{0.992pt}}
\multiput(1002.17,669.94)(13.000,-26.940){2}{\rule{0.400pt}{0.496pt}}
\multiput(1016.58,638.37)(0.493,-1.290){23}{\rule{0.119pt}{1.115pt}}
\multiput(1015.17,640.68)(13.000,-30.685){2}{\rule{0.400pt}{0.558pt}}
\multiput(1029.58,604.99)(0.493,-1.408){23}{\rule{0.119pt}{1.208pt}}
\multiput(1028.17,607.49)(13.000,-33.493){2}{\rule{0.400pt}{0.604pt}}
\multiput(1042.58,569.08)(0.494,-1.378){25}{\rule{0.119pt}{1.186pt}}
\multiput(1041.17,571.54)(14.000,-35.539){2}{\rule{0.400pt}{0.593pt}}
\multiput(1056.58,530.60)(0.493,-1.527){23}{\rule{0.119pt}{1.300pt}}
\multiput(1055.17,533.30)(13.000,-36.302){2}{\rule{0.400pt}{0.650pt}}
\multiput(1069.58,491.60)(0.493,-1.527){23}{\rule{0.119pt}{1.300pt}}
\multiput(1068.17,494.30)(13.000,-36.302){2}{\rule{0.400pt}{0.650pt}}
\multiput(1082.58,452.60)(0.493,-1.527){23}{\rule{0.119pt}{1.300pt}}
\multiput(1081.17,455.30)(13.000,-36.302){2}{\rule{0.400pt}{0.650pt}}
\multiput(1095.58,413.86)(0.493,-1.448){23}{\rule{0.119pt}{1.238pt}}
\multiput(1094.17,416.43)(13.000,-34.430){2}{\rule{0.400pt}{0.619pt}}
\multiput(1108.58,377.43)(0.494,-1.268){25}{\rule{0.119pt}{1.100pt}}
\multiput(1107.17,379.72)(14.000,-32.717){2}{\rule{0.400pt}{0.550pt}}
\multiput(1122.58,342.50)(0.493,-1.250){23}{\rule{0.119pt}{1.085pt}}
\multiput(1121.17,344.75)(13.000,-29.749){2}{\rule{0.400pt}{0.542pt}}
\multiput(1135.58,311.01)(0.493,-1.091){23}{\rule{0.119pt}{0.962pt}}
\multiput(1134.17,313.00)(13.000,-26.004){2}{\rule{0.400pt}{0.481pt}}
\multiput(1148.58,283.52)(0.493,-0.933){23}{\rule{0.119pt}{0.838pt}}
\multiput(1147.17,285.26)(13.000,-22.260){2}{\rule{0.400pt}{0.419pt}}
\multiput(1161.58,260.33)(0.494,-0.680){25}{\rule{0.119pt}{0.643pt}}
\multiput(1160.17,261.67)(14.000,-17.666){2}{\rule{0.400pt}{0.321pt}}
\multiput(1175.00,242.92)(0.497,-0.493){23}{\rule{0.500pt}{0.119pt}}
\multiput(1175.00,243.17)(11.962,-13.000){2}{\rule{0.250pt}{0.400pt}}
\multiput(1188.00,229.93)(0.728,-0.489){15}{\rule{0.678pt}{0.118pt}}
\multiput(1188.00,230.17)(11.593,-9.000){2}{\rule{0.339pt}{0.400pt}}
\multiput(1201.00,220.95)(2.695,-0.447){3}{\rule{1.833pt}{0.108pt}}
\multiput(1201.00,221.17)(9.195,-3.000){2}{\rule{0.917pt}{0.400pt}}
\multiput(1214.00,219.61)(2.695,0.447){3}{\rule{1.833pt}{0.108pt}}
\multiput(1214.00,218.17)(9.195,3.000){2}{\rule{0.917pt}{0.400pt}}
\multiput(1227.00,222.59)(0.786,0.489){15}{\rule{0.722pt}{0.118pt}}
\multiput(1227.00,221.17)(12.501,9.000){2}{\rule{0.361pt}{0.400pt}}
\multiput(1241.00,231.58)(0.497,0.493){23}{\rule{0.500pt}{0.119pt}}
\multiput(1241.00,230.17)(11.962,13.000){2}{\rule{0.250pt}{0.400pt}}
\multiput(1254.58,244.00)(0.493,0.734){23}{\rule{0.119pt}{0.685pt}}
\multiput(1253.17,244.00)(13.000,17.579){2}{\rule{0.400pt}{0.342pt}}
\multiput(1267.58,263.00)(0.493,0.893){23}{\rule{0.119pt}{0.808pt}}
\multiput(1266.17,263.00)(13.000,21.324){2}{\rule{0.400pt}{0.404pt}}
\multiput(1280.58,286.00)(0.494,0.974){25}{\rule{0.119pt}{0.871pt}}
\multiput(1279.17,286.00)(14.000,25.191){2}{\rule{0.400pt}{0.436pt}}
\multiput(1294.58,313.00)(0.493,1.171){23}{\rule{0.119pt}{1.023pt}}
\multiput(1293.17,313.00)(13.000,27.877){2}{\rule{0.400pt}{0.512pt}}
\multiput(1307.58,343.00)(0.493,1.290){23}{\rule{0.119pt}{1.115pt}}
\multiput(1306.17,343.00)(13.000,30.685){2}{\rule{0.400pt}{0.558pt}}
\multiput(1320.58,376.00)(0.493,1.329){23}{\rule{0.119pt}{1.146pt}}
\multiput(1319.17,376.00)(13.000,31.621){2}{\rule{0.400pt}{0.573pt}}
\multiput(1333.58,410.00)(0.493,1.408){23}{\rule{0.119pt}{1.208pt}}
\multiput(1332.17,410.00)(13.000,33.493){2}{\rule{0.400pt}{0.604pt}}
\multiput(1346.58,446.00)(0.494,1.305){25}{\rule{0.119pt}{1.129pt}}
\multiput(1345.17,446.00)(14.000,33.658){2}{\rule{0.400pt}{0.564pt}}
\multiput(1360.58,482.00)(0.493,1.369){23}{\rule{0.119pt}{1.177pt}}
\multiput(1359.17,482.00)(13.000,32.557){2}{\rule{0.400pt}{0.588pt}}
\multiput(1373.58,517.00)(0.493,1.329){23}{\rule{0.119pt}{1.146pt}}
\multiput(1372.17,517.00)(13.000,31.621){2}{\rule{0.400pt}{0.573pt}}
\multiput(1386.58,551.00)(0.493,1.250){23}{\rule{0.119pt}{1.085pt}}
\multiput(1385.17,551.00)(13.000,29.749){2}{\rule{0.400pt}{0.542pt}}
\multiput(1399.58,583.00)(0.494,1.048){25}{\rule{0.119pt}{0.929pt}}
\multiput(1398.17,583.00)(14.000,27.073){2}{\rule{0.400pt}{0.464pt}}
\multiput(1413.58,612.00)(0.493,1.012){23}{\rule{0.119pt}{0.900pt}}
\multiput(1412.17,612.00)(13.000,24.132){2}{\rule{0.400pt}{0.450pt}}
\multiput(1426.58,638.00)(0.493,0.853){23}{\rule{0.119pt}{0.777pt}}
\multiput(1425.17,638.00)(13.000,20.387){2}{\rule{0.400pt}{0.388pt}}
\put(1279,778){\makebox(0,0)[r]{namerané dáta}}
\put(130,129){\makebox(0,0){$\times$}}
\put(138,141){\makebox(0,0){$\times$}}
\put(146,154){\makebox(0,0){$\times$}}
\put(155,171){\makebox(0,0){$\times$}}
\put(163,191){\makebox(0,0){$\times$}}
\put(171,211){\makebox(0,0){$\times$}}
\put(179,235){\makebox(0,0){$\times$}}
\put(187,260){\makebox(0,0){$\times$}}
\put(195,287){\makebox(0,0){$\times$}}
\put(204,316){\makebox(0,0){$\times$}}
\put(212,346){\makebox(0,0){$\times$}}
\put(220,376){\makebox(0,0){$\times$}}
\put(228,406){\makebox(0,0){$\times$}}
\put(236,439){\makebox(0,0){$\times$}}
\put(245,471){\makebox(0,0){$\times$}}
\put(253,502){\makebox(0,0){$\times$}}
\put(261,533){\makebox(0,0){$\times$}}
\put(269,563){\makebox(0,0){$\times$}}
\put(277,592){\makebox(0,0){$\times$}}
\put(285,621){\makebox(0,0){$\times$}}
\put(294,649){\makebox(0,0){$\times$}}
\put(302,674){\makebox(0,0){$\times$}}
\put(310,697){\makebox(0,0){$\times$}}
\put(318,718){\makebox(0,0){$\times$}}
\put(326,738){\makebox(0,0){$\times$}}
\put(335,754){\makebox(0,0){$\times$}}
\put(343,768){\makebox(0,0){$\times$}}
\put(351,781){\makebox(0,0){$\times$}}
\put(359,789){\makebox(0,0){$\times$}}
\put(367,795){\makebox(0,0){$\times$}}
\put(375,798){\makebox(0,0){$\times$}}
\put(384,798){\makebox(0,0){$\times$}}
\put(392,796){\makebox(0,0){$\times$}}
\put(400,790){\makebox(0,0){$\times$}}
\put(408,782){\makebox(0,0){$\times$}}
\put(416,772){\makebox(0,0){$\times$}}
\put(425,759){\makebox(0,0){$\times$}}
\put(433,743){\makebox(0,0){$\times$}}
\put(441,725){\makebox(0,0){$\times$}}
\put(449,704){\makebox(0,0){$\times$}}
\put(457,683){\makebox(0,0){$\times$}}
\put(465,660){\makebox(0,0){$\times$}}
\put(474,634){\makebox(0,0){$\times$}}
\put(482,609){\makebox(0,0){$\times$}}
\put(490,581){\makebox(0,0){$\times$}}
\put(498,553){\makebox(0,0){$\times$}}
\put(506,524){\makebox(0,0){$\times$}}
\put(515,496){\makebox(0,0){$\times$}}
\put(523,467){\makebox(0,0){$\times$}}
\put(531,439){\makebox(0,0){$\times$}}
\put(539,410){\makebox(0,0){$\times$}}
\put(547,382){\makebox(0,0){$\times$}}
\put(555,355){\makebox(0,0){$\times$}}
\put(564,329){\makebox(0,0){$\times$}}
\put(572,305){\makebox(0,0){$\times$}}
\put(580,282){\makebox(0,0){$\times$}}
\put(588,261){\makebox(0,0){$\times$}}
\put(596,242){\makebox(0,0){$\times$}}
\put(605,223){\makebox(0,0){$\times$}}
\put(613,209){\makebox(0,0){$\times$}}
\put(621,196){\makebox(0,0){$\times$}}
\put(629,186){\makebox(0,0){$\times$}}
\put(637,178){\makebox(0,0){$\times$}}
\put(645,171){\makebox(0,0){$\times$}}
\put(654,169){\makebox(0,0){$\times$}}
\put(662,168){\makebox(0,0){$\times$}}
\put(670,171){\makebox(0,0){$\times$}}
\put(678,175){\makebox(0,0){$\times$}}
\put(686,182){\makebox(0,0){$\times$}}
\put(695,192){\makebox(0,0){$\times$}}
\put(703,204){\makebox(0,0){$\times$}}
\put(711,218){\makebox(0,0){$\times$}}
\put(719,235){\makebox(0,0){$\times$}}
\put(727,252){\makebox(0,0){$\times$}}
\put(735,272){\makebox(0,0){$\times$}}
\put(744,294){\makebox(0,0){$\times$}}
\put(752,316){\makebox(0,0){$\times$}}
\put(760,340){\makebox(0,0){$\times$}}
\put(768,366){\makebox(0,0){$\times$}}
\put(776,391){\makebox(0,0){$\times$}}
\put(785,417){\makebox(0,0){$\times$}}
\put(793,443){\makebox(0,0){$\times$}}
\put(801,471){\makebox(0,0){$\times$}}
\put(809,497){\makebox(0,0){$\times$}}
\put(817,524){\makebox(0,0){$\times$}}
\put(825,549){\makebox(0,0){$\times$}}
\put(834,573){\makebox(0,0){$\times$}}
\put(842,597){\makebox(0,0){$\times$}}
\put(850,619){\makebox(0,0){$\times$}}
\put(858,640){\makebox(0,0){$\times$}}
\put(866,660){\makebox(0,0){$\times$}}
\put(874,678){\makebox(0,0){$\times$}}
\put(883,695){\makebox(0,0){$\times$}}
\put(891,707){\makebox(0,0){$\times$}}
\put(899,720){\makebox(0,0){$\times$}}
\put(907,729){\makebox(0,0){$\times$}}
\put(915,736){\makebox(0,0){$\times$}}
\put(924,741){\makebox(0,0){$\times$}}
\put(932,744){\makebox(0,0){$\times$}}
\put(940,744){\makebox(0,0){$\times$}}
\put(948,743){\makebox(0,0){$\times$}}
\put(956,738){\makebox(0,0){$\times$}}
\put(964,731){\makebox(0,0){$\times$}}
\put(973,723){\makebox(0,0){$\times$}}
\put(981,711){\makebox(0,0){$\times$}}
\put(989,698){\makebox(0,0){$\times$}}
\put(997,683){\makebox(0,0){$\times$}}
\put(1005,667){\makebox(0,0){$\times$}}
\put(1014,649){\makebox(0,0){$\times$}}
\put(1022,629){\makebox(0,0){$\times$}}
\put(1030,608){\makebox(0,0){$\times$}}
\put(1038,586){\makebox(0,0){$\times$}}
\put(1046,563){\makebox(0,0){$\times$}}
\put(1054,540){\makebox(0,0){$\times$}}
\put(1063,516){\makebox(0,0){$\times$}}
\put(1071,492){\makebox(0,0){$\times$}}
\put(1079,467){\makebox(0,0){$\times$}}
\put(1087,444){\makebox(0,0){$\times$}}
\put(1095,421){\makebox(0,0){$\times$}}
\put(1104,397){\makebox(0,0){$\times$}}
\put(1112,375){\makebox(0,0){$\times$}}
\put(1120,353){\makebox(0,0){$\times$}}
\put(1128,333){\makebox(0,0){$\times$}}
\put(1136,314){\makebox(0,0){$\times$}}
\put(1144,296){\makebox(0,0){$\times$}}
\put(1153,280){\makebox(0,0){$\times$}}
\put(1161,266){\makebox(0,0){$\times$}}
\put(1169,253){\makebox(0,0){$\times$}}
\put(1177,243){\makebox(0,0){$\times$}}
\put(1185,235){\makebox(0,0){$\times$}}
\put(1194,228){\makebox(0,0){$\times$}}
\put(1202,223){\makebox(0,0){$\times$}}
\put(1210,221){\makebox(0,0){$\times$}}
\put(1218,221){\makebox(0,0){$\times$}}
\put(1226,223){\makebox(0,0){$\times$}}
\put(1234,226){\makebox(0,0){$\times$}}
\put(1243,232){\makebox(0,0){$\times$}}
\put(1251,242){\makebox(0,0){$\times$}}
\put(1259,252){\makebox(0,0){$\times$}}
\put(1267,264){\makebox(0,0){$\times$}}
\put(1275,278){\makebox(0,0){$\times$}}
\put(1284,292){\makebox(0,0){$\times$}}
\put(1292,310){\makebox(0,0){$\times$}}
\put(1300,328){\makebox(0,0){$\times$}}
\put(1308,347){\makebox(0,0){$\times$}}
\put(1316,368){\makebox(0,0){$\times$}}
\put(1324,388){\makebox(0,0){$\times$}}
\put(1333,410){\makebox(0,0){$\times$}}
\put(1341,432){\makebox(0,0){$\times$}}
\put(1349,454){\makebox(0,0){$\times$}}
\put(1357,476){\makebox(0,0){$\times$}}
\put(1365,499){\makebox(0,0){$\times$}}
\put(1374,522){\makebox(0,0){$\times$}}
\put(1382,543){\makebox(0,0){$\times$}}
\put(1390,563){\makebox(0,0){$\times$}}
\put(1398,583){\makebox(0,0){$\times$}}
\put(1406,602){\makebox(0,0){$\times$}}
\put(1414,619){\makebox(0,0){$\times$}}
\put(1423,635){\makebox(0,0){$\times$}}
\put(1431,651){\makebox(0,0){$\times$}}
\put(1439,663){\makebox(0,0){$\times$}}
\put(1349,778){\makebox(0,0){$\times$}}
\put(130.0,82.0){\rule[-0.200pt]{0.400pt}{187.179pt}}
\put(130.0,82.0){\rule[-0.200pt]{315.338pt}{0.400pt}}
\put(1439.0,82.0){\rule[-0.200pt]{0.400pt}{187.179pt}}
\put(130.0,859.0){\rule[-0.200pt]{315.338pt}{0.400pt}}
\end{picture}

\caption{Namerané hodnoty závislosti výchylky $x$ na čase $t$, preložené závislosťou $x = 2.4\cdot e^{-0.033 t}\cdot sin \(3.6t-3.4\)$}  \label{G_PB_1}
\end{figure}



Pre menovité prúdy $I= "0.5 A"$, $I= "1 A"$ a $I= "1.5 A"$ boli namerané dáta vynesené postupne do grafov postupn \ref{G_PT_.5}, \ref{G_PT_1} a \ref{G_PT_1.5}.
A z fitov získané postupne hodnoty $\gamma = \{ 0.17 , 0.40 , 0.90\}" Hz"$.


\subsection{Polohová podmienka}

Pre polohovú podmienku boli boli pre jednotlivé prúdy zaznamenané výchylky kyvadla na čase a vynesené do grafu Obr. \ref{G_7}.
Z grafu je vidieť že kritický útlm nastáva pre prúd $I= "\(2.6-3\pm0.2\) A"$.


\begin{figure}
% GNUPLOT: LaTeX picture
\setlength{\unitlength}{0.240900pt}
\ifx\plotpoint\undefined\newsavebox{\plotpoint}\fi
\sbox{\plotpoint}{\rule[-0.200pt]{0.400pt}{0.400pt}}%
\begin{picture}(1500,900)(0,0)
\sbox{\plotpoint}{\rule[-0.200pt]{0.400pt}{0.400pt}}%
\put(171.0,131.0){\rule[-0.200pt]{4.818pt}{0.400pt}}
\put(151,131){\makebox(0,0)[r]{-1.5}}
\put(1419.0,131.0){\rule[-0.200pt]{4.818pt}{0.400pt}}
\put(171.0,252.0){\rule[-0.200pt]{4.818pt}{0.400pt}}
\put(151,252){\makebox(0,0)[r]{-1}}
\put(1419.0,252.0){\rule[-0.200pt]{4.818pt}{0.400pt}}
\put(171.0,374.0){\rule[-0.200pt]{4.818pt}{0.400pt}}
\put(151,374){\makebox(0,0)[r]{-0.5}}
\put(1419.0,374.0){\rule[-0.200pt]{4.818pt}{0.400pt}}
\put(171.0,495.0){\rule[-0.200pt]{4.818pt}{0.400pt}}
\put(151,495){\makebox(0,0)[r]{ 0}}
\put(1419.0,495.0){\rule[-0.200pt]{4.818pt}{0.400pt}}
\put(171.0,616.0){\rule[-0.200pt]{4.818pt}{0.400pt}}
\put(151,616){\makebox(0,0)[r]{ 0.5}}
\put(1419.0,616.0){\rule[-0.200pt]{4.818pt}{0.400pt}}
\put(171.0,738.0){\rule[-0.200pt]{4.818pt}{0.400pt}}
\put(151,738){\makebox(0,0)[r]{ 1}}
\put(1419.0,738.0){\rule[-0.200pt]{4.818pt}{0.400pt}}
\put(171.0,859.0){\rule[-0.200pt]{4.818pt}{0.400pt}}
\put(151,859){\makebox(0,0)[r]{ 1.5}}
\put(1419.0,859.0){\rule[-0.200pt]{4.818pt}{0.400pt}}
\put(171.0,131.0){\rule[-0.200pt]{0.400pt}{4.818pt}}
\put(171,90){\makebox(0,0){ 2}}
\put(171.0,839.0){\rule[-0.200pt]{0.400pt}{4.818pt}}
\put(382.0,131.0){\rule[-0.200pt]{0.400pt}{4.818pt}}
\put(382,90){\makebox(0,0){ 2.5}}
\put(382.0,839.0){\rule[-0.200pt]{0.400pt}{4.818pt}}
\put(594.0,131.0){\rule[-0.200pt]{0.400pt}{4.818pt}}
\put(594,90){\makebox(0,0){ 3}}
\put(594.0,839.0){\rule[-0.200pt]{0.400pt}{4.818pt}}
\put(805.0,131.0){\rule[-0.200pt]{0.400pt}{4.818pt}}
\put(805,90){\makebox(0,0){ 3.5}}
\put(805.0,839.0){\rule[-0.200pt]{0.400pt}{4.818pt}}
\put(1016.0,131.0){\rule[-0.200pt]{0.400pt}{4.818pt}}
\put(1016,90){\makebox(0,0){ 4}}
\put(1016.0,839.0){\rule[-0.200pt]{0.400pt}{4.818pt}}
\put(1228.0,131.0){\rule[-0.200pt]{0.400pt}{4.818pt}}
\put(1228,90){\makebox(0,0){ 4.5}}
\put(1228.0,839.0){\rule[-0.200pt]{0.400pt}{4.818pt}}
\put(1439.0,131.0){\rule[-0.200pt]{0.400pt}{4.818pt}}
\put(1439,90){\makebox(0,0){ 5}}
\put(1439.0,839.0){\rule[-0.200pt]{0.400pt}{4.818pt}}
\put(171.0,131.0){\rule[-0.200pt]{0.400pt}{175.375pt}}
\put(171.0,131.0){\rule[-0.200pt]{305.461pt}{0.400pt}}
\put(1439.0,131.0){\rule[-0.200pt]{0.400pt}{175.375pt}}
\put(171.0,859.0){\rule[-0.200pt]{305.461pt}{0.400pt}}
\put(30,495){\makebox(0,0){\popi{x}{mm}}}
\put(805,29){\makebox(0,0){\popi{t}{s}}}
\put(1279,213){\makebox(0,0)[r]{namerané hodnoty}}
\put(171,795){\makebox(0,0){$+$}}
\put(182,794){\makebox(0,0){$+$}}
\put(192,790){\makebox(0,0){$+$}}
\put(203,783){\makebox(0,0){$+$}}
\put(213,774){\makebox(0,0){$+$}}
\put(224,763){\makebox(0,0){$+$}}
\put(234,750){\makebox(0,0){$+$}}
\put(245,735){\makebox(0,0){$+$}}
\put(256,719){\makebox(0,0){$+$}}
\put(266,700){\makebox(0,0){$+$}}
\put(277,680){\makebox(0,0){$+$}}
\put(287,659){\makebox(0,0){$+$}}
\put(298,636){\makebox(0,0){$+$}}
\put(308,613){\makebox(0,0){$+$}}
\put(319,589){\makebox(0,0){$+$}}
\put(330,564){\makebox(0,0){$+$}}
\put(340,539){\makebox(0,0){$+$}}
\put(351,514){\makebox(0,0){$+$}}
\put(361,489){\makebox(0,0){$+$}}
\put(372,464){\makebox(0,0){$+$}}
\put(382,440){\makebox(0,0){$+$}}
\put(393,416){\makebox(0,0){$+$}}
\put(403,394){\makebox(0,0){$+$}}
\put(414,372){\makebox(0,0){$+$}}
\put(425,352){\makebox(0,0){$+$}}
\put(435,333){\makebox(0,0){$+$}}
\put(446,315){\makebox(0,0){$+$}}
\put(456,299){\makebox(0,0){$+$}}
\put(467,285){\makebox(0,0){$+$}}
\put(477,273){\makebox(0,0){$+$}}
\put(488,263){\makebox(0,0){$+$}}
\put(499,255){\makebox(0,0){$+$}}
\put(509,249){\makebox(0,0){$+$}}
\put(520,246){\makebox(0,0){$+$}}
\put(530,244){\makebox(0,0){$+$}}
\put(541,245){\makebox(0,0){$+$}}
\put(551,248){\makebox(0,0){$+$}}
\put(562,253){\makebox(0,0){$+$}}
\put(573,260){\makebox(0,0){$+$}}
\put(583,269){\makebox(0,0){$+$}}
\put(594,280){\makebox(0,0){$+$}}
\put(604,292){\makebox(0,0){$+$}}
\put(615,306){\makebox(0,0){$+$}}
\put(625,322){\makebox(0,0){$+$}}
\put(636,339){\makebox(0,0){$+$}}
\put(647,358){\makebox(0,0){$+$}}
\put(657,378){\makebox(0,0){$+$}}
\put(668,398){\makebox(0,0){$+$}}
\put(678,419){\makebox(0,0){$+$}}
\put(689,440){\makebox(0,0){$+$}}
\put(699,462){\makebox(0,0){$+$}}
\put(710,485){\makebox(0,0){$+$}}
\put(720,507){\makebox(0,0){$+$}}
\put(731,529){\makebox(0,0){$+$}}
\put(742,550){\makebox(0,0){$+$}}
\put(752,571){\makebox(0,0){$+$}}
\put(763,591){\makebox(0,0){$+$}}
\put(773,611){\makebox(0,0){$+$}}
\put(784,629){\makebox(0,0){$+$}}
\put(794,646){\makebox(0,0){$+$}}
\put(805,661){\makebox(0,0){$+$}}
\put(816,676){\makebox(0,0){$+$}}
\put(826,688){\makebox(0,0){$+$}}
\put(837,699){\makebox(0,0){$+$}}
\put(847,708){\makebox(0,0){$+$}}
\put(858,715){\makebox(0,0){$+$}}
\put(868,721){\makebox(0,0){$+$}}
\put(879,725){\makebox(0,0){$+$}}
\put(890,726){\makebox(0,0){$+$}}
\put(900,726){\makebox(0,0){$+$}}
\put(911,724){\makebox(0,0){$+$}}
\put(921,719){\makebox(0,0){$+$}}
\put(932,714){\makebox(0,0){$+$}}
\put(942,707){\makebox(0,0){$+$}}
\put(953,697){\makebox(0,0){$+$}}
\put(964,687){\makebox(0,0){$+$}}
\put(974,675){\makebox(0,0){$+$}}
\put(985,661){\makebox(0,0){$+$}}
\put(995,646){\makebox(0,0){$+$}}
\put(1006,631){\makebox(0,0){$+$}}
\put(1016,614){\makebox(0,0){$+$}}
\put(1027,597){\makebox(0,0){$+$}}
\put(1037,578){\makebox(0,0){$+$}}
\put(1048,560){\makebox(0,0){$+$}}
\put(1059,541){\makebox(0,0){$+$}}
\put(1069,522){\makebox(0,0){$+$}}
\put(1080,503){\makebox(0,0){$+$}}
\put(1090,484){\makebox(0,0){$+$}}
\put(1101,465){\makebox(0,0){$+$}}
\put(1111,447){\makebox(0,0){$+$}}
\put(1122,429){\makebox(0,0){$+$}}
\put(1133,413){\makebox(0,0){$+$}}
\put(1143,397){\makebox(0,0){$+$}}
\put(1154,382){\makebox(0,0){$+$}}
\put(1164,368){\makebox(0,0){$+$}}
\put(1175,355){\makebox(0,0){$+$}}
\put(1185,344){\makebox(0,0){$+$}}
\put(1196,334){\makebox(0,0){$+$}}
\put(1207,325){\makebox(0,0){$+$}}
\put(1217,318){\makebox(0,0){$+$}}
\put(1228,313){\makebox(0,0){$+$}}
\put(1238,309){\makebox(0,0){$+$}}
\put(1249,307){\makebox(0,0){$+$}}
\put(1259,307){\makebox(0,0){$+$}}
\put(1270,308){\makebox(0,0){$+$}}
\put(1281,311){\makebox(0,0){$+$}}
\put(1291,315){\makebox(0,0){$+$}}
\put(1302,321){\makebox(0,0){$+$}}
\put(1312,329){\makebox(0,0){$+$}}
\put(1323,338){\makebox(0,0){$+$}}
\put(1333,347){\makebox(0,0){$+$}}
\put(1344,359){\makebox(0,0){$+$}}
\put(1354,371){\makebox(0,0){$+$}}
\put(1365,384){\makebox(0,0){$+$}}
\put(1376,399){\makebox(0,0){$+$}}
\put(1386,414){\makebox(0,0){$+$}}
\put(1397,429){\makebox(0,0){$+$}}
\put(1407,445){\makebox(0,0){$+$}}
\put(1418,462){\makebox(0,0){$+$}}
\put(1428,478){\makebox(0,0){$+$}}
\put(1439,495){\makebox(0,0){$+$}}
\put(1349,213){\makebox(0,0){$+$}}
\put(1279,172){\makebox(0,0)[r]{$x= 1.7 e^{-0.17 t} sin\(3.7 t + 6.7\)$}}
\multiput(1299,172)(20.756,0.000){5}{\usebox{\plotpoint}}
\put(1399,172){\usebox{\plotpoint}}
\put(171,792){\usebox{\plotpoint}}
\put(171.00,792.00){\usebox{\plotpoint}}
\put(190.90,786.81){\usebox{\plotpoint}}
\put(208.68,776.21){\usebox{\plotpoint}}
\put(223.30,761.50){\usebox{\plotpoint}}
\put(236.69,745.65){\usebox{\plotpoint}}
\multiput(248,730)(10.925,-17.648){2}{\usebox{\plotpoint}}
\put(269.17,692.67){\usebox{\plotpoint}}
\put(278.45,674.10){\usebox{\plotpoint}}
\multiput(286,659)(9.004,-18.701){2}{\usebox{\plotpoint}}
\put(305.30,617.94){\usebox{\plotpoint}}
\multiput(312,603)(8.253,-19.044){2}{\usebox{\plotpoint}}
\put(330.25,560.88){\usebox{\plotpoint}}
\multiput(338,543)(7.493,-19.356){2}{\usebox{\plotpoint}}
\multiput(350,512)(8.027,-19.141){2}{\usebox{\plotpoint}}
\put(369.93,465.01){\usebox{\plotpoint}}
\multiput(376,451)(8.490,-18.940){2}{\usebox{\plotpoint}}
\put(395.41,408.20){\usebox{\plotpoint}}
\multiput(402,394)(8.430,-18.967){2}{\usebox{\plotpoint}}
\put(421.87,351.86){\usebox{\plotpoint}}
\put(431.91,333.70){\usebox{\plotpoint}}
\put(442.64,315.94){\usebox{\plotpoint}}
\put(454.02,298.58){\usebox{\plotpoint}}
\put(466.19,281.76){\usebox{\plotpoint}}
\put(479.36,265.75){\usebox{\plotpoint}}
\put(495.18,252.43){\usebox{\plotpoint}}
\put(513.44,242.64){\usebox{\plotpoint}}
\put(533.69,239.00){\usebox{\plotpoint}}
\put(553.89,242.66){\usebox{\plotpoint}}
\put(571.83,252.95){\usebox{\plotpoint}}
\put(587.50,266.50){\usebox{\plotpoint}}
\put(601.22,282.02){\usebox{\plotpoint}}
\put(613.67,298.62){\usebox{\plotpoint}}
\put(625.35,315.77){\usebox{\plotpoint}}
\put(636.35,333.37){\usebox{\plotpoint}}
\multiput(645,348)(10.213,18.069){2}{\usebox{\plotpoint}}
\put(666.70,387.72){\usebox{\plotpoint}}
\put(675.79,406.38){\usebox{\plotpoint}}
\multiput(683,422)(9.282,18.564){2}{\usebox{\plotpoint}}
\put(702.94,462.41){\usebox{\plotpoint}}
\multiput(709,475)(9.282,18.564){2}{\usebox{\plotpoint}}
\put(730.60,518.19){\usebox{\plotpoint}}
\put(739.57,536.90){\usebox{\plotpoint}}
\multiput(747,553)(9.576,18.415){2}{\usebox{\plotpoint}}
\put(768.05,592.24){\usebox{\plotpoint}}
\put(778.44,610.21){\usebox{\plotpoint}}
\put(789.21,627.94){\usebox{\plotpoint}}
\put(800.55,645.33){\usebox{\plotpoint}}
\put(812.26,662.45){\usebox{\plotpoint}}
\put(825.92,678.07){\usebox{\plotpoint}}
\put(840.55,692.73){\usebox{\plotpoint}}
\put(857.26,705.03){\usebox{\plotpoint}}
\put(875.76,714.17){\usebox{\plotpoint}}
\put(896.19,717.00){\usebox{\plotpoint}}
\put(916.36,712.91){\usebox{\plotpoint}}
\put(934.52,702.98){\usebox{\plotpoint}}
\put(951.19,690.63){\usebox{\plotpoint}}
\put(965.88,675.98){\usebox{\plotpoint}}
\put(979.37,660.21){\usebox{\plotpoint}}
\multiput(991,645)(11.720,-17.130){2}{\usebox{\plotpoint}}
\put(1014.34,608.77){\usebox{\plotpoint}}
\put(1025.22,591.10){\usebox{\plotpoint}}
\put(1035.91,573.31){\usebox{\plotpoint}}
\put(1046.47,555.44){\usebox{\plotpoint}}
\multiput(1055,541)(10.213,-18.069){2}{\usebox{\plotpoint}}
\put(1076.82,501.09){\usebox{\plotpoint}}
\put(1086.83,482.91){\usebox{\plotpoint}}
\put(1097.18,464.92){\usebox{\plotpoint}}
\multiput(1106,450)(10.559,-17.869){2}{\usebox{\plotpoint}}
\put(1129.56,411.75){\usebox{\plotpoint}}
\put(1140.38,394.03){\usebox{\plotpoint}}
\put(1152.33,377.10){\usebox{\plotpoint}}
\put(1165.24,360.85){\usebox{\plotpoint}}
\put(1178.99,345.32){\usebox{\plotpoint}}
\put(1193.92,330.92){\usebox{\plotpoint}}
\put(1210.03,317.91){\usebox{\plotpoint}}
\put(1228.53,308.52){\usebox{\plotpoint}}
\put(1248.58,303.88){\usebox{\plotpoint}}
\put(1269.17,304.53){\usebox{\plotpoint}}
\put(1288.78,311.03){\usebox{\plotpoint}}
\put(1306.45,321.85){\usebox{\plotpoint}}
\put(1323.07,334.28){\usebox{\plotpoint}}
\put(1337.78,348.91){\usebox{\plotpoint}}
\put(1351.21,364.72){\usebox{\plotpoint}}
\put(1364.22,380.90){\usebox{\plotpoint}}
\put(1376.76,397.43){\usebox{\plotpoint}}
\multiput(1388,413)(11.720,17.130){2}{\usebox{\plotpoint}}
\put(1411.70,448.95){\usebox{\plotpoint}}
\put(1422.99,466.37){\usebox{\plotpoint}}
\put(1434.30,483.77){\usebox{\plotpoint}}
\put(1439,491){\usebox{\plotpoint}}
\put(171.0,131.0){\rule[-0.200pt]{0.400pt}{175.375pt}}
\put(171.0,131.0){\rule[-0.200pt]{305.461pt}{0.400pt}}
\put(1439.0,131.0){\rule[-0.200pt]{0.400pt}{175.375pt}}
\put(171.0,859.0){\rule[-0.200pt]{305.461pt}{0.400pt}}
\end{picture}

\caption{Závislosť výchylky $x$ na čase $t$ preložená závislosťou $x= 1.7 e^{-0.17 t} sin\(3.7 t + 6.7\)$ pre tlmiaci prúd $I="0.5 A"$}  \label{G_PT_.5}
\end{figure}

\begin{figure}
% GNUPLOT: LaTeX picture
\setlength{\unitlength}{0.240900pt}
\ifx\plotpoint\undefined\newsavebox{\plotpoint}\fi
\begin{picture}(1500,900)(0,0)
\sbox{\plotpoint}{\rule[-0.200pt]{0.400pt}{0.400pt}}%
\put(171.0,131.0){\rule[-0.200pt]{4.818pt}{0.400pt}}
\put(151,131){\makebox(0,0)[r]{-0.8}}
\put(1419.0,131.0){\rule[-0.200pt]{4.818pt}{0.400pt}}
\put(171.0,235.0){\rule[-0.200pt]{4.818pt}{0.400pt}}
\put(151,235){\makebox(0,0)[r]{-0.6}}
\put(1419.0,235.0){\rule[-0.200pt]{4.818pt}{0.400pt}}
\put(171.0,339.0){\rule[-0.200pt]{4.818pt}{0.400pt}}
\put(151,339){\makebox(0,0)[r]{-0.4}}
\put(1419.0,339.0){\rule[-0.200pt]{4.818pt}{0.400pt}}
\put(171.0,443.0){\rule[-0.200pt]{4.818pt}{0.400pt}}
\put(151,443){\makebox(0,0)[r]{-0.2}}
\put(1419.0,443.0){\rule[-0.200pt]{4.818pt}{0.400pt}}
\put(171.0,547.0){\rule[-0.200pt]{4.818pt}{0.400pt}}
\put(151,547){\makebox(0,0)[r]{ 0}}
\put(1419.0,547.0){\rule[-0.200pt]{4.818pt}{0.400pt}}
\put(171.0,651.0){\rule[-0.200pt]{4.818pt}{0.400pt}}
\put(151,651){\makebox(0,0)[r]{ 0.2}}
\put(1419.0,651.0){\rule[-0.200pt]{4.818pt}{0.400pt}}
\put(171.0,755.0){\rule[-0.200pt]{4.818pt}{0.400pt}}
\put(151,755){\makebox(0,0)[r]{ 0.4}}
\put(1419.0,755.0){\rule[-0.200pt]{4.818pt}{0.400pt}}
\put(171.0,859.0){\rule[-0.200pt]{4.818pt}{0.400pt}}
\put(151,859){\makebox(0,0)[r]{ 0.6}}
\put(1419.0,859.0){\rule[-0.200pt]{4.818pt}{0.400pt}}
\put(171.0,131.0){\rule[-0.200pt]{0.400pt}{4.818pt}}
\put(171,90){\makebox(0,0){ 2}}
\put(171.0,839.0){\rule[-0.200pt]{0.400pt}{4.818pt}}
\put(382.0,131.0){\rule[-0.200pt]{0.400pt}{4.818pt}}
\put(382,90){\makebox(0,0){ 2.5}}
\put(382.0,839.0){\rule[-0.200pt]{0.400pt}{4.818pt}}
\put(594.0,131.0){\rule[-0.200pt]{0.400pt}{4.818pt}}
\put(594,90){\makebox(0,0){ 3}}
\put(594.0,839.0){\rule[-0.200pt]{0.400pt}{4.818pt}}
\put(805.0,131.0){\rule[-0.200pt]{0.400pt}{4.818pt}}
\put(805,90){\makebox(0,0){ 3.5}}
\put(805.0,839.0){\rule[-0.200pt]{0.400pt}{4.818pt}}
\put(1016.0,131.0){\rule[-0.200pt]{0.400pt}{4.818pt}}
\put(1016,90){\makebox(0,0){ 4}}
\put(1016.0,839.0){\rule[-0.200pt]{0.400pt}{4.818pt}}
\put(1228.0,131.0){\rule[-0.200pt]{0.400pt}{4.818pt}}
\put(1228,90){\makebox(0,0){ 4.5}}
\put(1228.0,839.0){\rule[-0.200pt]{0.400pt}{4.818pt}}
\put(1439.0,131.0){\rule[-0.200pt]{0.400pt}{4.818pt}}
\put(1439,90){\makebox(0,0){ 5}}
\put(1439.0,839.0){\rule[-0.200pt]{0.400pt}{4.818pt}}
\put(171.0,131.0){\rule[-0.200pt]{0.400pt}{175.375pt}}
\put(171.0,131.0){\rule[-0.200pt]{305.461pt}{0.400pt}}
\put(1439.0,131.0){\rule[-0.200pt]{0.400pt}{175.375pt}}
\put(171.0,859.0){\rule[-0.200pt]{305.461pt}{0.400pt}}
\put(30,495){\makebox(0,0){\popi{x}{mm}}}
\put(805,29){\makebox(0,0){\popi{t}{s}}}
\put(1279,213){\makebox(0,0)[r]{namerané hodnoty}}
\put(171,360){\makebox(0,0){$+$}}
\put(182,325){\makebox(0,0){$+$}}
\put(192,295){\makebox(0,0){$+$}}
\put(203,266){\makebox(0,0){$+$}}
\put(213,240){\makebox(0,0){$+$}}
\put(224,217){\makebox(0,0){$+$}}
\put(234,198){\makebox(0,0){$+$}}
\put(245,182){\makebox(0,0){$+$}}
\put(256,169){\makebox(0,0){$+$}}
\put(266,161){\makebox(0,0){$+$}}
\put(277,155){\makebox(0,0){$+$}}
\put(287,152){\makebox(0,0){$+$}}
\put(298,154){\makebox(0,0){$+$}}
\put(308,158){\makebox(0,0){$+$}}
\put(319,167){\makebox(0,0){$+$}}
\put(330,178){\makebox(0,0){$+$}}
\put(340,192){\makebox(0,0){$+$}}
\put(351,210){\makebox(0,0){$+$}}
\put(361,229){\makebox(0,0){$+$}}
\put(372,251){\makebox(0,0){$+$}}
\put(382,275){\makebox(0,0){$+$}}
\put(393,301){\makebox(0,0){$+$}}
\put(403,329){\makebox(0,0){$+$}}
\put(414,358){\makebox(0,0){$+$}}
\put(425,388){\makebox(0,0){$+$}}
\put(435,420){\makebox(0,0){$+$}}
\put(446,452){\makebox(0,0){$+$}}
\put(456,484){\makebox(0,0){$+$}}
\put(467,517){\makebox(0,0){$+$}}
\put(477,548){\makebox(0,0){$+$}}
\put(488,580){\makebox(0,0){$+$}}
\put(499,610){\makebox(0,0){$+$}}
\put(509,640){\makebox(0,0){$+$}}
\put(520,667){\makebox(0,0){$+$}}
\put(530,694){\makebox(0,0){$+$}}
\put(541,719){\makebox(0,0){$+$}}
\put(551,741){\makebox(0,0){$+$}}
\put(562,763){\makebox(0,0){$+$}}
\put(573,782){\makebox(0,0){$+$}}
\put(583,798){\makebox(0,0){$+$}}
\put(594,813){\makebox(0,0){$+$}}
\put(604,824){\makebox(0,0){$+$}}
\put(615,834){\makebox(0,0){$+$}}
\put(625,840){\makebox(0,0){$+$}}
\put(636,845){\makebox(0,0){$+$}}
\put(647,847){\makebox(0,0){$+$}}
\put(657,847){\makebox(0,0){$+$}}
\put(668,843){\makebox(0,0){$+$}}
\put(678,838){\makebox(0,0){$+$}}
\put(689,830){\makebox(0,0){$+$}}
\put(699,821){\makebox(0,0){$+$}}
\put(710,809){\makebox(0,0){$+$}}
\put(720,796){\makebox(0,0){$+$}}
\put(731,780){\makebox(0,0){$+$}}
\put(742,763){\makebox(0,0){$+$}}
\put(752,746){\makebox(0,0){$+$}}
\put(763,726){\makebox(0,0){$+$}}
\put(773,706){\makebox(0,0){$+$}}
\put(784,684){\makebox(0,0){$+$}}
\put(794,662){\makebox(0,0){$+$}}
\put(805,640){\makebox(0,0){$+$}}
\put(816,618){\makebox(0,0){$+$}}
\put(826,596){\makebox(0,0){$+$}}
\put(837,573){\makebox(0,0){$+$}}
\put(847,551){\makebox(0,0){$+$}}
\put(858,530){\makebox(0,0){$+$}}
\put(868,510){\makebox(0,0){$+$}}
\put(879,490){\makebox(0,0){$+$}}
\put(890,471){\makebox(0,0){$+$}}
\put(900,453){\makebox(0,0){$+$}}
\put(911,437){\makebox(0,0){$+$}}
\put(921,422){\makebox(0,0){$+$}}
\put(932,408){\makebox(0,0){$+$}}
\put(942,396){\makebox(0,0){$+$}}
\put(953,386){\makebox(0,0){$+$}}
\put(964,377){\makebox(0,0){$+$}}
\put(974,371){\makebox(0,0){$+$}}
\put(985,365){\makebox(0,0){$+$}}
\put(995,361){\makebox(0,0){$+$}}
\put(1006,360){\makebox(0,0){$+$}}
\put(1016,360){\makebox(0,0){$+$}}
\put(1027,361){\makebox(0,0){$+$}}
\put(1037,365){\makebox(0,0){$+$}}
\put(1048,370){\makebox(0,0){$+$}}
\put(1059,376){\makebox(0,0){$+$}}
\put(1069,384){\makebox(0,0){$+$}}
\put(1080,394){\makebox(0,0){$+$}}
\put(1090,403){\makebox(0,0){$+$}}
\put(1101,415){\makebox(0,0){$+$}}
\put(1111,428){\makebox(0,0){$+$}}
\put(1122,441){\makebox(0,0){$+$}}
\put(1133,454){\makebox(0,0){$+$}}
\put(1143,470){\makebox(0,0){$+$}}
\put(1154,484){\makebox(0,0){$+$}}
\put(1164,500){\makebox(0,0){$+$}}
\put(1175,515){\makebox(0,0){$+$}}
\put(1185,531){\makebox(0,0){$+$}}
\put(1196,546){\makebox(0,0){$+$}}
\put(1207,561){\makebox(0,0){$+$}}
\put(1217,577){\makebox(0,0){$+$}}
\put(1228,590){\makebox(0,0){$+$}}
\put(1238,604){\makebox(0,0){$+$}}
\put(1249,617){\makebox(0,0){$+$}}
\put(1259,630){\makebox(0,0){$+$}}
\put(1270,641){\makebox(0,0){$+$}}
\put(1281,650){\makebox(0,0){$+$}}
\put(1291,660){\makebox(0,0){$+$}}
\put(1302,668){\makebox(0,0){$+$}}
\put(1312,674){\makebox(0,0){$+$}}
\put(1323,681){\makebox(0,0){$+$}}
\put(1333,685){\makebox(0,0){$+$}}
\put(1344,688){\makebox(0,0){$+$}}
\put(1354,690){\makebox(0,0){$+$}}
\put(1365,691){\makebox(0,0){$+$}}
\put(1376,691){\makebox(0,0){$+$}}
\put(1386,689){\makebox(0,0){$+$}}
\put(1397,687){\makebox(0,0){$+$}}
\put(1407,683){\makebox(0,0){$+$}}
\put(1418,678){\makebox(0,0){$+$}}
\put(1428,672){\makebox(0,0){$+$}}
\put(1439,666){\makebox(0,0){$+$}}
\put(1349,213){\makebox(0,0){$+$}}
\put(1279,172){\makebox(0,0)[r]{$x= 1.9 e^{-0.40 t} sin\(3.7 t - 16.3\)$}}
\multiput(1299,172)(20.756,0.000){5}{\usebox{\plotpoint}}
\put(1399,172){\usebox{\plotpoint}}
\put(171,352){\usebox{\plotpoint}}
\multiput(171,352)(6.273,-19.785){3}{\usebox{\plotpoint}}
\put(190.54,292.89){\usebox{\plotpoint}}
\multiput(197,275)(7.093,-19.506){2}{\usebox{\plotpoint}}
\multiput(209,242)(8.740,-18.825){2}{\usebox{\plotpoint}}
\put(230.94,197.49){\usebox{\plotpoint}}
\put(242.16,180.08){\usebox{\plotpoint}}
\put(255.63,164.37){\usebox{\plotpoint}}
\put(271.95,151.70){\usebox{\plotpoint}}
\put(292.08,148.93){\usebox{\plotpoint}}
\put(311.11,156.52){\usebox{\plotpoint}}
\put(326.71,170.10){\usebox{\plotpoint}}
\multiput(338,184)(10.298,18.021){2}{\usebox{\plotpoint}}
\put(359.93,222.56){\usebox{\plotpoint}}
\put(369.11,241.16){\usebox{\plotpoint}}
\multiput(376,256)(8.253,19.044){2}{\usebox{\plotpoint}}
\multiput(389,286)(7.607,19.311){2}{\usebox{\plotpoint}}
\put(408.41,337.16){\usebox{\plotpoint}}
\multiput(414,353)(6.880,19.582){2}{\usebox{\plotpoint}}
\multiput(427,390)(6.880,19.582){2}{\usebox{\plotpoint}}
\multiput(440,427)(6.718,19.638){2}{\usebox{\plotpoint}}
\multiput(453,465)(6.718,19.638){2}{\usebox{\plotpoint}}
\multiput(466,503)(6.250,19.792){2}{\usebox{\plotpoint}}
\multiput(478,541)(6.880,19.582){2}{\usebox{\plotpoint}}
\multiput(491,578)(7.227,19.457){2}{\usebox{\plotpoint}}
\put(510.92,631.09){\usebox{\plotpoint}}
\multiput(517,647)(7.812,19.229){2}{\usebox{\plotpoint}}
\multiput(530,679)(7.708,19.271){2}{\usebox{\plotpoint}}
\put(550.63,726.92){\usebox{\plotpoint}}
\put(560.26,745.30){\usebox{\plotpoint}}
\put(570.64,763.27){\usebox{\plotpoint}}
\put(581.65,780.86){\usebox{\plotpoint}}
\put(594.28,797.33){\usebox{\plotpoint}}
\put(608.18,812.68){\usebox{\plotpoint}}
\put(625.45,823.98){\usebox{\plotpoint}}
\put(645.20,829.98){\usebox{\plotpoint}}
\put(665.56,826.67){\usebox{\plotpoint}}
\put(683.90,817.24){\usebox{\plotpoint}}
\put(699.34,803.41){\usebox{\plotpoint}}
\put(713.13,787.91){\usebox{\plotpoint}}
\put(725.78,771.48){\usebox{\plotpoint}}
\put(737.20,754.16){\usebox{\plotpoint}}
\multiput(747,737)(10.213,-18.069){2}{\usebox{\plotpoint}}
\put(767.42,699.73){\usebox{\plotpoint}}
\put(777.00,681.31){\usebox{\plotpoint}}
\multiput(786,664)(9.004,-18.701){2}{\usebox{\plotpoint}}
\put(804.19,625.32){\usebox{\plotpoint}}
\multiput(811,610)(9.004,-18.701){2}{\usebox{\plotpoint}}
\put(830.74,569.00){\usebox{\plotpoint}}
\multiput(837,556)(9.282,-18.564){2}{\usebox{\plotpoint}}
\put(858.66,513.35){\usebox{\plotpoint}}
\put(867.91,494.77){\usebox{\plotpoint}}
\multiput(875,480)(10.213,-18.069){2}{\usebox{\plotpoint}}
\put(898.24,440.45){\usebox{\plotpoint}}
\put(909.76,423.19){\usebox{\plotpoint}}
\put(921.76,406.25){\usebox{\plotpoint}}
\put(934.38,389.78){\usebox{\plotpoint}}
\put(948.81,374.94){\usebox{\plotpoint}}
\put(964.53,361.40){\usebox{\plotpoint}}
\put(983.02,352.07){\usebox{\plotpoint}}
\put(1002.89,346.26){\usebox{\plotpoint}}
\put(1023.42,347.71){\usebox{\plotpoint}}
\put(1042.97,354.52){\usebox{\plotpoint}}
\put(1060.83,365.04){\usebox{\plotpoint}}
\put(1076.51,378.51){\usebox{\plotpoint}}
\put(1091.19,393.19){\usebox{\plotpoint}}
\put(1105.38,408.33){\usebox{\plotpoint}}
\put(1118.51,424.40){\usebox{\plotpoint}}
\put(1130.70,441.20){\usebox{\plotpoint}}
\put(1142.28,458.42){\usebox{\plotpoint}}
\put(1154.34,475.32){\usebox{\plotpoint}}
\put(1166.15,492.38){\usebox{\plotpoint}}
\put(1177.87,509.51){\usebox{\plotpoint}}
\put(1189.37,526.79){\usebox{\plotpoint}}
\put(1200.76,544.14){\usebox{\plotpoint}}
\put(1212.35,561.36){\usebox{\plotpoint}}
\put(1224.30,578.32){\usebox{\plotpoint}}
\put(1236.91,594.81){\usebox{\plotpoint}}
\put(1249.72,611.13){\usebox{\plotpoint}}
\put(1263.29,626.84){\usebox{\plotpoint}}
\put(1277.21,642.21){\usebox{\plotpoint}}
\put(1292.44,656.29){\usebox{\plotpoint}}
\put(1309.07,668.66){\usebox{\plotpoint}}
\put(1327.31,678.53){\usebox{\plotpoint}}
\put(1346.57,686.19){\usebox{\plotpoint}}
\put(1367.02,689.39){\usebox{\plotpoint}}
\put(1387.60,688.06){\usebox{\plotpoint}}
\put(1407.47,682.30){\usebox{\plotpoint}}
\put(1426.38,673.76){\usebox{\plotpoint}}
\put(1439,666){\usebox{\plotpoint}}
\put(171.0,131.0){\rule[-0.200pt]{0.400pt}{175.375pt}}
\put(171.0,131.0){\rule[-0.200pt]{305.461pt}{0.400pt}}
\put(1439.0,131.0){\rule[-0.200pt]{0.400pt}{175.375pt}}
\put(171.0,859.0){\rule[-0.200pt]{305.461pt}{0.400pt}}
\end{picture}

\caption{Závislosť výchylky $x$ na čase $t$ preložená závislosťou $x= 1.9 e^{-0.40 t} sin\(3.7 t - 16.3\)$ pre tlmiaci prúd $I="1.0 A"$}  \label{G_PT_1}
\end{figure}

\begin{figure}
% GNUPLOT: LaTeX picture
\setlength{\unitlength}{0.240900pt}
\ifx\plotpoint\undefined\newsavebox{\plotpoint}\fi
\begin{picture}(1500,900)(0,0)
\sbox{\plotpoint}{\rule[-0.200pt]{0.400pt}{0.400pt}}%
\put(171.0,131.0){\rule[-0.200pt]{4.818pt}{0.400pt}}
\put(151,131){\makebox(0,0)[r]{-0.8}}
\put(1419.0,131.0){\rule[-0.200pt]{4.818pt}{0.400pt}}
\put(171.0,222.0){\rule[-0.200pt]{4.818pt}{0.400pt}}
\put(151,222){\makebox(0,0)[r]{-0.6}}
\put(1419.0,222.0){\rule[-0.200pt]{4.818pt}{0.400pt}}
\put(171.0,313.0){\rule[-0.200pt]{4.818pt}{0.400pt}}
\put(151,313){\makebox(0,0)[r]{-0.4}}
\put(1419.0,313.0){\rule[-0.200pt]{4.818pt}{0.400pt}}
\put(171.0,404.0){\rule[-0.200pt]{4.818pt}{0.400pt}}
\put(151,404){\makebox(0,0)[r]{-0.2}}
\put(1419.0,404.0){\rule[-0.200pt]{4.818pt}{0.400pt}}
\put(171.0,495.0){\rule[-0.200pt]{4.818pt}{0.400pt}}
\put(151,495){\makebox(0,0)[r]{ 0}}
\put(1419.0,495.0){\rule[-0.200pt]{4.818pt}{0.400pt}}
\put(171.0,586.0){\rule[-0.200pt]{4.818pt}{0.400pt}}
\put(151,586){\makebox(0,0)[r]{ 0.2}}
\put(1419.0,586.0){\rule[-0.200pt]{4.818pt}{0.400pt}}
\put(171.0,677.0){\rule[-0.200pt]{4.818pt}{0.400pt}}
\put(151,677){\makebox(0,0)[r]{ 0.4}}
\put(1419.0,677.0){\rule[-0.200pt]{4.818pt}{0.400pt}}
\put(171.0,768.0){\rule[-0.200pt]{4.818pt}{0.400pt}}
\put(151,768){\makebox(0,0)[r]{ 0.6}}
\put(1419.0,768.0){\rule[-0.200pt]{4.818pt}{0.400pt}}
\put(171.0,859.0){\rule[-0.200pt]{4.818pt}{0.400pt}}
\put(151,859){\makebox(0,0)[r]{ 0.8}}
\put(1419.0,859.0){\rule[-0.200pt]{4.818pt}{0.400pt}}
\put(336.0,131.0){\rule[-0.200pt]{0.400pt}{4.818pt}}
\put(336,90){\makebox(0,0){ 1.5}}
\put(336.0,839.0){\rule[-0.200pt]{0.400pt}{4.818pt}}
\put(612.0,131.0){\rule[-0.200pt]{0.400pt}{4.818pt}}
\put(612,90){\makebox(0,0){ 2}}
\put(612.0,839.0){\rule[-0.200pt]{0.400pt}{4.818pt}}
\put(888.0,131.0){\rule[-0.200pt]{0.400pt}{4.818pt}}
\put(888,90){\makebox(0,0){ 2.5}}
\put(888.0,839.0){\rule[-0.200pt]{0.400pt}{4.818pt}}
\put(1163.0,131.0){\rule[-0.200pt]{0.400pt}{4.818pt}}
\put(1163,90){\makebox(0,0){ 3}}
\put(1163.0,839.0){\rule[-0.200pt]{0.400pt}{4.818pt}}
\put(1439.0,131.0){\rule[-0.200pt]{0.400pt}{4.818pt}}
\put(1439,90){\makebox(0,0){ 3.5}}
\put(1439.0,839.0){\rule[-0.200pt]{0.400pt}{4.818pt}}
\put(171.0,131.0){\rule[-0.200pt]{0.400pt}{175.375pt}}
\put(171.0,131.0){\rule[-0.200pt]{305.461pt}{0.400pt}}
\put(1439.0,131.0){\rule[-0.200pt]{0.400pt}{175.375pt}}
\put(171.0,859.0){\rule[-0.200pt]{305.461pt}{0.400pt}}
\put(30,495){\makebox(0,0){\popi{x}{mm}}}
\put(805,29){\makebox(0,0){\popi{t}{s}}}
\put(1279,213){\makebox(0,0)[r]{namerané hodnoty}}
\put(171,195){\makebox(0,0){$+$}}
\put(185,233){\makebox(0,0){$+$}}
\put(199,273){\makebox(0,0){$+$}}
\put(212,313){\makebox(0,0){$+$}}
\put(226,353){\makebox(0,0){$+$}}
\put(240,393){\makebox(0,0){$+$}}
\put(254,432){\makebox(0,0){$+$}}
\put(267,471){\makebox(0,0){$+$}}
\put(281,508){\makebox(0,0){$+$}}
\put(295,544){\makebox(0,0){$+$}}
\put(309,577){\makebox(0,0){$+$}}
\put(323,609){\makebox(0,0){$+$}}
\put(336,638){\makebox(0,0){$+$}}
\put(350,664){\makebox(0,0){$+$}}
\put(364,689){\makebox(0,0){$+$}}
\put(378,712){\makebox(0,0){$+$}}
\put(392,732){\makebox(0,0){$+$}}
\put(405,749){\makebox(0,0){$+$}}
\put(419,763){\makebox(0,0){$+$}}
\put(433,774){\makebox(0,0){$+$}}
\put(447,784){\makebox(0,0){$+$}}
\put(460,789){\makebox(0,0){$+$}}
\put(474,793){\makebox(0,0){$+$}}
\put(488,794){\makebox(0,0){$+$}}
\put(502,793){\makebox(0,0){$+$}}
\put(516,789){\makebox(0,0){$+$}}
\put(529,784){\makebox(0,0){$+$}}
\put(543,776){\makebox(0,0){$+$}}
\put(557,766){\makebox(0,0){$+$}}
\put(571,754){\makebox(0,0){$+$}}
\put(584,741){\makebox(0,0){$+$}}
\put(598,727){\makebox(0,0){$+$}}
\put(612,712){\makebox(0,0){$+$}}
\put(626,694){\makebox(0,0){$+$}}
\put(640,677){\makebox(0,0){$+$}}
\put(653,658){\makebox(0,0){$+$}}
\put(667,639){\makebox(0,0){$+$}}
\put(681,620){\makebox(0,0){$+$}}
\put(695,601){\makebox(0,0){$+$}}
\put(709,581){\makebox(0,0){$+$}}
\put(722,562){\makebox(0,0){$+$}}
\put(736,544){\makebox(0,0){$+$}}
\put(750,526){\makebox(0,0){$+$}}
\put(764,508){\makebox(0,0){$+$}}
\put(777,491){\makebox(0,0){$+$}}
\put(791,475){\makebox(0,0){$+$}}
\put(805,460){\makebox(0,0){$+$}}
\put(819,446){\makebox(0,0){$+$}}
\put(833,433){\makebox(0,0){$+$}}
\put(846,421){\makebox(0,0){$+$}}
\put(860,411){\makebox(0,0){$+$}}
\put(874,402){\makebox(0,0){$+$}}
\put(888,394){\makebox(0,0){$+$}}
\put(901,388){\makebox(0,0){$+$}}
\put(915,382){\makebox(0,0){$+$}}
\put(929,378){\makebox(0,0){$+$}}
\put(943,375){\makebox(0,0){$+$}}
\put(957,374){\makebox(0,0){$+$}}
\put(970,374){\makebox(0,0){$+$}}
\put(984,374){\makebox(0,0){$+$}}
\put(998,376){\makebox(0,0){$+$}}
\put(1012,379){\makebox(0,0){$+$}}
\put(1026,383){\makebox(0,0){$+$}}
\put(1039,387){\makebox(0,0){$+$}}
\put(1053,393){\makebox(0,0){$+$}}
\put(1067,399){\makebox(0,0){$+$}}
\put(1081,406){\makebox(0,0){$+$}}
\put(1094,413){\makebox(0,0){$+$}}
\put(1108,421){\makebox(0,0){$+$}}
\put(1122,429){\makebox(0,0){$+$}}
\put(1136,438){\makebox(0,0){$+$}}
\put(1150,446){\makebox(0,0){$+$}}
\put(1163,456){\makebox(0,0){$+$}}
\put(1177,464){\makebox(0,0){$+$}}
\put(1191,474){\makebox(0,0){$+$}}
\put(1205,482){\makebox(0,0){$+$}}
\put(1218,491){\makebox(0,0){$+$}}
\put(1232,499){\makebox(0,0){$+$}}
\put(1246,507){\makebox(0,0){$+$}}
\put(1260,515){\makebox(0,0){$+$}}
\put(1274,522){\makebox(0,0){$+$}}
\put(1287,529){\makebox(0,0){$+$}}
\put(1301,535){\makebox(0,0){$+$}}
\put(1315,541){\makebox(0,0){$+$}}
\put(1329,546){\makebox(0,0){$+$}}
\put(1343,550){\makebox(0,0){$+$}}
\put(1356,554){\makebox(0,0){$+$}}
\put(1370,557){\makebox(0,0){$+$}}
\put(1384,560){\makebox(0,0){$+$}}
\put(1398,562){\makebox(0,0){$+$}}
\put(1411,564){\makebox(0,0){$+$}}
\put(1425,564){\makebox(0,0){$+$}}
\put(1439,564){\makebox(0,0){$+$}}
\put(1349,213){\makebox(0,0){$+$}}
\put(1279,172){\makebox(0,0)[r]{$x= 3.3 e^{-0.9 t} sin\(3.6 t - 4.9\)$}}
\multiput(1299,172)(20.756,0.000){5}{\usebox{\plotpoint}}
\put(1399,172){\usebox{\plotpoint}}
\put(171,176){\usebox{\plotpoint}}
\multiput(171,176)(6.563,19.690){2}{\usebox{\plotpoint}}
\multiput(184,215)(6.415,19.739){3}{\usebox{\plotpoint}}
\multiput(197,255)(5.964,19.880){2}{\usebox{\plotpoint}}
\multiput(209,295)(6.415,19.739){2}{\usebox{\plotpoint}}
\multiput(222,335)(6.563,19.690){2}{\usebox{\plotpoint}}
\multiput(235,374)(6.563,19.690){2}{\usebox{\plotpoint}}
\put(254.83,432.45){\usebox{\plotpoint}}
\multiput(261,450)(6.403,19.743){2}{\usebox{\plotpoint}}
\multiput(273,487)(7.413,19.387){2}{\usebox{\plotpoint}}
\multiput(286,521)(7.607,19.311){2}{\usebox{\plotpoint}}
\put(304.99,568.75){\usebox{\plotpoint}}
\multiput(312,586)(8.490,18.940){2}{\usebox{\plotpoint}}
\put(330.00,625.76){\usebox{\plotpoint}}
\multiput(338,643)(8.982,18.712){2}{\usebox{\plotpoint}}
\put(357.64,681.52){\usebox{\plotpoint}}
\put(368.38,699.27){\usebox{\plotpoint}}
\put(379.96,716.49){\usebox{\plotpoint}}
\put(392.35,733.13){\usebox{\plotpoint}}
\put(405.70,749.01){\usebox{\plotpoint}}
\put(420.51,763.50){\usebox{\plotpoint}}
\put(437.07,775.97){\usebox{\plotpoint}}
\put(455.75,784.85){\usebox{\plotpoint}}
\put(475.99,788.83){\usebox{\plotpoint}}
\put(496.59,787.71){\usebox{\plotpoint}}
\put(516.28,781.28){\usebox{\plotpoint}}
\put(534.71,771.86){\usebox{\plotpoint}}
\put(551.51,759.68){\usebox{\plotpoint}}
\put(567.48,746.44){\usebox{\plotpoint}}
\put(582.20,731.80){\usebox{\plotpoint}}
\put(596.54,716.83){\usebox{\plotpoint}}
\put(609.54,700.65){\usebox{\plotpoint}}
\put(622.63,684.54){\usebox{\plotpoint}}
\put(635.58,668.32){\usebox{\plotpoint}}
\put(648.19,651.83){\usebox{\plotpoint}}
\put(660.69,635.27){\usebox{\plotpoint}}
\put(672.75,618.38){\usebox{\plotpoint}}
\put(684.33,601.16){\usebox{\plotpoint}}
\put(696.50,584.34){\usebox{\plotpoint}}
\multiput(709,568)(12.152,-16.826){2}{\usebox{\plotpoint}}
\put(733.84,534.52){\usebox{\plotpoint}}
\put(745.87,517.61){\usebox{\plotpoint}}
\put(758.85,501.42){\usebox{\plotpoint}}
\put(772.40,485.70){\usebox{\plotpoint}}
\put(785.99,470.01){\usebox{\plotpoint}}
\put(800.11,454.80){\usebox{\plotpoint}}
\put(814.32,439.68){\usebox{\plotpoint}}
\put(829.40,425.43){\usebox{\plotpoint}}
\put(845.56,412.42){\usebox{\plotpoint}}
\put(862.46,400.38){\usebox{\plotpoint}}
\put(879.49,388.58){\usebox{\plotpoint}}
\put(898.07,379.35){\usebox{\plotpoint}}
\put(917.44,371.94){\usebox{\plotpoint}}
\put(937.60,367.23){\usebox{\plotpoint}}
\put(958.17,364.53){\usebox{\plotpoint}}
\put(978.90,364.07){\usebox{\plotpoint}}
\put(999.52,366.31){\usebox{\plotpoint}}
\put(1019.68,371.13){\usebox{\plotpoint}}
\put(1039.52,377.24){\usebox{\plotpoint}}
\put(1058.95,384.52){\usebox{\plotpoint}}
\put(1077.89,392.95){\usebox{\plotpoint}}
\put(1096.20,402.72){\usebox{\plotpoint}}
\put(1114.48,412.56){\usebox{\plotpoint}}
\put(1132.30,423.18){\usebox{\plotpoint}}
\put(1150.14,433.78){\usebox{\plotpoint}}
\put(1167.82,444.66){\usebox{\plotpoint}}
\put(1185.50,455.54){\usebox{\plotpoint}}
\put(1203.01,466.67){\usebox{\plotpoint}}
\put(1220.99,477.00){\usebox{\plotpoint}}
\put(1238.83,487.60){\usebox{\plotpoint}}
\put(1257.10,497.44){\usebox{\plotpoint}}
\put(1275.24,507.50){\usebox{\plotpoint}}
\put(1294.09,516.20){\usebox{\plotpoint}}
\put(1313.35,523.90){\usebox{\plotpoint}}
\put(1332.72,531.36){\usebox{\plotpoint}}
\put(1352.37,538.04){\usebox{\plotpoint}}
\put(1372.41,543.40){\usebox{\plotpoint}}
\put(1392.70,547.72){\usebox{\plotpoint}}
\put(1413.19,551.01){\usebox{\plotpoint}}
\put(1433.88,552.61){\usebox{\plotpoint}}
\put(1439,553){\usebox{\plotpoint}}
\put(171.0,131.0){\rule[-0.200pt]{0.400pt}{175.375pt}}
\put(171.0,131.0){\rule[-0.200pt]{305.461pt}{0.400pt}}
\put(1439.0,131.0){\rule[-0.200pt]{0.400pt}{175.375pt}}
\put(171.0,859.0){\rule[-0.200pt]{305.461pt}{0.400pt}}
\end{picture}

\caption{Závislosť výchylky $x$ na čase $t$ preložená závislosťou $x= 3.3 e^{-0.9 t} sin\(3.6 t - 4.9\)$ pre tlmiací príd $I="1.5 A"$}  \label{G_PT_1.5}
\end{figure}

\begin{figure}
% GNUPLOT: LaTeX picture
\setlength{\unitlength}{0.240900pt}
\ifx\plotpoint\undefined\newsavebox{\plotpoint}\fi
\begin{picture}(1500,900)(0,0)
\sbox{\plotpoint}{\rule[-0.200pt]{0.400pt}{0.400pt}}%
\put(171.0,131.0){\rule[-0.200pt]{4.818pt}{0.400pt}}
\put(151,131){\makebox(0,0)[r]{-1.4}}
\put(1419.0,131.0){\rule[-0.200pt]{4.818pt}{0.400pt}}
\put(171.0,204.0){\rule[-0.200pt]{4.818pt}{0.400pt}}
\put(151,204){\makebox(0,0)[r]{-1.2}}
\put(1419.0,204.0){\rule[-0.200pt]{4.818pt}{0.400pt}}
\put(171.0,277.0){\rule[-0.200pt]{4.818pt}{0.400pt}}
\put(151,277){\makebox(0,0)[r]{-1}}
\put(1419.0,277.0){\rule[-0.200pt]{4.818pt}{0.400pt}}
\put(171.0,349.0){\rule[-0.200pt]{4.818pt}{0.400pt}}
\put(151,349){\makebox(0,0)[r]{-0.8}}
\put(1419.0,349.0){\rule[-0.200pt]{4.818pt}{0.400pt}}
\put(171.0,422.0){\rule[-0.200pt]{4.818pt}{0.400pt}}
\put(151,422){\makebox(0,0)[r]{-0.6}}
\put(1419.0,422.0){\rule[-0.200pt]{4.818pt}{0.400pt}}
\put(171.0,495.0){\rule[-0.200pt]{4.818pt}{0.400pt}}
\put(151,495){\makebox(0,0)[r]{-0.4}}
\put(1419.0,495.0){\rule[-0.200pt]{4.818pt}{0.400pt}}
\put(171.0,568.0){\rule[-0.200pt]{4.818pt}{0.400pt}}
\put(151,568){\makebox(0,0)[r]{-0.2}}
\put(1419.0,568.0){\rule[-0.200pt]{4.818pt}{0.400pt}}
\put(171.0,641.0){\rule[-0.200pt]{4.818pt}{0.400pt}}
\put(151,641){\makebox(0,0)[r]{ 0}}
\put(1419.0,641.0){\rule[-0.200pt]{4.818pt}{0.400pt}}
\put(171.0,713.0){\rule[-0.200pt]{4.818pt}{0.400pt}}
\put(151,713){\makebox(0,0)[r]{ 0.2}}
\put(1419.0,713.0){\rule[-0.200pt]{4.818pt}{0.400pt}}
\put(171.0,786.0){\rule[-0.200pt]{4.818pt}{0.400pt}}
\put(151,786){\makebox(0,0)[r]{ 0.4}}
\put(1419.0,786.0){\rule[-0.200pt]{4.818pt}{0.400pt}}
\put(171.0,859.0){\rule[-0.200pt]{4.818pt}{0.400pt}}
\put(151,859){\makebox(0,0)[r]{ 0.6}}
\put(1419.0,859.0){\rule[-0.200pt]{4.818pt}{0.400pt}}
\put(171.0,131.0){\rule[-0.200pt]{0.400pt}{4.818pt}}
\put(171,90){\makebox(0,0){ 0}}
\put(171.0,839.0){\rule[-0.200pt]{0.400pt}{4.818pt}}
\put(330.0,131.0){\rule[-0.200pt]{0.400pt}{4.818pt}}
\put(330,90){\makebox(0,0){ 0.5}}
\put(330.0,839.0){\rule[-0.200pt]{0.400pt}{4.818pt}}
\put(488.0,131.0){\rule[-0.200pt]{0.400pt}{4.818pt}}
\put(488,90){\makebox(0,0){ 1}}
\put(488.0,839.0){\rule[-0.200pt]{0.400pt}{4.818pt}}
\put(647.0,131.0){\rule[-0.200pt]{0.400pt}{4.818pt}}
\put(647,90){\makebox(0,0){ 1.5}}
\put(647.0,839.0){\rule[-0.200pt]{0.400pt}{4.818pt}}
\put(805.0,131.0){\rule[-0.200pt]{0.400pt}{4.818pt}}
\put(805,90){\makebox(0,0){ 2}}
\put(805.0,839.0){\rule[-0.200pt]{0.400pt}{4.818pt}}
\put(964.0,131.0){\rule[-0.200pt]{0.400pt}{4.818pt}}
\put(964,90){\makebox(0,0){ 2.5}}
\put(964.0,839.0){\rule[-0.200pt]{0.400pt}{4.818pt}}
\put(1122.0,131.0){\rule[-0.200pt]{0.400pt}{4.818pt}}
\put(1122,90){\makebox(0,0){ 3}}
\put(1122.0,839.0){\rule[-0.200pt]{0.400pt}{4.818pt}}
\put(1281.0,131.0){\rule[-0.200pt]{0.400pt}{4.818pt}}
\put(1281,90){\makebox(0,0){ 3.5}}
\put(1281.0,839.0){\rule[-0.200pt]{0.400pt}{4.818pt}}
\put(1439.0,131.0){\rule[-0.200pt]{0.400pt}{4.818pt}}
\put(1439,90){\makebox(0,0){ 4}}
\put(1439.0,839.0){\rule[-0.200pt]{0.400pt}{4.818pt}}
\put(171.0,131.0){\rule[-0.200pt]{0.400pt}{175.375pt}}
\put(171.0,131.0){\rule[-0.200pt]{305.461pt}{0.400pt}}
\put(1439.0,131.0){\rule[-0.200pt]{0.400pt}{175.375pt}}
\put(171.0,859.0){\rule[-0.200pt]{305.461pt}{0.400pt}}
\put(30,495){\makebox(0,0){\popi{x}{mm}}}
\put(805,29){\makebox(0,0){\popi{t}{s}}}
\put(1279,377){\makebox(0,0)[r]{$I="1.7 A"$}}
\put(1299.0,377.0){\rule[-0.200pt]{24.090pt}{0.400pt}}
\put(171,140){\usebox{\plotpoint}}
\put(187,138.67){\rule{1.927pt}{0.400pt}}
\multiput(187.00,139.17)(4.000,-1.000){2}{\rule{0.964pt}{0.400pt}}
\put(171.0,140.0){\rule[-0.200pt]{3.854pt}{0.400pt}}
\put(234,138.67){\rule{1.927pt}{0.400pt}}
\multiput(234.00,138.17)(4.000,1.000){2}{\rule{0.964pt}{0.400pt}}
\put(195.0,139.0){\rule[-0.200pt]{9.395pt}{0.400pt}}
\put(250,139.67){\rule{1.927pt}{0.400pt}}
\multiput(250.00,139.17)(4.000,1.000){2}{\rule{0.964pt}{0.400pt}}
\put(258,140.67){\rule{1.927pt}{0.400pt}}
\multiput(258.00,140.17)(4.000,1.000){2}{\rule{0.964pt}{0.400pt}}
\put(266,141.67){\rule{1.927pt}{0.400pt}}
\multiput(266.00,141.17)(4.000,1.000){2}{\rule{0.964pt}{0.400pt}}
\put(274,142.67){\rule{1.927pt}{0.400pt}}
\multiput(274.00,142.17)(4.000,1.000){2}{\rule{0.964pt}{0.400pt}}
\multiput(282.00,144.59)(0.671,0.482){9}{\rule{0.633pt}{0.116pt}}
\multiput(282.00,143.17)(6.685,6.000){2}{\rule{0.317pt}{0.400pt}}
\multiput(290.59,150.00)(0.488,0.560){13}{\rule{0.117pt}{0.550pt}}
\multiput(289.17,150.00)(8.000,7.858){2}{\rule{0.400pt}{0.275pt}}
\multiput(298.59,159.00)(0.488,0.758){13}{\rule{0.117pt}{0.700pt}}
\multiput(297.17,159.00)(8.000,10.547){2}{\rule{0.400pt}{0.350pt}}
\multiput(306.59,171.00)(0.488,1.022){13}{\rule{0.117pt}{0.900pt}}
\multiput(305.17,171.00)(8.000,14.132){2}{\rule{0.400pt}{0.450pt}}
\multiput(314.59,187.00)(0.488,1.220){13}{\rule{0.117pt}{1.050pt}}
\multiput(313.17,187.00)(8.000,16.821){2}{\rule{0.400pt}{0.525pt}}
\multiput(322.59,206.00)(0.488,1.352){13}{\rule{0.117pt}{1.150pt}}
\multiput(321.17,206.00)(8.000,18.613){2}{\rule{0.400pt}{0.575pt}}
\multiput(330.59,227.00)(0.485,1.789){11}{\rule{0.117pt}{1.471pt}}
\multiput(329.17,227.00)(7.000,20.946){2}{\rule{0.400pt}{0.736pt}}
\multiput(337.59,251.00)(0.488,1.748){13}{\rule{0.117pt}{1.450pt}}
\multiput(336.17,251.00)(8.000,23.990){2}{\rule{0.400pt}{0.725pt}}
\multiput(345.59,278.00)(0.488,1.748){13}{\rule{0.117pt}{1.450pt}}
\multiput(344.17,278.00)(8.000,23.990){2}{\rule{0.400pt}{0.725pt}}
\multiput(353.59,305.00)(0.488,1.880){13}{\rule{0.117pt}{1.550pt}}
\multiput(352.17,305.00)(8.000,25.783){2}{\rule{0.400pt}{0.775pt}}
\multiput(361.59,334.00)(0.488,2.013){13}{\rule{0.117pt}{1.650pt}}
\multiput(360.17,334.00)(8.000,27.575){2}{\rule{0.400pt}{0.825pt}}
\multiput(369.59,365.00)(0.488,2.013){13}{\rule{0.117pt}{1.650pt}}
\multiput(368.17,365.00)(8.000,27.575){2}{\rule{0.400pt}{0.825pt}}
\multiput(377.59,396.00)(0.488,2.145){13}{\rule{0.117pt}{1.750pt}}
\multiput(376.17,396.00)(8.000,29.368){2}{\rule{0.400pt}{0.875pt}}
\multiput(385.59,429.00)(0.488,2.145){13}{\rule{0.117pt}{1.750pt}}
\multiput(384.17,429.00)(8.000,29.368){2}{\rule{0.400pt}{0.875pt}}
\multiput(393.59,462.00)(0.488,2.079){13}{\rule{0.117pt}{1.700pt}}
\multiput(392.17,462.00)(8.000,28.472){2}{\rule{0.400pt}{0.850pt}}
\multiput(401.59,494.00)(0.488,2.079){13}{\rule{0.117pt}{1.700pt}}
\multiput(400.17,494.00)(8.000,28.472){2}{\rule{0.400pt}{0.850pt}}
\multiput(409.59,526.00)(0.488,2.079){13}{\rule{0.117pt}{1.700pt}}
\multiput(408.17,526.00)(8.000,28.472){2}{\rule{0.400pt}{0.850pt}}
\multiput(417.59,558.00)(0.488,2.013){13}{\rule{0.117pt}{1.650pt}}
\multiput(416.17,558.00)(8.000,27.575){2}{\rule{0.400pt}{0.825pt}}
\multiput(425.59,589.00)(0.488,1.947){13}{\rule{0.117pt}{1.600pt}}
\multiput(424.17,589.00)(8.000,26.679){2}{\rule{0.400pt}{0.800pt}}
\multiput(433.59,619.00)(0.485,2.171){11}{\rule{0.117pt}{1.757pt}}
\multiput(432.17,619.00)(7.000,25.353){2}{\rule{0.400pt}{0.879pt}}
\multiput(440.59,648.00)(0.488,1.682){13}{\rule{0.117pt}{1.400pt}}
\multiput(439.17,648.00)(8.000,23.094){2}{\rule{0.400pt}{0.700pt}}
\multiput(448.59,674.00)(0.488,1.682){13}{\rule{0.117pt}{1.400pt}}
\multiput(447.17,674.00)(8.000,23.094){2}{\rule{0.400pt}{0.700pt}}
\multiput(456.59,700.00)(0.488,1.484){13}{\rule{0.117pt}{1.250pt}}
\multiput(455.17,700.00)(8.000,20.406){2}{\rule{0.400pt}{0.625pt}}
\multiput(464.59,723.00)(0.488,1.418){13}{\rule{0.117pt}{1.200pt}}
\multiput(463.17,723.00)(8.000,19.509){2}{\rule{0.400pt}{0.600pt}}
\multiput(472.59,745.00)(0.488,1.220){13}{\rule{0.117pt}{1.050pt}}
\multiput(471.17,745.00)(8.000,16.821){2}{\rule{0.400pt}{0.525pt}}
\multiput(480.59,764.00)(0.488,1.220){13}{\rule{0.117pt}{1.050pt}}
\multiput(479.17,764.00)(8.000,16.821){2}{\rule{0.400pt}{0.525pt}}
\multiput(488.59,783.00)(0.488,1.022){13}{\rule{0.117pt}{0.900pt}}
\multiput(487.17,783.00)(8.000,14.132){2}{\rule{0.400pt}{0.450pt}}
\multiput(496.59,799.00)(0.488,0.824){13}{\rule{0.117pt}{0.750pt}}
\multiput(495.17,799.00)(8.000,11.443){2}{\rule{0.400pt}{0.375pt}}
\multiput(504.59,812.00)(0.488,0.758){13}{\rule{0.117pt}{0.700pt}}
\multiput(503.17,812.00)(8.000,10.547){2}{\rule{0.400pt}{0.350pt}}
\multiput(512.59,824.00)(0.488,0.626){13}{\rule{0.117pt}{0.600pt}}
\multiput(511.17,824.00)(8.000,8.755){2}{\rule{0.400pt}{0.300pt}}
\multiput(520.00,834.59)(0.569,0.485){11}{\rule{0.557pt}{0.117pt}}
\multiput(520.00,833.17)(6.844,7.000){2}{\rule{0.279pt}{0.400pt}}
\multiput(528.00,841.59)(0.671,0.482){9}{\rule{0.633pt}{0.116pt}}
\multiput(528.00,840.17)(6.685,6.000){2}{\rule{0.317pt}{0.400pt}}
\multiput(536.00,847.60)(0.920,0.468){5}{\rule{0.800pt}{0.113pt}}
\multiput(536.00,846.17)(5.340,4.000){2}{\rule{0.400pt}{0.400pt}}
\put(543,851.17){\rule{1.700pt}{0.400pt}}
\multiput(543.00,850.17)(4.472,2.000){2}{\rule{0.850pt}{0.400pt}}
\put(242.0,140.0){\rule[-0.200pt]{1.927pt}{0.400pt}}
\put(559,851.67){\rule{1.927pt}{0.400pt}}
\multiput(559.00,852.17)(4.000,-1.000){2}{\rule{0.964pt}{0.400pt}}
\multiput(567.00,850.95)(1.579,-0.447){3}{\rule{1.167pt}{0.108pt}}
\multiput(567.00,851.17)(5.579,-3.000){2}{\rule{0.583pt}{0.400pt}}
\multiput(575.00,847.93)(0.821,-0.477){7}{\rule{0.740pt}{0.115pt}}
\multiput(575.00,848.17)(6.464,-5.000){2}{\rule{0.370pt}{0.400pt}}
\multiput(583.00,842.93)(0.671,-0.482){9}{\rule{0.633pt}{0.116pt}}
\multiput(583.00,843.17)(6.685,-6.000){2}{\rule{0.317pt}{0.400pt}}
\multiput(591.00,836.93)(0.569,-0.485){11}{\rule{0.557pt}{0.117pt}}
\multiput(591.00,837.17)(6.844,-7.000){2}{\rule{0.279pt}{0.400pt}}
\multiput(599.59,828.72)(0.488,-0.560){13}{\rule{0.117pt}{0.550pt}}
\multiput(598.17,829.86)(8.000,-7.858){2}{\rule{0.400pt}{0.275pt}}
\multiput(607.59,819.72)(0.488,-0.560){13}{\rule{0.117pt}{0.550pt}}
\multiput(606.17,820.86)(8.000,-7.858){2}{\rule{0.400pt}{0.275pt}}
\multiput(615.59,810.51)(0.488,-0.626){13}{\rule{0.117pt}{0.600pt}}
\multiput(614.17,811.75)(8.000,-8.755){2}{\rule{0.400pt}{0.300pt}}
\multiput(623.59,800.09)(0.488,-0.758){13}{\rule{0.117pt}{0.700pt}}
\multiput(622.17,801.55)(8.000,-10.547){2}{\rule{0.400pt}{0.350pt}}
\multiput(631.59,788.30)(0.488,-0.692){13}{\rule{0.117pt}{0.650pt}}
\multiput(630.17,789.65)(8.000,-9.651){2}{\rule{0.400pt}{0.325pt}}
\multiput(639.59,776.89)(0.488,-0.824){13}{\rule{0.117pt}{0.750pt}}
\multiput(638.17,778.44)(8.000,-11.443){2}{\rule{0.400pt}{0.375pt}}
\multiput(647.59,763.74)(0.485,-0.874){11}{\rule{0.117pt}{0.786pt}}
\multiput(646.17,765.37)(7.000,-10.369){2}{\rule{0.400pt}{0.393pt}}
\multiput(654.59,751.89)(0.488,-0.824){13}{\rule{0.117pt}{0.750pt}}
\multiput(653.17,753.44)(8.000,-11.443){2}{\rule{0.400pt}{0.375pt}}
\multiput(662.59,738.89)(0.488,-0.824){13}{\rule{0.117pt}{0.750pt}}
\multiput(661.17,740.44)(8.000,-11.443){2}{\rule{0.400pt}{0.375pt}}
\multiput(670.59,725.89)(0.488,-0.824){13}{\rule{0.117pt}{0.750pt}}
\multiput(669.17,727.44)(8.000,-11.443){2}{\rule{0.400pt}{0.375pt}}
\multiput(678.59,712.89)(0.488,-0.824){13}{\rule{0.117pt}{0.750pt}}
\multiput(677.17,714.44)(8.000,-11.443){2}{\rule{0.400pt}{0.375pt}}
\multiput(686.59,700.09)(0.488,-0.758){13}{\rule{0.117pt}{0.700pt}}
\multiput(685.17,701.55)(8.000,-10.547){2}{\rule{0.400pt}{0.350pt}}
\multiput(694.59,688.09)(0.488,-0.758){13}{\rule{0.117pt}{0.700pt}}
\multiput(693.17,689.55)(8.000,-10.547){2}{\rule{0.400pt}{0.350pt}}
\multiput(702.59,676.09)(0.488,-0.758){13}{\rule{0.117pt}{0.700pt}}
\multiput(701.17,677.55)(8.000,-10.547){2}{\rule{0.400pt}{0.350pt}}
\multiput(710.59,664.30)(0.488,-0.692){13}{\rule{0.117pt}{0.650pt}}
\multiput(709.17,665.65)(8.000,-9.651){2}{\rule{0.400pt}{0.325pt}}
\multiput(718.59,653.30)(0.488,-0.692){13}{\rule{0.117pt}{0.650pt}}
\multiput(717.17,654.65)(8.000,-9.651){2}{\rule{0.400pt}{0.325pt}}
\multiput(726.59,642.51)(0.488,-0.626){13}{\rule{0.117pt}{0.600pt}}
\multiput(725.17,643.75)(8.000,-8.755){2}{\rule{0.400pt}{0.300pt}}
\multiput(734.59,632.51)(0.488,-0.626){13}{\rule{0.117pt}{0.600pt}}
\multiput(733.17,633.75)(8.000,-8.755){2}{\rule{0.400pt}{0.300pt}}
\multiput(742.00,623.93)(0.494,-0.488){13}{\rule{0.500pt}{0.117pt}}
\multiput(742.00,624.17)(6.962,-8.000){2}{\rule{0.250pt}{0.400pt}}
\multiput(750.59,614.69)(0.485,-0.569){11}{\rule{0.117pt}{0.557pt}}
\multiput(749.17,615.84)(7.000,-6.844){2}{\rule{0.400pt}{0.279pt}}
\multiput(757.00,607.93)(0.569,-0.485){11}{\rule{0.557pt}{0.117pt}}
\multiput(757.00,608.17)(6.844,-7.000){2}{\rule{0.279pt}{0.400pt}}
\multiput(765.00,600.93)(0.569,-0.485){11}{\rule{0.557pt}{0.117pt}}
\multiput(765.00,601.17)(6.844,-7.000){2}{\rule{0.279pt}{0.400pt}}
\multiput(773.00,593.93)(0.821,-0.477){7}{\rule{0.740pt}{0.115pt}}
\multiput(773.00,594.17)(6.464,-5.000){2}{\rule{0.370pt}{0.400pt}}
\multiput(781.00,588.93)(0.821,-0.477){7}{\rule{0.740pt}{0.115pt}}
\multiput(781.00,589.17)(6.464,-5.000){2}{\rule{0.370pt}{0.400pt}}
\multiput(789.00,583.94)(1.066,-0.468){5}{\rule{0.900pt}{0.113pt}}
\multiput(789.00,584.17)(6.132,-4.000){2}{\rule{0.450pt}{0.400pt}}
\multiput(797.00,579.95)(1.579,-0.447){3}{\rule{1.167pt}{0.108pt}}
\multiput(797.00,580.17)(5.579,-3.000){2}{\rule{0.583pt}{0.400pt}}
\put(805,576.17){\rule{1.700pt}{0.400pt}}
\multiput(805.00,577.17)(4.472,-2.000){2}{\rule{0.850pt}{0.400pt}}
\put(813,574.17){\rule{1.700pt}{0.400pt}}
\multiput(813.00,575.17)(4.472,-2.000){2}{\rule{0.850pt}{0.400pt}}
\put(551.0,853.0){\rule[-0.200pt]{1.927pt}{0.400pt}}
\put(845,573.67){\rule{1.927pt}{0.400pt}}
\multiput(845.00,573.17)(4.000,1.000){2}{\rule{0.964pt}{0.400pt}}
\put(853,575.17){\rule{1.500pt}{0.400pt}}
\multiput(853.00,574.17)(3.887,2.000){2}{\rule{0.750pt}{0.400pt}}
\put(860,577.17){\rule{1.700pt}{0.400pt}}
\multiput(860.00,576.17)(4.472,2.000){2}{\rule{0.850pt}{0.400pt}}
\multiput(868.00,579.61)(1.579,0.447){3}{\rule{1.167pt}{0.108pt}}
\multiput(868.00,578.17)(5.579,3.000){2}{\rule{0.583pt}{0.400pt}}
\multiput(876.00,582.61)(1.579,0.447){3}{\rule{1.167pt}{0.108pt}}
\multiput(876.00,581.17)(5.579,3.000){2}{\rule{0.583pt}{0.400pt}}
\multiput(884.00,585.60)(1.066,0.468){5}{\rule{0.900pt}{0.113pt}}
\multiput(884.00,584.17)(6.132,4.000){2}{\rule{0.450pt}{0.400pt}}
\multiput(892.00,589.60)(1.066,0.468){5}{\rule{0.900pt}{0.113pt}}
\multiput(892.00,588.17)(6.132,4.000){2}{\rule{0.450pt}{0.400pt}}
\multiput(900.00,593.60)(1.066,0.468){5}{\rule{0.900pt}{0.113pt}}
\multiput(900.00,592.17)(6.132,4.000){2}{\rule{0.450pt}{0.400pt}}
\multiput(908.00,597.60)(1.066,0.468){5}{\rule{0.900pt}{0.113pt}}
\multiput(908.00,596.17)(6.132,4.000){2}{\rule{0.450pt}{0.400pt}}
\multiput(916.00,601.59)(0.821,0.477){7}{\rule{0.740pt}{0.115pt}}
\multiput(916.00,600.17)(6.464,5.000){2}{\rule{0.370pt}{0.400pt}}
\multiput(924.00,606.60)(1.066,0.468){5}{\rule{0.900pt}{0.113pt}}
\multiput(924.00,605.17)(6.132,4.000){2}{\rule{0.450pt}{0.400pt}}
\multiput(932.00,610.59)(0.821,0.477){7}{\rule{0.740pt}{0.115pt}}
\multiput(932.00,609.17)(6.464,5.000){2}{\rule{0.370pt}{0.400pt}}
\multiput(940.00,615.60)(1.066,0.468){5}{\rule{0.900pt}{0.113pt}}
\multiput(940.00,614.17)(6.132,4.000){2}{\rule{0.450pt}{0.400pt}}
\multiput(948.00,619.59)(0.821,0.477){7}{\rule{0.740pt}{0.115pt}}
\multiput(948.00,618.17)(6.464,5.000){2}{\rule{0.370pt}{0.400pt}}
\multiput(956.00,624.59)(0.821,0.477){7}{\rule{0.740pt}{0.115pt}}
\multiput(956.00,623.17)(6.464,5.000){2}{\rule{0.370pt}{0.400pt}}
\multiput(964.00,629.60)(0.920,0.468){5}{\rule{0.800pt}{0.113pt}}
\multiput(964.00,628.17)(5.340,4.000){2}{\rule{0.400pt}{0.400pt}}
\multiput(971.00,633.59)(0.821,0.477){7}{\rule{0.740pt}{0.115pt}}
\multiput(971.00,632.17)(6.464,5.000){2}{\rule{0.370pt}{0.400pt}}
\multiput(979.00,638.60)(1.066,0.468){5}{\rule{0.900pt}{0.113pt}}
\multiput(979.00,637.17)(6.132,4.000){2}{\rule{0.450pt}{0.400pt}}
\multiput(987.00,642.60)(1.066,0.468){5}{\rule{0.900pt}{0.113pt}}
\multiput(987.00,641.17)(6.132,4.000){2}{\rule{0.450pt}{0.400pt}}
\multiput(995.00,646.60)(1.066,0.468){5}{\rule{0.900pt}{0.113pt}}
\multiput(995.00,645.17)(6.132,4.000){2}{\rule{0.450pt}{0.400pt}}
\multiput(1003.00,650.60)(1.066,0.468){5}{\rule{0.900pt}{0.113pt}}
\multiput(1003.00,649.17)(6.132,4.000){2}{\rule{0.450pt}{0.400pt}}
\multiput(1011.00,654.60)(1.066,0.468){5}{\rule{0.900pt}{0.113pt}}
\multiput(1011.00,653.17)(6.132,4.000){2}{\rule{0.450pt}{0.400pt}}
\multiput(1019.00,658.61)(1.579,0.447){3}{\rule{1.167pt}{0.108pt}}
\multiput(1019.00,657.17)(5.579,3.000){2}{\rule{0.583pt}{0.400pt}}
\multiput(1027.00,661.61)(1.579,0.447){3}{\rule{1.167pt}{0.108pt}}
\multiput(1027.00,660.17)(5.579,3.000){2}{\rule{0.583pt}{0.400pt}}
\put(1035,664.17){\rule{1.700pt}{0.400pt}}
\multiput(1035.00,663.17)(4.472,2.000){2}{\rule{0.850pt}{0.400pt}}
\multiput(1043.00,666.61)(1.579,0.447){3}{\rule{1.167pt}{0.108pt}}
\multiput(1043.00,665.17)(5.579,3.000){2}{\rule{0.583pt}{0.400pt}}
\put(1051,668.67){\rule{1.927pt}{0.400pt}}
\multiput(1051.00,668.17)(4.000,1.000){2}{\rule{0.964pt}{0.400pt}}
\multiput(1059.00,670.61)(1.579,0.447){3}{\rule{1.167pt}{0.108pt}}
\multiput(1059.00,669.17)(5.579,3.000){2}{\rule{0.583pt}{0.400pt}}
\put(1067,672.67){\rule{1.686pt}{0.400pt}}
\multiput(1067.00,672.17)(3.500,1.000){2}{\rule{0.843pt}{0.400pt}}
\put(1074,673.67){\rule{1.927pt}{0.400pt}}
\multiput(1074.00,673.17)(4.000,1.000){2}{\rule{0.964pt}{0.400pt}}
\put(1082,674.67){\rule{1.927pt}{0.400pt}}
\multiput(1082.00,674.17)(4.000,1.000){2}{\rule{0.964pt}{0.400pt}}
\put(821.0,574.0){\rule[-0.200pt]{5.782pt}{0.400pt}}
\put(1130,674.67){\rule{1.927pt}{0.400pt}}
\multiput(1130.00,675.17)(4.000,-1.000){2}{\rule{0.964pt}{0.400pt}}
\put(1138,673.67){\rule{1.927pt}{0.400pt}}
\multiput(1138.00,674.17)(4.000,-1.000){2}{\rule{0.964pt}{0.400pt}}
\put(1146,672.67){\rule{1.927pt}{0.400pt}}
\multiput(1146.00,673.17)(4.000,-1.000){2}{\rule{0.964pt}{0.400pt}}
\put(1154,671.67){\rule{1.927pt}{0.400pt}}
\multiput(1154.00,672.17)(4.000,-1.000){2}{\rule{0.964pt}{0.400pt}}
\put(1162,670.17){\rule{1.700pt}{0.400pt}}
\multiput(1162.00,671.17)(4.472,-2.000){2}{\rule{0.850pt}{0.400pt}}
\put(1170,668.67){\rule{1.686pt}{0.400pt}}
\multiput(1170.00,669.17)(3.500,-1.000){2}{\rule{0.843pt}{0.400pt}}
\put(1177,667.67){\rule{1.927pt}{0.400pt}}
\multiput(1177.00,668.17)(4.000,-1.000){2}{\rule{0.964pt}{0.400pt}}
\put(1185,666.17){\rule{1.700pt}{0.400pt}}
\multiput(1185.00,667.17)(4.472,-2.000){2}{\rule{0.850pt}{0.400pt}}
\put(1193,664.67){\rule{1.927pt}{0.400pt}}
\multiput(1193.00,665.17)(4.000,-1.000){2}{\rule{0.964pt}{0.400pt}}
\put(1201,663.67){\rule{1.927pt}{0.400pt}}
\multiput(1201.00,664.17)(4.000,-1.000){2}{\rule{0.964pt}{0.400pt}}
\put(1209,662.17){\rule{1.700pt}{0.400pt}}
\multiput(1209.00,663.17)(4.472,-2.000){2}{\rule{0.850pt}{0.400pt}}
\put(1217,660.67){\rule{1.927pt}{0.400pt}}
\multiput(1217.00,661.17)(4.000,-1.000){2}{\rule{0.964pt}{0.400pt}}
\put(1225,659.17){\rule{1.700pt}{0.400pt}}
\multiput(1225.00,660.17)(4.472,-2.000){2}{\rule{0.850pt}{0.400pt}}
\put(1233,657.17){\rule{1.700pt}{0.400pt}}
\multiput(1233.00,658.17)(4.472,-2.000){2}{\rule{0.850pt}{0.400pt}}
\put(1241,655.67){\rule{1.927pt}{0.400pt}}
\multiput(1241.00,656.17)(4.000,-1.000){2}{\rule{0.964pt}{0.400pt}}
\put(1249,654.17){\rule{1.700pt}{0.400pt}}
\multiput(1249.00,655.17)(4.472,-2.000){2}{\rule{0.850pt}{0.400pt}}
\put(1257,652.67){\rule{1.927pt}{0.400pt}}
\multiput(1257.00,653.17)(4.000,-1.000){2}{\rule{0.964pt}{0.400pt}}
\put(1265,651.67){\rule{1.927pt}{0.400pt}}
\multiput(1265.00,652.17)(4.000,-1.000){2}{\rule{0.964pt}{0.400pt}}
\put(1273,650.67){\rule{1.927pt}{0.400pt}}
\multiput(1273.00,651.17)(4.000,-1.000){2}{\rule{0.964pt}{0.400pt}}
\put(1281,649.67){\rule{1.686pt}{0.400pt}}
\multiput(1281.00,650.17)(3.500,-1.000){2}{\rule{0.843pt}{0.400pt}}
\put(1288,648.67){\rule{1.927pt}{0.400pt}}
\multiput(1288.00,649.17)(4.000,-1.000){2}{\rule{0.964pt}{0.400pt}}
\put(1296,647.67){\rule{1.927pt}{0.400pt}}
\multiput(1296.00,648.17)(4.000,-1.000){2}{\rule{0.964pt}{0.400pt}}
\put(1304,646.67){\rule{1.927pt}{0.400pt}}
\multiput(1304.00,647.17)(4.000,-1.000){2}{\rule{0.964pt}{0.400pt}}
\put(1312,645.67){\rule{1.927pt}{0.400pt}}
\multiput(1312.00,646.17)(4.000,-1.000){2}{\rule{0.964pt}{0.400pt}}
\put(1320,644.67){\rule{1.927pt}{0.400pt}}
\multiput(1320.00,645.17)(4.000,-1.000){2}{\rule{0.964pt}{0.400pt}}
\put(1090.0,676.0){\rule[-0.200pt]{9.636pt}{0.400pt}}
\put(1344,643.67){\rule{1.927pt}{0.400pt}}
\multiput(1344.00,644.17)(4.000,-1.000){2}{\rule{0.964pt}{0.400pt}}
\put(1328.0,645.0){\rule[-0.200pt]{3.854pt}{0.400pt}}
\put(1368,642.67){\rule{1.927pt}{0.400pt}}
\multiput(1368.00,643.17)(4.000,-1.000){2}{\rule{0.964pt}{0.400pt}}
\put(1352.0,644.0){\rule[-0.200pt]{3.854pt}{0.400pt}}
\put(1423,642.67){\rule{1.927pt}{0.400pt}}
\multiput(1423.00,642.17)(4.000,1.000){2}{\rule{0.964pt}{0.400pt}}
\put(1376.0,643.0){\rule[-0.200pt]{11.322pt}{0.400pt}}
\put(1431.0,644.0){\rule[-0.200pt]{1.927pt}{0.400pt}}
\put(1279,336){\makebox(0,0)[r]{$I="1.8 A"$}}
\multiput(1299,336)(20.756,0.000){5}{\usebox{\plotpoint}}
\put(1399,336){\usebox{\plotpoint}}
\put(171,146){\usebox{\plotpoint}}
\put(171.00,146.00){\usebox{\plotpoint}}
\put(191.69,147.00){\usebox{\plotpoint}}
\put(212.38,145.83){\usebox{\plotpoint}}
\put(233.01,144.00){\usebox{\plotpoint}}
\put(253.64,142.00){\usebox{\plotpoint}}
\put(274.40,142.00){\usebox{\plotpoint}}
\put(295.15,142.00){\usebox{\plotpoint}}
\put(315.91,142.00){\usebox{\plotpoint}}
\put(336.66,142.00){\usebox{\plotpoint}}
\put(353.96,152.20){\usebox{\plotpoint}}
\put(365.97,169.08){\usebox{\plotpoint}}
\put(375.38,187.56){\usebox{\plotpoint}}
\put(383.58,206.62){\usebox{\plotpoint}}
\put(391.08,225.97){\usebox{\plotpoint}}
\put(397.86,245.58){\usebox{\plotpoint}}
\put(404.18,265.35){\usebox{\plotpoint}}
\multiput(409,281)(5.896,19.900){2}{\usebox{\plotpoint}}
\put(421.87,325.05){\usebox{\plotpoint}}
\multiput(425,336)(5.348,20.055){2}{\usebox{\plotpoint}}
\put(437.51,385.31){\usebox{\plotpoint}}
\multiput(440,396)(5.186,20.097){2}{\usebox{\plotpoint}}
\put(452.82,445.66){\usebox{\plotpoint}}
\multiput(456,458)(5.348,20.055){2}{\usebox{\plotpoint}}
\put(468.62,505.89){\usebox{\plotpoint}}
\multiput(472,519)(5.348,20.055){2}{\usebox{\plotpoint}}
\put(484.70,566.04){\usebox{\plotpoint}}
\multiput(488,578)(5.702,19.957){2}{\usebox{\plotpoint}}
\put(502.10,625.83){\usebox{\plotpoint}}
\put(508.36,645.62){\usebox{\plotpoint}}
\put(514.78,665.35){\usebox{\plotpoint}}
\put(521.52,684.98){\usebox{\plotpoint}}
\put(528.94,704.36){\usebox{\plotpoint}}
\put(536.64,723.64){\usebox{\plotpoint}}
\put(544.43,742.86){\usebox{\plotpoint}}
\put(554.01,761.27){\usebox{\plotpoint}}
\put(564.61,779.11){\usebox{\plotpoint}}
\put(577.25,795.53){\usebox{\plotpoint}}
\put(592.17,809.88){\usebox{\plotpoint}}
\put(610.57,818.89){\usebox{\plotpoint}}
\put(631.12,818.97){\usebox{\plotpoint}}
\put(650.00,810.86){\usebox{\plotpoint}}
\put(666.82,798.78){\usebox{\plotpoint}}
\put(681.69,784.31){\usebox{\plotpoint}}
\put(695.74,769.04){\usebox{\plotpoint}}
\put(709.08,753.15){\usebox{\plotpoint}}
\put(722.05,736.94){\usebox{\plotpoint}}
\put(735.01,720.73){\usebox{\plotpoint}}
\put(747.98,704.53){\usebox{\plotpoint}}
\put(760.32,687.85){\usebox{\plotpoint}}
\put(773.81,672.09){\usebox{\plotpoint}}
\put(788.02,656.98){\usebox{\plotpoint}}
\put(803.07,642.69){\usebox{\plotpoint}}
\put(818.69,629.02){\usebox{\plotpoint}}
\put(836.00,617.63){\usebox{\plotpoint}}
\put(854.55,608.34){\usebox{\plotpoint}}
\put(874.09,601.48){\usebox{\plotpoint}}
\put(894.52,598.68){\usebox{\plotpoint}}
\put(915.18,598.90){\usebox{\plotpoint}}
\put(935.65,601.46){\usebox{\plotpoint}}
\put(955.77,605.97){\usebox{\plotpoint}}
\put(975.40,612.65){\usebox{\plotpoint}}
\put(994.83,619.94){\usebox{\plotpoint}}
\put(1014.27,627.23){\usebox{\plotpoint}}
\put(1033.33,635.37){\usebox{\plotpoint}}
\put(1053.04,641.77){\usebox{\plotpoint}}
\put(1072.62,648.61){\usebox{\plotpoint}}
\put(1092.46,654.61){\usebox{\plotpoint}}
\put(1112.77,658.69){\usebox{\plotpoint}}
\put(1133.15,662.39){\usebox{\plotpoint}}
\put(1153.81,664.00){\usebox{\plotpoint}}
\put(1174.50,665.00){\usebox{\plotpoint}}
\put(1195.26,665.00){\usebox{\plotpoint}}
\put(1215.96,664.13){\usebox{\plotpoint}}
\put(1236.62,662.55){\usebox{\plotpoint}}
\put(1257.28,661.00){\usebox{\plotpoint}}
\put(1277.93,659.38){\usebox{\plotpoint}}
\put(1298.53,657.00){\usebox{\plotpoint}}
\put(1319.17,655.10){\usebox{\plotpoint}}
\put(1339.82,653.52){\usebox{\plotpoint}}
\put(1360.48,652.00){\usebox{\plotpoint}}
\put(1381.14,650.36){\usebox{\plotpoint}}
\put(1401.85,649.64){\usebox{\plotpoint}}
\put(1422.56,649.00){\usebox{\plotpoint}}
\put(1439,649){\usebox{\plotpoint}}
\sbox{\plotpoint}{\rule[-0.400pt]{0.800pt}{0.800pt}}%
\sbox{\plotpoint}{\rule[-0.200pt]{0.400pt}{0.400pt}}%
\put(1279,295){\makebox(0,0)[r]{$I="2.5 A"$}}
\sbox{\plotpoint}{\rule[-0.400pt]{0.800pt}{0.800pt}}%
\put(1299.0,295.0){\rule[-0.400pt]{24.090pt}{0.800pt}}
\put(171,197){\usebox{\plotpoint}}
\put(171,195.84){\rule{1.927pt}{0.800pt}}
\multiput(171.00,195.34)(4.000,1.000){2}{\rule{0.964pt}{0.800pt}}
\put(187,196.84){\rule{1.927pt}{0.800pt}}
\multiput(187.00,196.34)(4.000,1.000){2}{\rule{0.964pt}{0.800pt}}
\multiput(195.00,200.38)(0.928,0.560){3}{\rule{1.480pt}{0.135pt}}
\multiput(195.00,197.34)(4.928,5.000){2}{\rule{0.740pt}{0.800pt}}
\multiput(203.00,205.40)(0.481,0.520){9}{\rule{1.000pt}{0.125pt}}
\multiput(203.00,202.34)(5.924,8.000){2}{\rule{0.500pt}{0.800pt}}
\multiput(212.40,212.00)(0.520,0.627){9}{\rule{0.125pt}{1.200pt}}
\multiput(209.34,212.00)(8.000,7.509){2}{\rule{0.800pt}{0.600pt}}
\multiput(220.40,222.00)(0.526,1.000){7}{\rule{0.127pt}{1.686pt}}
\multiput(217.34,222.00)(7.000,9.501){2}{\rule{0.800pt}{0.843pt}}
\multiput(227.40,235.00)(0.520,0.920){9}{\rule{0.125pt}{1.600pt}}
\multiput(224.34,235.00)(8.000,10.679){2}{\rule{0.800pt}{0.800pt}}
\multiput(235.40,249.00)(0.520,1.066){9}{\rule{0.125pt}{1.800pt}}
\multiput(232.34,249.00)(8.000,12.264){2}{\rule{0.800pt}{0.900pt}}
\multiput(243.40,265.00)(0.520,1.139){9}{\rule{0.125pt}{1.900pt}}
\multiput(240.34,265.00)(8.000,13.056){2}{\rule{0.800pt}{0.950pt}}
\multiput(251.40,282.00)(0.520,1.285){9}{\rule{0.125pt}{2.100pt}}
\multiput(248.34,282.00)(8.000,14.641){2}{\rule{0.800pt}{1.050pt}}
\multiput(259.40,301.00)(0.520,1.285){9}{\rule{0.125pt}{2.100pt}}
\multiput(256.34,301.00)(8.000,14.641){2}{\rule{0.800pt}{1.050pt}}
\multiput(267.40,320.00)(0.520,1.358){9}{\rule{0.125pt}{2.200pt}}
\multiput(264.34,320.00)(8.000,15.434){2}{\rule{0.800pt}{1.100pt}}
\multiput(275.40,340.00)(0.520,1.432){9}{\rule{0.125pt}{2.300pt}}
\multiput(272.34,340.00)(8.000,16.226){2}{\rule{0.800pt}{1.150pt}}
\multiput(283.40,361.00)(0.520,1.432){9}{\rule{0.125pt}{2.300pt}}
\multiput(280.34,361.00)(8.000,16.226){2}{\rule{0.800pt}{1.150pt}}
\multiput(291.40,382.00)(0.520,1.432){9}{\rule{0.125pt}{2.300pt}}
\multiput(288.34,382.00)(8.000,16.226){2}{\rule{0.800pt}{1.150pt}}
\multiput(299.40,403.00)(0.520,1.358){9}{\rule{0.125pt}{2.200pt}}
\multiput(296.34,403.00)(8.000,15.434){2}{\rule{0.800pt}{1.100pt}}
\multiput(307.40,423.00)(0.520,1.432){9}{\rule{0.125pt}{2.300pt}}
\multiput(304.34,423.00)(8.000,16.226){2}{\rule{0.800pt}{1.150pt}}
\multiput(315.40,444.00)(0.520,1.358){9}{\rule{0.125pt}{2.200pt}}
\multiput(312.34,444.00)(8.000,15.434){2}{\rule{0.800pt}{1.100pt}}
\multiput(323.40,464.00)(0.520,1.358){9}{\rule{0.125pt}{2.200pt}}
\multiput(320.34,464.00)(8.000,15.434){2}{\rule{0.800pt}{1.100pt}}
\multiput(331.40,484.00)(0.526,1.526){7}{\rule{0.127pt}{2.371pt}}
\multiput(328.34,484.00)(7.000,14.078){2}{\rule{0.800pt}{1.186pt}}
\multiput(338.40,503.00)(0.520,1.212){9}{\rule{0.125pt}{2.000pt}}
\multiput(335.34,503.00)(8.000,13.849){2}{\rule{0.800pt}{1.000pt}}
\multiput(346.40,521.00)(0.520,1.139){9}{\rule{0.125pt}{1.900pt}}
\multiput(343.34,521.00)(8.000,13.056){2}{\rule{0.800pt}{0.950pt}}
\multiput(354.40,538.00)(0.520,1.139){9}{\rule{0.125pt}{1.900pt}}
\multiput(351.34,538.00)(8.000,13.056){2}{\rule{0.800pt}{0.950pt}}
\multiput(362.40,555.00)(0.520,0.993){9}{\rule{0.125pt}{1.700pt}}
\multiput(359.34,555.00)(8.000,11.472){2}{\rule{0.800pt}{0.850pt}}
\multiput(370.40,570.00)(0.520,0.993){9}{\rule{0.125pt}{1.700pt}}
\multiput(367.34,570.00)(8.000,11.472){2}{\rule{0.800pt}{0.850pt}}
\multiput(378.40,585.00)(0.520,0.847){9}{\rule{0.125pt}{1.500pt}}
\multiput(375.34,585.00)(8.000,9.887){2}{\rule{0.800pt}{0.750pt}}
\multiput(386.40,598.00)(0.520,0.847){9}{\rule{0.125pt}{1.500pt}}
\multiput(383.34,598.00)(8.000,9.887){2}{\rule{0.800pt}{0.750pt}}
\multiput(394.40,611.00)(0.520,0.700){9}{\rule{0.125pt}{1.300pt}}
\multiput(391.34,611.00)(8.000,8.302){2}{\rule{0.800pt}{0.650pt}}
\multiput(402.40,622.00)(0.520,0.627){9}{\rule{0.125pt}{1.200pt}}
\multiput(399.34,622.00)(8.000,7.509){2}{\rule{0.800pt}{0.600pt}}
\multiput(409.00,633.40)(0.481,0.520){9}{\rule{1.000pt}{0.125pt}}
\multiput(409.00,630.34)(5.924,8.000){2}{\rule{0.500pt}{0.800pt}}
\multiput(417.00,641.40)(0.481,0.520){9}{\rule{1.000pt}{0.125pt}}
\multiput(417.00,638.34)(5.924,8.000){2}{\rule{0.500pt}{0.800pt}}
\multiput(425.00,649.40)(0.562,0.526){7}{\rule{1.114pt}{0.127pt}}
\multiput(425.00,646.34)(5.687,7.000){2}{\rule{0.557pt}{0.800pt}}
\multiput(433.00,656.39)(0.574,0.536){5}{\rule{1.133pt}{0.129pt}}
\multiput(433.00,653.34)(4.648,6.000){2}{\rule{0.567pt}{0.800pt}}
\multiput(440.00,662.38)(0.928,0.560){3}{\rule{1.480pt}{0.135pt}}
\multiput(440.00,659.34)(4.928,5.000){2}{\rule{0.740pt}{0.800pt}}
\put(448,666.34){\rule{1.800pt}{0.800pt}}
\multiput(448.00,664.34)(4.264,4.000){2}{\rule{0.900pt}{0.800pt}}
\put(456,670.34){\rule{1.800pt}{0.800pt}}
\multiput(456.00,668.34)(4.264,4.000){2}{\rule{0.900pt}{0.800pt}}
\put(464,673.84){\rule{1.927pt}{0.800pt}}
\multiput(464.00,672.34)(4.000,3.000){2}{\rule{0.964pt}{0.800pt}}
\put(472,676.84){\rule{1.927pt}{0.800pt}}
\multiput(472.00,675.34)(4.000,3.000){2}{\rule{0.964pt}{0.800pt}}
\put(480,678.84){\rule{1.927pt}{0.800pt}}
\multiput(480.00,678.34)(4.000,1.000){2}{\rule{0.964pt}{0.800pt}}
\put(488,679.84){\rule{1.927pt}{0.800pt}}
\multiput(488.00,679.34)(4.000,1.000){2}{\rule{0.964pt}{0.800pt}}
\put(496,680.84){\rule{1.927pt}{0.800pt}}
\multiput(496.00,680.34)(4.000,1.000){2}{\rule{0.964pt}{0.800pt}}
\put(179.0,198.0){\rule[-0.400pt]{1.927pt}{0.800pt}}
\put(520,680.84){\rule{1.927pt}{0.800pt}}
\multiput(520.00,681.34)(4.000,-1.000){2}{\rule{0.964pt}{0.800pt}}
\put(504.0,683.0){\rule[-0.400pt]{3.854pt}{0.800pt}}
\put(536,679.84){\rule{1.686pt}{0.800pt}}
\multiput(536.00,680.34)(3.500,-1.000){2}{\rule{0.843pt}{0.800pt}}
\put(543,678.84){\rule{1.927pt}{0.800pt}}
\multiput(543.00,679.34)(4.000,-1.000){2}{\rule{0.964pt}{0.800pt}}
\put(551,677.84){\rule{1.927pt}{0.800pt}}
\multiput(551.00,678.34)(4.000,-1.000){2}{\rule{0.964pt}{0.800pt}}
\put(559,676.34){\rule{1.927pt}{0.800pt}}
\multiput(559.00,677.34)(4.000,-2.000){2}{\rule{0.964pt}{0.800pt}}
\put(567,674.84){\rule{1.927pt}{0.800pt}}
\multiput(567.00,675.34)(4.000,-1.000){2}{\rule{0.964pt}{0.800pt}}
\put(575,673.84){\rule{1.927pt}{0.800pt}}
\multiput(575.00,674.34)(4.000,-1.000){2}{\rule{0.964pt}{0.800pt}}
\put(583,672.34){\rule{1.927pt}{0.800pt}}
\multiput(583.00,673.34)(4.000,-2.000){2}{\rule{0.964pt}{0.800pt}}
\put(591,670.84){\rule{1.927pt}{0.800pt}}
\multiput(591.00,671.34)(4.000,-1.000){2}{\rule{0.964pt}{0.800pt}}
\put(599,669.34){\rule{1.927pt}{0.800pt}}
\multiput(599.00,670.34)(4.000,-2.000){2}{\rule{0.964pt}{0.800pt}}
\put(607,667.84){\rule{1.927pt}{0.800pt}}
\multiput(607.00,668.34)(4.000,-1.000){2}{\rule{0.964pt}{0.800pt}}
\put(615,666.84){\rule{1.927pt}{0.800pt}}
\multiput(615.00,667.34)(4.000,-1.000){2}{\rule{0.964pt}{0.800pt}}
\put(623,665.34){\rule{1.927pt}{0.800pt}}
\multiput(623.00,666.34)(4.000,-2.000){2}{\rule{0.964pt}{0.800pt}}
\put(631,663.84){\rule{1.927pt}{0.800pt}}
\multiput(631.00,664.34)(4.000,-1.000){2}{\rule{0.964pt}{0.800pt}}
\put(639,662.84){\rule{1.927pt}{0.800pt}}
\multiput(639.00,663.34)(4.000,-1.000){2}{\rule{0.964pt}{0.800pt}}
\put(647,661.34){\rule{1.686pt}{0.800pt}}
\multiput(647.00,662.34)(3.500,-2.000){2}{\rule{0.843pt}{0.800pt}}
\put(654,659.84){\rule{1.927pt}{0.800pt}}
\multiput(654.00,660.34)(4.000,-1.000){2}{\rule{0.964pt}{0.800pt}}
\put(662,658.84){\rule{1.927pt}{0.800pt}}
\multiput(662.00,659.34)(4.000,-1.000){2}{\rule{0.964pt}{0.800pt}}
\put(670,657.34){\rule{1.927pt}{0.800pt}}
\multiput(670.00,658.34)(4.000,-2.000){2}{\rule{0.964pt}{0.800pt}}
\put(678,655.84){\rule{1.927pt}{0.800pt}}
\multiput(678.00,656.34)(4.000,-1.000){2}{\rule{0.964pt}{0.800pt}}
\put(686,654.84){\rule{1.927pt}{0.800pt}}
\multiput(686.00,655.34)(4.000,-1.000){2}{\rule{0.964pt}{0.800pt}}
\put(694,653.84){\rule{1.927pt}{0.800pt}}
\multiput(694.00,654.34)(4.000,-1.000){2}{\rule{0.964pt}{0.800pt}}
\put(702,652.84){\rule{1.927pt}{0.800pt}}
\multiput(702.00,653.34)(4.000,-1.000){2}{\rule{0.964pt}{0.800pt}}
\put(528.0,682.0){\rule[-0.400pt]{1.927pt}{0.800pt}}
\put(718,651.84){\rule{1.927pt}{0.800pt}}
\multiput(718.00,652.34)(4.000,-1.000){2}{\rule{0.964pt}{0.800pt}}
\put(726,650.84){\rule{1.927pt}{0.800pt}}
\multiput(726.00,651.34)(4.000,-1.000){2}{\rule{0.964pt}{0.800pt}}
\put(710.0,654.0){\rule[-0.400pt]{1.927pt}{0.800pt}}
\put(742,649.84){\rule{1.927pt}{0.800pt}}
\multiput(742.00,650.34)(4.000,-1.000){2}{\rule{0.964pt}{0.800pt}}
\put(750,648.84){\rule{1.686pt}{0.800pt}}
\multiput(750.00,649.34)(3.500,-1.000){2}{\rule{0.843pt}{0.800pt}}
\put(734.0,652.0){\rule[-0.400pt]{1.927pt}{0.800pt}}
\put(773,647.84){\rule{1.927pt}{0.800pt}}
\multiput(773.00,648.34)(4.000,-1.000){2}{\rule{0.964pt}{0.800pt}}
\put(757.0,650.0){\rule[-0.400pt]{3.854pt}{0.800pt}}
\put(813,646.84){\rule{1.927pt}{0.800pt}}
\multiput(813.00,647.34)(4.000,-1.000){2}{\rule{0.964pt}{0.800pt}}
\put(781.0,649.0){\rule[-0.400pt]{7.709pt}{0.800pt}}
\put(821.0,648.0){\rule[-0.400pt]{148.876pt}{0.800pt}}
\sbox{\plotpoint}{\rule[-0.500pt]{1.000pt}{1.000pt}}%
\sbox{\plotpoint}{\rule[-0.200pt]{0.400pt}{0.400pt}}%
\put(1279,254){\makebox(0,0)[r]{$I="2.6 A"$}}
\sbox{\plotpoint}{\rule[-0.500pt]{1.000pt}{1.000pt}}%
\multiput(1299,254)(20.756,0.000){5}{\usebox{\plotpoint}}
\put(1399,254){\usebox{\plotpoint}}
\put(171,618){\usebox{\plotpoint}}
\put(171.00,618.00){\usebox{\plotpoint}}
\put(182.75,600.91){\usebox{\plotpoint}}
\put(192.95,582.84){\usebox{\plotpoint}}
\put(201.98,564.16){\usebox{\plotpoint}}
\put(210.82,545.38){\usebox{\plotpoint}}
\put(218.89,526.26){\usebox{\plotpoint}}
\put(225.76,506.67){\usebox{\plotpoint}}
\put(233.44,487.39){\usebox{\plotpoint}}
\put(240.85,468.01){\usebox{\plotpoint}}
\put(248.24,448.61){\usebox{\plotpoint}}
\put(255.63,429.21){\usebox{\plotpoint}}
\put(263.02,409.82){\usebox{\plotpoint}}
\put(270.23,390.36){\usebox{\plotpoint}}
\put(277.20,370.81){\usebox{\plotpoint}}
\put(284.10,351.23){\usebox{\plotpoint}}
\put(291.24,331.74){\usebox{\plotpoint}}
\put(298.72,312.38){\usebox{\plotpoint}}
\put(307.40,293.55){\usebox{\plotpoint}}
\put(318.14,275.80){\usebox{\plotpoint}}
\put(330.10,258.84){\usebox{\plotpoint}}
\put(340.70,240.99){\usebox{\plotpoint}}
\put(351.57,223.32){\usebox{\plotpoint}}
\put(363.95,206.69){\usebox{\plotpoint}}
\put(379.95,193.90){\usebox{\plotpoint}}
\put(400.24,190.10){\usebox{\plotpoint}}
\put(420.90,191.49){\usebox{\plotpoint}}
\put(441.39,189.48){\usebox{\plotpoint}}
\put(460.88,182.56){\usebox{\plotpoint}}
\put(479.79,174.08){\usebox{\plotpoint}}
\put(499.08,166.46){\usebox{\plotpoint}}
\put(518.80,160.30){\usebox{\plotpoint}}
\put(539.45,161.00){\usebox{\plotpoint}}
\put(559.38,155.81){\usebox{\plotpoint}}
\put(578.58,148.11){\usebox{\plotpoint}}
\put(598.89,144.03){\usebox{\plotpoint}}
\put(619.12,139.49){\usebox{\plotpoint}}
\put(639.72,137.00){\usebox{\plotpoint}}
\put(660.35,138.79){\usebox{\plotpoint}}
\put(681.07,138.62){\usebox{\plotpoint}}
\put(701.67,136.04){\usebox{\plotpoint}}
\put(722.39,135.45){\usebox{\plotpoint}}
\put(742.10,140.05){\usebox{\plotpoint}}
\put(761.30,147.00){\usebox{\plotpoint}}
\put(781.99,145.88){\usebox{\plotpoint}}
\put(802.40,142.32){\usebox{\plotpoint}}
\put(819.50,152.50){\usebox{\plotpoint}}
\put(831.21,169.59){\usebox{\plotpoint}}
\put(841.34,187.68){\usebox{\plotpoint}}
\put(850.11,206.49){\usebox{\plotpoint}}
\put(857.71,225.79){\usebox{\plotpoint}}
\put(865.25,245.12){\usebox{\plotpoint}}
\put(872.96,264.40){\usebox{\plotpoint}}
\put(880.29,283.81){\usebox{\plotpoint}}
\put(887.53,303.26){\usebox{\plotpoint}}
\put(894.80,322.70){\usebox{\plotpoint}}
\put(901.89,342.21){\usebox{\plotpoint}}
\put(909.03,361.70){\usebox{\plotpoint}}
\multiput(916,380)(7.093,19.506){2}{\usebox{\plotpoint}}
\put(930.86,420.01){\usebox{\plotpoint}}
\put(938.25,439.41){\usebox{\plotpoint}}
\put(945.88,458.71){\usebox{\plotpoint}}
\put(953.84,477.87){\usebox{\plotpoint}}
\put(962.17,496.88){\usebox{\plotpoint}}
\put(969.89,516.15){\usebox{\plotpoint}}
\put(978.91,534.83){\usebox{\plotpoint}}
\put(988.33,553.32){\usebox{\plotpoint}}
\put(998.62,571.34){\usebox{\plotpoint}}
\put(1009.62,588.93){\usebox{\plotpoint}}
\put(1021.74,605.77){\usebox{\plotpoint}}
\put(1034.85,621.83){\usebox{\plotpoint}}
\put(1050.00,636.00){\usebox{\plotpoint}}
\put(1066.92,647.95){\usebox{\plotpoint}}
\put(1084.92,658.10){\usebox{\plotpoint}}
\put(1104.59,664.65){\usebox{\plotpoint}}
\put(1124.97,668.37){\usebox{\plotpoint}}
\put(1145.62,670.00){\usebox{\plotpoint}}
\put(1166.38,670.00){\usebox{\plotpoint}}
\put(1187.00,668.00){\usebox{\plotpoint}}
\put(1207.64,666.17){\usebox{\plotpoint}}
\put(1228.26,664.00){\usebox{\plotpoint}}
\put(1248.71,661.04){\usebox{\plotpoint}}
\put(1269.37,659.45){\usebox{\plotpoint}}
\put(1289.96,657.00){\usebox{\plotpoint}}
\put(1310.65,656.00){\usebox{\plotpoint}}
\put(1331.28,654.00){\usebox{\plotpoint}}
\put(1351.98,653.00){\usebox{\plotpoint}}
\put(1372.67,652.00){\usebox{\plotpoint}}
\put(1393.43,652.00){\usebox{\plotpoint}}
\put(1414.12,651.00){\usebox{\plotpoint}}
\put(1434.81,650.00){\usebox{\plotpoint}}
\put(1439,650){\usebox{\plotpoint}}
\sbox{\plotpoint}{\rule[-0.600pt]{1.200pt}{1.200pt}}%
\sbox{\plotpoint}{\rule[-0.200pt]{0.400pt}{0.400pt}}%
\put(1279,213){\makebox(0,0)[r]{$I="2 A"$}}
\sbox{\plotpoint}{\rule[-0.600pt]{1.200pt}{1.200pt}}%
\put(1299.0,213.0){\rule[-0.600pt]{24.090pt}{1.200pt}}
\put(171,230){\usebox{\plotpoint}}
\multiput(173.24,230.00)(0.503,1.034){6}{\rule{0.121pt}{2.850pt}}
\multiput(168.51,230.00)(8.000,11.085){2}{\rule{1.200pt}{1.425pt}}
\multiput(181.24,247.00)(0.503,1.260){6}{\rule{0.121pt}{3.300pt}}
\multiput(176.51,247.00)(8.000,13.151){2}{\rule{1.200pt}{1.650pt}}
\multiput(189.24,267.00)(0.503,1.336){6}{\rule{0.121pt}{3.450pt}}
\multiput(184.51,267.00)(8.000,13.839){2}{\rule{1.200pt}{1.725pt}}
\multiput(197.24,288.00)(0.503,1.487){6}{\rule{0.121pt}{3.750pt}}
\multiput(192.51,288.00)(8.000,15.217){2}{\rule{1.200pt}{1.875pt}}
\multiput(205.24,311.00)(0.503,1.562){6}{\rule{0.121pt}{3.900pt}}
\multiput(200.51,311.00)(8.000,15.905){2}{\rule{1.200pt}{1.950pt}}
\multiput(213.24,335.00)(0.503,1.638){6}{\rule{0.121pt}{4.050pt}}
\multiput(208.51,335.00)(8.000,16.594){2}{\rule{1.200pt}{2.025pt}}
\multiput(221.24,360.00)(0.505,2.180){4}{\rule{0.122pt}{4.929pt}}
\multiput(216.51,360.00)(7.000,16.771){2}{\rule{1.200pt}{2.464pt}}
\multiput(228.24,387.00)(0.503,1.713){6}{\rule{0.121pt}{4.200pt}}
\multiput(223.51,387.00)(8.000,17.283){2}{\rule{1.200pt}{2.100pt}}
\multiput(236.24,413.00)(0.503,1.713){6}{\rule{0.121pt}{4.200pt}}
\multiput(231.51,413.00)(8.000,17.283){2}{\rule{1.200pt}{2.100pt}}
\multiput(244.24,439.00)(0.503,1.789){6}{\rule{0.121pt}{4.350pt}}
\multiput(239.51,439.00)(8.000,17.971){2}{\rule{1.200pt}{2.175pt}}
\multiput(252.24,466.00)(0.503,1.713){6}{\rule{0.121pt}{4.200pt}}
\multiput(247.51,466.00)(8.000,17.283){2}{\rule{1.200pt}{2.100pt}}
\multiput(260.24,492.00)(0.503,1.713){6}{\rule{0.121pt}{4.200pt}}
\multiput(255.51,492.00)(8.000,17.283){2}{\rule{1.200pt}{2.100pt}}
\multiput(268.24,518.00)(0.503,1.713){6}{\rule{0.121pt}{4.200pt}}
\multiput(263.51,518.00)(8.000,17.283){2}{\rule{1.200pt}{2.100pt}}
\multiput(276.24,544.00)(0.503,1.638){6}{\rule{0.121pt}{4.050pt}}
\multiput(271.51,544.00)(8.000,16.594){2}{\rule{1.200pt}{2.025pt}}
\multiput(284.24,569.00)(0.503,1.487){6}{\rule{0.121pt}{3.750pt}}
\multiput(279.51,569.00)(8.000,15.217){2}{\rule{1.200pt}{1.875pt}}
\multiput(292.24,592.00)(0.503,1.411){6}{\rule{0.121pt}{3.600pt}}
\multiput(287.51,592.00)(8.000,14.528){2}{\rule{1.200pt}{1.800pt}}
\multiput(300.24,614.00)(0.503,1.336){6}{\rule{0.121pt}{3.450pt}}
\multiput(295.51,614.00)(8.000,13.839){2}{\rule{1.200pt}{1.725pt}}
\multiput(308.24,635.00)(0.503,1.185){6}{\rule{0.121pt}{3.150pt}}
\multiput(303.51,635.00)(8.000,12.462){2}{\rule{1.200pt}{1.575pt}}
\multiput(316.24,654.00)(0.503,1.109){6}{\rule{0.121pt}{3.000pt}}
\multiput(311.51,654.00)(8.000,11.773){2}{\rule{1.200pt}{1.500pt}}
\multiput(324.24,672.00)(0.503,1.034){6}{\rule{0.121pt}{2.850pt}}
\multiput(319.51,672.00)(8.000,11.085){2}{\rule{1.200pt}{1.425pt}}
\multiput(332.24,689.00)(0.505,1.027){4}{\rule{0.122pt}{2.871pt}}
\multiput(327.51,689.00)(7.000,9.040){2}{\rule{1.200pt}{1.436pt}}
\multiput(339.24,704.00)(0.503,0.732){6}{\rule{0.121pt}{2.250pt}}
\multiput(334.51,704.00)(8.000,8.330){2}{\rule{1.200pt}{1.125pt}}
\multiput(347.24,717.00)(0.503,0.581){6}{\rule{0.121pt}{1.950pt}}
\multiput(342.51,717.00)(8.000,6.953){2}{\rule{1.200pt}{0.975pt}}
\multiput(355.24,728.00)(0.503,0.581){6}{\rule{0.121pt}{1.950pt}}
\multiput(350.51,728.00)(8.000,6.953){2}{\rule{1.200pt}{0.975pt}}
\multiput(361.00,741.24)(0.355,0.503){6}{\rule{1.500pt}{0.121pt}}
\multiput(361.00,736.51)(4.887,8.000){2}{\rule{0.750pt}{1.200pt}}
\multiput(369.00,749.24)(0.354,0.505){4}{\rule{1.671pt}{0.122pt}}
\multiput(369.00,744.51)(4.531,7.000){2}{\rule{0.836pt}{1.200pt}}
\multiput(377.00,756.24)(0.113,0.509){2}{\rule{1.900pt}{0.123pt}}
\multiput(377.00,751.51)(4.056,6.000){2}{\rule{0.950pt}{1.200pt}}
\put(385,759.51){\rule{1.927pt}{1.200pt}}
\multiput(385.00,757.51)(4.000,4.000){2}{\rule{0.964pt}{1.200pt}}
\put(393,763.01){\rule{1.927pt}{1.200pt}}
\multiput(393.00,761.51)(4.000,3.000){2}{\rule{0.964pt}{1.200pt}}
\put(401,765.51){\rule{1.927pt}{1.200pt}}
\multiput(401.00,764.51)(4.000,2.000){2}{\rule{0.964pt}{1.200pt}}
\put(425,766.01){\rule{1.927pt}{1.200pt}}
\multiput(425.00,766.51)(4.000,-1.000){2}{\rule{0.964pt}{1.200pt}}
\put(433,764.01){\rule{1.686pt}{1.200pt}}
\multiput(433.00,765.51)(3.500,-3.000){2}{\rule{0.843pt}{1.200pt}}
\put(440,761.01){\rule{1.927pt}{1.200pt}}
\multiput(440.00,762.51)(4.000,-3.000){2}{\rule{0.964pt}{1.200pt}}
\put(448,758.01){\rule{1.927pt}{1.200pt}}
\multiput(448.00,759.51)(4.000,-3.000){2}{\rule{0.964pt}{1.200pt}}
\put(456,754.01){\rule{1.927pt}{1.200pt}}
\multiput(456.00,756.51)(4.000,-5.000){2}{\rule{0.964pt}{1.200pt}}
\put(464,749.01){\rule{1.927pt}{1.200pt}}
\multiput(464.00,751.51)(4.000,-5.000){2}{\rule{0.964pt}{1.200pt}}
\put(472,744.01){\rule{1.927pt}{1.200pt}}
\multiput(472.00,746.51)(4.000,-5.000){2}{\rule{0.964pt}{1.200pt}}
\multiput(480.00,741.26)(0.354,-0.505){4}{\rule{1.671pt}{0.122pt}}
\multiput(480.00,741.51)(4.531,-7.000){2}{\rule{0.836pt}{1.200pt}}
\put(488,732.01){\rule{1.927pt}{1.200pt}}
\multiput(488.00,734.51)(4.000,-5.000){2}{\rule{0.964pt}{1.200pt}}
\multiput(496.00,729.26)(0.354,-0.505){4}{\rule{1.671pt}{0.122pt}}
\multiput(496.00,729.51)(4.531,-7.000){2}{\rule{0.836pt}{1.200pt}}
\multiput(504.00,722.26)(0.354,-0.505){4}{\rule{1.671pt}{0.122pt}}
\multiput(504.00,722.51)(4.531,-7.000){2}{\rule{0.836pt}{1.200pt}}
\multiput(512.00,715.25)(0.113,-0.509){2}{\rule{1.900pt}{0.123pt}}
\multiput(512.00,715.51)(4.056,-6.000){2}{\rule{0.950pt}{1.200pt}}
\multiput(520.00,709.26)(0.354,-0.505){4}{\rule{1.671pt}{0.122pt}}
\multiput(520.00,709.51)(4.531,-7.000){2}{\rule{0.836pt}{1.200pt}}
\multiput(528.00,702.26)(0.354,-0.505){4}{\rule{1.671pt}{0.122pt}}
\multiput(528.00,702.51)(4.531,-7.000){2}{\rule{0.836pt}{1.200pt}}
\put(536,692.51){\rule{1.686pt}{1.200pt}}
\multiput(536.00,695.51)(3.500,-6.000){2}{\rule{0.843pt}{1.200pt}}
\multiput(543.00,689.25)(0.113,-0.509){2}{\rule{1.900pt}{0.123pt}}
\multiput(543.00,689.51)(4.056,-6.000){2}{\rule{0.950pt}{1.200pt}}
\multiput(551.00,683.25)(0.113,-0.509){2}{\rule{1.900pt}{0.123pt}}
\multiput(551.00,683.51)(4.056,-6.000){2}{\rule{0.950pt}{1.200pt}}
\multiput(559.00,677.25)(0.113,-0.509){2}{\rule{1.900pt}{0.123pt}}
\multiput(559.00,677.51)(4.056,-6.000){2}{\rule{0.950pt}{1.200pt}}
\multiput(567.00,671.25)(0.113,-0.509){2}{\rule{1.900pt}{0.123pt}}
\multiput(567.00,671.51)(4.056,-6.000){2}{\rule{0.950pt}{1.200pt}}
\multiput(575.00,665.25)(0.113,-0.509){2}{\rule{1.900pt}{0.123pt}}
\multiput(575.00,665.51)(4.056,-6.000){2}{\rule{0.950pt}{1.200pt}}
\put(583,657.01){\rule{1.927pt}{1.200pt}}
\multiput(583.00,659.51)(4.000,-5.000){2}{\rule{0.964pt}{1.200pt}}
\put(591,652.01){\rule{1.927pt}{1.200pt}}
\multiput(591.00,654.51)(4.000,-5.000){2}{\rule{0.964pt}{1.200pt}}
\put(599,647.51){\rule{1.927pt}{1.200pt}}
\multiput(599.00,649.51)(4.000,-4.000){2}{\rule{0.964pt}{1.200pt}}
\put(607,643.51){\rule{1.927pt}{1.200pt}}
\multiput(607.00,645.51)(4.000,-4.000){2}{\rule{0.964pt}{1.200pt}}
\put(615,639.51){\rule{1.927pt}{1.200pt}}
\multiput(615.00,641.51)(4.000,-4.000){2}{\rule{0.964pt}{1.200pt}}
\put(623,636.01){\rule{1.927pt}{1.200pt}}
\multiput(623.00,637.51)(4.000,-3.000){2}{\rule{0.964pt}{1.200pt}}
\put(631,633.01){\rule{1.927pt}{1.200pt}}
\multiput(631.00,634.51)(4.000,-3.000){2}{\rule{0.964pt}{1.200pt}}
\put(639,630.01){\rule{1.927pt}{1.200pt}}
\multiput(639.00,631.51)(4.000,-3.000){2}{\rule{0.964pt}{1.200pt}}
\put(647,627.51){\rule{1.686pt}{1.200pt}}
\multiput(647.00,628.51)(3.500,-2.000){2}{\rule{0.843pt}{1.200pt}}
\put(654,625.51){\rule{1.927pt}{1.200pt}}
\multiput(654.00,626.51)(4.000,-2.000){2}{\rule{0.964pt}{1.200pt}}
\put(662,623.51){\rule{1.927pt}{1.200pt}}
\multiput(662.00,624.51)(4.000,-2.000){2}{\rule{0.964pt}{1.200pt}}
\put(670,622.01){\rule{1.927pt}{1.200pt}}
\multiput(670.00,622.51)(4.000,-1.000){2}{\rule{0.964pt}{1.200pt}}
\put(678,621.01){\rule{1.927pt}{1.200pt}}
\multiput(678.00,621.51)(4.000,-1.000){2}{\rule{0.964pt}{1.200pt}}
\put(686,620.01){\rule{1.927pt}{1.200pt}}
\multiput(686.00,620.51)(4.000,-1.000){2}{\rule{0.964pt}{1.200pt}}
\put(409.0,769.0){\rule[-0.600pt]{3.854pt}{1.200pt}}
\put(726,620.01){\rule{1.927pt}{1.200pt}}
\multiput(726.00,619.51)(4.000,1.000){2}{\rule{0.964pt}{1.200pt}}
\put(734,621.01){\rule{1.927pt}{1.200pt}}
\multiput(734.00,620.51)(4.000,1.000){2}{\rule{0.964pt}{1.200pt}}
\put(742,622.01){\rule{1.927pt}{1.200pt}}
\multiput(742.00,621.51)(4.000,1.000){2}{\rule{0.964pt}{1.200pt}}
\put(750,623.01){\rule{1.686pt}{1.200pt}}
\multiput(750.00,622.51)(3.500,1.000){2}{\rule{0.843pt}{1.200pt}}
\put(757,624.01){\rule{1.927pt}{1.200pt}}
\multiput(757.00,623.51)(4.000,1.000){2}{\rule{0.964pt}{1.200pt}}
\put(765,625.51){\rule{1.927pt}{1.200pt}}
\multiput(765.00,624.51)(4.000,2.000){2}{\rule{0.964pt}{1.200pt}}
\put(773,627.01){\rule{1.927pt}{1.200pt}}
\multiput(773.00,626.51)(4.000,1.000){2}{\rule{0.964pt}{1.200pt}}
\put(781,628.51){\rule{1.927pt}{1.200pt}}
\multiput(781.00,627.51)(4.000,2.000){2}{\rule{0.964pt}{1.200pt}}
\put(789,630.01){\rule{1.927pt}{1.200pt}}
\multiput(789.00,629.51)(4.000,1.000){2}{\rule{0.964pt}{1.200pt}}
\put(797,631.01){\rule{1.927pt}{1.200pt}}
\multiput(797.00,630.51)(4.000,1.000){2}{\rule{0.964pt}{1.200pt}}
\put(805,632.51){\rule{1.927pt}{1.200pt}}
\multiput(805.00,631.51)(4.000,2.000){2}{\rule{0.964pt}{1.200pt}}
\put(813,634.01){\rule{1.927pt}{1.200pt}}
\multiput(813.00,633.51)(4.000,1.000){2}{\rule{0.964pt}{1.200pt}}
\put(821,635.01){\rule{1.927pt}{1.200pt}}
\multiput(821.00,634.51)(4.000,1.000){2}{\rule{0.964pt}{1.200pt}}
\put(829,636.51){\rule{1.927pt}{1.200pt}}
\multiput(829.00,635.51)(4.000,2.000){2}{\rule{0.964pt}{1.200pt}}
\put(837,638.51){\rule{1.927pt}{1.200pt}}
\multiput(837.00,637.51)(4.000,2.000){2}{\rule{0.964pt}{1.200pt}}
\put(845,640.01){\rule{1.927pt}{1.200pt}}
\multiput(845.00,639.51)(4.000,1.000){2}{\rule{0.964pt}{1.200pt}}
\put(853,641.51){\rule{1.686pt}{1.200pt}}
\multiput(853.00,640.51)(3.500,2.000){2}{\rule{0.843pt}{1.200pt}}
\put(694.0,622.0){\rule[-0.600pt]{7.709pt}{1.200pt}}
\put(868,643.51){\rule{1.927pt}{1.200pt}}
\multiput(868.00,642.51)(4.000,2.000){2}{\rule{0.964pt}{1.200pt}}
\put(876,645.01){\rule{1.927pt}{1.200pt}}
\multiput(876.00,644.51)(4.000,1.000){2}{\rule{0.964pt}{1.200pt}}
\put(884,646.01){\rule{1.927pt}{1.200pt}}
\multiput(884.00,645.51)(4.000,1.000){2}{\rule{0.964pt}{1.200pt}}
\put(892,647.01){\rule{1.927pt}{1.200pt}}
\multiput(892.00,646.51)(4.000,1.000){2}{\rule{0.964pt}{1.200pt}}
\put(900,648.01){\rule{1.927pt}{1.200pt}}
\multiput(900.00,647.51)(4.000,1.000){2}{\rule{0.964pt}{1.200pt}}
\put(908,649.01){\rule{1.927pt}{1.200pt}}
\multiput(908.00,648.51)(4.000,1.000){2}{\rule{0.964pt}{1.200pt}}
\put(860.0,645.0){\rule[-0.600pt]{1.927pt}{1.200pt}}
\put(924,650.01){\rule{1.927pt}{1.200pt}}
\multiput(924.00,649.51)(4.000,1.000){2}{\rule{0.964pt}{1.200pt}}
\put(916.0,652.0){\rule[-0.600pt]{1.927pt}{1.200pt}}
\put(940,651.01){\rule{1.927pt}{1.200pt}}
\multiput(940.00,650.51)(4.000,1.000){2}{\rule{0.964pt}{1.200pt}}
\put(932.0,653.0){\rule[-0.600pt]{1.927pt}{1.200pt}}
\put(1051,651.01){\rule{1.927pt}{1.200pt}}
\multiput(1051.00,651.51)(4.000,-1.000){2}{\rule{0.964pt}{1.200pt}}
\put(948.0,654.0){\rule[-0.600pt]{24.813pt}{1.200pt}}
\put(1074,650.01){\rule{1.927pt}{1.200pt}}
\multiput(1074.00,650.51)(4.000,-1.000){2}{\rule{0.964pt}{1.200pt}}
\put(1059.0,653.0){\rule[-0.600pt]{3.613pt}{1.200pt}}
\put(1114,649.01){\rule{1.927pt}{1.200pt}}
\multiput(1114.00,649.51)(4.000,-1.000){2}{\rule{0.964pt}{1.200pt}}
\put(1082.0,652.0){\rule[-0.600pt]{7.709pt}{1.200pt}}
\put(1130,648.01){\rule{1.927pt}{1.200pt}}
\multiput(1130.00,648.51)(4.000,-1.000){2}{\rule{0.964pt}{1.200pt}}
\put(1122.0,651.0){\rule[-0.600pt]{1.927pt}{1.200pt}}
\put(1193,647.01){\rule{1.927pt}{1.200pt}}
\multiput(1193.00,647.51)(4.000,-1.000){2}{\rule{0.964pt}{1.200pt}}
\put(1138.0,650.0){\rule[-0.600pt]{13.249pt}{1.200pt}}
\put(1201.0,649.0){\rule[-0.600pt]{57.334pt}{1.200pt}}
\sbox{\plotpoint}{\rule[-0.500pt]{1.000pt}{1.000pt}}%
\sbox{\plotpoint}{\rule[-0.200pt]{0.400pt}{0.400pt}}%
\put(1279,172){\makebox(0,0)[r]{$I="3 A"$}}
\sbox{\plotpoint}{\rule[-0.500pt]{1.000pt}{1.000pt}}%
\multiput(1299,172)(41.511,0.000){3}{\usebox{\plotpoint}}
\put(1399,172){\usebox{\plotpoint}}
\put(171,240){\usebox{\plotpoint}}
\put(171.00,240.00){\usebox{\plotpoint}}
\put(211.91,244.46){\usebox{\plotpoint}}
\put(241.25,272.97){\usebox{\plotpoint}}
\put(263.97,307.70){\usebox{\plotpoint}}
\put(284.67,343.66){\usebox{\plotpoint}}
\put(304.83,379.95){\usebox{\plotpoint}}
\put(324.99,416.23){\usebox{\plotpoint}}
\put(345.26,452.42){\usebox{\plotpoint}}
\put(367.83,487.25){\usebox{\plotpoint}}
\put(393.12,520.15){\usebox{\plotpoint}}
\put(420.92,550.92){\usebox{\plotpoint}}
\put(451.84,578.40){\usebox{\plotpoint}}
\put(487.86,598.93){\usebox{\plotpoint}}
\put(526.62,613.48){\usebox{\plotpoint}}
\put(566.96,623.00){\usebox{\plotpoint}}
\put(607.98,629.00){\usebox{\plotpoint}}
\put(649.30,632.00){\usebox{\plotpoint}}
\put(690.69,634.00){\usebox{\plotpoint}}
\put(732.15,634.77){\usebox{\plotpoint}}
\put(773.58,636.00){\usebox{\plotpoint}}
\put(815.09,636.00){\usebox{\plotpoint}}
\put(856.60,636.00){\usebox{\plotpoint}}
\put(898.11,636.00){\usebox{\plotpoint}}
\put(939.62,636.00){\usebox{\plotpoint}}
\put(981.13,636.00){\usebox{\plotpoint}}
\put(1022.64,636.00){\usebox{\plotpoint}}
\put(1064.15,636.00){\usebox{\plotpoint}}
\put(1105.60,637.00){\usebox{\plotpoint}}
\put(1147.04,638.13){\usebox{\plotpoint}}
\put(1188.38,641.00){\usebox{\plotpoint}}
\put(1229.82,642.00){\usebox{\plotpoint}}
\put(1271.34,642.00){\usebox{\plotpoint}}
\put(1312.85,642.00){\usebox{\plotpoint}}
\put(1354.36,642.00){\usebox{\plotpoint}}
\put(1395.87,642.00){\usebox{\plotpoint}}
\put(1437.38,642.00){\usebox{\plotpoint}}
\put(1439,642){\usebox{\plotpoint}}
\sbox{\plotpoint}{\rule[-0.200pt]{0.400pt}{0.400pt}}%
\put(171.0,131.0){\rule[-0.200pt]{0.400pt}{175.375pt}}
\put(171.0,131.0){\rule[-0.200pt]{305.461pt}{0.400pt}}
\put(1439.0,131.0){\rule[-0.200pt]{0.400pt}{175.375pt}}
\put(171.0,859.0){\rule[-0.200pt]{305.461pt}{0.400pt}}
\end{picture}

\caption{Namerané časové závislosti výchylky $x$ pre jednotlivé prúdy $I$ pre polohovú podmienku}  \label{G_7}
\end{figure}



\subsection{Pohybová podmienka}

Pre pohybovú podmienku boli namerané dáta vynesené do grafu Obr. \ref{G_8}, tu nastáva útlm približne pre $I="\(2.8\pm0.2\) A"$
\begin{figure}
% GNUPLOT: LaTeX picture
\setlength{\unitlength}{0.240900pt}
\ifx\plotpoint\undefined\newsavebox{\plotpoint}\fi
\begin{picture}(1500,900)(0,0)
\sbox{\plotpoint}{\rule[-0.200pt]{0.400pt}{0.400pt}}%
\put(171.0,131.0){\rule[-0.200pt]{4.818pt}{0.400pt}}
\put(151,131){\makebox(0,0)[r]{-3}}
\put(1419.0,131.0){\rule[-0.200pt]{4.818pt}{0.400pt}}
\put(171.0,204.0){\rule[-0.200pt]{4.818pt}{0.400pt}}
\put(151,204){\makebox(0,0)[r]{-2.5}}
\put(1419.0,204.0){\rule[-0.200pt]{4.818pt}{0.400pt}}
\put(171.0,277.0){\rule[-0.200pt]{4.818pt}{0.400pt}}
\put(151,277){\makebox(0,0)[r]{-2}}
\put(1419.0,277.0){\rule[-0.200pt]{4.818pt}{0.400pt}}
\put(171.0,349.0){\rule[-0.200pt]{4.818pt}{0.400pt}}
\put(151,349){\makebox(0,0)[r]{-1.5}}
\put(1419.0,349.0){\rule[-0.200pt]{4.818pt}{0.400pt}}
\put(171.0,422.0){\rule[-0.200pt]{4.818pt}{0.400pt}}
\put(151,422){\makebox(0,0)[r]{-1}}
\put(1419.0,422.0){\rule[-0.200pt]{4.818pt}{0.400pt}}
\put(171.0,495.0){\rule[-0.200pt]{4.818pt}{0.400pt}}
\put(151,495){\makebox(0,0)[r]{-0.5}}
\put(1419.0,495.0){\rule[-0.200pt]{4.818pt}{0.400pt}}
\put(171.0,568.0){\rule[-0.200pt]{4.818pt}{0.400pt}}
\put(151,568){\makebox(0,0)[r]{ 0}}
\put(1419.0,568.0){\rule[-0.200pt]{4.818pt}{0.400pt}}
\put(171.0,641.0){\rule[-0.200pt]{4.818pt}{0.400pt}}
\put(151,641){\makebox(0,0)[r]{ 0.5}}
\put(1419.0,641.0){\rule[-0.200pt]{4.818pt}{0.400pt}}
\put(171.0,713.0){\rule[-0.200pt]{4.818pt}{0.400pt}}
\put(151,713){\makebox(0,0)[r]{ 1}}
\put(1419.0,713.0){\rule[-0.200pt]{4.818pt}{0.400pt}}
\put(171.0,786.0){\rule[-0.200pt]{4.818pt}{0.400pt}}
\put(151,786){\makebox(0,0)[r]{ 1.5}}
\put(1419.0,786.0){\rule[-0.200pt]{4.818pt}{0.400pt}}
\put(171.0,859.0){\rule[-0.200pt]{4.818pt}{0.400pt}}
\put(151,859){\makebox(0,0)[r]{ 2}}
\put(1419.0,859.0){\rule[-0.200pt]{4.818pt}{0.400pt}}
\put(171.0,131.0){\rule[-0.200pt]{0.400pt}{4.818pt}}
\put(171,90){\makebox(0,0){ 0}}
\put(171.0,839.0){\rule[-0.200pt]{0.400pt}{4.818pt}}
\put(330.0,131.0){\rule[-0.200pt]{0.400pt}{4.818pt}}
\put(330,90){\makebox(0,0){ 0.5}}
\put(330.0,839.0){\rule[-0.200pt]{0.400pt}{4.818pt}}
\put(488.0,131.0){\rule[-0.200pt]{0.400pt}{4.818pt}}
\put(488,90){\makebox(0,0){ 1}}
\put(488.0,839.0){\rule[-0.200pt]{0.400pt}{4.818pt}}
\put(647.0,131.0){\rule[-0.200pt]{0.400pt}{4.818pt}}
\put(647,90){\makebox(0,0){ 1.5}}
\put(647.0,839.0){\rule[-0.200pt]{0.400pt}{4.818pt}}
\put(805.0,131.0){\rule[-0.200pt]{0.400pt}{4.818pt}}
\put(805,90){\makebox(0,0){ 2}}
\put(805.0,839.0){\rule[-0.200pt]{0.400pt}{4.818pt}}
\put(964.0,131.0){\rule[-0.200pt]{0.400pt}{4.818pt}}
\put(964,90){\makebox(0,0){ 2.5}}
\put(964.0,839.0){\rule[-0.200pt]{0.400pt}{4.818pt}}
\put(1122.0,131.0){\rule[-0.200pt]{0.400pt}{4.818pt}}
\put(1122,90){\makebox(0,0){ 3}}
\put(1122.0,839.0){\rule[-0.200pt]{0.400pt}{4.818pt}}
\put(1281.0,131.0){\rule[-0.200pt]{0.400pt}{4.818pt}}
\put(1281,90){\makebox(0,0){ 3.5}}
\put(1281.0,839.0){\rule[-0.200pt]{0.400pt}{4.818pt}}
\put(1439.0,131.0){\rule[-0.200pt]{0.400pt}{4.818pt}}
\put(1439,90){\makebox(0,0){ 4}}
\put(1439.0,839.0){\rule[-0.200pt]{0.400pt}{4.818pt}}
\put(171.0,131.0){\rule[-0.200pt]{0.400pt}{175.375pt}}
\put(171.0,131.0){\rule[-0.200pt]{305.461pt}{0.400pt}}
\put(1439.0,131.0){\rule[-0.200pt]{0.400pt}{175.375pt}}
\put(171.0,859.0){\rule[-0.200pt]{305.461pt}{0.400pt}}
\put(30,495){\makebox(0,0){\popi{x}{mm}}}
\put(805,29){\makebox(0,0){\popi{t}{s}}}
\put(1279,418){\makebox(0,0)[r]{$I="0.5 A"$}}
\put(1299.0,418.0){\rule[-0.200pt]{24.090pt}{0.400pt}}
\put(171,571){\usebox{\plotpoint}}
\put(266,569.17){\rule{1.700pt}{0.400pt}}
\multiput(266.00,570.17)(4.472,-2.000){2}{\rule{0.850pt}{0.400pt}}
\multiput(274.59,566.72)(0.488,-0.560){13}{\rule{0.117pt}{0.550pt}}
\multiput(273.17,567.86)(8.000,-7.858){2}{\rule{0.400pt}{0.275pt}}
\multiput(282.59,555.64)(0.488,-1.220){13}{\rule{0.117pt}{1.050pt}}
\multiput(281.17,557.82)(8.000,-16.821){2}{\rule{0.400pt}{0.525pt}}
\multiput(290.59,535.60)(0.488,-1.550){13}{\rule{0.117pt}{1.300pt}}
\multiput(289.17,538.30)(8.000,-21.302){2}{\rule{0.400pt}{0.650pt}}
\multiput(298.59,511.60)(0.488,-1.550){13}{\rule{0.117pt}{1.300pt}}
\multiput(297.17,514.30)(8.000,-21.302){2}{\rule{0.400pt}{0.650pt}}
\multiput(306.59,487.81)(0.488,-1.484){13}{\rule{0.117pt}{1.250pt}}
\multiput(305.17,490.41)(8.000,-20.406){2}{\rule{0.400pt}{0.625pt}}
\multiput(314.59,464.81)(0.488,-1.484){13}{\rule{0.117pt}{1.250pt}}
\multiput(313.17,467.41)(8.000,-20.406){2}{\rule{0.400pt}{0.625pt}}
\multiput(322.59,442.23)(0.488,-1.352){13}{\rule{0.117pt}{1.150pt}}
\multiput(321.17,444.61)(8.000,-18.613){2}{\rule{0.400pt}{0.575pt}}
\multiput(330.59,420.84)(0.485,-1.484){11}{\rule{0.117pt}{1.243pt}}
\multiput(329.17,423.42)(7.000,-17.420){2}{\rule{0.400pt}{0.621pt}}
\multiput(337.59,401.85)(0.488,-1.154){13}{\rule{0.117pt}{1.000pt}}
\multiput(336.17,403.92)(8.000,-15.924){2}{\rule{0.400pt}{0.500pt}}
\multiput(345.59,384.06)(0.488,-1.088){13}{\rule{0.117pt}{0.950pt}}
\multiput(344.17,386.03)(8.000,-15.028){2}{\rule{0.400pt}{0.475pt}}
\multiput(353.59,367.68)(0.488,-0.890){13}{\rule{0.117pt}{0.800pt}}
\multiput(352.17,369.34)(8.000,-12.340){2}{\rule{0.400pt}{0.400pt}}
\multiput(361.59,353.89)(0.488,-0.824){13}{\rule{0.117pt}{0.750pt}}
\multiput(360.17,355.44)(8.000,-11.443){2}{\rule{0.400pt}{0.375pt}}
\multiput(369.59,341.30)(0.488,-0.692){13}{\rule{0.117pt}{0.650pt}}
\multiput(368.17,342.65)(8.000,-9.651){2}{\rule{0.400pt}{0.325pt}}
\multiput(377.59,330.72)(0.488,-0.560){13}{\rule{0.117pt}{0.550pt}}
\multiput(376.17,331.86)(8.000,-7.858){2}{\rule{0.400pt}{0.275pt}}
\multiput(385.00,322.93)(0.671,-0.482){9}{\rule{0.633pt}{0.116pt}}
\multiput(385.00,323.17)(6.685,-6.000){2}{\rule{0.317pt}{0.400pt}}
\multiput(393.00,316.93)(0.821,-0.477){7}{\rule{0.740pt}{0.115pt}}
\multiput(393.00,317.17)(6.464,-5.000){2}{\rule{0.370pt}{0.400pt}}
\put(401,311.17){\rule{1.700pt}{0.400pt}}
\multiput(401.00,312.17)(4.472,-2.000){2}{\rule{0.850pt}{0.400pt}}
\put(171.0,571.0){\rule[-0.200pt]{22.885pt}{0.400pt}}
\multiput(417.00,311.61)(1.579,0.447){3}{\rule{1.167pt}{0.108pt}}
\multiput(417.00,310.17)(5.579,3.000){2}{\rule{0.583pt}{0.400pt}}
\multiput(425.00,314.59)(0.821,0.477){7}{\rule{0.740pt}{0.115pt}}
\multiput(425.00,313.17)(6.464,5.000){2}{\rule{0.370pt}{0.400pt}}
\multiput(433.00,319.59)(0.581,0.482){9}{\rule{0.567pt}{0.116pt}}
\multiput(433.00,318.17)(5.824,6.000){2}{\rule{0.283pt}{0.400pt}}
\multiput(440.59,325.00)(0.488,0.560){13}{\rule{0.117pt}{0.550pt}}
\multiput(439.17,325.00)(8.000,7.858){2}{\rule{0.400pt}{0.275pt}}
\multiput(448.59,334.00)(0.488,0.692){13}{\rule{0.117pt}{0.650pt}}
\multiput(447.17,334.00)(8.000,9.651){2}{\rule{0.400pt}{0.325pt}}
\multiput(456.59,345.00)(0.488,0.824){13}{\rule{0.117pt}{0.750pt}}
\multiput(455.17,345.00)(8.000,11.443){2}{\rule{0.400pt}{0.375pt}}
\multiput(464.59,358.00)(0.488,0.890){13}{\rule{0.117pt}{0.800pt}}
\multiput(463.17,358.00)(8.000,12.340){2}{\rule{0.400pt}{0.400pt}}
\multiput(472.59,372.00)(0.488,1.022){13}{\rule{0.117pt}{0.900pt}}
\multiput(471.17,372.00)(8.000,14.132){2}{\rule{0.400pt}{0.450pt}}
\multiput(480.59,388.00)(0.488,1.154){13}{\rule{0.117pt}{1.000pt}}
\multiput(479.17,388.00)(8.000,15.924){2}{\rule{0.400pt}{0.500pt}}
\multiput(488.59,406.00)(0.488,1.154){13}{\rule{0.117pt}{1.000pt}}
\multiput(487.17,406.00)(8.000,15.924){2}{\rule{0.400pt}{0.500pt}}
\multiput(496.59,424.00)(0.488,1.286){13}{\rule{0.117pt}{1.100pt}}
\multiput(495.17,424.00)(8.000,17.717){2}{\rule{0.400pt}{0.550pt}}
\multiput(504.59,444.00)(0.488,1.352){13}{\rule{0.117pt}{1.150pt}}
\multiput(503.17,444.00)(8.000,18.613){2}{\rule{0.400pt}{0.575pt}}
\multiput(512.59,465.00)(0.488,1.352){13}{\rule{0.117pt}{1.150pt}}
\multiput(511.17,465.00)(8.000,18.613){2}{\rule{0.400pt}{0.575pt}}
\multiput(520.59,486.00)(0.488,1.484){13}{\rule{0.117pt}{1.250pt}}
\multiput(519.17,486.00)(8.000,20.406){2}{\rule{0.400pt}{0.625pt}}
\multiput(528.59,509.00)(0.488,1.418){13}{\rule{0.117pt}{1.200pt}}
\multiput(527.17,509.00)(8.000,19.509){2}{\rule{0.400pt}{0.600pt}}
\multiput(536.59,531.00)(0.485,1.713){11}{\rule{0.117pt}{1.414pt}}
\multiput(535.17,531.00)(7.000,20.065){2}{\rule{0.400pt}{0.707pt}}
\multiput(543.59,554.00)(0.488,1.484){13}{\rule{0.117pt}{1.250pt}}
\multiput(542.17,554.00)(8.000,20.406){2}{\rule{0.400pt}{0.625pt}}
\multiput(551.59,577.00)(0.488,1.418){13}{\rule{0.117pt}{1.200pt}}
\multiput(550.17,577.00)(8.000,19.509){2}{\rule{0.400pt}{0.600pt}}
\multiput(559.59,599.00)(0.488,1.418){13}{\rule{0.117pt}{1.200pt}}
\multiput(558.17,599.00)(8.000,19.509){2}{\rule{0.400pt}{0.600pt}}
\multiput(567.59,621.00)(0.488,1.418){13}{\rule{0.117pt}{1.200pt}}
\multiput(566.17,621.00)(8.000,19.509){2}{\rule{0.400pt}{0.600pt}}
\multiput(575.59,643.00)(0.488,1.352){13}{\rule{0.117pt}{1.150pt}}
\multiput(574.17,643.00)(8.000,18.613){2}{\rule{0.400pt}{0.575pt}}
\multiput(583.59,664.00)(0.488,1.286){13}{\rule{0.117pt}{1.100pt}}
\multiput(582.17,664.00)(8.000,17.717){2}{\rule{0.400pt}{0.550pt}}
\multiput(591.59,684.00)(0.488,1.154){13}{\rule{0.117pt}{1.000pt}}
\multiput(590.17,684.00)(8.000,15.924){2}{\rule{0.400pt}{0.500pt}}
\multiput(599.59,702.00)(0.488,1.154){13}{\rule{0.117pt}{1.000pt}}
\multiput(598.17,702.00)(8.000,15.924){2}{\rule{0.400pt}{0.500pt}}
\multiput(607.59,720.00)(0.488,1.088){13}{\rule{0.117pt}{0.950pt}}
\multiput(606.17,720.00)(8.000,15.028){2}{\rule{0.400pt}{0.475pt}}
\multiput(615.59,737.00)(0.488,0.890){13}{\rule{0.117pt}{0.800pt}}
\multiput(614.17,737.00)(8.000,12.340){2}{\rule{0.400pt}{0.400pt}}
\multiput(623.59,751.00)(0.488,0.890){13}{\rule{0.117pt}{0.800pt}}
\multiput(622.17,751.00)(8.000,12.340){2}{\rule{0.400pt}{0.400pt}}
\multiput(631.59,765.00)(0.488,0.692){13}{\rule{0.117pt}{0.650pt}}
\multiput(630.17,765.00)(8.000,9.651){2}{\rule{0.400pt}{0.325pt}}
\multiput(639.59,776.00)(0.488,0.560){13}{\rule{0.117pt}{0.550pt}}
\multiput(638.17,776.00)(8.000,7.858){2}{\rule{0.400pt}{0.275pt}}
\multiput(647.59,785.00)(0.485,0.569){11}{\rule{0.117pt}{0.557pt}}
\multiput(646.17,785.00)(7.000,6.844){2}{\rule{0.400pt}{0.279pt}}
\multiput(654.00,793.59)(0.671,0.482){9}{\rule{0.633pt}{0.116pt}}
\multiput(654.00,792.17)(6.685,6.000){2}{\rule{0.317pt}{0.400pt}}
\multiput(662.00,799.60)(1.066,0.468){5}{\rule{0.900pt}{0.113pt}}
\multiput(662.00,798.17)(6.132,4.000){2}{\rule{0.450pt}{0.400pt}}
\put(670,803.17){\rule{1.700pt}{0.400pt}}
\multiput(670.00,802.17)(4.472,2.000){2}{\rule{0.850pt}{0.400pt}}
\put(409.0,311.0){\rule[-0.200pt]{1.927pt}{0.400pt}}
\put(686,803.17){\rule{1.700pt}{0.400pt}}
\multiput(686.00,804.17)(4.472,-2.000){2}{\rule{0.850pt}{0.400pt}}
\multiput(694.00,801.94)(1.066,-0.468){5}{\rule{0.900pt}{0.113pt}}
\multiput(694.00,802.17)(6.132,-4.000){2}{\rule{0.450pt}{0.400pt}}
\multiput(702.00,797.93)(0.671,-0.482){9}{\rule{0.633pt}{0.116pt}}
\multiput(702.00,798.17)(6.685,-6.000){2}{\rule{0.317pt}{0.400pt}}
\multiput(710.00,791.93)(0.494,-0.488){13}{\rule{0.500pt}{0.117pt}}
\multiput(710.00,792.17)(6.962,-8.000){2}{\rule{0.250pt}{0.400pt}}
\multiput(718.59,782.51)(0.488,-0.626){13}{\rule{0.117pt}{0.600pt}}
\multiput(717.17,783.75)(8.000,-8.755){2}{\rule{0.400pt}{0.300pt}}
\multiput(726.59,772.30)(0.488,-0.692){13}{\rule{0.117pt}{0.650pt}}
\multiput(725.17,773.65)(8.000,-9.651){2}{\rule{0.400pt}{0.325pt}}
\multiput(734.59,761.09)(0.488,-0.758){13}{\rule{0.117pt}{0.700pt}}
\multiput(733.17,762.55)(8.000,-10.547){2}{\rule{0.400pt}{0.350pt}}
\multiput(742.59,748.68)(0.488,-0.890){13}{\rule{0.117pt}{0.800pt}}
\multiput(741.17,750.34)(8.000,-12.340){2}{\rule{0.400pt}{0.400pt}}
\multiput(750.59,733.79)(0.485,-1.179){11}{\rule{0.117pt}{1.014pt}}
\multiput(749.17,735.89)(7.000,-13.895){2}{\rule{0.400pt}{0.507pt}}
\multiput(757.59,718.26)(0.488,-1.022){13}{\rule{0.117pt}{0.900pt}}
\multiput(756.17,720.13)(8.000,-14.132){2}{\rule{0.400pt}{0.450pt}}
\multiput(765.59,702.06)(0.488,-1.088){13}{\rule{0.117pt}{0.950pt}}
\multiput(764.17,704.03)(8.000,-15.028){2}{\rule{0.400pt}{0.475pt}}
\multiput(773.59,684.64)(0.488,-1.220){13}{\rule{0.117pt}{1.050pt}}
\multiput(772.17,686.82)(8.000,-16.821){2}{\rule{0.400pt}{0.525pt}}
\multiput(781.59,665.85)(0.488,-1.154){13}{\rule{0.117pt}{1.000pt}}
\multiput(780.17,667.92)(8.000,-15.924){2}{\rule{0.400pt}{0.500pt}}
\multiput(789.59,647.43)(0.488,-1.286){13}{\rule{0.117pt}{1.100pt}}
\multiput(788.17,649.72)(8.000,-17.717){2}{\rule{0.400pt}{0.550pt}}
\multiput(797.59,627.43)(0.488,-1.286){13}{\rule{0.117pt}{1.100pt}}
\multiput(796.17,629.72)(8.000,-17.717){2}{\rule{0.400pt}{0.550pt}}
\multiput(805.59,607.43)(0.488,-1.286){13}{\rule{0.117pt}{1.100pt}}
\multiput(804.17,609.72)(8.000,-17.717){2}{\rule{0.400pt}{0.550pt}}
\multiput(813.59,587.43)(0.488,-1.286){13}{\rule{0.117pt}{1.100pt}}
\multiput(812.17,589.72)(8.000,-17.717){2}{\rule{0.400pt}{0.550pt}}
\multiput(821.59,567.43)(0.488,-1.286){13}{\rule{0.117pt}{1.100pt}}
\multiput(820.17,569.72)(8.000,-17.717){2}{\rule{0.400pt}{0.550pt}}
\multiput(829.59,547.43)(0.488,-1.286){13}{\rule{0.117pt}{1.100pt}}
\multiput(828.17,549.72)(8.000,-17.717){2}{\rule{0.400pt}{0.550pt}}
\multiput(837.59,527.64)(0.488,-1.220){13}{\rule{0.117pt}{1.050pt}}
\multiput(836.17,529.82)(8.000,-16.821){2}{\rule{0.400pt}{0.525pt}}
\multiput(845.59,508.64)(0.488,-1.220){13}{\rule{0.117pt}{1.050pt}}
\multiput(844.17,510.82)(8.000,-16.821){2}{\rule{0.400pt}{0.525pt}}
\multiput(853.59,489.55)(0.485,-1.255){11}{\rule{0.117pt}{1.071pt}}
\multiput(852.17,491.78)(7.000,-14.776){2}{\rule{0.400pt}{0.536pt}}
\multiput(860.59,473.06)(0.488,-1.088){13}{\rule{0.117pt}{0.950pt}}
\multiput(859.17,475.03)(8.000,-15.028){2}{\rule{0.400pt}{0.475pt}}
\multiput(868.59,456.26)(0.488,-1.022){13}{\rule{0.117pt}{0.900pt}}
\multiput(867.17,458.13)(8.000,-14.132){2}{\rule{0.400pt}{0.450pt}}
\multiput(876.59,440.47)(0.488,-0.956){13}{\rule{0.117pt}{0.850pt}}
\multiput(875.17,442.24)(8.000,-13.236){2}{\rule{0.400pt}{0.425pt}}
\multiput(884.59,425.89)(0.488,-0.824){13}{\rule{0.117pt}{0.750pt}}
\multiput(883.17,427.44)(8.000,-11.443){2}{\rule{0.400pt}{0.375pt}}
\multiput(892.59,413.09)(0.488,-0.758){13}{\rule{0.117pt}{0.700pt}}
\multiput(891.17,414.55)(8.000,-10.547){2}{\rule{0.400pt}{0.350pt}}
\multiput(900.59,401.30)(0.488,-0.692){13}{\rule{0.117pt}{0.650pt}}
\multiput(899.17,402.65)(8.000,-9.651){2}{\rule{0.400pt}{0.325pt}}
\multiput(908.59,390.72)(0.488,-0.560){13}{\rule{0.117pt}{0.550pt}}
\multiput(907.17,391.86)(8.000,-7.858){2}{\rule{0.400pt}{0.275pt}}
\multiput(916.00,382.93)(0.569,-0.485){11}{\rule{0.557pt}{0.117pt}}
\multiput(916.00,383.17)(6.844,-7.000){2}{\rule{0.279pt}{0.400pt}}
\multiput(924.00,375.93)(0.821,-0.477){7}{\rule{0.740pt}{0.115pt}}
\multiput(924.00,376.17)(6.464,-5.000){2}{\rule{0.370pt}{0.400pt}}
\multiput(932.00,370.94)(1.066,-0.468){5}{\rule{0.900pt}{0.113pt}}
\multiput(932.00,371.17)(6.132,-4.000){2}{\rule{0.450pt}{0.400pt}}
\put(940,366.17){\rule{1.700pt}{0.400pt}}
\multiput(940.00,367.17)(4.472,-2.000){2}{\rule{0.850pt}{0.400pt}}
\put(948,364.67){\rule{1.927pt}{0.400pt}}
\multiput(948.00,365.17)(4.000,-1.000){2}{\rule{0.964pt}{0.400pt}}
\put(956,365.17){\rule{1.700pt}{0.400pt}}
\multiput(956.00,364.17)(4.472,2.000){2}{\rule{0.850pt}{0.400pt}}
\multiput(964.00,367.61)(1.355,0.447){3}{\rule{1.033pt}{0.108pt}}
\multiput(964.00,366.17)(4.855,3.000){2}{\rule{0.517pt}{0.400pt}}
\multiput(971.00,370.59)(0.821,0.477){7}{\rule{0.740pt}{0.115pt}}
\multiput(971.00,369.17)(6.464,5.000){2}{\rule{0.370pt}{0.400pt}}
\multiput(979.00,375.59)(0.569,0.485){11}{\rule{0.557pt}{0.117pt}}
\multiput(979.00,374.17)(6.844,7.000){2}{\rule{0.279pt}{0.400pt}}
\multiput(987.00,382.59)(0.494,0.488){13}{\rule{0.500pt}{0.117pt}}
\multiput(987.00,381.17)(6.962,8.000){2}{\rule{0.250pt}{0.400pt}}
\multiput(995.59,390.00)(0.488,0.560){13}{\rule{0.117pt}{0.550pt}}
\multiput(994.17,390.00)(8.000,7.858){2}{\rule{0.400pt}{0.275pt}}
\multiput(1003.59,399.00)(0.488,0.692){13}{\rule{0.117pt}{0.650pt}}
\multiput(1002.17,399.00)(8.000,9.651){2}{\rule{0.400pt}{0.325pt}}
\multiput(1011.59,410.00)(0.488,0.824){13}{\rule{0.117pt}{0.750pt}}
\multiput(1010.17,410.00)(8.000,11.443){2}{\rule{0.400pt}{0.375pt}}
\multiput(1019.59,423.00)(0.488,0.824){13}{\rule{0.117pt}{0.750pt}}
\multiput(1018.17,423.00)(8.000,11.443){2}{\rule{0.400pt}{0.375pt}}
\multiput(1027.59,436.00)(0.488,0.890){13}{\rule{0.117pt}{0.800pt}}
\multiput(1026.17,436.00)(8.000,12.340){2}{\rule{0.400pt}{0.400pt}}
\multiput(1035.59,450.00)(0.488,1.022){13}{\rule{0.117pt}{0.900pt}}
\multiput(1034.17,450.00)(8.000,14.132){2}{\rule{0.400pt}{0.450pt}}
\multiput(1043.59,466.00)(0.488,1.022){13}{\rule{0.117pt}{0.900pt}}
\multiput(1042.17,466.00)(8.000,14.132){2}{\rule{0.400pt}{0.450pt}}
\multiput(1051.59,482.00)(0.488,1.088){13}{\rule{0.117pt}{0.950pt}}
\multiput(1050.17,482.00)(8.000,15.028){2}{\rule{0.400pt}{0.475pt}}
\multiput(1059.59,499.00)(0.488,1.088){13}{\rule{0.117pt}{0.950pt}}
\multiput(1058.17,499.00)(8.000,15.028){2}{\rule{0.400pt}{0.475pt}}
\multiput(1067.59,516.00)(0.485,1.332){11}{\rule{0.117pt}{1.129pt}}
\multiput(1066.17,516.00)(7.000,15.658){2}{\rule{0.400pt}{0.564pt}}
\multiput(1074.59,534.00)(0.488,1.154){13}{\rule{0.117pt}{1.000pt}}
\multiput(1073.17,534.00)(8.000,15.924){2}{\rule{0.400pt}{0.500pt}}
\multiput(1082.59,552.00)(0.488,1.154){13}{\rule{0.117pt}{1.000pt}}
\multiput(1081.17,552.00)(8.000,15.924){2}{\rule{0.400pt}{0.500pt}}
\multiput(1090.59,570.00)(0.488,1.154){13}{\rule{0.117pt}{1.000pt}}
\multiput(1089.17,570.00)(8.000,15.924){2}{\rule{0.400pt}{0.500pt}}
\multiput(1098.59,588.00)(0.488,1.088){13}{\rule{0.117pt}{0.950pt}}
\multiput(1097.17,588.00)(8.000,15.028){2}{\rule{0.400pt}{0.475pt}}
\multiput(1106.59,605.00)(0.488,1.088){13}{\rule{0.117pt}{0.950pt}}
\multiput(1105.17,605.00)(8.000,15.028){2}{\rule{0.400pt}{0.475pt}}
\multiput(1114.59,622.00)(0.488,1.088){13}{\rule{0.117pt}{0.950pt}}
\multiput(1113.17,622.00)(8.000,15.028){2}{\rule{0.400pt}{0.475pt}}
\multiput(1122.59,639.00)(0.488,1.022){13}{\rule{0.117pt}{0.900pt}}
\multiput(1121.17,639.00)(8.000,14.132){2}{\rule{0.400pt}{0.450pt}}
\multiput(1130.59,655.00)(0.488,0.956){13}{\rule{0.117pt}{0.850pt}}
\multiput(1129.17,655.00)(8.000,13.236){2}{\rule{0.400pt}{0.425pt}}
\multiput(1138.59,670.00)(0.488,0.890){13}{\rule{0.117pt}{0.800pt}}
\multiput(1137.17,670.00)(8.000,12.340){2}{\rule{0.400pt}{0.400pt}}
\multiput(1146.59,684.00)(0.488,0.824){13}{\rule{0.117pt}{0.750pt}}
\multiput(1145.17,684.00)(8.000,11.443){2}{\rule{0.400pt}{0.375pt}}
\multiput(1154.59,697.00)(0.488,0.758){13}{\rule{0.117pt}{0.700pt}}
\multiput(1153.17,697.00)(8.000,10.547){2}{\rule{0.400pt}{0.350pt}}
\multiput(1162.59,709.00)(0.488,0.692){13}{\rule{0.117pt}{0.650pt}}
\multiput(1161.17,709.00)(8.000,9.651){2}{\rule{0.400pt}{0.325pt}}
\multiput(1170.59,720.00)(0.485,0.721){11}{\rule{0.117pt}{0.671pt}}
\multiput(1169.17,720.00)(7.000,8.606){2}{\rule{0.400pt}{0.336pt}}
\multiput(1177.00,730.59)(0.494,0.488){13}{\rule{0.500pt}{0.117pt}}
\multiput(1177.00,729.17)(6.962,8.000){2}{\rule{0.250pt}{0.400pt}}
\multiput(1185.00,738.59)(0.671,0.482){9}{\rule{0.633pt}{0.116pt}}
\multiput(1185.00,737.17)(6.685,6.000){2}{\rule{0.317pt}{0.400pt}}
\multiput(1193.00,744.59)(0.821,0.477){7}{\rule{0.740pt}{0.115pt}}
\multiput(1193.00,743.17)(6.464,5.000){2}{\rule{0.370pt}{0.400pt}}
\multiput(1201.00,749.60)(1.066,0.468){5}{\rule{0.900pt}{0.113pt}}
\multiput(1201.00,748.17)(6.132,4.000){2}{\rule{0.450pt}{0.400pt}}
\put(1209,753.17){\rule{1.700pt}{0.400pt}}
\multiput(1209.00,752.17)(4.472,2.000){2}{\rule{0.850pt}{0.400pt}}
\put(678.0,805.0){\rule[-0.200pt]{1.927pt}{0.400pt}}
\put(1225,753.67){\rule{1.927pt}{0.400pt}}
\multiput(1225.00,754.17)(4.000,-1.000){2}{\rule{0.964pt}{0.400pt}}
\multiput(1233.00,752.95)(1.579,-0.447){3}{\rule{1.167pt}{0.108pt}}
\multiput(1233.00,753.17)(5.579,-3.000){2}{\rule{0.583pt}{0.400pt}}
\multiput(1241.00,749.94)(1.066,-0.468){5}{\rule{0.900pt}{0.113pt}}
\multiput(1241.00,750.17)(6.132,-4.000){2}{\rule{0.450pt}{0.400pt}}
\multiput(1249.00,745.93)(0.821,-0.477){7}{\rule{0.740pt}{0.115pt}}
\multiput(1249.00,746.17)(6.464,-5.000){2}{\rule{0.370pt}{0.400pt}}
\multiput(1257.00,740.93)(0.569,-0.485){11}{\rule{0.557pt}{0.117pt}}
\multiput(1257.00,741.17)(6.844,-7.000){2}{\rule{0.279pt}{0.400pt}}
\multiput(1265.00,733.93)(0.494,-0.488){13}{\rule{0.500pt}{0.117pt}}
\multiput(1265.00,734.17)(6.962,-8.000){2}{\rule{0.250pt}{0.400pt}}
\multiput(1273.59,724.51)(0.488,-0.626){13}{\rule{0.117pt}{0.600pt}}
\multiput(1272.17,725.75)(8.000,-8.755){2}{\rule{0.400pt}{0.300pt}}
\multiput(1281.59,714.21)(0.485,-0.721){11}{\rule{0.117pt}{0.671pt}}
\multiput(1280.17,715.61)(7.000,-8.606){2}{\rule{0.400pt}{0.336pt}}
\multiput(1288.59,704.09)(0.488,-0.758){13}{\rule{0.117pt}{0.700pt}}
\multiput(1287.17,705.55)(8.000,-10.547){2}{\rule{0.400pt}{0.350pt}}
\multiput(1296.59,692.09)(0.488,-0.758){13}{\rule{0.117pt}{0.700pt}}
\multiput(1295.17,693.55)(8.000,-10.547){2}{\rule{0.400pt}{0.350pt}}
\multiput(1304.59,679.68)(0.488,-0.890){13}{\rule{0.117pt}{0.800pt}}
\multiput(1303.17,681.34)(8.000,-12.340){2}{\rule{0.400pt}{0.400pt}}
\multiput(1312.59,665.68)(0.488,-0.890){13}{\rule{0.117pt}{0.800pt}}
\multiput(1311.17,667.34)(8.000,-12.340){2}{\rule{0.400pt}{0.400pt}}
\multiput(1320.59,651.68)(0.488,-0.890){13}{\rule{0.117pt}{0.800pt}}
\multiput(1319.17,653.34)(8.000,-12.340){2}{\rule{0.400pt}{0.400pt}}
\multiput(1328.59,637.47)(0.488,-0.956){13}{\rule{0.117pt}{0.850pt}}
\multiput(1327.17,639.24)(8.000,-13.236){2}{\rule{0.400pt}{0.425pt}}
\multiput(1336.59,622.26)(0.488,-1.022){13}{\rule{0.117pt}{0.900pt}}
\multiput(1335.17,624.13)(8.000,-14.132){2}{\rule{0.400pt}{0.450pt}}
\multiput(1344.59,606.47)(0.488,-0.956){13}{\rule{0.117pt}{0.850pt}}
\multiput(1343.17,608.24)(8.000,-13.236){2}{\rule{0.400pt}{0.425pt}}
\multiput(1352.59,591.26)(0.488,-1.022){13}{\rule{0.117pt}{0.900pt}}
\multiput(1351.17,593.13)(8.000,-14.132){2}{\rule{0.400pt}{0.450pt}}
\multiput(1360.59,575.26)(0.488,-1.022){13}{\rule{0.117pt}{0.900pt}}
\multiput(1359.17,577.13)(8.000,-14.132){2}{\rule{0.400pt}{0.450pt}}
\multiput(1368.59,559.47)(0.488,-0.956){13}{\rule{0.117pt}{0.850pt}}
\multiput(1367.17,561.24)(8.000,-13.236){2}{\rule{0.400pt}{0.425pt}}
\multiput(1376.59,544.47)(0.488,-0.956){13}{\rule{0.117pt}{0.850pt}}
\multiput(1375.17,546.24)(8.000,-13.236){2}{\rule{0.400pt}{0.425pt}}
\multiput(1384.59,529.03)(0.485,-1.103){11}{\rule{0.117pt}{0.957pt}}
\multiput(1383.17,531.01)(7.000,-13.013){2}{\rule{0.400pt}{0.479pt}}
\multiput(1391.59,514.68)(0.488,-0.890){13}{\rule{0.117pt}{0.800pt}}
\multiput(1390.17,516.34)(8.000,-12.340){2}{\rule{0.400pt}{0.400pt}}
\multiput(1399.59,500.68)(0.488,-0.890){13}{\rule{0.117pt}{0.800pt}}
\multiput(1398.17,502.34)(8.000,-12.340){2}{\rule{0.400pt}{0.400pt}}
\multiput(1407.59,487.09)(0.488,-0.758){13}{\rule{0.117pt}{0.700pt}}
\multiput(1406.17,488.55)(8.000,-10.547){2}{\rule{0.400pt}{0.350pt}}
\multiput(1415.59,475.09)(0.488,-0.758){13}{\rule{0.117pt}{0.700pt}}
\multiput(1414.17,476.55)(8.000,-10.547){2}{\rule{0.400pt}{0.350pt}}
\multiput(1423.59,463.30)(0.488,-0.692){13}{\rule{0.117pt}{0.650pt}}
\multiput(1422.17,464.65)(8.000,-9.651){2}{\rule{0.400pt}{0.325pt}}
\multiput(1431.59,452.51)(0.488,-0.626){13}{\rule{0.117pt}{0.600pt}}
\multiput(1430.17,453.75)(8.000,-8.755){2}{\rule{0.400pt}{0.300pt}}
\put(1217.0,755.0){\rule[-0.200pt]{1.927pt}{0.400pt}}
\put(1439,445){\usebox{\plotpoint}}
\put(1279,377){\makebox(0,0)[r]{$I="1.5 A"$}}
\multiput(1299,377)(20.756,0.000){5}{\usebox{\plotpoint}}
\put(1399,377){\usebox{\plotpoint}}
\put(171,571){\usebox{\plotpoint}}
\put(171.00,571.00){\usebox{\plotpoint}}
\put(191.76,571.00){\usebox{\plotpoint}}
\put(212.51,571.00){\usebox{\plotpoint}}
\put(233.27,571.00){\usebox{\plotpoint}}
\put(254.02,571.00){\usebox{\plotpoint}}
\put(274.71,569.91){\usebox{\plotpoint}}
\put(295.01,565.75){\usebox{\plotpoint}}
\put(315.35,561.83){\usebox{\plotpoint}}
\put(335.34,556.71){\usebox{\plotpoint}}
\put(350.59,543.01){\usebox{\plotpoint}}
\put(360.56,524.89){\usebox{\plotpoint}}
\put(368.04,505.53){\usebox{\plotpoint}}
\put(374.13,485.69){\usebox{\plotpoint}}
\multiput(377,476)(4.276,-20.310){2}{\usebox{\plotpoint}}
\multiput(385,438)(3.796,-20.405){2}{\usebox{\plotpoint}}
\multiput(393,395)(3.975,-20.371){2}{\usebox{\plotpoint}}
\multiput(401,354)(4.276,-20.310){2}{\usebox{\plotpoint}}
\multiput(409,316)(4.754,-20.204){2}{\usebox{\plotpoint}}
\put(422.18,262.58){\usebox{\plotpoint}}
\put(427.78,242.60){\usebox{\plotpoint}}
\put(433.76,222.73){\usebox{\plotpoint}}
\put(440.81,203.39){\usebox{\plotpoint}}
\put(456.18,206.42){\usebox{\plotpoint}}
\multiput(464,225)(7.093,19.506){2}{\usebox{\plotpoint}}
\put(477.91,264.74){\usebox{\plotpoint}}
\put(484.16,284.53){\usebox{\plotpoint}}
\put(490.27,304.37){\usebox{\plotpoint}}
\multiput(496,323)(5.519,20.008){2}{\usebox{\plotpoint}}
\put(507.37,364.23){\usebox{\plotpoint}}
\multiput(512,381)(5.519,20.008){2}{\usebox{\plotpoint}}
\put(523.81,424.29){\usebox{\plotpoint}}
\multiput(528,440)(5.519,20.008){2}{\usebox{\plotpoint}}
\put(539.74,484.48){\usebox{\plotpoint}}
\multiput(543,498)(5.519,20.008){2}{\usebox{\plotpoint}}
\put(556.02,544.57){\usebox{\plotpoint}}
\put(561.72,564.53){\usebox{\plotpoint}}
\multiput(567,583)(6.104,19.838){2}{\usebox{\plotpoint}}
\put(580.01,624.04){\usebox{\plotpoint}}
\put(586.72,643.68){\usebox{\plotpoint}}
\put(593.64,663.25){\usebox{\plotpoint}}
\put(600.88,682.70){\usebox{\plotpoint}}
\put(608.74,701.91){\usebox{\plotpoint}}
\put(617.27,720.83){\usebox{\plotpoint}}
\put(626.62,739.34){\usebox{\plotpoint}}
\put(637.62,756.93){\usebox{\plotpoint}}
\put(650.35,773.31){\usebox{\plotpoint}}
\put(666.30,786.15){\usebox{\plotpoint}}
\put(686.37,789.95){\usebox{\plotpoint}}
\put(705.50,782.81){\usebox{\plotpoint}}
\put(721.85,770.15){\usebox{\plotpoint}}
\put(735.67,754.70){\usebox{\plotpoint}}
\put(747.88,737.91){\usebox{\plotpoint}}
\put(758.71,720.22){\usebox{\plotpoint}}
\put(769.59,702.54){\usebox{\plotpoint}}
\put(780.07,684.63){\usebox{\plotpoint}}
\put(789.93,666.37){\usebox{\plotpoint}}
\put(800.06,648.26){\usebox{\plotpoint}}
\put(809.83,629.94){\usebox{\plotpoint}}
\put(819.60,611.63){\usebox{\plotpoint}}
\put(829.82,593.57){\usebox{\plotpoint}}
\put(840.12,575.55){\usebox{\plotpoint}}
\put(851.05,557.92){\usebox{\plotpoint}}
\put(861.46,539.99){\usebox{\plotpoint}}
\put(873.67,523.21){\usebox{\plotpoint}}
\put(886.65,507.02){\usebox{\plotpoint}}
\put(901.04,492.09){\usebox{\plotpoint}}
\put(917.28,479.20){\usebox{\plotpoint}}
\put(935.64,469.64){\usebox{\plotpoint}}
\put(955.49,464.06){\usebox{\plotpoint}}
\put(976.13,465.64){\usebox{\plotpoint}}
\put(995.70,472.35){\usebox{\plotpoint}}
\put(1014.09,481.93){\usebox{\plotpoint}}
\put(1030.97,493.98){\usebox{\plotpoint}}
\put(1047.31,506.77){\usebox{\plotpoint}}
\put(1062.93,520.44){\usebox{\plotpoint}}
\put(1077.85,534.85){\usebox{\plotpoint}}
\put(1093.01,549.01){\usebox{\plotpoint}}
\put(1108.31,563.02){\usebox{\plotpoint}}
\put(1124.55,575.92){\usebox{\plotpoint}}
\put(1141.16,588.37){\usebox{\plotpoint}}
\put(1158.47,599.79){\usebox{\plotpoint}}
\put(1177.03,609.01){\usebox{\plotpoint}}
\put(1196.37,616.26){\usebox{\plotpoint}}
\put(1216.51,620.88){\usebox{\plotpoint}}
\put(1237.19,622.00){\usebox{\plotpoint}}
\put(1257.82,619.90){\usebox{\plotpoint}}
\put(1278.11,615.72){\usebox{\plotpoint}}
\put(1298.11,610.21){\usebox{\plotpoint}}
\put(1317.54,602.92){\usebox{\plotpoint}}
\put(1336.60,594.77){\usebox{\plotpoint}}
\put(1355.66,586.63){\usebox{\plotpoint}}
\put(1375.10,579.34){\usebox{\plotpoint}}
\put(1394.02,570.87){\usebox{\plotpoint}}
\put(1413.69,564.33){\usebox{\plotpoint}}
\put(1433.54,558.37){\usebox{\plotpoint}}
\put(1439,557){\usebox{\plotpoint}}
\sbox{\plotpoint}{\rule[-0.400pt]{0.800pt}{0.800pt}}%
\sbox{\plotpoint}{\rule[-0.200pt]{0.400pt}{0.400pt}}%
\put(1279,336){\makebox(0,0)[r]{$I="1 A"$}}
\sbox{\plotpoint}{\rule[-0.400pt]{0.800pt}{0.800pt}}%
\put(1299.0,336.0){\rule[-0.400pt]{24.090pt}{0.800pt}}
\put(171,571){\usebox{\plotpoint}}
\put(234,568.34){\rule{1.927pt}{0.800pt}}
\multiput(234.00,569.34)(4.000,-2.000){2}{\rule{0.964pt}{0.800pt}}
\put(242,565.84){\rule{1.927pt}{0.800pt}}
\multiput(242.00,567.34)(4.000,-3.000){2}{\rule{0.964pt}{0.800pt}}
\multiput(250.00,564.07)(0.685,-0.536){5}{\rule{1.267pt}{0.129pt}}
\multiput(250.00,564.34)(5.371,-6.000){2}{\rule{0.633pt}{0.800pt}}
\multiput(259.40,555.43)(0.520,-0.554){9}{\rule{0.125pt}{1.100pt}}
\multiput(256.34,557.72)(8.000,-6.717){2}{\rule{0.800pt}{0.550pt}}
\multiput(267.40,544.77)(0.520,-0.847){9}{\rule{0.125pt}{1.500pt}}
\multiput(264.34,547.89)(8.000,-9.887){2}{\rule{0.800pt}{0.750pt}}
\multiput(275.40,529.28)(0.520,-1.285){9}{\rule{0.125pt}{2.100pt}}
\multiput(272.34,533.64)(8.000,-14.641){2}{\rule{0.800pt}{1.050pt}}
\multiput(283.40,509.45)(0.520,-1.432){9}{\rule{0.125pt}{2.300pt}}
\multiput(280.34,514.23)(8.000,-16.226){2}{\rule{0.800pt}{1.150pt}}
\multiput(291.40,486.38)(0.520,-1.797){9}{\rule{0.125pt}{2.800pt}}
\multiput(288.34,492.19)(8.000,-20.188){2}{\rule{0.800pt}{1.400pt}}
\multiput(299.40,459.96)(0.520,-1.870){9}{\rule{0.125pt}{2.900pt}}
\multiput(296.34,465.98)(8.000,-20.981){2}{\rule{0.800pt}{1.450pt}}
\multiput(307.40,432.96)(0.520,-1.870){9}{\rule{0.125pt}{2.900pt}}
\multiput(304.34,438.98)(8.000,-20.981){2}{\rule{0.800pt}{1.450pt}}
\multiput(315.40,406.79)(0.520,-1.724){9}{\rule{0.125pt}{2.700pt}}
\multiput(312.34,412.40)(8.000,-19.396){2}{\rule{0.800pt}{1.350pt}}
\multiput(323.40,382.21)(0.520,-1.651){9}{\rule{0.125pt}{2.600pt}}
\multiput(320.34,387.60)(8.000,-18.604){2}{\rule{0.800pt}{1.300pt}}
\multiput(331.40,358.21)(0.526,-1.701){7}{\rule{0.127pt}{2.600pt}}
\multiput(328.34,363.60)(7.000,-15.604){2}{\rule{0.800pt}{1.300pt}}
\multiput(338.40,339.28)(0.520,-1.285){9}{\rule{0.125pt}{2.100pt}}
\multiput(335.34,343.64)(8.000,-14.641){2}{\rule{0.800pt}{1.050pt}}
\multiput(346.40,321.11)(0.520,-1.139){9}{\rule{0.125pt}{1.900pt}}
\multiput(343.34,325.06)(8.000,-13.056){2}{\rule{0.800pt}{0.950pt}}
\multiput(354.40,305.36)(0.520,-0.920){9}{\rule{0.125pt}{1.600pt}}
\multiput(351.34,308.68)(8.000,-10.679){2}{\rule{0.800pt}{0.800pt}}
\multiput(362.40,292.60)(0.520,-0.700){9}{\rule{0.125pt}{1.300pt}}
\multiput(359.34,295.30)(8.000,-8.302){2}{\rule{0.800pt}{0.650pt}}
\multiput(370.40,282.43)(0.520,-0.554){9}{\rule{0.125pt}{1.100pt}}
\multiput(367.34,284.72)(8.000,-6.717){2}{\rule{0.800pt}{0.550pt}}
\multiput(377.00,276.07)(0.685,-0.536){5}{\rule{1.267pt}{0.129pt}}
\multiput(377.00,276.34)(5.371,-6.000){2}{\rule{0.633pt}{0.800pt}}
\put(385,268.34){\rule{1.800pt}{0.800pt}}
\multiput(385.00,270.34)(4.264,-4.000){2}{\rule{0.900pt}{0.800pt}}
\put(171.0,571.0){\rule[-0.400pt]{15.177pt}{0.800pt}}
\put(401,266.84){\rule{1.927pt}{0.800pt}}
\multiput(401.00,266.34)(4.000,1.000){2}{\rule{0.964pt}{0.800pt}}
\multiput(409.00,270.38)(0.928,0.560){3}{\rule{1.480pt}{0.135pt}}
\multiput(409.00,267.34)(4.928,5.000){2}{\rule{0.740pt}{0.800pt}}
\multiput(417.00,275.40)(0.562,0.526){7}{\rule{1.114pt}{0.127pt}}
\multiput(417.00,272.34)(5.687,7.000){2}{\rule{0.557pt}{0.800pt}}
\multiput(426.40,281.00)(0.520,0.554){9}{\rule{0.125pt}{1.100pt}}
\multiput(423.34,281.00)(8.000,6.717){2}{\rule{0.800pt}{0.550pt}}
\multiput(434.40,290.00)(0.526,0.913){7}{\rule{0.127pt}{1.571pt}}
\multiput(431.34,290.00)(7.000,8.738){2}{\rule{0.800pt}{0.786pt}}
\multiput(441.40,302.00)(0.520,0.847){9}{\rule{0.125pt}{1.500pt}}
\multiput(438.34,302.00)(8.000,9.887){2}{\rule{0.800pt}{0.750pt}}
\multiput(449.40,315.00)(0.520,1.066){9}{\rule{0.125pt}{1.800pt}}
\multiput(446.34,315.00)(8.000,12.264){2}{\rule{0.800pt}{0.900pt}}
\multiput(457.40,331.00)(0.520,1.139){9}{\rule{0.125pt}{1.900pt}}
\multiput(454.34,331.00)(8.000,13.056){2}{\rule{0.800pt}{0.950pt}}
\multiput(465.40,348.00)(0.520,1.285){9}{\rule{0.125pt}{2.100pt}}
\multiput(462.34,348.00)(8.000,14.641){2}{\rule{0.800pt}{1.050pt}}
\multiput(473.40,367.00)(0.520,1.432){9}{\rule{0.125pt}{2.300pt}}
\multiput(470.34,367.00)(8.000,16.226){2}{\rule{0.800pt}{1.150pt}}
\multiput(481.40,388.00)(0.520,1.432){9}{\rule{0.125pt}{2.300pt}}
\multiput(478.34,388.00)(8.000,16.226){2}{\rule{0.800pt}{1.150pt}}
\multiput(489.40,409.00)(0.520,1.578){9}{\rule{0.125pt}{2.500pt}}
\multiput(486.34,409.00)(8.000,17.811){2}{\rule{0.800pt}{1.250pt}}
\multiput(497.40,432.00)(0.520,1.578){9}{\rule{0.125pt}{2.500pt}}
\multiput(494.34,432.00)(8.000,17.811){2}{\rule{0.800pt}{1.250pt}}
\multiput(505.40,455.00)(0.520,1.651){9}{\rule{0.125pt}{2.600pt}}
\multiput(502.34,455.00)(8.000,18.604){2}{\rule{0.800pt}{1.300pt}}
\multiput(513.40,479.00)(0.520,1.651){9}{\rule{0.125pt}{2.600pt}}
\multiput(510.34,479.00)(8.000,18.604){2}{\rule{0.800pt}{1.300pt}}
\multiput(521.40,503.00)(0.520,1.724){9}{\rule{0.125pt}{2.700pt}}
\multiput(518.34,503.00)(8.000,19.396){2}{\rule{0.800pt}{1.350pt}}
\multiput(529.40,528.00)(0.520,1.651){9}{\rule{0.125pt}{2.600pt}}
\multiput(526.34,528.00)(8.000,18.604){2}{\rule{0.800pt}{1.300pt}}
\multiput(537.40,552.00)(0.526,2.052){7}{\rule{0.127pt}{3.057pt}}
\multiput(534.34,552.00)(7.000,18.655){2}{\rule{0.800pt}{1.529pt}}
\multiput(544.40,577.00)(0.520,1.578){9}{\rule{0.125pt}{2.500pt}}
\multiput(541.34,577.00)(8.000,17.811){2}{\rule{0.800pt}{1.250pt}}
\multiput(552.40,600.00)(0.520,1.578){9}{\rule{0.125pt}{2.500pt}}
\multiput(549.34,600.00)(8.000,17.811){2}{\rule{0.800pt}{1.250pt}}
\multiput(560.40,623.00)(0.520,1.505){9}{\rule{0.125pt}{2.400pt}}
\multiput(557.34,623.00)(8.000,17.019){2}{\rule{0.800pt}{1.200pt}}
\multiput(568.40,645.00)(0.520,1.432){9}{\rule{0.125pt}{2.300pt}}
\multiput(565.34,645.00)(8.000,16.226){2}{\rule{0.800pt}{1.150pt}}
\multiput(576.40,666.00)(0.520,1.358){9}{\rule{0.125pt}{2.200pt}}
\multiput(573.34,666.00)(8.000,15.434){2}{\rule{0.800pt}{1.100pt}}
\multiput(584.40,686.00)(0.520,1.285){9}{\rule{0.125pt}{2.100pt}}
\multiput(581.34,686.00)(8.000,14.641){2}{\rule{0.800pt}{1.050pt}}
\multiput(592.40,705.00)(0.520,1.139){9}{\rule{0.125pt}{1.900pt}}
\multiput(589.34,705.00)(8.000,13.056){2}{\rule{0.800pt}{0.950pt}}
\multiput(600.40,722.00)(0.520,1.066){9}{\rule{0.125pt}{1.800pt}}
\multiput(597.34,722.00)(8.000,12.264){2}{\rule{0.800pt}{0.900pt}}
\multiput(608.40,738.00)(0.520,0.920){9}{\rule{0.125pt}{1.600pt}}
\multiput(605.34,738.00)(8.000,10.679){2}{\rule{0.800pt}{0.800pt}}
\multiput(616.40,752.00)(0.520,0.847){9}{\rule{0.125pt}{1.500pt}}
\multiput(613.34,752.00)(8.000,9.887){2}{\rule{0.800pt}{0.750pt}}
\multiput(624.40,765.00)(0.520,0.627){9}{\rule{0.125pt}{1.200pt}}
\multiput(621.34,765.00)(8.000,7.509){2}{\rule{0.800pt}{0.600pt}}
\multiput(631.00,776.40)(0.481,0.520){9}{\rule{1.000pt}{0.125pt}}
\multiput(631.00,773.34)(5.924,8.000){2}{\rule{0.500pt}{0.800pt}}
\multiput(639.00,784.40)(0.562,0.526){7}{\rule{1.114pt}{0.127pt}}
\multiput(639.00,781.34)(5.687,7.000){2}{\rule{0.557pt}{0.800pt}}
\put(647,790.34){\rule{1.600pt}{0.800pt}}
\multiput(647.00,788.34)(3.679,4.000){2}{\rule{0.800pt}{0.800pt}}
\put(654,793.84){\rule{1.927pt}{0.800pt}}
\multiput(654.00,792.34)(4.000,3.000){2}{\rule{0.964pt}{0.800pt}}
\put(662,795.84){\rule{1.927pt}{0.800pt}}
\multiput(662.00,795.34)(4.000,1.000){2}{\rule{0.964pt}{0.800pt}}
\put(670,795.84){\rule{1.927pt}{0.800pt}}
\multiput(670.00,796.34)(4.000,-1.000){2}{\rule{0.964pt}{0.800pt}}
\put(678,793.34){\rule{1.800pt}{0.800pt}}
\multiput(678.00,795.34)(4.264,-4.000){2}{\rule{0.900pt}{0.800pt}}
\multiput(686.00,791.06)(0.928,-0.560){3}{\rule{1.480pt}{0.135pt}}
\multiput(686.00,791.34)(4.928,-5.000){2}{\rule{0.740pt}{0.800pt}}
\multiput(694.00,786.07)(0.685,-0.536){5}{\rule{1.267pt}{0.129pt}}
\multiput(694.00,786.34)(5.371,-6.000){2}{\rule{0.633pt}{0.800pt}}
\multiput(703.40,777.43)(0.520,-0.554){9}{\rule{0.125pt}{1.100pt}}
\multiput(700.34,779.72)(8.000,-6.717){2}{\rule{0.800pt}{0.550pt}}
\multiput(711.40,768.43)(0.520,-0.554){9}{\rule{0.125pt}{1.100pt}}
\multiput(708.34,770.72)(8.000,-6.717){2}{\rule{0.800pt}{0.550pt}}
\multiput(719.40,758.19)(0.520,-0.774){9}{\rule{0.125pt}{1.400pt}}
\multiput(716.34,761.09)(8.000,-9.094){2}{\rule{0.800pt}{0.700pt}}
\multiput(727.40,746.19)(0.520,-0.774){9}{\rule{0.125pt}{1.400pt}}
\multiput(724.34,749.09)(8.000,-9.094){2}{\rule{0.800pt}{0.700pt}}
\multiput(735.40,733.36)(0.520,-0.920){9}{\rule{0.125pt}{1.600pt}}
\multiput(732.34,736.68)(8.000,-10.679){2}{\rule{0.800pt}{0.800pt}}
\multiput(743.40,718.94)(0.520,-0.993){9}{\rule{0.125pt}{1.700pt}}
\multiput(740.34,722.47)(8.000,-11.472){2}{\rule{0.800pt}{0.850pt}}
\multiput(751.40,703.05)(0.526,-1.176){7}{\rule{0.127pt}{1.914pt}}
\multiput(748.34,707.03)(7.000,-11.027){2}{\rule{0.800pt}{0.957pt}}
\multiput(758.40,688.53)(0.520,-1.066){9}{\rule{0.125pt}{1.800pt}}
\multiput(755.34,692.26)(8.000,-12.264){2}{\rule{0.800pt}{0.900pt}}
\multiput(766.40,672.11)(0.520,-1.139){9}{\rule{0.125pt}{1.900pt}}
\multiput(763.34,676.06)(8.000,-13.056){2}{\rule{0.800pt}{0.950pt}}
\multiput(774.40,654.70)(0.520,-1.212){9}{\rule{0.125pt}{2.000pt}}
\multiput(771.34,658.85)(8.000,-13.849){2}{\rule{0.800pt}{1.000pt}}
\multiput(782.40,637.11)(0.520,-1.139){9}{\rule{0.125pt}{1.900pt}}
\multiput(779.34,641.06)(8.000,-13.056){2}{\rule{0.800pt}{0.950pt}}
\multiput(790.40,619.70)(0.520,-1.212){9}{\rule{0.125pt}{2.000pt}}
\multiput(787.34,623.85)(8.000,-13.849){2}{\rule{0.800pt}{1.000pt}}
\multiput(798.40,601.70)(0.520,-1.212){9}{\rule{0.125pt}{2.000pt}}
\multiput(795.34,605.85)(8.000,-13.849){2}{\rule{0.800pt}{1.000pt}}
\multiput(806.40,583.70)(0.520,-1.212){9}{\rule{0.125pt}{2.000pt}}
\multiput(803.34,587.85)(8.000,-13.849){2}{\rule{0.800pt}{1.000pt}}
\multiput(814.40,566.11)(0.520,-1.139){9}{\rule{0.125pt}{1.900pt}}
\multiput(811.34,570.06)(8.000,-13.056){2}{\rule{0.800pt}{0.950pt}}
\multiput(822.40,549.11)(0.520,-1.139){9}{\rule{0.125pt}{1.900pt}}
\multiput(819.34,553.06)(8.000,-13.056){2}{\rule{0.800pt}{0.950pt}}
\multiput(830.40,532.11)(0.520,-1.139){9}{\rule{0.125pt}{1.900pt}}
\multiput(827.34,536.06)(8.000,-13.056){2}{\rule{0.800pt}{0.950pt}}
\multiput(838.40,515.94)(0.520,-0.993){9}{\rule{0.125pt}{1.700pt}}
\multiput(835.34,519.47)(8.000,-11.472){2}{\rule{0.800pt}{0.850pt}}
\multiput(846.40,500.94)(0.520,-0.993){9}{\rule{0.125pt}{1.700pt}}
\multiput(843.34,504.47)(8.000,-11.472){2}{\rule{0.800pt}{0.850pt}}
\multiput(854.40,485.53)(0.526,-1.088){7}{\rule{0.127pt}{1.800pt}}
\multiput(851.34,489.26)(7.000,-10.264){2}{\rule{0.800pt}{0.900pt}}
\multiput(861.40,472.77)(0.520,-0.847){9}{\rule{0.125pt}{1.500pt}}
\multiput(858.34,475.89)(8.000,-9.887){2}{\rule{0.800pt}{0.750pt}}
\multiput(869.40,460.19)(0.520,-0.774){9}{\rule{0.125pt}{1.400pt}}
\multiput(866.34,463.09)(8.000,-9.094){2}{\rule{0.800pt}{0.700pt}}
\multiput(877.40,448.60)(0.520,-0.700){9}{\rule{0.125pt}{1.300pt}}
\multiput(874.34,451.30)(8.000,-8.302){2}{\rule{0.800pt}{0.650pt}}
\multiput(885.40,438.43)(0.520,-0.554){9}{\rule{0.125pt}{1.100pt}}
\multiput(882.34,440.72)(8.000,-6.717){2}{\rule{0.800pt}{0.550pt}}
\multiput(892.00,432.08)(0.481,-0.520){9}{\rule{1.000pt}{0.125pt}}
\multiput(892.00,432.34)(5.924,-8.000){2}{\rule{0.500pt}{0.800pt}}
\multiput(900.00,424.08)(0.562,-0.526){7}{\rule{1.114pt}{0.127pt}}
\multiput(900.00,424.34)(5.687,-7.000){2}{\rule{0.557pt}{0.800pt}}
\multiput(908.00,417.06)(0.928,-0.560){3}{\rule{1.480pt}{0.135pt}}
\multiput(908.00,417.34)(4.928,-5.000){2}{\rule{0.740pt}{0.800pt}}
\put(916,410.34){\rule{1.800pt}{0.800pt}}
\multiput(916.00,412.34)(4.264,-4.000){2}{\rule{0.900pt}{0.800pt}}
\put(924,406.84){\rule{1.927pt}{0.800pt}}
\multiput(924.00,408.34)(4.000,-3.000){2}{\rule{0.964pt}{0.800pt}}
\put(932,404.84){\rule{1.927pt}{0.800pt}}
\multiput(932.00,405.34)(4.000,-1.000){2}{\rule{0.964pt}{0.800pt}}
\put(940,404.84){\rule{1.927pt}{0.800pt}}
\multiput(940.00,404.34)(4.000,1.000){2}{\rule{0.964pt}{0.800pt}}
\put(948,405.84){\rule{1.927pt}{0.800pt}}
\multiput(948.00,405.34)(4.000,1.000){2}{\rule{0.964pt}{0.800pt}}
\put(956,408.34){\rule{1.800pt}{0.800pt}}
\multiput(956.00,406.34)(4.264,4.000){2}{\rule{0.900pt}{0.800pt}}
\put(964,412.34){\rule{1.600pt}{0.800pt}}
\multiput(964.00,410.34)(3.679,4.000){2}{\rule{0.800pt}{0.800pt}}
\multiput(971.00,417.39)(0.685,0.536){5}{\rule{1.267pt}{0.129pt}}
\multiput(971.00,414.34)(5.371,6.000){2}{\rule{0.633pt}{0.800pt}}
\multiput(979.00,423.39)(0.685,0.536){5}{\rule{1.267pt}{0.129pt}}
\multiput(979.00,420.34)(5.371,6.000){2}{\rule{0.633pt}{0.800pt}}
\multiput(987.00,429.40)(0.481,0.520){9}{\rule{1.000pt}{0.125pt}}
\multiput(987.00,426.34)(5.924,8.000){2}{\rule{0.500pt}{0.800pt}}
\multiput(996.40,436.00)(0.520,0.554){9}{\rule{0.125pt}{1.100pt}}
\multiput(993.34,436.00)(8.000,6.717){2}{\rule{0.800pt}{0.550pt}}
\multiput(1004.40,445.00)(0.520,0.627){9}{\rule{0.125pt}{1.200pt}}
\multiput(1001.34,445.00)(8.000,7.509){2}{\rule{0.800pt}{0.600pt}}
\multiput(1012.40,455.00)(0.520,0.627){9}{\rule{0.125pt}{1.200pt}}
\multiput(1009.34,455.00)(8.000,7.509){2}{\rule{0.800pt}{0.600pt}}
\multiput(1020.40,465.00)(0.520,0.774){9}{\rule{0.125pt}{1.400pt}}
\multiput(1017.34,465.00)(8.000,9.094){2}{\rule{0.800pt}{0.700pt}}
\multiput(1028.40,477.00)(0.520,0.774){9}{\rule{0.125pt}{1.400pt}}
\multiput(1025.34,477.00)(8.000,9.094){2}{\rule{0.800pt}{0.700pt}}
\multiput(1036.40,489.00)(0.520,0.774){9}{\rule{0.125pt}{1.400pt}}
\multiput(1033.34,489.00)(8.000,9.094){2}{\rule{0.800pt}{0.700pt}}
\multiput(1044.40,501.00)(0.520,0.847){9}{\rule{0.125pt}{1.500pt}}
\multiput(1041.34,501.00)(8.000,9.887){2}{\rule{0.800pt}{0.750pt}}
\multiput(1052.40,514.00)(0.520,0.847){9}{\rule{0.125pt}{1.500pt}}
\multiput(1049.34,514.00)(8.000,9.887){2}{\rule{0.800pt}{0.750pt}}
\multiput(1060.40,527.00)(0.520,0.847){9}{\rule{0.125pt}{1.500pt}}
\multiput(1057.34,527.00)(8.000,9.887){2}{\rule{0.800pt}{0.750pt}}
\multiput(1068.40,540.00)(0.526,1.000){7}{\rule{0.127pt}{1.686pt}}
\multiput(1065.34,540.00)(7.000,9.501){2}{\rule{0.800pt}{0.843pt}}
\multiput(1075.40,553.00)(0.520,0.847){9}{\rule{0.125pt}{1.500pt}}
\multiput(1072.34,553.00)(8.000,9.887){2}{\rule{0.800pt}{0.750pt}}
\multiput(1083.40,566.00)(0.520,0.847){9}{\rule{0.125pt}{1.500pt}}
\multiput(1080.34,566.00)(8.000,9.887){2}{\rule{0.800pt}{0.750pt}}
\multiput(1091.40,579.00)(0.520,0.847){9}{\rule{0.125pt}{1.500pt}}
\multiput(1088.34,579.00)(8.000,9.887){2}{\rule{0.800pt}{0.750pt}}
\multiput(1099.40,592.00)(0.520,0.774){9}{\rule{0.125pt}{1.400pt}}
\multiput(1096.34,592.00)(8.000,9.094){2}{\rule{0.800pt}{0.700pt}}
\multiput(1107.40,604.00)(0.520,0.774){9}{\rule{0.125pt}{1.400pt}}
\multiput(1104.34,604.00)(8.000,9.094){2}{\rule{0.800pt}{0.700pt}}
\multiput(1115.40,616.00)(0.520,0.700){9}{\rule{0.125pt}{1.300pt}}
\multiput(1112.34,616.00)(8.000,8.302){2}{\rule{0.800pt}{0.650pt}}
\multiput(1123.40,627.00)(0.520,0.627){9}{\rule{0.125pt}{1.200pt}}
\multiput(1120.34,627.00)(8.000,7.509){2}{\rule{0.800pt}{0.600pt}}
\multiput(1131.40,637.00)(0.520,0.627){9}{\rule{0.125pt}{1.200pt}}
\multiput(1128.34,637.00)(8.000,7.509){2}{\rule{0.800pt}{0.600pt}}
\multiput(1139.40,647.00)(0.520,0.554){9}{\rule{0.125pt}{1.100pt}}
\multiput(1136.34,647.00)(8.000,6.717){2}{\rule{0.800pt}{0.550pt}}
\multiput(1146.00,657.40)(0.481,0.520){9}{\rule{1.000pt}{0.125pt}}
\multiput(1146.00,654.34)(5.924,8.000){2}{\rule{0.500pt}{0.800pt}}
\multiput(1154.00,665.40)(0.562,0.526){7}{\rule{1.114pt}{0.127pt}}
\multiput(1154.00,662.34)(5.687,7.000){2}{\rule{0.557pt}{0.800pt}}
\multiput(1162.00,672.39)(0.685,0.536){5}{\rule{1.267pt}{0.129pt}}
\multiput(1162.00,669.34)(5.371,6.000){2}{\rule{0.633pt}{0.800pt}}
\multiput(1170.00,678.38)(0.760,0.560){3}{\rule{1.320pt}{0.135pt}}
\multiput(1170.00,675.34)(4.260,5.000){2}{\rule{0.660pt}{0.800pt}}
\put(1177,682.34){\rule{1.800pt}{0.800pt}}
\multiput(1177.00,680.34)(4.264,4.000){2}{\rule{0.900pt}{0.800pt}}
\put(1185,686.34){\rule{1.800pt}{0.800pt}}
\multiput(1185.00,684.34)(4.264,4.000){2}{\rule{0.900pt}{0.800pt}}
\put(1193,689.34){\rule{1.927pt}{0.800pt}}
\multiput(1193.00,688.34)(4.000,2.000){2}{\rule{0.964pt}{0.800pt}}
\put(393.0,268.0){\rule[-0.400pt]{1.927pt}{0.800pt}}
\put(1217,689.84){\rule{1.927pt}{0.800pt}}
\multiput(1217.00,690.34)(4.000,-1.000){2}{\rule{0.964pt}{0.800pt}}
\put(1225,688.34){\rule{1.927pt}{0.800pt}}
\multiput(1225.00,689.34)(4.000,-2.000){2}{\rule{0.964pt}{0.800pt}}
\put(1233,685.84){\rule{1.927pt}{0.800pt}}
\multiput(1233.00,687.34)(4.000,-3.000){2}{\rule{0.964pt}{0.800pt}}
\put(1241,682.34){\rule{1.800pt}{0.800pt}}
\multiput(1241.00,684.34)(4.264,-4.000){2}{\rule{0.900pt}{0.800pt}}
\put(1249,678.34){\rule{1.800pt}{0.800pt}}
\multiput(1249.00,680.34)(4.264,-4.000){2}{\rule{0.900pt}{0.800pt}}
\multiput(1257.00,676.07)(0.685,-0.536){5}{\rule{1.267pt}{0.129pt}}
\multiput(1257.00,676.34)(5.371,-6.000){2}{\rule{0.633pt}{0.800pt}}
\multiput(1265.00,670.07)(0.685,-0.536){5}{\rule{1.267pt}{0.129pt}}
\multiput(1265.00,670.34)(5.371,-6.000){2}{\rule{0.633pt}{0.800pt}}
\multiput(1273.00,664.08)(0.562,-0.526){7}{\rule{1.114pt}{0.127pt}}
\multiput(1273.00,664.34)(5.687,-7.000){2}{\rule{0.557pt}{0.800pt}}
\multiput(1281.00,657.08)(0.475,-0.526){7}{\rule{1.000pt}{0.127pt}}
\multiput(1281.00,657.34)(4.924,-7.000){2}{\rule{0.500pt}{0.800pt}}
\multiput(1289.40,647.43)(0.520,-0.554){9}{\rule{0.125pt}{1.100pt}}
\multiput(1286.34,649.72)(8.000,-6.717){2}{\rule{0.800pt}{0.550pt}}
\multiput(1296.00,641.08)(0.481,-0.520){9}{\rule{1.000pt}{0.125pt}}
\multiput(1296.00,641.34)(5.924,-8.000){2}{\rule{0.500pt}{0.800pt}}
\multiput(1305.40,630.43)(0.520,-0.554){9}{\rule{0.125pt}{1.100pt}}
\multiput(1302.34,632.72)(8.000,-6.717){2}{\rule{0.800pt}{0.550pt}}
\multiput(1313.40,621.43)(0.520,-0.554){9}{\rule{0.125pt}{1.100pt}}
\multiput(1310.34,623.72)(8.000,-6.717){2}{\rule{0.800pt}{0.550pt}}
\multiput(1321.40,612.43)(0.520,-0.554){9}{\rule{0.125pt}{1.100pt}}
\multiput(1318.34,614.72)(8.000,-6.717){2}{\rule{0.800pt}{0.550pt}}
\multiput(1329.40,603.02)(0.520,-0.627){9}{\rule{0.125pt}{1.200pt}}
\multiput(1326.34,605.51)(8.000,-7.509){2}{\rule{0.800pt}{0.600pt}}
\multiput(1337.40,593.43)(0.520,-0.554){9}{\rule{0.125pt}{1.100pt}}
\multiput(1334.34,595.72)(8.000,-6.717){2}{\rule{0.800pt}{0.550pt}}
\multiput(1345.40,584.02)(0.520,-0.627){9}{\rule{0.125pt}{1.200pt}}
\multiput(1342.34,586.51)(8.000,-7.509){2}{\rule{0.800pt}{0.600pt}}
\multiput(1353.40,574.43)(0.520,-0.554){9}{\rule{0.125pt}{1.100pt}}
\multiput(1350.34,576.72)(8.000,-6.717){2}{\rule{0.800pt}{0.550pt}}
\multiput(1361.40,565.43)(0.520,-0.554){9}{\rule{0.125pt}{1.100pt}}
\multiput(1358.34,567.72)(8.000,-6.717){2}{\rule{0.800pt}{0.550pt}}
\multiput(1369.40,556.43)(0.520,-0.554){9}{\rule{0.125pt}{1.100pt}}
\multiput(1366.34,558.72)(8.000,-6.717){2}{\rule{0.800pt}{0.550pt}}
\multiput(1377.40,547.43)(0.520,-0.554){9}{\rule{0.125pt}{1.100pt}}
\multiput(1374.34,549.72)(8.000,-6.717){2}{\rule{0.800pt}{0.550pt}}
\multiput(1385.40,538.37)(0.526,-0.562){7}{\rule{0.127pt}{1.114pt}}
\multiput(1382.34,540.69)(7.000,-5.687){2}{\rule{0.800pt}{0.557pt}}
\multiput(1391.00,533.08)(0.562,-0.526){7}{\rule{1.114pt}{0.127pt}}
\multiput(1391.00,533.34)(5.687,-7.000){2}{\rule{0.557pt}{0.800pt}}
\multiput(1399.00,526.08)(0.481,-0.520){9}{\rule{1.000pt}{0.125pt}}
\multiput(1399.00,526.34)(5.924,-8.000){2}{\rule{0.500pt}{0.800pt}}
\multiput(1407.00,518.07)(0.685,-0.536){5}{\rule{1.267pt}{0.129pt}}
\multiput(1407.00,518.34)(5.371,-6.000){2}{\rule{0.633pt}{0.800pt}}
\multiput(1415.00,512.07)(0.685,-0.536){5}{\rule{1.267pt}{0.129pt}}
\multiput(1415.00,512.34)(5.371,-6.000){2}{\rule{0.633pt}{0.800pt}}
\multiput(1423.00,506.07)(0.685,-0.536){5}{\rule{1.267pt}{0.129pt}}
\multiput(1423.00,506.34)(5.371,-6.000){2}{\rule{0.633pt}{0.800pt}}
\put(1431,498.34){\rule{1.800pt}{0.800pt}}
\multiput(1431.00,500.34)(4.264,-4.000){2}{\rule{0.900pt}{0.800pt}}
\put(1201.0,692.0){\rule[-0.400pt]{3.854pt}{0.800pt}}
\put(1439,498){\usebox{\plotpoint}}
\sbox{\plotpoint}{\rule[-0.500pt]{1.000pt}{1.000pt}}%
\sbox{\plotpoint}{\rule[-0.200pt]{0.400pt}{0.400pt}}%
\put(1279,295){\makebox(0,0)[r]{$I="2.5 A"$}}
\sbox{\plotpoint}{\rule[-0.500pt]{1.000pt}{1.000pt}}%
\multiput(1299,295)(20.756,0.000){5}{\usebox{\plotpoint}}
\put(1399,295){\usebox{\plotpoint}}
\put(171,565){\usebox{\plotpoint}}
\put(171.00,565.00){\usebox{\plotpoint}}
\put(191.76,565.00){\usebox{\plotpoint}}
\put(212.35,562.66){\usebox{\plotpoint}}
\put(232.19,556.68){\usebox{\plotpoint}}
\put(251.14,548.29){\usebox{\plotpoint}}
\put(267.88,536.12){\usebox{\plotpoint}}
\put(282.04,520.95){\usebox{\plotpoint}}
\put(295.16,504.90){\usebox{\plotpoint}}
\put(307.73,488.40){\usebox{\plotpoint}}
\put(318.69,470.79){\usebox{\plotpoint}}
\put(328.30,452.40){\usebox{\plotpoint}}
\put(336.80,433.47){\usebox{\plotpoint}}
\multiput(345,415)(7.708,-19.271){2}{\usebox{\plotpoint}}
\put(361.39,376.28){\usebox{\plotpoint}}
\put(371.69,358.30){\usebox{\plotpoint}}
\put(384.79,342.23){\usebox{\plotpoint}}
\put(402.88,332.76){\usebox{\plotpoint}}
\put(423.11,335.29){\usebox{\plotpoint}}
\put(440.07,347.06){\usebox{\plotpoint}}
\put(455.22,361.22){\usebox{\plotpoint}}
\put(468.76,376.95){\usebox{\plotpoint}}
\put(481.62,393.23){\usebox{\plotpoint}}
\put(493.83,410.02){\usebox{\plotpoint}}
\put(506.16,426.70){\usebox{\plotpoint}}
\put(518.33,443.50){\usebox{\plotpoint}}
\put(530.91,459.99){\usebox{\plotpoint}}
\put(542.49,477.20){\usebox{\plotpoint}}
\put(555.38,493.47){\usebox{\plotpoint}}
\put(568.43,509.60){\usebox{\plotpoint}}
\put(582.68,524.68){\usebox{\plotpoint}}
\put(597.22,539.44){\usebox{\plotpoint}}
\put(612.84,553.11){\usebox{\plotpoint}}
\put(629.68,565.18){\usebox{\plotpoint}}
\put(647.72,575.41){\usebox{\plotpoint}}
\put(666.66,583.75){\usebox{\plotpoint}}
\put(686.67,589.17){\usebox{\plotpoint}}
\put(707.14,592.00){\usebox{\plotpoint}}
\put(727.83,593.00){\usebox{\plotpoint}}
\put(748.59,593.00){\usebox{\plotpoint}}
\put(769.21,591.00){\usebox{\plotpoint}}
\put(789.84,588.90){\usebox{\plotpoint}}
\put(810.43,586.32){\usebox{\plotpoint}}
\put(831.03,583.75){\usebox{\plotpoint}}
\put(851.62,581.17){\usebox{\plotpoint}}
\put(872.20,578.47){\usebox{\plotpoint}}
\put(892.86,576.89){\usebox{\plotpoint}}
\put(913.45,574.32){\usebox{\plotpoint}}
\put(934.13,573.00){\usebox{\plotpoint}}
\put(954.82,572.00){\usebox{\plotpoint}}
\put(975.51,571.00){\usebox{\plotpoint}}
\put(996.21,570.00){\usebox{\plotpoint}}
\put(1016.90,569.00){\usebox{\plotpoint}}
\put(1037.65,569.00){\usebox{\plotpoint}}
\put(1058.41,569.00){\usebox{\plotpoint}}
\put(1079.17,569.00){\usebox{\plotpoint}}
\put(1099.92,569.00){\usebox{\plotpoint}}
\put(1120.68,569.00){\usebox{\plotpoint}}
\put(1141.43,569.00){\usebox{\plotpoint}}
\put(1162.19,569.00){\usebox{\plotpoint}}
\put(1182.94,569.00){\usebox{\plotpoint}}
\put(1203.70,569.00){\usebox{\plotpoint}}
\put(1224.45,569.00){\usebox{\plotpoint}}
\put(1245.21,569.00){\usebox{\plotpoint}}
\put(1265.96,569.00){\usebox{\plotpoint}}
\put(1286.72,569.00){\usebox{\plotpoint}}
\put(1307.48,569.00){\usebox{\plotpoint}}
\put(1312,569){\usebox{\plotpoint}}
\sbox{\plotpoint}{\rule[-0.600pt]{1.200pt}{1.200pt}}%
\sbox{\plotpoint}{\rule[-0.200pt]{0.400pt}{0.400pt}}%
\put(1279,254){\makebox(0,0)[r]{$I="2.8 A"$}}
\sbox{\plotpoint}{\rule[-0.600pt]{1.200pt}{1.200pt}}%
\put(1299.0,254.0){\rule[-0.600pt]{24.090pt}{1.200pt}}
\put(171,564){\usebox{\plotpoint}}
\put(219,561.01){\rule{1.686pt}{1.200pt}}
\multiput(219.00,561.51)(3.500,-1.000){2}{\rule{0.843pt}{1.200pt}}
\put(171.0,564.0){\rule[-0.600pt]{11.563pt}{1.200pt}}
\put(250,560.01){\rule{1.927pt}{1.200pt}}
\multiput(250.00,560.51)(4.000,-1.000){2}{\rule{0.964pt}{1.200pt}}
\put(226.0,563.0){\rule[-0.600pt]{5.782pt}{1.200pt}}
\put(298,559.01){\rule{1.927pt}{1.200pt}}
\multiput(298.00,559.51)(4.000,-1.000){2}{\rule{0.964pt}{1.200pt}}
\put(258.0,562.0){\rule[-0.600pt]{9.636pt}{1.200pt}}
\put(330,558.01){\rule{1.686pt}{1.200pt}}
\multiput(330.00,558.51)(3.500,-1.000){2}{\rule{0.843pt}{1.200pt}}
\put(306.0,561.0){\rule[-0.600pt]{5.782pt}{1.200pt}}
\put(369,558.01){\rule{1.927pt}{1.200pt}}
\multiput(369.00,557.51)(4.000,1.000){2}{\rule{0.964pt}{1.200pt}}
\put(337.0,560.0){\rule[-0.600pt]{7.709pt}{1.200pt}}
\put(417,558.01){\rule{1.927pt}{1.200pt}}
\multiput(417.00,558.51)(4.000,-1.000){2}{\rule{0.964pt}{1.200pt}}
\put(425,557.01){\rule{1.927pt}{1.200pt}}
\multiput(425.00,557.51)(4.000,-1.000){2}{\rule{0.964pt}{1.200pt}}
\put(377.0,561.0){\rule[-0.600pt]{9.636pt}{1.200pt}}
\put(440,556.01){\rule{1.927pt}{1.200pt}}
\multiput(440.00,556.51)(4.000,-1.000){2}{\rule{0.964pt}{1.200pt}}
\put(433.0,559.0){\rule[-0.600pt]{1.686pt}{1.200pt}}
\put(456,555.01){\rule{1.927pt}{1.200pt}}
\multiput(456.00,555.51)(4.000,-1.000){2}{\rule{0.964pt}{1.200pt}}
\put(448.0,558.0){\rule[-0.600pt]{1.927pt}{1.200pt}}
\put(488,554.01){\rule{1.927pt}{1.200pt}}
\multiput(488.00,554.51)(4.000,-1.000){2}{\rule{0.964pt}{1.200pt}}
\put(464.0,557.0){\rule[-0.600pt]{5.782pt}{1.200pt}}
\put(504,553.01){\rule{1.927pt}{1.200pt}}
\multiput(504.00,553.51)(4.000,-1.000){2}{\rule{0.964pt}{1.200pt}}
\put(496.0,556.0){\rule[-0.600pt]{1.927pt}{1.200pt}}
\put(559,553.01){\rule{1.927pt}{1.200pt}}
\multiput(559.00,552.51)(4.000,1.000){2}{\rule{0.964pt}{1.200pt}}
\put(512.0,555.0){\rule[-0.600pt]{11.322pt}{1.200pt}}
\put(591,554.01){\rule{1.927pt}{1.200pt}}
\multiput(591.00,553.51)(4.000,1.000){2}{\rule{0.964pt}{1.200pt}}
\put(567.0,556.0){\rule[-0.600pt]{5.782pt}{1.200pt}}
\put(615,555.01){\rule{1.927pt}{1.200pt}}
\multiput(615.00,554.51)(4.000,1.000){2}{\rule{0.964pt}{1.200pt}}
\put(599.0,557.0){\rule[-0.600pt]{3.854pt}{1.200pt}}
\put(631,556.01){\rule{1.927pt}{1.200pt}}
\multiput(631.00,555.51)(4.000,1.000){2}{\rule{0.964pt}{1.200pt}}
\put(623.0,558.0){\rule[-0.600pt]{1.927pt}{1.200pt}}
\put(654,557.01){\rule{1.927pt}{1.200pt}}
\multiput(654.00,556.51)(4.000,1.000){2}{\rule{0.964pt}{1.200pt}}
\put(639.0,559.0){\rule[-0.600pt]{3.613pt}{1.200pt}}
\put(678,557.01){\rule{1.927pt}{1.200pt}}
\multiput(678.00,557.51)(4.000,-1.000){2}{\rule{0.964pt}{1.200pt}}
\put(662.0,560.0){\rule[-0.600pt]{3.854pt}{1.200pt}}
\put(750,557.01){\rule{1.686pt}{1.200pt}}
\multiput(750.00,556.51)(3.500,1.000){2}{\rule{0.843pt}{1.200pt}}
\put(686.0,559.0){\rule[-0.600pt]{15.418pt}{1.200pt}}
\put(845,558.01){\rule{1.927pt}{1.200pt}}
\multiput(845.00,557.51)(4.000,1.000){2}{\rule{0.964pt}{1.200pt}}
\put(757.0,560.0){\rule[-0.600pt]{21.199pt}{1.200pt}}
\put(884,559.01){\rule{1.927pt}{1.200pt}}
\multiput(884.00,558.51)(4.000,1.000){2}{\rule{0.964pt}{1.200pt}}
\put(853.0,561.0){\rule[-0.600pt]{7.468pt}{1.200pt}}
\put(908,560.01){\rule{1.927pt}{1.200pt}}
\multiput(908.00,559.51)(4.000,1.000){2}{\rule{0.964pt}{1.200pt}}
\put(892.0,562.0){\rule[-0.600pt]{3.854pt}{1.200pt}}
\put(932,561.01){\rule{1.927pt}{1.200pt}}
\multiput(932.00,560.51)(4.000,1.000){2}{\rule{0.964pt}{1.200pt}}
\put(916.0,563.0){\rule[-0.600pt]{3.854pt}{1.200pt}}
\put(964,562.01){\rule{1.686pt}{1.200pt}}
\multiput(964.00,561.51)(3.500,1.000){2}{\rule{0.843pt}{1.200pt}}
\put(940.0,564.0){\rule[-0.600pt]{5.782pt}{1.200pt}}
\put(995,562.01){\rule{1.927pt}{1.200pt}}
\multiput(995.00,562.51)(4.000,-1.000){2}{\rule{0.964pt}{1.200pt}}
\put(1003,561.01){\rule{1.927pt}{1.200pt}}
\multiput(1003.00,561.51)(4.000,-1.000){2}{\rule{0.964pt}{1.200pt}}
\put(971.0,565.0){\rule[-0.600pt]{5.782pt}{1.200pt}}
\put(1019,560.01){\rule{1.927pt}{1.200pt}}
\multiput(1019.00,560.51)(4.000,-1.000){2}{\rule{0.964pt}{1.200pt}}
\put(1027,559.01){\rule{1.927pt}{1.200pt}}
\multiput(1027.00,559.51)(4.000,-1.000){2}{\rule{0.964pt}{1.200pt}}
\put(1035,558.01){\rule{1.927pt}{1.200pt}}
\multiput(1035.00,558.51)(4.000,-1.000){2}{\rule{0.964pt}{1.200pt}}
\put(1011.0,563.0){\rule[-0.600pt]{1.927pt}{1.200pt}}
\put(1051,557.01){\rule{1.927pt}{1.200pt}}
\multiput(1051.00,557.51)(4.000,-1.000){2}{\rule{0.964pt}{1.200pt}}
\put(1059,556.01){\rule{1.927pt}{1.200pt}}
\multiput(1059.00,556.51)(4.000,-1.000){2}{\rule{0.964pt}{1.200pt}}
\put(1043.0,560.0){\rule[-0.600pt]{1.927pt}{1.200pt}}
\put(1114,555.01){\rule{1.927pt}{1.200pt}}
\multiput(1114.00,555.51)(4.000,-1.000){2}{\rule{0.964pt}{1.200pt}}
\put(1122,554.01){\rule{1.927pt}{1.200pt}}
\multiput(1122.00,554.51)(4.000,-1.000){2}{\rule{0.964pt}{1.200pt}}
\put(1130,553.01){\rule{1.927pt}{1.200pt}}
\multiput(1130.00,553.51)(4.000,-1.000){2}{\rule{0.964pt}{1.200pt}}
\put(1067.0,558.0){\rule[-0.600pt]{11.322pt}{1.200pt}}
\put(1146,552.01){\rule{1.927pt}{1.200pt}}
\multiput(1146.00,552.51)(4.000,-1.000){2}{\rule{0.964pt}{1.200pt}}
\put(1138.0,555.0){\rule[-0.600pt]{1.927pt}{1.200pt}}
\put(1162,551.01){\rule{1.927pt}{1.200pt}}
\multiput(1162.00,551.51)(4.000,-1.000){2}{\rule{0.964pt}{1.200pt}}
\put(1154.0,554.0){\rule[-0.600pt]{1.927pt}{1.200pt}}
\put(1193,551.01){\rule{1.927pt}{1.200pt}}
\multiput(1193.00,550.51)(4.000,1.000){2}{\rule{0.964pt}{1.200pt}}
\put(1170.0,553.0){\rule[-0.600pt]{5.541pt}{1.200pt}}
\put(1217,552.01){\rule{1.927pt}{1.200pt}}
\multiput(1217.00,551.51)(4.000,1.000){2}{\rule{0.964pt}{1.200pt}}
\put(1201.0,554.0){\rule[-0.600pt]{3.854pt}{1.200pt}}
\put(1257,552.01){\rule{1.927pt}{1.200pt}}
\multiput(1257.00,552.51)(4.000,-1.000){2}{\rule{0.964pt}{1.200pt}}
\put(1225.0,555.0){\rule[-0.600pt]{7.709pt}{1.200pt}}
\put(1281,551.01){\rule{1.686pt}{1.200pt}}
\multiput(1281.00,551.51)(3.500,-1.000){2}{\rule{0.843pt}{1.200pt}}
\put(1265.0,554.0){\rule[-0.600pt]{3.854pt}{1.200pt}}
\put(1344,550.01){\rule{1.927pt}{1.200pt}}
\multiput(1344.00,550.51)(4.000,-1.000){2}{\rule{0.964pt}{1.200pt}}
\put(1288.0,553.0){\rule[-0.600pt]{13.490pt}{1.200pt}}
\put(1360,549.01){\rule{1.927pt}{1.200pt}}
\multiput(1360.00,549.51)(4.000,-1.000){2}{\rule{0.964pt}{1.200pt}}
\put(1368,548.01){\rule{1.927pt}{1.200pt}}
\multiput(1368.00,548.51)(4.000,-1.000){2}{\rule{0.964pt}{1.200pt}}
\put(1376,547.01){\rule{1.927pt}{1.200pt}}
\multiput(1376.00,547.51)(4.000,-1.000){2}{\rule{0.964pt}{1.200pt}}
\put(1384,545.51){\rule{1.686pt}{1.200pt}}
\multiput(1384.00,546.51)(3.500,-2.000){2}{\rule{0.843pt}{1.200pt}}
\put(1391,543.51){\rule{1.927pt}{1.200pt}}
\multiput(1391.00,544.51)(4.000,-2.000){2}{\rule{0.964pt}{1.200pt}}
\multiput(1399.00,542.25)(0.113,-0.509){2}{\rule{1.900pt}{0.123pt}}
\multiput(1399.00,542.51)(4.056,-6.000){2}{\rule{0.950pt}{1.200pt}}
\multiput(1409.24,530.91)(0.503,-0.581){6}{\rule{0.121pt}{1.950pt}}
\multiput(1404.51,534.95)(8.000,-6.953){2}{\rule{1.200pt}{0.975pt}}
\multiput(1417.24,518.04)(0.503,-0.807){6}{\rule{0.121pt}{2.400pt}}
\multiput(1412.51,523.02)(8.000,-9.019){2}{\rule{1.200pt}{1.200pt}}
\multiput(1425.24,501.55)(0.503,-1.109){6}{\rule{0.121pt}{3.000pt}}
\multiput(1420.51,507.77)(8.000,-11.773){2}{\rule{1.200pt}{1.500pt}}
\multiput(1433.24,482.30)(0.503,-1.260){6}{\rule{0.121pt}{3.300pt}}
\multiput(1428.51,489.15)(8.000,-13.151){2}{\rule{1.200pt}{1.650pt}}
\put(1352.0,552.0){\rule[-0.600pt]{1.927pt}{1.200pt}}
\put(1439,476){\usebox{\plotpoint}}
\sbox{\plotpoint}{\rule[-0.500pt]{1.000pt}{1.000pt}}%
\sbox{\plotpoint}{\rule[-0.200pt]{0.400pt}{0.400pt}}%
\put(1279,213){\makebox(0,0)[r]{$I="2 A"$}}
\sbox{\plotpoint}{\rule[-0.500pt]{1.000pt}{1.000pt}}%
\multiput(1299,213)(41.511,0.000){3}{\usebox{\plotpoint}}
\put(1399,213){\usebox{\plotpoint}}
\put(171,565){\usebox{\plotpoint}}
\put(171.00,565.00){\usebox{\plotpoint}}
\put(212.51,565.00){\usebox{\plotpoint}}
\put(252.83,557.58){\usebox{\plotpoint}}
\put(277.72,525.63){\usebox{\plotpoint}}
\put(292.39,486.83){\usebox{\plotpoint}}
\put(306.14,447.66){\usebox{\plotpoint}}
\put(322.29,409.43){\usebox{\plotpoint}}
\put(343.48,373.90){\usebox{\plotpoint}}
\put(377.59,353.00){\usebox{\plotpoint}}
\put(414.53,369.15){\usebox{\plotpoint}}
\put(442.16,399.96){\usebox{\plotpoint}}
\put(466.43,433.64){\usebox{\plotpoint}}
\put(488.98,468.48){\usebox{\plotpoint}}
\put(511.54,503.31){\usebox{\plotpoint}}
\put(534.97,537.58){\usebox{\plotpoint}}
\put(560.19,570.49){\usebox{\plotpoint}}
\put(589.52,599.71){\usebox{\plotpoint}}
\put(624.62,621.61){\usebox{\plotpoint}}
\put(664.82,630.35){\usebox{\plotpoint}}
\put(705.62,625.09){\usebox{\plotpoint}}
\put(745.06,612.47){\usebox{\plotpoint}}
\put(783.09,595.96){\usebox{\plotpoint}}
\put(821.68,580.75){\usebox{\plotpoint}}
\put(861.05,567.74){\usebox{\plotpoint}}
\put(901.71,559.79){\usebox{\plotpoint}}
\put(942.99,556.00){\usebox{\plotpoint}}
\put(984.44,557.00){\usebox{\plotpoint}}
\put(1025.77,559.85){\usebox{\plotpoint}}
\put(1067.02,564.00){\usebox{\plotpoint}}
\put(1108.28,568.00){\usebox{\plotpoint}}
\put(1149.60,571.00){\usebox{\plotpoint}}
\put(1191.00,572.75){\usebox{\plotpoint}}
\put(1232.50,573.00){\usebox{\plotpoint}}
\put(1274.01,573.00){\usebox{\plotpoint}}
\put(1315.46,572.00){\usebox{\plotpoint}}
\put(1356.97,572.00){\usebox{\plotpoint}}
\put(1398.42,571.00){\usebox{\plotpoint}}
\put(1439,571){\usebox{\plotpoint}}
\sbox{\plotpoint}{\rule[-0.200pt]{0.400pt}{0.400pt}}%
\put(1279,172){\makebox(0,0)[r]{$I="3 A"$}}
\put(1299.0,172.0){\rule[-0.200pt]{24.090pt}{0.400pt}}
\put(171,564){\usebox{\plotpoint}}
\put(226,562.67){\rule{1.927pt}{0.400pt}}
\multiput(226.00,563.17)(4.000,-1.000){2}{\rule{0.964pt}{0.400pt}}
\put(234,561.67){\rule{1.927pt}{0.400pt}}
\multiput(234.00,562.17)(4.000,-1.000){2}{\rule{0.964pt}{0.400pt}}
\put(242,560.17){\rule{1.700pt}{0.400pt}}
\multiput(242.00,561.17)(4.472,-2.000){2}{\rule{0.850pt}{0.400pt}}
\put(250,558.67){\rule{1.927pt}{0.400pt}}
\multiput(250.00,559.17)(4.000,-1.000){2}{\rule{0.964pt}{0.400pt}}
\put(258,557.67){\rule{1.927pt}{0.400pt}}
\multiput(258.00,558.17)(4.000,-1.000){2}{\rule{0.964pt}{0.400pt}}
\put(266,556.67){\rule{1.927pt}{0.400pt}}
\multiput(266.00,557.17)(4.000,-1.000){2}{\rule{0.964pt}{0.400pt}}
\put(274,555.67){\rule{1.927pt}{0.400pt}}
\multiput(274.00,556.17)(4.000,-1.000){2}{\rule{0.964pt}{0.400pt}}
\put(282,554.67){\rule{1.927pt}{0.400pt}}
\multiput(282.00,555.17)(4.000,-1.000){2}{\rule{0.964pt}{0.400pt}}
\put(171.0,564.0){\rule[-0.200pt]{13.249pt}{0.400pt}}
\put(306,553.67){\rule{1.927pt}{0.400pt}}
\multiput(306.00,554.17)(4.000,-1.000){2}{\rule{0.964pt}{0.400pt}}
\multiput(314.00,552.95)(1.579,-0.447){3}{\rule{1.167pt}{0.108pt}}
\multiput(314.00,553.17)(5.579,-3.000){2}{\rule{0.583pt}{0.400pt}}
\put(322,549.17){\rule{1.700pt}{0.400pt}}
\multiput(322.00,550.17)(4.472,-2.000){2}{\rule{0.850pt}{0.400pt}}
\multiput(330.00,547.93)(0.581,-0.482){9}{\rule{0.567pt}{0.116pt}}
\multiput(330.00,548.17)(5.824,-6.000){2}{\rule{0.283pt}{0.400pt}}
\multiput(337.59,540.72)(0.488,-0.560){13}{\rule{0.117pt}{0.550pt}}
\multiput(336.17,541.86)(8.000,-7.858){2}{\rule{0.400pt}{0.275pt}}
\multiput(345.59,531.09)(0.488,-0.758){13}{\rule{0.117pt}{0.700pt}}
\multiput(344.17,532.55)(8.000,-10.547){2}{\rule{0.400pt}{0.350pt}}
\multiput(353.59,518.89)(0.488,-0.824){13}{\rule{0.117pt}{0.750pt}}
\multiput(352.17,520.44)(8.000,-11.443){2}{\rule{0.400pt}{0.375pt}}
\multiput(361.59,505.47)(0.488,-0.956){13}{\rule{0.117pt}{0.850pt}}
\multiput(360.17,507.24)(8.000,-13.236){2}{\rule{0.400pt}{0.425pt}}
\multiput(369.59,490.26)(0.488,-1.022){13}{\rule{0.117pt}{0.900pt}}
\multiput(368.17,492.13)(8.000,-14.132){2}{\rule{0.400pt}{0.450pt}}
\multiput(377.59,474.26)(0.488,-1.022){13}{\rule{0.117pt}{0.900pt}}
\multiput(376.17,476.13)(8.000,-14.132){2}{\rule{0.400pt}{0.450pt}}
\multiput(385.59,458.06)(0.488,-1.088){13}{\rule{0.117pt}{0.950pt}}
\multiput(384.17,460.03)(8.000,-15.028){2}{\rule{0.400pt}{0.475pt}}
\multiput(393.59,440.64)(0.488,-1.220){13}{\rule{0.117pt}{1.050pt}}
\multiput(392.17,442.82)(8.000,-16.821){2}{\rule{0.400pt}{0.525pt}}
\multiput(401.59,421.02)(0.488,-1.418){13}{\rule{0.117pt}{1.200pt}}
\multiput(400.17,423.51)(8.000,-19.509){2}{\rule{0.400pt}{0.600pt}}
\multiput(409.59,399.02)(0.488,-1.418){13}{\rule{0.117pt}{1.200pt}}
\multiput(408.17,401.51)(8.000,-19.509){2}{\rule{0.400pt}{0.600pt}}
\multiput(417.59,377.43)(0.488,-1.286){13}{\rule{0.117pt}{1.100pt}}
\multiput(416.17,379.72)(8.000,-17.717){2}{\rule{0.400pt}{0.550pt}}
\multiput(425.59,358.47)(0.488,-0.956){13}{\rule{0.117pt}{0.850pt}}
\multiput(424.17,360.24)(8.000,-13.236){2}{\rule{0.400pt}{0.425pt}}
\multiput(433.59,343.98)(0.485,-0.798){11}{\rule{0.117pt}{0.729pt}}
\multiput(432.17,345.49)(7.000,-9.488){2}{\rule{0.400pt}{0.364pt}}
\multiput(440.00,334.93)(0.569,-0.485){11}{\rule{0.557pt}{0.117pt}}
\multiput(440.00,335.17)(6.844,-7.000){2}{\rule{0.279pt}{0.400pt}}
\multiput(448.00,327.93)(0.821,-0.477){7}{\rule{0.740pt}{0.115pt}}
\multiput(448.00,328.17)(6.464,-5.000){2}{\rule{0.370pt}{0.400pt}}
\put(456,322.17){\rule{1.700pt}{0.400pt}}
\multiput(456.00,323.17)(4.472,-2.000){2}{\rule{0.850pt}{0.400pt}}
\put(290.0,555.0){\rule[-0.200pt]{3.854pt}{0.400pt}}
\multiput(472.00,322.61)(1.579,0.447){3}{\rule{1.167pt}{0.108pt}}
\multiput(472.00,321.17)(5.579,3.000){2}{\rule{0.583pt}{0.400pt}}
\multiput(480.00,325.60)(1.066,0.468){5}{\rule{0.900pt}{0.113pt}}
\multiput(480.00,324.17)(6.132,4.000){2}{\rule{0.450pt}{0.400pt}}
\multiput(488.00,329.59)(0.821,0.477){7}{\rule{0.740pt}{0.115pt}}
\multiput(488.00,328.17)(6.464,5.000){2}{\rule{0.370pt}{0.400pt}}
\multiput(496.00,334.59)(0.569,0.485){11}{\rule{0.557pt}{0.117pt}}
\multiput(496.00,333.17)(6.844,7.000){2}{\rule{0.279pt}{0.400pt}}
\multiput(504.00,341.59)(0.569,0.485){11}{\rule{0.557pt}{0.117pt}}
\multiput(504.00,340.17)(6.844,7.000){2}{\rule{0.279pt}{0.400pt}}
\multiput(512.00,348.59)(0.494,0.488){13}{\rule{0.500pt}{0.117pt}}
\multiput(512.00,347.17)(6.962,8.000){2}{\rule{0.250pt}{0.400pt}}
\multiput(520.59,356.00)(0.488,0.560){13}{\rule{0.117pt}{0.550pt}}
\multiput(519.17,356.00)(8.000,7.858){2}{\rule{0.400pt}{0.275pt}}
\multiput(528.00,365.59)(0.494,0.488){13}{\rule{0.500pt}{0.117pt}}
\multiput(528.00,364.17)(6.962,8.000){2}{\rule{0.250pt}{0.400pt}}
\multiput(536.59,373.00)(0.485,0.645){11}{\rule{0.117pt}{0.614pt}}
\multiput(535.17,373.00)(7.000,7.725){2}{\rule{0.400pt}{0.307pt}}
\multiput(543.59,382.00)(0.488,0.560){13}{\rule{0.117pt}{0.550pt}}
\multiput(542.17,382.00)(8.000,7.858){2}{\rule{0.400pt}{0.275pt}}
\multiput(551.59,391.00)(0.488,0.560){13}{\rule{0.117pt}{0.550pt}}
\multiput(550.17,391.00)(8.000,7.858){2}{\rule{0.400pt}{0.275pt}}
\multiput(559.59,400.00)(0.488,0.560){13}{\rule{0.117pt}{0.550pt}}
\multiput(558.17,400.00)(8.000,7.858){2}{\rule{0.400pt}{0.275pt}}
\multiput(567.59,409.00)(0.488,0.560){13}{\rule{0.117pt}{0.550pt}}
\multiput(566.17,409.00)(8.000,7.858){2}{\rule{0.400pt}{0.275pt}}
\multiput(575.00,418.59)(0.494,0.488){13}{\rule{0.500pt}{0.117pt}}
\multiput(575.00,417.17)(6.962,8.000){2}{\rule{0.250pt}{0.400pt}}
\multiput(583.59,426.00)(0.488,0.560){13}{\rule{0.117pt}{0.550pt}}
\multiput(582.17,426.00)(8.000,7.858){2}{\rule{0.400pt}{0.275pt}}
\multiput(591.00,435.59)(0.494,0.488){13}{\rule{0.500pt}{0.117pt}}
\multiput(591.00,434.17)(6.962,8.000){2}{\rule{0.250pt}{0.400pt}}
\multiput(599.00,443.59)(0.494,0.488){13}{\rule{0.500pt}{0.117pt}}
\multiput(599.00,442.17)(6.962,8.000){2}{\rule{0.250pt}{0.400pt}}
\multiput(607.00,451.59)(0.494,0.488){13}{\rule{0.500pt}{0.117pt}}
\multiput(607.00,450.17)(6.962,8.000){2}{\rule{0.250pt}{0.400pt}}
\multiput(615.00,459.59)(0.494,0.488){13}{\rule{0.500pt}{0.117pt}}
\multiput(615.00,458.17)(6.962,8.000){2}{\rule{0.250pt}{0.400pt}}
\multiput(623.00,467.59)(0.569,0.485){11}{\rule{0.557pt}{0.117pt}}
\multiput(623.00,466.17)(6.844,7.000){2}{\rule{0.279pt}{0.400pt}}
\multiput(631.00,474.59)(0.569,0.485){11}{\rule{0.557pt}{0.117pt}}
\multiput(631.00,473.17)(6.844,7.000){2}{\rule{0.279pt}{0.400pt}}
\multiput(639.00,481.59)(0.569,0.485){11}{\rule{0.557pt}{0.117pt}}
\multiput(639.00,480.17)(6.844,7.000){2}{\rule{0.279pt}{0.400pt}}
\multiput(647.00,488.59)(0.581,0.482){9}{\rule{0.567pt}{0.116pt}}
\multiput(647.00,487.17)(5.824,6.000){2}{\rule{0.283pt}{0.400pt}}
\multiput(654.00,494.59)(0.671,0.482){9}{\rule{0.633pt}{0.116pt}}
\multiput(654.00,493.17)(6.685,6.000){2}{\rule{0.317pt}{0.400pt}}
\multiput(662.00,500.59)(0.671,0.482){9}{\rule{0.633pt}{0.116pt}}
\multiput(662.00,499.17)(6.685,6.000){2}{\rule{0.317pt}{0.400pt}}
\multiput(670.00,506.59)(0.671,0.482){9}{\rule{0.633pt}{0.116pt}}
\multiput(670.00,505.17)(6.685,6.000){2}{\rule{0.317pt}{0.400pt}}
\multiput(678.00,512.59)(0.821,0.477){7}{\rule{0.740pt}{0.115pt}}
\multiput(678.00,511.17)(6.464,5.000){2}{\rule{0.370pt}{0.400pt}}
\multiput(686.00,517.60)(1.066,0.468){5}{\rule{0.900pt}{0.113pt}}
\multiput(686.00,516.17)(6.132,4.000){2}{\rule{0.450pt}{0.400pt}}
\multiput(694.00,521.59)(0.821,0.477){7}{\rule{0.740pt}{0.115pt}}
\multiput(694.00,520.17)(6.464,5.000){2}{\rule{0.370pt}{0.400pt}}
\multiput(702.00,526.60)(1.066,0.468){5}{\rule{0.900pt}{0.113pt}}
\multiput(702.00,525.17)(6.132,4.000){2}{\rule{0.450pt}{0.400pt}}
\multiput(710.00,530.60)(1.066,0.468){5}{\rule{0.900pt}{0.113pt}}
\multiput(710.00,529.17)(6.132,4.000){2}{\rule{0.450pt}{0.400pt}}
\multiput(718.00,534.61)(1.579,0.447){3}{\rule{1.167pt}{0.108pt}}
\multiput(718.00,533.17)(5.579,3.000){2}{\rule{0.583pt}{0.400pt}}
\multiput(726.00,537.60)(1.066,0.468){5}{\rule{0.900pt}{0.113pt}}
\multiput(726.00,536.17)(6.132,4.000){2}{\rule{0.450pt}{0.400pt}}
\multiput(734.00,541.61)(1.579,0.447){3}{\rule{1.167pt}{0.108pt}}
\multiput(734.00,540.17)(5.579,3.000){2}{\rule{0.583pt}{0.400pt}}
\multiput(742.00,544.61)(1.579,0.447){3}{\rule{1.167pt}{0.108pt}}
\multiput(742.00,543.17)(5.579,3.000){2}{\rule{0.583pt}{0.400pt}}
\put(750,547.17){\rule{1.500pt}{0.400pt}}
\multiput(750.00,546.17)(3.887,2.000){2}{\rule{0.750pt}{0.400pt}}
\put(757,549.17){\rule{1.700pt}{0.400pt}}
\multiput(757.00,548.17)(4.472,2.000){2}{\rule{0.850pt}{0.400pt}}
\put(765,551.17){\rule{1.700pt}{0.400pt}}
\multiput(765.00,550.17)(4.472,2.000){2}{\rule{0.850pt}{0.400pt}}
\put(773,553.17){\rule{1.700pt}{0.400pt}}
\multiput(773.00,552.17)(4.472,2.000){2}{\rule{0.850pt}{0.400pt}}
\put(781,554.67){\rule{1.927pt}{0.400pt}}
\multiput(781.00,554.17)(4.000,1.000){2}{\rule{0.964pt}{0.400pt}}
\put(789,556.17){\rule{1.700pt}{0.400pt}}
\multiput(789.00,555.17)(4.472,2.000){2}{\rule{0.850pt}{0.400pt}}
\put(797,557.67){\rule{1.927pt}{0.400pt}}
\multiput(797.00,557.17)(4.000,1.000){2}{\rule{0.964pt}{0.400pt}}
\put(805,558.67){\rule{1.927pt}{0.400pt}}
\multiput(805.00,558.17)(4.000,1.000){2}{\rule{0.964pt}{0.400pt}}
\put(813,559.67){\rule{1.927pt}{0.400pt}}
\multiput(813.00,559.17)(4.000,1.000){2}{\rule{0.964pt}{0.400pt}}
\put(821,560.67){\rule{1.927pt}{0.400pt}}
\multiput(821.00,560.17)(4.000,1.000){2}{\rule{0.964pt}{0.400pt}}
\put(829,561.67){\rule{1.927pt}{0.400pt}}
\multiput(829.00,561.17)(4.000,1.000){2}{\rule{0.964pt}{0.400pt}}
\put(464.0,322.0){\rule[-0.200pt]{1.927pt}{0.400pt}}
\put(845,562.67){\rule{1.927pt}{0.400pt}}
\multiput(845.00,562.17)(4.000,1.000){2}{\rule{0.964pt}{0.400pt}}
\put(837.0,563.0){\rule[-0.200pt]{1.927pt}{0.400pt}}
\put(860,563.67){\rule{1.927pt}{0.400pt}}
\multiput(860.00,563.17)(4.000,1.000){2}{\rule{0.964pt}{0.400pt}}
\put(853.0,564.0){\rule[-0.200pt]{1.686pt}{0.400pt}}
\put(884,564.67){\rule{1.927pt}{0.400pt}}
\multiput(884.00,564.17)(4.000,1.000){2}{\rule{0.964pt}{0.400pt}}
\put(868.0,565.0){\rule[-0.200pt]{3.854pt}{0.400pt}}
\put(924,565.67){\rule{1.927pt}{0.400pt}}
\multiput(924.00,565.17)(4.000,1.000){2}{\rule{0.964pt}{0.400pt}}
\put(892.0,566.0){\rule[-0.200pt]{7.709pt}{0.400pt}}
\put(932.0,567.0){\rule[-0.200pt]{116.355pt}{0.400pt}}
\put(171.0,131.0){\rule[-0.200pt]{0.400pt}{175.375pt}}
\put(171.0,131.0){\rule[-0.200pt]{305.461pt}{0.400pt}}
\put(1439.0,131.0){\rule[-0.200pt]{0.400pt}{175.375pt}}
\put(171.0,859.0){\rule[-0.200pt]{305.461pt}{0.400pt}}
\end{picture}

\caption{Namerané časové závislosti výchylky $x$ pre jednotlivé prúdy $I$ pre pohybovú podmienku}  \label{G_8}
\end{figure}


\section{Diskusia}

Meranie LHO pre stredné závažie ukazuje, že hodnoty $\omega_0$ sa pre jednotlivé spôsoby odlišujú.
Ak porovnáme veľké závažie pre statickú metódu a tlmené kmity tak vidíme, že sa výsledné hodnoty podobajú, čo však neplatí pre stredné závažie. 

Ako zdroj chýba je hlavne veľká nepresnosť fitu, pre $a$, ďalej vidíme aj postupnú zmenu pri útlme kmitov. Obe nepresnosti prisudzujem, zásahom meradla, ktoré má istý nenulový moment zotrvačnosti. 

Pri kalibrovaní táčok motorčeku sa pre napätie väčšie ako $U=11V$ používala lineárna extrapolácia, čo môže byť zdrojom ďalších chýb, keďže nevieme či otáčky motorčeku naozaj závisia lineárne na napätí.

Pri určovaní kritického útlmu pohlovho kyvadla, je nepresnosť určovania prúdu v úrovni $"10\%"$, pretože sa hlavne pri prúdoch nad $"2 A"$ nedalo presne a dlhodobou udržať stabilný prúd v obvode. 

Na presné nájdenie kritického útlmu by sme potrebovali zdroj ktorý nám dokáže dlhodobo dodávať min. prúd $I="3 A"$. A cievku ktorá by bola schopná dlhodobo takýto prúd zniesť. 

\section{Záver}
V prvej časti bola statickou metódou určená tuhosť 
\eq{
k = 27.5\pm0.7\,,
}
a vlastná uhlová frekvencia lineárne harmonického oscilátoru pre rôzne závažia (postupne malé, stredné, veľké)
\eq[m]{
\omega_0= "23.34 rad \cdot  s^{-1}"\,,\\
\omega_0= "21.09 rad \cdot  s^{-1}"\,,\\
\omega_0= "16.90 rad \cdot  s^{-1}"\,.
}
Metódou tlmených kmitov bola určená vlastná frekvencia pre stredné 
závažie ako $\omega_0="\(18.88\pm0.007\) rad s^{-1}"$ a $\omega_0="\(18.24\pm0.02\) rad \cdot  s^{-1}"$ a pre veľké 
$\omega_0="\(16.56\pm0.02\) rad s^{-1}"$ a $\omega_0="\(16.30\pm0.01\) rad \cdot  s^{-1}"$.

Pri budených tlmených kmitoch bola učená vlastná frekvenica ako $\gamma ="25.28 Hz"$.

 
Pri pohlovom kyvadle bola určená vlastná $\omega_0 = "\(3.69\pm0.01\) rad\cdot s^{-1}"$. 
A kritické tlmenie nastáva pri prúde v rozmedzí $I = "2.6-3 A"$.




\begin{thebibliography}{2}
\bibitem{C_1}
Lineární harmonický oscilátor [cit. 29.11.2016]Dostupné po prihlásení z Kurz: Fyzikální praktikum I:\url{https://praktikum.fjfi.cvut.cz/pluginfile.php/129/mod_resource/content/7/LHO-navod_160927.pdf}


\bibitem{C_2}
Pohlovo kyvadlo [cit. 29.11.2016]Dostupné po prihlásení z Kurz: Fyzikální praktikum I:\url{https://praktikum.fjfi.cvut.cz/pluginfile.php/130/mod_resource/content/10/PohlovoKyvadlo_161102.pdf}

\end{thebibliography}

\end{document}


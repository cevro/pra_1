\documentclass[a4paper,10pt]{article}
%\usepackage[IL2]{fontenc}
\usepackage[utf8x]{inputenc}
\usepackage[czech]{babel}
\usepackage{amsfonts,amsmath,amssymb,graphicx,color}
%\usepackage[total={17cm,27cm}, top=2cm, left=2cm, includefoot]{geometry}
%\usepackage{fancyhdr}
\usepackage{fkssugar}
\usepackage{hyperref}

%\usepackage{caption}
\renewcommand{\popi}[2]{$#1$[\jd{#2}]}
\renewcommand{\figurename}{Obr.}
\addto\captionsczech{\renewcommand{\figurename}{Obr.}}
\addto\captionsczech{\renewcommand{\tablename}{Tab.}}

\begin{document}
\def\mean#1{\left< #1 \right>}
\noindent
{\large Fyzikální praktikum 1.} \hfil {\large FJFI ČVUT V Praze}\\
\noindent
{\large\textbf{pracovní úkol \# 11}}
\begin{center}
{\large\textit{Sériový a vázaný rezonanční obvod }}
\end{center}
\noindent
\rule{\textwidth}{1px}
\vspace{\baselineskip}

\emph{Michal Červeňák}
\par
\vspace{\baselineskip}
\begin{minipage}[l]{0.5\textwidth}%
\textit{dátum merania:}~28.11. 2016\\%
%\vspace{\baselineskip}%
\par%
\noindent%
\textit{skupina:}~4\\%
%\vspace{\baselineskip}%
\par%
\noindent%
\textit{Klasifikace:}\dotfill\\%
\end{minipage}

\section{Pracovní úkol}

\begin{enumerate}
\item DU: Vypište diferenciální rovnice pro mechanický a elektrický harmonický os- ´
cilátor. Porovnáním členů určete, která veličiny si v obou oscilátorech odpovídají, a
pokuste se vysvětlit roli jednotlivých prvků v RLC obvodu. Nápověda na Obr. 2\cite{C_1}
\item Sestavte sáriový rezonanční obvod dle Obr. 9. Pozorujte vliv změny parametrů $R$, $L$ a $C$
na obvod. Určete frekvenci vlastních kmitů RLC obvodu pro hodnoty R = 50 Ω, L = 1 mH a
$C = "500 pF"$. Porovnejte s předpokládanou hodnotou získanou z Thompsonova vzorce (10).
\item Zobrazte v módu rozmítání proudovou rezonanční křivku na osciloskopu a slovně popište její
změny při zasouvání jádra do cívky. Na základě toho odhadněte magnetický charakter jádra.
\item Proměřte proudovou rezonanční křivku v závislosti na frekvenci. Měření proveďte dvakrát:
pro vzduchovou cívku a cívku s jádrem. Znázorněte obě rezonanční křivky do společnáho grafu a
fitováním stanovte činitele jakosti obou rezonančních obvodů. Na základě toho určete indukčnost
cívky s jádrem.
\item Určete kapacitu neznámáho kondenzátoru, o němž víte, že má kapacitu menší, než je maximální
hodnota kapacity kondenzátoru Tesla.
\item Sestavte induktivně vázaný obvod a v módu rozmítání zobrazte jeho napěťovou rezonanční
křivku. Cívky posouvejte tak, abyste dosáhli vazby nadkritická, kritická a podkritická. Nalezněte
vzdálenost, při níž dochází k vazbě kritická, a vzdálenost, při níž k vazbě již nedochází.
Nepovinná: Proměřte napěťovou rezonanční křivku pro vazbu nadkritickou a znázorněte do grafu.

\end{enumerate}



\section{Pomôcky}
Frekvenční generátor UNI-T UTG9020A, osciloskop GoldStar, 
odporová dekáda CMT R1-1000 $(1\Omega − 11 M\Omega)$, indukčná dekáda
CMT L3-250 (1 µH − 11 H), vzduchová cívka PHYWE $(1 mH, 0.4 m\Omega)$, jádro cívky, ladicí kapacitní
normál Tesla (100 − 1100 pF), kapacitní normál Ulrich (1000 pF), kondenzátor neznámé
kapacity, dva koaxiální kabely, spojovací vodiče, dvě cívky pro vázané obvody s ladicím kondenzátorem
na stavebnicových dílech PHYWE.

\section{Teória}
Rezonančnú frekvenciu $\Omega_0$ môžeme určiť pomocou Thomsonova vzorca
\eq{
\Omega_0 = \frac{1}{\sqrt{L C}} \,,\lbl{R_1}
}
kde $L$ je indukčnosť cievky a $C$ je kapacita kondenzátoru.

V našom prípade z zapojenia z úlohy 1. môžeme pre $L= "1 mH"$ a $C = "\mu F"$ vypočítať podľa vzťahu \ref{R_1} vlastnú frekvenciu  $\Omega_0= "1.41 MHz"$.

Pre RLC obvod závislosť prúdu $I$ na frekvencií $f$ môžeme opísať vzťahom
\eq{
I_{\(f\)} = \frac{I_max}{\sqrt{1 + Q^2 \(\frac{f_0}{f} - \frac{f_0}{f}\)^2}}\,, \lbl{R_2}
}
kde $f_0$ je rezonančná frekvencia a $Q$ činiteľ akosti obvodu. 

Kapacitu neznámeho kondenzátoru $C_x$ zapojenie paralelne ku kondenzátoru so známou kapacitou vypočítame
\eq{
C_x = C_1 - C_2 \,, \lbl{R_6}
}
kde $C_1$ je kapacita bez paralelného neznámeho kondenzátoru a $C_2$ je kapacita so zapojeným paralelným kondenzátorom.


Činiteľ akosti $Q$ vypočítame ako 
\eq{
Q = \frac{\alpha}{R}\sqrt{\frac{L}{C}}\,, \lbl{R_5}
}
kde $L$ je indukčnosť, $C$ je kapacita, $R$ je odpor a $\alpha$ je parameter pre koreláciu.

Pre výpočet indukčnosti cievky môžeme odvodiť zo vzťahu \ref{R_5}
\eq{
\(\frac{Q_2}{Q_1}\)^2 = \frac{C_1 L_1}{C_2 L_1} \,. \lbl{R_3}
}



\subsubsection{Spracovanie chýb merania}

Označme $\mean{t}$ aritmetický priemer nameraných hodnôt $t_i$, a $\Delta t$ hodnotu $\mean{t}-t$, pričom 
\eq{
\mean{t} = \frac{1}{n}\sum_{i=1}^n t_i \,, \lbl{SCH_1}
}  
a chybu aritmetického priemeru 
\eq{
  \sigma_0=\sqrt{\frac{\sum_{i=1}^n \(t_i - \mean{t}\)^2}{n\(n-1\)}}\,, \lbl{SCH_2}
}
pričom $n$ je počet meraní.

\section{Postup merania}
\begin{enumerate}
\item Bol zostavený obvod podľa Obr. 1 \cite{C_1} a odmeraná dĺžka jednej periódy kmitu a frekvencia kmitania.
\item Generátor frekvencií bol nastavený na mód rozmítaní a pozorované zmeny amplitúdy pri vkladaní a vyberaní jadra cievky
\item Pre rozsah 20 hodnôt frekvencie v okolí vlastnej frekvencie sa zistila amplitúda s jadrom a bez jadra cievky.
\item Obvod sa pomocou zmeny kapacity na normále uviedol do rezonancie a následne sa paralelne k normále umiestnil kondenzátor s neznámou kapacitou. A opäť bol obvod uvedený do rezonancie.
\item Obvod bol zostavený podľa Obr. 4. \cite{C_1}, nameraná hodnota vzdialeností pre maximum a úplný útlm.
\end{enumerate}

\section{Výsledky merania}
\subsection{Úloha 1.}
V tabuľke \ref{T_1} sú zaznamenané odmerané hodnoty vlastnenej frekvencie. 
Z týchto hodnôt bola vypočítaná podľa vzťahu \ref{SCH_1} priemerná hodnota
\eq[n]{
f_0 = "\(216\pm1\) kHz" \,,\\
T_0 = "\(4.60\pm0.03\) \nu s" \,,
}

\begin{table}[h]

\begin{center}
\begin{tabular}{| c | c |}
\hline
\popi{f_0}{kHz} & \popi{T_0}{\mu s} \\
\hline
$217.4$ & $4.6$ \\
$217.4$ & $4.6$ \\
$213.7$   & $4.68$ \\
\hline

\end{tabular}
\caption{Namerané hodnoty frekvencie $f_0$ a periódy $T_0$ vlastných kmitov RLC obvodu} \label{T_1}
\end{center}
\end{table}

\subsection{Úloha 3.}
Pre vzdušnú cievku s a bez jadra boli namerané hodnoty napätia $U$ v okolí vlastnej frekvencie obvodu, tie boli vynesené do grafov Obr. \ref{G_1} bez jadra a \ref{G_2} s jadrom a spoločne do Obr. \ref{G_3}.

Hodnoty získané pre cievku bez jadra $Q="\(8.5\pm0.3\)"$ a pre cievku s jadrom $Q = "\(2.5\pm0.1\)"$
Pre výpočet indukčnosti cievky sme využili vzťah \ref{R_3} pričom $C1=C2$. 
Z toho dostávame indukčnosť cievky $L_2 ="0.085 mH"$\label{MV_1}.

\subsection{Úloha 2.}
Pre cievku bez jadra sme činiteľ akosti $Q$ vypočítal podľa vzťahu \ref{R_5}, 
\eq{
Q= "28.28\alpha"\,,
}
pre cievku s jadrom sme využili hodnotu \ref{MV_1} a dosadili opäť do vzťahu \ref{R_5} a tým určili koeficient akosti
\eq{
Q= 8.32 \alpha\,.
}

\subsection{Úloha 4.}
Pri zisťovaní neznámej kapacity $C_x$ sme namerali hodnoty $C_1="406 pF"$ a $C_2="327 pF"$, následne podľa vzťahu \ref{R_6} sme dopočítali neznámu kapacitu
\eq{
C_x = "79 pF"\,.
}

\subsection{Úloha 5.}
K maximálnej amplitúde došlo vo vzdialenosti $x_1 = "\(4.4\pm0.1\) mm"$. 
Úplný útlm nastal vo vzdialenosti $x_2="\(14.3\pm0.1mm\)"$.



\begin{figure}
% GNUPLOT: LaTeX picture
\setlength{\unitlength}{0.240900pt}
\ifx\plotpoint\undefined\newsavebox{\plotpoint}\fi
\sbox{\plotpoint}{\rule[-0.200pt]{0.400pt}{0.400pt}}%
\begin{picture}(1500,900)(0,0)
\sbox{\plotpoint}{\rule[-0.200pt]{0.400pt}{0.400pt}}%
\put(151.0,131.0){\rule[-0.200pt]{4.818pt}{0.400pt}}
\put(131,131){\makebox(0,0)[r]{ 15}}
\put(1419.0,131.0){\rule[-0.200pt]{4.818pt}{0.400pt}}
\put(151.0,235.0){\rule[-0.200pt]{4.818pt}{0.400pt}}
\put(131,235){\makebox(0,0)[r]{ 20}}
\put(1419.0,235.0){\rule[-0.200pt]{4.818pt}{0.400pt}}
\put(151.0,339.0){\rule[-0.200pt]{4.818pt}{0.400pt}}
\put(131,339){\makebox(0,0)[r]{ 25}}
\put(1419.0,339.0){\rule[-0.200pt]{4.818pt}{0.400pt}}
\put(151.0,443.0){\rule[-0.200pt]{4.818pt}{0.400pt}}
\put(131,443){\makebox(0,0)[r]{ 30}}
\put(1419.0,443.0){\rule[-0.200pt]{4.818pt}{0.400pt}}
\put(151.0,547.0){\rule[-0.200pt]{4.818pt}{0.400pt}}
\put(131,547){\makebox(0,0)[r]{ 35}}
\put(1419.0,547.0){\rule[-0.200pt]{4.818pt}{0.400pt}}
\put(151.0,651.0){\rule[-0.200pt]{4.818pt}{0.400pt}}
\put(131,651){\makebox(0,0)[r]{ 40}}
\put(1419.0,651.0){\rule[-0.200pt]{4.818pt}{0.400pt}}
\put(151.0,755.0){\rule[-0.200pt]{4.818pt}{0.400pt}}
\put(131,755){\makebox(0,0)[r]{ 45}}
\put(1419.0,755.0){\rule[-0.200pt]{4.818pt}{0.400pt}}
\put(151.0,859.0){\rule[-0.200pt]{4.818pt}{0.400pt}}
\put(131,859){\makebox(0,0)[r]{ 50}}
\put(1419.0,859.0){\rule[-0.200pt]{4.818pt}{0.400pt}}
\put(151.0,131.0){\rule[-0.200pt]{0.400pt}{4.818pt}}
\put(151,90){\makebox(0,0){ 195}}
\put(151.0,839.0){\rule[-0.200pt]{0.400pt}{4.818pt}}
\put(294.0,131.0){\rule[-0.200pt]{0.400pt}{4.818pt}}
\put(294,90){\makebox(0,0){ 200}}
\put(294.0,839.0){\rule[-0.200pt]{0.400pt}{4.818pt}}
\put(437.0,131.0){\rule[-0.200pt]{0.400pt}{4.818pt}}
\put(437,90){\makebox(0,0){ 205}}
\put(437.0,839.0){\rule[-0.200pt]{0.400pt}{4.818pt}}
\put(580.0,131.0){\rule[-0.200pt]{0.400pt}{4.818pt}}
\put(580,90){\makebox(0,0){ 210}}
\put(580.0,839.0){\rule[-0.200pt]{0.400pt}{4.818pt}}
\put(723.0,131.0){\rule[-0.200pt]{0.400pt}{4.818pt}}
\put(723,90){\makebox(0,0){ 215}}
\put(723.0,839.0){\rule[-0.200pt]{0.400pt}{4.818pt}}
\put(867.0,131.0){\rule[-0.200pt]{0.400pt}{4.818pt}}
\put(867,90){\makebox(0,0){ 220}}
\put(867.0,839.0){\rule[-0.200pt]{0.400pt}{4.818pt}}
\put(1010.0,131.0){\rule[-0.200pt]{0.400pt}{4.818pt}}
\put(1010,90){\makebox(0,0){ 225}}
\put(1010.0,839.0){\rule[-0.200pt]{0.400pt}{4.818pt}}
\put(1153.0,131.0){\rule[-0.200pt]{0.400pt}{4.818pt}}
\put(1153,90){\makebox(0,0){ 230}}
\put(1153.0,839.0){\rule[-0.200pt]{0.400pt}{4.818pt}}
\put(1296.0,131.0){\rule[-0.200pt]{0.400pt}{4.818pt}}
\put(1296,90){\makebox(0,0){ 235}}
\put(1296.0,839.0){\rule[-0.200pt]{0.400pt}{4.818pt}}
\put(1439.0,131.0){\rule[-0.200pt]{0.400pt}{4.818pt}}
\put(1439,90){\makebox(0,0){ 240}}
\put(1439.0,839.0){\rule[-0.200pt]{0.400pt}{4.818pt}}
\put(151.0,131.0){\rule[-0.200pt]{0.400pt}{175.375pt}}
\put(151.0,131.0){\rule[-0.200pt]{310.279pt}{0.400pt}}
\put(1439.0,131.0){\rule[-0.200pt]{0.400pt}{175.375pt}}
\put(151.0,859.0){\rule[-0.200pt]{310.279pt}{0.400pt}}
\put(30,495){\makebox(0,0){\popi{I}{mA}}}
\put(795,29){\makebox(0,0){\popi{f}{kHz}}}
\put(1279,819){\makebox(0,0)[r]{fit $I = I(f)$}}
\put(1299.0,819.0){\rule[-0.200pt]{24.090pt}{0.400pt}}
\put(208,201){\usebox{\plotpoint}}
\multiput(208.00,201.59)(1.033,0.482){9}{\rule{0.900pt}{0.116pt}}
\multiput(208.00,200.17)(10.132,6.000){2}{\rule{0.450pt}{0.400pt}}
\multiput(220.00,207.59)(0.943,0.482){9}{\rule{0.833pt}{0.116pt}}
\multiput(220.00,206.17)(9.270,6.000){2}{\rule{0.417pt}{0.400pt}}
\multiput(231.00,213.59)(1.033,0.482){9}{\rule{0.900pt}{0.116pt}}
\multiput(231.00,212.17)(10.132,6.000){2}{\rule{0.450pt}{0.400pt}}
\multiput(243.00,219.59)(1.033,0.482){9}{\rule{0.900pt}{0.116pt}}
\multiput(243.00,218.17)(10.132,6.000){2}{\rule{0.450pt}{0.400pt}}
\multiput(255.00,225.59)(0.798,0.485){11}{\rule{0.729pt}{0.117pt}}
\multiput(255.00,224.17)(9.488,7.000){2}{\rule{0.364pt}{0.400pt}}
\multiput(266.00,232.59)(1.033,0.482){9}{\rule{0.900pt}{0.116pt}}
\multiput(266.00,231.17)(10.132,6.000){2}{\rule{0.450pt}{0.400pt}}
\multiput(278.00,238.59)(0.798,0.485){11}{\rule{0.729pt}{0.117pt}}
\multiput(278.00,237.17)(9.488,7.000){2}{\rule{0.364pt}{0.400pt}}
\multiput(289.00,245.59)(0.874,0.485){11}{\rule{0.786pt}{0.117pt}}
\multiput(289.00,244.17)(10.369,7.000){2}{\rule{0.393pt}{0.400pt}}
\multiput(301.00,252.59)(0.798,0.485){11}{\rule{0.729pt}{0.117pt}}
\multiput(301.00,251.17)(9.488,7.000){2}{\rule{0.364pt}{0.400pt}}
\multiput(312.00,259.59)(0.874,0.485){11}{\rule{0.786pt}{0.117pt}}
\multiput(312.00,258.17)(10.369,7.000){2}{\rule{0.393pt}{0.400pt}}
\multiput(324.00,266.59)(0.798,0.485){11}{\rule{0.729pt}{0.117pt}}
\multiput(324.00,265.17)(9.488,7.000){2}{\rule{0.364pt}{0.400pt}}
\multiput(335.00,273.59)(0.758,0.488){13}{\rule{0.700pt}{0.117pt}}
\multiput(335.00,272.17)(10.547,8.000){2}{\rule{0.350pt}{0.400pt}}
\multiput(347.00,281.59)(0.758,0.488){13}{\rule{0.700pt}{0.117pt}}
\multiput(347.00,280.17)(10.547,8.000){2}{\rule{0.350pt}{0.400pt}}
\multiput(359.00,289.59)(0.798,0.485){11}{\rule{0.729pt}{0.117pt}}
\multiput(359.00,288.17)(9.488,7.000){2}{\rule{0.364pt}{0.400pt}}
\multiput(370.00,296.59)(0.758,0.488){13}{\rule{0.700pt}{0.117pt}}
\multiput(370.00,295.17)(10.547,8.000){2}{\rule{0.350pt}{0.400pt}}
\multiput(382.00,304.59)(0.611,0.489){15}{\rule{0.589pt}{0.118pt}}
\multiput(382.00,303.17)(9.778,9.000){2}{\rule{0.294pt}{0.400pt}}
\multiput(393.00,313.59)(0.758,0.488){13}{\rule{0.700pt}{0.117pt}}
\multiput(393.00,312.17)(10.547,8.000){2}{\rule{0.350pt}{0.400pt}}
\multiput(405.00,321.59)(0.692,0.488){13}{\rule{0.650pt}{0.117pt}}
\multiput(405.00,320.17)(9.651,8.000){2}{\rule{0.325pt}{0.400pt}}
\multiput(416.00,329.59)(0.669,0.489){15}{\rule{0.633pt}{0.118pt}}
\multiput(416.00,328.17)(10.685,9.000){2}{\rule{0.317pt}{0.400pt}}
\multiput(428.00,338.59)(0.669,0.489){15}{\rule{0.633pt}{0.118pt}}
\multiput(428.00,337.17)(10.685,9.000){2}{\rule{0.317pt}{0.400pt}}
\multiput(440.00,347.59)(0.611,0.489){15}{\rule{0.589pt}{0.118pt}}
\multiput(440.00,346.17)(9.778,9.000){2}{\rule{0.294pt}{0.400pt}}
\multiput(451.00,356.59)(0.669,0.489){15}{\rule{0.633pt}{0.118pt}}
\multiput(451.00,355.17)(10.685,9.000){2}{\rule{0.317pt}{0.400pt}}
\multiput(463.00,365.59)(0.611,0.489){15}{\rule{0.589pt}{0.118pt}}
\multiput(463.00,364.17)(9.778,9.000){2}{\rule{0.294pt}{0.400pt}}
\multiput(474.00,374.59)(0.669,0.489){15}{\rule{0.633pt}{0.118pt}}
\multiput(474.00,373.17)(10.685,9.000){2}{\rule{0.317pt}{0.400pt}}
\multiput(486.00,383.58)(0.547,0.491){17}{\rule{0.540pt}{0.118pt}}
\multiput(486.00,382.17)(9.879,10.000){2}{\rule{0.270pt}{0.400pt}}
\multiput(497.00,393.59)(0.669,0.489){15}{\rule{0.633pt}{0.118pt}}
\multiput(497.00,392.17)(10.685,9.000){2}{\rule{0.317pt}{0.400pt}}
\multiput(509.00,402.58)(0.547,0.491){17}{\rule{0.540pt}{0.118pt}}
\multiput(509.00,401.17)(9.879,10.000){2}{\rule{0.270pt}{0.400pt}}
\multiput(520.00,412.59)(0.669,0.489){15}{\rule{0.633pt}{0.118pt}}
\multiput(520.00,411.17)(10.685,9.000){2}{\rule{0.317pt}{0.400pt}}
\multiput(532.00,421.58)(0.600,0.491){17}{\rule{0.580pt}{0.118pt}}
\multiput(532.00,420.17)(10.796,10.000){2}{\rule{0.290pt}{0.400pt}}
\multiput(544.00,431.58)(0.547,0.491){17}{\rule{0.540pt}{0.118pt}}
\multiput(544.00,430.17)(9.879,10.000){2}{\rule{0.270pt}{0.400pt}}
\multiput(555.00,441.58)(0.600,0.491){17}{\rule{0.580pt}{0.118pt}}
\multiput(555.00,440.17)(10.796,10.000){2}{\rule{0.290pt}{0.400pt}}
\multiput(567.00,451.59)(0.611,0.489){15}{\rule{0.589pt}{0.118pt}}
\multiput(567.00,450.17)(9.778,9.000){2}{\rule{0.294pt}{0.400pt}}
\multiput(578.00,460.58)(0.600,0.491){17}{\rule{0.580pt}{0.118pt}}
\multiput(578.00,459.17)(10.796,10.000){2}{\rule{0.290pt}{0.400pt}}
\multiput(590.00,470.59)(0.611,0.489){15}{\rule{0.589pt}{0.118pt}}
\multiput(590.00,469.17)(9.778,9.000){2}{\rule{0.294pt}{0.400pt}}
\multiput(601.00,479.58)(0.600,0.491){17}{\rule{0.580pt}{0.118pt}}
\multiput(601.00,478.17)(10.796,10.000){2}{\rule{0.290pt}{0.400pt}}
\multiput(613.00,489.59)(0.669,0.489){15}{\rule{0.633pt}{0.118pt}}
\multiput(613.00,488.17)(10.685,9.000){2}{\rule{0.317pt}{0.400pt}}
\multiput(625.00,498.59)(0.611,0.489){15}{\rule{0.589pt}{0.118pt}}
\multiput(625.00,497.17)(9.778,9.000){2}{\rule{0.294pt}{0.400pt}}
\multiput(636.00,507.59)(0.758,0.488){13}{\rule{0.700pt}{0.117pt}}
\multiput(636.00,506.17)(10.547,8.000){2}{\rule{0.350pt}{0.400pt}}
\multiput(648.00,515.59)(0.611,0.489){15}{\rule{0.589pt}{0.118pt}}
\multiput(648.00,514.17)(9.778,9.000){2}{\rule{0.294pt}{0.400pt}}
\multiput(659.00,524.59)(0.758,0.488){13}{\rule{0.700pt}{0.117pt}}
\multiput(659.00,523.17)(10.547,8.000){2}{\rule{0.350pt}{0.400pt}}
\multiput(671.00,532.59)(0.798,0.485){11}{\rule{0.729pt}{0.117pt}}
\multiput(671.00,531.17)(9.488,7.000){2}{\rule{0.364pt}{0.400pt}}
\multiput(682.00,539.59)(0.874,0.485){11}{\rule{0.786pt}{0.117pt}}
\multiput(682.00,538.17)(10.369,7.000){2}{\rule{0.393pt}{0.400pt}}
\multiput(694.00,546.59)(0.874,0.485){11}{\rule{0.786pt}{0.117pt}}
\multiput(694.00,545.17)(10.369,7.000){2}{\rule{0.393pt}{0.400pt}}
\multiput(706.00,553.59)(0.943,0.482){9}{\rule{0.833pt}{0.116pt}}
\multiput(706.00,552.17)(9.270,6.000){2}{\rule{0.417pt}{0.400pt}}
\multiput(717.00,559.59)(1.033,0.482){9}{\rule{0.900pt}{0.116pt}}
\multiput(717.00,558.17)(10.132,6.000){2}{\rule{0.450pt}{0.400pt}}
\multiput(729.00,565.59)(1.155,0.477){7}{\rule{0.980pt}{0.115pt}}
\multiput(729.00,564.17)(8.966,5.000){2}{\rule{0.490pt}{0.400pt}}
\multiput(740.00,570.60)(1.651,0.468){5}{\rule{1.300pt}{0.113pt}}
\multiput(740.00,569.17)(9.302,4.000){2}{\rule{0.650pt}{0.400pt}}
\multiput(752.00,574.61)(2.248,0.447){3}{\rule{1.567pt}{0.108pt}}
\multiput(752.00,573.17)(7.748,3.000){2}{\rule{0.783pt}{0.400pt}}
\multiput(763.00,577.61)(2.472,0.447){3}{\rule{1.700pt}{0.108pt}}
\multiput(763.00,576.17)(8.472,3.000){2}{\rule{0.850pt}{0.400pt}}
\put(775,580.17){\rule{2.300pt}{0.400pt}}
\multiput(775.00,579.17)(6.226,2.000){2}{\rule{1.150pt}{0.400pt}}
\put(786,582.17){\rule{2.500pt}{0.400pt}}
\multiput(786.00,581.17)(6.811,2.000){2}{\rule{1.250pt}{0.400pt}}
\put(821,582.67){\rule{2.891pt}{0.400pt}}
\multiput(821.00,583.17)(6.000,-1.000){2}{\rule{1.445pt}{0.400pt}}
\put(833,581.17){\rule{2.300pt}{0.400pt}}
\multiput(833.00,582.17)(6.226,-2.000){2}{\rule{1.150pt}{0.400pt}}
\put(844,579.17){\rule{2.500pt}{0.400pt}}
\multiput(844.00,580.17)(6.811,-2.000){2}{\rule{1.250pt}{0.400pt}}
\multiput(856.00,577.95)(2.248,-0.447){3}{\rule{1.567pt}{0.108pt}}
\multiput(856.00,578.17)(7.748,-3.000){2}{\rule{0.783pt}{0.400pt}}
\multiput(867.00,574.94)(1.651,-0.468){5}{\rule{1.300pt}{0.113pt}}
\multiput(867.00,575.17)(9.302,-4.000){2}{\rule{0.650pt}{0.400pt}}
\multiput(879.00,570.94)(1.651,-0.468){5}{\rule{1.300pt}{0.113pt}}
\multiput(879.00,571.17)(9.302,-4.000){2}{\rule{0.650pt}{0.400pt}}
\multiput(891.00,566.93)(1.155,-0.477){7}{\rule{0.980pt}{0.115pt}}
\multiput(891.00,567.17)(8.966,-5.000){2}{\rule{0.490pt}{0.400pt}}
\multiput(902.00,561.93)(1.033,-0.482){9}{\rule{0.900pt}{0.116pt}}
\multiput(902.00,562.17)(10.132,-6.000){2}{\rule{0.450pt}{0.400pt}}
\multiput(914.00,555.93)(0.943,-0.482){9}{\rule{0.833pt}{0.116pt}}
\multiput(914.00,556.17)(9.270,-6.000){2}{\rule{0.417pt}{0.400pt}}
\multiput(925.00,549.93)(1.033,-0.482){9}{\rule{0.900pt}{0.116pt}}
\multiput(925.00,550.17)(10.132,-6.000){2}{\rule{0.450pt}{0.400pt}}
\multiput(937.00,543.93)(0.798,-0.485){11}{\rule{0.729pt}{0.117pt}}
\multiput(937.00,544.17)(9.488,-7.000){2}{\rule{0.364pt}{0.400pt}}
\multiput(948.00,536.93)(0.758,-0.488){13}{\rule{0.700pt}{0.117pt}}
\multiput(948.00,537.17)(10.547,-8.000){2}{\rule{0.350pt}{0.400pt}}
\multiput(960.00,528.93)(0.874,-0.485){11}{\rule{0.786pt}{0.117pt}}
\multiput(960.00,529.17)(10.369,-7.000){2}{\rule{0.393pt}{0.400pt}}
\multiput(972.00,521.93)(0.692,-0.488){13}{\rule{0.650pt}{0.117pt}}
\multiput(972.00,522.17)(9.651,-8.000){2}{\rule{0.325pt}{0.400pt}}
\multiput(983.00,513.93)(0.669,-0.489){15}{\rule{0.633pt}{0.118pt}}
\multiput(983.00,514.17)(10.685,-9.000){2}{\rule{0.317pt}{0.400pt}}
\multiput(995.00,504.93)(0.692,-0.488){13}{\rule{0.650pt}{0.117pt}}
\multiput(995.00,505.17)(9.651,-8.000){2}{\rule{0.325pt}{0.400pt}}
\multiput(1006.00,496.93)(0.669,-0.489){15}{\rule{0.633pt}{0.118pt}}
\multiput(1006.00,497.17)(10.685,-9.000){2}{\rule{0.317pt}{0.400pt}}
\multiput(1018.00,487.93)(0.692,-0.488){13}{\rule{0.650pt}{0.117pt}}
\multiput(1018.00,488.17)(9.651,-8.000){2}{\rule{0.325pt}{0.400pt}}
\multiput(1029.00,479.93)(0.669,-0.489){15}{\rule{0.633pt}{0.118pt}}
\multiput(1029.00,480.17)(10.685,-9.000){2}{\rule{0.317pt}{0.400pt}}
\multiput(1041.00,470.93)(0.611,-0.489){15}{\rule{0.589pt}{0.118pt}}
\multiput(1041.00,471.17)(9.778,-9.000){2}{\rule{0.294pt}{0.400pt}}
\multiput(1052.00,461.93)(0.669,-0.489){15}{\rule{0.633pt}{0.118pt}}
\multiput(1052.00,462.17)(10.685,-9.000){2}{\rule{0.317pt}{0.400pt}}
\multiput(1064.00,452.93)(0.669,-0.489){15}{\rule{0.633pt}{0.118pt}}
\multiput(1064.00,453.17)(10.685,-9.000){2}{\rule{0.317pt}{0.400pt}}
\multiput(1076.00,443.93)(0.611,-0.489){15}{\rule{0.589pt}{0.118pt}}
\multiput(1076.00,444.17)(9.778,-9.000){2}{\rule{0.294pt}{0.400pt}}
\multiput(1087.00,434.93)(0.669,-0.489){15}{\rule{0.633pt}{0.118pt}}
\multiput(1087.00,435.17)(10.685,-9.000){2}{\rule{0.317pt}{0.400pt}}
\multiput(1099.00,425.93)(0.692,-0.488){13}{\rule{0.650pt}{0.117pt}}
\multiput(1099.00,426.17)(9.651,-8.000){2}{\rule{0.325pt}{0.400pt}}
\multiput(1110.00,417.93)(0.669,-0.489){15}{\rule{0.633pt}{0.118pt}}
\multiput(1110.00,418.17)(10.685,-9.000){2}{\rule{0.317pt}{0.400pt}}
\multiput(1122.00,408.93)(0.611,-0.489){15}{\rule{0.589pt}{0.118pt}}
\multiput(1122.00,409.17)(9.778,-9.000){2}{\rule{0.294pt}{0.400pt}}
\multiput(1133.00,399.93)(0.669,-0.489){15}{\rule{0.633pt}{0.118pt}}
\multiput(1133.00,400.17)(10.685,-9.000){2}{\rule{0.317pt}{0.400pt}}
\multiput(1145.00,390.93)(0.758,-0.488){13}{\rule{0.700pt}{0.117pt}}
\multiput(1145.00,391.17)(10.547,-8.000){2}{\rule{0.350pt}{0.400pt}}
\multiput(1157.00,382.93)(0.692,-0.488){13}{\rule{0.650pt}{0.117pt}}
\multiput(1157.00,383.17)(9.651,-8.000){2}{\rule{0.325pt}{0.400pt}}
\multiput(1168.00,374.93)(0.669,-0.489){15}{\rule{0.633pt}{0.118pt}}
\multiput(1168.00,375.17)(10.685,-9.000){2}{\rule{0.317pt}{0.400pt}}
\multiput(1180.00,365.93)(0.692,-0.488){13}{\rule{0.650pt}{0.117pt}}
\multiput(1180.00,366.17)(9.651,-8.000){2}{\rule{0.325pt}{0.400pt}}
\multiput(1191.00,357.93)(0.758,-0.488){13}{\rule{0.700pt}{0.117pt}}
\multiput(1191.00,358.17)(10.547,-8.000){2}{\rule{0.350pt}{0.400pt}}
\multiput(1203.00,349.93)(0.692,-0.488){13}{\rule{0.650pt}{0.117pt}}
\multiput(1203.00,350.17)(9.651,-8.000){2}{\rule{0.325pt}{0.400pt}}
\multiput(1214.00,341.93)(0.874,-0.485){11}{\rule{0.786pt}{0.117pt}}
\multiput(1214.00,342.17)(10.369,-7.000){2}{\rule{0.393pt}{0.400pt}}
\multiput(1226.00,334.93)(0.692,-0.488){13}{\rule{0.650pt}{0.117pt}}
\multiput(1226.00,335.17)(9.651,-8.000){2}{\rule{0.325pt}{0.400pt}}
\multiput(1237.00,326.93)(0.874,-0.485){11}{\rule{0.786pt}{0.117pt}}
\multiput(1237.00,327.17)(10.369,-7.000){2}{\rule{0.393pt}{0.400pt}}
\multiput(1249.00,319.93)(0.874,-0.485){11}{\rule{0.786pt}{0.117pt}}
\multiput(1249.00,320.17)(10.369,-7.000){2}{\rule{0.393pt}{0.400pt}}
\multiput(1261.00,312.93)(0.692,-0.488){13}{\rule{0.650pt}{0.117pt}}
\multiput(1261.00,313.17)(9.651,-8.000){2}{\rule{0.325pt}{0.400pt}}
\multiput(1272.00,304.93)(0.874,-0.485){11}{\rule{0.786pt}{0.117pt}}
\multiput(1272.00,305.17)(10.369,-7.000){2}{\rule{0.393pt}{0.400pt}}
\multiput(1284.00,297.93)(0.943,-0.482){9}{\rule{0.833pt}{0.116pt}}
\multiput(1284.00,298.17)(9.270,-6.000){2}{\rule{0.417pt}{0.400pt}}
\multiput(1295.00,291.93)(0.874,-0.485){11}{\rule{0.786pt}{0.117pt}}
\multiput(1295.00,292.17)(10.369,-7.000){2}{\rule{0.393pt}{0.400pt}}
\multiput(1307.00,284.93)(0.798,-0.485){11}{\rule{0.729pt}{0.117pt}}
\multiput(1307.00,285.17)(9.488,-7.000){2}{\rule{0.364pt}{0.400pt}}
\multiput(1318.00,277.93)(1.033,-0.482){9}{\rule{0.900pt}{0.116pt}}
\multiput(1318.00,278.17)(10.132,-6.000){2}{\rule{0.450pt}{0.400pt}}
\multiput(1330.00,271.93)(1.033,-0.482){9}{\rule{0.900pt}{0.116pt}}
\multiput(1330.00,272.17)(10.132,-6.000){2}{\rule{0.450pt}{0.400pt}}
\multiput(1342.00,265.93)(0.943,-0.482){9}{\rule{0.833pt}{0.116pt}}
\multiput(1342.00,266.17)(9.270,-6.000){2}{\rule{0.417pt}{0.400pt}}
\put(798.0,584.0){\rule[-0.200pt]{5.541pt}{0.400pt}}
\put(1279,778){\makebox(0,0)[r]{Namerané hodnoty pre cievku bez jadra}}
\put(781,609){\makebox(0,0){$\times$}}
\put(1067,485){\makebox(0,0){$\times$}}
\put(1353,235){\makebox(0,0){$\times$}}
\put(1210,339){\makebox(0,0){$\times$}}
\put(924,547){\makebox(0,0){$\times$}}
\put(638,485){\makebox(0,0){$\times$}}
\put(494,401){\makebox(0,0){$\times$}}
\put(351,297){\makebox(0,0){$\times$}}
\put(208,214){\makebox(0,0){$\times$}}
\put(294,256){\makebox(0,0){$\times$}}
\put(409,318){\makebox(0,0){$\times$}}
\put(580,464){\makebox(0,0){$\times$}}
\put(666,485){\makebox(0,0){$\times$}}
\put(723,568){\makebox(0,0){$\times$}}
\put(752,589){\makebox(0,0){$\times$}}
\put(809,589){\makebox(0,0){$\times$}}
\put(867,568){\makebox(0,0){$\times$}}
\put(1010,485){\makebox(0,0){$\times$}}
\put(1153,381){\makebox(0,0){$\times$}}
\put(1296,297){\makebox(0,0){$\times$}}
\put(1349,778){\makebox(0,0){$\times$}}
\put(151.0,131.0){\rule[-0.200pt]{0.400pt}{175.375pt}}
\put(151.0,131.0){\rule[-0.200pt]{310.279pt}{0.400pt}}
\put(1439.0,131.0){\rule[-0.200pt]{0.400pt}{175.375pt}}
\put(151.0,859.0){\rule[-0.200pt]{310.279pt}{0.400pt}}
\end{picture}

\caption{Závislosť veľkosti prúdu $I$ na rezonančnej frekvencií $f$ pre cievku bez jadra, preložená funkciou $I = \frac{36.8\pm0.4}{\sqrt{1+\(8.5\pm0.3\)\cdot\(\frac{f}{218.1\pm0.2}-\frac{218.1\pm0.2}{f}\)^2}}$.}  \label{G_1}
\end{figure}

\begin{figure}
% GNUPLOT: LaTeX picture
\setlength{\unitlength}{0.240900pt}
\ifx\plotpoint\undefined\newsavebox{\plotpoint}\fi
\begin{picture}(1500,900)(0,0)
\sbox{\plotpoint}{\rule[-0.200pt]{0.400pt}{0.400pt}}%
\put(151.0,131.0){\rule[-0.200pt]{4.818pt}{0.400pt}}
\put(131,131){\makebox(0,0)[r]{ 5}}
\put(1419.0,131.0){\rule[-0.200pt]{4.818pt}{0.400pt}}
\put(151.0,204.0){\rule[-0.200pt]{4.818pt}{0.400pt}}
\put(131,204){\makebox(0,0)[r]{ 6}}
\put(1419.0,204.0){\rule[-0.200pt]{4.818pt}{0.400pt}}
\put(151.0,277.0){\rule[-0.200pt]{4.818pt}{0.400pt}}
\put(131,277){\makebox(0,0)[r]{ 7}}
\put(1419.0,277.0){\rule[-0.200pt]{4.818pt}{0.400pt}}
\put(151.0,349.0){\rule[-0.200pt]{4.818pt}{0.400pt}}
\put(131,349){\makebox(0,0)[r]{ 8}}
\put(1419.0,349.0){\rule[-0.200pt]{4.818pt}{0.400pt}}
\put(151.0,422.0){\rule[-0.200pt]{4.818pt}{0.400pt}}
\put(131,422){\makebox(0,0)[r]{ 9}}
\put(1419.0,422.0){\rule[-0.200pt]{4.818pt}{0.400pt}}
\put(151.0,495.0){\rule[-0.200pt]{4.818pt}{0.400pt}}
\put(131,495){\makebox(0,0)[r]{ 10}}
\put(1419.0,495.0){\rule[-0.200pt]{4.818pt}{0.400pt}}
\put(151.0,568.0){\rule[-0.200pt]{4.818pt}{0.400pt}}
\put(131,568){\makebox(0,0)[r]{ 11}}
\put(1419.0,568.0){\rule[-0.200pt]{4.818pt}{0.400pt}}
\put(151.0,641.0){\rule[-0.200pt]{4.818pt}{0.400pt}}
\put(131,641){\makebox(0,0)[r]{ 12}}
\put(1419.0,641.0){\rule[-0.200pt]{4.818pt}{0.400pt}}
\put(151.0,713.0){\rule[-0.200pt]{4.818pt}{0.400pt}}
\put(131,713){\makebox(0,0)[r]{ 13}}
\put(1419.0,713.0){\rule[-0.200pt]{4.818pt}{0.400pt}}
\put(151.0,786.0){\rule[-0.200pt]{4.818pt}{0.400pt}}
\put(131,786){\makebox(0,0)[r]{ 14}}
\put(1419.0,786.0){\rule[-0.200pt]{4.818pt}{0.400pt}}
\put(151.0,859.0){\rule[-0.200pt]{4.818pt}{0.400pt}}
\put(131,859){\makebox(0,0)[r]{ 15}}
\put(1419.0,859.0){\rule[-0.200pt]{4.818pt}{0.400pt}}
\put(151.0,131.0){\rule[-0.200pt]{0.400pt}{4.818pt}}
\put(151,90){\makebox(0,0){ 140}}
\put(151.0,839.0){\rule[-0.200pt]{0.400pt}{4.818pt}}
\put(335.0,131.0){\rule[-0.200pt]{0.400pt}{4.818pt}}
\put(335,90){\makebox(0,0){ 160}}
\put(335.0,839.0){\rule[-0.200pt]{0.400pt}{4.818pt}}
\put(519.0,131.0){\rule[-0.200pt]{0.400pt}{4.818pt}}
\put(519,90){\makebox(0,0){ 180}}
\put(519.0,839.0){\rule[-0.200pt]{0.400pt}{4.818pt}}
\put(703.0,131.0){\rule[-0.200pt]{0.400pt}{4.818pt}}
\put(703,90){\makebox(0,0){ 200}}
\put(703.0,839.0){\rule[-0.200pt]{0.400pt}{4.818pt}}
\put(887.0,131.0){\rule[-0.200pt]{0.400pt}{4.818pt}}
\put(887,90){\makebox(0,0){ 220}}
\put(887.0,839.0){\rule[-0.200pt]{0.400pt}{4.818pt}}
\put(1071.0,131.0){\rule[-0.200pt]{0.400pt}{4.818pt}}
\put(1071,90){\makebox(0,0){ 240}}
\put(1071.0,839.0){\rule[-0.200pt]{0.400pt}{4.818pt}}
\put(1255.0,131.0){\rule[-0.200pt]{0.400pt}{4.818pt}}
\put(1255,90){\makebox(0,0){ 260}}
\put(1255.0,839.0){\rule[-0.200pt]{0.400pt}{4.818pt}}
\put(1439.0,131.0){\rule[-0.200pt]{0.400pt}{4.818pt}}
\put(1439,90){\makebox(0,0){ 280}}
\put(1439.0,839.0){\rule[-0.200pt]{0.400pt}{4.818pt}}
\put(151.0,131.0){\rule[-0.200pt]{0.400pt}{175.375pt}}
\put(151.0,131.0){\rule[-0.200pt]{310.279pt}{0.400pt}}
\put(1439.0,131.0){\rule[-0.200pt]{0.400pt}{175.375pt}}
\put(151.0,859.0){\rule[-0.200pt]{310.279pt}{0.400pt}}
\put(30,495){\makebox(0,0){\popi{I}{mA}}}
\put(795,29){\makebox(0,0){\popi{f}{kHz}}}
\put(1279,819){\makebox(0,0)[r]{fit $I = I(f)$}}
\put(1299.0,819.0){\rule[-0.200pt]{24.090pt}{0.400pt}}
\put(289,209){\usebox{\plotpoint}}
\multiput(289.00,209.59)(0.626,0.488){13}{\rule{0.600pt}{0.117pt}}
\multiput(289.00,208.17)(8.755,8.000){2}{\rule{0.300pt}{0.400pt}}
\multiput(299.00,217.59)(0.721,0.485){11}{\rule{0.671pt}{0.117pt}}
\multiput(299.00,216.17)(8.606,7.000){2}{\rule{0.336pt}{0.400pt}}
\multiput(309.00,224.59)(0.692,0.488){13}{\rule{0.650pt}{0.117pt}}
\multiput(309.00,223.17)(9.651,8.000){2}{\rule{0.325pt}{0.400pt}}
\multiput(320.00,232.59)(0.626,0.488){13}{\rule{0.600pt}{0.117pt}}
\multiput(320.00,231.17)(8.755,8.000){2}{\rule{0.300pt}{0.400pt}}
\multiput(330.00,240.59)(0.626,0.488){13}{\rule{0.600pt}{0.117pt}}
\multiput(330.00,239.17)(8.755,8.000){2}{\rule{0.300pt}{0.400pt}}
\multiput(340.00,248.59)(0.553,0.489){15}{\rule{0.544pt}{0.118pt}}
\multiput(340.00,247.17)(8.870,9.000){2}{\rule{0.272pt}{0.400pt}}
\multiput(350.00,257.59)(0.692,0.488){13}{\rule{0.650pt}{0.117pt}}
\multiput(350.00,256.17)(9.651,8.000){2}{\rule{0.325pt}{0.400pt}}
\multiput(361.00,265.59)(0.553,0.489){15}{\rule{0.544pt}{0.118pt}}
\multiput(361.00,264.17)(8.870,9.000){2}{\rule{0.272pt}{0.400pt}}
\multiput(371.00,274.59)(0.626,0.488){13}{\rule{0.600pt}{0.117pt}}
\multiput(371.00,273.17)(8.755,8.000){2}{\rule{0.300pt}{0.400pt}}
\multiput(381.00,282.59)(0.553,0.489){15}{\rule{0.544pt}{0.118pt}}
\multiput(381.00,281.17)(8.870,9.000){2}{\rule{0.272pt}{0.400pt}}
\multiput(391.00,291.59)(0.553,0.489){15}{\rule{0.544pt}{0.118pt}}
\multiput(391.00,290.17)(8.870,9.000){2}{\rule{0.272pt}{0.400pt}}
\multiput(401.00,300.59)(0.611,0.489){15}{\rule{0.589pt}{0.118pt}}
\multiput(401.00,299.17)(9.778,9.000){2}{\rule{0.294pt}{0.400pt}}
\multiput(412.00,309.59)(0.553,0.489){15}{\rule{0.544pt}{0.118pt}}
\multiput(412.00,308.17)(8.870,9.000){2}{\rule{0.272pt}{0.400pt}}
\multiput(422.00,318.58)(0.495,0.491){17}{\rule{0.500pt}{0.118pt}}
\multiput(422.00,317.17)(8.962,10.000){2}{\rule{0.250pt}{0.400pt}}
\multiput(432.00,328.59)(0.553,0.489){15}{\rule{0.544pt}{0.118pt}}
\multiput(432.00,327.17)(8.870,9.000){2}{\rule{0.272pt}{0.400pt}}
\multiput(442.00,337.58)(0.547,0.491){17}{\rule{0.540pt}{0.118pt}}
\multiput(442.00,336.17)(9.879,10.000){2}{\rule{0.270pt}{0.400pt}}
\multiput(453.00,347.59)(0.553,0.489){15}{\rule{0.544pt}{0.118pt}}
\multiput(453.00,346.17)(8.870,9.000){2}{\rule{0.272pt}{0.400pt}}
\multiput(463.00,356.58)(0.495,0.491){17}{\rule{0.500pt}{0.118pt}}
\multiput(463.00,355.17)(8.962,10.000){2}{\rule{0.250pt}{0.400pt}}
\multiput(473.00,366.58)(0.495,0.491){17}{\rule{0.500pt}{0.118pt}}
\multiput(473.00,365.17)(8.962,10.000){2}{\rule{0.250pt}{0.400pt}}
\multiput(483.00,376.58)(0.495,0.491){17}{\rule{0.500pt}{0.118pt}}
\multiput(483.00,375.17)(8.962,10.000){2}{\rule{0.250pt}{0.400pt}}
\multiput(493.00,386.58)(0.547,0.491){17}{\rule{0.540pt}{0.118pt}}
\multiput(493.00,385.17)(9.879,10.000){2}{\rule{0.270pt}{0.400pt}}
\multiput(504.00,396.58)(0.495,0.491){17}{\rule{0.500pt}{0.118pt}}
\multiput(504.00,395.17)(8.962,10.000){2}{\rule{0.250pt}{0.400pt}}
\multiput(514.00,406.58)(0.495,0.491){17}{\rule{0.500pt}{0.118pt}}
\multiput(514.00,405.17)(8.962,10.000){2}{\rule{0.250pt}{0.400pt}}
\multiput(524.00,416.58)(0.495,0.491){17}{\rule{0.500pt}{0.118pt}}
\multiput(524.00,415.17)(8.962,10.000){2}{\rule{0.250pt}{0.400pt}}
\multiput(534.00,426.58)(0.547,0.491){17}{\rule{0.540pt}{0.118pt}}
\multiput(534.00,425.17)(9.879,10.000){2}{\rule{0.270pt}{0.400pt}}
\multiput(545.00,436.58)(0.495,0.491){17}{\rule{0.500pt}{0.118pt}}
\multiput(545.00,435.17)(8.962,10.000){2}{\rule{0.250pt}{0.400pt}}
\multiput(555.00,446.58)(0.495,0.491){17}{\rule{0.500pt}{0.118pt}}
\multiput(555.00,445.17)(8.962,10.000){2}{\rule{0.250pt}{0.400pt}}
\multiput(565.00,456.59)(0.553,0.489){15}{\rule{0.544pt}{0.118pt}}
\multiput(565.00,455.17)(8.870,9.000){2}{\rule{0.272pt}{0.400pt}}
\multiput(575.00,465.58)(0.495,0.491){17}{\rule{0.500pt}{0.118pt}}
\multiput(575.00,464.17)(8.962,10.000){2}{\rule{0.250pt}{0.400pt}}
\multiput(585.00,475.58)(0.547,0.491){17}{\rule{0.540pt}{0.118pt}}
\multiput(585.00,474.17)(9.879,10.000){2}{\rule{0.270pt}{0.400pt}}
\multiput(596.00,485.59)(0.553,0.489){15}{\rule{0.544pt}{0.118pt}}
\multiput(596.00,484.17)(8.870,9.000){2}{\rule{0.272pt}{0.400pt}}
\multiput(606.00,494.58)(0.495,0.491){17}{\rule{0.500pt}{0.118pt}}
\multiput(606.00,493.17)(8.962,10.000){2}{\rule{0.250pt}{0.400pt}}
\multiput(616.00,504.59)(0.553,0.489){15}{\rule{0.544pt}{0.118pt}}
\multiput(616.00,503.17)(8.870,9.000){2}{\rule{0.272pt}{0.400pt}}
\multiput(626.00,513.59)(0.692,0.488){13}{\rule{0.650pt}{0.117pt}}
\multiput(626.00,512.17)(9.651,8.000){2}{\rule{0.325pt}{0.400pt}}
\multiput(637.00,521.59)(0.553,0.489){15}{\rule{0.544pt}{0.118pt}}
\multiput(637.00,520.17)(8.870,9.000){2}{\rule{0.272pt}{0.400pt}}
\multiput(647.00,530.59)(0.626,0.488){13}{\rule{0.600pt}{0.117pt}}
\multiput(647.00,529.17)(8.755,8.000){2}{\rule{0.300pt}{0.400pt}}
\multiput(657.00,538.59)(0.626,0.488){13}{\rule{0.600pt}{0.117pt}}
\multiput(657.00,537.17)(8.755,8.000){2}{\rule{0.300pt}{0.400pt}}
\multiput(667.00,546.59)(0.626,0.488){13}{\rule{0.600pt}{0.117pt}}
\multiput(667.00,545.17)(8.755,8.000){2}{\rule{0.300pt}{0.400pt}}
\multiput(677.00,554.59)(0.798,0.485){11}{\rule{0.729pt}{0.117pt}}
\multiput(677.00,553.17)(9.488,7.000){2}{\rule{0.364pt}{0.400pt}}
\multiput(688.00,561.59)(0.852,0.482){9}{\rule{0.767pt}{0.116pt}}
\multiput(688.00,560.17)(8.409,6.000){2}{\rule{0.383pt}{0.400pt}}
\multiput(698.00,567.59)(0.721,0.485){11}{\rule{0.671pt}{0.117pt}}
\multiput(698.00,566.17)(8.606,7.000){2}{\rule{0.336pt}{0.400pt}}
\multiput(708.00,574.59)(1.044,0.477){7}{\rule{0.900pt}{0.115pt}}
\multiput(708.00,573.17)(8.132,5.000){2}{\rule{0.450pt}{0.400pt}}
\multiput(718.00,579.59)(1.155,0.477){7}{\rule{0.980pt}{0.115pt}}
\multiput(718.00,578.17)(8.966,5.000){2}{\rule{0.490pt}{0.400pt}}
\multiput(729.00,584.59)(1.044,0.477){7}{\rule{0.900pt}{0.115pt}}
\multiput(729.00,583.17)(8.132,5.000){2}{\rule{0.450pt}{0.400pt}}
\multiput(739.00,589.60)(1.358,0.468){5}{\rule{1.100pt}{0.113pt}}
\multiput(739.00,588.17)(7.717,4.000){2}{\rule{0.550pt}{0.400pt}}
\multiput(749.00,593.60)(1.358,0.468){5}{\rule{1.100pt}{0.113pt}}
\multiput(749.00,592.17)(7.717,4.000){2}{\rule{0.550pt}{0.400pt}}
\multiput(759.00,597.61)(2.025,0.447){3}{\rule{1.433pt}{0.108pt}}
\multiput(759.00,596.17)(7.025,3.000){2}{\rule{0.717pt}{0.400pt}}
\put(769,600.17){\rule{2.300pt}{0.400pt}}
\multiput(769.00,599.17)(6.226,2.000){2}{\rule{1.150pt}{0.400pt}}
\put(780,602.17){\rule{2.100pt}{0.400pt}}
\multiput(780.00,601.17)(5.641,2.000){2}{\rule{1.050pt}{0.400pt}}
\put(790,603.67){\rule{2.409pt}{0.400pt}}
\multiput(790.00,603.17)(5.000,1.000){2}{\rule{1.204pt}{0.400pt}}
\put(800,604.67){\rule{2.409pt}{0.400pt}}
\multiput(800.00,604.17)(5.000,1.000){2}{\rule{1.204pt}{0.400pt}}
\put(821,604.67){\rule{2.409pt}{0.400pt}}
\multiput(821.00,605.17)(5.000,-1.000){2}{\rule{1.204pt}{0.400pt}}
\put(831,603.67){\rule{2.409pt}{0.400pt}}
\multiput(831.00,604.17)(5.000,-1.000){2}{\rule{1.204pt}{0.400pt}}
\put(841,602.17){\rule{2.100pt}{0.400pt}}
\multiput(841.00,603.17)(5.641,-2.000){2}{\rule{1.050pt}{0.400pt}}
\put(851,600.17){\rule{2.100pt}{0.400pt}}
\multiput(851.00,601.17)(5.641,-2.000){2}{\rule{1.050pt}{0.400pt}}
\multiput(861.00,598.95)(2.248,-0.447){3}{\rule{1.567pt}{0.108pt}}
\multiput(861.00,599.17)(7.748,-3.000){2}{\rule{0.783pt}{0.400pt}}
\multiput(872.00,595.95)(2.025,-0.447){3}{\rule{1.433pt}{0.108pt}}
\multiput(872.00,596.17)(7.025,-3.000){2}{\rule{0.717pt}{0.400pt}}
\multiput(882.00,592.94)(1.358,-0.468){5}{\rule{1.100pt}{0.113pt}}
\multiput(882.00,593.17)(7.717,-4.000){2}{\rule{0.550pt}{0.400pt}}
\multiput(892.00,588.94)(1.358,-0.468){5}{\rule{1.100pt}{0.113pt}}
\multiput(892.00,589.17)(7.717,-4.000){2}{\rule{0.550pt}{0.400pt}}
\multiput(902.00,584.94)(1.505,-0.468){5}{\rule{1.200pt}{0.113pt}}
\multiput(902.00,585.17)(8.509,-4.000){2}{\rule{0.600pt}{0.400pt}}
\multiput(913.00,580.93)(1.044,-0.477){7}{\rule{0.900pt}{0.115pt}}
\multiput(913.00,581.17)(8.132,-5.000){2}{\rule{0.450pt}{0.400pt}}
\multiput(923.00,575.93)(0.852,-0.482){9}{\rule{0.767pt}{0.116pt}}
\multiput(923.00,576.17)(8.409,-6.000){2}{\rule{0.383pt}{0.400pt}}
\multiput(933.00,569.93)(1.044,-0.477){7}{\rule{0.900pt}{0.115pt}}
\multiput(933.00,570.17)(8.132,-5.000){2}{\rule{0.450pt}{0.400pt}}
\multiput(943.00,564.93)(0.852,-0.482){9}{\rule{0.767pt}{0.116pt}}
\multiput(943.00,565.17)(8.409,-6.000){2}{\rule{0.383pt}{0.400pt}}
\multiput(953.00,558.93)(0.943,-0.482){9}{\rule{0.833pt}{0.116pt}}
\multiput(953.00,559.17)(9.270,-6.000){2}{\rule{0.417pt}{0.400pt}}
\multiput(964.00,552.93)(0.721,-0.485){11}{\rule{0.671pt}{0.117pt}}
\multiput(964.00,553.17)(8.606,-7.000){2}{\rule{0.336pt}{0.400pt}}
\multiput(974.00,545.93)(0.852,-0.482){9}{\rule{0.767pt}{0.116pt}}
\multiput(974.00,546.17)(8.409,-6.000){2}{\rule{0.383pt}{0.400pt}}
\multiput(984.00,539.93)(0.721,-0.485){11}{\rule{0.671pt}{0.117pt}}
\multiput(984.00,540.17)(8.606,-7.000){2}{\rule{0.336pt}{0.400pt}}
\multiput(994.00,532.93)(0.798,-0.485){11}{\rule{0.729pt}{0.117pt}}
\multiput(994.00,533.17)(9.488,-7.000){2}{\rule{0.364pt}{0.400pt}}
\multiput(1005.00,525.93)(0.721,-0.485){11}{\rule{0.671pt}{0.117pt}}
\multiput(1005.00,526.17)(8.606,-7.000){2}{\rule{0.336pt}{0.400pt}}
\multiput(1015.00,518.93)(0.626,-0.488){13}{\rule{0.600pt}{0.117pt}}
\multiput(1015.00,519.17)(8.755,-8.000){2}{\rule{0.300pt}{0.400pt}}
\multiput(1025.00,510.93)(0.721,-0.485){11}{\rule{0.671pt}{0.117pt}}
\multiput(1025.00,511.17)(8.606,-7.000){2}{\rule{0.336pt}{0.400pt}}
\multiput(1035.00,503.93)(0.721,-0.485){11}{\rule{0.671pt}{0.117pt}}
\multiput(1035.00,504.17)(8.606,-7.000){2}{\rule{0.336pt}{0.400pt}}
\multiput(1045.00,496.93)(0.692,-0.488){13}{\rule{0.650pt}{0.117pt}}
\multiput(1045.00,497.17)(9.651,-8.000){2}{\rule{0.325pt}{0.400pt}}
\multiput(1056.00,488.93)(0.721,-0.485){11}{\rule{0.671pt}{0.117pt}}
\multiput(1056.00,489.17)(8.606,-7.000){2}{\rule{0.336pt}{0.400pt}}
\multiput(1066.00,481.93)(0.626,-0.488){13}{\rule{0.600pt}{0.117pt}}
\multiput(1066.00,482.17)(8.755,-8.000){2}{\rule{0.300pt}{0.400pt}}
\multiput(1076.00,473.93)(0.721,-0.485){11}{\rule{0.671pt}{0.117pt}}
\multiput(1076.00,474.17)(8.606,-7.000){2}{\rule{0.336pt}{0.400pt}}
\multiput(1086.00,466.93)(0.692,-0.488){13}{\rule{0.650pt}{0.117pt}}
\multiput(1086.00,467.17)(9.651,-8.000){2}{\rule{0.325pt}{0.400pt}}
\multiput(1097.00,458.93)(0.721,-0.485){11}{\rule{0.671pt}{0.117pt}}
\multiput(1097.00,459.17)(8.606,-7.000){2}{\rule{0.336pt}{0.400pt}}
\multiput(1107.00,451.93)(0.626,-0.488){13}{\rule{0.600pt}{0.117pt}}
\multiput(1107.00,452.17)(8.755,-8.000){2}{\rule{0.300pt}{0.400pt}}
\multiput(1117.00,443.93)(0.721,-0.485){11}{\rule{0.671pt}{0.117pt}}
\multiput(1117.00,444.17)(8.606,-7.000){2}{\rule{0.336pt}{0.400pt}}
\multiput(1127.00,436.93)(0.626,-0.488){13}{\rule{0.600pt}{0.117pt}}
\multiput(1127.00,437.17)(8.755,-8.000){2}{\rule{0.300pt}{0.400pt}}
\multiput(1137.00,428.93)(0.798,-0.485){11}{\rule{0.729pt}{0.117pt}}
\multiput(1137.00,429.17)(9.488,-7.000){2}{\rule{0.364pt}{0.400pt}}
\multiput(1148.00,421.93)(0.721,-0.485){11}{\rule{0.671pt}{0.117pt}}
\multiput(1148.00,422.17)(8.606,-7.000){2}{\rule{0.336pt}{0.400pt}}
\multiput(1158.00,414.93)(0.721,-0.485){11}{\rule{0.671pt}{0.117pt}}
\multiput(1158.00,415.17)(8.606,-7.000){2}{\rule{0.336pt}{0.400pt}}
\multiput(1168.00,407.93)(0.626,-0.488){13}{\rule{0.600pt}{0.117pt}}
\multiput(1168.00,408.17)(8.755,-8.000){2}{\rule{0.300pt}{0.400pt}}
\multiput(1178.00,399.93)(0.798,-0.485){11}{\rule{0.729pt}{0.117pt}}
\multiput(1178.00,400.17)(9.488,-7.000){2}{\rule{0.364pt}{0.400pt}}
\multiput(1189.00,392.93)(0.721,-0.485){11}{\rule{0.671pt}{0.117pt}}
\multiput(1189.00,393.17)(8.606,-7.000){2}{\rule{0.336pt}{0.400pt}}
\multiput(1199.00,385.93)(0.721,-0.485){11}{\rule{0.671pt}{0.117pt}}
\multiput(1199.00,386.17)(8.606,-7.000){2}{\rule{0.336pt}{0.400pt}}
\multiput(1209.00,378.93)(0.852,-0.482){9}{\rule{0.767pt}{0.116pt}}
\multiput(1209.00,379.17)(8.409,-6.000){2}{\rule{0.383pt}{0.400pt}}
\multiput(1219.00,372.93)(0.721,-0.485){11}{\rule{0.671pt}{0.117pt}}
\multiput(1219.00,373.17)(8.606,-7.000){2}{\rule{0.336pt}{0.400pt}}
\multiput(1229.00,365.93)(0.798,-0.485){11}{\rule{0.729pt}{0.117pt}}
\multiput(1229.00,366.17)(9.488,-7.000){2}{\rule{0.364pt}{0.400pt}}
\multiput(1240.00,358.93)(0.852,-0.482){9}{\rule{0.767pt}{0.116pt}}
\multiput(1240.00,359.17)(8.409,-6.000){2}{\rule{0.383pt}{0.400pt}}
\multiput(1250.00,352.93)(0.721,-0.485){11}{\rule{0.671pt}{0.117pt}}
\multiput(1250.00,353.17)(8.606,-7.000){2}{\rule{0.336pt}{0.400pt}}
\multiput(1260.00,345.93)(0.852,-0.482){9}{\rule{0.767pt}{0.116pt}}
\multiput(1260.00,346.17)(8.409,-6.000){2}{\rule{0.383pt}{0.400pt}}
\multiput(1270.00,339.93)(0.943,-0.482){9}{\rule{0.833pt}{0.116pt}}
\multiput(1270.00,340.17)(9.270,-6.000){2}{\rule{0.417pt}{0.400pt}}
\multiput(1281.00,333.93)(0.852,-0.482){9}{\rule{0.767pt}{0.116pt}}
\multiput(1281.00,334.17)(8.409,-6.000){2}{\rule{0.383pt}{0.400pt}}
\multiput(1291.00,327.93)(0.852,-0.482){9}{\rule{0.767pt}{0.116pt}}
\multiput(1291.00,328.17)(8.409,-6.000){2}{\rule{0.383pt}{0.400pt}}
\put(810.0,606.0){\rule[-0.200pt]{2.650pt}{0.400pt}}
\put(1279,778){\makebox(0,0)[r]{Namerané hodnoty pre cievku s jadrom}}
\put(859,638){\makebox(0,0){$\times$}}
\put(850,594){\makebox(0,0){$\times$}}
\put(841,594){\makebox(0,0){$\times$}}
\put(832,594){\makebox(0,0){$\times$}}
\put(795,594){\makebox(0,0){$\times$}}
\put(749,551){\makebox(0,0){$\times$}}
\put(703,551){\makebox(0,0){$\times$}}
\put(611,507){\makebox(0,0){$\times$}}
\put(519,420){\makebox(0,0){$\times$}}
\put(933,594){\makebox(0,0){$\times$}}
\put(1025,594){\makebox(0,0){$\times$}}
\put(1117,420){\makebox(0,0){$\times$}}
\put(1071,464){\makebox(0,0){$\times$}}
\put(1301,290){\makebox(0,0){$\times$}}
\put(1255,333){\makebox(0,0){$\times$}}
\put(289,202){\makebox(0,0){$\times$}}
\put(381,290){\makebox(0,0){$\times$}}
\put(427,333){\makebox(0,0){$\times$}}
\put(473,376){\makebox(0,0){$\times$}}
\put(565,464){\makebox(0,0){$\times$}}
\put(1349,778){\makebox(0,0){$\times$}}
\put(151.0,131.0){\rule[-0.200pt]{0.400pt}{175.375pt}}
\put(151.0,131.0){\rule[-0.200pt]{310.279pt}{0.400pt}}
\put(1439.0,131.0){\rule[-0.200pt]{0.400pt}{175.375pt}}
\put(151.0,859.0){\rule[-0.200pt]{310.279pt}{0.400pt}}
\end{picture}

\caption{Závislosť veľkosti prúdu $I$ na rezonančnej frekvencií $f$ pre cievku s jadrom, preložená funkciou $I = \frac{11.5\pm0.14}{\sqrt{1+\(2.5\pm0.1\)\cdot\(\frac{f}{212.2\pm0.1}-\frac{212.2\pm0.1}{f}\)^2}}$.}  \label{G_2}
\end{figure}

\begin{figure}
% GNUPLOT: LaTeX picture
\setlength{\unitlength}{0.240900pt}
\ifx\plotpoint\undefined\newsavebox{\plotpoint}\fi
\begin{picture}(1500,900)(0,0)
\sbox{\plotpoint}{\rule[-0.200pt]{0.400pt}{0.400pt}}%
\put(151.0,131.0){\rule[-0.200pt]{4.818pt}{0.400pt}}
\put(131,131){\makebox(0,0)[r]{ 5}}
\put(1419.0,131.0){\rule[-0.200pt]{4.818pt}{0.400pt}}
\put(151.0,212.0){\rule[-0.200pt]{4.818pt}{0.400pt}}
\put(131,212){\makebox(0,0)[r]{ 10}}
\put(1419.0,212.0){\rule[-0.200pt]{4.818pt}{0.400pt}}
\put(151.0,293.0){\rule[-0.200pt]{4.818pt}{0.400pt}}
\put(131,293){\makebox(0,0)[r]{ 15}}
\put(1419.0,293.0){\rule[-0.200pt]{4.818pt}{0.400pt}}
\put(151.0,374.0){\rule[-0.200pt]{4.818pt}{0.400pt}}
\put(131,374){\makebox(0,0)[r]{ 20}}
\put(1419.0,374.0){\rule[-0.200pt]{4.818pt}{0.400pt}}
\put(151.0,455.0){\rule[-0.200pt]{4.818pt}{0.400pt}}
\put(131,455){\makebox(0,0)[r]{ 25}}
\put(1419.0,455.0){\rule[-0.200pt]{4.818pt}{0.400pt}}
\put(151.0,535.0){\rule[-0.200pt]{4.818pt}{0.400pt}}
\put(131,535){\makebox(0,0)[r]{ 30}}
\put(1419.0,535.0){\rule[-0.200pt]{4.818pt}{0.400pt}}
\put(151.0,616.0){\rule[-0.200pt]{4.818pt}{0.400pt}}
\put(131,616){\makebox(0,0)[r]{ 35}}
\put(1419.0,616.0){\rule[-0.200pt]{4.818pt}{0.400pt}}
\put(151.0,697.0){\rule[-0.200pt]{4.818pt}{0.400pt}}
\put(131,697){\makebox(0,0)[r]{ 40}}
\put(1419.0,697.0){\rule[-0.200pt]{4.818pt}{0.400pt}}
\put(151.0,778.0){\rule[-0.200pt]{4.818pt}{0.400pt}}
\put(131,778){\makebox(0,0)[r]{ 45}}
\put(1419.0,778.0){\rule[-0.200pt]{4.818pt}{0.400pt}}
\put(151.0,859.0){\rule[-0.200pt]{4.818pt}{0.400pt}}
\put(131,859){\makebox(0,0)[r]{ 50}}
\put(1419.0,859.0){\rule[-0.200pt]{4.818pt}{0.400pt}}
\put(151.0,131.0){\rule[-0.200pt]{0.400pt}{4.818pt}}
\put(151,90){\makebox(0,0){ 140}}
\put(151.0,839.0){\rule[-0.200pt]{0.400pt}{4.818pt}}
\put(335.0,131.0){\rule[-0.200pt]{0.400pt}{4.818pt}}
\put(335,90){\makebox(0,0){ 160}}
\put(335.0,839.0){\rule[-0.200pt]{0.400pt}{4.818pt}}
\put(519.0,131.0){\rule[-0.200pt]{0.400pt}{4.818pt}}
\put(519,90){\makebox(0,0){ 180}}
\put(519.0,839.0){\rule[-0.200pt]{0.400pt}{4.818pt}}
\put(703.0,131.0){\rule[-0.200pt]{0.400pt}{4.818pt}}
\put(703,90){\makebox(0,0){ 200}}
\put(703.0,839.0){\rule[-0.200pt]{0.400pt}{4.818pt}}
\put(887.0,131.0){\rule[-0.200pt]{0.400pt}{4.818pt}}
\put(887,90){\makebox(0,0){ 220}}
\put(887.0,839.0){\rule[-0.200pt]{0.400pt}{4.818pt}}
\put(1071.0,131.0){\rule[-0.200pt]{0.400pt}{4.818pt}}
\put(1071,90){\makebox(0,0){ 240}}
\put(1071.0,839.0){\rule[-0.200pt]{0.400pt}{4.818pt}}
\put(1255.0,131.0){\rule[-0.200pt]{0.400pt}{4.818pt}}
\put(1255,90){\makebox(0,0){ 260}}
\put(1255.0,839.0){\rule[-0.200pt]{0.400pt}{4.818pt}}
\put(1439.0,131.0){\rule[-0.200pt]{0.400pt}{4.818pt}}
\put(1439,90){\makebox(0,0){ 280}}
\put(1439.0,839.0){\rule[-0.200pt]{0.400pt}{4.818pt}}
\put(151.0,131.0){\rule[-0.200pt]{0.400pt}{175.375pt}}
\put(151.0,131.0){\rule[-0.200pt]{310.279pt}{0.400pt}}
\put(1439.0,131.0){\rule[-0.200pt]{0.400pt}{175.375pt}}
\put(151.0,859.0){\rule[-0.200pt]{310.279pt}{0.400pt}}
\put(30,495){\makebox(0,0){\popi{I}{mA}}}
\put(795,29){\makebox(0,0){\popi{f}{kHz}}}
\put(1279,819){\makebox(0,0)[r]{fit $I = I(f)$ pre cievku s jadrom}}
\put(1299.0,819.0){\rule[-0.200pt]{24.090pt}{0.400pt}}
\put(289,148){\usebox{\plotpoint}}
\put(289,148.17){\rule{2.100pt}{0.400pt}}
\multiput(289.00,147.17)(5.641,2.000){2}{\rule{1.050pt}{0.400pt}}
\put(299,150.17){\rule{2.100pt}{0.400pt}}
\multiput(299.00,149.17)(5.641,2.000){2}{\rule{1.050pt}{0.400pt}}
\put(309,151.67){\rule{2.650pt}{0.400pt}}
\multiput(309.00,151.17)(5.500,1.000){2}{\rule{1.325pt}{0.400pt}}
\put(320,153.17){\rule{2.100pt}{0.400pt}}
\multiput(320.00,152.17)(5.641,2.000){2}{\rule{1.050pt}{0.400pt}}
\put(330,155.17){\rule{2.100pt}{0.400pt}}
\multiput(330.00,154.17)(5.641,2.000){2}{\rule{1.050pt}{0.400pt}}
\put(340,157.17){\rule{2.100pt}{0.400pt}}
\multiput(340.00,156.17)(5.641,2.000){2}{\rule{1.050pt}{0.400pt}}
\put(350,159.17){\rule{2.300pt}{0.400pt}}
\multiput(350.00,158.17)(6.226,2.000){2}{\rule{1.150pt}{0.400pt}}
\put(361,161.17){\rule{2.100pt}{0.400pt}}
\multiput(361.00,160.17)(5.641,2.000){2}{\rule{1.050pt}{0.400pt}}
\put(371,163.17){\rule{2.100pt}{0.400pt}}
\multiput(371.00,162.17)(5.641,2.000){2}{\rule{1.050pt}{0.400pt}}
\put(381,165.17){\rule{2.100pt}{0.400pt}}
\multiput(381.00,164.17)(5.641,2.000){2}{\rule{1.050pt}{0.400pt}}
\put(391,167.17){\rule{2.100pt}{0.400pt}}
\multiput(391.00,166.17)(5.641,2.000){2}{\rule{1.050pt}{0.400pt}}
\put(401,169.17){\rule{2.300pt}{0.400pt}}
\multiput(401.00,168.17)(6.226,2.000){2}{\rule{1.150pt}{0.400pt}}
\put(412,171.17){\rule{2.100pt}{0.400pt}}
\multiput(412.00,170.17)(5.641,2.000){2}{\rule{1.050pt}{0.400pt}}
\put(422,173.17){\rule{2.100pt}{0.400pt}}
\multiput(422.00,172.17)(5.641,2.000){2}{\rule{1.050pt}{0.400pt}}
\put(432,175.17){\rule{2.100pt}{0.400pt}}
\multiput(432.00,174.17)(5.641,2.000){2}{\rule{1.050pt}{0.400pt}}
\put(442,177.17){\rule{2.300pt}{0.400pt}}
\multiput(442.00,176.17)(6.226,2.000){2}{\rule{1.150pt}{0.400pt}}
\put(453,179.17){\rule{2.100pt}{0.400pt}}
\multiput(453.00,178.17)(5.641,2.000){2}{\rule{1.050pt}{0.400pt}}
\put(463,181.17){\rule{2.100pt}{0.400pt}}
\multiput(463.00,180.17)(5.641,2.000){2}{\rule{1.050pt}{0.400pt}}
\put(473,183.17){\rule{2.100pt}{0.400pt}}
\multiput(473.00,182.17)(5.641,2.000){2}{\rule{1.050pt}{0.400pt}}
\multiput(483.00,185.61)(2.025,0.447){3}{\rule{1.433pt}{0.108pt}}
\multiput(483.00,184.17)(7.025,3.000){2}{\rule{0.717pt}{0.400pt}}
\put(493,188.17){\rule{2.300pt}{0.400pt}}
\multiput(493.00,187.17)(6.226,2.000){2}{\rule{1.150pt}{0.400pt}}
\put(504,190.17){\rule{2.100pt}{0.400pt}}
\multiput(504.00,189.17)(5.641,2.000){2}{\rule{1.050pt}{0.400pt}}
\put(514,192.17){\rule{2.100pt}{0.400pt}}
\multiput(514.00,191.17)(5.641,2.000){2}{\rule{1.050pt}{0.400pt}}
\put(524,194.17){\rule{2.100pt}{0.400pt}}
\multiput(524.00,193.17)(5.641,2.000){2}{\rule{1.050pt}{0.400pt}}
\multiput(534.00,196.61)(2.248,0.447){3}{\rule{1.567pt}{0.108pt}}
\multiput(534.00,195.17)(7.748,3.000){2}{\rule{0.783pt}{0.400pt}}
\put(545,199.17){\rule{2.100pt}{0.400pt}}
\multiput(545.00,198.17)(5.641,2.000){2}{\rule{1.050pt}{0.400pt}}
\put(555,201.17){\rule{2.100pt}{0.400pt}}
\multiput(555.00,200.17)(5.641,2.000){2}{\rule{1.050pt}{0.400pt}}
\put(565,203.17){\rule{2.100pt}{0.400pt}}
\multiput(565.00,202.17)(5.641,2.000){2}{\rule{1.050pt}{0.400pt}}
\put(575,205.17){\rule{2.100pt}{0.400pt}}
\multiput(575.00,204.17)(5.641,2.000){2}{\rule{1.050pt}{0.400pt}}
\multiput(585.00,207.61)(2.248,0.447){3}{\rule{1.567pt}{0.108pt}}
\multiput(585.00,206.17)(7.748,3.000){2}{\rule{0.783pt}{0.400pt}}
\put(596,210.17){\rule{2.100pt}{0.400pt}}
\multiput(596.00,209.17)(5.641,2.000){2}{\rule{1.050pt}{0.400pt}}
\put(606,212.17){\rule{2.100pt}{0.400pt}}
\multiput(606.00,211.17)(5.641,2.000){2}{\rule{1.050pt}{0.400pt}}
\put(616,214.17){\rule{2.100pt}{0.400pt}}
\multiput(616.00,213.17)(5.641,2.000){2}{\rule{1.050pt}{0.400pt}}
\put(626,216.17){\rule{2.300pt}{0.400pt}}
\multiput(626.00,215.17)(6.226,2.000){2}{\rule{1.150pt}{0.400pt}}
\put(637,218.17){\rule{2.100pt}{0.400pt}}
\multiput(637.00,217.17)(5.641,2.000){2}{\rule{1.050pt}{0.400pt}}
\put(647,219.67){\rule{2.409pt}{0.400pt}}
\multiput(647.00,219.17)(5.000,1.000){2}{\rule{1.204pt}{0.400pt}}
\put(657,221.17){\rule{2.100pt}{0.400pt}}
\multiput(657.00,220.17)(5.641,2.000){2}{\rule{1.050pt}{0.400pt}}
\put(667,223.17){\rule{2.100pt}{0.400pt}}
\multiput(667.00,222.17)(5.641,2.000){2}{\rule{1.050pt}{0.400pt}}
\put(677,224.67){\rule{2.650pt}{0.400pt}}
\multiput(677.00,224.17)(5.500,1.000){2}{\rule{1.325pt}{0.400pt}}
\put(688,226.17){\rule{2.100pt}{0.400pt}}
\multiput(688.00,225.17)(5.641,2.000){2}{\rule{1.050pt}{0.400pt}}
\put(698,227.67){\rule{2.409pt}{0.400pt}}
\multiput(698.00,227.17)(5.000,1.000){2}{\rule{1.204pt}{0.400pt}}
\put(708,229.17){\rule{2.100pt}{0.400pt}}
\multiput(708.00,228.17)(5.641,2.000){2}{\rule{1.050pt}{0.400pt}}
\put(718,230.67){\rule{2.650pt}{0.400pt}}
\multiput(718.00,230.17)(5.500,1.000){2}{\rule{1.325pt}{0.400pt}}
\put(729,231.67){\rule{2.409pt}{0.400pt}}
\multiput(729.00,231.17)(5.000,1.000){2}{\rule{1.204pt}{0.400pt}}
\put(739,232.67){\rule{2.409pt}{0.400pt}}
\multiput(739.00,232.17)(5.000,1.000){2}{\rule{1.204pt}{0.400pt}}
\put(749,233.67){\rule{2.409pt}{0.400pt}}
\multiput(749.00,233.17)(5.000,1.000){2}{\rule{1.204pt}{0.400pt}}
\put(769,234.67){\rule{2.650pt}{0.400pt}}
\multiput(769.00,234.17)(5.500,1.000){2}{\rule{1.325pt}{0.400pt}}
\put(759.0,235.0){\rule[-0.200pt]{2.409pt}{0.400pt}}
\put(851,234.67){\rule{2.409pt}{0.400pt}}
\multiput(851.00,235.17)(5.000,-1.000){2}{\rule{1.204pt}{0.400pt}}
\put(780.0,236.0){\rule[-0.200pt]{17.104pt}{0.400pt}}
\put(872,233.67){\rule{2.409pt}{0.400pt}}
\multiput(872.00,234.17)(5.000,-1.000){2}{\rule{1.204pt}{0.400pt}}
\put(882,232.67){\rule{2.409pt}{0.400pt}}
\multiput(882.00,233.17)(5.000,-1.000){2}{\rule{1.204pt}{0.400pt}}
\put(892,231.67){\rule{2.409pt}{0.400pt}}
\multiput(892.00,232.17)(5.000,-1.000){2}{\rule{1.204pt}{0.400pt}}
\put(902,230.67){\rule{2.650pt}{0.400pt}}
\multiput(902.00,231.17)(5.500,-1.000){2}{\rule{1.325pt}{0.400pt}}
\put(913,229.67){\rule{2.409pt}{0.400pt}}
\multiput(913.00,230.17)(5.000,-1.000){2}{\rule{1.204pt}{0.400pt}}
\put(923,228.67){\rule{2.409pt}{0.400pt}}
\multiput(923.00,229.17)(5.000,-1.000){2}{\rule{1.204pt}{0.400pt}}
\put(933,227.67){\rule{2.409pt}{0.400pt}}
\multiput(933.00,228.17)(5.000,-1.000){2}{\rule{1.204pt}{0.400pt}}
\put(943,226.17){\rule{2.100pt}{0.400pt}}
\multiput(943.00,227.17)(5.641,-2.000){2}{\rule{1.050pt}{0.400pt}}
\put(953,224.67){\rule{2.650pt}{0.400pt}}
\multiput(953.00,225.17)(5.500,-1.000){2}{\rule{1.325pt}{0.400pt}}
\put(964,223.67){\rule{2.409pt}{0.400pt}}
\multiput(964.00,224.17)(5.000,-1.000){2}{\rule{1.204pt}{0.400pt}}
\put(974,222.17){\rule{2.100pt}{0.400pt}}
\multiput(974.00,223.17)(5.641,-2.000){2}{\rule{1.050pt}{0.400pt}}
\put(984,220.67){\rule{2.409pt}{0.400pt}}
\multiput(984.00,221.17)(5.000,-1.000){2}{\rule{1.204pt}{0.400pt}}
\put(994,219.17){\rule{2.300pt}{0.400pt}}
\multiput(994.00,220.17)(6.226,-2.000){2}{\rule{1.150pt}{0.400pt}}
\put(1005,217.17){\rule{2.100pt}{0.400pt}}
\multiput(1005.00,218.17)(5.641,-2.000){2}{\rule{1.050pt}{0.400pt}}
\put(1015,215.67){\rule{2.409pt}{0.400pt}}
\multiput(1015.00,216.17)(5.000,-1.000){2}{\rule{1.204pt}{0.400pt}}
\put(1025,214.17){\rule{2.100pt}{0.400pt}}
\multiput(1025.00,215.17)(5.641,-2.000){2}{\rule{1.050pt}{0.400pt}}
\put(1035,212.17){\rule{2.100pt}{0.400pt}}
\multiput(1035.00,213.17)(5.641,-2.000){2}{\rule{1.050pt}{0.400pt}}
\put(1045,210.67){\rule{2.650pt}{0.400pt}}
\multiput(1045.00,211.17)(5.500,-1.000){2}{\rule{1.325pt}{0.400pt}}
\put(1056,209.17){\rule{2.100pt}{0.400pt}}
\multiput(1056.00,210.17)(5.641,-2.000){2}{\rule{1.050pt}{0.400pt}}
\put(1066,207.67){\rule{2.409pt}{0.400pt}}
\multiput(1066.00,208.17)(5.000,-1.000){2}{\rule{1.204pt}{0.400pt}}
\put(1076,206.17){\rule{2.100pt}{0.400pt}}
\multiput(1076.00,207.17)(5.641,-2.000){2}{\rule{1.050pt}{0.400pt}}
\put(1086,204.17){\rule{2.300pt}{0.400pt}}
\multiput(1086.00,205.17)(6.226,-2.000){2}{\rule{1.150pt}{0.400pt}}
\put(1097,202.17){\rule{2.100pt}{0.400pt}}
\multiput(1097.00,203.17)(5.641,-2.000){2}{\rule{1.050pt}{0.400pt}}
\put(1107,200.67){\rule{2.409pt}{0.400pt}}
\multiput(1107.00,201.17)(5.000,-1.000){2}{\rule{1.204pt}{0.400pt}}
\put(1117,199.17){\rule{2.100pt}{0.400pt}}
\multiput(1117.00,200.17)(5.641,-2.000){2}{\rule{1.050pt}{0.400pt}}
\put(1127,197.67){\rule{2.409pt}{0.400pt}}
\multiput(1127.00,198.17)(5.000,-1.000){2}{\rule{1.204pt}{0.400pt}}
\put(1137,196.17){\rule{2.300pt}{0.400pt}}
\multiput(1137.00,197.17)(6.226,-2.000){2}{\rule{1.150pt}{0.400pt}}
\put(1148,194.17){\rule{2.100pt}{0.400pt}}
\multiput(1148.00,195.17)(5.641,-2.000){2}{\rule{1.050pt}{0.400pt}}
\put(1158,192.67){\rule{2.409pt}{0.400pt}}
\multiput(1158.00,193.17)(5.000,-1.000){2}{\rule{1.204pt}{0.400pt}}
\put(1168,191.17){\rule{2.100pt}{0.400pt}}
\multiput(1168.00,192.17)(5.641,-2.000){2}{\rule{1.050pt}{0.400pt}}
\put(1178,189.67){\rule{2.650pt}{0.400pt}}
\multiput(1178.00,190.17)(5.500,-1.000){2}{\rule{1.325pt}{0.400pt}}
\put(1189,188.17){\rule{2.100pt}{0.400pt}}
\multiput(1189.00,189.17)(5.641,-2.000){2}{\rule{1.050pt}{0.400pt}}
\put(1199,186.17){\rule{2.100pt}{0.400pt}}
\multiput(1199.00,187.17)(5.641,-2.000){2}{\rule{1.050pt}{0.400pt}}
\put(1209,184.67){\rule{2.409pt}{0.400pt}}
\multiput(1209.00,185.17)(5.000,-1.000){2}{\rule{1.204pt}{0.400pt}}
\put(1219,183.17){\rule{2.100pt}{0.400pt}}
\multiput(1219.00,184.17)(5.641,-2.000){2}{\rule{1.050pt}{0.400pt}}
\put(1229,181.67){\rule{2.650pt}{0.400pt}}
\multiput(1229.00,182.17)(5.500,-1.000){2}{\rule{1.325pt}{0.400pt}}
\put(1240,180.67){\rule{2.409pt}{0.400pt}}
\multiput(1240.00,181.17)(5.000,-1.000){2}{\rule{1.204pt}{0.400pt}}
\put(1250,179.17){\rule{2.100pt}{0.400pt}}
\multiput(1250.00,180.17)(5.641,-2.000){2}{\rule{1.050pt}{0.400pt}}
\put(1260,177.67){\rule{2.409pt}{0.400pt}}
\multiput(1260.00,178.17)(5.000,-1.000){2}{\rule{1.204pt}{0.400pt}}
\put(1270,176.17){\rule{2.300pt}{0.400pt}}
\multiput(1270.00,177.17)(6.226,-2.000){2}{\rule{1.150pt}{0.400pt}}
\put(1281,174.67){\rule{2.409pt}{0.400pt}}
\multiput(1281.00,175.17)(5.000,-1.000){2}{\rule{1.204pt}{0.400pt}}
\put(1291,173.67){\rule{2.409pt}{0.400pt}}
\multiput(1291.00,174.17)(5.000,-1.000){2}{\rule{1.204pt}{0.400pt}}
\put(861.0,235.0){\rule[-0.200pt]{2.650pt}{0.400pt}}
\put(1279,778){\makebox(0,0)[r]{Namerané hodnoty pre cievku s jadrom}}
\put(859,244){\makebox(0,0){$\times$}}
\put(850,234){\makebox(0,0){$\times$}}
\put(841,234){\makebox(0,0){$\times$}}
\put(832,234){\makebox(0,0){$\times$}}
\put(795,234){\makebox(0,0){$\times$}}
\put(749,224){\makebox(0,0){$\times$}}
\put(703,224){\makebox(0,0){$\times$}}
\put(611,215){\makebox(0,0){$\times$}}
\put(519,195){\makebox(0,0){$\times$}}
\put(933,234){\makebox(0,0){$\times$}}
\put(1025,234){\makebox(0,0){$\times$}}
\put(1117,195){\makebox(0,0){$\times$}}
\put(1071,205){\makebox(0,0){$\times$}}
\put(1301,166){\makebox(0,0){$\times$}}
\put(1255,176){\makebox(0,0){$\times$}}
\put(289,147){\makebox(0,0){$\times$}}
\put(381,166){\makebox(0,0){$\times$}}
\put(427,176){\makebox(0,0){$\times$}}
\put(473,186){\makebox(0,0){$\times$}}
\put(565,205){\makebox(0,0){$\times$}}
\put(1349,778){\makebox(0,0){$\times$}}
\sbox{\plotpoint}{\rule[-0.400pt]{0.800pt}{0.800pt}}%
\sbox{\plotpoint}{\rule[-0.200pt]{0.400pt}{0.400pt}}%
\put(1279,737){\makebox(0,0)[r]{fit $I = I(f)$ pre cievku bez jadra}}
\sbox{\plotpoint}{\rule[-0.400pt]{0.800pt}{0.800pt}}%
\put(1299.0,737.0){\rule[-0.400pt]{24.090pt}{0.800pt}}
\put(289,147){\usebox{\plotpoint}}
\put(289,146.34){\rule{2.409pt}{0.800pt}}
\multiput(289.00,145.34)(5.000,2.000){2}{\rule{1.204pt}{0.800pt}}
\put(299,148.34){\rule{2.409pt}{0.800pt}}
\multiput(299.00,147.34)(5.000,2.000){2}{\rule{1.204pt}{0.800pt}}
\put(309,150.34){\rule{2.650pt}{0.800pt}}
\multiput(309.00,149.34)(5.500,2.000){2}{\rule{1.325pt}{0.800pt}}
\put(320,152.34){\rule{2.409pt}{0.800pt}}
\multiput(320.00,151.34)(5.000,2.000){2}{\rule{1.204pt}{0.800pt}}
\put(330,154.34){\rule{2.409pt}{0.800pt}}
\multiput(330.00,153.34)(5.000,2.000){2}{\rule{1.204pt}{0.800pt}}
\put(340,156.84){\rule{2.409pt}{0.800pt}}
\multiput(340.00,155.34)(5.000,3.000){2}{\rule{1.204pt}{0.800pt}}
\put(350,159.34){\rule{2.650pt}{0.800pt}}
\multiput(350.00,158.34)(5.500,2.000){2}{\rule{1.325pt}{0.800pt}}
\put(361,161.84){\rule{2.409pt}{0.800pt}}
\multiput(361.00,160.34)(5.000,3.000){2}{\rule{1.204pt}{0.800pt}}
\put(371,164.84){\rule{2.409pt}{0.800pt}}
\multiput(371.00,163.34)(5.000,3.000){2}{\rule{1.204pt}{0.800pt}}
\put(381,167.34){\rule{2.409pt}{0.800pt}}
\multiput(381.00,166.34)(5.000,2.000){2}{\rule{1.204pt}{0.800pt}}
\put(391,169.84){\rule{2.409pt}{0.800pt}}
\multiput(391.00,168.34)(5.000,3.000){2}{\rule{1.204pt}{0.800pt}}
\put(401,172.84){\rule{2.650pt}{0.800pt}}
\multiput(401.00,171.34)(5.500,3.000){2}{\rule{1.325pt}{0.800pt}}
\put(412,175.84){\rule{2.409pt}{0.800pt}}
\multiput(412.00,174.34)(5.000,3.000){2}{\rule{1.204pt}{0.800pt}}
\put(422,179.34){\rule{2.200pt}{0.800pt}}
\multiput(422.00,177.34)(5.434,4.000){2}{\rule{1.100pt}{0.800pt}}
\put(432,182.84){\rule{2.409pt}{0.800pt}}
\multiput(432.00,181.34)(5.000,3.000){2}{\rule{1.204pt}{0.800pt}}
\put(442,185.84){\rule{2.650pt}{0.800pt}}
\multiput(442.00,184.34)(5.500,3.000){2}{\rule{1.325pt}{0.800pt}}
\put(453,189.34){\rule{2.200pt}{0.800pt}}
\multiput(453.00,187.34)(5.434,4.000){2}{\rule{1.100pt}{0.800pt}}
\put(463,193.34){\rule{2.200pt}{0.800pt}}
\multiput(463.00,191.34)(5.434,4.000){2}{\rule{1.100pt}{0.800pt}}
\put(473,197.34){\rule{2.200pt}{0.800pt}}
\multiput(473.00,195.34)(5.434,4.000){2}{\rule{1.100pt}{0.800pt}}
\put(483,201.34){\rule{2.200pt}{0.800pt}}
\multiput(483.00,199.34)(5.434,4.000){2}{\rule{1.100pt}{0.800pt}}
\put(493,205.34){\rule{2.400pt}{0.800pt}}
\multiput(493.00,203.34)(6.019,4.000){2}{\rule{1.200pt}{0.800pt}}
\multiput(504.00,210.38)(1.264,0.560){3}{\rule{1.800pt}{0.135pt}}
\multiput(504.00,207.34)(6.264,5.000){2}{\rule{0.900pt}{0.800pt}}
\multiput(514.00,215.38)(1.264,0.560){3}{\rule{1.800pt}{0.135pt}}
\multiput(514.00,212.34)(6.264,5.000){2}{\rule{0.900pt}{0.800pt}}
\multiput(524.00,220.38)(1.264,0.560){3}{\rule{1.800pt}{0.135pt}}
\multiput(524.00,217.34)(6.264,5.000){2}{\rule{0.900pt}{0.800pt}}
\multiput(534.00,225.38)(1.432,0.560){3}{\rule{1.960pt}{0.135pt}}
\multiput(534.00,222.34)(6.932,5.000){2}{\rule{0.980pt}{0.800pt}}
\multiput(545.00,230.38)(1.264,0.560){3}{\rule{1.800pt}{0.135pt}}
\multiput(545.00,227.34)(6.264,5.000){2}{\rule{0.900pt}{0.800pt}}
\multiput(555.00,235.39)(0.909,0.536){5}{\rule{1.533pt}{0.129pt}}
\multiput(555.00,232.34)(6.817,6.000){2}{\rule{0.767pt}{0.800pt}}
\multiput(565.00,241.39)(0.909,0.536){5}{\rule{1.533pt}{0.129pt}}
\multiput(565.00,238.34)(6.817,6.000){2}{\rule{0.767pt}{0.800pt}}
\multiput(575.00,247.40)(0.738,0.526){7}{\rule{1.343pt}{0.127pt}}
\multiput(575.00,244.34)(7.213,7.000){2}{\rule{0.671pt}{0.800pt}}
\multiput(585.00,254.40)(0.825,0.526){7}{\rule{1.457pt}{0.127pt}}
\multiput(585.00,251.34)(7.976,7.000){2}{\rule{0.729pt}{0.800pt}}
\multiput(596.00,261.40)(0.738,0.526){7}{\rule{1.343pt}{0.127pt}}
\multiput(596.00,258.34)(7.213,7.000){2}{\rule{0.671pt}{0.800pt}}
\multiput(606.00,268.40)(0.627,0.520){9}{\rule{1.200pt}{0.125pt}}
\multiput(606.00,265.34)(7.509,8.000){2}{\rule{0.600pt}{0.800pt}}
\multiput(616.00,276.40)(0.627,0.520){9}{\rule{1.200pt}{0.125pt}}
\multiput(616.00,273.34)(7.509,8.000){2}{\rule{0.600pt}{0.800pt}}
\multiput(626.00,284.40)(0.700,0.520){9}{\rule{1.300pt}{0.125pt}}
\multiput(626.00,281.34)(8.302,8.000){2}{\rule{0.650pt}{0.800pt}}
\multiput(637.00,292.40)(0.548,0.516){11}{\rule{1.089pt}{0.124pt}}
\multiput(637.00,289.34)(7.740,9.000){2}{\rule{0.544pt}{0.800pt}}
\multiput(647.00,301.40)(0.487,0.514){13}{\rule{1.000pt}{0.124pt}}
\multiput(647.00,298.34)(7.924,10.000){2}{\rule{0.500pt}{0.800pt}}
\multiput(657.00,311.40)(0.487,0.514){13}{\rule{1.000pt}{0.124pt}}
\multiput(657.00,308.34)(7.924,10.000){2}{\rule{0.500pt}{0.800pt}}
\multiput(668.40,320.00)(0.514,0.543){13}{\rule{0.124pt}{1.080pt}}
\multiput(665.34,320.00)(10.000,8.758){2}{\rule{0.800pt}{0.540pt}}
\multiput(678.40,331.00)(0.512,0.539){15}{\rule{0.123pt}{1.073pt}}
\multiput(675.34,331.00)(11.000,9.774){2}{\rule{0.800pt}{0.536pt}}
\multiput(689.40,343.00)(0.514,0.654){13}{\rule{0.124pt}{1.240pt}}
\multiput(686.34,343.00)(10.000,10.426){2}{\rule{0.800pt}{0.620pt}}
\multiput(699.40,356.00)(0.514,0.654){13}{\rule{0.124pt}{1.240pt}}
\multiput(696.34,356.00)(10.000,10.426){2}{\rule{0.800pt}{0.620pt}}
\multiput(709.40,369.00)(0.514,0.710){13}{\rule{0.124pt}{1.320pt}}
\multiput(706.34,369.00)(10.000,11.260){2}{\rule{0.800pt}{0.660pt}}
\multiput(719.40,383.00)(0.512,0.689){15}{\rule{0.123pt}{1.291pt}}
\multiput(716.34,383.00)(11.000,12.321){2}{\rule{0.800pt}{0.645pt}}
\multiput(730.40,398.00)(0.514,0.821){13}{\rule{0.124pt}{1.480pt}}
\multiput(727.34,398.00)(10.000,12.928){2}{\rule{0.800pt}{0.740pt}}
\multiput(740.40,414.00)(0.514,0.877){13}{\rule{0.124pt}{1.560pt}}
\multiput(737.34,414.00)(10.000,13.762){2}{\rule{0.800pt}{0.780pt}}
\multiput(750.40,431.00)(0.514,0.877){13}{\rule{0.124pt}{1.560pt}}
\multiput(747.34,431.00)(10.000,13.762){2}{\rule{0.800pt}{0.780pt}}
\multiput(760.40,448.00)(0.514,0.988){13}{\rule{0.124pt}{1.720pt}}
\multiput(757.34,448.00)(10.000,15.430){2}{\rule{0.800pt}{0.860pt}}
\multiput(770.40,467.00)(0.512,0.888){15}{\rule{0.123pt}{1.582pt}}
\multiput(767.34,467.00)(11.000,15.717){2}{\rule{0.800pt}{0.791pt}}
\multiput(781.40,486.00)(0.514,1.044){13}{\rule{0.124pt}{1.800pt}}
\multiput(778.34,486.00)(10.000,16.264){2}{\rule{0.800pt}{0.900pt}}
\multiput(791.40,506.00)(0.514,1.100){13}{\rule{0.124pt}{1.880pt}}
\multiput(788.34,506.00)(10.000,17.098){2}{\rule{0.800pt}{0.940pt}}
\multiput(801.40,527.00)(0.514,1.044){13}{\rule{0.124pt}{1.800pt}}
\multiput(798.34,527.00)(10.000,16.264){2}{\rule{0.800pt}{0.900pt}}
\multiput(811.40,547.00)(0.512,0.988){15}{\rule{0.123pt}{1.727pt}}
\multiput(808.34,547.00)(11.000,17.415){2}{\rule{0.800pt}{0.864pt}}
\multiput(822.40,568.00)(0.514,0.988){13}{\rule{0.124pt}{1.720pt}}
\multiput(819.34,568.00)(10.000,15.430){2}{\rule{0.800pt}{0.860pt}}
\multiput(832.40,587.00)(0.514,0.877){13}{\rule{0.124pt}{1.560pt}}
\multiput(829.34,587.00)(10.000,13.762){2}{\rule{0.800pt}{0.780pt}}
\multiput(842.40,604.00)(0.514,0.821){13}{\rule{0.124pt}{1.480pt}}
\multiput(839.34,604.00)(10.000,12.928){2}{\rule{0.800pt}{0.740pt}}
\multiput(852.40,620.00)(0.514,0.599){13}{\rule{0.124pt}{1.160pt}}
\multiput(849.34,620.00)(10.000,9.592){2}{\rule{0.800pt}{0.580pt}}
\multiput(861.00,633.40)(0.700,0.520){9}{\rule{1.300pt}{0.125pt}}
\multiput(861.00,630.34)(8.302,8.000){2}{\rule{0.650pt}{0.800pt}}
\multiput(872.00,641.38)(1.264,0.560){3}{\rule{1.800pt}{0.135pt}}
\multiput(872.00,638.34)(6.264,5.000){2}{\rule{0.900pt}{0.800pt}}
\multiput(892.00,643.06)(1.264,-0.560){3}{\rule{1.800pt}{0.135pt}}
\multiput(892.00,643.34)(6.264,-5.000){2}{\rule{0.900pt}{0.800pt}}
\multiput(902.00,638.08)(0.700,-0.520){9}{\rule{1.300pt}{0.125pt}}
\multiput(902.00,638.34)(8.302,-8.000){2}{\rule{0.650pt}{0.800pt}}
\multiput(914.40,627.18)(0.514,-0.599){13}{\rule{0.124pt}{1.160pt}}
\multiput(911.34,629.59)(10.000,-9.592){2}{\rule{0.800pt}{0.580pt}}
\multiput(924.40,614.52)(0.514,-0.710){13}{\rule{0.124pt}{1.320pt}}
\multiput(921.34,617.26)(10.000,-11.260){2}{\rule{0.800pt}{0.660pt}}
\multiput(934.40,599.52)(0.514,-0.877){13}{\rule{0.124pt}{1.560pt}}
\multiput(931.34,602.76)(10.000,-13.762){2}{\rule{0.800pt}{0.780pt}}
\multiput(944.40,582.52)(0.514,-0.877){13}{\rule{0.124pt}{1.560pt}}
\multiput(941.34,585.76)(10.000,-13.762){2}{\rule{0.800pt}{0.780pt}}
\multiput(954.40,565.43)(0.512,-0.888){15}{\rule{0.123pt}{1.582pt}}
\multiput(951.34,568.72)(11.000,-15.717){2}{\rule{0.800pt}{0.791pt}}
\multiput(965.40,545.86)(0.514,-0.988){13}{\rule{0.124pt}{1.720pt}}
\multiput(962.34,549.43)(10.000,-15.430){2}{\rule{0.800pt}{0.860pt}}
\multiput(975.40,526.86)(0.514,-0.988){13}{\rule{0.124pt}{1.720pt}}
\multiput(972.34,530.43)(10.000,-15.430){2}{\rule{0.800pt}{0.860pt}}
\multiput(985.40,508.19)(0.514,-0.933){13}{\rule{0.124pt}{1.640pt}}
\multiput(982.34,511.60)(10.000,-14.596){2}{\rule{0.800pt}{0.820pt}}
\multiput(995.40,490.74)(0.512,-0.838){15}{\rule{0.123pt}{1.509pt}}
\multiput(992.34,493.87)(11.000,-14.868){2}{\rule{0.800pt}{0.755pt}}
\multiput(1006.40,472.52)(0.514,-0.877){13}{\rule{0.124pt}{1.560pt}}
\multiput(1003.34,475.76)(10.000,-13.762){2}{\rule{0.800pt}{0.780pt}}
\multiput(1016.40,455.86)(0.514,-0.821){13}{\rule{0.124pt}{1.480pt}}
\multiput(1013.34,458.93)(10.000,-12.928){2}{\rule{0.800pt}{0.740pt}}
\multiput(1026.40,439.86)(0.514,-0.821){13}{\rule{0.124pt}{1.480pt}}
\multiput(1023.34,442.93)(10.000,-12.928){2}{\rule{0.800pt}{0.740pt}}
\multiput(1036.40,424.52)(0.514,-0.710){13}{\rule{0.124pt}{1.320pt}}
\multiput(1033.34,427.26)(10.000,-11.260){2}{\rule{0.800pt}{0.660pt}}
\multiput(1046.40,410.94)(0.512,-0.639){15}{\rule{0.123pt}{1.218pt}}
\multiput(1043.34,413.47)(11.000,-11.472){2}{\rule{0.800pt}{0.609pt}}
\multiput(1057.40,396.85)(0.514,-0.654){13}{\rule{0.124pt}{1.240pt}}
\multiput(1054.34,399.43)(10.000,-10.426){2}{\rule{0.800pt}{0.620pt}}
\multiput(1067.40,384.18)(0.514,-0.599){13}{\rule{0.124pt}{1.160pt}}
\multiput(1064.34,386.59)(10.000,-9.592){2}{\rule{0.800pt}{0.580pt}}
\multiput(1077.40,372.18)(0.514,-0.599){13}{\rule{0.124pt}{1.160pt}}
\multiput(1074.34,374.59)(10.000,-9.592){2}{\rule{0.800pt}{0.580pt}}
\multiput(1086.00,363.08)(0.543,-0.514){13}{\rule{1.080pt}{0.124pt}}
\multiput(1086.00,363.34)(8.758,-10.000){2}{\rule{0.540pt}{0.800pt}}
\multiput(1098.40,350.52)(0.514,-0.543){13}{\rule{0.124pt}{1.080pt}}
\multiput(1095.34,352.76)(10.000,-8.758){2}{\rule{0.800pt}{0.540pt}}
\multiput(1107.00,342.08)(0.548,-0.516){11}{\rule{1.089pt}{0.124pt}}
\multiput(1107.00,342.34)(7.740,-9.000){2}{\rule{0.544pt}{0.800pt}}
\multiput(1117.00,333.08)(0.548,-0.516){11}{\rule{1.089pt}{0.124pt}}
\multiput(1117.00,333.34)(7.740,-9.000){2}{\rule{0.544pt}{0.800pt}}
\multiput(1127.00,324.08)(0.548,-0.516){11}{\rule{1.089pt}{0.124pt}}
\multiput(1127.00,324.34)(7.740,-9.000){2}{\rule{0.544pt}{0.800pt}}
\multiput(1137.00,315.08)(0.700,-0.520){9}{\rule{1.300pt}{0.125pt}}
\multiput(1137.00,315.34)(8.302,-8.000){2}{\rule{0.650pt}{0.800pt}}
\multiput(1148.00,307.08)(0.738,-0.526){7}{\rule{1.343pt}{0.127pt}}
\multiput(1148.00,307.34)(7.213,-7.000){2}{\rule{0.671pt}{0.800pt}}
\multiput(1158.00,300.08)(0.738,-0.526){7}{\rule{1.343pt}{0.127pt}}
\multiput(1158.00,300.34)(7.213,-7.000){2}{\rule{0.671pt}{0.800pt}}
\multiput(1168.00,293.08)(0.738,-0.526){7}{\rule{1.343pt}{0.127pt}}
\multiput(1168.00,293.34)(7.213,-7.000){2}{\rule{0.671pt}{0.800pt}}
\multiput(1178.00,286.08)(0.825,-0.526){7}{\rule{1.457pt}{0.127pt}}
\multiput(1178.00,286.34)(7.976,-7.000){2}{\rule{0.729pt}{0.800pt}}
\multiput(1189.00,279.07)(0.909,-0.536){5}{\rule{1.533pt}{0.129pt}}
\multiput(1189.00,279.34)(6.817,-6.000){2}{\rule{0.767pt}{0.800pt}}
\multiput(1199.00,273.07)(0.909,-0.536){5}{\rule{1.533pt}{0.129pt}}
\multiput(1199.00,273.34)(6.817,-6.000){2}{\rule{0.767pt}{0.800pt}}
\multiput(1209.00,267.06)(1.264,-0.560){3}{\rule{1.800pt}{0.135pt}}
\multiput(1209.00,267.34)(6.264,-5.000){2}{\rule{0.900pt}{0.800pt}}
\multiput(1219.00,262.06)(1.264,-0.560){3}{\rule{1.800pt}{0.135pt}}
\multiput(1219.00,262.34)(6.264,-5.000){2}{\rule{0.900pt}{0.800pt}}
\multiput(1229.00,257.06)(1.432,-0.560){3}{\rule{1.960pt}{0.135pt}}
\multiput(1229.00,257.34)(6.932,-5.000){2}{\rule{0.980pt}{0.800pt}}
\multiput(1240.00,252.06)(1.264,-0.560){3}{\rule{1.800pt}{0.135pt}}
\multiput(1240.00,252.34)(6.264,-5.000){2}{\rule{0.900pt}{0.800pt}}
\multiput(1250.00,247.06)(1.264,-0.560){3}{\rule{1.800pt}{0.135pt}}
\multiput(1250.00,247.34)(6.264,-5.000){2}{\rule{0.900pt}{0.800pt}}
\put(1260,240.34){\rule{2.200pt}{0.800pt}}
\multiput(1260.00,242.34)(5.434,-4.000){2}{\rule{1.100pt}{0.800pt}}
\put(1270,236.34){\rule{2.400pt}{0.800pt}}
\multiput(1270.00,238.34)(6.019,-4.000){2}{\rule{1.200pt}{0.800pt}}
\put(1281,232.34){\rule{2.200pt}{0.800pt}}
\multiput(1281.00,234.34)(5.434,-4.000){2}{\rule{1.100pt}{0.800pt}}
\put(1291,228.34){\rule{2.200pt}{0.800pt}}
\multiput(1291.00,230.34)(5.434,-4.000){2}{\rule{1.100pt}{0.800pt}}
\put(882.0,645.0){\rule[-0.400pt]{2.409pt}{0.800pt}}
\sbox{\plotpoint}{\rule[-0.500pt]{1.000pt}{1.000pt}}%
\sbox{\plotpoint}{\rule[-0.200pt]{0.400pt}{0.400pt}}%
\put(1279,696){\makebox(0,0)[r]{Namerané hodnoty pre cievku bez jadra}}
\sbox{\plotpoint}{\rule[-0.500pt]{1.000pt}{1.000pt}}%
\put(859,665){\raisebox{-.8pt}{\makebox(0,0){$\Box$}}}
\put(951,568){\raisebox{-.8pt}{\makebox(0,0){$\Box$}}}
\put(1043,374){\raisebox{-.8pt}{\makebox(0,0){$\Box$}}}
\put(997,455){\raisebox{-.8pt}{\makebox(0,0){$\Box$}}}
\put(905,616){\raisebox{-.8pt}{\makebox(0,0){$\Box$}}}
\put(813,568){\raisebox{-.8pt}{\makebox(0,0){$\Box$}}}
\put(767,503){\raisebox{-.8pt}{\makebox(0,0){$\Box$}}}
\put(721,422){\raisebox{-.8pt}{\makebox(0,0){$\Box$}}}
\put(675,357){\raisebox{-.8pt}{\makebox(0,0){$\Box$}}}
\put(703,390){\raisebox{-.8pt}{\makebox(0,0){$\Box$}}}
\put(740,438){\raisebox{-.8pt}{\makebox(0,0){$\Box$}}}
\put(795,552){\raisebox{-.8pt}{\makebox(0,0){$\Box$}}}
\put(823,568){\raisebox{-.8pt}{\makebox(0,0){$\Box$}}}
\put(841,633){\raisebox{-.8pt}{\makebox(0,0){$\Box$}}}
\put(850,649){\raisebox{-.8pt}{\makebox(0,0){$\Box$}}}
\put(869,649){\raisebox{-.8pt}{\makebox(0,0){$\Box$}}}
\put(887,633){\raisebox{-.8pt}{\makebox(0,0){$\Box$}}}
\put(933,568){\raisebox{-.8pt}{\makebox(0,0){$\Box$}}}
\put(979,487){\raisebox{-.8pt}{\makebox(0,0){$\Box$}}}
\put(1025,422){\raisebox{-.8pt}{\makebox(0,0){$\Box$}}}
\put(1349,696){\raisebox{-.8pt}{\makebox(0,0){$\Box$}}}
\sbox{\plotpoint}{\rule[-0.200pt]{0.400pt}{0.400pt}}%
\put(151.0,131.0){\rule[-0.200pt]{0.400pt}{175.375pt}}
\put(151.0,131.0){\rule[-0.200pt]{310.279pt}{0.400pt}}
\put(1439.0,131.0){\rule[-0.200pt]{0.400pt}{175.375pt}}
\put(151.0,859.0){\rule[-0.200pt]{310.279pt}{0.400pt}}
\end{picture}

\caption{Závislosť veľkosti prúdu $I$ na rezonančnej frekvencií $f$ pre cievku bez jadra, preložená funkciou $I = \frac{36.8\pm0.4}{\sqrt{1+\(8.5\pm0.3\)\cdot\(\frac{f}{218.1\pm0.2}-\frac{218.1\pm0.2}{f}\)^2}}$ a závislosť veľkosti prúdu $I$ na rezonančnej frekvencií $f$ pre cievku s jadrom, preložená funkciou $I = \frac{11.5\pm0.14}{\sqrt{1+\(2.5\pm0.1\)\cdot\(\frac{f}{212.2\pm0.1}-\frac{212.2\pm0.1}{f}\)^2}}$.}  \label{G_3}
\end{figure}



\section{Diskusia}
V Prvej časti sme spočítali teoretickú rezonančnú frekvenciu obvodu ako $\Omega_0= "1.41 MHz"$ ale nameraná hodnota je $f_0 = "\(216\pm1\) kHz"$. Čo je chyba o skoro celý rád. Vyzerá to pravdepodobne na chybu vo výpočtoch alebo použitých hodnôt pre výpočet.

V druhej časti sa pri vložení jadra do cievky sa zmenšil pretekajúci prúd a teda zmenšila akosť. Teda môžeme predpokladať, že jadro je za paramagnetického materiálu.

Dátam z predchádzajúceho bodu odpovedajú aj dáta vynesené v grafe Obr. \ref{G_3} kde je jasne vidieť pokles prídu v oblasti rezonančnej frekvencie pri vloženom jadre v cievke. 

Pri meraní kapacity neznámeho kondenzátoru bolo meranie uskutočnené len raz, však podľa \cite{C_1} sme mali meranie previesť 5 krát. 

Posledné meranie vzdialenosti je predovšetkým pri hľadaní maxima zaťažené mnohými systematickými chybami. 
V prvej rade sú obe cievky za v plastovom obale teda neviem odmerať ich presnú vzdialenosť ale len vzdialenosť krytov. 
Cievky niesu dokonalé rovnobežné voči sebe a teda ich vzdialenosť sa v rôznych miestach po obvode líši, odhadom $\sim "0.2 cm"$. 


\section{Záver}

Vlastná frekvencia obvodu bola určená $f_0 = "\(216\pm1\) kHz"$.
Cievka s jadrom má indukčnosť $L_2 ="0.085 mH"$.
Neznámy kondenzátor má kapacitu $C_x = "79 pF"$.


\begin{thebibliography}{2}
\bibitem{C_1}
Sériový a vázaný rezonanční obvod [cit. 4.12.2016]Dostupné po prihlásení z Kurz: Fyzikální praktikum I:\url{https://praktikum.fjfi.cvut.cz/pluginfile.php/123/mod_resource/content/7/navod_rezonancni_obvody_161005.pdf}

\end{thebibliography}

\end{document}


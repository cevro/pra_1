\documentclass[a4paper,10pt]{article}
%\usepackage[IL2]{fontenc}
\usepackage[utf8x]{inputenc}
\usepackage[czech]{babel}
\usepackage{amsfonts,amsmath,amssymb,graphicx,color}
%\usepackage[total={17cm,27cm}, top=2cm, left=2cm, includefoot]{geometry}
%\usepackage{fancyhdr}
\usepackage{fkssugar}
\usepackage{hyperref}

%\usepackage{caption}
\renewcommand{\popi}[2]{$#1$[\jd{#2}]}
\renewcommand{\figurename}{Obr.}
\addto\captionsczech{\renewcommand{\figurename}{Obr.}}
\addto\captionsczech{\renewcommand{\tablename}{Tab.}}

\begin{document}
\def\mean#1{\left< #1 \right>}
\noindent
{\large Fyzikální praktikum 1.} \hfil {\large FJFI ČVUT V Praze}\\
\noindent
{\large\textbf{pracovní úkol \# 12}}
\begin{center}
{\large\textit{Sonar}}
\end{center}
\noindent
\rule{\textwidth}{1px}
\vspace{\baselineskip}

\emph{Michal Červeňák}
\par
\vspace{\baselineskip}
\begin{minipage}[l]{0.5\textwidth}%
\textit{dátum merania:}~05.12. 2016\\%
%\vspace{\baselineskip}%
\par%
\noindent%
\textit{skupina:}~4\\%
%\vspace{\baselineskip}%
\par%
\noindent%
\textit{Klasifikace:}\dotfill\\%
\end{minipage}

\section{Pracovní úkol}

\begin{enumerate}
\item DU: V domácí přípravě spočítejte úhel prvních pěti maxim dle vztahu (6) pro 1,
2 a 5 štěrbin, znáte-li mřížkovou konstantu $d = "3 cm"$, šířku štěrbiny a = 1 cm
a frekvenci vlnění $f = "40 kHz"$. Tvar difrakčního obrazce graficky znázorněte.
\item  Změřte velikost přijímaného signálu v závislosti na úhlu mezi přijímačem a kolmicí k odrazové
ploše. Výsledky zpracujte tabulkově i graficky a ověřte, zda-li platí zákon odrazu pro ultrazvukové
vlny. Měření proveďte pro 3 různé pevně zvolené úhly dopadu.
\item  Změřte rychlost zvuku ve vzduchu. Proveďte alespoň deset měření při různých vzdálenostech
vysílače od přijímače a výsledky zpracujte statisticky. Porovnejte váš výsledek se vztahem (3).
\item  Změřte alespoˇn pět vzdáleností odrazové plochy od vysílače/přijímače pomocí ultrazvukových
vln (princip sonaru). Porovnejte vzdálenosti měřené sonarem a měčítkem. Použijte vámi experimentálně
stanovenou rychlost zvuku z úkolu 2.
\item  Proměřte závislost intenzity zvukového signálu po průchodu zvukových vln soustavou štěrbin
pro N (počet štěrbin) = 1,2,5. Výsledky zpracujte graficky a okomentujte v protokolu.
\item Změřte Dopplerův jev pro dvě rychlosti vozíčku pro jeden případ (přijímač klid nebo přijímač
pohyb) a porovnejte vyýsledky s teoretickými výpočty. Měření proveďte pro kaˇzdou rychlost
minimálné 5-krát.
\end{enumerate}



\section{Pomôcky}
Generátor 40 kHz vln, zesilovač, 1 mikrofon, dvoukanálový digitální osciloskop, čítač
Tesla, odrazová kovová deska, laboratorní stojan, parabolický odrýžeč, difrakční mřížka s nastavitelným
počtem šterbin, elektický vozíček s nastavitelnou rychlostí pojezdu, pojezdová lavice
s měřítkem (2 ks), stopky, výsuvné měřítko 50 cm, úhloměr, kabely, sada držákuů pro mikrofony

\section{Teória}
Závislosť rýchlosti zvuku $v_z$ vzduchu od teploty $T$ od nám vyjadruje vzťah
\eq{
v_z = 331.3 \sqrt{1+\frac{T}{273.15}}\,. \lbl{R_0}
}

K výpočtu rýchlosti zvuku $v_z$ pomocou zvukových vĺn, pre priame usporiadanie použijeme vzťah
\eq{
v_z= \frac{s}{t} \,, \lbl{R_1}
}
kde $s$ je vzdialenosť vysielaču a prijímaču signálu, v prípade odrazu signálu od prekážky vzorec transformujeme
\eq{
v_z = \frac{2s}{t}\,, \lbl{R_2}
}
za predpokladu, že vysielač a prijímač je umiestnený v rovnakej vzdialenosti $s$ od odrazovej plochy a $t$ je v oboch prípadoch čas.


Pre výpočet rýchlosti vozíku môžeme použiť vzťah z klasickej mechaniky 
\eq{
v = \frac{s}{t} \,, \lbl{R_6}
}

Pre pozorovanie zmeny pozorovanej frekvencie $f$ pri pohybe vozíku rýchlosťou $v$ pričom zdroj vysiela signál o frekvencií $f_0$ platí podľa Dopplerova javu vzťah
\eq{
f = \frac{v_z\pm v}{v_z} f_0\,, \lbl{R_7}
}
kde $v_z$ je rýchlosť zvuku.

Pre teoretickú závislosť difrakčného obrazca môžeme odvodiť vzťah
\eq[m]{
I = I_0\(\frac{\sin\(\alpha\) \sin\(N\beta\)}{\alpha N \cdot \sin\(\beta\)}\)^2\,,\\
\alpha = \frac{\pi}{\lambda}d\sin\(\theta\)\,,\\
\beta = \frac{\pi}{\lambda}a\sin\(\theta\)\,,
}
kde $d$ je mriežková konštanta a $a$ je šírka štrbiny a vlnová dĺžka $\lambda = "0.0083 m"$.

\subsubsection{Spracovanie chýb merania}

Označme $\mean{t}$ aritmetický priemer nameraných hodnôt $t_i$, a $\Delta t$ hodnotu $\mean{t}-t$, pričom 
\eq{
\mean{t} = \frac{1}{n}\sum_{i=1}^n t_i \,, \lbl{SCH_1}
}  
a chybu aritmetického priemeru 
\eq{
  \sigma_0=\sqrt{\frac{\sum_{i=1}^n \(t_i - \mean{t}\)^2}{n\(n-1\)}}\,, \lbl{SCH_2}
}
pričom $n$ je počet meraní.

\section{Postup merania}
\begin{enumerate}
\item Podľa schémy Obr. 1 \cite{C_1} boli všetky súčiastky pripojené o odkúšané.
\item Pre 3 rôzne uhly bola odmeraná amplitúda napätia na osciloskope.
\item Obvod bol prepojený na Obr. \cite{C_1} a bola domeraná rýchlosť zvuku priamo a odrazom.
\item Mikrofón bol presunutý do aparatúry a odmerané hodnoty napätia pre  1,2,5 štrbín.
\item Náslende bol mikrofón presunutý na vozíček a bol meraný doplerov jav.
\item Pre jazdu vpred a vzad bola odemeraná rýchlosť vozíčku.
\end{enumerate}

\section{Výsledky merania}
\subsection{Úloha 1.}
V tabuľke Tab. \ref{T_1} sú zaznamenané odmerané hodnoty amplitúdy napätie $U$ na rozdiely uhlu prijímača a vysielača $\Delta \sigma$. A vynesené do grafu Obr. \ref{G_1}.

\begin{figure}
% GNUPLOT: LaTeX picture
\setlength{\unitlength}{0.240900pt}
\ifx\plotpoint\undefined\newsavebox{\plotpoint}\fi
\sbox{\plotpoint}{\rule[-0.200pt]{0.400pt}{0.400pt}}%
\begin{picture}(1500,900)(0,0)
\sbox{\plotpoint}{\rule[-0.200pt]{0.400pt}{0.400pt}}%
\put(151.0,131.0){\rule[-0.200pt]{4.818pt}{0.400pt}}
\put(131,131){\makebox(0,0)[r]{ 15}}
\put(1419.0,131.0){\rule[-0.200pt]{4.818pt}{0.400pt}}
\put(151.0,235.0){\rule[-0.200pt]{4.818pt}{0.400pt}}
\put(131,235){\makebox(0,0)[r]{ 20}}
\put(1419.0,235.0){\rule[-0.200pt]{4.818pt}{0.400pt}}
\put(151.0,339.0){\rule[-0.200pt]{4.818pt}{0.400pt}}
\put(131,339){\makebox(0,0)[r]{ 25}}
\put(1419.0,339.0){\rule[-0.200pt]{4.818pt}{0.400pt}}
\put(151.0,443.0){\rule[-0.200pt]{4.818pt}{0.400pt}}
\put(131,443){\makebox(0,0)[r]{ 30}}
\put(1419.0,443.0){\rule[-0.200pt]{4.818pt}{0.400pt}}
\put(151.0,547.0){\rule[-0.200pt]{4.818pt}{0.400pt}}
\put(131,547){\makebox(0,0)[r]{ 35}}
\put(1419.0,547.0){\rule[-0.200pt]{4.818pt}{0.400pt}}
\put(151.0,651.0){\rule[-0.200pt]{4.818pt}{0.400pt}}
\put(131,651){\makebox(0,0)[r]{ 40}}
\put(1419.0,651.0){\rule[-0.200pt]{4.818pt}{0.400pt}}
\put(151.0,755.0){\rule[-0.200pt]{4.818pt}{0.400pt}}
\put(131,755){\makebox(0,0)[r]{ 45}}
\put(1419.0,755.0){\rule[-0.200pt]{4.818pt}{0.400pt}}
\put(151.0,859.0){\rule[-0.200pt]{4.818pt}{0.400pt}}
\put(131,859){\makebox(0,0)[r]{ 50}}
\put(1419.0,859.0){\rule[-0.200pt]{4.818pt}{0.400pt}}
\put(151.0,131.0){\rule[-0.200pt]{0.400pt}{4.818pt}}
\put(151,90){\makebox(0,0){ 195}}
\put(151.0,839.0){\rule[-0.200pt]{0.400pt}{4.818pt}}
\put(294.0,131.0){\rule[-0.200pt]{0.400pt}{4.818pt}}
\put(294,90){\makebox(0,0){ 200}}
\put(294.0,839.0){\rule[-0.200pt]{0.400pt}{4.818pt}}
\put(437.0,131.0){\rule[-0.200pt]{0.400pt}{4.818pt}}
\put(437,90){\makebox(0,0){ 205}}
\put(437.0,839.0){\rule[-0.200pt]{0.400pt}{4.818pt}}
\put(580.0,131.0){\rule[-0.200pt]{0.400pt}{4.818pt}}
\put(580,90){\makebox(0,0){ 210}}
\put(580.0,839.0){\rule[-0.200pt]{0.400pt}{4.818pt}}
\put(723.0,131.0){\rule[-0.200pt]{0.400pt}{4.818pt}}
\put(723,90){\makebox(0,0){ 215}}
\put(723.0,839.0){\rule[-0.200pt]{0.400pt}{4.818pt}}
\put(867.0,131.0){\rule[-0.200pt]{0.400pt}{4.818pt}}
\put(867,90){\makebox(0,0){ 220}}
\put(867.0,839.0){\rule[-0.200pt]{0.400pt}{4.818pt}}
\put(1010.0,131.0){\rule[-0.200pt]{0.400pt}{4.818pt}}
\put(1010,90){\makebox(0,0){ 225}}
\put(1010.0,839.0){\rule[-0.200pt]{0.400pt}{4.818pt}}
\put(1153.0,131.0){\rule[-0.200pt]{0.400pt}{4.818pt}}
\put(1153,90){\makebox(0,0){ 230}}
\put(1153.0,839.0){\rule[-0.200pt]{0.400pt}{4.818pt}}
\put(1296.0,131.0){\rule[-0.200pt]{0.400pt}{4.818pt}}
\put(1296,90){\makebox(0,0){ 235}}
\put(1296.0,839.0){\rule[-0.200pt]{0.400pt}{4.818pt}}
\put(1439.0,131.0){\rule[-0.200pt]{0.400pt}{4.818pt}}
\put(1439,90){\makebox(0,0){ 240}}
\put(1439.0,839.0){\rule[-0.200pt]{0.400pt}{4.818pt}}
\put(151.0,131.0){\rule[-0.200pt]{0.400pt}{175.375pt}}
\put(151.0,131.0){\rule[-0.200pt]{310.279pt}{0.400pt}}
\put(1439.0,131.0){\rule[-0.200pt]{0.400pt}{175.375pt}}
\put(151.0,859.0){\rule[-0.200pt]{310.279pt}{0.400pt}}
\put(30,495){\makebox(0,0){\popi{I}{mA}}}
\put(795,29){\makebox(0,0){\popi{f}{kHz}}}
\put(1279,819){\makebox(0,0)[r]{fit $I = I(f)$}}
\put(1299.0,819.0){\rule[-0.200pt]{24.090pt}{0.400pt}}
\put(208,201){\usebox{\plotpoint}}
\multiput(208.00,201.59)(1.033,0.482){9}{\rule{0.900pt}{0.116pt}}
\multiput(208.00,200.17)(10.132,6.000){2}{\rule{0.450pt}{0.400pt}}
\multiput(220.00,207.59)(0.943,0.482){9}{\rule{0.833pt}{0.116pt}}
\multiput(220.00,206.17)(9.270,6.000){2}{\rule{0.417pt}{0.400pt}}
\multiput(231.00,213.59)(1.033,0.482){9}{\rule{0.900pt}{0.116pt}}
\multiput(231.00,212.17)(10.132,6.000){2}{\rule{0.450pt}{0.400pt}}
\multiput(243.00,219.59)(1.033,0.482){9}{\rule{0.900pt}{0.116pt}}
\multiput(243.00,218.17)(10.132,6.000){2}{\rule{0.450pt}{0.400pt}}
\multiput(255.00,225.59)(0.798,0.485){11}{\rule{0.729pt}{0.117pt}}
\multiput(255.00,224.17)(9.488,7.000){2}{\rule{0.364pt}{0.400pt}}
\multiput(266.00,232.59)(1.033,0.482){9}{\rule{0.900pt}{0.116pt}}
\multiput(266.00,231.17)(10.132,6.000){2}{\rule{0.450pt}{0.400pt}}
\multiput(278.00,238.59)(0.798,0.485){11}{\rule{0.729pt}{0.117pt}}
\multiput(278.00,237.17)(9.488,7.000){2}{\rule{0.364pt}{0.400pt}}
\multiput(289.00,245.59)(0.874,0.485){11}{\rule{0.786pt}{0.117pt}}
\multiput(289.00,244.17)(10.369,7.000){2}{\rule{0.393pt}{0.400pt}}
\multiput(301.00,252.59)(0.798,0.485){11}{\rule{0.729pt}{0.117pt}}
\multiput(301.00,251.17)(9.488,7.000){2}{\rule{0.364pt}{0.400pt}}
\multiput(312.00,259.59)(0.874,0.485){11}{\rule{0.786pt}{0.117pt}}
\multiput(312.00,258.17)(10.369,7.000){2}{\rule{0.393pt}{0.400pt}}
\multiput(324.00,266.59)(0.798,0.485){11}{\rule{0.729pt}{0.117pt}}
\multiput(324.00,265.17)(9.488,7.000){2}{\rule{0.364pt}{0.400pt}}
\multiput(335.00,273.59)(0.758,0.488){13}{\rule{0.700pt}{0.117pt}}
\multiput(335.00,272.17)(10.547,8.000){2}{\rule{0.350pt}{0.400pt}}
\multiput(347.00,281.59)(0.758,0.488){13}{\rule{0.700pt}{0.117pt}}
\multiput(347.00,280.17)(10.547,8.000){2}{\rule{0.350pt}{0.400pt}}
\multiput(359.00,289.59)(0.798,0.485){11}{\rule{0.729pt}{0.117pt}}
\multiput(359.00,288.17)(9.488,7.000){2}{\rule{0.364pt}{0.400pt}}
\multiput(370.00,296.59)(0.758,0.488){13}{\rule{0.700pt}{0.117pt}}
\multiput(370.00,295.17)(10.547,8.000){2}{\rule{0.350pt}{0.400pt}}
\multiput(382.00,304.59)(0.611,0.489){15}{\rule{0.589pt}{0.118pt}}
\multiput(382.00,303.17)(9.778,9.000){2}{\rule{0.294pt}{0.400pt}}
\multiput(393.00,313.59)(0.758,0.488){13}{\rule{0.700pt}{0.117pt}}
\multiput(393.00,312.17)(10.547,8.000){2}{\rule{0.350pt}{0.400pt}}
\multiput(405.00,321.59)(0.692,0.488){13}{\rule{0.650pt}{0.117pt}}
\multiput(405.00,320.17)(9.651,8.000){2}{\rule{0.325pt}{0.400pt}}
\multiput(416.00,329.59)(0.669,0.489){15}{\rule{0.633pt}{0.118pt}}
\multiput(416.00,328.17)(10.685,9.000){2}{\rule{0.317pt}{0.400pt}}
\multiput(428.00,338.59)(0.669,0.489){15}{\rule{0.633pt}{0.118pt}}
\multiput(428.00,337.17)(10.685,9.000){2}{\rule{0.317pt}{0.400pt}}
\multiput(440.00,347.59)(0.611,0.489){15}{\rule{0.589pt}{0.118pt}}
\multiput(440.00,346.17)(9.778,9.000){2}{\rule{0.294pt}{0.400pt}}
\multiput(451.00,356.59)(0.669,0.489){15}{\rule{0.633pt}{0.118pt}}
\multiput(451.00,355.17)(10.685,9.000){2}{\rule{0.317pt}{0.400pt}}
\multiput(463.00,365.59)(0.611,0.489){15}{\rule{0.589pt}{0.118pt}}
\multiput(463.00,364.17)(9.778,9.000){2}{\rule{0.294pt}{0.400pt}}
\multiput(474.00,374.59)(0.669,0.489){15}{\rule{0.633pt}{0.118pt}}
\multiput(474.00,373.17)(10.685,9.000){2}{\rule{0.317pt}{0.400pt}}
\multiput(486.00,383.58)(0.547,0.491){17}{\rule{0.540pt}{0.118pt}}
\multiput(486.00,382.17)(9.879,10.000){2}{\rule{0.270pt}{0.400pt}}
\multiput(497.00,393.59)(0.669,0.489){15}{\rule{0.633pt}{0.118pt}}
\multiput(497.00,392.17)(10.685,9.000){2}{\rule{0.317pt}{0.400pt}}
\multiput(509.00,402.58)(0.547,0.491){17}{\rule{0.540pt}{0.118pt}}
\multiput(509.00,401.17)(9.879,10.000){2}{\rule{0.270pt}{0.400pt}}
\multiput(520.00,412.59)(0.669,0.489){15}{\rule{0.633pt}{0.118pt}}
\multiput(520.00,411.17)(10.685,9.000){2}{\rule{0.317pt}{0.400pt}}
\multiput(532.00,421.58)(0.600,0.491){17}{\rule{0.580pt}{0.118pt}}
\multiput(532.00,420.17)(10.796,10.000){2}{\rule{0.290pt}{0.400pt}}
\multiput(544.00,431.58)(0.547,0.491){17}{\rule{0.540pt}{0.118pt}}
\multiput(544.00,430.17)(9.879,10.000){2}{\rule{0.270pt}{0.400pt}}
\multiput(555.00,441.58)(0.600,0.491){17}{\rule{0.580pt}{0.118pt}}
\multiput(555.00,440.17)(10.796,10.000){2}{\rule{0.290pt}{0.400pt}}
\multiput(567.00,451.59)(0.611,0.489){15}{\rule{0.589pt}{0.118pt}}
\multiput(567.00,450.17)(9.778,9.000){2}{\rule{0.294pt}{0.400pt}}
\multiput(578.00,460.58)(0.600,0.491){17}{\rule{0.580pt}{0.118pt}}
\multiput(578.00,459.17)(10.796,10.000){2}{\rule{0.290pt}{0.400pt}}
\multiput(590.00,470.59)(0.611,0.489){15}{\rule{0.589pt}{0.118pt}}
\multiput(590.00,469.17)(9.778,9.000){2}{\rule{0.294pt}{0.400pt}}
\multiput(601.00,479.58)(0.600,0.491){17}{\rule{0.580pt}{0.118pt}}
\multiput(601.00,478.17)(10.796,10.000){2}{\rule{0.290pt}{0.400pt}}
\multiput(613.00,489.59)(0.669,0.489){15}{\rule{0.633pt}{0.118pt}}
\multiput(613.00,488.17)(10.685,9.000){2}{\rule{0.317pt}{0.400pt}}
\multiput(625.00,498.59)(0.611,0.489){15}{\rule{0.589pt}{0.118pt}}
\multiput(625.00,497.17)(9.778,9.000){2}{\rule{0.294pt}{0.400pt}}
\multiput(636.00,507.59)(0.758,0.488){13}{\rule{0.700pt}{0.117pt}}
\multiput(636.00,506.17)(10.547,8.000){2}{\rule{0.350pt}{0.400pt}}
\multiput(648.00,515.59)(0.611,0.489){15}{\rule{0.589pt}{0.118pt}}
\multiput(648.00,514.17)(9.778,9.000){2}{\rule{0.294pt}{0.400pt}}
\multiput(659.00,524.59)(0.758,0.488){13}{\rule{0.700pt}{0.117pt}}
\multiput(659.00,523.17)(10.547,8.000){2}{\rule{0.350pt}{0.400pt}}
\multiput(671.00,532.59)(0.798,0.485){11}{\rule{0.729pt}{0.117pt}}
\multiput(671.00,531.17)(9.488,7.000){2}{\rule{0.364pt}{0.400pt}}
\multiput(682.00,539.59)(0.874,0.485){11}{\rule{0.786pt}{0.117pt}}
\multiput(682.00,538.17)(10.369,7.000){2}{\rule{0.393pt}{0.400pt}}
\multiput(694.00,546.59)(0.874,0.485){11}{\rule{0.786pt}{0.117pt}}
\multiput(694.00,545.17)(10.369,7.000){2}{\rule{0.393pt}{0.400pt}}
\multiput(706.00,553.59)(0.943,0.482){9}{\rule{0.833pt}{0.116pt}}
\multiput(706.00,552.17)(9.270,6.000){2}{\rule{0.417pt}{0.400pt}}
\multiput(717.00,559.59)(1.033,0.482){9}{\rule{0.900pt}{0.116pt}}
\multiput(717.00,558.17)(10.132,6.000){2}{\rule{0.450pt}{0.400pt}}
\multiput(729.00,565.59)(1.155,0.477){7}{\rule{0.980pt}{0.115pt}}
\multiput(729.00,564.17)(8.966,5.000){2}{\rule{0.490pt}{0.400pt}}
\multiput(740.00,570.60)(1.651,0.468){5}{\rule{1.300pt}{0.113pt}}
\multiput(740.00,569.17)(9.302,4.000){2}{\rule{0.650pt}{0.400pt}}
\multiput(752.00,574.61)(2.248,0.447){3}{\rule{1.567pt}{0.108pt}}
\multiput(752.00,573.17)(7.748,3.000){2}{\rule{0.783pt}{0.400pt}}
\multiput(763.00,577.61)(2.472,0.447){3}{\rule{1.700pt}{0.108pt}}
\multiput(763.00,576.17)(8.472,3.000){2}{\rule{0.850pt}{0.400pt}}
\put(775,580.17){\rule{2.300pt}{0.400pt}}
\multiput(775.00,579.17)(6.226,2.000){2}{\rule{1.150pt}{0.400pt}}
\put(786,582.17){\rule{2.500pt}{0.400pt}}
\multiput(786.00,581.17)(6.811,2.000){2}{\rule{1.250pt}{0.400pt}}
\put(821,582.67){\rule{2.891pt}{0.400pt}}
\multiput(821.00,583.17)(6.000,-1.000){2}{\rule{1.445pt}{0.400pt}}
\put(833,581.17){\rule{2.300pt}{0.400pt}}
\multiput(833.00,582.17)(6.226,-2.000){2}{\rule{1.150pt}{0.400pt}}
\put(844,579.17){\rule{2.500pt}{0.400pt}}
\multiput(844.00,580.17)(6.811,-2.000){2}{\rule{1.250pt}{0.400pt}}
\multiput(856.00,577.95)(2.248,-0.447){3}{\rule{1.567pt}{0.108pt}}
\multiput(856.00,578.17)(7.748,-3.000){2}{\rule{0.783pt}{0.400pt}}
\multiput(867.00,574.94)(1.651,-0.468){5}{\rule{1.300pt}{0.113pt}}
\multiput(867.00,575.17)(9.302,-4.000){2}{\rule{0.650pt}{0.400pt}}
\multiput(879.00,570.94)(1.651,-0.468){5}{\rule{1.300pt}{0.113pt}}
\multiput(879.00,571.17)(9.302,-4.000){2}{\rule{0.650pt}{0.400pt}}
\multiput(891.00,566.93)(1.155,-0.477){7}{\rule{0.980pt}{0.115pt}}
\multiput(891.00,567.17)(8.966,-5.000){2}{\rule{0.490pt}{0.400pt}}
\multiput(902.00,561.93)(1.033,-0.482){9}{\rule{0.900pt}{0.116pt}}
\multiput(902.00,562.17)(10.132,-6.000){2}{\rule{0.450pt}{0.400pt}}
\multiput(914.00,555.93)(0.943,-0.482){9}{\rule{0.833pt}{0.116pt}}
\multiput(914.00,556.17)(9.270,-6.000){2}{\rule{0.417pt}{0.400pt}}
\multiput(925.00,549.93)(1.033,-0.482){9}{\rule{0.900pt}{0.116pt}}
\multiput(925.00,550.17)(10.132,-6.000){2}{\rule{0.450pt}{0.400pt}}
\multiput(937.00,543.93)(0.798,-0.485){11}{\rule{0.729pt}{0.117pt}}
\multiput(937.00,544.17)(9.488,-7.000){2}{\rule{0.364pt}{0.400pt}}
\multiput(948.00,536.93)(0.758,-0.488){13}{\rule{0.700pt}{0.117pt}}
\multiput(948.00,537.17)(10.547,-8.000){2}{\rule{0.350pt}{0.400pt}}
\multiput(960.00,528.93)(0.874,-0.485){11}{\rule{0.786pt}{0.117pt}}
\multiput(960.00,529.17)(10.369,-7.000){2}{\rule{0.393pt}{0.400pt}}
\multiput(972.00,521.93)(0.692,-0.488){13}{\rule{0.650pt}{0.117pt}}
\multiput(972.00,522.17)(9.651,-8.000){2}{\rule{0.325pt}{0.400pt}}
\multiput(983.00,513.93)(0.669,-0.489){15}{\rule{0.633pt}{0.118pt}}
\multiput(983.00,514.17)(10.685,-9.000){2}{\rule{0.317pt}{0.400pt}}
\multiput(995.00,504.93)(0.692,-0.488){13}{\rule{0.650pt}{0.117pt}}
\multiput(995.00,505.17)(9.651,-8.000){2}{\rule{0.325pt}{0.400pt}}
\multiput(1006.00,496.93)(0.669,-0.489){15}{\rule{0.633pt}{0.118pt}}
\multiput(1006.00,497.17)(10.685,-9.000){2}{\rule{0.317pt}{0.400pt}}
\multiput(1018.00,487.93)(0.692,-0.488){13}{\rule{0.650pt}{0.117pt}}
\multiput(1018.00,488.17)(9.651,-8.000){2}{\rule{0.325pt}{0.400pt}}
\multiput(1029.00,479.93)(0.669,-0.489){15}{\rule{0.633pt}{0.118pt}}
\multiput(1029.00,480.17)(10.685,-9.000){2}{\rule{0.317pt}{0.400pt}}
\multiput(1041.00,470.93)(0.611,-0.489){15}{\rule{0.589pt}{0.118pt}}
\multiput(1041.00,471.17)(9.778,-9.000){2}{\rule{0.294pt}{0.400pt}}
\multiput(1052.00,461.93)(0.669,-0.489){15}{\rule{0.633pt}{0.118pt}}
\multiput(1052.00,462.17)(10.685,-9.000){2}{\rule{0.317pt}{0.400pt}}
\multiput(1064.00,452.93)(0.669,-0.489){15}{\rule{0.633pt}{0.118pt}}
\multiput(1064.00,453.17)(10.685,-9.000){2}{\rule{0.317pt}{0.400pt}}
\multiput(1076.00,443.93)(0.611,-0.489){15}{\rule{0.589pt}{0.118pt}}
\multiput(1076.00,444.17)(9.778,-9.000){2}{\rule{0.294pt}{0.400pt}}
\multiput(1087.00,434.93)(0.669,-0.489){15}{\rule{0.633pt}{0.118pt}}
\multiput(1087.00,435.17)(10.685,-9.000){2}{\rule{0.317pt}{0.400pt}}
\multiput(1099.00,425.93)(0.692,-0.488){13}{\rule{0.650pt}{0.117pt}}
\multiput(1099.00,426.17)(9.651,-8.000){2}{\rule{0.325pt}{0.400pt}}
\multiput(1110.00,417.93)(0.669,-0.489){15}{\rule{0.633pt}{0.118pt}}
\multiput(1110.00,418.17)(10.685,-9.000){2}{\rule{0.317pt}{0.400pt}}
\multiput(1122.00,408.93)(0.611,-0.489){15}{\rule{0.589pt}{0.118pt}}
\multiput(1122.00,409.17)(9.778,-9.000){2}{\rule{0.294pt}{0.400pt}}
\multiput(1133.00,399.93)(0.669,-0.489){15}{\rule{0.633pt}{0.118pt}}
\multiput(1133.00,400.17)(10.685,-9.000){2}{\rule{0.317pt}{0.400pt}}
\multiput(1145.00,390.93)(0.758,-0.488){13}{\rule{0.700pt}{0.117pt}}
\multiput(1145.00,391.17)(10.547,-8.000){2}{\rule{0.350pt}{0.400pt}}
\multiput(1157.00,382.93)(0.692,-0.488){13}{\rule{0.650pt}{0.117pt}}
\multiput(1157.00,383.17)(9.651,-8.000){2}{\rule{0.325pt}{0.400pt}}
\multiput(1168.00,374.93)(0.669,-0.489){15}{\rule{0.633pt}{0.118pt}}
\multiput(1168.00,375.17)(10.685,-9.000){2}{\rule{0.317pt}{0.400pt}}
\multiput(1180.00,365.93)(0.692,-0.488){13}{\rule{0.650pt}{0.117pt}}
\multiput(1180.00,366.17)(9.651,-8.000){2}{\rule{0.325pt}{0.400pt}}
\multiput(1191.00,357.93)(0.758,-0.488){13}{\rule{0.700pt}{0.117pt}}
\multiput(1191.00,358.17)(10.547,-8.000){2}{\rule{0.350pt}{0.400pt}}
\multiput(1203.00,349.93)(0.692,-0.488){13}{\rule{0.650pt}{0.117pt}}
\multiput(1203.00,350.17)(9.651,-8.000){2}{\rule{0.325pt}{0.400pt}}
\multiput(1214.00,341.93)(0.874,-0.485){11}{\rule{0.786pt}{0.117pt}}
\multiput(1214.00,342.17)(10.369,-7.000){2}{\rule{0.393pt}{0.400pt}}
\multiput(1226.00,334.93)(0.692,-0.488){13}{\rule{0.650pt}{0.117pt}}
\multiput(1226.00,335.17)(9.651,-8.000){2}{\rule{0.325pt}{0.400pt}}
\multiput(1237.00,326.93)(0.874,-0.485){11}{\rule{0.786pt}{0.117pt}}
\multiput(1237.00,327.17)(10.369,-7.000){2}{\rule{0.393pt}{0.400pt}}
\multiput(1249.00,319.93)(0.874,-0.485){11}{\rule{0.786pt}{0.117pt}}
\multiput(1249.00,320.17)(10.369,-7.000){2}{\rule{0.393pt}{0.400pt}}
\multiput(1261.00,312.93)(0.692,-0.488){13}{\rule{0.650pt}{0.117pt}}
\multiput(1261.00,313.17)(9.651,-8.000){2}{\rule{0.325pt}{0.400pt}}
\multiput(1272.00,304.93)(0.874,-0.485){11}{\rule{0.786pt}{0.117pt}}
\multiput(1272.00,305.17)(10.369,-7.000){2}{\rule{0.393pt}{0.400pt}}
\multiput(1284.00,297.93)(0.943,-0.482){9}{\rule{0.833pt}{0.116pt}}
\multiput(1284.00,298.17)(9.270,-6.000){2}{\rule{0.417pt}{0.400pt}}
\multiput(1295.00,291.93)(0.874,-0.485){11}{\rule{0.786pt}{0.117pt}}
\multiput(1295.00,292.17)(10.369,-7.000){2}{\rule{0.393pt}{0.400pt}}
\multiput(1307.00,284.93)(0.798,-0.485){11}{\rule{0.729pt}{0.117pt}}
\multiput(1307.00,285.17)(9.488,-7.000){2}{\rule{0.364pt}{0.400pt}}
\multiput(1318.00,277.93)(1.033,-0.482){9}{\rule{0.900pt}{0.116pt}}
\multiput(1318.00,278.17)(10.132,-6.000){2}{\rule{0.450pt}{0.400pt}}
\multiput(1330.00,271.93)(1.033,-0.482){9}{\rule{0.900pt}{0.116pt}}
\multiput(1330.00,272.17)(10.132,-6.000){2}{\rule{0.450pt}{0.400pt}}
\multiput(1342.00,265.93)(0.943,-0.482){9}{\rule{0.833pt}{0.116pt}}
\multiput(1342.00,266.17)(9.270,-6.000){2}{\rule{0.417pt}{0.400pt}}
\put(798.0,584.0){\rule[-0.200pt]{5.541pt}{0.400pt}}
\put(1279,778){\makebox(0,0)[r]{Namerané hodnoty pre cievku bez jadra}}
\put(781,609){\makebox(0,0){$\times$}}
\put(1067,485){\makebox(0,0){$\times$}}
\put(1353,235){\makebox(0,0){$\times$}}
\put(1210,339){\makebox(0,0){$\times$}}
\put(924,547){\makebox(0,0){$\times$}}
\put(638,485){\makebox(0,0){$\times$}}
\put(494,401){\makebox(0,0){$\times$}}
\put(351,297){\makebox(0,0){$\times$}}
\put(208,214){\makebox(0,0){$\times$}}
\put(294,256){\makebox(0,0){$\times$}}
\put(409,318){\makebox(0,0){$\times$}}
\put(580,464){\makebox(0,0){$\times$}}
\put(666,485){\makebox(0,0){$\times$}}
\put(723,568){\makebox(0,0){$\times$}}
\put(752,589){\makebox(0,0){$\times$}}
\put(809,589){\makebox(0,0){$\times$}}
\put(867,568){\makebox(0,0){$\times$}}
\put(1010,485){\makebox(0,0){$\times$}}
\put(1153,381){\makebox(0,0){$\times$}}
\put(1296,297){\makebox(0,0){$\times$}}
\put(1349,778){\makebox(0,0){$\times$}}
\put(151.0,131.0){\rule[-0.200pt]{0.400pt}{175.375pt}}
\put(151.0,131.0){\rule[-0.200pt]{310.279pt}{0.400pt}}
\put(1439.0,131.0){\rule[-0.200pt]{0.400pt}{175.375pt}}
\put(151.0,859.0){\rule[-0.200pt]{310.279pt}{0.400pt}}
\end{picture}

\caption{Závislosť amplitúdy napätia $U$ od rozdielu uhlu $\Delta \sigma$, pri stálom umiestnení vysielaču s dopadovým uhlom $\sigma_v = "45 \deg"$}  \label{G_1}
\end{figure}

\begin{table}[h]
\begin{center}
\begin{tabular}{| c | c | c | c |}
\hline
\popi{U}{mV} & \popi{\varphi_v}{\deg} & \popi{\varphi_p}{\deg} & \popi{\Delta\varphi_v}{\deg}\\
\hline
$36.4$&$40$&$130$&$90$\\
$34.5$&$40$&$130$&$90$\\
$35.9$&$40$&$130$&$90$\\
$36.1$&$40$&$130$&$90$\\
$35.3$&$40$&$130$&$90$\\
$25.0$&$40$&$110$&$70$\\
$24.2$&$40$&$110$&$70$\\
$24.7$&$40$&$110$&$70$\\
$22.7$&$40$&$110$&$70$\\
$22.7$&$40$&$110$&$70$\\
$13.8$&$40$&$90$&$50$\\
$14.0$&$40$&$90$&$50$\\
$14.8$&$40$&$90$&$50$\\
$12.8$&$40$&$90$&$50$\\
$14.4$&$40$&$90$&$50$\\
\hline

\end{tabular}
\caption{Namerané hodnoty napätia $U$ na uhle prijímača $\varphi_p$, vysielača $\varphi_v$} \label{T_1}
\end{center}
\end{table}

\subsection{Úloha 2.}


\begin{table}[h]
\begin{center}
\begin{tabular}{| c | c |}
\hline
\popi{T}{\nu s} & \popi{x}{cm} \\
\hline
$100$&$0$ \\
$330$&$8.5$ \\
$280$&$6.7$ \\
$240$&$5.5$ \\
$210$&$5$ \\
$190$&$4$ \\
$170$&$3$ \\
$440$&$11$ \\
$460$&$12$ \\
$680$&$20$ \\
\hline
\end{tabular}
\caption{Namerané hodnoty vzdialenosti vysielaču a prijímaču $x$ a doba oneskorenia signálu $T$ } \label{T_2}
\end{center}
\end{table}

V tabuľke Tab. \ref{T_2}. sú zaznamenané hodnoty vzdialenosti vysielaču 
a prijímaču $x$ k dobe oneskorenia signálu na osciloskope $t$ pre usporiadanie na priamo.
Z týchto hodnôt bola vypočítaná podľa vzťahov \ref{SCH_1} a \ref{R_1} 
rýchlosť zvuku $v_v="\(384.9\pm48.5\) m\cdot s^{-1}"$

\subsection{Úloha 3}

\begin{table}[h]
\begin{center}
\begin{tabular}{| c | c |}
\hline
\popi{T}{\nu s} & \popi{x}{cm} \\
\hline
$9$&$640$ \\
$3.2$&$740$ \\
$37.2$&$2300$ \\
$32$&$2000$ \\
$28$&$1700$ \\
$25$&$1600$ \\
$23$&$1400$ \\
$20$&$1200$ \\
$18$&$1100$ \\
$98$&$6000$ \\
\hline
\end{tabular}
\caption{Namerané hodnoty vzdialenosti prekážky a vysielači $x$ a doba oneskorenia signálu $T$ } \label{T_3}
\end{center}
\end{table}

V tabuľke Tab. \ref{T_3}. sú zaznamenané hodnoty vzdialenosti vysielaču 
a prijímaču $l$ k dobe oneskorenia signálu na osciloskope $t$ pre meranie 
s odrazom od prekážky.
Z týchto hodnôt bola vypočítaná podľa vzťahov \ref{SCH_1} a \ref{R_1} 
rýchlosť zvuku $v_z="\(320.1\pm78.2\) m\cdot s^{-1}"$

\subsection{Úloha 4.}
Do grafov Obr. \ref{G_D1}, Obr. \ref{G_D2} a Obr. \ref{G_D5}. Boli postupne namerané dáta pre jedno- dv-e a päť-štrbínové difrakčné obrazce.

\begin{figure}
% GNUPLOT: LaTeX picture
\setlength{\unitlength}{0.240900pt}
\ifx\plotpoint\undefined\newsavebox{\plotpoint}\fi
\begin{picture}(1500,900)(0,0)
\sbox{\plotpoint}{\rule[-0.200pt]{0.400pt}{0.400pt}}%
\put(191.0,131.0){\rule[-0.200pt]{4.818pt}{0.400pt}}
\put(171,131){\makebox(0,0)[r]{ 0.75}}
\put(1419.0,131.0){\rule[-0.200pt]{4.818pt}{0.400pt}}
\put(191.0,277.0){\rule[-0.200pt]{4.818pt}{0.400pt}}
\put(171,277){\makebox(0,0)[r]{ 0.8}}
\put(1419.0,277.0){\rule[-0.200pt]{4.818pt}{0.400pt}}
\put(191.0,422.0){\rule[-0.200pt]{4.818pt}{0.400pt}}
\put(171,422){\makebox(0,0)[r]{ 0.85}}
\put(1419.0,422.0){\rule[-0.200pt]{4.818pt}{0.400pt}}
\put(191.0,568.0){\rule[-0.200pt]{4.818pt}{0.400pt}}
\put(171,568){\makebox(0,0)[r]{ 0.9}}
\put(1419.0,568.0){\rule[-0.200pt]{4.818pt}{0.400pt}}
\put(191.0,713.0){\rule[-0.200pt]{4.818pt}{0.400pt}}
\put(171,713){\makebox(0,0)[r]{ 0.95}}
\put(1419.0,713.0){\rule[-0.200pt]{4.818pt}{0.400pt}}
\put(191.0,859.0){\rule[-0.200pt]{4.818pt}{0.400pt}}
\put(171,859){\makebox(0,0)[r]{ 1}}
\put(1419.0,859.0){\rule[-0.200pt]{4.818pt}{0.400pt}}
\put(191.0,131.0){\rule[-0.200pt]{0.400pt}{4.818pt}}
\put(191,90){\makebox(0,0){-0.2}}
\put(191.0,839.0){\rule[-0.200pt]{0.400pt}{4.818pt}}
\put(347.0,131.0){\rule[-0.200pt]{0.400pt}{4.818pt}}
\put(347,90){\makebox(0,0){-0.15}}
\put(347.0,839.0){\rule[-0.200pt]{0.400pt}{4.818pt}}
\put(503.0,131.0){\rule[-0.200pt]{0.400pt}{4.818pt}}
\put(503,90){\makebox(0,0){-0.1}}
\put(503.0,839.0){\rule[-0.200pt]{0.400pt}{4.818pt}}
\put(659.0,131.0){\rule[-0.200pt]{0.400pt}{4.818pt}}
\put(659,90){\makebox(0,0){-0.05}}
\put(659.0,839.0){\rule[-0.200pt]{0.400pt}{4.818pt}}
\put(815.0,131.0){\rule[-0.200pt]{0.400pt}{4.818pt}}
\put(815,90){\makebox(0,0){ 0}}
\put(815.0,839.0){\rule[-0.200pt]{0.400pt}{4.818pt}}
\put(971.0,131.0){\rule[-0.200pt]{0.400pt}{4.818pt}}
\put(971,90){\makebox(0,0){ 0.05}}
\put(971.0,839.0){\rule[-0.200pt]{0.400pt}{4.818pt}}
\put(1127.0,131.0){\rule[-0.200pt]{0.400pt}{4.818pt}}
\put(1127,90){\makebox(0,0){ 0.1}}
\put(1127.0,839.0){\rule[-0.200pt]{0.400pt}{4.818pt}}
\put(1283.0,131.0){\rule[-0.200pt]{0.400pt}{4.818pt}}
\put(1283,90){\makebox(0,0){ 0.15}}
\put(1283.0,839.0){\rule[-0.200pt]{0.400pt}{4.818pt}}
\put(1439.0,131.0){\rule[-0.200pt]{0.400pt}{4.818pt}}
\put(1439,90){\makebox(0,0){ 0.2}}
\put(1439.0,839.0){\rule[-0.200pt]{0.400pt}{4.818pt}}
\put(191.0,131.0){\rule[-0.200pt]{0.400pt}{175.375pt}}
\put(191.0,131.0){\rule[-0.200pt]{300.643pt}{0.400pt}}
\put(1439.0,131.0){\rule[-0.200pt]{0.400pt}{175.375pt}}
\put(191.0,859.0){\rule[-0.200pt]{300.643pt}{0.400pt}}
\put(30,495){\makebox(0,0){\popi{U/U_0}{-}}}
\put(815,29){\makebox(0,0){\popi{\theta}{rad}}}
\put(1279,819){\makebox(0,0)[r]{Teoretická zavislosť}}
\put(1299.0,819.0){\rule[-0.200pt]{24.090pt}{0.400pt}}
\put(191,350){\usebox{\plotpoint}}
\put(191.0,350.0){\usebox{\plotpoint}}
\put(191.0,351.0){\usebox{\plotpoint}}
\put(192.0,351.0){\rule[-0.200pt]{0.400pt}{0.482pt}}
\put(192.0,353.0){\usebox{\plotpoint}}
\put(193.0,353.0){\usebox{\plotpoint}}
\put(193.0,354.0){\usebox{\plotpoint}}
\put(194.0,354.0){\rule[-0.200pt]{0.400pt}{0.482pt}}
\put(194.0,356.0){\usebox{\plotpoint}}
\put(195.0,356.0){\usebox{\plotpoint}}
\put(195.0,357.0){\usebox{\plotpoint}}
\put(196.0,357.0){\rule[-0.200pt]{0.400pt}{0.482pt}}
\put(196.0,359.0){\usebox{\plotpoint}}
\put(197.0,359.0){\usebox{\plotpoint}}
\put(197.0,360.0){\usebox{\plotpoint}}
\put(198,360.67){\rule{0.241pt}{0.400pt}}
\multiput(198.00,360.17)(0.500,1.000){2}{\rule{0.120pt}{0.400pt}}
\put(198.0,360.0){\usebox{\plotpoint}}
\put(199,362){\usebox{\plotpoint}}
\put(199,362){\usebox{\plotpoint}}
\put(199,362){\usebox{\plotpoint}}
\put(199,362){\usebox{\plotpoint}}
\put(199.0,362.0){\usebox{\plotpoint}}
\put(199.0,363.0){\usebox{\plotpoint}}
\put(200,363.67){\rule{0.241pt}{0.400pt}}
\multiput(200.00,363.17)(0.500,1.000){2}{\rule{0.120pt}{0.400pt}}
\put(200.0,363.0){\usebox{\plotpoint}}
\put(201,365){\usebox{\plotpoint}}
\put(201,365){\usebox{\plotpoint}}
\put(201,365){\usebox{\plotpoint}}
\put(201,365){\usebox{\plotpoint}}
\put(201.0,365.0){\usebox{\plotpoint}}
\put(201.0,366.0){\usebox{\plotpoint}}
\put(202,366.67){\rule{0.241pt}{0.400pt}}
\multiput(202.00,366.17)(0.500,1.000){2}{\rule{0.120pt}{0.400pt}}
\put(202.0,366.0){\usebox{\plotpoint}}
\put(203,368){\usebox{\plotpoint}}
\put(203,368){\usebox{\plotpoint}}
\put(203,368){\usebox{\plotpoint}}
\put(203,368){\usebox{\plotpoint}}
\put(203,368){\usebox{\plotpoint}}
\put(203.0,368.0){\usebox{\plotpoint}}
\put(203.0,369.0){\usebox{\plotpoint}}
\put(204.0,369.0){\usebox{\plotpoint}}
\put(204.0,370.0){\usebox{\plotpoint}}
\put(205.0,370.0){\rule[-0.200pt]{0.400pt}{0.482pt}}
\put(205.0,372.0){\usebox{\plotpoint}}
\put(206.0,372.0){\usebox{\plotpoint}}
\put(206.0,373.0){\usebox{\plotpoint}}
\put(207.0,373.0){\rule[-0.200pt]{0.400pt}{0.482pt}}
\put(207.0,375.0){\usebox{\plotpoint}}
\put(208.0,375.0){\usebox{\plotpoint}}
\put(208.0,376.0){\usebox{\plotpoint}}
\put(209.0,376.0){\rule[-0.200pt]{0.400pt}{0.482pt}}
\put(209.0,378.0){\usebox{\plotpoint}}
\put(210.0,378.0){\usebox{\plotpoint}}
\put(210.0,379.0){\usebox{\plotpoint}}
\put(211,379.67){\rule{0.241pt}{0.400pt}}
\multiput(211.00,379.17)(0.500,1.000){2}{\rule{0.120pt}{0.400pt}}
\put(211.0,379.0){\usebox{\plotpoint}}
\put(212,381){\usebox{\plotpoint}}
\put(212,381){\usebox{\plotpoint}}
\put(212,381){\usebox{\plotpoint}}
\put(212,381){\usebox{\plotpoint}}
\put(212.0,381.0){\usebox{\plotpoint}}
\put(212.0,382.0){\usebox{\plotpoint}}
\put(213,382.67){\rule{0.241pt}{0.400pt}}
\multiput(213.00,382.17)(0.500,1.000){2}{\rule{0.120pt}{0.400pt}}
\put(213.0,382.0){\usebox{\plotpoint}}
\put(214,384){\usebox{\plotpoint}}
\put(214,384){\usebox{\plotpoint}}
\put(214,384){\usebox{\plotpoint}}
\put(214,384){\usebox{\plotpoint}}
\put(214,384){\usebox{\plotpoint}}
\put(214.0,384.0){\usebox{\plotpoint}}
\put(214.0,385.0){\usebox{\plotpoint}}
\put(215.0,385.0){\usebox{\plotpoint}}
\put(215.0,386.0){\usebox{\plotpoint}}
\put(216.0,386.0){\rule[-0.200pt]{0.400pt}{0.482pt}}
\put(216.0,388.0){\usebox{\plotpoint}}
\put(217.0,388.0){\usebox{\plotpoint}}
\put(217.0,389.0){\usebox{\plotpoint}}
\put(218.0,389.0){\rule[-0.200pt]{0.400pt}{0.482pt}}
\put(218.0,391.0){\usebox{\plotpoint}}
\put(219.0,391.0){\usebox{\plotpoint}}
\put(219.0,392.0){\usebox{\plotpoint}}
\put(220.0,392.0){\rule[-0.200pt]{0.400pt}{0.482pt}}
\put(220.0,394.0){\usebox{\plotpoint}}
\put(221.0,394.0){\usebox{\plotpoint}}
\put(221.0,395.0){\usebox{\plotpoint}}
\put(222,395.67){\rule{0.241pt}{0.400pt}}
\multiput(222.00,395.17)(0.500,1.000){2}{\rule{0.120pt}{0.400pt}}
\put(222.0,395.0){\usebox{\plotpoint}}
\put(223,397){\usebox{\plotpoint}}
\put(223,397){\usebox{\plotpoint}}
\put(223,397){\usebox{\plotpoint}}
\put(223,397){\usebox{\plotpoint}}
\put(223,397){\usebox{\plotpoint}}
\put(223.0,397.0){\usebox{\plotpoint}}
\put(223.0,398.0){\usebox{\plotpoint}}
\put(224.0,398.0){\usebox{\plotpoint}}
\put(224.0,399.0){\usebox{\plotpoint}}
\put(225.0,399.0){\rule[-0.200pt]{0.400pt}{0.482pt}}
\put(225.0,401.0){\usebox{\plotpoint}}
\put(226.0,401.0){\usebox{\plotpoint}}
\put(226.0,402.0){\usebox{\plotpoint}}
\put(227.0,402.0){\rule[-0.200pt]{0.400pt}{0.482pt}}
\put(227.0,404.0){\usebox{\plotpoint}}
\put(228.0,404.0){\usebox{\plotpoint}}
\put(228.0,405.0){\usebox{\plotpoint}}
\put(229,405.67){\rule{0.241pt}{0.400pt}}
\multiput(229.00,405.17)(0.500,1.000){2}{\rule{0.120pt}{0.400pt}}
\put(229.0,405.0){\usebox{\plotpoint}}
\put(230,407){\usebox{\plotpoint}}
\put(230,407){\usebox{\plotpoint}}
\put(230,407){\usebox{\plotpoint}}
\put(230,407){\usebox{\plotpoint}}
\put(230,407){\usebox{\plotpoint}}
\put(230.0,407.0){\usebox{\plotpoint}}
\put(230.0,408.0){\usebox{\plotpoint}}
\put(231.0,408.0){\usebox{\plotpoint}}
\put(231.0,409.0){\usebox{\plotpoint}}
\put(232.0,409.0){\rule[-0.200pt]{0.400pt}{0.482pt}}
\put(232.0,411.0){\usebox{\plotpoint}}
\put(233.0,411.0){\usebox{\plotpoint}}
\put(233.0,412.0){\usebox{\plotpoint}}
\put(234,412.67){\rule{0.241pt}{0.400pt}}
\multiput(234.00,412.17)(0.500,1.000){2}{\rule{0.120pt}{0.400pt}}
\put(234.0,412.0){\usebox{\plotpoint}}
\put(235,414){\usebox{\plotpoint}}
\put(235,414){\usebox{\plotpoint}}
\put(235,414){\usebox{\plotpoint}}
\put(235,414){\usebox{\plotpoint}}
\put(235.0,414.0){\usebox{\plotpoint}}
\put(235.0,415.0){\usebox{\plotpoint}}
\put(236,415.67){\rule{0.241pt}{0.400pt}}
\multiput(236.00,415.17)(0.500,1.000){2}{\rule{0.120pt}{0.400pt}}
\put(236.0,415.0){\usebox{\plotpoint}}
\put(237,417){\usebox{\plotpoint}}
\put(237,417){\usebox{\plotpoint}}
\put(237,417){\usebox{\plotpoint}}
\put(237,417){\usebox{\plotpoint}}
\put(237,417){\usebox{\plotpoint}}
\put(237.0,417.0){\usebox{\plotpoint}}
\put(237.0,418.0){\usebox{\plotpoint}}
\put(238.0,418.0){\usebox{\plotpoint}}
\put(238.0,419.0){\usebox{\plotpoint}}
\put(239.0,419.0){\rule[-0.200pt]{0.400pt}{0.482pt}}
\put(239.0,421.0){\usebox{\plotpoint}}
\put(240.0,421.0){\usebox{\plotpoint}}
\put(240.0,422.0){\usebox{\plotpoint}}
\put(241,422.67){\rule{0.241pt}{0.400pt}}
\multiput(241.00,422.17)(0.500,1.000){2}{\rule{0.120pt}{0.400pt}}
\put(241.0,422.0){\usebox{\plotpoint}}
\put(242,424){\usebox{\plotpoint}}
\put(242,424){\usebox{\plotpoint}}
\put(242,424){\usebox{\plotpoint}}
\put(242,424){\usebox{\plotpoint}}
\put(242,424){\usebox{\plotpoint}}
\put(242.0,424.0){\usebox{\plotpoint}}
\put(242.0,425.0){\usebox{\plotpoint}}
\put(243.0,425.0){\usebox{\plotpoint}}
\put(243.0,426.0){\usebox{\plotpoint}}
\put(244.0,426.0){\rule[-0.200pt]{0.400pt}{0.482pt}}
\put(244.0,428.0){\usebox{\plotpoint}}
\put(245.0,428.0){\usebox{\plotpoint}}
\put(245.0,429.0){\usebox{\plotpoint}}
\put(246,429.67){\rule{0.241pt}{0.400pt}}
\multiput(246.00,429.17)(0.500,1.000){2}{\rule{0.120pt}{0.400pt}}
\put(246.0,429.0){\usebox{\plotpoint}}
\put(247,431){\usebox{\plotpoint}}
\put(247,431){\usebox{\plotpoint}}
\put(247,431){\usebox{\plotpoint}}
\put(247,431){\usebox{\plotpoint}}
\put(247,431){\usebox{\plotpoint}}
\put(247.0,431.0){\usebox{\plotpoint}}
\put(247.0,432.0){\usebox{\plotpoint}}
\put(248.0,432.0){\usebox{\plotpoint}}
\put(248.0,433.0){\usebox{\plotpoint}}
\put(249.0,433.0){\rule[-0.200pt]{0.400pt}{0.482pt}}
\put(249.0,435.0){\usebox{\plotpoint}}
\put(250.0,435.0){\usebox{\plotpoint}}
\put(250.0,436.0){\usebox{\plotpoint}}
\put(251,436.67){\rule{0.241pt}{0.400pt}}
\multiput(251.00,436.17)(0.500,1.000){2}{\rule{0.120pt}{0.400pt}}
\put(251.0,436.0){\usebox{\plotpoint}}
\put(252,438){\usebox{\plotpoint}}
\put(252,438){\usebox{\plotpoint}}
\put(252,438){\usebox{\plotpoint}}
\put(252,438){\usebox{\plotpoint}}
\put(252,438){\usebox{\plotpoint}}
\put(252.0,438.0){\usebox{\plotpoint}}
\put(252.0,439.0){\usebox{\plotpoint}}
\put(253.0,439.0){\usebox{\plotpoint}}
\put(253.0,440.0){\usebox{\plotpoint}}
\put(254,440.67){\rule{0.241pt}{0.400pt}}
\multiput(254.00,440.17)(0.500,1.000){2}{\rule{0.120pt}{0.400pt}}
\put(254.0,440.0){\usebox{\plotpoint}}
\put(255,442){\usebox{\plotpoint}}
\put(255,442){\usebox{\plotpoint}}
\put(255,442){\usebox{\plotpoint}}
\put(255,442){\usebox{\plotpoint}}
\put(255.0,442.0){\usebox{\plotpoint}}
\put(255.0,443.0){\usebox{\plotpoint}}
\put(256.0,443.0){\usebox{\plotpoint}}
\put(256.0,444.0){\usebox{\plotpoint}}
\put(257.0,444.0){\rule[-0.200pt]{0.400pt}{0.482pt}}
\put(257.0,446.0){\usebox{\plotpoint}}
\put(258.0,446.0){\usebox{\plotpoint}}
\put(258.0,447.0){\usebox{\plotpoint}}
\put(259,447.67){\rule{0.241pt}{0.400pt}}
\multiput(259.00,447.17)(0.500,1.000){2}{\rule{0.120pt}{0.400pt}}
\put(259.0,447.0){\usebox{\plotpoint}}
\put(260,449){\usebox{\plotpoint}}
\put(260,449){\usebox{\plotpoint}}
\put(260,449){\usebox{\plotpoint}}
\put(260,449){\usebox{\plotpoint}}
\put(260,449){\usebox{\plotpoint}}
\put(260.0,449.0){\usebox{\plotpoint}}
\put(260.0,450.0){\usebox{\plotpoint}}
\put(261.0,450.0){\usebox{\plotpoint}}
\put(261.0,451.0){\usebox{\plotpoint}}
\put(262,451.67){\rule{0.241pt}{0.400pt}}
\multiput(262.00,451.17)(0.500,1.000){2}{\rule{0.120pt}{0.400pt}}
\put(262.0,451.0){\usebox{\plotpoint}}
\put(263,453){\usebox{\plotpoint}}
\put(263,453){\usebox{\plotpoint}}
\put(263,453){\usebox{\plotpoint}}
\put(263,453){\usebox{\plotpoint}}
\put(263.0,453.0){\usebox{\plotpoint}}
\put(263.0,454.0){\usebox{\plotpoint}}
\put(264.0,454.0){\usebox{\plotpoint}}
\put(264.0,455.0){\usebox{\plotpoint}}
\put(265.0,455.0){\rule[-0.200pt]{0.400pt}{0.482pt}}
\put(265.0,457.0){\usebox{\plotpoint}}
\put(266.0,457.0){\usebox{\plotpoint}}
\put(266.0,458.0){\usebox{\plotpoint}}
\put(267.0,458.0){\usebox{\plotpoint}}
\put(267.0,459.0){\usebox{\plotpoint}}
\put(268.0,459.0){\rule[-0.200pt]{0.400pt}{0.482pt}}
\put(268.0,461.0){\usebox{\plotpoint}}
\put(269.0,461.0){\usebox{\plotpoint}}
\put(269.0,462.0){\usebox{\plotpoint}}
\put(270,462.67){\rule{0.241pt}{0.400pt}}
\multiput(270.00,462.17)(0.500,1.000){2}{\rule{0.120pt}{0.400pt}}
\put(270.0,462.0){\usebox{\plotpoint}}
\put(271,464){\usebox{\plotpoint}}
\put(271,464){\usebox{\plotpoint}}
\put(271,464){\usebox{\plotpoint}}
\put(271,464){\usebox{\plotpoint}}
\put(271,464){\usebox{\plotpoint}}
\put(271.0,464.0){\usebox{\plotpoint}}
\put(271.0,465.0){\usebox{\plotpoint}}
\put(272.0,465.0){\usebox{\plotpoint}}
\put(272.0,466.0){\usebox{\plotpoint}}
\put(273,466.67){\rule{0.241pt}{0.400pt}}
\multiput(273.00,466.17)(0.500,1.000){2}{\rule{0.120pt}{0.400pt}}
\put(273.0,466.0){\usebox{\plotpoint}}
\put(274,468){\usebox{\plotpoint}}
\put(274,468){\usebox{\plotpoint}}
\put(274,468){\usebox{\plotpoint}}
\put(274,468){\usebox{\plotpoint}}
\put(274,468){\usebox{\plotpoint}}
\put(274.0,468.0){\usebox{\plotpoint}}
\put(274.0,469.0){\usebox{\plotpoint}}
\put(275.0,469.0){\usebox{\plotpoint}}
\put(275.0,470.0){\usebox{\plotpoint}}
\put(276.0,470.0){\rule[-0.200pt]{0.400pt}{0.482pt}}
\put(276.0,472.0){\usebox{\plotpoint}}
\put(277.0,472.0){\usebox{\plotpoint}}
\put(277.0,473.0){\usebox{\plotpoint}}
\put(278.0,473.0){\usebox{\plotpoint}}
\put(278.0,474.0){\usebox{\plotpoint}}
\put(279.0,474.0){\rule[-0.200pt]{0.400pt}{0.482pt}}
\put(279.0,476.0){\usebox{\plotpoint}}
\put(280.0,476.0){\usebox{\plotpoint}}
\put(280.0,477.0){\usebox{\plotpoint}}
\put(281.0,477.0){\usebox{\plotpoint}}
\put(281.0,478.0){\usebox{\plotpoint}}
\put(282.0,478.0){\rule[-0.200pt]{0.400pt}{0.482pt}}
\put(282.0,480.0){\usebox{\plotpoint}}
\put(283.0,480.0){\usebox{\plotpoint}}
\put(283.0,481.0){\usebox{\plotpoint}}
\put(284.0,481.0){\usebox{\plotpoint}}
\put(284.0,482.0){\usebox{\plotpoint}}
\put(285.0,482.0){\rule[-0.200pt]{0.400pt}{0.482pt}}
\put(285.0,484.0){\usebox{\plotpoint}}
\put(286.0,484.0){\usebox{\plotpoint}}
\put(286.0,485.0){\usebox{\plotpoint}}
\put(287.0,485.0){\usebox{\plotpoint}}
\put(287.0,486.0){\usebox{\plotpoint}}
\put(288.0,486.0){\rule[-0.200pt]{0.400pt}{0.482pt}}
\put(288.0,488.0){\usebox{\plotpoint}}
\put(289.0,488.0){\usebox{\plotpoint}}
\put(289.0,489.0){\usebox{\plotpoint}}
\put(290.0,489.0){\usebox{\plotpoint}}
\put(290.0,490.0){\usebox{\plotpoint}}
\put(291.0,490.0){\rule[-0.200pt]{0.400pt}{0.482pt}}
\put(291.0,492.0){\usebox{\plotpoint}}
\put(292.0,492.0){\usebox{\plotpoint}}
\put(292.0,493.0){\usebox{\plotpoint}}
\put(293.0,493.0){\usebox{\plotpoint}}
\put(293.0,494.0){\usebox{\plotpoint}}
\put(294,494.67){\rule{0.241pt}{0.400pt}}
\multiput(294.00,494.17)(0.500,1.000){2}{\rule{0.120pt}{0.400pt}}
\put(294.0,494.0){\usebox{\plotpoint}}
\put(295,496){\usebox{\plotpoint}}
\put(295,496){\usebox{\plotpoint}}
\put(295,496){\usebox{\plotpoint}}
\put(295,496){\usebox{\plotpoint}}
\put(295,496){\usebox{\plotpoint}}
\put(295.0,496.0){\usebox{\plotpoint}}
\put(295.0,497.0){\usebox{\plotpoint}}
\put(296.0,497.0){\usebox{\plotpoint}}
\put(296.0,498.0){\usebox{\plotpoint}}
\put(297,498.67){\rule{0.241pt}{0.400pt}}
\multiput(297.00,498.17)(0.500,1.000){2}{\rule{0.120pt}{0.400pt}}
\put(297.0,498.0){\usebox{\plotpoint}}
\put(298,500){\usebox{\plotpoint}}
\put(298,500){\usebox{\plotpoint}}
\put(298,500){\usebox{\plotpoint}}
\put(298,500){\usebox{\plotpoint}}
\put(298,500){\usebox{\plotpoint}}
\put(298.0,500.0){\usebox{\plotpoint}}
\put(298.0,501.0){\usebox{\plotpoint}}
\put(299.0,501.0){\usebox{\plotpoint}}
\put(299.0,502.0){\usebox{\plotpoint}}
\put(300.0,502.0){\usebox{\plotpoint}}
\put(300.0,503.0){\usebox{\plotpoint}}
\put(301.0,503.0){\rule[-0.200pt]{0.400pt}{0.482pt}}
\put(301.0,505.0){\usebox{\plotpoint}}
\put(302.0,505.0){\usebox{\plotpoint}}
\put(302.0,506.0){\usebox{\plotpoint}}
\put(303.0,506.0){\usebox{\plotpoint}}
\put(303.0,507.0){\usebox{\plotpoint}}
\put(304,507.67){\rule{0.241pt}{0.400pt}}
\multiput(304.00,507.17)(0.500,1.000){2}{\rule{0.120pt}{0.400pt}}
\put(304.0,507.0){\usebox{\plotpoint}}
\put(305,509){\usebox{\plotpoint}}
\put(305,509){\usebox{\plotpoint}}
\put(305,509){\usebox{\plotpoint}}
\put(305,509){\usebox{\plotpoint}}
\put(305,509){\usebox{\plotpoint}}
\put(305.0,509.0){\usebox{\plotpoint}}
\put(305.0,510.0){\usebox{\plotpoint}}
\put(306.0,510.0){\usebox{\plotpoint}}
\put(306.0,511.0){\usebox{\plotpoint}}
\put(307,511.67){\rule{0.241pt}{0.400pt}}
\multiput(307.00,511.17)(0.500,1.000){2}{\rule{0.120pt}{0.400pt}}
\put(307.0,511.0){\usebox{\plotpoint}}
\put(308,513){\usebox{\plotpoint}}
\put(308,513){\usebox{\plotpoint}}
\put(308,513){\usebox{\plotpoint}}
\put(308,513){\usebox{\plotpoint}}
\put(308,513){\usebox{\plotpoint}}
\put(308,513){\usebox{\plotpoint}}
\put(308.0,513.0){\usebox{\plotpoint}}
\put(308.0,514.0){\usebox{\plotpoint}}
\put(309.0,514.0){\usebox{\plotpoint}}
\put(309.0,515.0){\usebox{\plotpoint}}
\put(310.0,515.0){\usebox{\plotpoint}}
\put(310.0,516.0){\usebox{\plotpoint}}
\put(311,516.67){\rule{0.241pt}{0.400pt}}
\multiput(311.00,516.17)(0.500,1.000){2}{\rule{0.120pt}{0.400pt}}
\put(311.0,516.0){\usebox{\plotpoint}}
\put(312,518){\usebox{\plotpoint}}
\put(312,518){\usebox{\plotpoint}}
\put(312,518){\usebox{\plotpoint}}
\put(312,518){\usebox{\plotpoint}}
\put(312,518){\usebox{\plotpoint}}
\put(312.0,518.0){\usebox{\plotpoint}}
\put(312.0,519.0){\usebox{\plotpoint}}
\put(313.0,519.0){\usebox{\plotpoint}}
\put(313.0,520.0){\usebox{\plotpoint}}
\put(314.0,520.0){\usebox{\plotpoint}}
\put(314.0,521.0){\usebox{\plotpoint}}
\put(315.0,521.0){\rule[-0.200pt]{0.400pt}{0.482pt}}
\put(315.0,523.0){\usebox{\plotpoint}}
\put(316.0,523.0){\usebox{\plotpoint}}
\put(316.0,524.0){\usebox{\plotpoint}}
\put(317.0,524.0){\usebox{\plotpoint}}
\put(317.0,525.0){\usebox{\plotpoint}}
\put(318,525.67){\rule{0.241pt}{0.400pt}}
\multiput(318.00,525.17)(0.500,1.000){2}{\rule{0.120pt}{0.400pt}}
\put(318.0,525.0){\usebox{\plotpoint}}
\put(319,527){\usebox{\plotpoint}}
\put(319,527){\usebox{\plotpoint}}
\put(319,527){\usebox{\plotpoint}}
\put(319,527){\usebox{\plotpoint}}
\put(319,527){\usebox{\plotpoint}}
\put(319,527){\usebox{\plotpoint}}
\put(319.0,527.0){\usebox{\plotpoint}}
\put(319.0,528.0){\usebox{\plotpoint}}
\put(320.0,528.0){\usebox{\plotpoint}}
\put(320.0,529.0){\usebox{\plotpoint}}
\put(321.0,529.0){\usebox{\plotpoint}}
\put(321.0,530.0){\usebox{\plotpoint}}
\put(322,530.67){\rule{0.241pt}{0.400pt}}
\multiput(322.00,530.17)(0.500,1.000){2}{\rule{0.120pt}{0.400pt}}
\put(322.0,530.0){\usebox{\plotpoint}}
\put(323,532){\usebox{\plotpoint}}
\put(323,532){\usebox{\plotpoint}}
\put(323,532){\usebox{\plotpoint}}
\put(323,532){\usebox{\plotpoint}}
\put(323,532){\usebox{\plotpoint}}
\put(323,532){\usebox{\plotpoint}}
\put(323.0,532.0){\usebox{\plotpoint}}
\put(323.0,533.0){\usebox{\plotpoint}}
\put(324.0,533.0){\usebox{\plotpoint}}
\put(324.0,534.0){\usebox{\plotpoint}}
\put(325.0,534.0){\usebox{\plotpoint}}
\put(325.0,535.0){\usebox{\plotpoint}}
\put(326,535.67){\rule{0.241pt}{0.400pt}}
\multiput(326.00,535.17)(0.500,1.000){2}{\rule{0.120pt}{0.400pt}}
\put(326.0,535.0){\usebox{\plotpoint}}
\put(327,537){\usebox{\plotpoint}}
\put(327,537){\usebox{\plotpoint}}
\put(327,537){\usebox{\plotpoint}}
\put(327,537){\usebox{\plotpoint}}
\put(327,537){\usebox{\plotpoint}}
\put(327,537){\usebox{\plotpoint}}
\put(327.0,537.0){\usebox{\plotpoint}}
\put(327.0,538.0){\usebox{\plotpoint}}
\put(328.0,538.0){\usebox{\plotpoint}}
\put(328.0,539.0){\usebox{\plotpoint}}
\put(329.0,539.0){\usebox{\plotpoint}}
\put(329.0,540.0){\usebox{\plotpoint}}
\put(330,540.67){\rule{0.241pt}{0.400pt}}
\multiput(330.00,540.17)(0.500,1.000){2}{\rule{0.120pt}{0.400pt}}
\put(330.0,540.0){\usebox{\plotpoint}}
\put(331,542){\usebox{\plotpoint}}
\put(331,542){\usebox{\plotpoint}}
\put(331,542){\usebox{\plotpoint}}
\put(331,542){\usebox{\plotpoint}}
\put(331,542){\usebox{\plotpoint}}
\put(331,542){\usebox{\plotpoint}}
\put(331.0,542.0){\usebox{\plotpoint}}
\put(331.0,543.0){\usebox{\plotpoint}}
\put(332.0,543.0){\usebox{\plotpoint}}
\put(332.0,544.0){\usebox{\plotpoint}}
\put(333.0,544.0){\usebox{\plotpoint}}
\put(333.0,545.0){\usebox{\plotpoint}}
\put(334.0,545.0){\usebox{\plotpoint}}
\put(334.0,546.0){\usebox{\plotpoint}}
\put(335.0,546.0){\rule[-0.200pt]{0.400pt}{0.482pt}}
\put(335.0,548.0){\usebox{\plotpoint}}
\put(336.0,548.0){\usebox{\plotpoint}}
\put(336.0,549.0){\usebox{\plotpoint}}
\put(337.0,549.0){\usebox{\plotpoint}}
\put(337.0,550.0){\usebox{\plotpoint}}
\put(338.0,550.0){\usebox{\plotpoint}}
\put(338.0,551.0){\usebox{\plotpoint}}
\put(339,551.67){\rule{0.241pt}{0.400pt}}
\multiput(339.00,551.17)(0.500,1.000){2}{\rule{0.120pt}{0.400pt}}
\put(339.0,551.0){\usebox{\plotpoint}}
\put(340,553){\usebox{\plotpoint}}
\put(340,553){\usebox{\plotpoint}}
\put(340,553){\usebox{\plotpoint}}
\put(340,553){\usebox{\plotpoint}}
\put(340,553){\usebox{\plotpoint}}
\put(340.0,553.0){\usebox{\plotpoint}}
\put(340.0,554.0){\usebox{\plotpoint}}
\put(341.0,554.0){\usebox{\plotpoint}}
\put(341.0,555.0){\usebox{\plotpoint}}
\put(342.0,555.0){\usebox{\plotpoint}}
\put(342.0,556.0){\usebox{\plotpoint}}
\put(343.0,556.0){\usebox{\plotpoint}}
\put(343.0,557.0){\usebox{\plotpoint}}
\put(344.0,557.0){\rule[-0.200pt]{0.400pt}{0.482pt}}
\put(344.0,559.0){\usebox{\plotpoint}}
\put(345.0,559.0){\usebox{\plotpoint}}
\put(345.0,560.0){\usebox{\plotpoint}}
\put(346.0,560.0){\usebox{\plotpoint}}
\put(346.0,561.0){\usebox{\plotpoint}}
\put(347.0,561.0){\usebox{\plotpoint}}
\put(347.0,562.0){\usebox{\plotpoint}}
\put(348,562.67){\rule{0.241pt}{0.400pt}}
\multiput(348.00,562.17)(0.500,1.000){2}{\rule{0.120pt}{0.400pt}}
\put(348.0,562.0){\usebox{\plotpoint}}
\put(349,564){\usebox{\plotpoint}}
\put(349,564){\usebox{\plotpoint}}
\put(349,564){\usebox{\plotpoint}}
\put(349,564){\usebox{\plotpoint}}
\put(349,564){\usebox{\plotpoint}}
\put(349,564){\usebox{\plotpoint}}
\put(349.0,564.0){\usebox{\plotpoint}}
\put(349.0,565.0){\usebox{\plotpoint}}
\put(350.0,565.0){\usebox{\plotpoint}}
\put(350.0,566.0){\usebox{\plotpoint}}
\put(351.0,566.0){\usebox{\plotpoint}}
\put(351.0,567.0){\usebox{\plotpoint}}
\put(352.0,567.0){\usebox{\plotpoint}}
\put(352.0,568.0){\usebox{\plotpoint}}
\put(353.0,568.0){\usebox{\plotpoint}}
\put(353.0,569.0){\usebox{\plotpoint}}
\put(354.0,569.0){\rule[-0.200pt]{0.400pt}{0.482pt}}
\put(354.0,571.0){\usebox{\plotpoint}}
\put(355.0,571.0){\usebox{\plotpoint}}
\put(355.0,572.0){\usebox{\plotpoint}}
\put(356.0,572.0){\usebox{\plotpoint}}
\put(356.0,573.0){\usebox{\plotpoint}}
\put(357.0,573.0){\usebox{\plotpoint}}
\put(357.0,574.0){\usebox{\plotpoint}}
\put(358,574.67){\rule{0.241pt}{0.400pt}}
\multiput(358.00,574.17)(0.500,1.000){2}{\rule{0.120pt}{0.400pt}}
\put(358.0,574.0){\usebox{\plotpoint}}
\put(359,576){\usebox{\plotpoint}}
\put(359,576){\usebox{\plotpoint}}
\put(359,576){\usebox{\plotpoint}}
\put(359,576){\usebox{\plotpoint}}
\put(359,576){\usebox{\plotpoint}}
\put(359,576){\usebox{\plotpoint}}
\put(359.0,576.0){\usebox{\plotpoint}}
\put(359.0,577.0){\usebox{\plotpoint}}
\put(360.0,577.0){\usebox{\plotpoint}}
\put(360.0,578.0){\usebox{\plotpoint}}
\put(361.0,578.0){\usebox{\plotpoint}}
\put(361.0,579.0){\usebox{\plotpoint}}
\put(362.0,579.0){\usebox{\plotpoint}}
\put(362.0,580.0){\usebox{\plotpoint}}
\put(363.0,580.0){\usebox{\plotpoint}}
\put(363.0,581.0){\usebox{\plotpoint}}
\put(364,581.67){\rule{0.241pt}{0.400pt}}
\multiput(364.00,581.17)(0.500,1.000){2}{\rule{0.120pt}{0.400pt}}
\put(364.0,581.0){\usebox{\plotpoint}}
\put(365,583){\usebox{\plotpoint}}
\put(365,583){\usebox{\plotpoint}}
\put(365,583){\usebox{\plotpoint}}
\put(365,583){\usebox{\plotpoint}}
\put(365,583){\usebox{\plotpoint}}
\put(365.0,583.0){\usebox{\plotpoint}}
\put(365.0,584.0){\usebox{\plotpoint}}
\put(366.0,584.0){\usebox{\plotpoint}}
\put(366.0,585.0){\usebox{\plotpoint}}
\put(367.0,585.0){\usebox{\plotpoint}}
\put(367.0,586.0){\usebox{\plotpoint}}
\put(368.0,586.0){\usebox{\plotpoint}}
\put(368.0,587.0){\usebox{\plotpoint}}
\put(369.0,587.0){\usebox{\plotpoint}}
\put(369.0,588.0){\usebox{\plotpoint}}
\put(370,588.67){\rule{0.241pt}{0.400pt}}
\multiput(370.00,588.17)(0.500,1.000){2}{\rule{0.120pt}{0.400pt}}
\put(370.0,588.0){\usebox{\plotpoint}}
\put(371,590){\usebox{\plotpoint}}
\put(371,590){\usebox{\plotpoint}}
\put(371,590){\usebox{\plotpoint}}
\put(371,590){\usebox{\plotpoint}}
\put(371,590){\usebox{\plotpoint}}
\put(371,590){\usebox{\plotpoint}}
\put(371.0,590.0){\usebox{\plotpoint}}
\put(371.0,591.0){\usebox{\plotpoint}}
\put(372.0,591.0){\usebox{\plotpoint}}
\put(372.0,592.0){\usebox{\plotpoint}}
\put(373.0,592.0){\usebox{\plotpoint}}
\put(373.0,593.0){\usebox{\plotpoint}}
\put(374.0,593.0){\usebox{\plotpoint}}
\put(374.0,594.0){\usebox{\plotpoint}}
\put(375.0,594.0){\usebox{\plotpoint}}
\put(375.0,595.0){\usebox{\plotpoint}}
\put(376,595.67){\rule{0.241pt}{0.400pt}}
\multiput(376.00,595.17)(0.500,1.000){2}{\rule{0.120pt}{0.400pt}}
\put(376.0,595.0){\usebox{\plotpoint}}
\put(377,597){\usebox{\plotpoint}}
\put(377,597){\usebox{\plotpoint}}
\put(377,597){\usebox{\plotpoint}}
\put(377,597){\usebox{\plotpoint}}
\put(377,597){\usebox{\plotpoint}}
\put(377,597){\usebox{\plotpoint}}
\put(377.0,597.0){\usebox{\plotpoint}}
\put(377.0,598.0){\usebox{\plotpoint}}
\put(378.0,598.0){\usebox{\plotpoint}}
\put(378.0,599.0){\usebox{\plotpoint}}
\put(379.0,599.0){\usebox{\plotpoint}}
\put(379.0,600.0){\usebox{\plotpoint}}
\put(380.0,600.0){\usebox{\plotpoint}}
\put(380.0,601.0){\usebox{\plotpoint}}
\put(381.0,601.0){\usebox{\plotpoint}}
\put(381.0,602.0){\usebox{\plotpoint}}
\put(382.0,602.0){\usebox{\plotpoint}}
\put(382.0,603.0){\usebox{\plotpoint}}
\put(383.0,603.0){\usebox{\plotpoint}}
\put(383.0,604.0){\usebox{\plotpoint}}
\put(384,604.67){\rule{0.241pt}{0.400pt}}
\multiput(384.00,604.17)(0.500,1.000){2}{\rule{0.120pt}{0.400pt}}
\put(384.0,604.0){\usebox{\plotpoint}}
\put(385,606){\usebox{\plotpoint}}
\put(385,606){\usebox{\plotpoint}}
\put(385,606){\usebox{\plotpoint}}
\put(385,606){\usebox{\plotpoint}}
\put(385,606){\usebox{\plotpoint}}
\put(385,606){\usebox{\plotpoint}}
\put(385.0,606.0){\usebox{\plotpoint}}
\put(385.0,607.0){\usebox{\plotpoint}}
\put(386.0,607.0){\usebox{\plotpoint}}
\put(386.0,608.0){\usebox{\plotpoint}}
\put(387.0,608.0){\usebox{\plotpoint}}
\put(387.0,609.0){\usebox{\plotpoint}}
\put(388.0,609.0){\usebox{\plotpoint}}
\put(388.0,610.0){\usebox{\plotpoint}}
\put(389.0,610.0){\usebox{\plotpoint}}
\put(389.0,611.0){\usebox{\plotpoint}}
\put(390.0,611.0){\usebox{\plotpoint}}
\put(390.0,612.0){\usebox{\plotpoint}}
\put(391.0,612.0){\usebox{\plotpoint}}
\put(391.0,613.0){\usebox{\plotpoint}}
\put(392,613.67){\rule{0.241pt}{0.400pt}}
\multiput(392.00,613.17)(0.500,1.000){2}{\rule{0.120pt}{0.400pt}}
\put(392.0,613.0){\usebox{\plotpoint}}
\put(393,615){\usebox{\plotpoint}}
\put(393,615){\usebox{\plotpoint}}
\put(393,615){\usebox{\plotpoint}}
\put(393,615){\usebox{\plotpoint}}
\put(393,615){\usebox{\plotpoint}}
\put(393,615){\usebox{\plotpoint}}
\put(393,615){\usebox{\plotpoint}}
\put(393,614.67){\rule{0.241pt}{0.400pt}}
\multiput(393.00,614.17)(0.500,1.000){2}{\rule{0.120pt}{0.400pt}}
\put(394,616){\usebox{\plotpoint}}
\put(394,616){\usebox{\plotpoint}}
\put(394,616){\usebox{\plotpoint}}
\put(394,616){\usebox{\plotpoint}}
\put(394,616){\usebox{\plotpoint}}
\put(394,616){\usebox{\plotpoint}}
\put(394.0,616.0){\usebox{\plotpoint}}
\put(394.0,617.0){\usebox{\plotpoint}}
\put(395.0,617.0){\usebox{\plotpoint}}
\put(395.0,618.0){\usebox{\plotpoint}}
\put(396.0,618.0){\usebox{\plotpoint}}
\put(396.0,619.0){\usebox{\plotpoint}}
\put(397.0,619.0){\usebox{\plotpoint}}
\put(397.0,620.0){\usebox{\plotpoint}}
\put(398.0,620.0){\usebox{\plotpoint}}
\put(398.0,621.0){\usebox{\plotpoint}}
\put(399.0,621.0){\usebox{\plotpoint}}
\put(399.0,622.0){\usebox{\plotpoint}}
\put(400.0,622.0){\usebox{\plotpoint}}
\put(400.0,623.0){\usebox{\plotpoint}}
\put(401.0,623.0){\usebox{\plotpoint}}
\put(401.0,624.0){\usebox{\plotpoint}}
\put(402,624.67){\rule{0.241pt}{0.400pt}}
\multiput(402.00,624.17)(0.500,1.000){2}{\rule{0.120pt}{0.400pt}}
\put(402.0,624.0){\usebox{\plotpoint}}
\put(403,626){\usebox{\plotpoint}}
\put(403,626){\usebox{\plotpoint}}
\put(403,626){\usebox{\plotpoint}}
\put(403,626){\usebox{\plotpoint}}
\put(403,626){\usebox{\plotpoint}}
\put(403,626){\usebox{\plotpoint}}
\put(403,626){\usebox{\plotpoint}}
\put(403,625.67){\rule{0.241pt}{0.400pt}}
\multiput(403.00,625.17)(0.500,1.000){2}{\rule{0.120pt}{0.400pt}}
\put(404,627){\usebox{\plotpoint}}
\put(404,627){\usebox{\plotpoint}}
\put(404,627){\usebox{\plotpoint}}
\put(404,627){\usebox{\plotpoint}}
\put(404,627){\usebox{\plotpoint}}
\put(404,627){\usebox{\plotpoint}}
\put(404.0,627.0){\usebox{\plotpoint}}
\put(404.0,628.0){\usebox{\plotpoint}}
\put(405.0,628.0){\usebox{\plotpoint}}
\put(405.0,629.0){\usebox{\plotpoint}}
\put(406.0,629.0){\usebox{\plotpoint}}
\put(406.0,630.0){\usebox{\plotpoint}}
\put(407.0,630.0){\usebox{\plotpoint}}
\put(407.0,631.0){\usebox{\plotpoint}}
\put(408.0,631.0){\usebox{\plotpoint}}
\put(408.0,632.0){\usebox{\plotpoint}}
\put(409.0,632.0){\usebox{\plotpoint}}
\put(409.0,633.0){\usebox{\plotpoint}}
\put(410.0,633.0){\usebox{\plotpoint}}
\put(410.0,634.0){\usebox{\plotpoint}}
\put(411.0,634.0){\usebox{\plotpoint}}
\put(411.0,635.0){\usebox{\plotpoint}}
\put(412.0,635.0){\usebox{\plotpoint}}
\put(412.0,636.0){\usebox{\plotpoint}}
\put(413.0,636.0){\usebox{\plotpoint}}
\put(413.0,637.0){\usebox{\plotpoint}}
\put(414.0,637.0){\usebox{\plotpoint}}
\put(414.0,638.0){\usebox{\plotpoint}}
\put(415.0,638.0){\usebox{\plotpoint}}
\put(415.0,639.0){\usebox{\plotpoint}}
\put(416,639.67){\rule{0.241pt}{0.400pt}}
\multiput(416.00,639.17)(0.500,1.000){2}{\rule{0.120pt}{0.400pt}}
\put(416.0,639.0){\usebox{\plotpoint}}
\put(417,641){\usebox{\plotpoint}}
\put(417,641){\usebox{\plotpoint}}
\put(417,641){\usebox{\plotpoint}}
\put(417,641){\usebox{\plotpoint}}
\put(417,641){\usebox{\plotpoint}}
\put(417,641){\usebox{\plotpoint}}
\put(417,641){\usebox{\plotpoint}}
\put(417,640.67){\rule{0.241pt}{0.400pt}}
\multiput(417.00,640.17)(0.500,1.000){2}{\rule{0.120pt}{0.400pt}}
\put(418,642){\usebox{\plotpoint}}
\put(418,642){\usebox{\plotpoint}}
\put(418,642){\usebox{\plotpoint}}
\put(418,642){\usebox{\plotpoint}}
\put(418,642){\usebox{\plotpoint}}
\put(418,642){\usebox{\plotpoint}}
\put(418,642){\usebox{\plotpoint}}
\put(418,641.67){\rule{0.241pt}{0.400pt}}
\multiput(418.00,641.17)(0.500,1.000){2}{\rule{0.120pt}{0.400pt}}
\put(419,643){\usebox{\plotpoint}}
\put(419,643){\usebox{\plotpoint}}
\put(419,643){\usebox{\plotpoint}}
\put(419,643){\usebox{\plotpoint}}
\put(419,643){\usebox{\plotpoint}}
\put(419,643){\usebox{\plotpoint}}
\put(419.0,643.0){\usebox{\plotpoint}}
\put(419.0,644.0){\usebox{\plotpoint}}
\put(420.0,644.0){\usebox{\plotpoint}}
\put(420.0,645.0){\usebox{\plotpoint}}
\put(421.0,645.0){\usebox{\plotpoint}}
\put(421.0,646.0){\usebox{\plotpoint}}
\put(422.0,646.0){\usebox{\plotpoint}}
\put(422.0,647.0){\usebox{\plotpoint}}
\put(423.0,647.0){\usebox{\plotpoint}}
\put(423.0,648.0){\usebox{\plotpoint}}
\put(424.0,648.0){\usebox{\plotpoint}}
\put(424.0,649.0){\usebox{\plotpoint}}
\put(425.0,649.0){\usebox{\plotpoint}}
\put(425.0,650.0){\usebox{\plotpoint}}
\put(426.0,650.0){\usebox{\plotpoint}}
\put(426.0,651.0){\usebox{\plotpoint}}
\put(427.0,651.0){\usebox{\plotpoint}}
\put(427.0,652.0){\usebox{\plotpoint}}
\put(428.0,652.0){\usebox{\plotpoint}}
\put(428.0,653.0){\usebox{\plotpoint}}
\put(429.0,653.0){\usebox{\plotpoint}}
\put(429.0,654.0){\usebox{\plotpoint}}
\put(430.0,654.0){\usebox{\plotpoint}}
\put(430.0,655.0){\usebox{\plotpoint}}
\put(431.0,655.0){\usebox{\plotpoint}}
\put(431.0,656.0){\usebox{\plotpoint}}
\put(432.0,656.0){\usebox{\plotpoint}}
\put(432.0,657.0){\usebox{\plotpoint}}
\put(433.0,657.0){\usebox{\plotpoint}}
\put(433.0,658.0){\usebox{\plotpoint}}
\put(434.0,658.0){\usebox{\plotpoint}}
\put(434.0,659.0){\usebox{\plotpoint}}
\put(435.0,659.0){\usebox{\plotpoint}}
\put(435.0,660.0){\usebox{\plotpoint}}
\put(436.0,660.0){\usebox{\plotpoint}}
\put(436.0,661.0){\usebox{\plotpoint}}
\put(437.0,661.0){\usebox{\plotpoint}}
\put(437.0,662.0){\usebox{\plotpoint}}
\put(438.0,662.0){\usebox{\plotpoint}}
\put(438.0,663.0){\usebox{\plotpoint}}
\put(439.0,663.0){\usebox{\plotpoint}}
\put(439.0,664.0){\usebox{\plotpoint}}
\put(440.0,664.0){\usebox{\plotpoint}}
\put(440.0,665.0){\usebox{\plotpoint}}
\put(441.0,665.0){\usebox{\plotpoint}}
\put(441.0,666.0){\usebox{\plotpoint}}
\put(442.0,666.0){\usebox{\plotpoint}}
\put(442.0,667.0){\usebox{\plotpoint}}
\put(443.0,667.0){\usebox{\plotpoint}}
\put(443.0,668.0){\usebox{\plotpoint}}
\put(444.0,668.0){\usebox{\plotpoint}}
\put(444.0,669.0){\usebox{\plotpoint}}
\put(445.0,669.0){\usebox{\plotpoint}}
\put(445.0,670.0){\usebox{\plotpoint}}
\put(446.0,670.0){\usebox{\plotpoint}}
\put(446.0,671.0){\usebox{\plotpoint}}
\put(447.0,671.0){\usebox{\plotpoint}}
\put(447.0,672.0){\usebox{\plotpoint}}
\put(448.0,672.0){\usebox{\plotpoint}}
\put(448.0,673.0){\usebox{\plotpoint}}
\put(449.0,673.0){\usebox{\plotpoint}}
\put(449.0,674.0){\usebox{\plotpoint}}
\put(450.0,674.0){\usebox{\plotpoint}}
\put(450.0,675.0){\usebox{\plotpoint}}
\put(451.0,675.0){\usebox{\plotpoint}}
\put(451.0,676.0){\usebox{\plotpoint}}
\put(452.0,676.0){\usebox{\plotpoint}}
\put(452.0,677.0){\usebox{\plotpoint}}
\put(453.0,677.0){\usebox{\plotpoint}}
\put(453.0,678.0){\usebox{\plotpoint}}
\put(454.0,678.0){\usebox{\plotpoint}}
\put(454.0,679.0){\usebox{\plotpoint}}
\put(455.0,679.0){\usebox{\plotpoint}}
\put(455.0,680.0){\usebox{\plotpoint}}
\put(456.0,680.0){\usebox{\plotpoint}}
\put(456.0,681.0){\usebox{\plotpoint}}
\put(457.0,681.0){\usebox{\plotpoint}}
\put(457.0,682.0){\usebox{\plotpoint}}
\put(458.0,682.0){\usebox{\plotpoint}}
\put(458.0,683.0){\usebox{\plotpoint}}
\put(459.0,683.0){\usebox{\plotpoint}}
\put(459.0,684.0){\usebox{\plotpoint}}
\put(460.0,684.0){\usebox{\plotpoint}}
\put(460.0,685.0){\usebox{\plotpoint}}
\put(461.0,685.0){\usebox{\plotpoint}}
\put(461.0,686.0){\usebox{\plotpoint}}
\put(462.0,686.0){\usebox{\plotpoint}}
\put(462.0,687.0){\usebox{\plotpoint}}
\put(463.0,687.0){\usebox{\plotpoint}}
\put(463.0,688.0){\usebox{\plotpoint}}
\put(464.0,688.0){\usebox{\plotpoint}}
\put(465,688.67){\rule{0.241pt}{0.400pt}}
\multiput(465.00,688.17)(0.500,1.000){2}{\rule{0.120pt}{0.400pt}}
\put(464.0,689.0){\usebox{\plotpoint}}
\put(466,690){\usebox{\plotpoint}}
\put(466,690){\usebox{\plotpoint}}
\put(466,690){\usebox{\plotpoint}}
\put(466,690){\usebox{\plotpoint}}
\put(466,690){\usebox{\plotpoint}}
\put(466,690){\usebox{\plotpoint}}
\put(466,690){\usebox{\plotpoint}}
\put(466,689.67){\rule{0.241pt}{0.400pt}}
\multiput(466.00,689.17)(0.500,1.000){2}{\rule{0.120pt}{0.400pt}}
\put(467,691){\usebox{\plotpoint}}
\put(467,691){\usebox{\plotpoint}}
\put(467,691){\usebox{\plotpoint}}
\put(467,691){\usebox{\plotpoint}}
\put(467,691){\usebox{\plotpoint}}
\put(467,691){\usebox{\plotpoint}}
\put(467,691){\usebox{\plotpoint}}
\put(467.0,691.0){\usebox{\plotpoint}}
\put(468.0,691.0){\usebox{\plotpoint}}
\put(468.0,692.0){\usebox{\plotpoint}}
\put(469.0,692.0){\usebox{\plotpoint}}
\put(469.0,693.0){\usebox{\plotpoint}}
\put(470.0,693.0){\usebox{\plotpoint}}
\put(470.0,694.0){\usebox{\plotpoint}}
\put(471.0,694.0){\usebox{\plotpoint}}
\put(471.0,695.0){\usebox{\plotpoint}}
\put(472.0,695.0){\usebox{\plotpoint}}
\put(472.0,696.0){\usebox{\plotpoint}}
\put(473.0,696.0){\usebox{\plotpoint}}
\put(473.0,697.0){\usebox{\plotpoint}}
\put(474.0,697.0){\usebox{\plotpoint}}
\put(474.0,698.0){\usebox{\plotpoint}}
\put(475.0,698.0){\usebox{\plotpoint}}
\put(475.0,699.0){\usebox{\plotpoint}}
\put(476.0,699.0){\usebox{\plotpoint}}
\put(476.0,700.0){\usebox{\plotpoint}}
\put(477.0,700.0){\usebox{\plotpoint}}
\put(478,700.67){\rule{0.241pt}{0.400pt}}
\multiput(478.00,700.17)(0.500,1.000){2}{\rule{0.120pt}{0.400pt}}
\put(477.0,701.0){\usebox{\plotpoint}}
\put(479,702){\usebox{\plotpoint}}
\put(479,702){\usebox{\plotpoint}}
\put(479,702){\usebox{\plotpoint}}
\put(479,702){\usebox{\plotpoint}}
\put(479,702){\usebox{\plotpoint}}
\put(479,702){\usebox{\plotpoint}}
\put(479,702){\usebox{\plotpoint}}
\put(479,701.67){\rule{0.241pt}{0.400pt}}
\multiput(479.00,701.17)(0.500,1.000){2}{\rule{0.120pt}{0.400pt}}
\put(480,703){\usebox{\plotpoint}}
\put(480,703){\usebox{\plotpoint}}
\put(480,703){\usebox{\plotpoint}}
\put(480,703){\usebox{\plotpoint}}
\put(480,703){\usebox{\plotpoint}}
\put(480,703){\usebox{\plotpoint}}
\put(480,703){\usebox{\plotpoint}}
\put(480.0,703.0){\usebox{\plotpoint}}
\put(481.0,703.0){\usebox{\plotpoint}}
\put(481.0,704.0){\usebox{\plotpoint}}
\put(482.0,704.0){\usebox{\plotpoint}}
\put(482.0,705.0){\usebox{\plotpoint}}
\put(483.0,705.0){\usebox{\plotpoint}}
\put(483.0,706.0){\usebox{\plotpoint}}
\put(484.0,706.0){\usebox{\plotpoint}}
\put(484.0,707.0){\usebox{\plotpoint}}
\put(485.0,707.0){\usebox{\plotpoint}}
\put(485.0,708.0){\usebox{\plotpoint}}
\put(486.0,708.0){\usebox{\plotpoint}}
\put(486.0,709.0){\usebox{\plotpoint}}
\put(487.0,709.0){\usebox{\plotpoint}}
\put(488,709.67){\rule{0.241pt}{0.400pt}}
\multiput(488.00,709.17)(0.500,1.000){2}{\rule{0.120pt}{0.400pt}}
\put(487.0,710.0){\usebox{\plotpoint}}
\put(489,711){\usebox{\plotpoint}}
\put(489,711){\usebox{\plotpoint}}
\put(489,711){\usebox{\plotpoint}}
\put(489,711){\usebox{\plotpoint}}
\put(489,711){\usebox{\plotpoint}}
\put(489,711){\usebox{\plotpoint}}
\put(489,711){\usebox{\plotpoint}}
\put(489.0,711.0){\usebox{\plotpoint}}
\put(490.0,711.0){\usebox{\plotpoint}}
\put(490.0,712.0){\usebox{\plotpoint}}
\put(491.0,712.0){\usebox{\plotpoint}}
\put(491.0,713.0){\usebox{\plotpoint}}
\put(492.0,713.0){\usebox{\plotpoint}}
\put(492.0,714.0){\usebox{\plotpoint}}
\put(493.0,714.0){\usebox{\plotpoint}}
\put(493.0,715.0){\usebox{\plotpoint}}
\put(494.0,715.0){\usebox{\plotpoint}}
\put(494.0,716.0){\usebox{\plotpoint}}
\put(495.0,716.0){\usebox{\plotpoint}}
\put(496,716.67){\rule{0.241pt}{0.400pt}}
\multiput(496.00,716.17)(0.500,1.000){2}{\rule{0.120pt}{0.400pt}}
\put(495.0,717.0){\usebox{\plotpoint}}
\put(497,718){\usebox{\plotpoint}}
\put(497,718){\usebox{\plotpoint}}
\put(497,718){\usebox{\plotpoint}}
\put(497,718){\usebox{\plotpoint}}
\put(497,718){\usebox{\plotpoint}}
\put(497,718){\usebox{\plotpoint}}
\put(497,718){\usebox{\plotpoint}}
\put(497.0,718.0){\usebox{\plotpoint}}
\put(498.0,718.0){\usebox{\plotpoint}}
\put(498.0,719.0){\usebox{\plotpoint}}
\put(499.0,719.0){\usebox{\plotpoint}}
\put(499.0,720.0){\usebox{\plotpoint}}
\put(500.0,720.0){\usebox{\plotpoint}}
\put(500.0,721.0){\usebox{\plotpoint}}
\put(501.0,721.0){\usebox{\plotpoint}}
\put(501.0,722.0){\usebox{\plotpoint}}
\put(502.0,722.0){\usebox{\plotpoint}}
\put(503,722.67){\rule{0.241pt}{0.400pt}}
\multiput(503.00,722.17)(0.500,1.000){2}{\rule{0.120pt}{0.400pt}}
\put(502.0,723.0){\usebox{\plotpoint}}
\put(504,724){\usebox{\plotpoint}}
\put(504,724){\usebox{\plotpoint}}
\put(504,724){\usebox{\plotpoint}}
\put(504,724){\usebox{\plotpoint}}
\put(504,724){\usebox{\plotpoint}}
\put(504,724){\usebox{\plotpoint}}
\put(504,724){\usebox{\plotpoint}}
\put(504.0,724.0){\usebox{\plotpoint}}
\put(505.0,724.0){\usebox{\plotpoint}}
\put(505.0,725.0){\usebox{\plotpoint}}
\put(506.0,725.0){\usebox{\plotpoint}}
\put(506.0,726.0){\usebox{\plotpoint}}
\put(507.0,726.0){\usebox{\plotpoint}}
\put(507.0,727.0){\usebox{\plotpoint}}
\put(508.0,727.0){\usebox{\plotpoint}}
\put(508.0,728.0){\usebox{\plotpoint}}
\put(509.0,728.0){\usebox{\plotpoint}}
\put(509.0,729.0){\rule[-0.200pt]{0.482pt}{0.400pt}}
\put(511.0,729.0){\usebox{\plotpoint}}
\put(511.0,730.0){\usebox{\plotpoint}}
\put(512.0,730.0){\usebox{\plotpoint}}
\put(512.0,731.0){\usebox{\plotpoint}}
\put(513.0,731.0){\usebox{\plotpoint}}
\put(513.0,732.0){\usebox{\plotpoint}}
\put(514.0,732.0){\usebox{\plotpoint}}
\put(514.0,733.0){\usebox{\plotpoint}}
\put(515.0,733.0){\usebox{\plotpoint}}
\put(515.0,734.0){\rule[-0.200pt]{0.482pt}{0.400pt}}
\put(517.0,734.0){\usebox{\plotpoint}}
\put(517.0,735.0){\usebox{\plotpoint}}
\put(518.0,735.0){\usebox{\plotpoint}}
\put(518.0,736.0){\usebox{\plotpoint}}
\put(519.0,736.0){\usebox{\plotpoint}}
\put(519.0,737.0){\usebox{\plotpoint}}
\put(520.0,737.0){\usebox{\plotpoint}}
\put(520.0,738.0){\rule[-0.200pt]{0.482pt}{0.400pt}}
\put(522.0,738.0){\usebox{\plotpoint}}
\put(522.0,739.0){\usebox{\plotpoint}}
\put(523.0,739.0){\usebox{\plotpoint}}
\put(523.0,740.0){\usebox{\plotpoint}}
\put(524.0,740.0){\usebox{\plotpoint}}
\put(524.0,741.0){\usebox{\plotpoint}}
\put(525.0,741.0){\usebox{\plotpoint}}
\put(526,741.67){\rule{0.241pt}{0.400pt}}
\multiput(526.00,741.17)(0.500,1.000){2}{\rule{0.120pt}{0.400pt}}
\put(525.0,742.0){\usebox{\plotpoint}}
\put(527,743){\usebox{\plotpoint}}
\put(527,743){\usebox{\plotpoint}}
\put(527,743){\usebox{\plotpoint}}
\put(527,743){\usebox{\plotpoint}}
\put(527,743){\usebox{\plotpoint}}
\put(527,743){\usebox{\plotpoint}}
\put(527,743){\usebox{\plotpoint}}
\put(527.0,743.0){\usebox{\plotpoint}}
\put(528.0,743.0){\usebox{\plotpoint}}
\put(528.0,744.0){\usebox{\plotpoint}}
\put(529.0,744.0){\usebox{\plotpoint}}
\put(529.0,745.0){\usebox{\plotpoint}}
\put(530.0,745.0){\usebox{\plotpoint}}
\put(530.0,746.0){\rule[-0.200pt]{0.482pt}{0.400pt}}
\put(532.0,746.0){\usebox{\plotpoint}}
\put(532.0,747.0){\usebox{\plotpoint}}
\put(533.0,747.0){\usebox{\plotpoint}}
\put(533.0,748.0){\usebox{\plotpoint}}
\put(534.0,748.0){\usebox{\plotpoint}}
\put(535,748.67){\rule{0.241pt}{0.400pt}}
\multiput(535.00,748.17)(0.500,1.000){2}{\rule{0.120pt}{0.400pt}}
\put(534.0,749.0){\usebox{\plotpoint}}
\put(536,750){\usebox{\plotpoint}}
\put(536,750){\usebox{\plotpoint}}
\put(536,750){\usebox{\plotpoint}}
\put(536,750){\usebox{\plotpoint}}
\put(536,750){\usebox{\plotpoint}}
\put(536,750){\usebox{\plotpoint}}
\put(536,750){\usebox{\plotpoint}}
\put(536.0,750.0){\usebox{\plotpoint}}
\put(537.0,750.0){\usebox{\plotpoint}}
\put(537.0,751.0){\usebox{\plotpoint}}
\put(538.0,751.0){\usebox{\plotpoint}}
\put(538.0,752.0){\usebox{\plotpoint}}
\put(539.0,752.0){\usebox{\plotpoint}}
\put(539.0,753.0){\rule[-0.200pt]{0.482pt}{0.400pt}}
\put(541.0,753.0){\usebox{\plotpoint}}
\put(541.0,754.0){\usebox{\plotpoint}}
\put(542.0,754.0){\usebox{\plotpoint}}
\put(542.0,755.0){\usebox{\plotpoint}}
\put(543.0,755.0){\usebox{\plotpoint}}
\put(543.0,756.0){\rule[-0.200pt]{0.482pt}{0.400pt}}
\put(545.0,756.0){\usebox{\plotpoint}}
\put(545.0,757.0){\usebox{\plotpoint}}
\put(546.0,757.0){\usebox{\plotpoint}}
\put(546.0,758.0){\usebox{\plotpoint}}
\put(547.0,758.0){\usebox{\plotpoint}}
\put(547.0,759.0){\rule[-0.200pt]{0.482pt}{0.400pt}}
\put(549.0,759.0){\usebox{\plotpoint}}
\put(549.0,760.0){\usebox{\plotpoint}}
\put(550.0,760.0){\usebox{\plotpoint}}
\put(551,760.67){\rule{0.241pt}{0.400pt}}
\multiput(551.00,760.17)(0.500,1.000){2}{\rule{0.120pt}{0.400pt}}
\put(550.0,761.0){\usebox{\plotpoint}}
\put(552,762){\usebox{\plotpoint}}
\put(552,762){\usebox{\plotpoint}}
\put(552,762){\usebox{\plotpoint}}
\put(552,762){\usebox{\plotpoint}}
\put(552,762){\usebox{\plotpoint}}
\put(552,762){\usebox{\plotpoint}}
\put(552,762){\usebox{\plotpoint}}
\put(552.0,762.0){\usebox{\plotpoint}}
\put(553.0,762.0){\usebox{\plotpoint}}
\put(553.0,763.0){\usebox{\plotpoint}}
\put(554.0,763.0){\usebox{\plotpoint}}
\put(554.0,764.0){\rule[-0.200pt]{0.482pt}{0.400pt}}
\put(556.0,764.0){\usebox{\plotpoint}}
\put(556.0,765.0){\usebox{\plotpoint}}
\put(557.0,765.0){\usebox{\plotpoint}}
\put(558,765.67){\rule{0.241pt}{0.400pt}}
\multiput(558.00,765.17)(0.500,1.000){2}{\rule{0.120pt}{0.400pt}}
\put(557.0,766.0){\usebox{\plotpoint}}
\put(559,767){\usebox{\plotpoint}}
\put(559,767){\usebox{\plotpoint}}
\put(559,767){\usebox{\plotpoint}}
\put(559,767){\usebox{\plotpoint}}
\put(559,767){\usebox{\plotpoint}}
\put(559,767){\usebox{\plotpoint}}
\put(559,767){\usebox{\plotpoint}}
\put(559.0,767.0){\usebox{\plotpoint}}
\put(560.0,767.0){\usebox{\plotpoint}}
\put(560.0,768.0){\usebox{\plotpoint}}
\put(561.0,768.0){\usebox{\plotpoint}}
\put(561.0,769.0){\rule[-0.200pt]{0.482pt}{0.400pt}}
\put(563.0,769.0){\usebox{\plotpoint}}
\put(563.0,770.0){\usebox{\plotpoint}}
\put(564.0,770.0){\usebox{\plotpoint}}
\put(564.0,771.0){\rule[-0.200pt]{0.482pt}{0.400pt}}
\put(566.0,771.0){\usebox{\plotpoint}}
\put(566.0,772.0){\usebox{\plotpoint}}
\put(567.0,772.0){\usebox{\plotpoint}}
\put(568,772.67){\rule{0.241pt}{0.400pt}}
\multiput(568.00,772.17)(0.500,1.000){2}{\rule{0.120pt}{0.400pt}}
\put(567.0,773.0){\usebox{\plotpoint}}
\put(569,774){\usebox{\plotpoint}}
\put(569,774){\usebox{\plotpoint}}
\put(569,774){\usebox{\plotpoint}}
\put(569,774){\usebox{\plotpoint}}
\put(569,774){\usebox{\plotpoint}}
\put(569,774){\usebox{\plotpoint}}
\put(569,774){\usebox{\plotpoint}}
\put(569.0,774.0){\usebox{\plotpoint}}
\put(570.0,774.0){\usebox{\plotpoint}}
\put(571,774.67){\rule{0.241pt}{0.400pt}}
\multiput(571.00,774.17)(0.500,1.000){2}{\rule{0.120pt}{0.400pt}}
\put(570.0,775.0){\usebox{\plotpoint}}
\put(572,776){\usebox{\plotpoint}}
\put(572,776){\usebox{\plotpoint}}
\put(572,776){\usebox{\plotpoint}}
\put(572,776){\usebox{\plotpoint}}
\put(572,776){\usebox{\plotpoint}}
\put(572,776){\usebox{\plotpoint}}
\put(572,776){\usebox{\plotpoint}}
\put(572.0,776.0){\usebox{\plotpoint}}
\put(573.0,776.0){\usebox{\plotpoint}}
\put(574,776.67){\rule{0.241pt}{0.400pt}}
\multiput(574.00,776.17)(0.500,1.000){2}{\rule{0.120pt}{0.400pt}}
\put(573.0,777.0){\usebox{\plotpoint}}
\put(575,778){\usebox{\plotpoint}}
\put(575,778){\usebox{\plotpoint}}
\put(575,778){\usebox{\plotpoint}}
\put(575,778){\usebox{\plotpoint}}
\put(575,778){\usebox{\plotpoint}}
\put(575,778){\usebox{\plotpoint}}
\put(575,778){\usebox{\plotpoint}}
\put(575.0,778.0){\usebox{\plotpoint}}
\put(576.0,778.0){\usebox{\plotpoint}}
\put(577,778.67){\rule{0.241pt}{0.400pt}}
\multiput(577.00,778.17)(0.500,1.000){2}{\rule{0.120pt}{0.400pt}}
\put(576.0,779.0){\usebox{\plotpoint}}
\put(578,780){\usebox{\plotpoint}}
\put(578,780){\usebox{\plotpoint}}
\put(578,780){\usebox{\plotpoint}}
\put(578,780){\usebox{\plotpoint}}
\put(578,780){\usebox{\plotpoint}}
\put(578,780){\usebox{\plotpoint}}
\put(578,780){\usebox{\plotpoint}}
\put(578.0,780.0){\usebox{\plotpoint}}
\put(579.0,780.0){\usebox{\plotpoint}}
\put(580,780.67){\rule{0.241pt}{0.400pt}}
\multiput(580.00,780.17)(0.500,1.000){2}{\rule{0.120pt}{0.400pt}}
\put(579.0,781.0){\usebox{\plotpoint}}
\put(581,782){\usebox{\plotpoint}}
\put(581,782){\usebox{\plotpoint}}
\put(581,782){\usebox{\plotpoint}}
\put(581,782){\usebox{\plotpoint}}
\put(581,782){\usebox{\plotpoint}}
\put(581,782){\usebox{\plotpoint}}
\put(581,782){\usebox{\plotpoint}}
\put(581.0,782.0){\usebox{\plotpoint}}
\put(582.0,782.0){\usebox{\plotpoint}}
\put(582.0,783.0){\rule[-0.200pt]{0.482pt}{0.400pt}}
\put(584.0,783.0){\usebox{\plotpoint}}
\put(584.0,784.0){\usebox{\plotpoint}}
\put(585.0,784.0){\usebox{\plotpoint}}
\put(585.0,785.0){\rule[-0.200pt]{0.482pt}{0.400pt}}
\put(587.0,785.0){\usebox{\plotpoint}}
\put(587.0,786.0){\usebox{\plotpoint}}
\put(588.0,786.0){\usebox{\plotpoint}}
\put(588.0,787.0){\rule[-0.200pt]{0.482pt}{0.400pt}}
\put(590.0,787.0){\usebox{\plotpoint}}
\put(591,787.67){\rule{0.241pt}{0.400pt}}
\multiput(591.00,787.17)(0.500,1.000){2}{\rule{0.120pt}{0.400pt}}
\put(590.0,788.0){\usebox{\plotpoint}}
\put(592,789){\usebox{\plotpoint}}
\put(592,789){\usebox{\plotpoint}}
\put(592,789){\usebox{\plotpoint}}
\put(592,789){\usebox{\plotpoint}}
\put(592,789){\usebox{\plotpoint}}
\put(592,789){\usebox{\plotpoint}}
\put(592,789){\usebox{\plotpoint}}
\put(592.0,789.0){\usebox{\plotpoint}}
\put(593.0,789.0){\usebox{\plotpoint}}
\put(593.0,790.0){\rule[-0.200pt]{0.482pt}{0.400pt}}
\put(595.0,790.0){\usebox{\plotpoint}}
\put(595.0,791.0){\usebox{\plotpoint}}
\put(596.0,791.0){\usebox{\plotpoint}}
\put(596.0,792.0){\rule[-0.200pt]{0.482pt}{0.400pt}}
\put(598.0,792.0){\usebox{\plotpoint}}
\put(598.0,793.0){\rule[-0.200pt]{0.482pt}{0.400pt}}
\put(600.0,793.0){\usebox{\plotpoint}}
\put(600.0,794.0){\usebox{\plotpoint}}
\put(601.0,794.0){\usebox{\plotpoint}}
\put(601.0,795.0){\rule[-0.200pt]{0.482pt}{0.400pt}}
\put(603.0,795.0){\usebox{\plotpoint}}
\put(603.0,796.0){\rule[-0.200pt]{0.482pt}{0.400pt}}
\put(605.0,796.0){\usebox{\plotpoint}}
\put(605.0,797.0){\usebox{\plotpoint}}
\put(606.0,797.0){\usebox{\plotpoint}}
\put(606.0,798.0){\rule[-0.200pt]{0.482pt}{0.400pt}}
\put(608.0,798.0){\usebox{\plotpoint}}
\put(608.0,799.0){\rule[-0.200pt]{0.482pt}{0.400pt}}
\put(610.0,799.0){\usebox{\plotpoint}}
\put(611,799.67){\rule{0.241pt}{0.400pt}}
\multiput(611.00,799.17)(0.500,1.000){2}{\rule{0.120pt}{0.400pt}}
\put(610.0,800.0){\usebox{\plotpoint}}
\put(612,801){\usebox{\plotpoint}}
\put(612,801){\usebox{\plotpoint}}
\put(612,801){\usebox{\plotpoint}}
\put(612,801){\usebox{\plotpoint}}
\put(612,801){\usebox{\plotpoint}}
\put(612,801){\usebox{\plotpoint}}
\put(612,801){\usebox{\plotpoint}}
\put(612.0,801.0){\usebox{\plotpoint}}
\put(613.0,801.0){\usebox{\plotpoint}}
\put(613.0,802.0){\rule[-0.200pt]{0.482pt}{0.400pt}}
\put(615.0,802.0){\usebox{\plotpoint}}
\put(615.0,803.0){\rule[-0.200pt]{0.482pt}{0.400pt}}
\put(617.0,803.0){\usebox{\plotpoint}}
\put(617.0,804.0){\rule[-0.200pt]{0.482pt}{0.400pt}}
\put(619.0,804.0){\usebox{\plotpoint}}
\put(620,804.67){\rule{0.241pt}{0.400pt}}
\multiput(620.00,804.17)(0.500,1.000){2}{\rule{0.120pt}{0.400pt}}
\put(619.0,805.0){\usebox{\plotpoint}}
\put(621,806){\usebox{\plotpoint}}
\put(621,806){\usebox{\plotpoint}}
\put(621,806){\usebox{\plotpoint}}
\put(621,806){\usebox{\plotpoint}}
\put(621,806){\usebox{\plotpoint}}
\put(621,806){\usebox{\plotpoint}}
\put(621,806){\usebox{\plotpoint}}
\put(621.0,806.0){\usebox{\plotpoint}}
\put(622.0,806.0){\usebox{\plotpoint}}
\put(622.0,807.0){\rule[-0.200pt]{0.482pt}{0.400pt}}
\put(624.0,807.0){\usebox{\plotpoint}}
\put(624.0,808.0){\rule[-0.200pt]{0.482pt}{0.400pt}}
\put(626.0,808.0){\usebox{\plotpoint}}
\put(626.0,809.0){\rule[-0.200pt]{0.482pt}{0.400pt}}
\put(628.0,809.0){\usebox{\plotpoint}}
\put(628.0,810.0){\rule[-0.200pt]{0.482pt}{0.400pt}}
\put(630.0,810.0){\usebox{\plotpoint}}
\put(630.0,811.0){\rule[-0.200pt]{0.482pt}{0.400pt}}
\put(632.0,811.0){\usebox{\plotpoint}}
\put(632.0,812.0){\rule[-0.200pt]{0.482pt}{0.400pt}}
\put(634.0,812.0){\usebox{\plotpoint}}
\put(634.0,813.0){\rule[-0.200pt]{0.482pt}{0.400pt}}
\put(636.0,813.0){\usebox{\plotpoint}}
\put(636.0,814.0){\rule[-0.200pt]{0.482pt}{0.400pt}}
\put(638.0,814.0){\usebox{\plotpoint}}
\put(638.0,815.0){\rule[-0.200pt]{0.482pt}{0.400pt}}
\put(640.0,815.0){\usebox{\plotpoint}}
\put(640.0,816.0){\rule[-0.200pt]{0.482pt}{0.400pt}}
\put(642.0,816.0){\usebox{\plotpoint}}
\put(642.0,817.0){\rule[-0.200pt]{0.482pt}{0.400pt}}
\put(644.0,817.0){\usebox{\plotpoint}}
\put(644.0,818.0){\rule[-0.200pt]{0.482pt}{0.400pt}}
\put(646.0,818.0){\usebox{\plotpoint}}
\put(646.0,819.0){\rule[-0.200pt]{0.482pt}{0.400pt}}
\put(648.0,819.0){\usebox{\plotpoint}}
\put(648.0,820.0){\rule[-0.200pt]{0.482pt}{0.400pt}}
\put(650.0,820.0){\usebox{\plotpoint}}
\put(650.0,821.0){\rule[-0.200pt]{0.482pt}{0.400pt}}
\put(652.0,821.0){\usebox{\plotpoint}}
\put(652.0,822.0){\rule[-0.200pt]{0.723pt}{0.400pt}}
\put(655.0,822.0){\usebox{\plotpoint}}
\put(655.0,823.0){\rule[-0.200pt]{0.482pt}{0.400pt}}
\put(657.0,823.0){\usebox{\plotpoint}}
\put(657.0,824.0){\rule[-0.200pt]{0.482pt}{0.400pt}}
\put(659.0,824.0){\usebox{\plotpoint}}
\put(659.0,825.0){\rule[-0.200pt]{0.482pt}{0.400pt}}
\put(661.0,825.0){\usebox{\plotpoint}}
\put(661.0,826.0){\rule[-0.200pt]{0.723pt}{0.400pt}}
\put(664.0,826.0){\usebox{\plotpoint}}
\put(664.0,827.0){\rule[-0.200pt]{0.482pt}{0.400pt}}
\put(666.0,827.0){\usebox{\plotpoint}}
\put(668,827.67){\rule{0.241pt}{0.400pt}}
\multiput(668.00,827.17)(0.500,1.000){2}{\rule{0.120pt}{0.400pt}}
\put(666.0,828.0){\rule[-0.200pt]{0.482pt}{0.400pt}}
\put(669,829){\usebox{\plotpoint}}
\put(669,829){\usebox{\plotpoint}}
\put(669,829){\usebox{\plotpoint}}
\put(669,829){\usebox{\plotpoint}}
\put(669,829){\usebox{\plotpoint}}
\put(669,829){\usebox{\plotpoint}}
\put(669,829){\usebox{\plotpoint}}
\put(669.0,829.0){\rule[-0.200pt]{0.482pt}{0.400pt}}
\put(671.0,829.0){\usebox{\plotpoint}}
\put(671.0,830.0){\rule[-0.200pt]{0.482pt}{0.400pt}}
\put(673.0,830.0){\usebox{\plotpoint}}
\put(673.0,831.0){\rule[-0.200pt]{0.723pt}{0.400pt}}
\put(676.0,831.0){\usebox{\plotpoint}}
\put(678,831.67){\rule{0.241pt}{0.400pt}}
\multiput(678.00,831.17)(0.500,1.000){2}{\rule{0.120pt}{0.400pt}}
\put(676.0,832.0){\rule[-0.200pt]{0.482pt}{0.400pt}}
\put(679,833){\usebox{\plotpoint}}
\put(679,833){\usebox{\plotpoint}}
\put(679,833){\usebox{\plotpoint}}
\put(679,833){\usebox{\plotpoint}}
\put(679,833){\usebox{\plotpoint}}
\put(679,833){\usebox{\plotpoint}}
\put(679,833){\usebox{\plotpoint}}
\put(679.0,833.0){\rule[-0.200pt]{0.482pt}{0.400pt}}
\put(681.0,833.0){\usebox{\plotpoint}}
\put(681.0,834.0){\rule[-0.200pt]{0.723pt}{0.400pt}}
\put(684.0,834.0){\usebox{\plotpoint}}
\put(686,834.67){\rule{0.241pt}{0.400pt}}
\multiput(686.00,834.17)(0.500,1.000){2}{\rule{0.120pt}{0.400pt}}
\put(684.0,835.0){\rule[-0.200pt]{0.482pt}{0.400pt}}
\put(687,836){\usebox{\plotpoint}}
\put(687,836){\usebox{\plotpoint}}
\put(687,836){\usebox{\plotpoint}}
\put(687,836){\usebox{\plotpoint}}
\put(687,836){\usebox{\plotpoint}}
\put(687,836){\usebox{\plotpoint}}
\put(687,836){\usebox{\plotpoint}}
\put(687.0,836.0){\rule[-0.200pt]{0.482pt}{0.400pt}}
\put(689.0,836.0){\usebox{\plotpoint}}
\put(689.0,837.0){\rule[-0.200pt]{0.723pt}{0.400pt}}
\put(692.0,837.0){\usebox{\plotpoint}}
\put(692.0,838.0){\rule[-0.200pt]{0.723pt}{0.400pt}}
\put(695.0,838.0){\usebox{\plotpoint}}
\put(695.0,839.0){\rule[-0.200pt]{0.723pt}{0.400pt}}
\put(698.0,839.0){\usebox{\plotpoint}}
\put(698.0,840.0){\rule[-0.200pt]{0.723pt}{0.400pt}}
\put(701.0,840.0){\usebox{\plotpoint}}
\put(701.0,841.0){\rule[-0.200pt]{0.723pt}{0.400pt}}
\put(704.0,841.0){\usebox{\plotpoint}}
\put(704.0,842.0){\rule[-0.200pt]{0.723pt}{0.400pt}}
\put(707.0,842.0){\usebox{\plotpoint}}
\put(707.0,843.0){\rule[-0.200pt]{0.964pt}{0.400pt}}
\put(711.0,843.0){\usebox{\plotpoint}}
\put(711.0,844.0){\rule[-0.200pt]{0.723pt}{0.400pt}}
\put(714.0,844.0){\usebox{\plotpoint}}
\put(714.0,845.0){\rule[-0.200pt]{0.964pt}{0.400pt}}
\put(718.0,845.0){\usebox{\plotpoint}}
\put(718.0,846.0){\rule[-0.200pt]{0.723pt}{0.400pt}}
\put(721.0,846.0){\usebox{\plotpoint}}
\put(721.0,847.0){\rule[-0.200pt]{0.964pt}{0.400pt}}
\put(725.0,847.0){\usebox{\plotpoint}}
\put(725.0,848.0){\rule[-0.200pt]{0.964pt}{0.400pt}}
\put(729.0,848.0){\usebox{\plotpoint}}
\put(729.0,849.0){\rule[-0.200pt]{0.964pt}{0.400pt}}
\put(733.0,849.0){\usebox{\plotpoint}}
\put(733.0,850.0){\rule[-0.200pt]{1.204pt}{0.400pt}}
\put(738.0,850.0){\usebox{\plotpoint}}
\put(742,850.67){\rule{0.241pt}{0.400pt}}
\multiput(742.00,850.17)(0.500,1.000){2}{\rule{0.120pt}{0.400pt}}
\put(738.0,851.0){\rule[-0.200pt]{0.964pt}{0.400pt}}
\put(743,852){\usebox{\plotpoint}}
\put(743,852){\usebox{\plotpoint}}
\put(743,852){\usebox{\plotpoint}}
\put(743,852){\usebox{\plotpoint}}
\put(743,852){\usebox{\plotpoint}}
\put(743,852){\usebox{\plotpoint}}
\put(743,852){\usebox{\plotpoint}}
\put(747,851.67){\rule{0.241pt}{0.400pt}}
\multiput(747.00,851.17)(0.500,1.000){2}{\rule{0.120pt}{0.400pt}}
\put(743.0,852.0){\rule[-0.200pt]{0.964pt}{0.400pt}}
\put(748,853){\usebox{\plotpoint}}
\put(748,853){\usebox{\plotpoint}}
\put(748,853){\usebox{\plotpoint}}
\put(748,853){\usebox{\plotpoint}}
\put(748,853){\usebox{\plotpoint}}
\put(748,853){\usebox{\plotpoint}}
\put(748,853){\usebox{\plotpoint}}
\put(748.0,853.0){\rule[-0.200pt]{1.204pt}{0.400pt}}
\put(753.0,853.0){\usebox{\plotpoint}}
\put(753.0,854.0){\rule[-0.200pt]{1.445pt}{0.400pt}}
\put(759.0,854.0){\usebox{\plotpoint}}
\put(765,854.67){\rule{0.241pt}{0.400pt}}
\multiput(765.00,854.17)(0.500,1.000){2}{\rule{0.120pt}{0.400pt}}
\put(759.0,855.0){\rule[-0.200pt]{1.445pt}{0.400pt}}
\put(766,856){\usebox{\plotpoint}}
\put(766,856){\usebox{\plotpoint}}
\put(766,856){\usebox{\plotpoint}}
\put(766,856){\usebox{\plotpoint}}
\put(766,856){\usebox{\plotpoint}}
\put(766,856){\usebox{\plotpoint}}
\put(766,856){\usebox{\plotpoint}}
\put(766.0,856.0){\rule[-0.200pt]{1.686pt}{0.400pt}}
\put(773.0,856.0){\usebox{\plotpoint}}
\put(782,856.67){\rule{0.241pt}{0.400pt}}
\multiput(782.00,856.17)(0.500,1.000){2}{\rule{0.120pt}{0.400pt}}
\put(773.0,857.0){\rule[-0.200pt]{2.168pt}{0.400pt}}
\put(783,858){\usebox{\plotpoint}}
\put(783,858){\usebox{\plotpoint}}
\put(783,858){\usebox{\plotpoint}}
\put(783,858){\usebox{\plotpoint}}
\put(783,858){\usebox{\plotpoint}}
\put(783,858){\usebox{\plotpoint}}
\put(783,858){\usebox{\plotpoint}}
\put(783.0,858.0){\rule[-0.200pt]{3.132pt}{0.400pt}}
\put(796.0,858.0){\usebox{\plotpoint}}
\put(796.0,859.0){\rule[-0.200pt]{9.154pt}{0.400pt}}
\put(834.0,858.0){\usebox{\plotpoint}}
\put(847,856.67){\rule{0.241pt}{0.400pt}}
\multiput(847.00,857.17)(0.500,-1.000){2}{\rule{0.120pt}{0.400pt}}
\put(834.0,858.0){\rule[-0.200pt]{3.132pt}{0.400pt}}
\put(848,857){\usebox{\plotpoint}}
\put(848,857){\usebox{\plotpoint}}
\put(848,857){\usebox{\plotpoint}}
\put(848,857){\usebox{\plotpoint}}
\put(848,857){\usebox{\plotpoint}}
\put(848,857){\usebox{\plotpoint}}
\put(848,857){\usebox{\plotpoint}}
\put(848.0,857.0){\rule[-0.200pt]{2.168pt}{0.400pt}}
\put(857.0,856.0){\usebox{\plotpoint}}
\put(864,854.67){\rule{0.241pt}{0.400pt}}
\multiput(864.00,855.17)(0.500,-1.000){2}{\rule{0.120pt}{0.400pt}}
\put(857.0,856.0){\rule[-0.200pt]{1.686pt}{0.400pt}}
\put(865,855){\usebox{\plotpoint}}
\put(865,855){\usebox{\plotpoint}}
\put(865,855){\usebox{\plotpoint}}
\put(865,855){\usebox{\plotpoint}}
\put(865,855){\usebox{\plotpoint}}
\put(865,855){\usebox{\plotpoint}}
\put(865,855){\usebox{\plotpoint}}
\put(865.0,855.0){\rule[-0.200pt]{1.445pt}{0.400pt}}
\put(871.0,854.0){\usebox{\plotpoint}}
\put(871.0,854.0){\rule[-0.200pt]{1.445pt}{0.400pt}}
\put(877.0,853.0){\usebox{\plotpoint}}
\put(882,851.67){\rule{0.241pt}{0.400pt}}
\multiput(882.00,852.17)(0.500,-1.000){2}{\rule{0.120pt}{0.400pt}}
\put(877.0,853.0){\rule[-0.200pt]{1.204pt}{0.400pt}}
\put(883,852){\usebox{\plotpoint}}
\put(883,852){\usebox{\plotpoint}}
\put(883,852){\usebox{\plotpoint}}
\put(883,852){\usebox{\plotpoint}}
\put(883,852){\usebox{\plotpoint}}
\put(883,852){\usebox{\plotpoint}}
\put(883,852){\usebox{\plotpoint}}
\put(887,850.67){\rule{0.241pt}{0.400pt}}
\multiput(887.00,851.17)(0.500,-1.000){2}{\rule{0.120pt}{0.400pt}}
\put(883.0,852.0){\rule[-0.200pt]{0.964pt}{0.400pt}}
\put(888,851){\usebox{\plotpoint}}
\put(888,851){\usebox{\plotpoint}}
\put(888,851){\usebox{\plotpoint}}
\put(888,851){\usebox{\plotpoint}}
\put(888,851){\usebox{\plotpoint}}
\put(888,851){\usebox{\plotpoint}}
\put(888,851){\usebox{\plotpoint}}
\put(888.0,851.0){\rule[-0.200pt]{0.964pt}{0.400pt}}
\put(892.0,850.0){\usebox{\plotpoint}}
\put(892.0,850.0){\rule[-0.200pt]{1.204pt}{0.400pt}}
\put(897.0,849.0){\usebox{\plotpoint}}
\put(897.0,849.0){\rule[-0.200pt]{0.964pt}{0.400pt}}
\put(901.0,848.0){\usebox{\plotpoint}}
\put(901.0,848.0){\rule[-0.200pt]{0.964pt}{0.400pt}}
\put(905.0,847.0){\usebox{\plotpoint}}
\put(905.0,847.0){\rule[-0.200pt]{0.964pt}{0.400pt}}
\put(909.0,846.0){\usebox{\plotpoint}}
\put(909.0,846.0){\rule[-0.200pt]{0.723pt}{0.400pt}}
\put(912.0,845.0){\usebox{\plotpoint}}
\put(912.0,845.0){\rule[-0.200pt]{0.964pt}{0.400pt}}
\put(916.0,844.0){\usebox{\plotpoint}}
\put(916.0,844.0){\rule[-0.200pt]{0.723pt}{0.400pt}}
\put(919.0,843.0){\usebox{\plotpoint}}
\put(919.0,843.0){\rule[-0.200pt]{0.964pt}{0.400pt}}
\put(923.0,842.0){\usebox{\plotpoint}}
\put(923.0,842.0){\rule[-0.200pt]{0.723pt}{0.400pt}}
\put(926.0,841.0){\usebox{\plotpoint}}
\put(926.0,841.0){\rule[-0.200pt]{0.723pt}{0.400pt}}
\put(929.0,840.0){\usebox{\plotpoint}}
\put(929.0,840.0){\rule[-0.200pt]{0.723pt}{0.400pt}}
\put(932.0,839.0){\usebox{\plotpoint}}
\put(932.0,839.0){\rule[-0.200pt]{0.723pt}{0.400pt}}
\put(935.0,838.0){\usebox{\plotpoint}}
\put(935.0,838.0){\rule[-0.200pt]{0.723pt}{0.400pt}}
\put(938.0,837.0){\usebox{\plotpoint}}
\put(938.0,837.0){\rule[-0.200pt]{0.723pt}{0.400pt}}
\put(941.0,836.0){\usebox{\plotpoint}}
\put(943,834.67){\rule{0.241pt}{0.400pt}}
\multiput(943.00,835.17)(0.500,-1.000){2}{\rule{0.120pt}{0.400pt}}
\put(941.0,836.0){\rule[-0.200pt]{0.482pt}{0.400pt}}
\put(944,835){\usebox{\plotpoint}}
\put(944,835){\usebox{\plotpoint}}
\put(944,835){\usebox{\plotpoint}}
\put(944,835){\usebox{\plotpoint}}
\put(944,835){\usebox{\plotpoint}}
\put(944,835){\usebox{\plotpoint}}
\put(944,835){\usebox{\plotpoint}}
\put(944.0,835.0){\rule[-0.200pt]{0.482pt}{0.400pt}}
\put(946.0,834.0){\usebox{\plotpoint}}
\put(946.0,834.0){\rule[-0.200pt]{0.723pt}{0.400pt}}
\put(949.0,833.0){\usebox{\plotpoint}}
\put(951,831.67){\rule{0.241pt}{0.400pt}}
\multiput(951.00,832.17)(0.500,-1.000){2}{\rule{0.120pt}{0.400pt}}
\put(949.0,833.0){\rule[-0.200pt]{0.482pt}{0.400pt}}
\put(952,832){\usebox{\plotpoint}}
\put(952,832){\usebox{\plotpoint}}
\put(952,832){\usebox{\plotpoint}}
\put(952,832){\usebox{\plotpoint}}
\put(952,832){\usebox{\plotpoint}}
\put(952,832){\usebox{\plotpoint}}
\put(952,832){\usebox{\plotpoint}}
\put(952.0,832.0){\rule[-0.200pt]{0.482pt}{0.400pt}}
\put(954.0,831.0){\usebox{\plotpoint}}
\put(954.0,831.0){\rule[-0.200pt]{0.723pt}{0.400pt}}
\put(957.0,830.0){\usebox{\plotpoint}}
\put(957.0,830.0){\rule[-0.200pt]{0.482pt}{0.400pt}}
\put(959.0,829.0){\usebox{\plotpoint}}
\put(961,827.67){\rule{0.241pt}{0.400pt}}
\multiput(961.00,828.17)(0.500,-1.000){2}{\rule{0.120pt}{0.400pt}}
\put(959.0,829.0){\rule[-0.200pt]{0.482pt}{0.400pt}}
\put(962,828){\usebox{\plotpoint}}
\put(962,828){\usebox{\plotpoint}}
\put(962,828){\usebox{\plotpoint}}
\put(962,828){\usebox{\plotpoint}}
\put(962,828){\usebox{\plotpoint}}
\put(962,828){\usebox{\plotpoint}}
\put(962,828){\usebox{\plotpoint}}
\put(962.0,828.0){\rule[-0.200pt]{0.482pt}{0.400pt}}
\put(964.0,827.0){\usebox{\plotpoint}}
\put(964.0,827.0){\rule[-0.200pt]{0.482pt}{0.400pt}}
\put(966.0,826.0){\usebox{\plotpoint}}
\put(966.0,826.0){\rule[-0.200pt]{0.723pt}{0.400pt}}
\put(969.0,825.0){\usebox{\plotpoint}}
\put(969.0,825.0){\rule[-0.200pt]{0.482pt}{0.400pt}}
\put(971.0,824.0){\usebox{\plotpoint}}
\put(971.0,824.0){\rule[-0.200pt]{0.482pt}{0.400pt}}
\put(973.0,823.0){\usebox{\plotpoint}}
\put(973.0,823.0){\rule[-0.200pt]{0.482pt}{0.400pt}}
\put(975.0,822.0){\usebox{\plotpoint}}
\put(975.0,822.0){\rule[-0.200pt]{0.723pt}{0.400pt}}
\put(978.0,821.0){\usebox{\plotpoint}}
\put(978.0,821.0){\rule[-0.200pt]{0.482pt}{0.400pt}}
\put(980.0,820.0){\usebox{\plotpoint}}
\put(980.0,820.0){\rule[-0.200pt]{0.482pt}{0.400pt}}
\put(982.0,819.0){\usebox{\plotpoint}}
\put(982.0,819.0){\rule[-0.200pt]{0.482pt}{0.400pt}}
\put(984.0,818.0){\usebox{\plotpoint}}
\put(984.0,818.0){\rule[-0.200pt]{0.482pt}{0.400pt}}
\put(986.0,817.0){\usebox{\plotpoint}}
\put(986.0,817.0){\rule[-0.200pt]{0.482pt}{0.400pt}}
\put(988.0,816.0){\usebox{\plotpoint}}
\put(988.0,816.0){\rule[-0.200pt]{0.482pt}{0.400pt}}
\put(990.0,815.0){\usebox{\plotpoint}}
\put(990.0,815.0){\rule[-0.200pt]{0.482pt}{0.400pt}}
\put(992.0,814.0){\usebox{\plotpoint}}
\put(992.0,814.0){\rule[-0.200pt]{0.482pt}{0.400pt}}
\put(994.0,813.0){\usebox{\plotpoint}}
\put(994.0,813.0){\rule[-0.200pt]{0.482pt}{0.400pt}}
\put(996.0,812.0){\usebox{\plotpoint}}
\put(996.0,812.0){\rule[-0.200pt]{0.482pt}{0.400pt}}
\put(998.0,811.0){\usebox{\plotpoint}}
\put(998.0,811.0){\rule[-0.200pt]{0.482pt}{0.400pt}}
\put(1000.0,810.0){\usebox{\plotpoint}}
\put(1000.0,810.0){\rule[-0.200pt]{0.482pt}{0.400pt}}
\put(1002.0,809.0){\usebox{\plotpoint}}
\put(1002.0,809.0){\rule[-0.200pt]{0.482pt}{0.400pt}}
\put(1004.0,808.0){\usebox{\plotpoint}}
\put(1004.0,808.0){\rule[-0.200pt]{0.482pt}{0.400pt}}
\put(1006.0,807.0){\usebox{\plotpoint}}
\put(1006.0,807.0){\rule[-0.200pt]{0.482pt}{0.400pt}}
\put(1008.0,806.0){\usebox{\plotpoint}}
\put(1009,804.67){\rule{0.241pt}{0.400pt}}
\multiput(1009.00,805.17)(0.500,-1.000){2}{\rule{0.120pt}{0.400pt}}
\put(1008.0,806.0){\usebox{\plotpoint}}
\put(1010,805){\usebox{\plotpoint}}
\put(1010,805){\usebox{\plotpoint}}
\put(1010,805){\usebox{\plotpoint}}
\put(1010,805){\usebox{\plotpoint}}
\put(1010,805){\usebox{\plotpoint}}
\put(1010,805){\usebox{\plotpoint}}
\put(1010,805){\usebox{\plotpoint}}
\put(1010.0,805.0){\usebox{\plotpoint}}
\put(1011.0,804.0){\usebox{\plotpoint}}
\put(1011.0,804.0){\rule[-0.200pt]{0.482pt}{0.400pt}}
\put(1013.0,803.0){\usebox{\plotpoint}}
\put(1013.0,803.0){\rule[-0.200pt]{0.482pt}{0.400pt}}
\put(1015.0,802.0){\usebox{\plotpoint}}
\put(1015.0,802.0){\rule[-0.200pt]{0.482pt}{0.400pt}}
\put(1017.0,801.0){\usebox{\plotpoint}}
\put(1018,799.67){\rule{0.241pt}{0.400pt}}
\multiput(1018.00,800.17)(0.500,-1.000){2}{\rule{0.120pt}{0.400pt}}
\put(1017.0,801.0){\usebox{\plotpoint}}
\put(1019,800){\usebox{\plotpoint}}
\put(1019,800){\usebox{\plotpoint}}
\put(1019,800){\usebox{\plotpoint}}
\put(1019,800){\usebox{\plotpoint}}
\put(1019,800){\usebox{\plotpoint}}
\put(1019,800){\usebox{\plotpoint}}
\put(1019,800){\usebox{\plotpoint}}
\put(1019.0,800.0){\usebox{\plotpoint}}
\put(1020.0,799.0){\usebox{\plotpoint}}
\put(1020.0,799.0){\rule[-0.200pt]{0.482pt}{0.400pt}}
\put(1022.0,798.0){\usebox{\plotpoint}}
\put(1022.0,798.0){\rule[-0.200pt]{0.482pt}{0.400pt}}
\put(1024.0,797.0){\usebox{\plotpoint}}
\put(1024.0,797.0){\usebox{\plotpoint}}
\put(1025.0,796.0){\usebox{\plotpoint}}
\put(1025.0,796.0){\rule[-0.200pt]{0.482pt}{0.400pt}}
\put(1027.0,795.0){\usebox{\plotpoint}}
\put(1027.0,795.0){\rule[-0.200pt]{0.482pt}{0.400pt}}
\put(1029.0,794.0){\usebox{\plotpoint}}
\put(1029.0,794.0){\usebox{\plotpoint}}
\put(1030.0,793.0){\usebox{\plotpoint}}
\put(1030.0,793.0){\rule[-0.200pt]{0.482pt}{0.400pt}}
\put(1032.0,792.0){\usebox{\plotpoint}}
\put(1032.0,792.0){\rule[-0.200pt]{0.482pt}{0.400pt}}
\put(1034.0,791.0){\usebox{\plotpoint}}
\put(1034.0,791.0){\usebox{\plotpoint}}
\put(1035.0,790.0){\usebox{\plotpoint}}
\put(1035.0,790.0){\rule[-0.200pt]{0.482pt}{0.400pt}}
\put(1037.0,789.0){\usebox{\plotpoint}}
\put(1038,787.67){\rule{0.241pt}{0.400pt}}
\multiput(1038.00,788.17)(0.500,-1.000){2}{\rule{0.120pt}{0.400pt}}
\put(1037.0,789.0){\usebox{\plotpoint}}
\put(1039,788){\usebox{\plotpoint}}
\put(1039,788){\usebox{\plotpoint}}
\put(1039,788){\usebox{\plotpoint}}
\put(1039,788){\usebox{\plotpoint}}
\put(1039,788){\usebox{\plotpoint}}
\put(1039,788){\usebox{\plotpoint}}
\put(1039,788){\usebox{\plotpoint}}
\put(1039.0,788.0){\usebox{\plotpoint}}
\put(1040.0,787.0){\usebox{\plotpoint}}
\put(1040.0,787.0){\rule[-0.200pt]{0.482pt}{0.400pt}}
\put(1042.0,786.0){\usebox{\plotpoint}}
\put(1042.0,786.0){\usebox{\plotpoint}}
\put(1043.0,785.0){\usebox{\plotpoint}}
\put(1043.0,785.0){\rule[-0.200pt]{0.482pt}{0.400pt}}
\put(1045.0,784.0){\usebox{\plotpoint}}
\put(1045.0,784.0){\usebox{\plotpoint}}
\put(1046.0,783.0){\usebox{\plotpoint}}
\put(1046.0,783.0){\rule[-0.200pt]{0.482pt}{0.400pt}}
\put(1048.0,782.0){\usebox{\plotpoint}}
\put(1049,780.67){\rule{0.241pt}{0.400pt}}
\multiput(1049.00,781.17)(0.500,-1.000){2}{\rule{0.120pt}{0.400pt}}
\put(1048.0,782.0){\usebox{\plotpoint}}
\put(1050,781){\usebox{\plotpoint}}
\put(1050,781){\usebox{\plotpoint}}
\put(1050,781){\usebox{\plotpoint}}
\put(1050,781){\usebox{\plotpoint}}
\put(1050,781){\usebox{\plotpoint}}
\put(1050,781){\usebox{\plotpoint}}
\put(1050,781){\usebox{\plotpoint}}
\put(1050.0,781.0){\usebox{\plotpoint}}
\put(1051.0,780.0){\usebox{\plotpoint}}
\put(1052,778.67){\rule{0.241pt}{0.400pt}}
\multiput(1052.00,779.17)(0.500,-1.000){2}{\rule{0.120pt}{0.400pt}}
\put(1051.0,780.0){\usebox{\plotpoint}}
\put(1053,779){\usebox{\plotpoint}}
\put(1053,779){\usebox{\plotpoint}}
\put(1053,779){\usebox{\plotpoint}}
\put(1053,779){\usebox{\plotpoint}}
\put(1053,779){\usebox{\plotpoint}}
\put(1053,779){\usebox{\plotpoint}}
\put(1053,779){\usebox{\plotpoint}}
\put(1053.0,779.0){\usebox{\plotpoint}}
\put(1054.0,778.0){\usebox{\plotpoint}}
\put(1055,776.67){\rule{0.241pt}{0.400pt}}
\multiput(1055.00,777.17)(0.500,-1.000){2}{\rule{0.120pt}{0.400pt}}
\put(1054.0,778.0){\usebox{\plotpoint}}
\put(1056,777){\usebox{\plotpoint}}
\put(1056,777){\usebox{\plotpoint}}
\put(1056,777){\usebox{\plotpoint}}
\put(1056,777){\usebox{\plotpoint}}
\put(1056,777){\usebox{\plotpoint}}
\put(1056,777){\usebox{\plotpoint}}
\put(1056,777){\usebox{\plotpoint}}
\put(1056.0,777.0){\usebox{\plotpoint}}
\put(1057.0,776.0){\usebox{\plotpoint}}
\put(1058,774.67){\rule{0.241pt}{0.400pt}}
\multiput(1058.00,775.17)(0.500,-1.000){2}{\rule{0.120pt}{0.400pt}}
\put(1057.0,776.0){\usebox{\plotpoint}}
\put(1059,775){\usebox{\plotpoint}}
\put(1059,775){\usebox{\plotpoint}}
\put(1059,775){\usebox{\plotpoint}}
\put(1059,775){\usebox{\plotpoint}}
\put(1059,775){\usebox{\plotpoint}}
\put(1059,775){\usebox{\plotpoint}}
\put(1059,775){\usebox{\plotpoint}}
\put(1059.0,775.0){\usebox{\plotpoint}}
\put(1060.0,774.0){\usebox{\plotpoint}}
\put(1061,772.67){\rule{0.241pt}{0.400pt}}
\multiput(1061.00,773.17)(0.500,-1.000){2}{\rule{0.120pt}{0.400pt}}
\put(1060.0,774.0){\usebox{\plotpoint}}
\put(1062,773){\usebox{\plotpoint}}
\put(1062,773){\usebox{\plotpoint}}
\put(1062,773){\usebox{\plotpoint}}
\put(1062,773){\usebox{\plotpoint}}
\put(1062,773){\usebox{\plotpoint}}
\put(1062,773){\usebox{\plotpoint}}
\put(1062,773){\usebox{\plotpoint}}
\put(1062.0,773.0){\usebox{\plotpoint}}
\put(1063.0,772.0){\usebox{\plotpoint}}
\put(1063.0,772.0){\usebox{\plotpoint}}
\put(1064.0,771.0){\usebox{\plotpoint}}
\put(1064.0,771.0){\rule[-0.200pt]{0.482pt}{0.400pt}}
\put(1066.0,770.0){\usebox{\plotpoint}}
\put(1066.0,770.0){\usebox{\plotpoint}}
\put(1067.0,769.0){\usebox{\plotpoint}}
\put(1067.0,769.0){\rule[-0.200pt]{0.482pt}{0.400pt}}
\put(1069.0,768.0){\usebox{\plotpoint}}
\put(1069.0,768.0){\usebox{\plotpoint}}
\put(1070.0,767.0){\usebox{\plotpoint}}
\put(1071,765.67){\rule{0.241pt}{0.400pt}}
\multiput(1071.00,766.17)(0.500,-1.000){2}{\rule{0.120pt}{0.400pt}}
\put(1070.0,767.0){\usebox{\plotpoint}}
\put(1072,766){\usebox{\plotpoint}}
\put(1072,766){\usebox{\plotpoint}}
\put(1072,766){\usebox{\plotpoint}}
\put(1072,766){\usebox{\plotpoint}}
\put(1072,766){\usebox{\plotpoint}}
\put(1072,766){\usebox{\plotpoint}}
\put(1072,766){\usebox{\plotpoint}}
\put(1072.0,766.0){\usebox{\plotpoint}}
\put(1073.0,765.0){\usebox{\plotpoint}}
\put(1073.0,765.0){\usebox{\plotpoint}}
\put(1074.0,764.0){\usebox{\plotpoint}}
\put(1074.0,764.0){\rule[-0.200pt]{0.482pt}{0.400pt}}
\put(1076.0,763.0){\usebox{\plotpoint}}
\put(1076.0,763.0){\usebox{\plotpoint}}
\put(1077.0,762.0){\usebox{\plotpoint}}
\put(1078,760.67){\rule{0.241pt}{0.400pt}}
\multiput(1078.00,761.17)(0.500,-1.000){2}{\rule{0.120pt}{0.400pt}}
\put(1077.0,762.0){\usebox{\plotpoint}}
\put(1079,761){\usebox{\plotpoint}}
\put(1079,761){\usebox{\plotpoint}}
\put(1079,761){\usebox{\plotpoint}}
\put(1079,761){\usebox{\plotpoint}}
\put(1079,761){\usebox{\plotpoint}}
\put(1079,761){\usebox{\plotpoint}}
\put(1079,761){\usebox{\plotpoint}}
\put(1079.0,761.0){\usebox{\plotpoint}}
\put(1080.0,760.0){\usebox{\plotpoint}}
\put(1080.0,760.0){\usebox{\plotpoint}}
\put(1081.0,759.0){\usebox{\plotpoint}}
\put(1081.0,759.0){\rule[-0.200pt]{0.482pt}{0.400pt}}
\put(1083.0,758.0){\usebox{\plotpoint}}
\put(1083.0,758.0){\usebox{\plotpoint}}
\put(1084.0,757.0){\usebox{\plotpoint}}
\put(1084.0,757.0){\usebox{\plotpoint}}
\put(1085.0,756.0){\usebox{\plotpoint}}
\put(1085.0,756.0){\rule[-0.200pt]{0.482pt}{0.400pt}}
\put(1087.0,755.0){\usebox{\plotpoint}}
\put(1087.0,755.0){\usebox{\plotpoint}}
\put(1088.0,754.0){\usebox{\plotpoint}}
\put(1088.0,754.0){\usebox{\plotpoint}}
\put(1089.0,753.0){\usebox{\plotpoint}}
\put(1089.0,753.0){\rule[-0.200pt]{0.482pt}{0.400pt}}
\put(1091.0,752.0){\usebox{\plotpoint}}
\put(1091.0,752.0){\usebox{\plotpoint}}
\put(1092.0,751.0){\usebox{\plotpoint}}
\put(1092.0,751.0){\usebox{\plotpoint}}
\put(1093.0,750.0){\usebox{\plotpoint}}
\put(1094,748.67){\rule{0.241pt}{0.400pt}}
\multiput(1094.00,749.17)(0.500,-1.000){2}{\rule{0.120pt}{0.400pt}}
\put(1093.0,750.0){\usebox{\plotpoint}}
\put(1095,749){\usebox{\plotpoint}}
\put(1095,749){\usebox{\plotpoint}}
\put(1095,749){\usebox{\plotpoint}}
\put(1095,749){\usebox{\plotpoint}}
\put(1095,749){\usebox{\plotpoint}}
\put(1095,749){\usebox{\plotpoint}}
\put(1095,749){\usebox{\plotpoint}}
\put(1095.0,749.0){\usebox{\plotpoint}}
\put(1096.0,748.0){\usebox{\plotpoint}}
\put(1096.0,748.0){\usebox{\plotpoint}}
\put(1097.0,747.0){\usebox{\plotpoint}}
\put(1097.0,747.0){\usebox{\plotpoint}}
\put(1098.0,746.0){\usebox{\plotpoint}}
\put(1098.0,746.0){\rule[-0.200pt]{0.482pt}{0.400pt}}
\put(1100.0,745.0){\usebox{\plotpoint}}
\put(1100.0,745.0){\usebox{\plotpoint}}
\put(1101.0,744.0){\usebox{\plotpoint}}
\put(1101.0,744.0){\usebox{\plotpoint}}
\put(1102.0,743.0){\usebox{\plotpoint}}
\put(1103,741.67){\rule{0.241pt}{0.400pt}}
\multiput(1103.00,742.17)(0.500,-1.000){2}{\rule{0.120pt}{0.400pt}}
\put(1102.0,743.0){\usebox{\plotpoint}}
\put(1104,742){\usebox{\plotpoint}}
\put(1104,742){\usebox{\plotpoint}}
\put(1104,742){\usebox{\plotpoint}}
\put(1104,742){\usebox{\plotpoint}}
\put(1104,742){\usebox{\plotpoint}}
\put(1104,742){\usebox{\plotpoint}}
\put(1104,742){\usebox{\plotpoint}}
\put(1104.0,742.0){\usebox{\plotpoint}}
\put(1105.0,741.0){\usebox{\plotpoint}}
\put(1105.0,741.0){\usebox{\plotpoint}}
\put(1106.0,740.0){\usebox{\plotpoint}}
\put(1106.0,740.0){\usebox{\plotpoint}}
\put(1107.0,739.0){\usebox{\plotpoint}}
\put(1107.0,739.0){\usebox{\plotpoint}}
\put(1108.0,738.0){\usebox{\plotpoint}}
\put(1108.0,738.0){\rule[-0.200pt]{0.482pt}{0.400pt}}
\put(1110.0,737.0){\usebox{\plotpoint}}
\put(1110.0,737.0){\usebox{\plotpoint}}
\put(1111.0,736.0){\usebox{\plotpoint}}
\put(1111.0,736.0){\usebox{\plotpoint}}
\put(1112.0,735.0){\usebox{\plotpoint}}
\put(1112.0,735.0){\usebox{\plotpoint}}
\put(1113.0,734.0){\usebox{\plotpoint}}
\put(1113.0,734.0){\rule[-0.200pt]{0.482pt}{0.400pt}}
\put(1115.0,733.0){\usebox{\plotpoint}}
\put(1115.0,733.0){\usebox{\plotpoint}}
\put(1116.0,732.0){\usebox{\plotpoint}}
\put(1116.0,732.0){\usebox{\plotpoint}}
\put(1117.0,731.0){\usebox{\plotpoint}}
\put(1117.0,731.0){\usebox{\plotpoint}}
\put(1118.0,730.0){\usebox{\plotpoint}}
\put(1118.0,730.0){\usebox{\plotpoint}}
\put(1119.0,729.0){\usebox{\plotpoint}}
\put(1119.0,729.0){\rule[-0.200pt]{0.482pt}{0.400pt}}
\put(1121.0,728.0){\usebox{\plotpoint}}
\put(1121.0,728.0){\usebox{\plotpoint}}
\put(1122.0,727.0){\usebox{\plotpoint}}
\put(1122.0,727.0){\usebox{\plotpoint}}
\put(1123.0,726.0){\usebox{\plotpoint}}
\put(1123.0,726.0){\usebox{\plotpoint}}
\put(1124.0,725.0){\usebox{\plotpoint}}
\put(1124.0,725.0){\usebox{\plotpoint}}
\put(1125.0,724.0){\usebox{\plotpoint}}
\put(1126,722.67){\rule{0.241pt}{0.400pt}}
\multiput(1126.00,723.17)(0.500,-1.000){2}{\rule{0.120pt}{0.400pt}}
\put(1125.0,724.0){\usebox{\plotpoint}}
\put(1127,723){\usebox{\plotpoint}}
\put(1127,723){\usebox{\plotpoint}}
\put(1127,723){\usebox{\plotpoint}}
\put(1127,723){\usebox{\plotpoint}}
\put(1127,723){\usebox{\plotpoint}}
\put(1127,723){\usebox{\plotpoint}}
\put(1127,723){\usebox{\plotpoint}}
\put(1127.0,723.0){\usebox{\plotpoint}}
\put(1128.0,722.0){\usebox{\plotpoint}}
\put(1128.0,722.0){\usebox{\plotpoint}}
\put(1129.0,721.0){\usebox{\plotpoint}}
\put(1129.0,721.0){\usebox{\plotpoint}}
\put(1130.0,720.0){\usebox{\plotpoint}}
\put(1130.0,720.0){\usebox{\plotpoint}}
\put(1131.0,719.0){\usebox{\plotpoint}}
\put(1131.0,719.0){\usebox{\plotpoint}}
\put(1132.0,718.0){\usebox{\plotpoint}}
\put(1133,716.67){\rule{0.241pt}{0.400pt}}
\multiput(1133.00,717.17)(0.500,-1.000){2}{\rule{0.120pt}{0.400pt}}
\put(1132.0,718.0){\usebox{\plotpoint}}
\put(1134,717){\usebox{\plotpoint}}
\put(1134,717){\usebox{\plotpoint}}
\put(1134,717){\usebox{\plotpoint}}
\put(1134,717){\usebox{\plotpoint}}
\put(1134,717){\usebox{\plotpoint}}
\put(1134,717){\usebox{\plotpoint}}
\put(1134,717){\usebox{\plotpoint}}
\put(1134.0,717.0){\usebox{\plotpoint}}
\put(1135.0,716.0){\usebox{\plotpoint}}
\put(1135.0,716.0){\usebox{\plotpoint}}
\put(1136.0,715.0){\usebox{\plotpoint}}
\put(1136.0,715.0){\usebox{\plotpoint}}
\put(1137.0,714.0){\usebox{\plotpoint}}
\put(1137.0,714.0){\usebox{\plotpoint}}
\put(1138.0,713.0){\usebox{\plotpoint}}
\put(1138.0,713.0){\usebox{\plotpoint}}
\put(1139.0,712.0){\usebox{\plotpoint}}
\put(1139.0,712.0){\usebox{\plotpoint}}
\put(1140.0,711.0){\usebox{\plotpoint}}
\put(1141,709.67){\rule{0.241pt}{0.400pt}}
\multiput(1141.00,710.17)(0.500,-1.000){2}{\rule{0.120pt}{0.400pt}}
\put(1140.0,711.0){\usebox{\plotpoint}}
\put(1142,710){\usebox{\plotpoint}}
\put(1142,710){\usebox{\plotpoint}}
\put(1142,710){\usebox{\plotpoint}}
\put(1142,710){\usebox{\plotpoint}}
\put(1142,710){\usebox{\plotpoint}}
\put(1142,710){\usebox{\plotpoint}}
\put(1142,710){\usebox{\plotpoint}}
\put(1142.0,710.0){\usebox{\plotpoint}}
\put(1143.0,709.0){\usebox{\plotpoint}}
\put(1143.0,709.0){\usebox{\plotpoint}}
\put(1144.0,708.0){\usebox{\plotpoint}}
\put(1144.0,708.0){\usebox{\plotpoint}}
\put(1145.0,707.0){\usebox{\plotpoint}}
\put(1145.0,707.0){\usebox{\plotpoint}}
\put(1146.0,706.0){\usebox{\plotpoint}}
\put(1146.0,706.0){\usebox{\plotpoint}}
\put(1147.0,705.0){\usebox{\plotpoint}}
\put(1147.0,705.0){\usebox{\plotpoint}}
\put(1148.0,704.0){\usebox{\plotpoint}}
\put(1148.0,704.0){\usebox{\plotpoint}}
\put(1149.0,703.0){\usebox{\plotpoint}}
\put(1150,701.67){\rule{0.241pt}{0.400pt}}
\multiput(1150.00,702.17)(0.500,-1.000){2}{\rule{0.120pt}{0.400pt}}
\put(1149.0,703.0){\usebox{\plotpoint}}
\put(1151,702){\usebox{\plotpoint}}
\put(1151,702){\usebox{\plotpoint}}
\put(1151,702){\usebox{\plotpoint}}
\put(1151,702){\usebox{\plotpoint}}
\put(1151,702){\usebox{\plotpoint}}
\put(1151,702){\usebox{\plotpoint}}
\put(1151,702){\usebox{\plotpoint}}
\put(1151,700.67){\rule{0.241pt}{0.400pt}}
\multiput(1151.00,701.17)(0.500,-1.000){2}{\rule{0.120pt}{0.400pt}}
\put(1152,701){\usebox{\plotpoint}}
\put(1152,701){\usebox{\plotpoint}}
\put(1152,701){\usebox{\plotpoint}}
\put(1152,701){\usebox{\plotpoint}}
\put(1152,701){\usebox{\plotpoint}}
\put(1152,701){\usebox{\plotpoint}}
\put(1152,701){\usebox{\plotpoint}}
\put(1152.0,701.0){\usebox{\plotpoint}}
\put(1153.0,700.0){\usebox{\plotpoint}}
\put(1153.0,700.0){\usebox{\plotpoint}}
\put(1154.0,699.0){\usebox{\plotpoint}}
\put(1154.0,699.0){\usebox{\plotpoint}}
\put(1155.0,698.0){\usebox{\plotpoint}}
\put(1155.0,698.0){\usebox{\plotpoint}}
\put(1156.0,697.0){\usebox{\plotpoint}}
\put(1156.0,697.0){\usebox{\plotpoint}}
\put(1157.0,696.0){\usebox{\plotpoint}}
\put(1157.0,696.0){\usebox{\plotpoint}}
\put(1158.0,695.0){\usebox{\plotpoint}}
\put(1158.0,695.0){\usebox{\plotpoint}}
\put(1159.0,694.0){\usebox{\plotpoint}}
\put(1159.0,694.0){\usebox{\plotpoint}}
\put(1160.0,693.0){\usebox{\plotpoint}}
\put(1160.0,693.0){\usebox{\plotpoint}}
\put(1161.0,692.0){\usebox{\plotpoint}}
\put(1161.0,692.0){\usebox{\plotpoint}}
\put(1162.0,691.0){\usebox{\plotpoint}}
\put(1163,689.67){\rule{0.241pt}{0.400pt}}
\multiput(1163.00,690.17)(0.500,-1.000){2}{\rule{0.120pt}{0.400pt}}
\put(1162.0,691.0){\usebox{\plotpoint}}
\put(1164,690){\usebox{\plotpoint}}
\put(1164,690){\usebox{\plotpoint}}
\put(1164,690){\usebox{\plotpoint}}
\put(1164,690){\usebox{\plotpoint}}
\put(1164,690){\usebox{\plotpoint}}
\put(1164,690){\usebox{\plotpoint}}
\put(1164,690){\usebox{\plotpoint}}
\put(1164,688.67){\rule{0.241pt}{0.400pt}}
\multiput(1164.00,689.17)(0.500,-1.000){2}{\rule{0.120pt}{0.400pt}}
\put(1165,689){\usebox{\plotpoint}}
\put(1165,689){\usebox{\plotpoint}}
\put(1165,689){\usebox{\plotpoint}}
\put(1165,689){\usebox{\plotpoint}}
\put(1165,689){\usebox{\plotpoint}}
\put(1165,689){\usebox{\plotpoint}}
\put(1165,689){\usebox{\plotpoint}}
\put(1165.0,689.0){\usebox{\plotpoint}}
\put(1166.0,688.0){\usebox{\plotpoint}}
\put(1166.0,688.0){\usebox{\plotpoint}}
\put(1167.0,687.0){\usebox{\plotpoint}}
\put(1167.0,687.0){\usebox{\plotpoint}}
\put(1168.0,686.0){\usebox{\plotpoint}}
\put(1168.0,686.0){\usebox{\plotpoint}}
\put(1169.0,685.0){\usebox{\plotpoint}}
\put(1169.0,685.0){\usebox{\plotpoint}}
\put(1170.0,684.0){\usebox{\plotpoint}}
\put(1170.0,684.0){\usebox{\plotpoint}}
\put(1171.0,683.0){\usebox{\plotpoint}}
\put(1171.0,683.0){\usebox{\plotpoint}}
\put(1172.0,682.0){\usebox{\plotpoint}}
\put(1172.0,682.0){\usebox{\plotpoint}}
\put(1173.0,681.0){\usebox{\plotpoint}}
\put(1173.0,681.0){\usebox{\plotpoint}}
\put(1174.0,680.0){\usebox{\plotpoint}}
\put(1174.0,680.0){\usebox{\plotpoint}}
\put(1175.0,679.0){\usebox{\plotpoint}}
\put(1175.0,679.0){\usebox{\plotpoint}}
\put(1176.0,678.0){\usebox{\plotpoint}}
\put(1176.0,678.0){\usebox{\plotpoint}}
\put(1177.0,677.0){\usebox{\plotpoint}}
\put(1177.0,677.0){\usebox{\plotpoint}}
\put(1178.0,676.0){\usebox{\plotpoint}}
\put(1178.0,676.0){\usebox{\plotpoint}}
\put(1179.0,675.0){\usebox{\plotpoint}}
\put(1179.0,675.0){\usebox{\plotpoint}}
\put(1180.0,674.0){\usebox{\plotpoint}}
\put(1180.0,674.0){\usebox{\plotpoint}}
\put(1181.0,673.0){\usebox{\plotpoint}}
\put(1181.0,673.0){\usebox{\plotpoint}}
\put(1182.0,672.0){\usebox{\plotpoint}}
\put(1182.0,672.0){\usebox{\plotpoint}}
\put(1183.0,671.0){\usebox{\plotpoint}}
\put(1183.0,671.0){\usebox{\plotpoint}}
\put(1184.0,670.0){\usebox{\plotpoint}}
\put(1184.0,670.0){\usebox{\plotpoint}}
\put(1185.0,669.0){\usebox{\plotpoint}}
\put(1185.0,669.0){\usebox{\plotpoint}}
\put(1186.0,668.0){\usebox{\plotpoint}}
\put(1186.0,668.0){\usebox{\plotpoint}}
\put(1187.0,667.0){\usebox{\plotpoint}}
\put(1187.0,667.0){\usebox{\plotpoint}}
\put(1188.0,666.0){\usebox{\plotpoint}}
\put(1188.0,666.0){\usebox{\plotpoint}}
\put(1189.0,665.0){\usebox{\plotpoint}}
\put(1189.0,665.0){\usebox{\plotpoint}}
\put(1190.0,664.0){\usebox{\plotpoint}}
\put(1190.0,664.0){\usebox{\plotpoint}}
\put(1191.0,663.0){\usebox{\plotpoint}}
\put(1191.0,663.0){\usebox{\plotpoint}}
\put(1192.0,662.0){\usebox{\plotpoint}}
\put(1192.0,662.0){\usebox{\plotpoint}}
\put(1193.0,661.0){\usebox{\plotpoint}}
\put(1193.0,661.0){\usebox{\plotpoint}}
\put(1194.0,660.0){\usebox{\plotpoint}}
\put(1194.0,660.0){\usebox{\plotpoint}}
\put(1195.0,659.0){\usebox{\plotpoint}}
\put(1195.0,659.0){\usebox{\plotpoint}}
\put(1196.0,658.0){\usebox{\plotpoint}}
\put(1196.0,658.0){\usebox{\plotpoint}}
\put(1197.0,657.0){\usebox{\plotpoint}}
\put(1197.0,657.0){\usebox{\plotpoint}}
\put(1198.0,656.0){\usebox{\plotpoint}}
\put(1198.0,656.0){\usebox{\plotpoint}}
\put(1199.0,655.0){\usebox{\plotpoint}}
\put(1199.0,655.0){\usebox{\plotpoint}}
\put(1200.0,654.0){\usebox{\plotpoint}}
\put(1200.0,654.0){\usebox{\plotpoint}}
\put(1201.0,653.0){\usebox{\plotpoint}}
\put(1201.0,653.0){\usebox{\plotpoint}}
\put(1202.0,652.0){\usebox{\plotpoint}}
\put(1202.0,652.0){\usebox{\plotpoint}}
\put(1203.0,651.0){\usebox{\plotpoint}}
\put(1203.0,651.0){\usebox{\plotpoint}}
\put(1204.0,650.0){\usebox{\plotpoint}}
\put(1204.0,650.0){\usebox{\plotpoint}}
\put(1205.0,649.0){\usebox{\plotpoint}}
\put(1205.0,649.0){\usebox{\plotpoint}}
\put(1206.0,648.0){\usebox{\plotpoint}}
\put(1206.0,648.0){\usebox{\plotpoint}}
\put(1207.0,647.0){\usebox{\plotpoint}}
\put(1207.0,647.0){\usebox{\plotpoint}}
\put(1208.0,646.0){\usebox{\plotpoint}}
\put(1208.0,646.0){\usebox{\plotpoint}}
\put(1209.0,645.0){\usebox{\plotpoint}}
\put(1209.0,645.0){\usebox{\plotpoint}}
\put(1210.0,644.0){\usebox{\plotpoint}}
\put(1210.0,644.0){\usebox{\plotpoint}}
\put(1211,641.67){\rule{0.241pt}{0.400pt}}
\multiput(1211.00,642.17)(0.500,-1.000){2}{\rule{0.120pt}{0.400pt}}
\put(1211.0,643.0){\usebox{\plotpoint}}
\put(1212,642){\usebox{\plotpoint}}
\put(1212,642){\usebox{\plotpoint}}
\put(1212,642){\usebox{\plotpoint}}
\put(1212,642){\usebox{\plotpoint}}
\put(1212,642){\usebox{\plotpoint}}
\put(1212,642){\usebox{\plotpoint}}
\put(1212,642){\usebox{\plotpoint}}
\put(1212,640.67){\rule{0.241pt}{0.400pt}}
\multiput(1212.00,641.17)(0.500,-1.000){2}{\rule{0.120pt}{0.400pt}}
\put(1213,641){\usebox{\plotpoint}}
\put(1213,641){\usebox{\plotpoint}}
\put(1213,641){\usebox{\plotpoint}}
\put(1213,641){\usebox{\plotpoint}}
\put(1213,641){\usebox{\plotpoint}}
\put(1213,641){\usebox{\plotpoint}}
\put(1213,641){\usebox{\plotpoint}}
\put(1213,639.67){\rule{0.241pt}{0.400pt}}
\multiput(1213.00,640.17)(0.500,-1.000){2}{\rule{0.120pt}{0.400pt}}
\put(1214,640){\usebox{\plotpoint}}
\put(1214,640){\usebox{\plotpoint}}
\put(1214,640){\usebox{\plotpoint}}
\put(1214,640){\usebox{\plotpoint}}
\put(1214,640){\usebox{\plotpoint}}
\put(1214,640){\usebox{\plotpoint}}
\put(1214.0,639.0){\usebox{\plotpoint}}
\put(1214.0,639.0){\usebox{\plotpoint}}
\put(1215.0,638.0){\usebox{\plotpoint}}
\put(1215.0,638.0){\usebox{\plotpoint}}
\put(1216.0,637.0){\usebox{\plotpoint}}
\put(1216.0,637.0){\usebox{\plotpoint}}
\put(1217.0,636.0){\usebox{\plotpoint}}
\put(1217.0,636.0){\usebox{\plotpoint}}
\put(1218.0,635.0){\usebox{\plotpoint}}
\put(1218.0,635.0){\usebox{\plotpoint}}
\put(1219.0,634.0){\usebox{\plotpoint}}
\put(1219.0,634.0){\usebox{\plotpoint}}
\put(1220.0,633.0){\usebox{\plotpoint}}
\put(1220.0,633.0){\usebox{\plotpoint}}
\put(1221.0,632.0){\usebox{\plotpoint}}
\put(1221.0,632.0){\usebox{\plotpoint}}
\put(1222.0,631.0){\usebox{\plotpoint}}
\put(1222.0,631.0){\usebox{\plotpoint}}
\put(1223.0,630.0){\usebox{\plotpoint}}
\put(1223.0,630.0){\usebox{\plotpoint}}
\put(1224.0,629.0){\usebox{\plotpoint}}
\put(1224.0,629.0){\usebox{\plotpoint}}
\put(1225.0,628.0){\usebox{\plotpoint}}
\put(1225.0,628.0){\usebox{\plotpoint}}
\put(1226,625.67){\rule{0.241pt}{0.400pt}}
\multiput(1226.00,626.17)(0.500,-1.000){2}{\rule{0.120pt}{0.400pt}}
\put(1226.0,627.0){\usebox{\plotpoint}}
\put(1227,626){\usebox{\plotpoint}}
\put(1227,626){\usebox{\plotpoint}}
\put(1227,626){\usebox{\plotpoint}}
\put(1227,626){\usebox{\plotpoint}}
\put(1227,626){\usebox{\plotpoint}}
\put(1227,626){\usebox{\plotpoint}}
\put(1227,626){\usebox{\plotpoint}}
\put(1227,624.67){\rule{0.241pt}{0.400pt}}
\multiput(1227.00,625.17)(0.500,-1.000){2}{\rule{0.120pt}{0.400pt}}
\put(1228,625){\usebox{\plotpoint}}
\put(1228,625){\usebox{\plotpoint}}
\put(1228,625){\usebox{\plotpoint}}
\put(1228,625){\usebox{\plotpoint}}
\put(1228,625){\usebox{\plotpoint}}
\put(1228,625){\usebox{\plotpoint}}
\put(1228.0,624.0){\usebox{\plotpoint}}
\put(1228.0,624.0){\usebox{\plotpoint}}
\put(1229.0,623.0){\usebox{\plotpoint}}
\put(1229.0,623.0){\usebox{\plotpoint}}
\put(1230.0,622.0){\usebox{\plotpoint}}
\put(1230.0,622.0){\usebox{\plotpoint}}
\put(1231.0,621.0){\usebox{\plotpoint}}
\put(1231.0,621.0){\usebox{\plotpoint}}
\put(1232.0,620.0){\usebox{\plotpoint}}
\put(1232.0,620.0){\usebox{\plotpoint}}
\put(1233.0,619.0){\usebox{\plotpoint}}
\put(1233.0,619.0){\usebox{\plotpoint}}
\put(1234.0,618.0){\usebox{\plotpoint}}
\put(1234.0,618.0){\usebox{\plotpoint}}
\put(1235.0,617.0){\usebox{\plotpoint}}
\put(1235.0,617.0){\usebox{\plotpoint}}
\put(1236,614.67){\rule{0.241pt}{0.400pt}}
\multiput(1236.00,615.17)(0.500,-1.000){2}{\rule{0.120pt}{0.400pt}}
\put(1236.0,616.0){\usebox{\plotpoint}}
\put(1237,615){\usebox{\plotpoint}}
\put(1237,615){\usebox{\plotpoint}}
\put(1237,615){\usebox{\plotpoint}}
\put(1237,615){\usebox{\plotpoint}}
\put(1237,615){\usebox{\plotpoint}}
\put(1237,615){\usebox{\plotpoint}}
\put(1237,615){\usebox{\plotpoint}}
\put(1237,613.67){\rule{0.241pt}{0.400pt}}
\multiput(1237.00,614.17)(0.500,-1.000){2}{\rule{0.120pt}{0.400pt}}
\put(1238,614){\usebox{\plotpoint}}
\put(1238,614){\usebox{\plotpoint}}
\put(1238,614){\usebox{\plotpoint}}
\put(1238,614){\usebox{\plotpoint}}
\put(1238,614){\usebox{\plotpoint}}
\put(1238,614){\usebox{\plotpoint}}
\put(1238.0,613.0){\usebox{\plotpoint}}
\put(1238.0,613.0){\usebox{\plotpoint}}
\put(1239.0,612.0){\usebox{\plotpoint}}
\put(1239.0,612.0){\usebox{\plotpoint}}
\put(1240.0,611.0){\usebox{\plotpoint}}
\put(1240.0,611.0){\usebox{\plotpoint}}
\put(1241.0,610.0){\usebox{\plotpoint}}
\put(1241.0,610.0){\usebox{\plotpoint}}
\put(1242.0,609.0){\usebox{\plotpoint}}
\put(1242.0,609.0){\usebox{\plotpoint}}
\put(1243.0,608.0){\usebox{\plotpoint}}
\put(1243.0,608.0){\usebox{\plotpoint}}
\put(1244.0,607.0){\usebox{\plotpoint}}
\put(1244.0,607.0){\usebox{\plotpoint}}
\put(1245,604.67){\rule{0.241pt}{0.400pt}}
\multiput(1245.00,605.17)(0.500,-1.000){2}{\rule{0.120pt}{0.400pt}}
\put(1245.0,606.0){\usebox{\plotpoint}}
\put(1246,605){\usebox{\plotpoint}}
\put(1246,605){\usebox{\plotpoint}}
\put(1246,605){\usebox{\plotpoint}}
\put(1246,605){\usebox{\plotpoint}}
\put(1246,605){\usebox{\plotpoint}}
\put(1246,605){\usebox{\plotpoint}}
\put(1246.0,604.0){\usebox{\plotpoint}}
\put(1246.0,604.0){\usebox{\plotpoint}}
\put(1247.0,603.0){\usebox{\plotpoint}}
\put(1247.0,603.0){\usebox{\plotpoint}}
\put(1248.0,602.0){\usebox{\plotpoint}}
\put(1248.0,602.0){\usebox{\plotpoint}}
\put(1249.0,601.0){\usebox{\plotpoint}}
\put(1249.0,601.0){\usebox{\plotpoint}}
\put(1250.0,600.0){\usebox{\plotpoint}}
\put(1250.0,600.0){\usebox{\plotpoint}}
\put(1251.0,599.0){\usebox{\plotpoint}}
\put(1251.0,599.0){\usebox{\plotpoint}}
\put(1252.0,598.0){\usebox{\plotpoint}}
\put(1252.0,598.0){\usebox{\plotpoint}}
\put(1253,595.67){\rule{0.241pt}{0.400pt}}
\multiput(1253.00,596.17)(0.500,-1.000){2}{\rule{0.120pt}{0.400pt}}
\put(1253.0,597.0){\usebox{\plotpoint}}
\put(1254,596){\usebox{\plotpoint}}
\put(1254,596){\usebox{\plotpoint}}
\put(1254,596){\usebox{\plotpoint}}
\put(1254,596){\usebox{\plotpoint}}
\put(1254,596){\usebox{\plotpoint}}
\put(1254,596){\usebox{\plotpoint}}
\put(1254.0,595.0){\usebox{\plotpoint}}
\put(1254.0,595.0){\usebox{\plotpoint}}
\put(1255.0,594.0){\usebox{\plotpoint}}
\put(1255.0,594.0){\usebox{\plotpoint}}
\put(1256.0,593.0){\usebox{\plotpoint}}
\put(1256.0,593.0){\usebox{\plotpoint}}
\put(1257.0,592.0){\usebox{\plotpoint}}
\put(1257.0,592.0){\usebox{\plotpoint}}
\put(1258.0,591.0){\usebox{\plotpoint}}
\put(1258.0,591.0){\usebox{\plotpoint}}
\put(1259,588.67){\rule{0.241pt}{0.400pt}}
\multiput(1259.00,589.17)(0.500,-1.000){2}{\rule{0.120pt}{0.400pt}}
\put(1259.0,590.0){\usebox{\plotpoint}}
\put(1260,589){\usebox{\plotpoint}}
\put(1260,589){\usebox{\plotpoint}}
\put(1260,589){\usebox{\plotpoint}}
\put(1260,589){\usebox{\plotpoint}}
\put(1260,589){\usebox{\plotpoint}}
\put(1260,589){\usebox{\plotpoint}}
\put(1260.0,588.0){\usebox{\plotpoint}}
\put(1260.0,588.0){\usebox{\plotpoint}}
\put(1261.0,587.0){\usebox{\plotpoint}}
\put(1261.0,587.0){\usebox{\plotpoint}}
\put(1262.0,586.0){\usebox{\plotpoint}}
\put(1262.0,586.0){\usebox{\plotpoint}}
\put(1263.0,585.0){\usebox{\plotpoint}}
\put(1263.0,585.0){\usebox{\plotpoint}}
\put(1264.0,584.0){\usebox{\plotpoint}}
\put(1264.0,584.0){\usebox{\plotpoint}}
\put(1265,581.67){\rule{0.241pt}{0.400pt}}
\multiput(1265.00,582.17)(0.500,-1.000){2}{\rule{0.120pt}{0.400pt}}
\put(1265.0,583.0){\usebox{\plotpoint}}
\put(1266,582){\usebox{\plotpoint}}
\put(1266,582){\usebox{\plotpoint}}
\put(1266,582){\usebox{\plotpoint}}
\put(1266,582){\usebox{\plotpoint}}
\put(1266,582){\usebox{\plotpoint}}
\put(1266,582){\usebox{\plotpoint}}
\put(1266.0,581.0){\usebox{\plotpoint}}
\put(1266.0,581.0){\usebox{\plotpoint}}
\put(1267.0,580.0){\usebox{\plotpoint}}
\put(1267.0,580.0){\usebox{\plotpoint}}
\put(1268.0,579.0){\usebox{\plotpoint}}
\put(1268.0,579.0){\usebox{\plotpoint}}
\put(1269.0,578.0){\usebox{\plotpoint}}
\put(1269.0,578.0){\usebox{\plotpoint}}
\put(1270.0,577.0){\usebox{\plotpoint}}
\put(1270.0,577.0){\usebox{\plotpoint}}
\put(1271,574.67){\rule{0.241pt}{0.400pt}}
\multiput(1271.00,575.17)(0.500,-1.000){2}{\rule{0.120pt}{0.400pt}}
\put(1271.0,576.0){\usebox{\plotpoint}}
\put(1272,575){\usebox{\plotpoint}}
\put(1272,575){\usebox{\plotpoint}}
\put(1272,575){\usebox{\plotpoint}}
\put(1272,575){\usebox{\plotpoint}}
\put(1272,575){\usebox{\plotpoint}}
\put(1272,575){\usebox{\plotpoint}}
\put(1272.0,574.0){\usebox{\plotpoint}}
\put(1272.0,574.0){\usebox{\plotpoint}}
\put(1273.0,573.0){\usebox{\plotpoint}}
\put(1273.0,573.0){\usebox{\plotpoint}}
\put(1274.0,572.0){\usebox{\plotpoint}}
\put(1274.0,572.0){\usebox{\plotpoint}}
\put(1275.0,571.0){\usebox{\plotpoint}}
\put(1275.0,571.0){\usebox{\plotpoint}}
\put(1276.0,569.0){\rule[-0.200pt]{0.400pt}{0.482pt}}
\put(1276.0,569.0){\usebox{\plotpoint}}
\put(1277.0,568.0){\usebox{\plotpoint}}
\put(1277.0,568.0){\usebox{\plotpoint}}
\put(1278.0,567.0){\usebox{\plotpoint}}
\put(1278.0,567.0){\usebox{\plotpoint}}
\put(1279.0,566.0){\usebox{\plotpoint}}
\put(1279.0,566.0){\usebox{\plotpoint}}
\put(1280.0,565.0){\usebox{\plotpoint}}
\put(1280.0,565.0){\usebox{\plotpoint}}
\put(1281,562.67){\rule{0.241pt}{0.400pt}}
\multiput(1281.00,563.17)(0.500,-1.000){2}{\rule{0.120pt}{0.400pt}}
\put(1281.0,564.0){\usebox{\plotpoint}}
\put(1282,563){\usebox{\plotpoint}}
\put(1282,563){\usebox{\plotpoint}}
\put(1282,563){\usebox{\plotpoint}}
\put(1282,563){\usebox{\plotpoint}}
\put(1282,563){\usebox{\plotpoint}}
\put(1282.0,562.0){\usebox{\plotpoint}}
\put(1282.0,562.0){\usebox{\plotpoint}}
\put(1283.0,561.0){\usebox{\plotpoint}}
\put(1283.0,561.0){\usebox{\plotpoint}}
\put(1284.0,560.0){\usebox{\plotpoint}}
\put(1284.0,560.0){\usebox{\plotpoint}}
\put(1285.0,559.0){\usebox{\plotpoint}}
\put(1285.0,559.0){\usebox{\plotpoint}}
\put(1286.0,557.0){\rule[-0.200pt]{0.400pt}{0.482pt}}
\put(1286.0,557.0){\usebox{\plotpoint}}
\put(1287.0,556.0){\usebox{\plotpoint}}
\put(1287.0,556.0){\usebox{\plotpoint}}
\put(1288.0,555.0){\usebox{\plotpoint}}
\put(1288.0,555.0){\usebox{\plotpoint}}
\put(1289.0,554.0){\usebox{\plotpoint}}
\put(1289.0,554.0){\usebox{\plotpoint}}
\put(1290,551.67){\rule{0.241pt}{0.400pt}}
\multiput(1290.00,552.17)(0.500,-1.000){2}{\rule{0.120pt}{0.400pt}}
\put(1290.0,553.0){\usebox{\plotpoint}}
\put(1291,552){\usebox{\plotpoint}}
\put(1291,552){\usebox{\plotpoint}}
\put(1291,552){\usebox{\plotpoint}}
\put(1291,552){\usebox{\plotpoint}}
\put(1291,552){\usebox{\plotpoint}}
\put(1291,552){\usebox{\plotpoint}}
\put(1291.0,551.0){\usebox{\plotpoint}}
\put(1291.0,551.0){\usebox{\plotpoint}}
\put(1292.0,550.0){\usebox{\plotpoint}}
\put(1292.0,550.0){\usebox{\plotpoint}}
\put(1293.0,549.0){\usebox{\plotpoint}}
\put(1293.0,549.0){\usebox{\plotpoint}}
\put(1294.0,548.0){\usebox{\plotpoint}}
\put(1294.0,548.0){\usebox{\plotpoint}}
\put(1295.0,546.0){\rule[-0.200pt]{0.400pt}{0.482pt}}
\put(1295.0,546.0){\usebox{\plotpoint}}
\put(1296.0,545.0){\usebox{\plotpoint}}
\put(1296.0,545.0){\usebox{\plotpoint}}
\put(1297.0,544.0){\usebox{\plotpoint}}
\put(1297.0,544.0){\usebox{\plotpoint}}
\put(1298.0,543.0){\usebox{\plotpoint}}
\put(1298.0,543.0){\usebox{\plotpoint}}
\put(1299,540.67){\rule{0.241pt}{0.400pt}}
\multiput(1299.00,541.17)(0.500,-1.000){2}{\rule{0.120pt}{0.400pt}}
\put(1299.0,542.0){\usebox{\plotpoint}}
\put(1300,541){\usebox{\plotpoint}}
\put(1300,541){\usebox{\plotpoint}}
\put(1300,541){\usebox{\plotpoint}}
\put(1300,541){\usebox{\plotpoint}}
\put(1300,541){\usebox{\plotpoint}}
\put(1300.0,540.0){\usebox{\plotpoint}}
\put(1300.0,540.0){\usebox{\plotpoint}}
\put(1301.0,539.0){\usebox{\plotpoint}}
\put(1301.0,539.0){\usebox{\plotpoint}}
\put(1302.0,538.0){\usebox{\plotpoint}}
\put(1302.0,538.0){\usebox{\plotpoint}}
\put(1303,535.67){\rule{0.241pt}{0.400pt}}
\multiput(1303.00,536.17)(0.500,-1.000){2}{\rule{0.120pt}{0.400pt}}
\put(1303.0,537.0){\usebox{\plotpoint}}
\put(1304,536){\usebox{\plotpoint}}
\put(1304,536){\usebox{\plotpoint}}
\put(1304,536){\usebox{\plotpoint}}
\put(1304,536){\usebox{\plotpoint}}
\put(1304,536){\usebox{\plotpoint}}
\put(1304.0,535.0){\usebox{\plotpoint}}
\put(1304.0,535.0){\usebox{\plotpoint}}
\put(1305.0,534.0){\usebox{\plotpoint}}
\put(1305.0,534.0){\usebox{\plotpoint}}
\put(1306.0,533.0){\usebox{\plotpoint}}
\put(1306.0,533.0){\usebox{\plotpoint}}
\put(1307,530.67){\rule{0.241pt}{0.400pt}}
\multiput(1307.00,531.17)(0.500,-1.000){2}{\rule{0.120pt}{0.400pt}}
\put(1307.0,532.0){\usebox{\plotpoint}}
\put(1308,531){\usebox{\plotpoint}}
\put(1308,531){\usebox{\plotpoint}}
\put(1308,531){\usebox{\plotpoint}}
\put(1308,531){\usebox{\plotpoint}}
\put(1308,531){\usebox{\plotpoint}}
\put(1308.0,530.0){\usebox{\plotpoint}}
\put(1308.0,530.0){\usebox{\plotpoint}}
\put(1309.0,529.0){\usebox{\plotpoint}}
\put(1309.0,529.0){\usebox{\plotpoint}}
\put(1310.0,528.0){\usebox{\plotpoint}}
\put(1310.0,528.0){\usebox{\plotpoint}}
\put(1311,525.67){\rule{0.241pt}{0.400pt}}
\multiput(1311.00,526.17)(0.500,-1.000){2}{\rule{0.120pt}{0.400pt}}
\put(1311.0,527.0){\usebox{\plotpoint}}
\put(1312,526){\usebox{\plotpoint}}
\put(1312,526){\usebox{\plotpoint}}
\put(1312,526){\usebox{\plotpoint}}
\put(1312,526){\usebox{\plotpoint}}
\put(1312,526){\usebox{\plotpoint}}
\put(1312.0,525.0){\usebox{\plotpoint}}
\put(1312.0,525.0){\usebox{\plotpoint}}
\put(1313.0,524.0){\usebox{\plotpoint}}
\put(1313.0,524.0){\usebox{\plotpoint}}
\put(1314.0,523.0){\usebox{\plotpoint}}
\put(1314.0,523.0){\usebox{\plotpoint}}
\put(1315.0,521.0){\rule[-0.200pt]{0.400pt}{0.482pt}}
\put(1315.0,521.0){\usebox{\plotpoint}}
\put(1316.0,520.0){\usebox{\plotpoint}}
\put(1316.0,520.0){\usebox{\plotpoint}}
\put(1317.0,519.0){\usebox{\plotpoint}}
\put(1317.0,519.0){\usebox{\plotpoint}}
\put(1318,516.67){\rule{0.241pt}{0.400pt}}
\multiput(1318.00,517.17)(0.500,-1.000){2}{\rule{0.120pt}{0.400pt}}
\put(1318.0,518.0){\usebox{\plotpoint}}
\put(1319,517){\usebox{\plotpoint}}
\put(1319,517){\usebox{\plotpoint}}
\put(1319,517){\usebox{\plotpoint}}
\put(1319,517){\usebox{\plotpoint}}
\put(1319,517){\usebox{\plotpoint}}
\put(1319,517){\usebox{\plotpoint}}
\put(1319.0,516.0){\usebox{\plotpoint}}
\put(1319.0,516.0){\usebox{\plotpoint}}
\put(1320.0,515.0){\usebox{\plotpoint}}
\put(1320.0,515.0){\usebox{\plotpoint}}
\put(1321.0,514.0){\usebox{\plotpoint}}
\put(1321.0,514.0){\usebox{\plotpoint}}
\put(1322,511.67){\rule{0.241pt}{0.400pt}}
\multiput(1322.00,512.17)(0.500,-1.000){2}{\rule{0.120pt}{0.400pt}}
\put(1322.0,513.0){\usebox{\plotpoint}}
\put(1323,512){\usebox{\plotpoint}}
\put(1323,512){\usebox{\plotpoint}}
\put(1323,512){\usebox{\plotpoint}}
\put(1323,512){\usebox{\plotpoint}}
\put(1323,512){\usebox{\plotpoint}}
\put(1323.0,511.0){\usebox{\plotpoint}}
\put(1323.0,511.0){\usebox{\plotpoint}}
\put(1324.0,510.0){\usebox{\plotpoint}}
\put(1324.0,510.0){\usebox{\plotpoint}}
\put(1325,507.67){\rule{0.241pt}{0.400pt}}
\multiput(1325.00,508.17)(0.500,-1.000){2}{\rule{0.120pt}{0.400pt}}
\put(1325.0,509.0){\usebox{\plotpoint}}
\put(1326,508){\usebox{\plotpoint}}
\put(1326,508){\usebox{\plotpoint}}
\put(1326,508){\usebox{\plotpoint}}
\put(1326,508){\usebox{\plotpoint}}
\put(1326,508){\usebox{\plotpoint}}
\put(1326,508){\usebox{\plotpoint}}
\put(1326.0,507.0){\usebox{\plotpoint}}
\put(1326.0,507.0){\usebox{\plotpoint}}
\put(1327.0,506.0){\usebox{\plotpoint}}
\put(1327.0,506.0){\usebox{\plotpoint}}
\put(1328.0,505.0){\usebox{\plotpoint}}
\put(1328.0,505.0){\usebox{\plotpoint}}
\put(1329.0,503.0){\rule[-0.200pt]{0.400pt}{0.482pt}}
\put(1329.0,503.0){\usebox{\plotpoint}}
\put(1330.0,502.0){\usebox{\plotpoint}}
\put(1330.0,502.0){\usebox{\plotpoint}}
\put(1331.0,501.0){\usebox{\plotpoint}}
\put(1331.0,501.0){\usebox{\plotpoint}}
\put(1332,498.67){\rule{0.241pt}{0.400pt}}
\multiput(1332.00,499.17)(0.500,-1.000){2}{\rule{0.120pt}{0.400pt}}
\put(1332.0,500.0){\usebox{\plotpoint}}
\put(1333,499){\usebox{\plotpoint}}
\put(1333,499){\usebox{\plotpoint}}
\put(1333,499){\usebox{\plotpoint}}
\put(1333,499){\usebox{\plotpoint}}
\put(1333,499){\usebox{\plotpoint}}
\put(1333.0,498.0){\usebox{\plotpoint}}
\put(1333.0,498.0){\usebox{\plotpoint}}
\put(1334.0,497.0){\usebox{\plotpoint}}
\put(1334.0,497.0){\usebox{\plotpoint}}
\put(1335,494.67){\rule{0.241pt}{0.400pt}}
\multiput(1335.00,495.17)(0.500,-1.000){2}{\rule{0.120pt}{0.400pt}}
\put(1335.0,496.0){\usebox{\plotpoint}}
\put(1336,495){\usebox{\plotpoint}}
\put(1336,495){\usebox{\plotpoint}}
\put(1336,495){\usebox{\plotpoint}}
\put(1336,495){\usebox{\plotpoint}}
\put(1336,495){\usebox{\plotpoint}}
\put(1336,495){\usebox{\plotpoint}}
\put(1336.0,494.0){\usebox{\plotpoint}}
\put(1336.0,494.0){\usebox{\plotpoint}}
\put(1337.0,493.0){\usebox{\plotpoint}}
\put(1337.0,493.0){\usebox{\plotpoint}}
\put(1338.0,492.0){\usebox{\plotpoint}}
\put(1338.0,492.0){\usebox{\plotpoint}}
\put(1339.0,490.0){\rule[-0.200pt]{0.400pt}{0.482pt}}
\put(1339.0,490.0){\usebox{\plotpoint}}
\put(1340.0,489.0){\usebox{\plotpoint}}
\put(1340.0,489.0){\usebox{\plotpoint}}
\put(1341.0,488.0){\usebox{\plotpoint}}
\put(1341.0,488.0){\usebox{\plotpoint}}
\put(1342.0,486.0){\rule[-0.200pt]{0.400pt}{0.482pt}}
\put(1342.0,486.0){\usebox{\plotpoint}}
\put(1343.0,485.0){\usebox{\plotpoint}}
\put(1343.0,485.0){\usebox{\plotpoint}}
\put(1344.0,484.0){\usebox{\plotpoint}}
\put(1344.0,484.0){\usebox{\plotpoint}}
\put(1345.0,482.0){\rule[-0.200pt]{0.400pt}{0.482pt}}
\put(1345.0,482.0){\usebox{\plotpoint}}
\put(1346.0,481.0){\usebox{\plotpoint}}
\put(1346.0,481.0){\usebox{\plotpoint}}
\put(1347.0,480.0){\usebox{\plotpoint}}
\put(1347.0,480.0){\usebox{\plotpoint}}
\put(1348.0,478.0){\rule[-0.200pt]{0.400pt}{0.482pt}}
\put(1348.0,478.0){\usebox{\plotpoint}}
\put(1349.0,477.0){\usebox{\plotpoint}}
\put(1349.0,477.0){\usebox{\plotpoint}}
\put(1350.0,476.0){\usebox{\plotpoint}}
\put(1350.0,476.0){\usebox{\plotpoint}}
\put(1351.0,474.0){\rule[-0.200pt]{0.400pt}{0.482pt}}
\put(1351.0,474.0){\usebox{\plotpoint}}
\put(1352.0,473.0){\usebox{\plotpoint}}
\put(1352.0,473.0){\usebox{\plotpoint}}
\put(1353.0,472.0){\usebox{\plotpoint}}
\put(1353.0,472.0){\usebox{\plotpoint}}
\put(1354.0,470.0){\rule[-0.200pt]{0.400pt}{0.482pt}}
\put(1354.0,470.0){\usebox{\plotpoint}}
\put(1355.0,469.0){\usebox{\plotpoint}}
\put(1355.0,469.0){\usebox{\plotpoint}}
\put(1356,466.67){\rule{0.241pt}{0.400pt}}
\multiput(1356.00,467.17)(0.500,-1.000){2}{\rule{0.120pt}{0.400pt}}
\put(1356.0,468.0){\usebox{\plotpoint}}
\put(1357,467){\usebox{\plotpoint}}
\put(1357,467){\usebox{\plotpoint}}
\put(1357,467){\usebox{\plotpoint}}
\put(1357,467){\usebox{\plotpoint}}
\put(1357,467){\usebox{\plotpoint}}
\put(1357.0,466.0){\usebox{\plotpoint}}
\put(1357.0,466.0){\usebox{\plotpoint}}
\put(1358.0,465.0){\usebox{\plotpoint}}
\put(1358.0,465.0){\usebox{\plotpoint}}
\put(1359,462.67){\rule{0.241pt}{0.400pt}}
\multiput(1359.00,463.17)(0.500,-1.000){2}{\rule{0.120pt}{0.400pt}}
\put(1359.0,464.0){\usebox{\plotpoint}}
\put(1360,463){\usebox{\plotpoint}}
\put(1360,463){\usebox{\plotpoint}}
\put(1360,463){\usebox{\plotpoint}}
\put(1360,463){\usebox{\plotpoint}}
\put(1360,463){\usebox{\plotpoint}}
\put(1360.0,462.0){\usebox{\plotpoint}}
\put(1360.0,462.0){\usebox{\plotpoint}}
\put(1361.0,461.0){\usebox{\plotpoint}}
\put(1361.0,461.0){\usebox{\plotpoint}}
\put(1362.0,459.0){\rule[-0.200pt]{0.400pt}{0.482pt}}
\put(1362.0,459.0){\usebox{\plotpoint}}
\put(1363.0,458.0){\usebox{\plotpoint}}
\put(1363.0,458.0){\usebox{\plotpoint}}
\put(1364.0,457.0){\usebox{\plotpoint}}
\put(1364.0,457.0){\usebox{\plotpoint}}
\put(1365.0,455.0){\rule[-0.200pt]{0.400pt}{0.482pt}}
\put(1365.0,455.0){\usebox{\plotpoint}}
\put(1366.0,454.0){\usebox{\plotpoint}}
\put(1366.0,454.0){\usebox{\plotpoint}}
\put(1367,451.67){\rule{0.241pt}{0.400pt}}
\multiput(1367.00,452.17)(0.500,-1.000){2}{\rule{0.120pt}{0.400pt}}
\put(1367.0,453.0){\usebox{\plotpoint}}
\put(1368,452){\usebox{\plotpoint}}
\put(1368,452){\usebox{\plotpoint}}
\put(1368,452){\usebox{\plotpoint}}
\put(1368,452){\usebox{\plotpoint}}
\put(1368,452){\usebox{\plotpoint}}
\put(1368.0,451.0){\usebox{\plotpoint}}
\put(1368.0,451.0){\usebox{\plotpoint}}
\put(1369.0,450.0){\usebox{\plotpoint}}
\put(1369.0,450.0){\usebox{\plotpoint}}
\put(1370,447.67){\rule{0.241pt}{0.400pt}}
\multiput(1370.00,448.17)(0.500,-1.000){2}{\rule{0.120pt}{0.400pt}}
\put(1370.0,449.0){\usebox{\plotpoint}}
\put(1371,448){\usebox{\plotpoint}}
\put(1371,448){\usebox{\plotpoint}}
\put(1371,448){\usebox{\plotpoint}}
\put(1371,448){\usebox{\plotpoint}}
\put(1371,448){\usebox{\plotpoint}}
\put(1371.0,447.0){\usebox{\plotpoint}}
\put(1371.0,447.0){\usebox{\plotpoint}}
\put(1372.0,446.0){\usebox{\plotpoint}}
\put(1372.0,446.0){\usebox{\plotpoint}}
\put(1373.0,444.0){\rule[-0.200pt]{0.400pt}{0.482pt}}
\put(1373.0,444.0){\usebox{\plotpoint}}
\put(1374.0,443.0){\usebox{\plotpoint}}
\put(1374.0,443.0){\usebox{\plotpoint}}
\put(1375,440.67){\rule{0.241pt}{0.400pt}}
\multiput(1375.00,441.17)(0.500,-1.000){2}{\rule{0.120pt}{0.400pt}}
\put(1375.0,442.0){\usebox{\plotpoint}}
\put(1376,441){\usebox{\plotpoint}}
\put(1376,441){\usebox{\plotpoint}}
\put(1376,441){\usebox{\plotpoint}}
\put(1376,441){\usebox{\plotpoint}}
\put(1376,441){\usebox{\plotpoint}}
\put(1376.0,440.0){\usebox{\plotpoint}}
\put(1376.0,440.0){\usebox{\plotpoint}}
\put(1377.0,439.0){\usebox{\plotpoint}}
\put(1377.0,439.0){\usebox{\plotpoint}}
\put(1378,436.67){\rule{0.241pt}{0.400pt}}
\multiput(1378.00,437.17)(0.500,-1.000){2}{\rule{0.120pt}{0.400pt}}
\put(1378.0,438.0){\usebox{\plotpoint}}
\put(1379,437){\usebox{\plotpoint}}
\put(1379,437){\usebox{\plotpoint}}
\put(1379,437){\usebox{\plotpoint}}
\put(1379,437){\usebox{\plotpoint}}
\put(1379.0,436.0){\usebox{\plotpoint}}
\put(1379.0,436.0){\usebox{\plotpoint}}
\put(1380.0,435.0){\usebox{\plotpoint}}
\put(1380.0,435.0){\usebox{\plotpoint}}
\put(1381.0,433.0){\rule[-0.200pt]{0.400pt}{0.482pt}}
\put(1381.0,433.0){\usebox{\plotpoint}}
\put(1382.0,432.0){\usebox{\plotpoint}}
\put(1382.0,432.0){\usebox{\plotpoint}}
\put(1383,429.67){\rule{0.241pt}{0.400pt}}
\multiput(1383.00,430.17)(0.500,-1.000){2}{\rule{0.120pt}{0.400pt}}
\put(1383.0,431.0){\usebox{\plotpoint}}
\put(1384,430){\usebox{\plotpoint}}
\put(1384,430){\usebox{\plotpoint}}
\put(1384,430){\usebox{\plotpoint}}
\put(1384,430){\usebox{\plotpoint}}
\put(1384.0,429.0){\usebox{\plotpoint}}
\put(1384.0,429.0){\usebox{\plotpoint}}
\put(1385.0,428.0){\usebox{\plotpoint}}
\put(1385.0,428.0){\usebox{\plotpoint}}
\put(1386.0,426.0){\rule[-0.200pt]{0.400pt}{0.482pt}}
\put(1386.0,426.0){\usebox{\plotpoint}}
\put(1387.0,425.0){\usebox{\plotpoint}}
\put(1387.0,425.0){\usebox{\plotpoint}}
\put(1388,422.67){\rule{0.241pt}{0.400pt}}
\multiput(1388.00,423.17)(0.500,-1.000){2}{\rule{0.120pt}{0.400pt}}
\put(1388.0,424.0){\usebox{\plotpoint}}
\put(1389,423){\usebox{\plotpoint}}
\put(1389,423){\usebox{\plotpoint}}
\put(1389,423){\usebox{\plotpoint}}
\put(1389,423){\usebox{\plotpoint}}
\put(1389.0,422.0){\usebox{\plotpoint}}
\put(1389.0,422.0){\usebox{\plotpoint}}
\put(1390.0,421.0){\usebox{\plotpoint}}
\put(1390.0,421.0){\usebox{\plotpoint}}
\put(1391.0,419.0){\rule[-0.200pt]{0.400pt}{0.482pt}}
\put(1391.0,419.0){\usebox{\plotpoint}}
\put(1392.0,418.0){\usebox{\plotpoint}}
\put(1392.0,418.0){\usebox{\plotpoint}}
\put(1393,415.67){\rule{0.241pt}{0.400pt}}
\multiput(1393.00,416.17)(0.500,-1.000){2}{\rule{0.120pt}{0.400pt}}
\put(1393.0,417.0){\usebox{\plotpoint}}
\put(1394,416){\usebox{\plotpoint}}
\put(1394,416){\usebox{\plotpoint}}
\put(1394,416){\usebox{\plotpoint}}
\put(1394,416){\usebox{\plotpoint}}
\put(1394.0,415.0){\usebox{\plotpoint}}
\put(1394.0,415.0){\usebox{\plotpoint}}
\put(1395,412.67){\rule{0.241pt}{0.400pt}}
\multiput(1395.00,413.17)(0.500,-1.000){2}{\rule{0.120pt}{0.400pt}}
\put(1395.0,414.0){\usebox{\plotpoint}}
\put(1396,413){\usebox{\plotpoint}}
\put(1396,413){\usebox{\plotpoint}}
\put(1396,413){\usebox{\plotpoint}}
\put(1396,413){\usebox{\plotpoint}}
\put(1396,413){\usebox{\plotpoint}}
\put(1396.0,412.0){\usebox{\plotpoint}}
\put(1396.0,412.0){\usebox{\plotpoint}}
\put(1397.0,411.0){\usebox{\plotpoint}}
\put(1397.0,411.0){\usebox{\plotpoint}}
\put(1398.0,409.0){\rule[-0.200pt]{0.400pt}{0.482pt}}
\put(1398.0,409.0){\usebox{\plotpoint}}
\put(1399.0,408.0){\usebox{\plotpoint}}
\put(1399.0,408.0){\usebox{\plotpoint}}
\put(1400,405.67){\rule{0.241pt}{0.400pt}}
\multiput(1400.00,406.17)(0.500,-1.000){2}{\rule{0.120pt}{0.400pt}}
\put(1400.0,407.0){\usebox{\plotpoint}}
\put(1401,406){\usebox{\plotpoint}}
\put(1401,406){\usebox{\plotpoint}}
\put(1401,406){\usebox{\plotpoint}}
\put(1401,406){\usebox{\plotpoint}}
\put(1401,406){\usebox{\plotpoint}}
\put(1401.0,405.0){\usebox{\plotpoint}}
\put(1401.0,405.0){\usebox{\plotpoint}}
\put(1402.0,404.0){\usebox{\plotpoint}}
\put(1402.0,404.0){\usebox{\plotpoint}}
\put(1403.0,402.0){\rule[-0.200pt]{0.400pt}{0.482pt}}
\put(1403.0,402.0){\usebox{\plotpoint}}
\put(1404.0,401.0){\usebox{\plotpoint}}
\put(1404.0,401.0){\usebox{\plotpoint}}
\put(1405.0,399.0){\rule[-0.200pt]{0.400pt}{0.482pt}}
\put(1405.0,399.0){\usebox{\plotpoint}}
\put(1406.0,398.0){\usebox{\plotpoint}}
\put(1406.0,398.0){\usebox{\plotpoint}}
\put(1407,395.67){\rule{0.241pt}{0.400pt}}
\multiput(1407.00,396.17)(0.500,-1.000){2}{\rule{0.120pt}{0.400pt}}
\put(1407.0,397.0){\usebox{\plotpoint}}
\put(1408,396){\usebox{\plotpoint}}
\put(1408,396){\usebox{\plotpoint}}
\put(1408,396){\usebox{\plotpoint}}
\put(1408,396){\usebox{\plotpoint}}
\put(1408.0,395.0){\usebox{\plotpoint}}
\put(1408.0,395.0){\usebox{\plotpoint}}
\put(1409.0,394.0){\usebox{\plotpoint}}
\put(1409.0,394.0){\usebox{\plotpoint}}
\put(1410.0,392.0){\rule[-0.200pt]{0.400pt}{0.482pt}}
\put(1410.0,392.0){\usebox{\plotpoint}}
\put(1411.0,391.0){\usebox{\plotpoint}}
\put(1411.0,391.0){\usebox{\plotpoint}}
\put(1412.0,389.0){\rule[-0.200pt]{0.400pt}{0.482pt}}
\put(1412.0,389.0){\usebox{\plotpoint}}
\put(1413.0,388.0){\usebox{\plotpoint}}
\put(1413.0,388.0){\usebox{\plotpoint}}
\put(1414.0,386.0){\rule[-0.200pt]{0.400pt}{0.482pt}}
\put(1414.0,386.0){\usebox{\plotpoint}}
\put(1415.0,385.0){\usebox{\plotpoint}}
\put(1415.0,385.0){\usebox{\plotpoint}}
\put(1416,382.67){\rule{0.241pt}{0.400pt}}
\multiput(1416.00,383.17)(0.500,-1.000){2}{\rule{0.120pt}{0.400pt}}
\put(1416.0,384.0){\usebox{\plotpoint}}
\put(1417,383){\usebox{\plotpoint}}
\put(1417,383){\usebox{\plotpoint}}
\put(1417,383){\usebox{\plotpoint}}
\put(1417,383){\usebox{\plotpoint}}
\put(1417.0,382.0){\usebox{\plotpoint}}
\put(1417.0,382.0){\usebox{\plotpoint}}
\put(1418,379.67){\rule{0.241pt}{0.400pt}}
\multiput(1418.00,380.17)(0.500,-1.000){2}{\rule{0.120pt}{0.400pt}}
\put(1418.0,381.0){\usebox{\plotpoint}}
\put(1419,380){\usebox{\plotpoint}}
\put(1419,380){\usebox{\plotpoint}}
\put(1419,380){\usebox{\plotpoint}}
\put(1419,380){\usebox{\plotpoint}}
\put(1419,380){\usebox{\plotpoint}}
\put(1419.0,379.0){\usebox{\plotpoint}}
\put(1419.0,379.0){\usebox{\plotpoint}}
\put(1420.0,378.0){\usebox{\plotpoint}}
\put(1420.0,378.0){\usebox{\plotpoint}}
\put(1421.0,376.0){\rule[-0.200pt]{0.400pt}{0.482pt}}
\put(1421.0,376.0){\usebox{\plotpoint}}
\put(1422.0,375.0){\usebox{\plotpoint}}
\put(1422.0,375.0){\usebox{\plotpoint}}
\put(1423.0,373.0){\rule[-0.200pt]{0.400pt}{0.482pt}}
\put(1423.0,373.0){\usebox{\plotpoint}}
\put(1424.0,372.0){\usebox{\plotpoint}}
\put(1424.0,372.0){\usebox{\plotpoint}}
\put(1425.0,370.0){\rule[-0.200pt]{0.400pt}{0.482pt}}
\put(1425.0,370.0){\usebox{\plotpoint}}
\put(1426.0,369.0){\usebox{\plotpoint}}
\put(1426.0,369.0){\usebox{\plotpoint}}
\put(1427,366.67){\rule{0.241pt}{0.400pt}}
\multiput(1427.00,367.17)(0.500,-1.000){2}{\rule{0.120pt}{0.400pt}}
\put(1427.0,368.0){\usebox{\plotpoint}}
\put(1428,367){\usebox{\plotpoint}}
\put(1428,367){\usebox{\plotpoint}}
\put(1428,367){\usebox{\plotpoint}}
\put(1428,367){\usebox{\plotpoint}}
\put(1428.0,366.0){\usebox{\plotpoint}}
\put(1428.0,366.0){\usebox{\plotpoint}}
\put(1429,363.67){\rule{0.241pt}{0.400pt}}
\multiput(1429.00,364.17)(0.500,-1.000){2}{\rule{0.120pt}{0.400pt}}
\put(1429.0,365.0){\usebox{\plotpoint}}
\put(1430,364){\usebox{\plotpoint}}
\put(1430,364){\usebox{\plotpoint}}
\put(1430,364){\usebox{\plotpoint}}
\put(1430,364){\usebox{\plotpoint}}
\put(1430.0,363.0){\usebox{\plotpoint}}
\put(1430.0,363.0){\usebox{\plotpoint}}
\put(1431,360.67){\rule{0.241pt}{0.400pt}}
\multiput(1431.00,361.17)(0.500,-1.000){2}{\rule{0.120pt}{0.400pt}}
\put(1431.0,362.0){\usebox{\plotpoint}}
\put(1432,361){\usebox{\plotpoint}}
\put(1432,361){\usebox{\plotpoint}}
\put(1432,361){\usebox{\plotpoint}}
\put(1432,361){\usebox{\plotpoint}}
\put(1432,361){\usebox{\plotpoint}}
\put(1432.0,360.0){\usebox{\plotpoint}}
\put(1432.0,360.0){\usebox{\plotpoint}}
\put(1433.0,359.0){\usebox{\plotpoint}}
\put(1433.0,359.0){\usebox{\plotpoint}}
\put(1434.0,357.0){\rule[-0.200pt]{0.400pt}{0.482pt}}
\put(1434.0,357.0){\usebox{\plotpoint}}
\put(1435.0,356.0){\usebox{\plotpoint}}
\put(1435.0,356.0){\usebox{\plotpoint}}
\put(1436.0,354.0){\rule[-0.200pt]{0.400pt}{0.482pt}}
\put(1436.0,354.0){\usebox{\plotpoint}}
\put(1437.0,353.0){\usebox{\plotpoint}}
\put(1437.0,353.0){\usebox{\plotpoint}}
\put(1438.0,351.0){\rule[-0.200pt]{0.400pt}{0.482pt}}
\put(1438.0,351.0){\usebox{\plotpoint}}
\put(1439.0,350.0){\usebox{\plotpoint}}
\put(1279,778){\makebox(0,0)[r]{Namerné hodnoty}}
\put(753,192){\makebox(0,0){$\times$}}
\put(815,859){\makebox(0,0){$\times$}}
\put(862,161){\makebox(0,0){$\times$}}
\put(909,637){\makebox(0,0){$\times$}}
\put(955,283){\makebox(0,0){$\times$}}
\put(1349,778){\makebox(0,0){$\times$}}
\put(191.0,131.0){\rule[-0.200pt]{0.400pt}{175.375pt}}
\put(191.0,131.0){\rule[-0.200pt]{300.643pt}{0.400pt}}
\put(1439.0,131.0){\rule[-0.200pt]{0.400pt}{175.375pt}}
\put(191.0,859.0){\rule[-0.200pt]{300.643pt}{0.400pt}}
\end{picture}

\caption{Difrakčný obrazec pre 1 štrbinu, v porovnaní s teoretickou závislosťou. Kde $U/U_0$ je relatívny úbytok napätia a $\theta$ je pozorovaný uhol.}  \label{G_D1}
\end{figure}

\begin{figure}
% GNUPLOT: LaTeX picture
\setlength{\unitlength}{0.240900pt}
\ifx\plotpoint\undefined\newsavebox{\plotpoint}\fi
\begin{picture}(1500,900)(0,0)
\sbox{\plotpoint}{\rule[-0.200pt]{0.400pt}{0.400pt}}%
\put(171.0,131.0){\rule[-0.200pt]{4.818pt}{0.400pt}}
\put(151,131){\makebox(0,0)[r]{ 0}}
\put(1419.0,131.0){\rule[-0.200pt]{4.818pt}{0.400pt}}
\put(171.0,204.0){\rule[-0.200pt]{4.818pt}{0.400pt}}
\put(151,204){\makebox(0,0)[r]{ 0.1}}
\put(1419.0,204.0){\rule[-0.200pt]{4.818pt}{0.400pt}}
\put(171.0,277.0){\rule[-0.200pt]{4.818pt}{0.400pt}}
\put(151,277){\makebox(0,0)[r]{ 0.2}}
\put(1419.0,277.0){\rule[-0.200pt]{4.818pt}{0.400pt}}
\put(171.0,349.0){\rule[-0.200pt]{4.818pt}{0.400pt}}
\put(151,349){\makebox(0,0)[r]{ 0.3}}
\put(1419.0,349.0){\rule[-0.200pt]{4.818pt}{0.400pt}}
\put(171.0,422.0){\rule[-0.200pt]{4.818pt}{0.400pt}}
\put(151,422){\makebox(0,0)[r]{ 0.4}}
\put(1419.0,422.0){\rule[-0.200pt]{4.818pt}{0.400pt}}
\put(171.0,495.0){\rule[-0.200pt]{4.818pt}{0.400pt}}
\put(151,495){\makebox(0,0)[r]{ 0.5}}
\put(1419.0,495.0){\rule[-0.200pt]{4.818pt}{0.400pt}}
\put(171.0,568.0){\rule[-0.200pt]{4.818pt}{0.400pt}}
\put(151,568){\makebox(0,0)[r]{ 0.6}}
\put(1419.0,568.0){\rule[-0.200pt]{4.818pt}{0.400pt}}
\put(171.0,641.0){\rule[-0.200pt]{4.818pt}{0.400pt}}
\put(151,641){\makebox(0,0)[r]{ 0.7}}
\put(1419.0,641.0){\rule[-0.200pt]{4.818pt}{0.400pt}}
\put(171.0,713.0){\rule[-0.200pt]{4.818pt}{0.400pt}}
\put(151,713){\makebox(0,0)[r]{ 0.8}}
\put(1419.0,713.0){\rule[-0.200pt]{4.818pt}{0.400pt}}
\put(171.0,786.0){\rule[-0.200pt]{4.818pt}{0.400pt}}
\put(151,786){\makebox(0,0)[r]{ 0.9}}
\put(1419.0,786.0){\rule[-0.200pt]{4.818pt}{0.400pt}}
\put(171.0,859.0){\rule[-0.200pt]{4.818pt}{0.400pt}}
\put(151,859){\makebox(0,0)[r]{ 1}}
\put(1419.0,859.0){\rule[-0.200pt]{4.818pt}{0.400pt}}
\put(171.0,131.0){\rule[-0.200pt]{0.400pt}{4.818pt}}
\put(171,90){\makebox(0,0){-0.2}}
\put(171.0,839.0){\rule[-0.200pt]{0.400pt}{4.818pt}}
\put(330.0,131.0){\rule[-0.200pt]{0.400pt}{4.818pt}}
\put(330,90){\makebox(0,0){-0.15}}
\put(330.0,839.0){\rule[-0.200pt]{0.400pt}{4.818pt}}
\put(488.0,131.0){\rule[-0.200pt]{0.400pt}{4.818pt}}
\put(488,90){\makebox(0,0){-0.1}}
\put(488.0,839.0){\rule[-0.200pt]{0.400pt}{4.818pt}}
\put(647.0,131.0){\rule[-0.200pt]{0.400pt}{4.818pt}}
\put(647,90){\makebox(0,0){-0.05}}
\put(647.0,839.0){\rule[-0.200pt]{0.400pt}{4.818pt}}
\put(805.0,131.0){\rule[-0.200pt]{0.400pt}{4.818pt}}
\put(805,90){\makebox(0,0){ 0}}
\put(805.0,839.0){\rule[-0.200pt]{0.400pt}{4.818pt}}
\put(964.0,131.0){\rule[-0.200pt]{0.400pt}{4.818pt}}
\put(964,90){\makebox(0,0){ 0.05}}
\put(964.0,839.0){\rule[-0.200pt]{0.400pt}{4.818pt}}
\put(1122.0,131.0){\rule[-0.200pt]{0.400pt}{4.818pt}}
\put(1122,90){\makebox(0,0){ 0.1}}
\put(1122.0,839.0){\rule[-0.200pt]{0.400pt}{4.818pt}}
\put(1281.0,131.0){\rule[-0.200pt]{0.400pt}{4.818pt}}
\put(1281,90){\makebox(0,0){ 0.15}}
\put(1281.0,839.0){\rule[-0.200pt]{0.400pt}{4.818pt}}
\put(1439.0,131.0){\rule[-0.200pt]{0.400pt}{4.818pt}}
\put(1439,90){\makebox(0,0){ 0.2}}
\put(1439.0,839.0){\rule[-0.200pt]{0.400pt}{4.818pt}}
\put(171.0,131.0){\rule[-0.200pt]{0.400pt}{175.375pt}}
\put(171.0,131.0){\rule[-0.200pt]{305.461pt}{0.400pt}}
\put(1439.0,131.0){\rule[-0.200pt]{0.400pt}{175.375pt}}
\put(171.0,859.0){\rule[-0.200pt]{305.461pt}{0.400pt}}
\put(30,495){\makebox(0,0){\popi{U/U_0}{-}}}
\put(805,29){\makebox(0,0){\popi{\theta}{rad}}}
\put(1279,819){\makebox(0,0)[r]{Teoretická zavislosť}}
\put(1299.0,819.0){\rule[-0.200pt]{24.090pt}{0.400pt}}
\put(171,371){\usebox{\plotpoint}}
\put(171,371){\usebox{\plotpoint}}
\put(171.0,370.0){\usebox{\plotpoint}}
\put(171.0,370.0){\usebox{\plotpoint}}
\put(172.0,368.0){\rule[-0.200pt]{0.400pt}{0.482pt}}
\put(172.0,368.0){\usebox{\plotpoint}}
\put(173.0,366.0){\rule[-0.200pt]{0.400pt}{0.482pt}}
\put(173.0,366.0){\usebox{\plotpoint}}
\put(174.0,364.0){\rule[-0.200pt]{0.400pt}{0.482pt}}
\put(174.0,364.0){\usebox{\plotpoint}}
\put(175.0,362.0){\rule[-0.200pt]{0.400pt}{0.482pt}}
\put(175.0,362.0){\usebox{\plotpoint}}
\put(176.0,360.0){\rule[-0.200pt]{0.400pt}{0.482pt}}
\put(176.0,360.0){\usebox{\plotpoint}}
\put(177.0,358.0){\rule[-0.200pt]{0.400pt}{0.482pt}}
\put(177.0,358.0){\usebox{\plotpoint}}
\put(178,355.67){\rule{0.241pt}{0.400pt}}
\multiput(178.00,356.17)(0.500,-1.000){2}{\rule{0.120pt}{0.400pt}}
\put(178.0,357.0){\usebox{\plotpoint}}
\put(179,356){\usebox{\plotpoint}}
\put(179,356){\usebox{\plotpoint}}
\put(179,356){\usebox{\plotpoint}}
\put(179,353.67){\rule{0.241pt}{0.400pt}}
\multiput(179.00,354.17)(0.500,-1.000){2}{\rule{0.120pt}{0.400pt}}
\put(179.0,355.0){\usebox{\plotpoint}}
\put(180,354){\usebox{\plotpoint}}
\put(180,354){\usebox{\plotpoint}}
\put(180,354){\usebox{\plotpoint}}
\put(180.0,353.0){\usebox{\plotpoint}}
\put(180.0,353.0){\usebox{\plotpoint}}
\put(181.0,351.0){\rule[-0.200pt]{0.400pt}{0.482pt}}
\put(181.0,351.0){\usebox{\plotpoint}}
\put(182.0,349.0){\rule[-0.200pt]{0.400pt}{0.482pt}}
\put(182.0,349.0){\usebox{\plotpoint}}
\put(183.0,347.0){\rule[-0.200pt]{0.400pt}{0.482pt}}
\put(183.0,347.0){\usebox{\plotpoint}}
\put(184.0,345.0){\rule[-0.200pt]{0.400pt}{0.482pt}}
\put(184.0,345.0){\usebox{\plotpoint}}
\put(185.0,343.0){\rule[-0.200pt]{0.400pt}{0.482pt}}
\put(185.0,343.0){\usebox{\plotpoint}}
\put(186.0,341.0){\rule[-0.200pt]{0.400pt}{0.482pt}}
\put(186.0,341.0){\usebox{\plotpoint}}
\put(187.0,339.0){\rule[-0.200pt]{0.400pt}{0.482pt}}
\put(187.0,339.0){\usebox{\plotpoint}}
\put(188,336.67){\rule{0.241pt}{0.400pt}}
\multiput(188.00,337.17)(0.500,-1.000){2}{\rule{0.120pt}{0.400pt}}
\put(188.0,338.0){\usebox{\plotpoint}}
\put(189,337){\usebox{\plotpoint}}
\put(189,337){\usebox{\plotpoint}}
\put(189,337){\usebox{\plotpoint}}
\put(189.0,336.0){\usebox{\plotpoint}}
\put(189.0,336.0){\usebox{\plotpoint}}
\put(190.0,334.0){\rule[-0.200pt]{0.400pt}{0.482pt}}
\put(190.0,334.0){\usebox{\plotpoint}}
\put(191.0,332.0){\rule[-0.200pt]{0.400pt}{0.482pt}}
\put(191.0,332.0){\usebox{\plotpoint}}
\put(192.0,330.0){\rule[-0.200pt]{0.400pt}{0.482pt}}
\put(192.0,330.0){\usebox{\plotpoint}}
\put(193.0,328.0){\rule[-0.200pt]{0.400pt}{0.482pt}}
\put(193.0,328.0){\usebox{\plotpoint}}
\put(194.0,326.0){\rule[-0.200pt]{0.400pt}{0.482pt}}
\put(194.0,326.0){\usebox{\plotpoint}}
\put(195.0,324.0){\rule[-0.200pt]{0.400pt}{0.482pt}}
\put(195.0,324.0){\usebox{\plotpoint}}
\put(196.0,322.0){\rule[-0.200pt]{0.400pt}{0.482pt}}
\put(196.0,322.0){\usebox{\plotpoint}}
\put(197,319.67){\rule{0.241pt}{0.400pt}}
\multiput(197.00,320.17)(0.500,-1.000){2}{\rule{0.120pt}{0.400pt}}
\put(197.0,321.0){\usebox{\plotpoint}}
\put(198,320){\usebox{\plotpoint}}
\put(198,320){\usebox{\plotpoint}}
\put(198,320){\usebox{\plotpoint}}
\put(198,320){\usebox{\plotpoint}}
\put(198.0,319.0){\usebox{\plotpoint}}
\put(198.0,319.0){\usebox{\plotpoint}}
\put(199.0,317.0){\rule[-0.200pt]{0.400pt}{0.482pt}}
\put(199.0,317.0){\usebox{\plotpoint}}
\put(200.0,315.0){\rule[-0.200pt]{0.400pt}{0.482pt}}
\put(200.0,315.0){\usebox{\plotpoint}}
\put(201.0,313.0){\rule[-0.200pt]{0.400pt}{0.482pt}}
\put(201.0,313.0){\usebox{\plotpoint}}
\put(202.0,311.0){\rule[-0.200pt]{0.400pt}{0.482pt}}
\put(202.0,311.0){\usebox{\plotpoint}}
\put(203.0,309.0){\rule[-0.200pt]{0.400pt}{0.482pt}}
\put(203.0,309.0){\usebox{\plotpoint}}
\put(204.0,307.0){\rule[-0.200pt]{0.400pt}{0.482pt}}
\put(204.0,307.0){\usebox{\plotpoint}}
\put(205,304.67){\rule{0.241pt}{0.400pt}}
\multiput(205.00,305.17)(0.500,-1.000){2}{\rule{0.120pt}{0.400pt}}
\put(205.0,306.0){\usebox{\plotpoint}}
\put(206,305){\usebox{\plotpoint}}
\put(206,305){\usebox{\plotpoint}}
\put(206,305){\usebox{\plotpoint}}
\put(206.0,304.0){\usebox{\plotpoint}}
\put(206.0,304.0){\usebox{\plotpoint}}
\put(207.0,302.0){\rule[-0.200pt]{0.400pt}{0.482pt}}
\put(207.0,302.0){\usebox{\plotpoint}}
\put(208.0,300.0){\rule[-0.200pt]{0.400pt}{0.482pt}}
\put(208.0,300.0){\usebox{\plotpoint}}
\put(209.0,298.0){\rule[-0.200pt]{0.400pt}{0.482pt}}
\put(209.0,298.0){\usebox{\plotpoint}}
\put(210,295.67){\rule{0.241pt}{0.400pt}}
\multiput(210.00,296.17)(0.500,-1.000){2}{\rule{0.120pt}{0.400pt}}
\put(210.0,297.0){\usebox{\plotpoint}}
\put(211,296){\usebox{\plotpoint}}
\put(211,296){\usebox{\plotpoint}}
\put(211,296){\usebox{\plotpoint}}
\put(211,293.67){\rule{0.241pt}{0.400pt}}
\multiput(211.00,294.17)(0.500,-1.000){2}{\rule{0.120pt}{0.400pt}}
\put(211.0,295.0){\usebox{\plotpoint}}
\put(212,294){\usebox{\plotpoint}}
\put(212,294){\usebox{\plotpoint}}
\put(212,294){\usebox{\plotpoint}}
\put(212,294){\usebox{\plotpoint}}
\put(212.0,293.0){\usebox{\plotpoint}}
\put(212.0,293.0){\usebox{\plotpoint}}
\put(213.0,291.0){\rule[-0.200pt]{0.400pt}{0.482pt}}
\put(213.0,291.0){\usebox{\plotpoint}}
\put(214.0,289.0){\rule[-0.200pt]{0.400pt}{0.482pt}}
\put(214.0,289.0){\usebox{\plotpoint}}
\put(215,286.67){\rule{0.241pt}{0.400pt}}
\multiput(215.00,287.17)(0.500,-1.000){2}{\rule{0.120pt}{0.400pt}}
\put(215.0,288.0){\usebox{\plotpoint}}
\put(216,287){\usebox{\plotpoint}}
\put(216,287){\usebox{\plotpoint}}
\put(216,287){\usebox{\plotpoint}}
\put(216,284.67){\rule{0.241pt}{0.400pt}}
\multiput(216.00,285.17)(0.500,-1.000){2}{\rule{0.120pt}{0.400pt}}
\put(216.0,286.0){\usebox{\plotpoint}}
\put(217,285){\usebox{\plotpoint}}
\put(217,285){\usebox{\plotpoint}}
\put(217,285){\usebox{\plotpoint}}
\put(217,285){\usebox{\plotpoint}}
\put(217.0,284.0){\usebox{\plotpoint}}
\put(217.0,284.0){\usebox{\plotpoint}}
\put(218.0,282.0){\rule[-0.200pt]{0.400pt}{0.482pt}}
\put(218.0,282.0){\usebox{\plotpoint}}
\put(219.0,280.0){\rule[-0.200pt]{0.400pt}{0.482pt}}
\put(219.0,280.0){\usebox{\plotpoint}}
\put(220.0,278.0){\rule[-0.200pt]{0.400pt}{0.482pt}}
\put(220.0,278.0){\usebox{\plotpoint}}
\put(221,275.67){\rule{0.241pt}{0.400pt}}
\multiput(221.00,276.17)(0.500,-1.000){2}{\rule{0.120pt}{0.400pt}}
\put(221.0,277.0){\usebox{\plotpoint}}
\put(222,276){\usebox{\plotpoint}}
\put(222,276){\usebox{\plotpoint}}
\put(222,276){\usebox{\plotpoint}}
\put(222,276){\usebox{\plotpoint}}
\put(222.0,275.0){\usebox{\plotpoint}}
\put(222.0,275.0){\usebox{\plotpoint}}
\put(223.0,273.0){\rule[-0.200pt]{0.400pt}{0.482pt}}
\put(223.0,273.0){\usebox{\plotpoint}}
\put(224,270.67){\rule{0.241pt}{0.400pt}}
\multiput(224.00,271.17)(0.500,-1.000){2}{\rule{0.120pt}{0.400pt}}
\put(224.0,272.0){\usebox{\plotpoint}}
\put(225,271){\usebox{\plotpoint}}
\put(225,271){\usebox{\plotpoint}}
\put(225,271){\usebox{\plotpoint}}
\put(225.0,270.0){\usebox{\plotpoint}}
\put(225.0,270.0){\usebox{\plotpoint}}
\put(226.0,268.0){\rule[-0.200pt]{0.400pt}{0.482pt}}
\put(226.0,268.0){\usebox{\plotpoint}}
\put(227.0,266.0){\rule[-0.200pt]{0.400pt}{0.482pt}}
\put(227.0,266.0){\usebox{\plotpoint}}
\put(228,263.67){\rule{0.241pt}{0.400pt}}
\multiput(228.00,264.17)(0.500,-1.000){2}{\rule{0.120pt}{0.400pt}}
\put(228.0,265.0){\usebox{\plotpoint}}
\put(229,264){\usebox{\plotpoint}}
\put(229,264){\usebox{\plotpoint}}
\put(229,264){\usebox{\plotpoint}}
\put(229.0,263.0){\usebox{\plotpoint}}
\put(229.0,263.0){\usebox{\plotpoint}}
\put(230.0,261.0){\rule[-0.200pt]{0.400pt}{0.482pt}}
\put(230.0,261.0){\usebox{\plotpoint}}
\put(231.0,259.0){\rule[-0.200pt]{0.400pt}{0.482pt}}
\put(231.0,259.0){\usebox{\plotpoint}}
\put(232.0,258.0){\usebox{\plotpoint}}
\put(232.0,258.0){\usebox{\plotpoint}}
\put(233.0,256.0){\rule[-0.200pt]{0.400pt}{0.482pt}}
\put(233.0,256.0){\usebox{\plotpoint}}
\put(234.0,254.0){\rule[-0.200pt]{0.400pt}{0.482pt}}
\put(234.0,254.0){\usebox{\plotpoint}}
\put(235,251.67){\rule{0.241pt}{0.400pt}}
\multiput(235.00,252.17)(0.500,-1.000){2}{\rule{0.120pt}{0.400pt}}
\put(235.0,253.0){\usebox{\plotpoint}}
\put(236,252){\usebox{\plotpoint}}
\put(236,252){\usebox{\plotpoint}}
\put(236,252){\usebox{\plotpoint}}
\put(236,252){\usebox{\plotpoint}}
\put(236.0,251.0){\usebox{\plotpoint}}
\put(236.0,251.0){\usebox{\plotpoint}}
\put(237.0,249.0){\rule[-0.200pt]{0.400pt}{0.482pt}}
\put(237.0,249.0){\usebox{\plotpoint}}
\put(238,246.67){\rule{0.241pt}{0.400pt}}
\multiput(238.00,247.17)(0.500,-1.000){2}{\rule{0.120pt}{0.400pt}}
\put(238.0,248.0){\usebox{\plotpoint}}
\put(239,247){\usebox{\plotpoint}}
\put(239,247){\usebox{\plotpoint}}
\put(239,247){\usebox{\plotpoint}}
\put(239,247){\usebox{\plotpoint}}
\put(239.0,246.0){\usebox{\plotpoint}}
\put(239.0,246.0){\usebox{\plotpoint}}
\put(240.0,244.0){\rule[-0.200pt]{0.400pt}{0.482pt}}
\put(240.0,244.0){\usebox{\plotpoint}}
\put(241.0,243.0){\usebox{\plotpoint}}
\put(241.0,243.0){\usebox{\plotpoint}}
\put(242.0,241.0){\rule[-0.200pt]{0.400pt}{0.482pt}}
\put(242.0,241.0){\usebox{\plotpoint}}
\put(243,238.67){\rule{0.241pt}{0.400pt}}
\multiput(243.00,239.17)(0.500,-1.000){2}{\rule{0.120pt}{0.400pt}}
\put(243.0,240.0){\usebox{\plotpoint}}
\put(244,239){\usebox{\plotpoint}}
\put(244,239){\usebox{\plotpoint}}
\put(244,239){\usebox{\plotpoint}}
\put(244,239){\usebox{\plotpoint}}
\put(244.0,238.0){\usebox{\plotpoint}}
\put(244.0,238.0){\usebox{\plotpoint}}
\put(245.0,236.0){\rule[-0.200pt]{0.400pt}{0.482pt}}
\put(245.0,236.0){\usebox{\plotpoint}}
\put(246,233.67){\rule{0.241pt}{0.400pt}}
\multiput(246.00,234.17)(0.500,-1.000){2}{\rule{0.120pt}{0.400pt}}
\put(246.0,235.0){\usebox{\plotpoint}}
\put(247,234){\usebox{\plotpoint}}
\put(247,234){\usebox{\plotpoint}}
\put(247,234){\usebox{\plotpoint}}
\put(247,234){\usebox{\plotpoint}}
\put(247.0,233.0){\usebox{\plotpoint}}
\put(247.0,233.0){\usebox{\plotpoint}}
\put(248.0,231.0){\rule[-0.200pt]{0.400pt}{0.482pt}}
\put(248.0,231.0){\usebox{\plotpoint}}
\put(249.0,230.0){\usebox{\plotpoint}}
\put(249.0,230.0){\usebox{\plotpoint}}
\put(250.0,228.0){\rule[-0.200pt]{0.400pt}{0.482pt}}
\put(250.0,228.0){\usebox{\plotpoint}}
\put(251.0,227.0){\usebox{\plotpoint}}
\put(251.0,227.0){\usebox{\plotpoint}}
\put(252.0,225.0){\rule[-0.200pt]{0.400pt}{0.482pt}}
\put(252.0,225.0){\usebox{\plotpoint}}
\put(253.0,224.0){\usebox{\plotpoint}}
\put(253.0,224.0){\usebox{\plotpoint}}
\put(254.0,222.0){\rule[-0.200pt]{0.400pt}{0.482pt}}
\put(254.0,222.0){\usebox{\plotpoint}}
\put(255.0,221.0){\usebox{\plotpoint}}
\put(255.0,221.0){\usebox{\plotpoint}}
\put(256.0,219.0){\rule[-0.200pt]{0.400pt}{0.482pt}}
\put(256.0,219.0){\usebox{\plotpoint}}
\put(257.0,218.0){\usebox{\plotpoint}}
\put(257.0,218.0){\usebox{\plotpoint}}
\put(258.0,216.0){\rule[-0.200pt]{0.400pt}{0.482pt}}
\put(258.0,216.0){\usebox{\plotpoint}}
\put(259.0,215.0){\usebox{\plotpoint}}
\put(259.0,215.0){\usebox{\plotpoint}}
\put(260.0,213.0){\rule[-0.200pt]{0.400pt}{0.482pt}}
\put(260.0,213.0){\usebox{\plotpoint}}
\put(261.0,212.0){\usebox{\plotpoint}}
\put(261.0,212.0){\usebox{\plotpoint}}
\put(262.0,210.0){\rule[-0.200pt]{0.400pt}{0.482pt}}
\put(262.0,210.0){\usebox{\plotpoint}}
\put(263.0,209.0){\usebox{\plotpoint}}
\put(263.0,209.0){\usebox{\plotpoint}}
\put(264,206.67){\rule{0.241pt}{0.400pt}}
\multiput(264.00,207.17)(0.500,-1.000){2}{\rule{0.120pt}{0.400pt}}
\put(264.0,208.0){\usebox{\plotpoint}}
\put(265,207){\usebox{\plotpoint}}
\put(265,207){\usebox{\plotpoint}}
\put(265,207){\usebox{\plotpoint}}
\put(265,207){\usebox{\plotpoint}}
\put(265.0,206.0){\usebox{\plotpoint}}
\put(265.0,206.0){\usebox{\plotpoint}}
\put(266.0,205.0){\usebox{\plotpoint}}
\put(266.0,205.0){\usebox{\plotpoint}}
\put(267.0,203.0){\rule[-0.200pt]{0.400pt}{0.482pt}}
\put(267.0,203.0){\usebox{\plotpoint}}
\put(268.0,202.0){\usebox{\plotpoint}}
\put(268.0,202.0){\usebox{\plotpoint}}
\put(269.0,201.0){\usebox{\plotpoint}}
\put(269.0,201.0){\usebox{\plotpoint}}
\put(270.0,199.0){\rule[-0.200pt]{0.400pt}{0.482pt}}
\put(270.0,199.0){\usebox{\plotpoint}}
\put(271.0,198.0){\usebox{\plotpoint}}
\put(271.0,198.0){\usebox{\plotpoint}}
\put(272,195.67){\rule{0.241pt}{0.400pt}}
\multiput(272.00,196.17)(0.500,-1.000){2}{\rule{0.120pt}{0.400pt}}
\put(272.0,197.0){\usebox{\plotpoint}}
\put(273,196){\usebox{\plotpoint}}
\put(273,196){\usebox{\plotpoint}}
\put(273,196){\usebox{\plotpoint}}
\put(273,196){\usebox{\plotpoint}}
\put(273,196){\usebox{\plotpoint}}
\put(273.0,195.0){\usebox{\plotpoint}}
\put(273.0,195.0){\usebox{\plotpoint}}
\put(274.0,194.0){\usebox{\plotpoint}}
\put(274.0,194.0){\usebox{\plotpoint}}
\put(275,191.67){\rule{0.241pt}{0.400pt}}
\multiput(275.00,192.17)(0.500,-1.000){2}{\rule{0.120pt}{0.400pt}}
\put(275.0,193.0){\usebox{\plotpoint}}
\put(276,192){\usebox{\plotpoint}}
\put(276,192){\usebox{\plotpoint}}
\put(276,192){\usebox{\plotpoint}}
\put(276,192){\usebox{\plotpoint}}
\put(276,192){\usebox{\plotpoint}}
\put(276.0,191.0){\usebox{\plotpoint}}
\put(276.0,191.0){\usebox{\plotpoint}}
\put(277.0,190.0){\usebox{\plotpoint}}
\put(277.0,190.0){\usebox{\plotpoint}}
\put(278.0,189.0){\usebox{\plotpoint}}
\put(278.0,189.0){\usebox{\plotpoint}}
\put(279,186.67){\rule{0.241pt}{0.400pt}}
\multiput(279.00,187.17)(0.500,-1.000){2}{\rule{0.120pt}{0.400pt}}
\put(279.0,188.0){\usebox{\plotpoint}}
\put(280,187){\usebox{\plotpoint}}
\put(280,187){\usebox{\plotpoint}}
\put(280,187){\usebox{\plotpoint}}
\put(280,187){\usebox{\plotpoint}}
\put(280,187){\usebox{\plotpoint}}
\put(280.0,186.0){\usebox{\plotpoint}}
\put(280.0,186.0){\usebox{\plotpoint}}
\put(281.0,185.0){\usebox{\plotpoint}}
\put(281.0,185.0){\usebox{\plotpoint}}
\put(282.0,184.0){\usebox{\plotpoint}}
\put(282.0,184.0){\usebox{\plotpoint}}
\put(283,181.67){\rule{0.241pt}{0.400pt}}
\multiput(283.00,182.17)(0.500,-1.000){2}{\rule{0.120pt}{0.400pt}}
\put(283.0,183.0){\usebox{\plotpoint}}
\put(284,182){\usebox{\plotpoint}}
\put(284,182){\usebox{\plotpoint}}
\put(284,182){\usebox{\plotpoint}}
\put(284,182){\usebox{\plotpoint}}
\put(284,182){\usebox{\plotpoint}}
\put(284,182){\usebox{\plotpoint}}
\put(284.0,181.0){\usebox{\plotpoint}}
\put(284.0,181.0){\usebox{\plotpoint}}
\put(285.0,180.0){\usebox{\plotpoint}}
\put(285.0,180.0){\usebox{\plotpoint}}
\put(286.0,179.0){\usebox{\plotpoint}}
\put(286.0,179.0){\usebox{\plotpoint}}
\put(287.0,178.0){\usebox{\plotpoint}}
\put(287.0,178.0){\usebox{\plotpoint}}
\put(288.0,177.0){\usebox{\plotpoint}}
\put(288.0,177.0){\usebox{\plotpoint}}
\put(289,174.67){\rule{0.241pt}{0.400pt}}
\multiput(289.00,175.17)(0.500,-1.000){2}{\rule{0.120pt}{0.400pt}}
\put(289.0,176.0){\usebox{\plotpoint}}
\put(290,175){\usebox{\plotpoint}}
\put(290,175){\usebox{\plotpoint}}
\put(290,175){\usebox{\plotpoint}}
\put(290,175){\usebox{\plotpoint}}
\put(290,175){\usebox{\plotpoint}}
\put(290,175){\usebox{\plotpoint}}
\put(290.0,174.0){\usebox{\plotpoint}}
\put(290.0,174.0){\usebox{\plotpoint}}
\put(291.0,173.0){\usebox{\plotpoint}}
\put(291.0,173.0){\usebox{\plotpoint}}
\put(292.0,172.0){\usebox{\plotpoint}}
\put(292.0,172.0){\usebox{\plotpoint}}
\put(293.0,171.0){\usebox{\plotpoint}}
\put(293.0,171.0){\usebox{\plotpoint}}
\put(294.0,170.0){\usebox{\plotpoint}}
\put(294.0,170.0){\usebox{\plotpoint}}
\put(295.0,169.0){\usebox{\plotpoint}}
\put(295.0,169.0){\usebox{\plotpoint}}
\put(296.0,168.0){\usebox{\plotpoint}}
\put(296.0,168.0){\usebox{\plotpoint}}
\put(297.0,167.0){\usebox{\plotpoint}}
\put(297.0,167.0){\usebox{\plotpoint}}
\put(298.0,166.0){\usebox{\plotpoint}}
\put(298.0,166.0){\usebox{\plotpoint}}
\put(299.0,165.0){\usebox{\plotpoint}}
\put(299.0,165.0){\usebox{\plotpoint}}
\put(300.0,164.0){\usebox{\plotpoint}}
\put(300.0,164.0){\usebox{\plotpoint}}
\put(301.0,163.0){\usebox{\plotpoint}}
\put(301.0,163.0){\usebox{\plotpoint}}
\put(302.0,162.0){\usebox{\plotpoint}}
\put(302.0,162.0){\usebox{\plotpoint}}
\put(303.0,161.0){\usebox{\plotpoint}}
\put(303.0,161.0){\usebox{\plotpoint}}
\put(304.0,160.0){\usebox{\plotpoint}}
\put(304.0,160.0){\usebox{\plotpoint}}
\put(305.0,159.0){\usebox{\plotpoint}}
\put(305.0,159.0){\usebox{\plotpoint}}
\put(306.0,158.0){\usebox{\plotpoint}}
\put(306.0,158.0){\usebox{\plotpoint}}
\put(307.0,157.0){\usebox{\plotpoint}}
\put(307.0,157.0){\usebox{\plotpoint}}
\put(308.0,156.0){\usebox{\plotpoint}}
\put(309,154.67){\rule{0.241pt}{0.400pt}}
\multiput(309.00,155.17)(0.500,-1.000){2}{\rule{0.120pt}{0.400pt}}
\put(308.0,156.0){\usebox{\plotpoint}}
\put(310,155){\usebox{\plotpoint}}
\put(310,155){\usebox{\plotpoint}}
\put(310,155){\usebox{\plotpoint}}
\put(310,155){\usebox{\plotpoint}}
\put(310,155){\usebox{\plotpoint}}
\put(310,155){\usebox{\plotpoint}}
\put(310,155){\usebox{\plotpoint}}
\put(310.0,155.0){\usebox{\plotpoint}}
\put(311.0,154.0){\usebox{\plotpoint}}
\put(311.0,154.0){\usebox{\plotpoint}}
\put(312.0,153.0){\usebox{\plotpoint}}
\put(312.0,153.0){\usebox{\plotpoint}}
\put(313.0,152.0){\usebox{\plotpoint}}
\put(313.0,152.0){\usebox{\plotpoint}}
\put(314.0,151.0){\usebox{\plotpoint}}
\put(314.0,151.0){\rule[-0.200pt]{0.482pt}{0.400pt}}
\put(316.0,150.0){\usebox{\plotpoint}}
\put(316.0,150.0){\usebox{\plotpoint}}
\put(317.0,149.0){\usebox{\plotpoint}}
\put(317.0,149.0){\usebox{\plotpoint}}
\put(318.0,148.0){\usebox{\plotpoint}}
\put(318.0,148.0){\rule[-0.200pt]{0.482pt}{0.400pt}}
\put(320.0,147.0){\usebox{\plotpoint}}
\put(320.0,147.0){\usebox{\plotpoint}}
\put(321.0,146.0){\usebox{\plotpoint}}
\put(321.0,146.0){\rule[-0.200pt]{0.482pt}{0.400pt}}
\put(323.0,145.0){\usebox{\plotpoint}}
\put(323.0,145.0){\usebox{\plotpoint}}
\put(324.0,144.0){\usebox{\plotpoint}}
\put(324.0,144.0){\rule[-0.200pt]{0.482pt}{0.400pt}}
\put(326.0,143.0){\usebox{\plotpoint}}
\put(326.0,143.0){\usebox{\plotpoint}}
\put(327.0,142.0){\usebox{\plotpoint}}
\put(327.0,142.0){\rule[-0.200pt]{0.482pt}{0.400pt}}
\put(329.0,141.0){\usebox{\plotpoint}}
\put(329.0,141.0){\rule[-0.200pt]{0.482pt}{0.400pt}}
\put(331.0,140.0){\usebox{\plotpoint}}
\put(331.0,140.0){\rule[-0.200pt]{0.482pt}{0.400pt}}
\put(333.0,139.0){\usebox{\plotpoint}}
\put(333.0,139.0){\rule[-0.200pt]{0.482pt}{0.400pt}}
\put(335.0,138.0){\usebox{\plotpoint}}
\put(335.0,138.0){\rule[-0.200pt]{0.482pt}{0.400pt}}
\put(337.0,137.0){\usebox{\plotpoint}}
\put(337.0,137.0){\rule[-0.200pt]{0.482pt}{0.400pt}}
\put(339.0,136.0){\usebox{\plotpoint}}
\put(341,134.67){\rule{0.241pt}{0.400pt}}
\multiput(341.00,135.17)(0.500,-1.000){2}{\rule{0.120pt}{0.400pt}}
\put(339.0,136.0){\rule[-0.200pt]{0.482pt}{0.400pt}}
\put(342,135){\usebox{\plotpoint}}
\put(342,135){\usebox{\plotpoint}}
\put(342,135){\usebox{\plotpoint}}
\put(342,135){\usebox{\plotpoint}}
\put(342,135){\usebox{\plotpoint}}
\put(342,135){\usebox{\plotpoint}}
\put(342,135){\usebox{\plotpoint}}
\put(342.0,135.0){\rule[-0.200pt]{0.482pt}{0.400pt}}
\put(344.0,134.0){\usebox{\plotpoint}}
\put(347,132.67){\rule{0.241pt}{0.400pt}}
\multiput(347.00,133.17)(0.500,-1.000){2}{\rule{0.120pt}{0.400pt}}
\put(344.0,134.0){\rule[-0.200pt]{0.723pt}{0.400pt}}
\put(348,133){\usebox{\plotpoint}}
\put(348,133){\usebox{\plotpoint}}
\put(348,133){\usebox{\plotpoint}}
\put(348,133){\usebox{\plotpoint}}
\put(348,133){\usebox{\plotpoint}}
\put(348,133){\usebox{\plotpoint}}
\put(348,133){\usebox{\plotpoint}}
\put(348.0,133.0){\rule[-0.200pt]{0.723pt}{0.400pt}}
\put(351.0,132.0){\usebox{\plotpoint}}
\put(351.0,132.0){\rule[-0.200pt]{1.445pt}{0.400pt}}
\put(357.0,131.0){\usebox{\plotpoint}}
\put(357.0,131.0){\rule[-0.200pt]{3.854pt}{0.400pt}}
\put(373.0,131.0){\usebox{\plotpoint}}
\put(373.0,132.0){\rule[-0.200pt]{1.204pt}{0.400pt}}
\put(378.0,132.0){\usebox{\plotpoint}}
\put(378.0,133.0){\rule[-0.200pt]{0.964pt}{0.400pt}}
\put(382.0,133.0){\usebox{\plotpoint}}
\put(382.0,134.0){\rule[-0.200pt]{0.723pt}{0.400pt}}
\put(385.0,134.0){\usebox{\plotpoint}}
\put(385.0,135.0){\rule[-0.200pt]{0.723pt}{0.400pt}}
\put(388.0,135.0){\usebox{\plotpoint}}
\put(390,135.67){\rule{0.241pt}{0.400pt}}
\multiput(390.00,135.17)(0.500,1.000){2}{\rule{0.120pt}{0.400pt}}
\put(388.0,136.0){\rule[-0.200pt]{0.482pt}{0.400pt}}
\put(391,137){\usebox{\plotpoint}}
\put(391,137){\usebox{\plotpoint}}
\put(391,137){\usebox{\plotpoint}}
\put(391,137){\usebox{\plotpoint}}
\put(391,137){\usebox{\plotpoint}}
\put(391,137){\usebox{\plotpoint}}
\put(391,137){\usebox{\plotpoint}}
\put(391.0,137.0){\rule[-0.200pt]{0.482pt}{0.400pt}}
\put(393.0,137.0){\usebox{\plotpoint}}
\put(393.0,138.0){\rule[-0.200pt]{0.482pt}{0.400pt}}
\put(395.0,138.0){\usebox{\plotpoint}}
\put(395.0,139.0){\rule[-0.200pt]{0.482pt}{0.400pt}}
\put(397.0,139.0){\usebox{\plotpoint}}
\put(398,139.67){\rule{0.241pt}{0.400pt}}
\multiput(398.00,139.17)(0.500,1.000){2}{\rule{0.120pt}{0.400pt}}
\put(397.0,140.0){\usebox{\plotpoint}}
\put(399,141){\usebox{\plotpoint}}
\put(399,141){\usebox{\plotpoint}}
\put(399,141){\usebox{\plotpoint}}
\put(399,141){\usebox{\plotpoint}}
\put(399,141){\usebox{\plotpoint}}
\put(399,141){\usebox{\plotpoint}}
\put(399,141){\usebox{\plotpoint}}
\put(399.0,141.0){\usebox{\plotpoint}}
\put(400.0,141.0){\usebox{\plotpoint}}
\put(400.0,142.0){\rule[-0.200pt]{0.482pt}{0.400pt}}
\put(402.0,142.0){\usebox{\plotpoint}}
\put(402.0,143.0){\usebox{\plotpoint}}
\put(403.0,143.0){\usebox{\plotpoint}}
\put(403.0,144.0){\rule[-0.200pt]{0.482pt}{0.400pt}}
\put(405.0,144.0){\usebox{\plotpoint}}
\put(405.0,145.0){\usebox{\plotpoint}}
\put(406.0,145.0){\usebox{\plotpoint}}
\put(406.0,146.0){\rule[-0.200pt]{0.482pt}{0.400pt}}
\put(408.0,146.0){\usebox{\plotpoint}}
\put(408.0,147.0){\usebox{\plotpoint}}
\put(409.0,147.0){\usebox{\plotpoint}}
\put(409.0,148.0){\usebox{\plotpoint}}
\put(410.0,148.0){\usebox{\plotpoint}}
\put(410.0,149.0){\rule[-0.200pt]{0.482pt}{0.400pt}}
\put(412.0,149.0){\usebox{\plotpoint}}
\put(412.0,150.0){\usebox{\plotpoint}}
\put(413.0,150.0){\usebox{\plotpoint}}
\put(413.0,151.0){\usebox{\plotpoint}}
\put(414.0,151.0){\usebox{\plotpoint}}
\put(414.0,152.0){\usebox{\plotpoint}}
\put(415.0,152.0){\usebox{\plotpoint}}
\put(416,152.67){\rule{0.241pt}{0.400pt}}
\multiput(416.00,152.17)(0.500,1.000){2}{\rule{0.120pt}{0.400pt}}
\put(415.0,153.0){\usebox{\plotpoint}}
\put(417,154){\usebox{\plotpoint}}
\put(417,154){\usebox{\plotpoint}}
\put(417,154){\usebox{\plotpoint}}
\put(417,154){\usebox{\plotpoint}}
\put(417,154){\usebox{\plotpoint}}
\put(417,154){\usebox{\plotpoint}}
\put(417,154){\usebox{\plotpoint}}
\put(417.0,154.0){\usebox{\plotpoint}}
\put(418.0,154.0){\usebox{\plotpoint}}
\put(418.0,155.0){\usebox{\plotpoint}}
\put(419.0,155.0){\usebox{\plotpoint}}
\put(419.0,156.0){\usebox{\plotpoint}}
\put(420.0,156.0){\usebox{\plotpoint}}
\put(420.0,157.0){\usebox{\plotpoint}}
\put(421.0,157.0){\usebox{\plotpoint}}
\put(421.0,158.0){\usebox{\plotpoint}}
\put(422.0,158.0){\usebox{\plotpoint}}
\put(422.0,159.0){\usebox{\plotpoint}}
\put(423.0,159.0){\usebox{\plotpoint}}
\put(423.0,160.0){\usebox{\plotpoint}}
\put(424.0,160.0){\usebox{\plotpoint}}
\put(424.0,161.0){\usebox{\plotpoint}}
\put(425.0,161.0){\usebox{\plotpoint}}
\put(425.0,162.0){\usebox{\plotpoint}}
\put(426.0,162.0){\usebox{\plotpoint}}
\put(426.0,163.0){\usebox{\plotpoint}}
\put(427.0,163.0){\usebox{\plotpoint}}
\put(427.0,164.0){\usebox{\plotpoint}}
\put(428.0,164.0){\usebox{\plotpoint}}
\put(428.0,165.0){\usebox{\plotpoint}}
\put(429.0,165.0){\usebox{\plotpoint}}
\put(429.0,166.0){\usebox{\plotpoint}}
\put(430.0,166.0){\usebox{\plotpoint}}
\put(430.0,167.0){\usebox{\plotpoint}}
\put(431,167.67){\rule{0.241pt}{0.400pt}}
\multiput(431.00,167.17)(0.500,1.000){2}{\rule{0.120pt}{0.400pt}}
\put(431.0,167.0){\usebox{\plotpoint}}
\put(432,169){\usebox{\plotpoint}}
\put(432,169){\usebox{\plotpoint}}
\put(432,169){\usebox{\plotpoint}}
\put(432,169){\usebox{\plotpoint}}
\put(432,169){\usebox{\plotpoint}}
\put(432,169){\usebox{\plotpoint}}
\put(432.0,169.0){\usebox{\plotpoint}}
\put(432.0,170.0){\usebox{\plotpoint}}
\put(433.0,170.0){\usebox{\plotpoint}}
\put(433.0,171.0){\usebox{\plotpoint}}
\put(434.0,171.0){\usebox{\plotpoint}}
\put(434.0,172.0){\usebox{\plotpoint}}
\put(435.0,172.0){\usebox{\plotpoint}}
\put(435.0,173.0){\usebox{\plotpoint}}
\put(436.0,173.0){\usebox{\plotpoint}}
\put(436.0,174.0){\usebox{\plotpoint}}
\put(437,174.67){\rule{0.241pt}{0.400pt}}
\multiput(437.00,174.17)(0.500,1.000){2}{\rule{0.120pt}{0.400pt}}
\put(437.0,174.0){\usebox{\plotpoint}}
\put(438,176){\usebox{\plotpoint}}
\put(438,176){\usebox{\plotpoint}}
\put(438,176){\usebox{\plotpoint}}
\put(438,176){\usebox{\plotpoint}}
\put(438,176){\usebox{\plotpoint}}
\put(438,176){\usebox{\plotpoint}}
\put(438.0,176.0){\usebox{\plotpoint}}
\put(438.0,177.0){\usebox{\plotpoint}}
\put(439.0,177.0){\usebox{\plotpoint}}
\put(439.0,178.0){\usebox{\plotpoint}}
\put(440.0,178.0){\usebox{\plotpoint}}
\put(440.0,179.0){\usebox{\plotpoint}}
\put(441,179.67){\rule{0.241pt}{0.400pt}}
\multiput(441.00,179.17)(0.500,1.000){2}{\rule{0.120pt}{0.400pt}}
\put(441.0,179.0){\usebox{\plotpoint}}
\put(442,181){\usebox{\plotpoint}}
\put(442,181){\usebox{\plotpoint}}
\put(442,181){\usebox{\plotpoint}}
\put(442,181){\usebox{\plotpoint}}
\put(442,181){\usebox{\plotpoint}}
\put(442.0,181.0){\usebox{\plotpoint}}
\put(442.0,182.0){\usebox{\plotpoint}}
\put(443.0,182.0){\usebox{\plotpoint}}
\put(443.0,183.0){\usebox{\plotpoint}}
\put(444.0,183.0){\usebox{\plotpoint}}
\put(444.0,184.0){\usebox{\plotpoint}}
\put(445.0,184.0){\rule[-0.200pt]{0.400pt}{0.482pt}}
\put(445.0,186.0){\usebox{\plotpoint}}
\put(446.0,186.0){\usebox{\plotpoint}}
\put(446.0,187.0){\usebox{\plotpoint}}
\put(447.0,187.0){\usebox{\plotpoint}}
\put(447.0,188.0){\usebox{\plotpoint}}
\put(448.0,188.0){\rule[-0.200pt]{0.400pt}{0.482pt}}
\put(448.0,190.0){\usebox{\plotpoint}}
\put(449.0,190.0){\usebox{\plotpoint}}
\put(449.0,191.0){\usebox{\plotpoint}}
\put(450.0,191.0){\rule[-0.200pt]{0.400pt}{0.482pt}}
\put(450.0,193.0){\usebox{\plotpoint}}
\put(451.0,193.0){\usebox{\plotpoint}}
\put(451.0,194.0){\usebox{\plotpoint}}
\put(452.0,194.0){\usebox{\plotpoint}}
\put(452.0,195.0){\usebox{\plotpoint}}
\put(453.0,195.0){\rule[-0.200pt]{0.400pt}{0.482pt}}
\put(453.0,197.0){\usebox{\plotpoint}}
\put(454.0,197.0){\usebox{\plotpoint}}
\put(454.0,198.0){\usebox{\plotpoint}}
\put(455.0,198.0){\rule[-0.200pt]{0.400pt}{0.482pt}}
\put(455.0,200.0){\usebox{\plotpoint}}
\put(456.0,200.0){\usebox{\plotpoint}}
\put(456.0,201.0){\usebox{\plotpoint}}
\put(457.0,201.0){\rule[-0.200pt]{0.400pt}{0.482pt}}
\put(457.0,203.0){\usebox{\plotpoint}}
\put(458,203.67){\rule{0.241pt}{0.400pt}}
\multiput(458.00,203.17)(0.500,1.000){2}{\rule{0.120pt}{0.400pt}}
\put(458.0,203.0){\usebox{\plotpoint}}
\put(459,205){\usebox{\plotpoint}}
\put(459,205){\usebox{\plotpoint}}
\put(459,205){\usebox{\plotpoint}}
\put(459,205){\usebox{\plotpoint}}
\put(459,205){\usebox{\plotpoint}}
\put(459.0,205.0){\usebox{\plotpoint}}
\put(459.0,206.0){\usebox{\plotpoint}}
\put(460.0,206.0){\usebox{\plotpoint}}
\put(460.0,207.0){\usebox{\plotpoint}}
\put(461.0,207.0){\rule[-0.200pt]{0.400pt}{0.482pt}}
\put(461.0,209.0){\usebox{\plotpoint}}
\put(462,209.67){\rule{0.241pt}{0.400pt}}
\multiput(462.00,209.17)(0.500,1.000){2}{\rule{0.120pt}{0.400pt}}
\put(462.0,209.0){\usebox{\plotpoint}}
\put(463,211){\usebox{\plotpoint}}
\put(463,211){\usebox{\plotpoint}}
\put(463,211){\usebox{\plotpoint}}
\put(463,211){\usebox{\plotpoint}}
\put(463.0,211.0){\usebox{\plotpoint}}
\put(463.0,212.0){\usebox{\plotpoint}}
\put(464.0,212.0){\rule[-0.200pt]{0.400pt}{0.482pt}}
\put(464.0,214.0){\usebox{\plotpoint}}
\put(465,214.67){\rule{0.241pt}{0.400pt}}
\multiput(465.00,214.17)(0.500,1.000){2}{\rule{0.120pt}{0.400pt}}
\put(465.0,214.0){\usebox{\plotpoint}}
\put(466,216){\usebox{\plotpoint}}
\put(466,216){\usebox{\plotpoint}}
\put(466,216){\usebox{\plotpoint}}
\put(466,216){\usebox{\plotpoint}}
\put(466.0,216.0){\usebox{\plotpoint}}
\put(466.0,217.0){\usebox{\plotpoint}}
\put(467.0,217.0){\rule[-0.200pt]{0.400pt}{0.482pt}}
\put(467.0,219.0){\usebox{\plotpoint}}
\put(468.0,219.0){\usebox{\plotpoint}}
\put(468.0,220.0){\usebox{\plotpoint}}
\put(469.0,220.0){\rule[-0.200pt]{0.400pt}{0.482pt}}
\put(469.0,222.0){\usebox{\plotpoint}}
\put(470.0,222.0){\rule[-0.200pt]{0.400pt}{0.482pt}}
\put(470.0,224.0){\usebox{\plotpoint}}
\put(471,224.67){\rule{0.241pt}{0.400pt}}
\multiput(471.00,224.17)(0.500,1.000){2}{\rule{0.120pt}{0.400pt}}
\put(471.0,224.0){\usebox{\plotpoint}}
\put(472,226){\usebox{\plotpoint}}
\put(472,226){\usebox{\plotpoint}}
\put(472,226){\usebox{\plotpoint}}
\put(472,226){\usebox{\plotpoint}}
\put(472.0,226.0){\usebox{\plotpoint}}
\put(472.0,227.0){\usebox{\plotpoint}}
\put(473.0,227.0){\rule[-0.200pt]{0.400pt}{0.482pt}}
\put(473.0,229.0){\usebox{\plotpoint}}
\put(474.0,229.0){\rule[-0.200pt]{0.400pt}{0.482pt}}
\put(474.0,231.0){\usebox{\plotpoint}}
\put(475,231.67){\rule{0.241pt}{0.400pt}}
\multiput(475.00,231.17)(0.500,1.000){2}{\rule{0.120pt}{0.400pt}}
\put(475.0,231.0){\usebox{\plotpoint}}
\put(476,233){\usebox{\plotpoint}}
\put(476,233){\usebox{\plotpoint}}
\put(476,233){\usebox{\plotpoint}}
\put(476,233.67){\rule{0.241pt}{0.400pt}}
\multiput(476.00,233.17)(0.500,1.000){2}{\rule{0.120pt}{0.400pt}}
\put(476.0,233.0){\usebox{\plotpoint}}
\put(477,235){\usebox{\plotpoint}}
\put(477,235){\usebox{\plotpoint}}
\put(477,235){\usebox{\plotpoint}}
\put(477,235){\usebox{\plotpoint}}
\put(477.0,235.0){\usebox{\plotpoint}}
\put(477.0,236.0){\usebox{\plotpoint}}
\put(478.0,236.0){\rule[-0.200pt]{0.400pt}{0.482pt}}
\put(478.0,238.0){\usebox{\plotpoint}}
\put(479.0,238.0){\rule[-0.200pt]{0.400pt}{0.482pt}}
\put(479.0,240.0){\usebox{\plotpoint}}
\put(480,240.67){\rule{0.241pt}{0.400pt}}
\multiput(480.00,240.17)(0.500,1.000){2}{\rule{0.120pt}{0.400pt}}
\put(480.0,240.0){\usebox{\plotpoint}}
\put(481,242){\usebox{\plotpoint}}
\put(481,242){\usebox{\plotpoint}}
\put(481,242){\usebox{\plotpoint}}
\put(481,242.67){\rule{0.241pt}{0.400pt}}
\multiput(481.00,242.17)(0.500,1.000){2}{\rule{0.120pt}{0.400pt}}
\put(481.0,242.0){\usebox{\plotpoint}}
\put(482,244){\usebox{\plotpoint}}
\put(482,244){\usebox{\plotpoint}}
\put(482,244){\usebox{\plotpoint}}
\put(482,244){\usebox{\plotpoint}}
\put(482.0,244.0){\usebox{\plotpoint}}
\put(482.0,245.0){\usebox{\plotpoint}}
\put(483.0,245.0){\rule[-0.200pt]{0.400pt}{0.482pt}}
\put(483.0,247.0){\usebox{\plotpoint}}
\put(484.0,247.0){\rule[-0.200pt]{0.400pt}{0.482pt}}
\put(484.0,249.0){\usebox{\plotpoint}}
\put(485.0,249.0){\rule[-0.200pt]{0.400pt}{0.482pt}}
\put(485.0,251.0){\usebox{\plotpoint}}
\put(486.0,251.0){\rule[-0.200pt]{0.400pt}{0.482pt}}
\put(486.0,253.0){\usebox{\plotpoint}}
\put(487.0,253.0){\rule[-0.200pt]{0.400pt}{0.482pt}}
\put(487.0,255.0){\usebox{\plotpoint}}
\put(488.0,255.0){\rule[-0.200pt]{0.400pt}{0.482pt}}
\put(488.0,257.0){\usebox{\plotpoint}}
\put(489,257.67){\rule{0.241pt}{0.400pt}}
\multiput(489.00,257.17)(0.500,1.000){2}{\rule{0.120pt}{0.400pt}}
\put(489.0,257.0){\usebox{\plotpoint}}
\put(490,259){\usebox{\plotpoint}}
\put(490,259){\usebox{\plotpoint}}
\put(490,259){\usebox{\plotpoint}}
\put(490,259.67){\rule{0.241pt}{0.400pt}}
\multiput(490.00,259.17)(0.500,1.000){2}{\rule{0.120pt}{0.400pt}}
\put(490.0,259.0){\usebox{\plotpoint}}
\put(491,261){\usebox{\plotpoint}}
\put(491,261){\usebox{\plotpoint}}
\put(491,261){\usebox{\plotpoint}}
\put(491,261.67){\rule{0.241pt}{0.400pt}}
\multiput(491.00,261.17)(0.500,1.000){2}{\rule{0.120pt}{0.400pt}}
\put(491.0,261.0){\usebox{\plotpoint}}
\put(492,263){\usebox{\plotpoint}}
\put(492,263){\usebox{\plotpoint}}
\put(492,263){\usebox{\plotpoint}}
\put(492,263.67){\rule{0.241pt}{0.400pt}}
\multiput(492.00,263.17)(0.500,1.000){2}{\rule{0.120pt}{0.400pt}}
\put(492.0,263.0){\usebox{\plotpoint}}
\put(493,265){\usebox{\plotpoint}}
\put(493,265){\usebox{\plotpoint}}
\put(493.0,265.0){\rule[-0.200pt]{0.400pt}{0.482pt}}
\put(493.0,267.0){\usebox{\plotpoint}}
\put(494.0,267.0){\rule[-0.200pt]{0.400pt}{0.482pt}}
\put(494.0,269.0){\usebox{\plotpoint}}
\put(495,269.67){\rule{0.241pt}{0.400pt}}
\multiput(495.00,269.17)(0.500,1.000){2}{\rule{0.120pt}{0.400pt}}
\put(495.0,269.0){\usebox{\plotpoint}}
\put(496,271){\usebox{\plotpoint}}
\put(496,271){\usebox{\plotpoint}}
\put(496,271){\usebox{\plotpoint}}
\put(496,271.67){\rule{0.241pt}{0.400pt}}
\multiput(496.00,271.17)(0.500,1.000){2}{\rule{0.120pt}{0.400pt}}
\put(496.0,271.0){\usebox{\plotpoint}}
\put(497,273){\usebox{\plotpoint}}
\put(497,273){\usebox{\plotpoint}}
\put(497,273){\usebox{\plotpoint}}
\put(497,273.67){\rule{0.241pt}{0.400pt}}
\multiput(497.00,273.17)(0.500,1.000){2}{\rule{0.120pt}{0.400pt}}
\put(497.0,273.0){\usebox{\plotpoint}}
\put(498,275){\usebox{\plotpoint}}
\put(498,275){\usebox{\plotpoint}}
\put(498.0,275.0){\rule[-0.200pt]{0.400pt}{0.482pt}}
\put(498.0,277.0){\usebox{\plotpoint}}
\put(499.0,277.0){\rule[-0.200pt]{0.400pt}{0.482pt}}
\put(499.0,279.0){\usebox{\plotpoint}}
\put(500.0,279.0){\rule[-0.200pt]{0.400pt}{0.482pt}}
\put(500.0,281.0){\usebox{\plotpoint}}
\put(501.0,281.0){\rule[-0.200pt]{0.400pt}{0.482pt}}
\put(501.0,283.0){\usebox{\plotpoint}}
\put(502.0,283.0){\rule[-0.200pt]{0.400pt}{0.482pt}}
\put(502.0,285.0){\usebox{\plotpoint}}
\put(503.0,285.0){\rule[-0.200pt]{0.400pt}{0.482pt}}
\put(503.0,287.0){\usebox{\plotpoint}}
\put(504.0,287.0){\rule[-0.200pt]{0.400pt}{0.482pt}}
\put(504.0,289.0){\usebox{\plotpoint}}
\put(505.0,289.0){\rule[-0.200pt]{0.400pt}{0.482pt}}
\put(505.0,291.0){\usebox{\plotpoint}}
\put(506,292.67){\rule{0.241pt}{0.400pt}}
\multiput(506.00,292.17)(0.500,1.000){2}{\rule{0.120pt}{0.400pt}}
\put(506.0,291.0){\rule[-0.200pt]{0.400pt}{0.482pt}}
\put(507,294){\usebox{\plotpoint}}
\put(507,294){\usebox{\plotpoint}}
\put(507,294){\usebox{\plotpoint}}
\put(507.0,294.0){\rule[-0.200pt]{0.400pt}{0.482pt}}
\put(507.0,296.0){\usebox{\plotpoint}}
\put(508.0,296.0){\rule[-0.200pt]{0.400pt}{0.482pt}}
\put(508.0,298.0){\usebox{\plotpoint}}
\put(509.0,298.0){\rule[-0.200pt]{0.400pt}{0.482pt}}
\put(509.0,300.0){\usebox{\plotpoint}}
\put(510.0,300.0){\rule[-0.200pt]{0.400pt}{0.482pt}}
\put(510.0,302.0){\usebox{\plotpoint}}
\put(511,303.67){\rule{0.241pt}{0.400pt}}
\multiput(511.00,303.17)(0.500,1.000){2}{\rule{0.120pt}{0.400pt}}
\put(511.0,302.0){\rule[-0.200pt]{0.400pt}{0.482pt}}
\put(512,305){\usebox{\plotpoint}}
\put(512,305){\usebox{\plotpoint}}
\put(512,305.67){\rule{0.241pt}{0.400pt}}
\multiput(512.00,305.17)(0.500,1.000){2}{\rule{0.120pt}{0.400pt}}
\put(512.0,305.0){\usebox{\plotpoint}}
\put(513,307){\usebox{\plotpoint}}
\put(513,307){\usebox{\plotpoint}}
\put(513,307){\usebox{\plotpoint}}
\put(513.0,307.0){\rule[-0.200pt]{0.400pt}{0.482pt}}
\put(513.0,309.0){\usebox{\plotpoint}}
\put(514.0,309.0){\rule[-0.200pt]{0.400pt}{0.482pt}}
\put(514.0,311.0){\usebox{\plotpoint}}
\put(515.0,311.0){\rule[-0.200pt]{0.400pt}{0.482pt}}
\put(515.0,313.0){\usebox{\plotpoint}}
\put(516,314.67){\rule{0.241pt}{0.400pt}}
\multiput(516.00,314.17)(0.500,1.000){2}{\rule{0.120pt}{0.400pt}}
\put(516.0,313.0){\rule[-0.200pt]{0.400pt}{0.482pt}}
\put(517,316){\usebox{\plotpoint}}
\put(517,316){\usebox{\plotpoint}}
\put(517.0,316.0){\rule[-0.200pt]{0.400pt}{0.482pt}}
\put(517.0,318.0){\usebox{\plotpoint}}
\put(518.0,318.0){\rule[-0.200pt]{0.400pt}{0.482pt}}
\put(518.0,320.0){\usebox{\plotpoint}}
\put(519,321.67){\rule{0.241pt}{0.400pt}}
\multiput(519.00,321.17)(0.500,1.000){2}{\rule{0.120pt}{0.400pt}}
\put(519.0,320.0){\rule[-0.200pt]{0.400pt}{0.482pt}}
\put(520,323){\usebox{\plotpoint}}
\put(520,323){\usebox{\plotpoint}}
\put(520,323){\usebox{\plotpoint}}
\put(520.0,323.0){\rule[-0.200pt]{0.400pt}{0.482pt}}
\put(520.0,325.0){\usebox{\plotpoint}}
\put(521.0,325.0){\rule[-0.200pt]{0.400pt}{0.482pt}}
\put(521.0,327.0){\usebox{\plotpoint}}
\put(522.0,327.0){\rule[-0.200pt]{0.400pt}{0.482pt}}
\put(522.0,329.0){\usebox{\plotpoint}}
\put(523,330.67){\rule{0.241pt}{0.400pt}}
\multiput(523.00,330.17)(0.500,1.000){2}{\rule{0.120pt}{0.400pt}}
\put(523.0,329.0){\rule[-0.200pt]{0.400pt}{0.482pt}}
\put(524,332){\usebox{\plotpoint}}
\put(524,332){\usebox{\plotpoint}}
\put(524,332){\usebox{\plotpoint}}
\put(524.0,332.0){\rule[-0.200pt]{0.400pt}{0.482pt}}
\put(524.0,334.0){\usebox{\plotpoint}}
\put(525.0,334.0){\rule[-0.200pt]{0.400pt}{0.482pt}}
\put(525.0,336.0){\usebox{\plotpoint}}
\put(526,337.67){\rule{0.241pt}{0.400pt}}
\multiput(526.00,337.17)(0.500,1.000){2}{\rule{0.120pt}{0.400pt}}
\put(526.0,336.0){\rule[-0.200pt]{0.400pt}{0.482pt}}
\put(527,339){\usebox{\plotpoint}}
\put(527,339){\usebox{\plotpoint}}
\put(527.0,339.0){\rule[-0.200pt]{0.400pt}{0.482pt}}
\put(527.0,341.0){\usebox{\plotpoint}}
\put(528.0,341.0){\rule[-0.200pt]{0.400pt}{0.482pt}}
\put(528.0,343.0){\usebox{\plotpoint}}
\put(529.0,343.0){\rule[-0.200pt]{0.400pt}{0.723pt}}
\put(529.0,346.0){\usebox{\plotpoint}}
\put(530.0,346.0){\rule[-0.200pt]{0.400pt}{0.482pt}}
\put(530.0,348.0){\usebox{\plotpoint}}
\put(531.0,348.0){\rule[-0.200pt]{0.400pt}{0.482pt}}
\put(531.0,350.0){\usebox{\plotpoint}}
\put(532,351.67){\rule{0.241pt}{0.400pt}}
\multiput(532.00,351.17)(0.500,1.000){2}{\rule{0.120pt}{0.400pt}}
\put(532.0,350.0){\rule[-0.200pt]{0.400pt}{0.482pt}}
\put(533,353){\usebox{\plotpoint}}
\put(533,353){\usebox{\plotpoint}}
\put(533.0,353.0){\rule[-0.200pt]{0.400pt}{0.482pt}}
\put(533.0,355.0){\usebox{\plotpoint}}
\put(534,356.67){\rule{0.241pt}{0.400pt}}
\multiput(534.00,356.17)(0.500,1.000){2}{\rule{0.120pt}{0.400pt}}
\put(534.0,355.0){\rule[-0.200pt]{0.400pt}{0.482pt}}
\put(535,358){\usebox{\plotpoint}}
\put(535,358){\usebox{\plotpoint}}
\put(535.0,358.0){\rule[-0.200pt]{0.400pt}{0.482pt}}
\put(535.0,360.0){\usebox{\plotpoint}}
\put(536,361.67){\rule{0.241pt}{0.400pt}}
\multiput(536.00,361.17)(0.500,1.000){2}{\rule{0.120pt}{0.400pt}}
\put(536.0,360.0){\rule[-0.200pt]{0.400pt}{0.482pt}}
\put(537,363){\usebox{\plotpoint}}
\put(537,363){\usebox{\plotpoint}}
\put(537,363){\usebox{\plotpoint}}
\put(537.0,363.0){\rule[-0.200pt]{0.400pt}{0.482pt}}
\put(537.0,365.0){\usebox{\plotpoint}}
\put(538.0,365.0){\rule[-0.200pt]{0.400pt}{0.482pt}}
\put(538.0,367.0){\usebox{\plotpoint}}
\put(539,368.67){\rule{0.241pt}{0.400pt}}
\multiput(539.00,368.17)(0.500,1.000){2}{\rule{0.120pt}{0.400pt}}
\put(539.0,367.0){\rule[-0.200pt]{0.400pt}{0.482pt}}
\put(540,370){\usebox{\plotpoint}}
\put(540,370){\usebox{\plotpoint}}
\put(540.0,370.0){\rule[-0.200pt]{0.400pt}{0.482pt}}
\put(540.0,372.0){\usebox{\plotpoint}}
\put(541,373.67){\rule{0.241pt}{0.400pt}}
\multiput(541.00,373.17)(0.500,1.000){2}{\rule{0.120pt}{0.400pt}}
\put(541.0,372.0){\rule[-0.200pt]{0.400pt}{0.482pt}}
\put(542,375){\usebox{\plotpoint}}
\put(542,375){\usebox{\plotpoint}}
\put(542.0,375.0){\rule[-0.200pt]{0.400pt}{0.482pt}}
\put(542.0,377.0){\usebox{\plotpoint}}
\put(543,378.67){\rule{0.241pt}{0.400pt}}
\multiput(543.00,378.17)(0.500,1.000){2}{\rule{0.120pt}{0.400pt}}
\put(543.0,377.0){\rule[-0.200pt]{0.400pt}{0.482pt}}
\put(544,380){\usebox{\plotpoint}}
\put(544,380){\usebox{\plotpoint}}
\put(544.0,380.0){\rule[-0.200pt]{0.400pt}{0.482pt}}
\put(544.0,382.0){\usebox{\plotpoint}}
\put(545,383.67){\rule{0.241pt}{0.400pt}}
\multiput(545.00,383.17)(0.500,1.000){2}{\rule{0.120pt}{0.400pt}}
\put(545.0,382.0){\rule[-0.200pt]{0.400pt}{0.482pt}}
\put(546,385){\usebox{\plotpoint}}
\put(546,385){\usebox{\plotpoint}}
\put(546.0,385.0){\rule[-0.200pt]{0.400pt}{0.482pt}}
\put(546.0,387.0){\usebox{\plotpoint}}
\put(547.0,387.0){\rule[-0.200pt]{0.400pt}{0.482pt}}
\put(547.0,389.0){\usebox{\plotpoint}}
\put(548.0,389.0){\rule[-0.200pt]{0.400pt}{0.723pt}}
\put(548.0,392.0){\usebox{\plotpoint}}
\put(549.0,392.0){\rule[-0.200pt]{0.400pt}{0.482pt}}
\put(549.0,394.0){\usebox{\plotpoint}}
\put(550.0,394.0){\rule[-0.200pt]{0.400pt}{0.723pt}}
\put(550.0,397.0){\usebox{\plotpoint}}
\put(551.0,397.0){\rule[-0.200pt]{0.400pt}{0.482pt}}
\put(551.0,399.0){\usebox{\plotpoint}}
\put(552.0,399.0){\rule[-0.200pt]{0.400pt}{0.723pt}}
\put(552.0,402.0){\usebox{\plotpoint}}
\put(553,403.67){\rule{0.241pt}{0.400pt}}
\multiput(553.00,403.17)(0.500,1.000){2}{\rule{0.120pt}{0.400pt}}
\put(553.0,402.0){\rule[-0.200pt]{0.400pt}{0.482pt}}
\put(554,405){\usebox{\plotpoint}}
\put(554,405){\usebox{\plotpoint}}
\put(554.0,405.0){\rule[-0.200pt]{0.400pt}{0.482pt}}
\put(554.0,407.0){\usebox{\plotpoint}}
\put(555,408.67){\rule{0.241pt}{0.400pt}}
\multiput(555.00,408.17)(0.500,1.000){2}{\rule{0.120pt}{0.400pt}}
\put(555.0,407.0){\rule[-0.200pt]{0.400pt}{0.482pt}}
\put(556,410){\usebox{\plotpoint}}
\put(556,410){\usebox{\plotpoint}}
\put(556.0,410.0){\rule[-0.200pt]{0.400pt}{0.482pt}}
\put(556.0,412.0){\usebox{\plotpoint}}
\put(557,413.67){\rule{0.241pt}{0.400pt}}
\multiput(557.00,413.17)(0.500,1.000){2}{\rule{0.120pt}{0.400pt}}
\put(557.0,412.0){\rule[-0.200pt]{0.400pt}{0.482pt}}
\put(558,415){\usebox{\plotpoint}}
\put(558,415){\usebox{\plotpoint}}
\put(558,415){\usebox{\plotpoint}}
\put(558.0,415.0){\rule[-0.200pt]{0.400pt}{0.482pt}}
\put(558.0,417.0){\usebox{\plotpoint}}
\put(559,418.67){\rule{0.241pt}{0.400pt}}
\multiput(559.00,418.17)(0.500,1.000){2}{\rule{0.120pt}{0.400pt}}
\put(559.0,417.0){\rule[-0.200pt]{0.400pt}{0.482pt}}
\put(560,420){\usebox{\plotpoint}}
\put(560,420){\usebox{\plotpoint}}
\put(560.0,420.0){\rule[-0.200pt]{0.400pt}{0.482pt}}
\put(560.0,422.0){\usebox{\plotpoint}}
\put(561.0,422.0){\rule[-0.200pt]{0.400pt}{0.723pt}}
\put(561.0,425.0){\usebox{\plotpoint}}
\put(562.0,425.0){\rule[-0.200pt]{0.400pt}{0.482pt}}
\put(562.0,427.0){\usebox{\plotpoint}}
\put(563.0,427.0){\rule[-0.200pt]{0.400pt}{0.723pt}}
\put(563.0,430.0){\usebox{\plotpoint}}
\put(564,431.67){\rule{0.241pt}{0.400pt}}
\multiput(564.00,431.17)(0.500,1.000){2}{\rule{0.120pt}{0.400pt}}
\put(564.0,430.0){\rule[-0.200pt]{0.400pt}{0.482pt}}
\put(565,433){\usebox{\plotpoint}}
\put(565,433){\usebox{\plotpoint}}
\put(565.0,433.0){\rule[-0.200pt]{0.400pt}{0.482pt}}
\put(565.0,435.0){\usebox{\plotpoint}}
\put(566,436.67){\rule{0.241pt}{0.400pt}}
\multiput(566.00,436.17)(0.500,1.000){2}{\rule{0.120pt}{0.400pt}}
\put(566.0,435.0){\rule[-0.200pt]{0.400pt}{0.482pt}}
\put(567,438){\usebox{\plotpoint}}
\put(567,438){\usebox{\plotpoint}}
\put(567.0,438.0){\rule[-0.200pt]{0.400pt}{0.482pt}}
\put(567.0,440.0){\usebox{\plotpoint}}
\put(568,441.67){\rule{0.241pt}{0.400pt}}
\multiput(568.00,441.17)(0.500,1.000){2}{\rule{0.120pt}{0.400pt}}
\put(568.0,440.0){\rule[-0.200pt]{0.400pt}{0.482pt}}
\put(569,443){\usebox{\plotpoint}}
\put(569,443){\usebox{\plotpoint}}
\put(569.0,443.0){\rule[-0.200pt]{0.400pt}{0.482pt}}
\put(569.0,445.0){\usebox{\plotpoint}}
\put(570.0,445.0){\rule[-0.200pt]{0.400pt}{0.723pt}}
\put(570.0,448.0){\usebox{\plotpoint}}
\put(571,449.67){\rule{0.241pt}{0.400pt}}
\multiput(571.00,449.17)(0.500,1.000){2}{\rule{0.120pt}{0.400pt}}
\put(571.0,448.0){\rule[-0.200pt]{0.400pt}{0.482pt}}
\put(572,451){\usebox{\plotpoint}}
\put(572,451){\usebox{\plotpoint}}
\put(572.0,451.0){\rule[-0.200pt]{0.400pt}{0.482pt}}
\put(572.0,453.0){\usebox{\plotpoint}}
\put(573,454.67){\rule{0.241pt}{0.400pt}}
\multiput(573.00,454.17)(0.500,1.000){2}{\rule{0.120pt}{0.400pt}}
\put(573.0,453.0){\rule[-0.200pt]{0.400pt}{0.482pt}}
\put(574,456){\usebox{\plotpoint}}
\put(574,456){\usebox{\plotpoint}}
\put(574.0,456.0){\rule[-0.200pt]{0.400pt}{0.482pt}}
\put(574.0,458.0){\usebox{\plotpoint}}
\put(575.0,458.0){\rule[-0.200pt]{0.400pt}{0.723pt}}
\put(575.0,461.0){\usebox{\plotpoint}}
\put(576,462.67){\rule{0.241pt}{0.400pt}}
\multiput(576.00,462.17)(0.500,1.000){2}{\rule{0.120pt}{0.400pt}}
\put(576.0,461.0){\rule[-0.200pt]{0.400pt}{0.482pt}}
\put(577,464){\usebox{\plotpoint}}
\put(577,464){\usebox{\plotpoint}}
\put(577,464){\usebox{\plotpoint}}
\put(577.0,464.0){\rule[-0.200pt]{0.400pt}{0.482pt}}
\put(577.0,466.0){\usebox{\plotpoint}}
\put(578,467.67){\rule{0.241pt}{0.400pt}}
\multiput(578.00,467.17)(0.500,1.000){2}{\rule{0.120pt}{0.400pt}}
\put(578.0,466.0){\rule[-0.200pt]{0.400pt}{0.482pt}}
\put(579,469){\usebox{\plotpoint}}
\put(579,469){\usebox{\plotpoint}}
\put(579.0,469.0){\rule[-0.200pt]{0.400pt}{0.482pt}}
\put(579.0,471.0){\usebox{\plotpoint}}
\put(580.0,471.0){\rule[-0.200pt]{0.400pt}{0.723pt}}
\put(580.0,474.0){\usebox{\plotpoint}}
\put(581,475.67){\rule{0.241pt}{0.400pt}}
\multiput(581.00,475.17)(0.500,1.000){2}{\rule{0.120pt}{0.400pt}}
\put(581.0,474.0){\rule[-0.200pt]{0.400pt}{0.482pt}}
\put(582,477){\usebox{\plotpoint}}
\put(582,477){\usebox{\plotpoint}}
\put(582.0,477.0){\rule[-0.200pt]{0.400pt}{0.482pt}}
\put(582.0,479.0){\usebox{\plotpoint}}
\put(583,480.67){\rule{0.241pt}{0.400pt}}
\multiput(583.00,480.17)(0.500,1.000){2}{\rule{0.120pt}{0.400pt}}
\put(583.0,479.0){\rule[-0.200pt]{0.400pt}{0.482pt}}
\put(584,482){\usebox{\plotpoint}}
\put(584,482){\usebox{\plotpoint}}
\put(584.0,482.0){\rule[-0.200pt]{0.400pt}{0.482pt}}
\put(584.0,484.0){\usebox{\plotpoint}}
\put(585.0,484.0){\rule[-0.200pt]{0.400pt}{0.723pt}}
\put(585.0,487.0){\usebox{\plotpoint}}
\put(586,488.67){\rule{0.241pt}{0.400pt}}
\multiput(586.00,488.17)(0.500,1.000){2}{\rule{0.120pt}{0.400pt}}
\put(586.0,487.0){\rule[-0.200pt]{0.400pt}{0.482pt}}
\put(587,490){\usebox{\plotpoint}}
\put(587,490){\usebox{\plotpoint}}
\put(587.0,490.0){\rule[-0.200pt]{0.400pt}{0.482pt}}
\put(587.0,492.0){\usebox{\plotpoint}}
\put(588.0,492.0){\rule[-0.200pt]{0.400pt}{0.723pt}}
\put(588.0,495.0){\usebox{\plotpoint}}
\put(589,496.67){\rule{0.241pt}{0.400pt}}
\multiput(589.00,496.17)(0.500,1.000){2}{\rule{0.120pt}{0.400pt}}
\put(589.0,495.0){\rule[-0.200pt]{0.400pt}{0.482pt}}
\put(590,498){\usebox{\plotpoint}}
\put(590,498){\usebox{\plotpoint}}
\put(590.0,498.0){\rule[-0.200pt]{0.400pt}{0.482pt}}
\put(590.0,500.0){\usebox{\plotpoint}}
\put(591,501.67){\rule{0.241pt}{0.400pt}}
\multiput(591.00,501.17)(0.500,1.000){2}{\rule{0.120pt}{0.400pt}}
\put(591.0,500.0){\rule[-0.200pt]{0.400pt}{0.482pt}}
\put(592,503){\usebox{\plotpoint}}
\put(592,503){\usebox{\plotpoint}}
\put(592.0,503.0){\rule[-0.200pt]{0.400pt}{0.482pt}}
\put(592.0,505.0){\usebox{\plotpoint}}
\put(593.0,505.0){\rule[-0.200pt]{0.400pt}{0.723pt}}
\put(593.0,508.0){\usebox{\plotpoint}}
\put(594,509.67){\rule{0.241pt}{0.400pt}}
\multiput(594.00,509.17)(0.500,1.000){2}{\rule{0.120pt}{0.400pt}}
\put(594.0,508.0){\rule[-0.200pt]{0.400pt}{0.482pt}}
\put(595,511){\usebox{\plotpoint}}
\put(595,511){\usebox{\plotpoint}}
\put(595.0,511.0){\rule[-0.200pt]{0.400pt}{0.482pt}}
\put(595.0,513.0){\usebox{\plotpoint}}
\put(596.0,513.0){\rule[-0.200pt]{0.400pt}{0.723pt}}
\put(596.0,516.0){\usebox{\plotpoint}}
\put(597,517.67){\rule{0.241pt}{0.400pt}}
\multiput(597.00,517.17)(0.500,1.000){2}{\rule{0.120pt}{0.400pt}}
\put(597.0,516.0){\rule[-0.200pt]{0.400pt}{0.482pt}}
\put(598,519){\usebox{\plotpoint}}
\put(598,519){\usebox{\plotpoint}}
\put(598.0,519.0){\rule[-0.200pt]{0.400pt}{0.482pt}}
\put(598.0,521.0){\usebox{\plotpoint}}
\put(599,522.67){\rule{0.241pt}{0.400pt}}
\multiput(599.00,522.17)(0.500,1.000){2}{\rule{0.120pt}{0.400pt}}
\put(599.0,521.0){\rule[-0.200pt]{0.400pt}{0.482pt}}
\put(600,524){\usebox{\plotpoint}}
\put(600,524){\usebox{\plotpoint}}
\put(600.0,524.0){\rule[-0.200pt]{0.400pt}{0.482pt}}
\put(600.0,526.0){\usebox{\plotpoint}}
\put(601,527.67){\rule{0.241pt}{0.400pt}}
\multiput(601.00,527.17)(0.500,1.000){2}{\rule{0.120pt}{0.400pt}}
\put(601.0,526.0){\rule[-0.200pt]{0.400pt}{0.482pt}}
\put(602,529){\usebox{\plotpoint}}
\put(602,529){\usebox{\plotpoint}}
\put(602.0,529.0){\rule[-0.200pt]{0.400pt}{0.482pt}}
\put(602.0,531.0){\usebox{\plotpoint}}
\put(603.0,531.0){\rule[-0.200pt]{0.400pt}{0.723pt}}
\put(603.0,534.0){\usebox{\plotpoint}}
\put(604,535.67){\rule{0.241pt}{0.400pt}}
\multiput(604.00,535.17)(0.500,1.000){2}{\rule{0.120pt}{0.400pt}}
\put(604.0,534.0){\rule[-0.200pt]{0.400pt}{0.482pt}}
\put(605,537){\usebox{\plotpoint}}
\put(605,537){\usebox{\plotpoint}}
\put(605.0,537.0){\rule[-0.200pt]{0.400pt}{0.482pt}}
\put(605.0,539.0){\usebox{\plotpoint}}
\put(606.0,539.0){\rule[-0.200pt]{0.400pt}{0.723pt}}
\put(606.0,542.0){\usebox{\plotpoint}}
\put(607,543.67){\rule{0.241pt}{0.400pt}}
\multiput(607.00,543.17)(0.500,1.000){2}{\rule{0.120pt}{0.400pt}}
\put(607.0,542.0){\rule[-0.200pt]{0.400pt}{0.482pt}}
\put(608,545){\usebox{\plotpoint}}
\put(608,545){\usebox{\plotpoint}}
\put(608.0,545.0){\rule[-0.200pt]{0.400pt}{0.482pt}}
\put(608.0,547.0){\usebox{\plotpoint}}
\put(609,548.67){\rule{0.241pt}{0.400pt}}
\multiput(609.00,548.17)(0.500,1.000){2}{\rule{0.120pt}{0.400pt}}
\put(609.0,547.0){\rule[-0.200pt]{0.400pt}{0.482pt}}
\put(610,550){\usebox{\plotpoint}}
\put(610,550){\usebox{\plotpoint}}
\put(610.0,550.0){\rule[-0.200pt]{0.400pt}{0.482pt}}
\put(610.0,552.0){\usebox{\plotpoint}}
\put(611.0,552.0){\rule[-0.200pt]{0.400pt}{0.723pt}}
\put(611.0,555.0){\usebox{\plotpoint}}
\put(612.0,555.0){\rule[-0.200pt]{0.400pt}{0.482pt}}
\put(612.0,557.0){\usebox{\plotpoint}}
\put(613.0,557.0){\rule[-0.200pt]{0.400pt}{0.723pt}}
\put(613.0,560.0){\usebox{\plotpoint}}
\put(614,561.67){\rule{0.241pt}{0.400pt}}
\multiput(614.00,561.17)(0.500,1.000){2}{\rule{0.120pt}{0.400pt}}
\put(614.0,560.0){\rule[-0.200pt]{0.400pt}{0.482pt}}
\put(615,563){\usebox{\plotpoint}}
\put(615,563){\usebox{\plotpoint}}
\put(615.0,563.0){\rule[-0.200pt]{0.400pt}{0.482pt}}
\put(615.0,565.0){\usebox{\plotpoint}}
\put(616.0,565.0){\rule[-0.200pt]{0.400pt}{0.723pt}}
\put(616.0,568.0){\usebox{\plotpoint}}
\put(617.0,568.0){\rule[-0.200pt]{0.400pt}{0.482pt}}
\put(617.0,570.0){\usebox{\plotpoint}}
\put(618.0,570.0){\rule[-0.200pt]{0.400pt}{0.723pt}}
\put(618.0,573.0){\usebox{\plotpoint}}
\put(619.0,573.0){\rule[-0.200pt]{0.400pt}{0.482pt}}
\put(619.0,575.0){\usebox{\plotpoint}}
\put(620.0,575.0){\rule[-0.200pt]{0.400pt}{0.723pt}}
\put(620.0,578.0){\usebox{\plotpoint}}
\put(621,579.67){\rule{0.241pt}{0.400pt}}
\multiput(621.00,579.17)(0.500,1.000){2}{\rule{0.120pt}{0.400pt}}
\put(621.0,578.0){\rule[-0.200pt]{0.400pt}{0.482pt}}
\put(622,581){\usebox{\plotpoint}}
\put(622,581){\usebox{\plotpoint}}
\put(622.0,581.0){\rule[-0.200pt]{0.400pt}{0.482pt}}
\put(622.0,583.0){\usebox{\plotpoint}}
\put(623,584.67){\rule{0.241pt}{0.400pt}}
\multiput(623.00,584.17)(0.500,1.000){2}{\rule{0.120pt}{0.400pt}}
\put(623.0,583.0){\rule[-0.200pt]{0.400pt}{0.482pt}}
\put(624,586){\usebox{\plotpoint}}
\put(624,586){\usebox{\plotpoint}}
\put(624.0,586.0){\rule[-0.200pt]{0.400pt}{0.482pt}}
\put(624.0,588.0){\usebox{\plotpoint}}
\put(625.0,588.0){\rule[-0.200pt]{0.400pt}{0.723pt}}
\put(625.0,591.0){\usebox{\plotpoint}}
\put(626.0,591.0){\rule[-0.200pt]{0.400pt}{0.482pt}}
\put(626.0,593.0){\usebox{\plotpoint}}
\put(627,594.67){\rule{0.241pt}{0.400pt}}
\multiput(627.00,594.17)(0.500,1.000){2}{\rule{0.120pt}{0.400pt}}
\put(627.0,593.0){\rule[-0.200pt]{0.400pt}{0.482pt}}
\put(628,596){\usebox{\plotpoint}}
\put(628,596){\usebox{\plotpoint}}
\put(628.0,596.0){\rule[-0.200pt]{0.400pt}{0.482pt}}
\put(628.0,598.0){\usebox{\plotpoint}}
\put(629.0,598.0){\rule[-0.200pt]{0.400pt}{0.723pt}}
\put(629.0,601.0){\usebox{\plotpoint}}
\put(630.0,601.0){\rule[-0.200pt]{0.400pt}{0.482pt}}
\put(630.0,603.0){\usebox{\plotpoint}}
\put(631.0,603.0){\rule[-0.200pt]{0.400pt}{0.723pt}}
\put(631.0,606.0){\usebox{\plotpoint}}
\put(632,607.67){\rule{0.241pt}{0.400pt}}
\multiput(632.00,607.17)(0.500,1.000){2}{\rule{0.120pt}{0.400pt}}
\put(632.0,606.0){\rule[-0.200pt]{0.400pt}{0.482pt}}
\put(633,609){\usebox{\plotpoint}}
\put(633,609){\usebox{\plotpoint}}
\put(633,609){\usebox{\plotpoint}}
\put(633.0,609.0){\rule[-0.200pt]{0.400pt}{0.482pt}}
\put(633.0,611.0){\usebox{\plotpoint}}
\put(634.0,611.0){\rule[-0.200pt]{0.400pt}{0.482pt}}
\put(634.0,613.0){\usebox{\plotpoint}}
\put(635,614.67){\rule{0.241pt}{0.400pt}}
\multiput(635.00,614.17)(0.500,1.000){2}{\rule{0.120pt}{0.400pt}}
\put(635.0,613.0){\rule[-0.200pt]{0.400pt}{0.482pt}}
\put(636,616){\usebox{\plotpoint}}
\put(636,616){\usebox{\plotpoint}}
\put(636.0,616.0){\rule[-0.200pt]{0.400pt}{0.482pt}}
\put(636.0,618.0){\usebox{\plotpoint}}
\put(637.0,618.0){\rule[-0.200pt]{0.400pt}{0.723pt}}
\put(637.0,621.0){\usebox{\plotpoint}}
\put(638.0,621.0){\rule[-0.200pt]{0.400pt}{0.482pt}}
\put(638.0,623.0){\usebox{\plotpoint}}
\put(639.0,623.0){\rule[-0.200pt]{0.400pt}{0.723pt}}
\put(639.0,626.0){\usebox{\plotpoint}}
\put(640.0,626.0){\rule[-0.200pt]{0.400pt}{0.482pt}}
\put(640.0,628.0){\usebox{\plotpoint}}
\put(641,629.67){\rule{0.241pt}{0.400pt}}
\multiput(641.00,629.17)(0.500,1.000){2}{\rule{0.120pt}{0.400pt}}
\put(641.0,628.0){\rule[-0.200pt]{0.400pt}{0.482pt}}
\put(642,631){\usebox{\plotpoint}}
\put(642,631){\usebox{\plotpoint}}
\put(642.0,631.0){\rule[-0.200pt]{0.400pt}{0.482pt}}
\put(642.0,633.0){\usebox{\plotpoint}}
\put(643.0,633.0){\rule[-0.200pt]{0.400pt}{0.482pt}}
\put(643.0,635.0){\usebox{\plotpoint}}
\put(644.0,635.0){\rule[-0.200pt]{0.400pt}{0.723pt}}
\put(644.0,638.0){\usebox{\plotpoint}}
\put(645.0,638.0){\rule[-0.200pt]{0.400pt}{0.482pt}}
\put(645.0,640.0){\usebox{\plotpoint}}
\put(646,641.67){\rule{0.241pt}{0.400pt}}
\multiput(646.00,641.17)(0.500,1.000){2}{\rule{0.120pt}{0.400pt}}
\put(646.0,640.0){\rule[-0.200pt]{0.400pt}{0.482pt}}
\put(647,643){\usebox{\plotpoint}}
\put(647,643){\usebox{\plotpoint}}
\put(647.0,643.0){\rule[-0.200pt]{0.400pt}{0.482pt}}
\put(647.0,645.0){\usebox{\plotpoint}}
\put(648,646.67){\rule{0.241pt}{0.400pt}}
\multiput(648.00,646.17)(0.500,1.000){2}{\rule{0.120pt}{0.400pt}}
\put(648.0,645.0){\rule[-0.200pt]{0.400pt}{0.482pt}}
\put(649,648){\usebox{\plotpoint}}
\put(649,648){\usebox{\plotpoint}}
\put(649.0,648.0){\rule[-0.200pt]{0.400pt}{0.482pt}}
\put(649.0,650.0){\usebox{\plotpoint}}
\put(650,651.67){\rule{0.241pt}{0.400pt}}
\multiput(650.00,651.17)(0.500,1.000){2}{\rule{0.120pt}{0.400pt}}
\put(650.0,650.0){\rule[-0.200pt]{0.400pt}{0.482pt}}
\put(651,653){\usebox{\plotpoint}}
\put(651,653){\usebox{\plotpoint}}
\put(651,653){\usebox{\plotpoint}}
\put(651.0,653.0){\rule[-0.200pt]{0.400pt}{0.482pt}}
\put(651.0,655.0){\usebox{\plotpoint}}
\put(652.0,655.0){\rule[-0.200pt]{0.400pt}{0.482pt}}
\put(652.0,657.0){\usebox{\plotpoint}}
\put(653.0,657.0){\rule[-0.200pt]{0.400pt}{0.482pt}}
\put(653.0,659.0){\usebox{\plotpoint}}
\put(654,660.67){\rule{0.241pt}{0.400pt}}
\multiput(654.00,660.17)(0.500,1.000){2}{\rule{0.120pt}{0.400pt}}
\put(654.0,659.0){\rule[-0.200pt]{0.400pt}{0.482pt}}
\put(655,662){\usebox{\plotpoint}}
\put(655,662){\usebox{\plotpoint}}
\put(655.0,662.0){\rule[-0.200pt]{0.400pt}{0.482pt}}
\put(655.0,664.0){\usebox{\plotpoint}}
\put(656,665.67){\rule{0.241pt}{0.400pt}}
\multiput(656.00,665.17)(0.500,1.000){2}{\rule{0.120pt}{0.400pt}}
\put(656.0,664.0){\rule[-0.200pt]{0.400pt}{0.482pt}}
\put(657,667){\usebox{\plotpoint}}
\put(657,667){\usebox{\plotpoint}}
\put(657,667){\usebox{\plotpoint}}
\put(657.0,667.0){\rule[-0.200pt]{0.400pt}{0.482pt}}
\put(657.0,669.0){\usebox{\plotpoint}}
\put(658.0,669.0){\rule[-0.200pt]{0.400pt}{0.482pt}}
\put(658.0,671.0){\usebox{\plotpoint}}
\put(659,672.67){\rule{0.241pt}{0.400pt}}
\multiput(659.00,672.17)(0.500,1.000){2}{\rule{0.120pt}{0.400pt}}
\put(659.0,671.0){\rule[-0.200pt]{0.400pt}{0.482pt}}
\put(660,674){\usebox{\plotpoint}}
\put(660,674){\usebox{\plotpoint}}
\put(660,674){\usebox{\plotpoint}}
\put(660.0,674.0){\rule[-0.200pt]{0.400pt}{0.482pt}}
\put(660.0,676.0){\usebox{\plotpoint}}
\put(661.0,676.0){\rule[-0.200pt]{0.400pt}{0.482pt}}
\put(661.0,678.0){\usebox{\plotpoint}}
\put(662.0,678.0){\rule[-0.200pt]{0.400pt}{0.482pt}}
\put(662.0,680.0){\usebox{\plotpoint}}
\put(663,681.67){\rule{0.241pt}{0.400pt}}
\multiput(663.00,681.17)(0.500,1.000){2}{\rule{0.120pt}{0.400pt}}
\put(663.0,680.0){\rule[-0.200pt]{0.400pt}{0.482pt}}
\put(664,683){\usebox{\plotpoint}}
\put(664,683){\usebox{\plotpoint}}
\put(664,683){\usebox{\plotpoint}}
\put(664.0,683.0){\rule[-0.200pt]{0.400pt}{0.482pt}}
\put(664.0,685.0){\usebox{\plotpoint}}
\put(665.0,685.0){\rule[-0.200pt]{0.400pt}{0.482pt}}
\put(665.0,687.0){\usebox{\plotpoint}}
\put(666.0,687.0){\rule[-0.200pt]{0.400pt}{0.482pt}}
\put(666.0,689.0){\usebox{\plotpoint}}
\put(667,690.67){\rule{0.241pt}{0.400pt}}
\multiput(667.00,690.17)(0.500,1.000){2}{\rule{0.120pt}{0.400pt}}
\put(667.0,689.0){\rule[-0.200pt]{0.400pt}{0.482pt}}
\put(668,692){\usebox{\plotpoint}}
\put(668,692){\usebox{\plotpoint}}
\put(668,692){\usebox{\plotpoint}}
\put(668.0,692.0){\rule[-0.200pt]{0.400pt}{0.482pt}}
\put(668.0,694.0){\usebox{\plotpoint}}
\put(669.0,694.0){\rule[-0.200pt]{0.400pt}{0.482pt}}
\put(669.0,696.0){\usebox{\plotpoint}}
\put(670.0,696.0){\rule[-0.200pt]{0.400pt}{0.482pt}}
\put(670.0,698.0){\usebox{\plotpoint}}
\put(671.0,698.0){\rule[-0.200pt]{0.400pt}{0.482pt}}
\put(671.0,700.0){\usebox{\plotpoint}}
\put(672.0,700.0){\rule[-0.200pt]{0.400pt}{0.482pt}}
\put(672.0,702.0){\usebox{\plotpoint}}
\put(673,703.67){\rule{0.241pt}{0.400pt}}
\multiput(673.00,703.17)(0.500,1.000){2}{\rule{0.120pt}{0.400pt}}
\put(673.0,702.0){\rule[-0.200pt]{0.400pt}{0.482pt}}
\put(674,705){\usebox{\plotpoint}}
\put(674,705){\usebox{\plotpoint}}
\put(674.0,705.0){\rule[-0.200pt]{0.400pt}{0.482pt}}
\put(674.0,707.0){\usebox{\plotpoint}}
\put(675.0,707.0){\rule[-0.200pt]{0.400pt}{0.482pt}}
\put(675.0,709.0){\usebox{\plotpoint}}
\put(676.0,709.0){\rule[-0.200pt]{0.400pt}{0.482pt}}
\put(676.0,711.0){\usebox{\plotpoint}}
\put(677.0,711.0){\rule[-0.200pt]{0.400pt}{0.482pt}}
\put(677.0,713.0){\usebox{\plotpoint}}
\put(678.0,713.0){\rule[-0.200pt]{0.400pt}{0.482pt}}
\put(678.0,715.0){\usebox{\plotpoint}}
\put(679.0,715.0){\rule[-0.200pt]{0.400pt}{0.482pt}}
\put(679.0,717.0){\usebox{\plotpoint}}
\put(680.0,717.0){\rule[-0.200pt]{0.400pt}{0.482pt}}
\put(680.0,719.0){\usebox{\plotpoint}}
\put(681,720.67){\rule{0.241pt}{0.400pt}}
\multiput(681.00,720.17)(0.500,1.000){2}{\rule{0.120pt}{0.400pt}}
\put(681.0,719.0){\rule[-0.200pt]{0.400pt}{0.482pt}}
\put(682,722){\usebox{\plotpoint}}
\put(682,722){\usebox{\plotpoint}}
\put(682,722){\usebox{\plotpoint}}
\put(682,722.67){\rule{0.241pt}{0.400pt}}
\multiput(682.00,722.17)(0.500,1.000){2}{\rule{0.120pt}{0.400pt}}
\put(682.0,722.0){\usebox{\plotpoint}}
\put(683,724){\usebox{\plotpoint}}
\put(683,724){\usebox{\plotpoint}}
\put(683,724){\usebox{\plotpoint}}
\put(683,724.67){\rule{0.241pt}{0.400pt}}
\multiput(683.00,724.17)(0.500,1.000){2}{\rule{0.120pt}{0.400pt}}
\put(683.0,724.0){\usebox{\plotpoint}}
\put(684,726){\usebox{\plotpoint}}
\put(684,726){\usebox{\plotpoint}}
\put(684.0,726.0){\rule[-0.200pt]{0.400pt}{0.482pt}}
\put(684.0,728.0){\usebox{\plotpoint}}
\put(685.0,728.0){\rule[-0.200pt]{0.400pt}{0.482pt}}
\put(685.0,730.0){\usebox{\plotpoint}}
\put(686.0,730.0){\rule[-0.200pt]{0.400pt}{0.482pt}}
\put(686.0,732.0){\usebox{\plotpoint}}
\put(687,732.67){\rule{0.241pt}{0.400pt}}
\multiput(687.00,732.17)(0.500,1.000){2}{\rule{0.120pt}{0.400pt}}
\put(687.0,732.0){\usebox{\plotpoint}}
\put(688,734){\usebox{\plotpoint}}
\put(688,734){\usebox{\plotpoint}}
\put(688,734){\usebox{\plotpoint}}
\put(688,734.67){\rule{0.241pt}{0.400pt}}
\multiput(688.00,734.17)(0.500,1.000){2}{\rule{0.120pt}{0.400pt}}
\put(688.0,734.0){\usebox{\plotpoint}}
\put(689,736){\usebox{\plotpoint}}
\put(689,736){\usebox{\plotpoint}}
\put(689,736){\usebox{\plotpoint}}
\put(689,736.67){\rule{0.241pt}{0.400pt}}
\multiput(689.00,736.17)(0.500,1.000){2}{\rule{0.120pt}{0.400pt}}
\put(689.0,736.0){\usebox{\plotpoint}}
\put(690,738){\usebox{\plotpoint}}
\put(690,738){\usebox{\plotpoint}}
\put(690,738){\usebox{\plotpoint}}
\put(690,738.67){\rule{0.241pt}{0.400pt}}
\multiput(690.00,738.17)(0.500,1.000){2}{\rule{0.120pt}{0.400pt}}
\put(690.0,738.0){\usebox{\plotpoint}}
\put(691,740){\usebox{\plotpoint}}
\put(691,740){\usebox{\plotpoint}}
\put(691,740){\usebox{\plotpoint}}
\put(691,740.67){\rule{0.241pt}{0.400pt}}
\multiput(691.00,740.17)(0.500,1.000){2}{\rule{0.120pt}{0.400pt}}
\put(691.0,740.0){\usebox{\plotpoint}}
\put(692,742){\usebox{\plotpoint}}
\put(692,742){\usebox{\plotpoint}}
\put(692,742){\usebox{\plotpoint}}
\put(692,742.67){\rule{0.241pt}{0.400pt}}
\multiput(692.00,742.17)(0.500,1.000){2}{\rule{0.120pt}{0.400pt}}
\put(692.0,742.0){\usebox{\plotpoint}}
\put(693,744){\usebox{\plotpoint}}
\put(693,744){\usebox{\plotpoint}}
\put(693,744){\usebox{\plotpoint}}
\put(693,744.67){\rule{0.241pt}{0.400pt}}
\multiput(693.00,744.17)(0.500,1.000){2}{\rule{0.120pt}{0.400pt}}
\put(693.0,744.0){\usebox{\plotpoint}}
\put(694,746){\usebox{\plotpoint}}
\put(694,746){\usebox{\plotpoint}}
\put(694,746){\usebox{\plotpoint}}
\put(694,746.67){\rule{0.241pt}{0.400pt}}
\multiput(694.00,746.17)(0.500,1.000){2}{\rule{0.120pt}{0.400pt}}
\put(694.0,746.0){\usebox{\plotpoint}}
\put(695,748){\usebox{\plotpoint}}
\put(695,748){\usebox{\plotpoint}}
\put(695,748){\usebox{\plotpoint}}
\put(695.0,748.0){\usebox{\plotpoint}}
\put(695.0,749.0){\usebox{\plotpoint}}
\put(696.0,749.0){\rule[-0.200pt]{0.400pt}{0.482pt}}
\put(696.0,751.0){\usebox{\plotpoint}}
\put(697.0,751.0){\rule[-0.200pt]{0.400pt}{0.482pt}}
\put(697.0,753.0){\usebox{\plotpoint}}
\put(698.0,753.0){\rule[-0.200pt]{0.400pt}{0.482pt}}
\put(698.0,755.0){\usebox{\plotpoint}}
\put(699.0,755.0){\rule[-0.200pt]{0.400pt}{0.482pt}}
\put(699.0,757.0){\usebox{\plotpoint}}
\put(700,757.67){\rule{0.241pt}{0.400pt}}
\multiput(700.00,757.17)(0.500,1.000){2}{\rule{0.120pt}{0.400pt}}
\put(700.0,757.0){\usebox{\plotpoint}}
\put(701,759){\usebox{\plotpoint}}
\put(701,759){\usebox{\plotpoint}}
\put(701,759){\usebox{\plotpoint}}
\put(701,759.67){\rule{0.241pt}{0.400pt}}
\multiput(701.00,759.17)(0.500,1.000){2}{\rule{0.120pt}{0.400pt}}
\put(701.0,759.0){\usebox{\plotpoint}}
\put(702,761){\usebox{\plotpoint}}
\put(702,761){\usebox{\plotpoint}}
\put(702,761){\usebox{\plotpoint}}
\put(702,761){\usebox{\plotpoint}}
\put(702.0,761.0){\usebox{\plotpoint}}
\put(702.0,762.0){\usebox{\plotpoint}}
\put(703.0,762.0){\rule[-0.200pt]{0.400pt}{0.482pt}}
\put(703.0,764.0){\usebox{\plotpoint}}
\put(704.0,764.0){\rule[-0.200pt]{0.400pt}{0.482pt}}
\put(704.0,766.0){\usebox{\plotpoint}}
\put(705,766.67){\rule{0.241pt}{0.400pt}}
\multiput(705.00,766.17)(0.500,1.000){2}{\rule{0.120pt}{0.400pt}}
\put(705.0,766.0){\usebox{\plotpoint}}
\put(706,768){\usebox{\plotpoint}}
\put(706,768){\usebox{\plotpoint}}
\put(706,768){\usebox{\plotpoint}}
\put(706,768){\usebox{\plotpoint}}
\put(706.0,768.0){\usebox{\plotpoint}}
\put(706.0,769.0){\usebox{\plotpoint}}
\put(707.0,769.0){\rule[-0.200pt]{0.400pt}{0.482pt}}
\put(707.0,771.0){\usebox{\plotpoint}}
\put(708.0,771.0){\rule[-0.200pt]{0.400pt}{0.482pt}}
\put(708.0,773.0){\usebox{\plotpoint}}
\put(709,773.67){\rule{0.241pt}{0.400pt}}
\multiput(709.00,773.17)(0.500,1.000){2}{\rule{0.120pt}{0.400pt}}
\put(709.0,773.0){\usebox{\plotpoint}}
\put(710,775){\usebox{\plotpoint}}
\put(710,775){\usebox{\plotpoint}}
\put(710,775){\usebox{\plotpoint}}
\put(710,775){\usebox{\plotpoint}}
\put(710.0,775.0){\usebox{\plotpoint}}
\put(710.0,776.0){\usebox{\plotpoint}}
\put(711.0,776.0){\rule[-0.200pt]{0.400pt}{0.482pt}}
\put(711.0,778.0){\usebox{\plotpoint}}
\put(712,778.67){\rule{0.241pt}{0.400pt}}
\multiput(712.00,778.17)(0.500,1.000){2}{\rule{0.120pt}{0.400pt}}
\put(712.0,778.0){\usebox{\plotpoint}}
\put(713,780){\usebox{\plotpoint}}
\put(713,780){\usebox{\plotpoint}}
\put(713,780){\usebox{\plotpoint}}
\put(713,780){\usebox{\plotpoint}}
\put(713.0,780.0){\usebox{\plotpoint}}
\put(713.0,781.0){\usebox{\plotpoint}}
\put(714.0,781.0){\rule[-0.200pt]{0.400pt}{0.482pt}}
\put(714.0,783.0){\usebox{\plotpoint}}
\put(715.0,783.0){\usebox{\plotpoint}}
\put(715.0,784.0){\usebox{\plotpoint}}
\put(716.0,784.0){\rule[-0.200pt]{0.400pt}{0.482pt}}
\put(716.0,786.0){\usebox{\plotpoint}}
\put(717,786.67){\rule{0.241pt}{0.400pt}}
\multiput(717.00,786.17)(0.500,1.000){2}{\rule{0.120pt}{0.400pt}}
\put(717.0,786.0){\usebox{\plotpoint}}
\put(718,788){\usebox{\plotpoint}}
\put(718,788){\usebox{\plotpoint}}
\put(718,788){\usebox{\plotpoint}}
\put(718,788){\usebox{\plotpoint}}
\put(718.0,788.0){\usebox{\plotpoint}}
\put(718.0,789.0){\usebox{\plotpoint}}
\put(719.0,789.0){\rule[-0.200pt]{0.400pt}{0.482pt}}
\put(719.0,791.0){\usebox{\plotpoint}}
\put(720.0,791.0){\usebox{\plotpoint}}
\put(720.0,792.0){\usebox{\plotpoint}}
\put(721.0,792.0){\rule[-0.200pt]{0.400pt}{0.482pt}}
\put(721.0,794.0){\usebox{\plotpoint}}
\put(722.0,794.0){\usebox{\plotpoint}}
\put(722.0,795.0){\usebox{\plotpoint}}
\put(723.0,795.0){\rule[-0.200pt]{0.400pt}{0.482pt}}
\put(723.0,797.0){\usebox{\plotpoint}}
\put(724.0,797.0){\usebox{\plotpoint}}
\put(724.0,798.0){\usebox{\plotpoint}}
\put(725,798.67){\rule{0.241pt}{0.400pt}}
\multiput(725.00,798.17)(0.500,1.000){2}{\rule{0.120pt}{0.400pt}}
\put(725.0,798.0){\usebox{\plotpoint}}
\put(726,800){\usebox{\plotpoint}}
\put(726,800){\usebox{\plotpoint}}
\put(726,800){\usebox{\plotpoint}}
\put(726,800){\usebox{\plotpoint}}
\put(726.0,800.0){\usebox{\plotpoint}}
\put(726.0,801.0){\usebox{\plotpoint}}
\put(727,801.67){\rule{0.241pt}{0.400pt}}
\multiput(727.00,801.17)(0.500,1.000){2}{\rule{0.120pt}{0.400pt}}
\put(727.0,801.0){\usebox{\plotpoint}}
\put(728,803){\usebox{\plotpoint}}
\put(728,803){\usebox{\plotpoint}}
\put(728,803){\usebox{\plotpoint}}
\put(728,803){\usebox{\plotpoint}}
\put(728,803){\usebox{\plotpoint}}
\put(728.0,803.0){\usebox{\plotpoint}}
\put(728.0,804.0){\usebox{\plotpoint}}
\put(729.0,804.0){\usebox{\plotpoint}}
\put(729.0,805.0){\usebox{\plotpoint}}
\put(730.0,805.0){\rule[-0.200pt]{0.400pt}{0.482pt}}
\put(730.0,807.0){\usebox{\plotpoint}}
\put(731.0,807.0){\usebox{\plotpoint}}
\put(731.0,808.0){\usebox{\plotpoint}}
\put(732.0,808.0){\usebox{\plotpoint}}
\put(732.0,809.0){\usebox{\plotpoint}}
\put(733.0,809.0){\rule[-0.200pt]{0.400pt}{0.482pt}}
\put(733.0,811.0){\usebox{\plotpoint}}
\put(734.0,811.0){\usebox{\plotpoint}}
\put(734.0,812.0){\usebox{\plotpoint}}
\put(735.0,812.0){\usebox{\plotpoint}}
\put(735.0,813.0){\usebox{\plotpoint}}
\put(736.0,813.0){\rule[-0.200pt]{0.400pt}{0.482pt}}
\put(736.0,815.0){\usebox{\plotpoint}}
\put(737.0,815.0){\usebox{\plotpoint}}
\put(737.0,816.0){\usebox{\plotpoint}}
\put(738.0,816.0){\usebox{\plotpoint}}
\put(738.0,817.0){\usebox{\plotpoint}}
\put(739.0,817.0){\usebox{\plotpoint}}
\put(739.0,818.0){\usebox{\plotpoint}}
\put(740,818.67){\rule{0.241pt}{0.400pt}}
\multiput(740.00,818.17)(0.500,1.000){2}{\rule{0.120pt}{0.400pt}}
\put(740.0,818.0){\usebox{\plotpoint}}
\put(741,820){\usebox{\plotpoint}}
\put(741,820){\usebox{\plotpoint}}
\put(741,820){\usebox{\plotpoint}}
\put(741,820){\usebox{\plotpoint}}
\put(741,820){\usebox{\plotpoint}}
\put(741,820){\usebox{\plotpoint}}
\put(741.0,820.0){\usebox{\plotpoint}}
\put(741.0,821.0){\usebox{\plotpoint}}
\put(742.0,821.0){\usebox{\plotpoint}}
\put(742.0,822.0){\usebox{\plotpoint}}
\put(743.0,822.0){\usebox{\plotpoint}}
\put(743.0,823.0){\usebox{\plotpoint}}
\put(744.0,823.0){\usebox{\plotpoint}}
\put(744.0,824.0){\usebox{\plotpoint}}
\put(745.0,824.0){\usebox{\plotpoint}}
\put(745.0,825.0){\usebox{\plotpoint}}
\put(746,825.67){\rule{0.241pt}{0.400pt}}
\multiput(746.00,825.17)(0.500,1.000){2}{\rule{0.120pt}{0.400pt}}
\put(746.0,825.0){\usebox{\plotpoint}}
\put(747,827){\usebox{\plotpoint}}
\put(747,827){\usebox{\plotpoint}}
\put(747,827){\usebox{\plotpoint}}
\put(747,827){\usebox{\plotpoint}}
\put(747,827){\usebox{\plotpoint}}
\put(747,827){\usebox{\plotpoint}}
\put(747,827){\usebox{\plotpoint}}
\put(747,826.67){\rule{0.241pt}{0.400pt}}
\multiput(747.00,826.17)(0.500,1.000){2}{\rule{0.120pt}{0.400pt}}
\put(748,828){\usebox{\plotpoint}}
\put(748,828){\usebox{\plotpoint}}
\put(748,828){\usebox{\plotpoint}}
\put(748,828){\usebox{\plotpoint}}
\put(748,828){\usebox{\plotpoint}}
\put(748,828){\usebox{\plotpoint}}
\put(748,827.67){\rule{0.241pt}{0.400pt}}
\multiput(748.00,827.17)(0.500,1.000){2}{\rule{0.120pt}{0.400pt}}
\put(749,829){\usebox{\plotpoint}}
\put(749,829){\usebox{\plotpoint}}
\put(749,829){\usebox{\plotpoint}}
\put(749,829){\usebox{\plotpoint}}
\put(749,829){\usebox{\plotpoint}}
\put(749,829){\usebox{\plotpoint}}
\put(749.0,829.0){\usebox{\plotpoint}}
\put(749.0,830.0){\usebox{\plotpoint}}
\put(750.0,830.0){\usebox{\plotpoint}}
\put(750.0,831.0){\usebox{\plotpoint}}
\put(751.0,831.0){\usebox{\plotpoint}}
\put(751.0,832.0){\usebox{\plotpoint}}
\put(752.0,832.0){\usebox{\plotpoint}}
\put(752.0,833.0){\usebox{\plotpoint}}
\put(753.0,833.0){\usebox{\plotpoint}}
\put(753.0,834.0){\usebox{\plotpoint}}
\put(754.0,834.0){\usebox{\plotpoint}}
\put(754.0,835.0){\usebox{\plotpoint}}
\put(755.0,835.0){\usebox{\plotpoint}}
\put(756,835.67){\rule{0.241pt}{0.400pt}}
\multiput(756.00,835.17)(0.500,1.000){2}{\rule{0.120pt}{0.400pt}}
\put(755.0,836.0){\usebox{\plotpoint}}
\put(757,837){\usebox{\plotpoint}}
\put(757,837){\usebox{\plotpoint}}
\put(757,837){\usebox{\plotpoint}}
\put(757,837){\usebox{\plotpoint}}
\put(757,837){\usebox{\plotpoint}}
\put(757,837){\usebox{\plotpoint}}
\put(757.0,837.0){\usebox{\plotpoint}}
\put(758.0,837.0){\usebox{\plotpoint}}
\put(758.0,838.0){\usebox{\plotpoint}}
\put(759.0,838.0){\usebox{\plotpoint}}
\put(759.0,839.0){\usebox{\plotpoint}}
\put(760.0,839.0){\usebox{\plotpoint}}
\put(760.0,840.0){\usebox{\plotpoint}}
\put(761.0,840.0){\usebox{\plotpoint}}
\put(761.0,841.0){\usebox{\plotpoint}}
\put(762.0,841.0){\usebox{\plotpoint}}
\put(763,841.67){\rule{0.241pt}{0.400pt}}
\multiput(763.00,841.17)(0.500,1.000){2}{\rule{0.120pt}{0.400pt}}
\put(762.0,842.0){\usebox{\plotpoint}}
\put(764,843){\usebox{\plotpoint}}
\put(764,843){\usebox{\plotpoint}}
\put(764,843){\usebox{\plotpoint}}
\put(764,843){\usebox{\plotpoint}}
\put(764,843){\usebox{\plotpoint}}
\put(764,843){\usebox{\plotpoint}}
\put(764,843){\usebox{\plotpoint}}
\put(764.0,843.0){\usebox{\plotpoint}}
\put(765.0,843.0){\usebox{\plotpoint}}
\put(765.0,844.0){\usebox{\plotpoint}}
\put(766.0,844.0){\usebox{\plotpoint}}
\put(766.0,845.0){\rule[-0.200pt]{0.482pt}{0.400pt}}
\put(768.0,845.0){\usebox{\plotpoint}}
\put(768.0,846.0){\usebox{\plotpoint}}
\put(769.0,846.0){\usebox{\plotpoint}}
\put(769.0,847.0){\usebox{\plotpoint}}
\put(770.0,847.0){\usebox{\plotpoint}}
\put(770.0,848.0){\rule[-0.200pt]{0.482pt}{0.400pt}}
\put(772.0,848.0){\usebox{\plotpoint}}
\put(773,848.67){\rule{0.241pt}{0.400pt}}
\multiput(773.00,848.17)(0.500,1.000){2}{\rule{0.120pt}{0.400pt}}
\put(772.0,849.0){\usebox{\plotpoint}}
\put(774,850){\usebox{\plotpoint}}
\put(774,850){\usebox{\plotpoint}}
\put(774,850){\usebox{\plotpoint}}
\put(774,850){\usebox{\plotpoint}}
\put(774,850){\usebox{\plotpoint}}
\put(774,850){\usebox{\plotpoint}}
\put(774.0,850.0){\usebox{\plotpoint}}
\put(775.0,850.0){\usebox{\plotpoint}}
\put(775.0,851.0){\rule[-0.200pt]{0.482pt}{0.400pt}}
\put(777.0,851.0){\usebox{\plotpoint}}
\put(777.0,852.0){\rule[-0.200pt]{0.482pt}{0.400pt}}
\put(779.0,852.0){\usebox{\plotpoint}}
\put(779.0,853.0){\rule[-0.200pt]{0.482pt}{0.400pt}}
\put(781.0,853.0){\usebox{\plotpoint}}
\put(783,853.67){\rule{0.241pt}{0.400pt}}
\multiput(783.00,853.17)(0.500,1.000){2}{\rule{0.120pt}{0.400pt}}
\put(781.0,854.0){\rule[-0.200pt]{0.482pt}{0.400pt}}
\put(784,855){\usebox{\plotpoint}}
\put(784,855){\usebox{\plotpoint}}
\put(784,855){\usebox{\plotpoint}}
\put(784,855){\usebox{\plotpoint}}
\put(784,855){\usebox{\plotpoint}}
\put(784,855){\usebox{\plotpoint}}
\put(784,855){\usebox{\plotpoint}}
\put(784.0,855.0){\rule[-0.200pt]{0.482pt}{0.400pt}}
\put(786.0,855.0){\usebox{\plotpoint}}
\put(786.0,856.0){\rule[-0.200pt]{0.723pt}{0.400pt}}
\put(789.0,856.0){\usebox{\plotpoint}}
\put(789.0,857.0){\rule[-0.200pt]{0.964pt}{0.400pt}}
\put(793.0,857.0){\usebox{\plotpoint}}
\put(793.0,858.0){\rule[-0.200pt]{1.204pt}{0.400pt}}
\put(798.0,858.0){\usebox{\plotpoint}}
\put(798.0,859.0){\rule[-0.200pt]{3.373pt}{0.400pt}}
\put(812.0,858.0){\usebox{\plotpoint}}
\put(812.0,858.0){\rule[-0.200pt]{1.204pt}{0.400pt}}
\put(817.0,857.0){\usebox{\plotpoint}}
\put(817.0,857.0){\rule[-0.200pt]{0.964pt}{0.400pt}}
\put(821.0,856.0){\usebox{\plotpoint}}
\put(821.0,856.0){\rule[-0.200pt]{0.723pt}{0.400pt}}
\put(824.0,855.0){\usebox{\plotpoint}}
\put(826,853.67){\rule{0.241pt}{0.400pt}}
\multiput(826.00,854.17)(0.500,-1.000){2}{\rule{0.120pt}{0.400pt}}
\put(824.0,855.0){\rule[-0.200pt]{0.482pt}{0.400pt}}
\put(827,854){\usebox{\plotpoint}}
\put(827,854){\usebox{\plotpoint}}
\put(827,854){\usebox{\plotpoint}}
\put(827,854){\usebox{\plotpoint}}
\put(827,854){\usebox{\plotpoint}}
\put(827,854){\usebox{\plotpoint}}
\put(827.0,854.0){\rule[-0.200pt]{0.482pt}{0.400pt}}
\put(829.0,853.0){\usebox{\plotpoint}}
\put(829.0,853.0){\rule[-0.200pt]{0.482pt}{0.400pt}}
\put(831.0,852.0){\usebox{\plotpoint}}
\put(831.0,852.0){\rule[-0.200pt]{0.482pt}{0.400pt}}
\put(833.0,851.0){\usebox{\plotpoint}}
\put(833.0,851.0){\rule[-0.200pt]{0.482pt}{0.400pt}}
\put(835.0,850.0){\usebox{\plotpoint}}
\put(836,848.67){\rule{0.241pt}{0.400pt}}
\multiput(836.00,849.17)(0.500,-1.000){2}{\rule{0.120pt}{0.400pt}}
\put(835.0,850.0){\usebox{\plotpoint}}
\put(837,849){\usebox{\plotpoint}}
\put(837,849){\usebox{\plotpoint}}
\put(837,849){\usebox{\plotpoint}}
\put(837,849){\usebox{\plotpoint}}
\put(837,849){\usebox{\plotpoint}}
\put(837,849){\usebox{\plotpoint}}
\put(837,849){\usebox{\plotpoint}}
\put(837.0,849.0){\usebox{\plotpoint}}
\put(838.0,848.0){\usebox{\plotpoint}}
\put(838.0,848.0){\rule[-0.200pt]{0.482pt}{0.400pt}}
\put(840.0,847.0){\usebox{\plotpoint}}
\put(840.0,847.0){\usebox{\plotpoint}}
\put(841.0,846.0){\usebox{\plotpoint}}
\put(841.0,846.0){\usebox{\plotpoint}}
\put(842.0,845.0){\usebox{\plotpoint}}
\put(842.0,845.0){\rule[-0.200pt]{0.482pt}{0.400pt}}
\put(844.0,844.0){\usebox{\plotpoint}}
\put(844.0,844.0){\usebox{\plotpoint}}
\put(845.0,843.0){\usebox{\plotpoint}}
\put(846,841.67){\rule{0.241pt}{0.400pt}}
\multiput(846.00,842.17)(0.500,-1.000){2}{\rule{0.120pt}{0.400pt}}
\put(845.0,843.0){\usebox{\plotpoint}}
\put(847,842){\usebox{\plotpoint}}
\put(847,842){\usebox{\plotpoint}}
\put(847,842){\usebox{\plotpoint}}
\put(847,842){\usebox{\plotpoint}}
\put(847,842){\usebox{\plotpoint}}
\put(847,842){\usebox{\plotpoint}}
\put(847,842){\usebox{\plotpoint}}
\put(847.0,842.0){\usebox{\plotpoint}}
\put(848.0,841.0){\usebox{\plotpoint}}
\put(848.0,841.0){\usebox{\plotpoint}}
\put(849.0,840.0){\usebox{\plotpoint}}
\put(849.0,840.0){\usebox{\plotpoint}}
\put(850.0,839.0){\usebox{\plotpoint}}
\put(850.0,839.0){\usebox{\plotpoint}}
\put(851.0,838.0){\usebox{\plotpoint}}
\put(851.0,838.0){\usebox{\plotpoint}}
\put(852.0,837.0){\usebox{\plotpoint}}
\put(853,835.67){\rule{0.241pt}{0.400pt}}
\multiput(853.00,836.17)(0.500,-1.000){2}{\rule{0.120pt}{0.400pt}}
\put(852.0,837.0){\usebox{\plotpoint}}
\put(854,836){\usebox{\plotpoint}}
\put(854,836){\usebox{\plotpoint}}
\put(854,836){\usebox{\plotpoint}}
\put(854,836){\usebox{\plotpoint}}
\put(854,836){\usebox{\plotpoint}}
\put(854,836){\usebox{\plotpoint}}
\put(854,836){\usebox{\plotpoint}}
\put(854.0,836.0){\usebox{\plotpoint}}
\put(855.0,835.0){\usebox{\plotpoint}}
\put(855.0,835.0){\usebox{\plotpoint}}
\put(856.0,834.0){\usebox{\plotpoint}}
\put(856.0,834.0){\usebox{\plotpoint}}
\put(857.0,833.0){\usebox{\plotpoint}}
\put(857.0,833.0){\usebox{\plotpoint}}
\put(858.0,832.0){\usebox{\plotpoint}}
\put(858.0,832.0){\usebox{\plotpoint}}
\put(859.0,831.0){\usebox{\plotpoint}}
\put(859.0,831.0){\usebox{\plotpoint}}
\put(860.0,830.0){\usebox{\plotpoint}}
\put(860.0,830.0){\usebox{\plotpoint}}
\put(861,827.67){\rule{0.241pt}{0.400pt}}
\multiput(861.00,828.17)(0.500,-1.000){2}{\rule{0.120pt}{0.400pt}}
\put(861.0,829.0){\usebox{\plotpoint}}
\put(862,828){\usebox{\plotpoint}}
\put(862,828){\usebox{\plotpoint}}
\put(862,828){\usebox{\plotpoint}}
\put(862,828){\usebox{\plotpoint}}
\put(862,828){\usebox{\plotpoint}}
\put(862,828){\usebox{\plotpoint}}
\put(862,826.67){\rule{0.241pt}{0.400pt}}
\multiput(862.00,827.17)(0.500,-1.000){2}{\rule{0.120pt}{0.400pt}}
\put(863,827){\usebox{\plotpoint}}
\put(863,827){\usebox{\plotpoint}}
\put(863,827){\usebox{\plotpoint}}
\put(863,827){\usebox{\plotpoint}}
\put(863,827){\usebox{\plotpoint}}
\put(863,827){\usebox{\plotpoint}}
\put(863,827){\usebox{\plotpoint}}
\put(863,825.67){\rule{0.241pt}{0.400pt}}
\multiput(863.00,826.17)(0.500,-1.000){2}{\rule{0.120pt}{0.400pt}}
\put(864,826){\usebox{\plotpoint}}
\put(864,826){\usebox{\plotpoint}}
\put(864,826){\usebox{\plotpoint}}
\put(864,826){\usebox{\plotpoint}}
\put(864,826){\usebox{\plotpoint}}
\put(864,826){\usebox{\plotpoint}}
\put(864.0,825.0){\usebox{\plotpoint}}
\put(864.0,825.0){\usebox{\plotpoint}}
\put(865.0,824.0){\usebox{\plotpoint}}
\put(865.0,824.0){\usebox{\plotpoint}}
\put(866.0,823.0){\usebox{\plotpoint}}
\put(866.0,823.0){\usebox{\plotpoint}}
\put(867.0,822.0){\usebox{\plotpoint}}
\put(867.0,822.0){\usebox{\plotpoint}}
\put(868.0,821.0){\usebox{\plotpoint}}
\put(868.0,821.0){\usebox{\plotpoint}}
\put(869,818.67){\rule{0.241pt}{0.400pt}}
\multiput(869.00,819.17)(0.500,-1.000){2}{\rule{0.120pt}{0.400pt}}
\put(869.0,820.0){\usebox{\plotpoint}}
\put(870,819){\usebox{\plotpoint}}
\put(870,819){\usebox{\plotpoint}}
\put(870,819){\usebox{\plotpoint}}
\put(870,819){\usebox{\plotpoint}}
\put(870,819){\usebox{\plotpoint}}
\put(870.0,818.0){\usebox{\plotpoint}}
\put(870.0,818.0){\usebox{\plotpoint}}
\put(871.0,817.0){\usebox{\plotpoint}}
\put(871.0,817.0){\usebox{\plotpoint}}
\put(872.0,816.0){\usebox{\plotpoint}}
\put(872.0,816.0){\usebox{\plotpoint}}
\put(873.0,815.0){\usebox{\plotpoint}}
\put(873.0,815.0){\usebox{\plotpoint}}
\put(874.0,813.0){\rule[-0.200pt]{0.400pt}{0.482pt}}
\put(874.0,813.0){\usebox{\plotpoint}}
\put(875.0,812.0){\usebox{\plotpoint}}
\put(875.0,812.0){\usebox{\plotpoint}}
\put(876.0,811.0){\usebox{\plotpoint}}
\put(876.0,811.0){\usebox{\plotpoint}}
\put(877.0,809.0){\rule[-0.200pt]{0.400pt}{0.482pt}}
\put(877.0,809.0){\usebox{\plotpoint}}
\put(878.0,808.0){\usebox{\plotpoint}}
\put(878.0,808.0){\usebox{\plotpoint}}
\put(879.0,807.0){\usebox{\plotpoint}}
\put(879.0,807.0){\usebox{\plotpoint}}
\put(880.0,805.0){\rule[-0.200pt]{0.400pt}{0.482pt}}
\put(880.0,805.0){\usebox{\plotpoint}}
\put(881.0,804.0){\usebox{\plotpoint}}
\put(881.0,804.0){\usebox{\plotpoint}}
\put(882,801.67){\rule{0.241pt}{0.400pt}}
\multiput(882.00,802.17)(0.500,-1.000){2}{\rule{0.120pt}{0.400pt}}
\put(882.0,803.0){\usebox{\plotpoint}}
\put(883,802){\usebox{\plotpoint}}
\put(883,802){\usebox{\plotpoint}}
\put(883,802){\usebox{\plotpoint}}
\put(883,802){\usebox{\plotpoint}}
\put(883,802){\usebox{\plotpoint}}
\put(883.0,801.0){\usebox{\plotpoint}}
\put(883.0,801.0){\usebox{\plotpoint}}
\put(884,798.67){\rule{0.241pt}{0.400pt}}
\multiput(884.00,799.17)(0.500,-1.000){2}{\rule{0.120pt}{0.400pt}}
\put(884.0,800.0){\usebox{\plotpoint}}
\put(885,799){\usebox{\plotpoint}}
\put(885,799){\usebox{\plotpoint}}
\put(885,799){\usebox{\plotpoint}}
\put(885,799){\usebox{\plotpoint}}
\put(885,799){\usebox{\plotpoint}}
\put(885.0,798.0){\usebox{\plotpoint}}
\put(885.0,798.0){\usebox{\plotpoint}}
\put(886.0,797.0){\usebox{\plotpoint}}
\put(886.0,797.0){\usebox{\plotpoint}}
\put(887.0,795.0){\rule[-0.200pt]{0.400pt}{0.482pt}}
\put(887.0,795.0){\usebox{\plotpoint}}
\put(888.0,794.0){\usebox{\plotpoint}}
\put(888.0,794.0){\usebox{\plotpoint}}
\put(889.0,792.0){\rule[-0.200pt]{0.400pt}{0.482pt}}
\put(889.0,792.0){\usebox{\plotpoint}}
\put(890.0,791.0){\usebox{\plotpoint}}
\put(890.0,791.0){\usebox{\plotpoint}}
\put(891.0,789.0){\rule[-0.200pt]{0.400pt}{0.482pt}}
\put(891.0,789.0){\usebox{\plotpoint}}
\put(892,786.67){\rule{0.241pt}{0.400pt}}
\multiput(892.00,787.17)(0.500,-1.000){2}{\rule{0.120pt}{0.400pt}}
\put(892.0,788.0){\usebox{\plotpoint}}
\put(893,787){\usebox{\plotpoint}}
\put(893,787){\usebox{\plotpoint}}
\put(893,787){\usebox{\plotpoint}}
\put(893,787){\usebox{\plotpoint}}
\put(893.0,786.0){\usebox{\plotpoint}}
\put(893.0,786.0){\usebox{\plotpoint}}
\put(894.0,784.0){\rule[-0.200pt]{0.400pt}{0.482pt}}
\put(894.0,784.0){\usebox{\plotpoint}}
\put(895.0,783.0){\usebox{\plotpoint}}
\put(895.0,783.0){\usebox{\plotpoint}}
\put(896.0,781.0){\rule[-0.200pt]{0.400pt}{0.482pt}}
\put(896.0,781.0){\usebox{\plotpoint}}
\put(897,778.67){\rule{0.241pt}{0.400pt}}
\multiput(897.00,779.17)(0.500,-1.000){2}{\rule{0.120pt}{0.400pt}}
\put(897.0,780.0){\usebox{\plotpoint}}
\put(898,779){\usebox{\plotpoint}}
\put(898,779){\usebox{\plotpoint}}
\put(898,779){\usebox{\plotpoint}}
\put(898,779){\usebox{\plotpoint}}
\put(898.0,778.0){\usebox{\plotpoint}}
\put(898.0,778.0){\usebox{\plotpoint}}
\put(899.0,776.0){\rule[-0.200pt]{0.400pt}{0.482pt}}
\put(899.0,776.0){\usebox{\plotpoint}}
\put(900,773.67){\rule{0.241pt}{0.400pt}}
\multiput(900.00,774.17)(0.500,-1.000){2}{\rule{0.120pt}{0.400pt}}
\put(900.0,775.0){\usebox{\plotpoint}}
\put(901,774){\usebox{\plotpoint}}
\put(901,774){\usebox{\plotpoint}}
\put(901,774){\usebox{\plotpoint}}
\put(901.0,773.0){\usebox{\plotpoint}}
\put(901.0,773.0){\usebox{\plotpoint}}
\put(902.0,771.0){\rule[-0.200pt]{0.400pt}{0.482pt}}
\put(902.0,771.0){\usebox{\plotpoint}}
\put(903.0,769.0){\rule[-0.200pt]{0.400pt}{0.482pt}}
\put(903.0,769.0){\usebox{\plotpoint}}
\put(904,766.67){\rule{0.241pt}{0.400pt}}
\multiput(904.00,767.17)(0.500,-1.000){2}{\rule{0.120pt}{0.400pt}}
\put(904.0,768.0){\usebox{\plotpoint}}
\put(905,767){\usebox{\plotpoint}}
\put(905,767){\usebox{\plotpoint}}
\put(905,767){\usebox{\plotpoint}}
\put(905.0,766.0){\usebox{\plotpoint}}
\put(905.0,766.0){\usebox{\plotpoint}}
\put(906.0,764.0){\rule[-0.200pt]{0.400pt}{0.482pt}}
\put(906.0,764.0){\usebox{\plotpoint}}
\put(907.0,762.0){\rule[-0.200pt]{0.400pt}{0.482pt}}
\put(907.0,762.0){\usebox{\plotpoint}}
\put(908,759.67){\rule{0.241pt}{0.400pt}}
\multiput(908.00,760.17)(0.500,-1.000){2}{\rule{0.120pt}{0.400pt}}
\put(908.0,761.0){\usebox{\plotpoint}}
\put(909,760){\usebox{\plotpoint}}
\put(909,760){\usebox{\plotpoint}}
\put(909,760){\usebox{\plotpoint}}
\put(909,757.67){\rule{0.241pt}{0.400pt}}
\multiput(909.00,758.17)(0.500,-1.000){2}{\rule{0.120pt}{0.400pt}}
\put(909.0,759.0){\usebox{\plotpoint}}
\put(910,758){\usebox{\plotpoint}}
\put(910,758){\usebox{\plotpoint}}
\put(910,758){\usebox{\plotpoint}}
\put(910,758){\usebox{\plotpoint}}
\put(910.0,757.0){\usebox{\plotpoint}}
\put(910.0,757.0){\usebox{\plotpoint}}
\put(911.0,755.0){\rule[-0.200pt]{0.400pt}{0.482pt}}
\put(911.0,755.0){\usebox{\plotpoint}}
\put(912.0,753.0){\rule[-0.200pt]{0.400pt}{0.482pt}}
\put(912.0,753.0){\usebox{\plotpoint}}
\put(913.0,751.0){\rule[-0.200pt]{0.400pt}{0.482pt}}
\put(913.0,751.0){\usebox{\plotpoint}}
\put(914.0,749.0){\rule[-0.200pt]{0.400pt}{0.482pt}}
\put(914.0,749.0){\usebox{\plotpoint}}
\put(915,746.67){\rule{0.241pt}{0.400pt}}
\multiput(915.00,747.17)(0.500,-1.000){2}{\rule{0.120pt}{0.400pt}}
\put(915.0,748.0){\usebox{\plotpoint}}
\put(916,747){\usebox{\plotpoint}}
\put(916,747){\usebox{\plotpoint}}
\put(916,747){\usebox{\plotpoint}}
\put(916,744.67){\rule{0.241pt}{0.400pt}}
\multiput(916.00,745.17)(0.500,-1.000){2}{\rule{0.120pt}{0.400pt}}
\put(916.0,746.0){\usebox{\plotpoint}}
\put(917,745){\usebox{\plotpoint}}
\put(917,745){\usebox{\plotpoint}}
\put(917,745){\usebox{\plotpoint}}
\put(917,742.67){\rule{0.241pt}{0.400pt}}
\multiput(917.00,743.17)(0.500,-1.000){2}{\rule{0.120pt}{0.400pt}}
\put(917.0,744.0){\usebox{\plotpoint}}
\put(918,743){\usebox{\plotpoint}}
\put(918,743){\usebox{\plotpoint}}
\put(918,743){\usebox{\plotpoint}}
\put(918,740.67){\rule{0.241pt}{0.400pt}}
\multiput(918.00,741.17)(0.500,-1.000){2}{\rule{0.120pt}{0.400pt}}
\put(918.0,742.0){\usebox{\plotpoint}}
\put(919,741){\usebox{\plotpoint}}
\put(919,741){\usebox{\plotpoint}}
\put(919,741){\usebox{\plotpoint}}
\put(919,738.67){\rule{0.241pt}{0.400pt}}
\multiput(919.00,739.17)(0.500,-1.000){2}{\rule{0.120pt}{0.400pt}}
\put(919.0,740.0){\usebox{\plotpoint}}
\put(920,739){\usebox{\plotpoint}}
\put(920,739){\usebox{\plotpoint}}
\put(920,739){\usebox{\plotpoint}}
\put(920,736.67){\rule{0.241pt}{0.400pt}}
\multiput(920.00,737.17)(0.500,-1.000){2}{\rule{0.120pt}{0.400pt}}
\put(920.0,738.0){\usebox{\plotpoint}}
\put(921,737){\usebox{\plotpoint}}
\put(921,737){\usebox{\plotpoint}}
\put(921,737){\usebox{\plotpoint}}
\put(921,734.67){\rule{0.241pt}{0.400pt}}
\multiput(921.00,735.17)(0.500,-1.000){2}{\rule{0.120pt}{0.400pt}}
\put(921.0,736.0){\usebox{\plotpoint}}
\put(922,735){\usebox{\plotpoint}}
\put(922,735){\usebox{\plotpoint}}
\put(922,735){\usebox{\plotpoint}}
\put(922,732.67){\rule{0.241pt}{0.400pt}}
\multiput(922.00,733.17)(0.500,-1.000){2}{\rule{0.120pt}{0.400pt}}
\put(922.0,734.0){\usebox{\plotpoint}}
\put(923,733){\usebox{\plotpoint}}
\put(923,733){\usebox{\plotpoint}}
\put(923,733){\usebox{\plotpoint}}
\put(923.0,732.0){\usebox{\plotpoint}}
\put(923.0,732.0){\usebox{\plotpoint}}
\put(924.0,730.0){\rule[-0.200pt]{0.400pt}{0.482pt}}
\put(924.0,730.0){\usebox{\plotpoint}}
\put(925.0,728.0){\rule[-0.200pt]{0.400pt}{0.482pt}}
\put(925.0,728.0){\usebox{\plotpoint}}
\put(926,724.67){\rule{0.241pt}{0.400pt}}
\multiput(926.00,725.17)(0.500,-1.000){2}{\rule{0.120pt}{0.400pt}}
\put(926.0,726.0){\rule[-0.200pt]{0.400pt}{0.482pt}}
\put(927,725){\usebox{\plotpoint}}
\put(927,725){\usebox{\plotpoint}}
\put(927,725){\usebox{\plotpoint}}
\put(927,722.67){\rule{0.241pt}{0.400pt}}
\multiput(927.00,723.17)(0.500,-1.000){2}{\rule{0.120pt}{0.400pt}}
\put(927.0,724.0){\usebox{\plotpoint}}
\put(928,723){\usebox{\plotpoint}}
\put(928,723){\usebox{\plotpoint}}
\put(928,723){\usebox{\plotpoint}}
\put(928,720.67){\rule{0.241pt}{0.400pt}}
\multiput(928.00,721.17)(0.500,-1.000){2}{\rule{0.120pt}{0.400pt}}
\put(928.0,722.0){\usebox{\plotpoint}}
\put(929,721){\usebox{\plotpoint}}
\put(929,721){\usebox{\plotpoint}}
\put(929,721){\usebox{\plotpoint}}
\put(929.0,719.0){\rule[-0.200pt]{0.400pt}{0.482pt}}
\put(929.0,719.0){\usebox{\plotpoint}}
\put(930.0,717.0){\rule[-0.200pt]{0.400pt}{0.482pt}}
\put(930.0,717.0){\usebox{\plotpoint}}
\put(931.0,715.0){\rule[-0.200pt]{0.400pt}{0.482pt}}
\put(931.0,715.0){\usebox{\plotpoint}}
\put(932.0,713.0){\rule[-0.200pt]{0.400pt}{0.482pt}}
\put(932.0,713.0){\usebox{\plotpoint}}
\put(933.0,711.0){\rule[-0.200pt]{0.400pt}{0.482pt}}
\put(933.0,711.0){\usebox{\plotpoint}}
\put(934.0,709.0){\rule[-0.200pt]{0.400pt}{0.482pt}}
\put(934.0,709.0){\usebox{\plotpoint}}
\put(935.0,707.0){\rule[-0.200pt]{0.400pt}{0.482pt}}
\put(935.0,707.0){\usebox{\plotpoint}}
\put(936,703.67){\rule{0.241pt}{0.400pt}}
\multiput(936.00,704.17)(0.500,-1.000){2}{\rule{0.120pt}{0.400pt}}
\put(936.0,705.0){\rule[-0.200pt]{0.400pt}{0.482pt}}
\put(937,704){\usebox{\plotpoint}}
\put(937,704){\usebox{\plotpoint}}
\put(937,704){\usebox{\plotpoint}}
\put(937.0,702.0){\rule[-0.200pt]{0.400pt}{0.482pt}}
\put(937.0,702.0){\usebox{\plotpoint}}
\put(938.0,700.0){\rule[-0.200pt]{0.400pt}{0.482pt}}
\put(938.0,700.0){\usebox{\plotpoint}}
\put(939.0,698.0){\rule[-0.200pt]{0.400pt}{0.482pt}}
\put(939.0,698.0){\usebox{\plotpoint}}
\put(940.0,696.0){\rule[-0.200pt]{0.400pt}{0.482pt}}
\put(940.0,696.0){\usebox{\plotpoint}}
\put(941.0,694.0){\rule[-0.200pt]{0.400pt}{0.482pt}}
\put(941.0,694.0){\usebox{\plotpoint}}
\put(942,690.67){\rule{0.241pt}{0.400pt}}
\multiput(942.00,691.17)(0.500,-1.000){2}{\rule{0.120pt}{0.400pt}}
\put(942.0,692.0){\rule[-0.200pt]{0.400pt}{0.482pt}}
\put(943,691){\usebox{\plotpoint}}
\put(943,691){\usebox{\plotpoint}}
\put(943,691){\usebox{\plotpoint}}
\put(943.0,689.0){\rule[-0.200pt]{0.400pt}{0.482pt}}
\put(943.0,689.0){\usebox{\plotpoint}}
\put(944.0,687.0){\rule[-0.200pt]{0.400pt}{0.482pt}}
\put(944.0,687.0){\usebox{\plotpoint}}
\put(945.0,685.0){\rule[-0.200pt]{0.400pt}{0.482pt}}
\put(945.0,685.0){\usebox{\plotpoint}}
\put(946,681.67){\rule{0.241pt}{0.400pt}}
\multiput(946.00,682.17)(0.500,-1.000){2}{\rule{0.120pt}{0.400pt}}
\put(946.0,683.0){\rule[-0.200pt]{0.400pt}{0.482pt}}
\put(947,682){\usebox{\plotpoint}}
\put(947,682){\usebox{\plotpoint}}
\put(947.0,680.0){\rule[-0.200pt]{0.400pt}{0.482pt}}
\put(947.0,680.0){\usebox{\plotpoint}}
\put(948.0,678.0){\rule[-0.200pt]{0.400pt}{0.482pt}}
\put(948.0,678.0){\usebox{\plotpoint}}
\put(949.0,676.0){\rule[-0.200pt]{0.400pt}{0.482pt}}
\put(949.0,676.0){\usebox{\plotpoint}}
\put(950,672.67){\rule{0.241pt}{0.400pt}}
\multiput(950.00,673.17)(0.500,-1.000){2}{\rule{0.120pt}{0.400pt}}
\put(950.0,674.0){\rule[-0.200pt]{0.400pt}{0.482pt}}
\put(951,673){\usebox{\plotpoint}}
\put(951,673){\usebox{\plotpoint}}
\put(951.0,671.0){\rule[-0.200pt]{0.400pt}{0.482pt}}
\put(951.0,671.0){\usebox{\plotpoint}}
\put(952.0,669.0){\rule[-0.200pt]{0.400pt}{0.482pt}}
\put(952.0,669.0){\usebox{\plotpoint}}
\put(953,665.67){\rule{0.241pt}{0.400pt}}
\multiput(953.00,666.17)(0.500,-1.000){2}{\rule{0.120pt}{0.400pt}}
\put(953.0,667.0){\rule[-0.200pt]{0.400pt}{0.482pt}}
\put(954,666){\usebox{\plotpoint}}
\put(954,666){\usebox{\plotpoint}}
\put(954.0,664.0){\rule[-0.200pt]{0.400pt}{0.482pt}}
\put(954.0,664.0){\usebox{\plotpoint}}
\put(955,660.67){\rule{0.241pt}{0.400pt}}
\multiput(955.00,661.17)(0.500,-1.000){2}{\rule{0.120pt}{0.400pt}}
\put(955.0,662.0){\rule[-0.200pt]{0.400pt}{0.482pt}}
\put(956,661){\usebox{\plotpoint}}
\put(956,661){\usebox{\plotpoint}}
\put(956,661){\usebox{\plotpoint}}
\put(956.0,659.0){\rule[-0.200pt]{0.400pt}{0.482pt}}
\put(956.0,659.0){\usebox{\plotpoint}}
\put(957.0,657.0){\rule[-0.200pt]{0.400pt}{0.482pt}}
\put(957.0,657.0){\usebox{\plotpoint}}
\put(958.0,655.0){\rule[-0.200pt]{0.400pt}{0.482pt}}
\put(958.0,655.0){\usebox{\plotpoint}}
\put(959,651.67){\rule{0.241pt}{0.400pt}}
\multiput(959.00,652.17)(0.500,-1.000){2}{\rule{0.120pt}{0.400pt}}
\put(959.0,653.0){\rule[-0.200pt]{0.400pt}{0.482pt}}
\put(960,652){\usebox{\plotpoint}}
\put(960,652){\usebox{\plotpoint}}
\put(960.0,650.0){\rule[-0.200pt]{0.400pt}{0.482pt}}
\put(960.0,650.0){\usebox{\plotpoint}}
\put(961,646.67){\rule{0.241pt}{0.400pt}}
\multiput(961.00,647.17)(0.500,-1.000){2}{\rule{0.120pt}{0.400pt}}
\put(961.0,648.0){\rule[-0.200pt]{0.400pt}{0.482pt}}
\put(962,647){\usebox{\plotpoint}}
\put(962,647){\usebox{\plotpoint}}
\put(962.0,645.0){\rule[-0.200pt]{0.400pt}{0.482pt}}
\put(962.0,645.0){\usebox{\plotpoint}}
\put(963,641.67){\rule{0.241pt}{0.400pt}}
\multiput(963.00,642.17)(0.500,-1.000){2}{\rule{0.120pt}{0.400pt}}
\put(963.0,643.0){\rule[-0.200pt]{0.400pt}{0.482pt}}
\put(964,642){\usebox{\plotpoint}}
\put(964,642){\usebox{\plotpoint}}
\put(964,642){\usebox{\plotpoint}}
\put(964.0,640.0){\rule[-0.200pt]{0.400pt}{0.482pt}}
\put(964.0,640.0){\usebox{\plotpoint}}
\put(965.0,638.0){\rule[-0.200pt]{0.400pt}{0.482pt}}
\put(965.0,638.0){\usebox{\plotpoint}}
\put(966.0,635.0){\rule[-0.200pt]{0.400pt}{0.723pt}}
\put(966.0,635.0){\usebox{\plotpoint}}
\put(967.0,633.0){\rule[-0.200pt]{0.400pt}{0.482pt}}
\put(967.0,633.0){\usebox{\plotpoint}}
\put(968,629.67){\rule{0.241pt}{0.400pt}}
\multiput(968.00,630.17)(0.500,-1.000){2}{\rule{0.120pt}{0.400pt}}
\put(968.0,631.0){\rule[-0.200pt]{0.400pt}{0.482pt}}
\put(969,630){\usebox{\plotpoint}}
\put(969,630){\usebox{\plotpoint}}
\put(969,630){\usebox{\plotpoint}}
\put(969.0,628.0){\rule[-0.200pt]{0.400pt}{0.482pt}}
\put(969.0,628.0){\usebox{\plotpoint}}
\put(970.0,626.0){\rule[-0.200pt]{0.400pt}{0.482pt}}
\put(970.0,626.0){\usebox{\plotpoint}}
\put(971.0,623.0){\rule[-0.200pt]{0.400pt}{0.723pt}}
\put(971.0,623.0){\usebox{\plotpoint}}
\put(972.0,621.0){\rule[-0.200pt]{0.400pt}{0.482pt}}
\put(972.0,621.0){\usebox{\plotpoint}}
\put(973.0,618.0){\rule[-0.200pt]{0.400pt}{0.723pt}}
\put(973.0,618.0){\usebox{\plotpoint}}
\put(974,614.67){\rule{0.241pt}{0.400pt}}
\multiput(974.00,615.17)(0.500,-1.000){2}{\rule{0.120pt}{0.400pt}}
\put(974.0,616.0){\rule[-0.200pt]{0.400pt}{0.482pt}}
\put(975,615){\usebox{\plotpoint}}
\put(975,615){\usebox{\plotpoint}}
\put(975,615){\usebox{\plotpoint}}
\put(975.0,613.0){\rule[-0.200pt]{0.400pt}{0.482pt}}
\put(975.0,613.0){\usebox{\plotpoint}}
\put(976.0,611.0){\rule[-0.200pt]{0.400pt}{0.482pt}}
\put(976.0,611.0){\usebox{\plotpoint}}
\put(977,607.67){\rule{0.241pt}{0.400pt}}
\multiput(977.00,608.17)(0.500,-1.000){2}{\rule{0.120pt}{0.400pt}}
\put(977.0,609.0){\rule[-0.200pt]{0.400pt}{0.482pt}}
\put(978,608){\usebox{\plotpoint}}
\put(978,608){\usebox{\plotpoint}}
\put(978.0,606.0){\rule[-0.200pt]{0.400pt}{0.482pt}}
\put(978.0,606.0){\usebox{\plotpoint}}
\put(979.0,603.0){\rule[-0.200pt]{0.400pt}{0.723pt}}
\put(979.0,603.0){\usebox{\plotpoint}}
\put(980.0,601.0){\rule[-0.200pt]{0.400pt}{0.482pt}}
\put(980.0,601.0){\usebox{\plotpoint}}
\put(981.0,598.0){\rule[-0.200pt]{0.400pt}{0.723pt}}
\put(981.0,598.0){\usebox{\plotpoint}}
\put(982,594.67){\rule{0.241pt}{0.400pt}}
\multiput(982.00,595.17)(0.500,-1.000){2}{\rule{0.120pt}{0.400pt}}
\put(982.0,596.0){\rule[-0.200pt]{0.400pt}{0.482pt}}
\put(983,595){\usebox{\plotpoint}}
\put(983,595){\usebox{\plotpoint}}
\put(983,595){\usebox{\plotpoint}}
\put(983.0,593.0){\rule[-0.200pt]{0.400pt}{0.482pt}}
\put(983.0,593.0){\usebox{\plotpoint}}
\put(984.0,591.0){\rule[-0.200pt]{0.400pt}{0.482pt}}
\put(984.0,591.0){\usebox{\plotpoint}}
\put(985.0,588.0){\rule[-0.200pt]{0.400pt}{0.723pt}}
\put(985.0,588.0){\usebox{\plotpoint}}
\put(986,584.67){\rule{0.241pt}{0.400pt}}
\multiput(986.00,585.17)(0.500,-1.000){2}{\rule{0.120pt}{0.400pt}}
\put(986.0,586.0){\rule[-0.200pt]{0.400pt}{0.482pt}}
\put(987,585){\usebox{\plotpoint}}
\put(987,585){\usebox{\plotpoint}}
\put(987,585){\usebox{\plotpoint}}
\put(987.0,583.0){\rule[-0.200pt]{0.400pt}{0.482pt}}
\put(987.0,583.0){\usebox{\plotpoint}}
\put(988,579.67){\rule{0.241pt}{0.400pt}}
\multiput(988.00,580.17)(0.500,-1.000){2}{\rule{0.120pt}{0.400pt}}
\put(988.0,581.0){\rule[-0.200pt]{0.400pt}{0.482pt}}
\put(989,580){\usebox{\plotpoint}}
\put(989,580){\usebox{\plotpoint}}
\put(989.0,578.0){\rule[-0.200pt]{0.400pt}{0.482pt}}
\put(989.0,578.0){\usebox{\plotpoint}}
\put(990.0,575.0){\rule[-0.200pt]{0.400pt}{0.723pt}}
\put(990.0,575.0){\usebox{\plotpoint}}
\put(991.0,573.0){\rule[-0.200pt]{0.400pt}{0.482pt}}
\put(991.0,573.0){\usebox{\plotpoint}}
\put(992.0,570.0){\rule[-0.200pt]{0.400pt}{0.723pt}}
\put(992.0,570.0){\usebox{\plotpoint}}
\put(993.0,568.0){\rule[-0.200pt]{0.400pt}{0.482pt}}
\put(993.0,568.0){\usebox{\plotpoint}}
\put(994.0,565.0){\rule[-0.200pt]{0.400pt}{0.723pt}}
\put(994.0,565.0){\usebox{\plotpoint}}
\put(995,561.67){\rule{0.241pt}{0.400pt}}
\multiput(995.00,562.17)(0.500,-1.000){2}{\rule{0.120pt}{0.400pt}}
\put(995.0,563.0){\rule[-0.200pt]{0.400pt}{0.482pt}}
\put(996,562){\usebox{\plotpoint}}
\put(996,562){\usebox{\plotpoint}}
\put(996.0,560.0){\rule[-0.200pt]{0.400pt}{0.482pt}}
\put(996.0,560.0){\usebox{\plotpoint}}
\put(997.0,557.0){\rule[-0.200pt]{0.400pt}{0.723pt}}
\put(997.0,557.0){\usebox{\plotpoint}}
\put(998.0,555.0){\rule[-0.200pt]{0.400pt}{0.482pt}}
\put(998.0,555.0){\usebox{\plotpoint}}
\put(999.0,552.0){\rule[-0.200pt]{0.400pt}{0.723pt}}
\put(999.0,552.0){\usebox{\plotpoint}}
\put(1000,548.67){\rule{0.241pt}{0.400pt}}
\multiput(1000.00,549.17)(0.500,-1.000){2}{\rule{0.120pt}{0.400pt}}
\put(1000.0,550.0){\rule[-0.200pt]{0.400pt}{0.482pt}}
\put(1001,549){\usebox{\plotpoint}}
\put(1001,549){\usebox{\plotpoint}}
\put(1001.0,547.0){\rule[-0.200pt]{0.400pt}{0.482pt}}
\put(1001.0,547.0){\usebox{\plotpoint}}
\put(1002,543.67){\rule{0.241pt}{0.400pt}}
\multiput(1002.00,544.17)(0.500,-1.000){2}{\rule{0.120pt}{0.400pt}}
\put(1002.0,545.0){\rule[-0.200pt]{0.400pt}{0.482pt}}
\put(1003,544){\usebox{\plotpoint}}
\put(1003,544){\usebox{\plotpoint}}
\put(1003.0,542.0){\rule[-0.200pt]{0.400pt}{0.482pt}}
\put(1003.0,542.0){\usebox{\plotpoint}}
\put(1004.0,539.0){\rule[-0.200pt]{0.400pt}{0.723pt}}
\put(1004.0,539.0){\usebox{\plotpoint}}
\put(1005,535.67){\rule{0.241pt}{0.400pt}}
\multiput(1005.00,536.17)(0.500,-1.000){2}{\rule{0.120pt}{0.400pt}}
\put(1005.0,537.0){\rule[-0.200pt]{0.400pt}{0.482pt}}
\put(1006,536){\usebox{\plotpoint}}
\put(1006,536){\usebox{\plotpoint}}
\put(1006.0,534.0){\rule[-0.200pt]{0.400pt}{0.482pt}}
\put(1006.0,534.0){\usebox{\plotpoint}}
\put(1007.0,531.0){\rule[-0.200pt]{0.400pt}{0.723pt}}
\put(1007.0,531.0){\usebox{\plotpoint}}
\put(1008,527.67){\rule{0.241pt}{0.400pt}}
\multiput(1008.00,528.17)(0.500,-1.000){2}{\rule{0.120pt}{0.400pt}}
\put(1008.0,529.0){\rule[-0.200pt]{0.400pt}{0.482pt}}
\put(1009,528){\usebox{\plotpoint}}
\put(1009,528){\usebox{\plotpoint}}
\put(1009.0,526.0){\rule[-0.200pt]{0.400pt}{0.482pt}}
\put(1009.0,526.0){\usebox{\plotpoint}}
\put(1010,522.67){\rule{0.241pt}{0.400pt}}
\multiput(1010.00,523.17)(0.500,-1.000){2}{\rule{0.120pt}{0.400pt}}
\put(1010.0,524.0){\rule[-0.200pt]{0.400pt}{0.482pt}}
\put(1011,523){\usebox{\plotpoint}}
\put(1011,523){\usebox{\plotpoint}}
\put(1011.0,521.0){\rule[-0.200pt]{0.400pt}{0.482pt}}
\put(1011.0,521.0){\usebox{\plotpoint}}
\put(1012,517.67){\rule{0.241pt}{0.400pt}}
\multiput(1012.00,518.17)(0.500,-1.000){2}{\rule{0.120pt}{0.400pt}}
\put(1012.0,519.0){\rule[-0.200pt]{0.400pt}{0.482pt}}
\put(1013,518){\usebox{\plotpoint}}
\put(1013,518){\usebox{\plotpoint}}
\put(1013.0,516.0){\rule[-0.200pt]{0.400pt}{0.482pt}}
\put(1013.0,516.0){\usebox{\plotpoint}}
\put(1014.0,513.0){\rule[-0.200pt]{0.400pt}{0.723pt}}
\put(1014.0,513.0){\usebox{\plotpoint}}
\put(1015,509.67){\rule{0.241pt}{0.400pt}}
\multiput(1015.00,510.17)(0.500,-1.000){2}{\rule{0.120pt}{0.400pt}}
\put(1015.0,511.0){\rule[-0.200pt]{0.400pt}{0.482pt}}
\put(1016,510){\usebox{\plotpoint}}
\put(1016,510){\usebox{\plotpoint}}
\put(1016.0,508.0){\rule[-0.200pt]{0.400pt}{0.482pt}}
\put(1016.0,508.0){\usebox{\plotpoint}}
\put(1017.0,505.0){\rule[-0.200pt]{0.400pt}{0.723pt}}
\put(1017.0,505.0){\usebox{\plotpoint}}
\put(1018,501.67){\rule{0.241pt}{0.400pt}}
\multiput(1018.00,502.17)(0.500,-1.000){2}{\rule{0.120pt}{0.400pt}}
\put(1018.0,503.0){\rule[-0.200pt]{0.400pt}{0.482pt}}
\put(1019,502){\usebox{\plotpoint}}
\put(1019,502){\usebox{\plotpoint}}
\put(1019.0,500.0){\rule[-0.200pt]{0.400pt}{0.482pt}}
\put(1019.0,500.0){\usebox{\plotpoint}}
\put(1020,496.67){\rule{0.241pt}{0.400pt}}
\multiput(1020.00,497.17)(0.500,-1.000){2}{\rule{0.120pt}{0.400pt}}
\put(1020.0,498.0){\rule[-0.200pt]{0.400pt}{0.482pt}}
\put(1021,497){\usebox{\plotpoint}}
\put(1021,497){\usebox{\plotpoint}}
\put(1021.0,495.0){\rule[-0.200pt]{0.400pt}{0.482pt}}
\put(1021.0,495.0){\usebox{\plotpoint}}
\put(1022.0,492.0){\rule[-0.200pt]{0.400pt}{0.723pt}}
\put(1022.0,492.0){\usebox{\plotpoint}}
\put(1023,488.67){\rule{0.241pt}{0.400pt}}
\multiput(1023.00,489.17)(0.500,-1.000){2}{\rule{0.120pt}{0.400pt}}
\put(1023.0,490.0){\rule[-0.200pt]{0.400pt}{0.482pt}}
\put(1024,489){\usebox{\plotpoint}}
\put(1024,489){\usebox{\plotpoint}}
\put(1024.0,487.0){\rule[-0.200pt]{0.400pt}{0.482pt}}
\put(1024.0,487.0){\usebox{\plotpoint}}
\put(1025.0,484.0){\rule[-0.200pt]{0.400pt}{0.723pt}}
\put(1025.0,484.0){\usebox{\plotpoint}}
\put(1026,480.67){\rule{0.241pt}{0.400pt}}
\multiput(1026.00,481.17)(0.500,-1.000){2}{\rule{0.120pt}{0.400pt}}
\put(1026.0,482.0){\rule[-0.200pt]{0.400pt}{0.482pt}}
\put(1027,481){\usebox{\plotpoint}}
\put(1027,481){\usebox{\plotpoint}}
\put(1027.0,479.0){\rule[-0.200pt]{0.400pt}{0.482pt}}
\put(1027.0,479.0){\usebox{\plotpoint}}
\put(1028,475.67){\rule{0.241pt}{0.400pt}}
\multiput(1028.00,476.17)(0.500,-1.000){2}{\rule{0.120pt}{0.400pt}}
\put(1028.0,477.0){\rule[-0.200pt]{0.400pt}{0.482pt}}
\put(1029,476){\usebox{\plotpoint}}
\put(1029,476){\usebox{\plotpoint}}
\put(1029.0,474.0){\rule[-0.200pt]{0.400pt}{0.482pt}}
\put(1029.0,474.0){\usebox{\plotpoint}}
\put(1030.0,471.0){\rule[-0.200pt]{0.400pt}{0.723pt}}
\put(1030.0,471.0){\usebox{\plotpoint}}
\put(1031,467.67){\rule{0.241pt}{0.400pt}}
\multiput(1031.00,468.17)(0.500,-1.000){2}{\rule{0.120pt}{0.400pt}}
\put(1031.0,469.0){\rule[-0.200pt]{0.400pt}{0.482pt}}
\put(1032,468){\usebox{\plotpoint}}
\put(1032,468){\usebox{\plotpoint}}
\put(1032.0,466.0){\rule[-0.200pt]{0.400pt}{0.482pt}}
\put(1032.0,466.0){\usebox{\plotpoint}}
\put(1033,462.67){\rule{0.241pt}{0.400pt}}
\multiput(1033.00,463.17)(0.500,-1.000){2}{\rule{0.120pt}{0.400pt}}
\put(1033.0,464.0){\rule[-0.200pt]{0.400pt}{0.482pt}}
\put(1034,463){\usebox{\plotpoint}}
\put(1034,463){\usebox{\plotpoint}}
\put(1034.0,461.0){\rule[-0.200pt]{0.400pt}{0.482pt}}
\put(1034.0,461.0){\usebox{\plotpoint}}
\put(1035.0,458.0){\rule[-0.200pt]{0.400pt}{0.723pt}}
\put(1035.0,458.0){\usebox{\plotpoint}}
\put(1036,454.67){\rule{0.241pt}{0.400pt}}
\multiput(1036.00,455.17)(0.500,-1.000){2}{\rule{0.120pt}{0.400pt}}
\put(1036.0,456.0){\rule[-0.200pt]{0.400pt}{0.482pt}}
\put(1037,455){\usebox{\plotpoint}}
\put(1037,455){\usebox{\plotpoint}}
\put(1037.0,453.0){\rule[-0.200pt]{0.400pt}{0.482pt}}
\put(1037.0,453.0){\usebox{\plotpoint}}
\put(1038,449.67){\rule{0.241pt}{0.400pt}}
\multiput(1038.00,450.17)(0.500,-1.000){2}{\rule{0.120pt}{0.400pt}}
\put(1038.0,451.0){\rule[-0.200pt]{0.400pt}{0.482pt}}
\put(1039,450){\usebox{\plotpoint}}
\put(1039,450){\usebox{\plotpoint}}
\put(1039.0,448.0){\rule[-0.200pt]{0.400pt}{0.482pt}}
\put(1039.0,448.0){\usebox{\plotpoint}}
\put(1040.0,445.0){\rule[-0.200pt]{0.400pt}{0.723pt}}
\put(1040.0,445.0){\usebox{\plotpoint}}
\put(1041,441.67){\rule{0.241pt}{0.400pt}}
\multiput(1041.00,442.17)(0.500,-1.000){2}{\rule{0.120pt}{0.400pt}}
\put(1041.0,443.0){\rule[-0.200pt]{0.400pt}{0.482pt}}
\put(1042,442){\usebox{\plotpoint}}
\put(1042,442){\usebox{\plotpoint}}
\put(1042.0,440.0){\rule[-0.200pt]{0.400pt}{0.482pt}}
\put(1042.0,440.0){\usebox{\plotpoint}}
\put(1043,436.67){\rule{0.241pt}{0.400pt}}
\multiput(1043.00,437.17)(0.500,-1.000){2}{\rule{0.120pt}{0.400pt}}
\put(1043.0,438.0){\rule[-0.200pt]{0.400pt}{0.482pt}}
\put(1044,437){\usebox{\plotpoint}}
\put(1044,437){\usebox{\plotpoint}}
\put(1044.0,435.0){\rule[-0.200pt]{0.400pt}{0.482pt}}
\put(1044.0,435.0){\usebox{\plotpoint}}
\put(1045,431.67){\rule{0.241pt}{0.400pt}}
\multiput(1045.00,432.17)(0.500,-1.000){2}{\rule{0.120pt}{0.400pt}}
\put(1045.0,433.0){\rule[-0.200pt]{0.400pt}{0.482pt}}
\put(1046,432){\usebox{\plotpoint}}
\put(1046,432){\usebox{\plotpoint}}
\put(1046.0,430.0){\rule[-0.200pt]{0.400pt}{0.482pt}}
\put(1046.0,430.0){\usebox{\plotpoint}}
\put(1047.0,427.0){\rule[-0.200pt]{0.400pt}{0.723pt}}
\put(1047.0,427.0){\usebox{\plotpoint}}
\put(1048.0,425.0){\rule[-0.200pt]{0.400pt}{0.482pt}}
\put(1048.0,425.0){\usebox{\plotpoint}}
\put(1049.0,422.0){\rule[-0.200pt]{0.400pt}{0.723pt}}
\put(1049.0,422.0){\usebox{\plotpoint}}
\put(1050,418.67){\rule{0.241pt}{0.400pt}}
\multiput(1050.00,419.17)(0.500,-1.000){2}{\rule{0.120pt}{0.400pt}}
\put(1050.0,420.0){\rule[-0.200pt]{0.400pt}{0.482pt}}
\put(1051,419){\usebox{\plotpoint}}
\put(1051,419){\usebox{\plotpoint}}
\put(1051.0,417.0){\rule[-0.200pt]{0.400pt}{0.482pt}}
\put(1051.0,417.0){\usebox{\plotpoint}}
\put(1052,413.67){\rule{0.241pt}{0.400pt}}
\multiput(1052.00,414.17)(0.500,-1.000){2}{\rule{0.120pt}{0.400pt}}
\put(1052.0,415.0){\rule[-0.200pt]{0.400pt}{0.482pt}}
\put(1053,414){\usebox{\plotpoint}}
\put(1053,414){\usebox{\plotpoint}}
\put(1053.0,412.0){\rule[-0.200pt]{0.400pt}{0.482pt}}
\put(1053.0,412.0){\usebox{\plotpoint}}
\put(1054,408.67){\rule{0.241pt}{0.400pt}}
\multiput(1054.00,409.17)(0.500,-1.000){2}{\rule{0.120pt}{0.400pt}}
\put(1054.0,410.0){\rule[-0.200pt]{0.400pt}{0.482pt}}
\put(1055,409){\usebox{\plotpoint}}
\put(1055,409){\usebox{\plotpoint}}
\put(1055.0,407.0){\rule[-0.200pt]{0.400pt}{0.482pt}}
\put(1055.0,407.0){\usebox{\plotpoint}}
\put(1056,403.67){\rule{0.241pt}{0.400pt}}
\multiput(1056.00,404.17)(0.500,-1.000){2}{\rule{0.120pt}{0.400pt}}
\put(1056.0,405.0){\rule[-0.200pt]{0.400pt}{0.482pt}}
\put(1057,404){\usebox{\plotpoint}}
\put(1057,404){\usebox{\plotpoint}}
\put(1057.0,402.0){\rule[-0.200pt]{0.400pt}{0.482pt}}
\put(1057.0,402.0){\usebox{\plotpoint}}
\put(1058.0,399.0){\rule[-0.200pt]{0.400pt}{0.723pt}}
\put(1058.0,399.0){\usebox{\plotpoint}}
\put(1059.0,397.0){\rule[-0.200pt]{0.400pt}{0.482pt}}
\put(1059.0,397.0){\usebox{\plotpoint}}
\put(1060.0,394.0){\rule[-0.200pt]{0.400pt}{0.723pt}}
\put(1060.0,394.0){\usebox{\plotpoint}}
\put(1061.0,392.0){\rule[-0.200pt]{0.400pt}{0.482pt}}
\put(1061.0,392.0){\usebox{\plotpoint}}
\put(1062.0,389.0){\rule[-0.200pt]{0.400pt}{0.723pt}}
\put(1062.0,389.0){\usebox{\plotpoint}}
\put(1063.0,387.0){\rule[-0.200pt]{0.400pt}{0.482pt}}
\put(1063.0,387.0){\usebox{\plotpoint}}
\put(1064,383.67){\rule{0.241pt}{0.400pt}}
\multiput(1064.00,384.17)(0.500,-1.000){2}{\rule{0.120pt}{0.400pt}}
\put(1064.0,385.0){\rule[-0.200pt]{0.400pt}{0.482pt}}
\put(1065,384){\usebox{\plotpoint}}
\put(1065,384){\usebox{\plotpoint}}
\put(1065.0,382.0){\rule[-0.200pt]{0.400pt}{0.482pt}}
\put(1065.0,382.0){\usebox{\plotpoint}}
\put(1066,378.67){\rule{0.241pt}{0.400pt}}
\multiput(1066.00,379.17)(0.500,-1.000){2}{\rule{0.120pt}{0.400pt}}
\put(1066.0,380.0){\rule[-0.200pt]{0.400pt}{0.482pt}}
\put(1067,379){\usebox{\plotpoint}}
\put(1067,379){\usebox{\plotpoint}}
\put(1067.0,377.0){\rule[-0.200pt]{0.400pt}{0.482pt}}
\put(1067.0,377.0){\usebox{\plotpoint}}
\put(1068,373.67){\rule{0.241pt}{0.400pt}}
\multiput(1068.00,374.17)(0.500,-1.000){2}{\rule{0.120pt}{0.400pt}}
\put(1068.0,375.0){\rule[-0.200pt]{0.400pt}{0.482pt}}
\put(1069,374){\usebox{\plotpoint}}
\put(1069,374){\usebox{\plotpoint}}
\put(1069.0,372.0){\rule[-0.200pt]{0.400pt}{0.482pt}}
\put(1069.0,372.0){\usebox{\plotpoint}}
\put(1070,368.67){\rule{0.241pt}{0.400pt}}
\multiput(1070.00,369.17)(0.500,-1.000){2}{\rule{0.120pt}{0.400pt}}
\put(1070.0,370.0){\rule[-0.200pt]{0.400pt}{0.482pt}}
\put(1071,369){\usebox{\plotpoint}}
\put(1071,369){\usebox{\plotpoint}}
\put(1071.0,367.0){\rule[-0.200pt]{0.400pt}{0.482pt}}
\put(1071.0,367.0){\usebox{\plotpoint}}
\put(1072.0,365.0){\rule[-0.200pt]{0.400pt}{0.482pt}}
\put(1072.0,365.0){\usebox{\plotpoint}}
\put(1073,361.67){\rule{0.241pt}{0.400pt}}
\multiput(1073.00,362.17)(0.500,-1.000){2}{\rule{0.120pt}{0.400pt}}
\put(1073.0,363.0){\rule[-0.200pt]{0.400pt}{0.482pt}}
\put(1074,362){\usebox{\plotpoint}}
\put(1074,362){\usebox{\plotpoint}}
\put(1074.0,360.0){\rule[-0.200pt]{0.400pt}{0.482pt}}
\put(1074.0,360.0){\usebox{\plotpoint}}
\put(1075,356.67){\rule{0.241pt}{0.400pt}}
\multiput(1075.00,357.17)(0.500,-1.000){2}{\rule{0.120pt}{0.400pt}}
\put(1075.0,358.0){\rule[-0.200pt]{0.400pt}{0.482pt}}
\put(1076,357){\usebox{\plotpoint}}
\put(1076,357){\usebox{\plotpoint}}
\put(1076.0,355.0){\rule[-0.200pt]{0.400pt}{0.482pt}}
\put(1076.0,355.0){\usebox{\plotpoint}}
\put(1077,351.67){\rule{0.241pt}{0.400pt}}
\multiput(1077.00,352.17)(0.500,-1.000){2}{\rule{0.120pt}{0.400pt}}
\put(1077.0,353.0){\rule[-0.200pt]{0.400pt}{0.482pt}}
\put(1078,352){\usebox{\plotpoint}}
\put(1078,352){\usebox{\plotpoint}}
\put(1078,352){\usebox{\plotpoint}}
\put(1078.0,350.0){\rule[-0.200pt]{0.400pt}{0.482pt}}
\put(1078.0,350.0){\usebox{\plotpoint}}
\put(1079.0,348.0){\rule[-0.200pt]{0.400pt}{0.482pt}}
\put(1079.0,348.0){\usebox{\plotpoint}}
\put(1080.0,346.0){\rule[-0.200pt]{0.400pt}{0.482pt}}
\put(1080.0,346.0){\usebox{\plotpoint}}
\put(1081.0,343.0){\rule[-0.200pt]{0.400pt}{0.723pt}}
\put(1081.0,343.0){\usebox{\plotpoint}}
\put(1082.0,341.0){\rule[-0.200pt]{0.400pt}{0.482pt}}
\put(1082.0,341.0){\usebox{\plotpoint}}
\put(1083,337.67){\rule{0.241pt}{0.400pt}}
\multiput(1083.00,338.17)(0.500,-1.000){2}{\rule{0.120pt}{0.400pt}}
\put(1083.0,339.0){\rule[-0.200pt]{0.400pt}{0.482pt}}
\put(1084,338){\usebox{\plotpoint}}
\put(1084,338){\usebox{\plotpoint}}
\put(1084,338){\usebox{\plotpoint}}
\put(1084.0,336.0){\rule[-0.200pt]{0.400pt}{0.482pt}}
\put(1084.0,336.0){\usebox{\plotpoint}}
\put(1085.0,334.0){\rule[-0.200pt]{0.400pt}{0.482pt}}
\put(1085.0,334.0){\usebox{\plotpoint}}
\put(1086,330.67){\rule{0.241pt}{0.400pt}}
\multiput(1086.00,331.17)(0.500,-1.000){2}{\rule{0.120pt}{0.400pt}}
\put(1086.0,332.0){\rule[-0.200pt]{0.400pt}{0.482pt}}
\put(1087,331){\usebox{\plotpoint}}
\put(1087,331){\usebox{\plotpoint}}
\put(1087.0,329.0){\rule[-0.200pt]{0.400pt}{0.482pt}}
\put(1087.0,329.0){\usebox{\plotpoint}}
\put(1088.0,327.0){\rule[-0.200pt]{0.400pt}{0.482pt}}
\put(1088.0,327.0){\usebox{\plotpoint}}
\put(1089.0,325.0){\rule[-0.200pt]{0.400pt}{0.482pt}}
\put(1089.0,325.0){\usebox{\plotpoint}}
\put(1090,321.67){\rule{0.241pt}{0.400pt}}
\multiput(1090.00,322.17)(0.500,-1.000){2}{\rule{0.120pt}{0.400pt}}
\put(1090.0,323.0){\rule[-0.200pt]{0.400pt}{0.482pt}}
\put(1091,322){\usebox{\plotpoint}}
\put(1091,322){\usebox{\plotpoint}}
\put(1091.0,320.0){\rule[-0.200pt]{0.400pt}{0.482pt}}
\put(1091.0,320.0){\usebox{\plotpoint}}
\put(1092.0,318.0){\rule[-0.200pt]{0.400pt}{0.482pt}}
\put(1092.0,318.0){\usebox{\plotpoint}}
\put(1093,314.67){\rule{0.241pt}{0.400pt}}
\multiput(1093.00,315.17)(0.500,-1.000){2}{\rule{0.120pt}{0.400pt}}
\put(1093.0,316.0){\rule[-0.200pt]{0.400pt}{0.482pt}}
\put(1094,315){\usebox{\plotpoint}}
\put(1094,315){\usebox{\plotpoint}}
\put(1094,315){\usebox{\plotpoint}}
\put(1094.0,313.0){\rule[-0.200pt]{0.400pt}{0.482pt}}
\put(1094.0,313.0){\usebox{\plotpoint}}
\put(1095.0,311.0){\rule[-0.200pt]{0.400pt}{0.482pt}}
\put(1095.0,311.0){\usebox{\plotpoint}}
\put(1096.0,309.0){\rule[-0.200pt]{0.400pt}{0.482pt}}
\put(1096.0,309.0){\usebox{\plotpoint}}
\put(1097,305.67){\rule{0.241pt}{0.400pt}}
\multiput(1097.00,306.17)(0.500,-1.000){2}{\rule{0.120pt}{0.400pt}}
\put(1097.0,307.0){\rule[-0.200pt]{0.400pt}{0.482pt}}
\put(1098,306){\usebox{\plotpoint}}
\put(1098,306){\usebox{\plotpoint}}
\put(1098,306){\usebox{\plotpoint}}
\put(1098,303.67){\rule{0.241pt}{0.400pt}}
\multiput(1098.00,304.17)(0.500,-1.000){2}{\rule{0.120pt}{0.400pt}}
\put(1098.0,305.0){\usebox{\plotpoint}}
\put(1099,304){\usebox{\plotpoint}}
\put(1099,304){\usebox{\plotpoint}}
\put(1099,304){\usebox{\plotpoint}}
\put(1099.0,302.0){\rule[-0.200pt]{0.400pt}{0.482pt}}
\put(1099.0,302.0){\usebox{\plotpoint}}
\put(1100.0,300.0){\rule[-0.200pt]{0.400pt}{0.482pt}}
\put(1100.0,300.0){\usebox{\plotpoint}}
\put(1101.0,298.0){\rule[-0.200pt]{0.400pt}{0.482pt}}
\put(1101.0,298.0){\usebox{\plotpoint}}
\put(1102.0,296.0){\rule[-0.200pt]{0.400pt}{0.482pt}}
\put(1102.0,296.0){\usebox{\plotpoint}}
\put(1103,292.67){\rule{0.241pt}{0.400pt}}
\multiput(1103.00,293.17)(0.500,-1.000){2}{\rule{0.120pt}{0.400pt}}
\put(1103.0,294.0){\rule[-0.200pt]{0.400pt}{0.482pt}}
\put(1104,293){\usebox{\plotpoint}}
\put(1104,293){\usebox{\plotpoint}}
\put(1104,293){\usebox{\plotpoint}}
\put(1104.0,291.0){\rule[-0.200pt]{0.400pt}{0.482pt}}
\put(1104.0,291.0){\usebox{\plotpoint}}
\put(1105.0,289.0){\rule[-0.200pt]{0.400pt}{0.482pt}}
\put(1105.0,289.0){\usebox{\plotpoint}}
\put(1106.0,287.0){\rule[-0.200pt]{0.400pt}{0.482pt}}
\put(1106.0,287.0){\usebox{\plotpoint}}
\put(1107.0,285.0){\rule[-0.200pt]{0.400pt}{0.482pt}}
\put(1107.0,285.0){\usebox{\plotpoint}}
\put(1108.0,283.0){\rule[-0.200pt]{0.400pt}{0.482pt}}
\put(1108.0,283.0){\usebox{\plotpoint}}
\put(1109.0,281.0){\rule[-0.200pt]{0.400pt}{0.482pt}}
\put(1109.0,281.0){\usebox{\plotpoint}}
\put(1110.0,279.0){\rule[-0.200pt]{0.400pt}{0.482pt}}
\put(1110.0,279.0){\usebox{\plotpoint}}
\put(1111.0,277.0){\rule[-0.200pt]{0.400pt}{0.482pt}}
\put(1111.0,277.0){\usebox{\plotpoint}}
\put(1112,273.67){\rule{0.241pt}{0.400pt}}
\multiput(1112.00,274.17)(0.500,-1.000){2}{\rule{0.120pt}{0.400pt}}
\put(1112.0,275.0){\rule[-0.200pt]{0.400pt}{0.482pt}}
\put(1113,274){\usebox{\plotpoint}}
\put(1113,274){\usebox{\plotpoint}}
\put(1113,274){\usebox{\plotpoint}}
\put(1113,271.67){\rule{0.241pt}{0.400pt}}
\multiput(1113.00,272.17)(0.500,-1.000){2}{\rule{0.120pt}{0.400pt}}
\put(1113.0,273.0){\usebox{\plotpoint}}
\put(1114,272){\usebox{\plotpoint}}
\put(1114,272){\usebox{\plotpoint}}
\put(1114,272){\usebox{\plotpoint}}
\put(1114,269.67){\rule{0.241pt}{0.400pt}}
\multiput(1114.00,270.17)(0.500,-1.000){2}{\rule{0.120pt}{0.400pt}}
\put(1114.0,271.0){\usebox{\plotpoint}}
\put(1115,270){\usebox{\plotpoint}}
\put(1115,270){\usebox{\plotpoint}}
\put(1115,270){\usebox{\plotpoint}}
\put(1115.0,269.0){\usebox{\plotpoint}}
\put(1115.0,269.0){\usebox{\plotpoint}}
\put(1116.0,267.0){\rule[-0.200pt]{0.400pt}{0.482pt}}
\put(1116.0,267.0){\usebox{\plotpoint}}
\put(1117,263.67){\rule{0.241pt}{0.400pt}}
\multiput(1117.00,264.17)(0.500,-1.000){2}{\rule{0.120pt}{0.400pt}}
\put(1117.0,265.0){\rule[-0.200pt]{0.400pt}{0.482pt}}
\put(1118,264){\usebox{\plotpoint}}
\put(1118,264){\usebox{\plotpoint}}
\put(1118,264){\usebox{\plotpoint}}
\put(1118,261.67){\rule{0.241pt}{0.400pt}}
\multiput(1118.00,262.17)(0.500,-1.000){2}{\rule{0.120pt}{0.400pt}}
\put(1118.0,263.0){\usebox{\plotpoint}}
\put(1119,262){\usebox{\plotpoint}}
\put(1119,262){\usebox{\plotpoint}}
\put(1119,262){\usebox{\plotpoint}}
\put(1119,259.67){\rule{0.241pt}{0.400pt}}
\multiput(1119.00,260.17)(0.500,-1.000){2}{\rule{0.120pt}{0.400pt}}
\put(1119.0,261.0){\usebox{\plotpoint}}
\put(1120,260){\usebox{\plotpoint}}
\put(1120,260){\usebox{\plotpoint}}
\put(1120,260){\usebox{\plotpoint}}
\put(1120,257.67){\rule{0.241pt}{0.400pt}}
\multiput(1120.00,258.17)(0.500,-1.000){2}{\rule{0.120pt}{0.400pt}}
\put(1120.0,259.0){\usebox{\plotpoint}}
\put(1121,258){\usebox{\plotpoint}}
\put(1121,258){\usebox{\plotpoint}}
\put(1121,258){\usebox{\plotpoint}}
\put(1121.0,257.0){\usebox{\plotpoint}}
\put(1121.0,257.0){\usebox{\plotpoint}}
\put(1122.0,255.0){\rule[-0.200pt]{0.400pt}{0.482pt}}
\put(1122.0,255.0){\usebox{\plotpoint}}
\put(1123.0,253.0){\rule[-0.200pt]{0.400pt}{0.482pt}}
\put(1123.0,253.0){\usebox{\plotpoint}}
\put(1124.0,251.0){\rule[-0.200pt]{0.400pt}{0.482pt}}
\put(1124.0,251.0){\usebox{\plotpoint}}
\put(1125.0,249.0){\rule[-0.200pt]{0.400pt}{0.482pt}}
\put(1125.0,249.0){\usebox{\plotpoint}}
\put(1126.0,247.0){\rule[-0.200pt]{0.400pt}{0.482pt}}
\put(1126.0,247.0){\usebox{\plotpoint}}
\put(1127.0,245.0){\rule[-0.200pt]{0.400pt}{0.482pt}}
\put(1127.0,245.0){\usebox{\plotpoint}}
\put(1128,242.67){\rule{0.241pt}{0.400pt}}
\multiput(1128.00,243.17)(0.500,-1.000){2}{\rule{0.120pt}{0.400pt}}
\put(1128.0,244.0){\usebox{\plotpoint}}
\put(1129,243){\usebox{\plotpoint}}
\put(1129,243){\usebox{\plotpoint}}
\put(1129,243){\usebox{\plotpoint}}
\put(1129,240.67){\rule{0.241pt}{0.400pt}}
\multiput(1129.00,241.17)(0.500,-1.000){2}{\rule{0.120pt}{0.400pt}}
\put(1129.0,242.0){\usebox{\plotpoint}}
\put(1130,241){\usebox{\plotpoint}}
\put(1130,241){\usebox{\plotpoint}}
\put(1130,241){\usebox{\plotpoint}}
\put(1130.0,240.0){\usebox{\plotpoint}}
\put(1130.0,240.0){\usebox{\plotpoint}}
\put(1131.0,238.0){\rule[-0.200pt]{0.400pt}{0.482pt}}
\put(1131.0,238.0){\usebox{\plotpoint}}
\put(1132.0,236.0){\rule[-0.200pt]{0.400pt}{0.482pt}}
\put(1132.0,236.0){\usebox{\plotpoint}}
\put(1133,233.67){\rule{0.241pt}{0.400pt}}
\multiput(1133.00,234.17)(0.500,-1.000){2}{\rule{0.120pt}{0.400pt}}
\put(1133.0,235.0){\usebox{\plotpoint}}
\put(1134,234){\usebox{\plotpoint}}
\put(1134,234){\usebox{\plotpoint}}
\put(1134,234){\usebox{\plotpoint}}
\put(1134,231.67){\rule{0.241pt}{0.400pt}}
\multiput(1134.00,232.17)(0.500,-1.000){2}{\rule{0.120pt}{0.400pt}}
\put(1134.0,233.0){\usebox{\plotpoint}}
\put(1135,232){\usebox{\plotpoint}}
\put(1135,232){\usebox{\plotpoint}}
\put(1135,232){\usebox{\plotpoint}}
\put(1135,232){\usebox{\plotpoint}}
\put(1135.0,231.0){\usebox{\plotpoint}}
\put(1135.0,231.0){\usebox{\plotpoint}}
\put(1136.0,229.0){\rule[-0.200pt]{0.400pt}{0.482pt}}
\put(1136.0,229.0){\usebox{\plotpoint}}
\put(1137.0,227.0){\rule[-0.200pt]{0.400pt}{0.482pt}}
\put(1137.0,227.0){\usebox{\plotpoint}}
\put(1138,224.67){\rule{0.241pt}{0.400pt}}
\multiput(1138.00,225.17)(0.500,-1.000){2}{\rule{0.120pt}{0.400pt}}
\put(1138.0,226.0){\usebox{\plotpoint}}
\put(1139,225){\usebox{\plotpoint}}
\put(1139,225){\usebox{\plotpoint}}
\put(1139,225){\usebox{\plotpoint}}
\put(1139,225){\usebox{\plotpoint}}
\put(1139.0,224.0){\usebox{\plotpoint}}
\put(1139.0,224.0){\usebox{\plotpoint}}
\put(1140.0,222.0){\rule[-0.200pt]{0.400pt}{0.482pt}}
\put(1140.0,222.0){\usebox{\plotpoint}}
\put(1141.0,220.0){\rule[-0.200pt]{0.400pt}{0.482pt}}
\put(1141.0,220.0){\usebox{\plotpoint}}
\put(1142.0,219.0){\usebox{\plotpoint}}
\put(1142.0,219.0){\usebox{\plotpoint}}
\put(1143.0,217.0){\rule[-0.200pt]{0.400pt}{0.482pt}}
\put(1143.0,217.0){\usebox{\plotpoint}}
\put(1144,214.67){\rule{0.241pt}{0.400pt}}
\multiput(1144.00,215.17)(0.500,-1.000){2}{\rule{0.120pt}{0.400pt}}
\put(1144.0,216.0){\usebox{\plotpoint}}
\put(1145,215){\usebox{\plotpoint}}
\put(1145,215){\usebox{\plotpoint}}
\put(1145,215){\usebox{\plotpoint}}
\put(1145,215){\usebox{\plotpoint}}
\put(1145.0,214.0){\usebox{\plotpoint}}
\put(1145.0,214.0){\usebox{\plotpoint}}
\put(1146.0,212.0){\rule[-0.200pt]{0.400pt}{0.482pt}}
\put(1146.0,212.0){\usebox{\plotpoint}}
\put(1147,209.67){\rule{0.241pt}{0.400pt}}
\multiput(1147.00,210.17)(0.500,-1.000){2}{\rule{0.120pt}{0.400pt}}
\put(1147.0,211.0){\usebox{\plotpoint}}
\put(1148,210){\usebox{\plotpoint}}
\put(1148,210){\usebox{\plotpoint}}
\put(1148,210){\usebox{\plotpoint}}
\put(1148,210){\usebox{\plotpoint}}
\put(1148.0,209.0){\usebox{\plotpoint}}
\put(1148.0,209.0){\usebox{\plotpoint}}
\put(1149.0,207.0){\rule[-0.200pt]{0.400pt}{0.482pt}}
\put(1149.0,207.0){\usebox{\plotpoint}}
\put(1150.0,206.0){\usebox{\plotpoint}}
\put(1150.0,206.0){\usebox{\plotpoint}}
\put(1151,203.67){\rule{0.241pt}{0.400pt}}
\multiput(1151.00,204.17)(0.500,-1.000){2}{\rule{0.120pt}{0.400pt}}
\put(1151.0,205.0){\usebox{\plotpoint}}
\put(1152,204){\usebox{\plotpoint}}
\put(1152,204){\usebox{\plotpoint}}
\put(1152,204){\usebox{\plotpoint}}
\put(1152,204){\usebox{\plotpoint}}
\put(1152.0,203.0){\usebox{\plotpoint}}
\put(1152.0,203.0){\usebox{\plotpoint}}
\put(1153.0,201.0){\rule[-0.200pt]{0.400pt}{0.482pt}}
\put(1153.0,201.0){\usebox{\plotpoint}}
\put(1154.0,200.0){\usebox{\plotpoint}}
\put(1154.0,200.0){\usebox{\plotpoint}}
\put(1155.0,198.0){\rule[-0.200pt]{0.400pt}{0.482pt}}
\put(1155.0,198.0){\usebox{\plotpoint}}
\put(1156.0,197.0){\usebox{\plotpoint}}
\put(1156.0,197.0){\usebox{\plotpoint}}
\put(1157.0,195.0){\rule[-0.200pt]{0.400pt}{0.482pt}}
\put(1157.0,195.0){\usebox{\plotpoint}}
\put(1158.0,194.0){\usebox{\plotpoint}}
\put(1158.0,194.0){\usebox{\plotpoint}}
\put(1159.0,193.0){\usebox{\plotpoint}}
\put(1159.0,193.0){\usebox{\plotpoint}}
\put(1160.0,191.0){\rule[-0.200pt]{0.400pt}{0.482pt}}
\put(1160.0,191.0){\usebox{\plotpoint}}
\put(1161.0,190.0){\usebox{\plotpoint}}
\put(1161.0,190.0){\usebox{\plotpoint}}
\put(1162.0,188.0){\rule[-0.200pt]{0.400pt}{0.482pt}}
\put(1162.0,188.0){\usebox{\plotpoint}}
\put(1163.0,187.0){\usebox{\plotpoint}}
\put(1163.0,187.0){\usebox{\plotpoint}}
\put(1164.0,186.0){\usebox{\plotpoint}}
\put(1164.0,186.0){\usebox{\plotpoint}}
\put(1165.0,184.0){\rule[-0.200pt]{0.400pt}{0.482pt}}
\put(1165.0,184.0){\usebox{\plotpoint}}
\put(1166.0,183.0){\usebox{\plotpoint}}
\put(1166.0,183.0){\usebox{\plotpoint}}
\put(1167.0,182.0){\usebox{\plotpoint}}
\put(1167.0,182.0){\usebox{\plotpoint}}
\put(1168,179.67){\rule{0.241pt}{0.400pt}}
\multiput(1168.00,180.17)(0.500,-1.000){2}{\rule{0.120pt}{0.400pt}}
\put(1168.0,181.0){\usebox{\plotpoint}}
\put(1169,180){\usebox{\plotpoint}}
\put(1169,180){\usebox{\plotpoint}}
\put(1169,180){\usebox{\plotpoint}}
\put(1169,180){\usebox{\plotpoint}}
\put(1169,180){\usebox{\plotpoint}}
\put(1169.0,179.0){\usebox{\plotpoint}}
\put(1169.0,179.0){\usebox{\plotpoint}}
\put(1170.0,178.0){\usebox{\plotpoint}}
\put(1170.0,178.0){\usebox{\plotpoint}}
\put(1171.0,177.0){\usebox{\plotpoint}}
\put(1171.0,177.0){\usebox{\plotpoint}}
\put(1172,174.67){\rule{0.241pt}{0.400pt}}
\multiput(1172.00,175.17)(0.500,-1.000){2}{\rule{0.120pt}{0.400pt}}
\put(1172.0,176.0){\usebox{\plotpoint}}
\put(1173,175){\usebox{\plotpoint}}
\put(1173,175){\usebox{\plotpoint}}
\put(1173,175){\usebox{\plotpoint}}
\put(1173,175){\usebox{\plotpoint}}
\put(1173,175){\usebox{\plotpoint}}
\put(1173.0,174.0){\usebox{\plotpoint}}
\put(1173.0,174.0){\usebox{\plotpoint}}
\put(1174.0,173.0){\usebox{\plotpoint}}
\put(1174.0,173.0){\usebox{\plotpoint}}
\put(1175.0,172.0){\usebox{\plotpoint}}
\put(1175.0,172.0){\usebox{\plotpoint}}
\put(1176.0,171.0){\usebox{\plotpoint}}
\put(1176.0,171.0){\usebox{\plotpoint}}
\put(1177.0,170.0){\usebox{\plotpoint}}
\put(1177.0,170.0){\usebox{\plotpoint}}
\put(1178,167.67){\rule{0.241pt}{0.400pt}}
\multiput(1178.00,168.17)(0.500,-1.000){2}{\rule{0.120pt}{0.400pt}}
\put(1178.0,169.0){\usebox{\plotpoint}}
\put(1179,168){\usebox{\plotpoint}}
\put(1179,168){\usebox{\plotpoint}}
\put(1179,168){\usebox{\plotpoint}}
\put(1179,168){\usebox{\plotpoint}}
\put(1179,168){\usebox{\plotpoint}}
\put(1179,168){\usebox{\plotpoint}}
\put(1179.0,167.0){\usebox{\plotpoint}}
\put(1179.0,167.0){\usebox{\plotpoint}}
\put(1180.0,166.0){\usebox{\plotpoint}}
\put(1180.0,166.0){\usebox{\plotpoint}}
\put(1181.0,165.0){\usebox{\plotpoint}}
\put(1181.0,165.0){\usebox{\plotpoint}}
\put(1182.0,164.0){\usebox{\plotpoint}}
\put(1182.0,164.0){\usebox{\plotpoint}}
\put(1183.0,163.0){\usebox{\plotpoint}}
\put(1183.0,163.0){\usebox{\plotpoint}}
\put(1184.0,162.0){\usebox{\plotpoint}}
\put(1184.0,162.0){\usebox{\plotpoint}}
\put(1185.0,161.0){\usebox{\plotpoint}}
\put(1185.0,161.0){\usebox{\plotpoint}}
\put(1186.0,160.0){\usebox{\plotpoint}}
\put(1186.0,160.0){\usebox{\plotpoint}}
\put(1187.0,159.0){\usebox{\plotpoint}}
\put(1187.0,159.0){\usebox{\plotpoint}}
\put(1188.0,158.0){\usebox{\plotpoint}}
\put(1188.0,158.0){\usebox{\plotpoint}}
\put(1189.0,157.0){\usebox{\plotpoint}}
\put(1189.0,157.0){\usebox{\plotpoint}}
\put(1190.0,156.0){\usebox{\plotpoint}}
\put(1190.0,156.0){\usebox{\plotpoint}}
\put(1191.0,155.0){\usebox{\plotpoint}}
\put(1191.0,155.0){\usebox{\plotpoint}}
\put(1192.0,154.0){\usebox{\plotpoint}}
\put(1193,152.67){\rule{0.241pt}{0.400pt}}
\multiput(1193.00,153.17)(0.500,-1.000){2}{\rule{0.120pt}{0.400pt}}
\put(1192.0,154.0){\usebox{\plotpoint}}
\put(1194,153){\usebox{\plotpoint}}
\put(1194,153){\usebox{\plotpoint}}
\put(1194,153){\usebox{\plotpoint}}
\put(1194,153){\usebox{\plotpoint}}
\put(1194,153){\usebox{\plotpoint}}
\put(1194,153){\usebox{\plotpoint}}
\put(1194.0,153.0){\usebox{\plotpoint}}
\put(1195.0,152.0){\usebox{\plotpoint}}
\put(1195.0,152.0){\usebox{\plotpoint}}
\put(1196.0,151.0){\usebox{\plotpoint}}
\put(1196.0,151.0){\usebox{\plotpoint}}
\put(1197.0,150.0){\usebox{\plotpoint}}
\put(1197.0,150.0){\usebox{\plotpoint}}
\put(1198.0,149.0){\usebox{\plotpoint}}
\put(1198.0,149.0){\rule[-0.200pt]{0.482pt}{0.400pt}}
\put(1200.0,148.0){\usebox{\plotpoint}}
\put(1200.0,148.0){\usebox{\plotpoint}}
\put(1201.0,147.0){\usebox{\plotpoint}}
\put(1201.0,147.0){\usebox{\plotpoint}}
\put(1202.0,146.0){\usebox{\plotpoint}}
\put(1202.0,146.0){\rule[-0.200pt]{0.482pt}{0.400pt}}
\put(1204.0,145.0){\usebox{\plotpoint}}
\put(1204.0,145.0){\usebox{\plotpoint}}
\put(1205.0,144.0){\usebox{\plotpoint}}
\put(1205.0,144.0){\rule[-0.200pt]{0.482pt}{0.400pt}}
\put(1207.0,143.0){\usebox{\plotpoint}}
\put(1207.0,143.0){\usebox{\plotpoint}}
\put(1208.0,142.0){\usebox{\plotpoint}}
\put(1208.0,142.0){\rule[-0.200pt]{0.482pt}{0.400pt}}
\put(1210.0,141.0){\usebox{\plotpoint}}
\put(1211,139.67){\rule{0.241pt}{0.400pt}}
\multiput(1211.00,140.17)(0.500,-1.000){2}{\rule{0.120pt}{0.400pt}}
\put(1210.0,141.0){\usebox{\plotpoint}}
\put(1212,140){\usebox{\plotpoint}}
\put(1212,140){\usebox{\plotpoint}}
\put(1212,140){\usebox{\plotpoint}}
\put(1212,140){\usebox{\plotpoint}}
\put(1212,140){\usebox{\plotpoint}}
\put(1212,140){\usebox{\plotpoint}}
\put(1212.0,140.0){\usebox{\plotpoint}}
\put(1213.0,139.0){\usebox{\plotpoint}}
\put(1213.0,139.0){\rule[-0.200pt]{0.482pt}{0.400pt}}
\put(1215.0,138.0){\usebox{\plotpoint}}
\put(1215.0,138.0){\rule[-0.200pt]{0.482pt}{0.400pt}}
\put(1217.0,137.0){\usebox{\plotpoint}}
\put(1219,135.67){\rule{0.241pt}{0.400pt}}
\multiput(1219.00,136.17)(0.500,-1.000){2}{\rule{0.120pt}{0.400pt}}
\put(1217.0,137.0){\rule[-0.200pt]{0.482pt}{0.400pt}}
\put(1220,136){\usebox{\plotpoint}}
\put(1220,136){\usebox{\plotpoint}}
\put(1220,136){\usebox{\plotpoint}}
\put(1220,136){\usebox{\plotpoint}}
\put(1220,136){\usebox{\plotpoint}}
\put(1220,136){\usebox{\plotpoint}}
\put(1220.0,136.0){\rule[-0.200pt]{0.482pt}{0.400pt}}
\put(1222.0,135.0){\usebox{\plotpoint}}
\put(1222.0,135.0){\rule[-0.200pt]{0.723pt}{0.400pt}}
\put(1225.0,134.0){\usebox{\plotpoint}}
\put(1225.0,134.0){\rule[-0.200pt]{0.723pt}{0.400pt}}
\put(1228.0,133.0){\usebox{\plotpoint}}
\put(1228.0,133.0){\rule[-0.200pt]{0.964pt}{0.400pt}}
\put(1232.0,132.0){\usebox{\plotpoint}}
\put(1232.0,132.0){\rule[-0.200pt]{1.204pt}{0.400pt}}
\put(1237.0,131.0){\usebox{\plotpoint}}
\put(1237.0,131.0){\rule[-0.200pt]{3.854pt}{0.400pt}}
\put(1253.0,131.0){\usebox{\plotpoint}}
\put(1253.0,132.0){\rule[-0.200pt]{1.445pt}{0.400pt}}
\put(1259.0,132.0){\usebox{\plotpoint}}
\put(1262,132.67){\rule{0.241pt}{0.400pt}}
\multiput(1262.00,132.17)(0.500,1.000){2}{\rule{0.120pt}{0.400pt}}
\put(1259.0,133.0){\rule[-0.200pt]{0.723pt}{0.400pt}}
\put(1263,134){\usebox{\plotpoint}}
\put(1263,134){\usebox{\plotpoint}}
\put(1263,134){\usebox{\plotpoint}}
\put(1263,134){\usebox{\plotpoint}}
\put(1263,134){\usebox{\plotpoint}}
\put(1263,134){\usebox{\plotpoint}}
\put(1263,134){\usebox{\plotpoint}}
\put(1263.0,134.0){\rule[-0.200pt]{0.723pt}{0.400pt}}
\put(1266.0,134.0){\usebox{\plotpoint}}
\put(1268,134.67){\rule{0.241pt}{0.400pt}}
\multiput(1268.00,134.17)(0.500,1.000){2}{\rule{0.120pt}{0.400pt}}
\put(1266.0,135.0){\rule[-0.200pt]{0.482pt}{0.400pt}}
\put(1269,136){\usebox{\plotpoint}}
\put(1269,136){\usebox{\plotpoint}}
\put(1269,136){\usebox{\plotpoint}}
\put(1269,136){\usebox{\plotpoint}}
\put(1269,136){\usebox{\plotpoint}}
\put(1269,136){\usebox{\plotpoint}}
\put(1269,136){\usebox{\plotpoint}}
\put(1269.0,136.0){\rule[-0.200pt]{0.482pt}{0.400pt}}
\put(1271.0,136.0){\usebox{\plotpoint}}
\put(1271.0,137.0){\rule[-0.200pt]{0.482pt}{0.400pt}}
\put(1273.0,137.0){\usebox{\plotpoint}}
\put(1273.0,138.0){\rule[-0.200pt]{0.482pt}{0.400pt}}
\put(1275.0,138.0){\usebox{\plotpoint}}
\put(1275.0,139.0){\rule[-0.200pt]{0.482pt}{0.400pt}}
\put(1277.0,139.0){\usebox{\plotpoint}}
\put(1277.0,140.0){\rule[-0.200pt]{0.482pt}{0.400pt}}
\put(1279.0,140.0){\usebox{\plotpoint}}
\put(1279.0,141.0){\rule[-0.200pt]{0.482pt}{0.400pt}}
\put(1281.0,141.0){\usebox{\plotpoint}}
\put(1281.0,142.0){\rule[-0.200pt]{0.482pt}{0.400pt}}
\put(1283.0,142.0){\usebox{\plotpoint}}
\put(1283.0,143.0){\usebox{\plotpoint}}
\put(1284.0,143.0){\usebox{\plotpoint}}
\put(1284.0,144.0){\rule[-0.200pt]{0.482pt}{0.400pt}}
\put(1286.0,144.0){\usebox{\plotpoint}}
\put(1286.0,145.0){\usebox{\plotpoint}}
\put(1287.0,145.0){\usebox{\plotpoint}}
\put(1287.0,146.0){\rule[-0.200pt]{0.482pt}{0.400pt}}
\put(1289.0,146.0){\usebox{\plotpoint}}
\put(1289.0,147.0){\usebox{\plotpoint}}
\put(1290.0,147.0){\usebox{\plotpoint}}
\put(1290.0,148.0){\rule[-0.200pt]{0.482pt}{0.400pt}}
\put(1292.0,148.0){\usebox{\plotpoint}}
\put(1292.0,149.0){\usebox{\plotpoint}}
\put(1293.0,149.0){\usebox{\plotpoint}}
\put(1293.0,150.0){\usebox{\plotpoint}}
\put(1294.0,150.0){\usebox{\plotpoint}}
\put(1294.0,151.0){\rule[-0.200pt]{0.482pt}{0.400pt}}
\put(1296.0,151.0){\usebox{\plotpoint}}
\put(1296.0,152.0){\usebox{\plotpoint}}
\put(1297.0,152.0){\usebox{\plotpoint}}
\put(1297.0,153.0){\usebox{\plotpoint}}
\put(1298.0,153.0){\usebox{\plotpoint}}
\put(1298.0,154.0){\usebox{\plotpoint}}
\put(1299.0,154.0){\usebox{\plotpoint}}
\put(1300,154.67){\rule{0.241pt}{0.400pt}}
\multiput(1300.00,154.17)(0.500,1.000){2}{\rule{0.120pt}{0.400pt}}
\put(1299.0,155.0){\usebox{\plotpoint}}
\put(1301,156){\usebox{\plotpoint}}
\put(1301,156){\usebox{\plotpoint}}
\put(1301,156){\usebox{\plotpoint}}
\put(1301,156){\usebox{\plotpoint}}
\put(1301,156){\usebox{\plotpoint}}
\put(1301,156){\usebox{\plotpoint}}
\put(1301,156){\usebox{\plotpoint}}
\put(1301.0,156.0){\usebox{\plotpoint}}
\put(1302.0,156.0){\usebox{\plotpoint}}
\put(1302.0,157.0){\usebox{\plotpoint}}
\put(1303.0,157.0){\usebox{\plotpoint}}
\put(1303.0,158.0){\usebox{\plotpoint}}
\put(1304.0,158.0){\usebox{\plotpoint}}
\put(1304.0,159.0){\usebox{\plotpoint}}
\put(1305.0,159.0){\usebox{\plotpoint}}
\put(1305.0,160.0){\usebox{\plotpoint}}
\put(1306.0,160.0){\usebox{\plotpoint}}
\put(1306.0,161.0){\usebox{\plotpoint}}
\put(1307.0,161.0){\usebox{\plotpoint}}
\put(1307.0,162.0){\usebox{\plotpoint}}
\put(1308.0,162.0){\usebox{\plotpoint}}
\put(1308.0,163.0){\usebox{\plotpoint}}
\put(1309.0,163.0){\usebox{\plotpoint}}
\put(1309.0,164.0){\usebox{\plotpoint}}
\put(1310.0,164.0){\usebox{\plotpoint}}
\put(1310.0,165.0){\usebox{\plotpoint}}
\put(1311.0,165.0){\usebox{\plotpoint}}
\put(1311.0,166.0){\usebox{\plotpoint}}
\put(1312.0,166.0){\usebox{\plotpoint}}
\put(1312.0,167.0){\usebox{\plotpoint}}
\put(1313.0,167.0){\usebox{\plotpoint}}
\put(1313.0,168.0){\usebox{\plotpoint}}
\put(1314.0,168.0){\usebox{\plotpoint}}
\put(1314.0,169.0){\usebox{\plotpoint}}
\put(1315.0,169.0){\usebox{\plotpoint}}
\put(1315.0,170.0){\usebox{\plotpoint}}
\put(1316.0,170.0){\usebox{\plotpoint}}
\put(1316.0,171.0){\usebox{\plotpoint}}
\put(1317.0,171.0){\usebox{\plotpoint}}
\put(1317.0,172.0){\usebox{\plotpoint}}
\put(1318.0,172.0){\usebox{\plotpoint}}
\put(1318.0,173.0){\usebox{\plotpoint}}
\put(1319.0,173.0){\usebox{\plotpoint}}
\put(1319.0,174.0){\usebox{\plotpoint}}
\put(1320,174.67){\rule{0.241pt}{0.400pt}}
\multiput(1320.00,174.17)(0.500,1.000){2}{\rule{0.120pt}{0.400pt}}
\put(1320.0,174.0){\usebox{\plotpoint}}
\put(1321,176){\usebox{\plotpoint}}
\put(1321,176){\usebox{\plotpoint}}
\put(1321,176){\usebox{\plotpoint}}
\put(1321,176){\usebox{\plotpoint}}
\put(1321,176){\usebox{\plotpoint}}
\put(1321,176){\usebox{\plotpoint}}
\put(1321.0,176.0){\usebox{\plotpoint}}
\put(1321.0,177.0){\usebox{\plotpoint}}
\put(1322.0,177.0){\usebox{\plotpoint}}
\put(1322.0,178.0){\usebox{\plotpoint}}
\put(1323.0,178.0){\usebox{\plotpoint}}
\put(1323.0,179.0){\usebox{\plotpoint}}
\put(1324.0,179.0){\usebox{\plotpoint}}
\put(1324.0,180.0){\usebox{\plotpoint}}
\put(1325.0,180.0){\usebox{\plotpoint}}
\put(1325.0,181.0){\usebox{\plotpoint}}
\put(1326,181.67){\rule{0.241pt}{0.400pt}}
\multiput(1326.00,181.17)(0.500,1.000){2}{\rule{0.120pt}{0.400pt}}
\put(1326.0,181.0){\usebox{\plotpoint}}
\put(1327,183){\usebox{\plotpoint}}
\put(1327,183){\usebox{\plotpoint}}
\put(1327,183){\usebox{\plotpoint}}
\put(1327,183){\usebox{\plotpoint}}
\put(1327,183){\usebox{\plotpoint}}
\put(1327,183){\usebox{\plotpoint}}
\put(1327.0,183.0){\usebox{\plotpoint}}
\put(1327.0,184.0){\usebox{\plotpoint}}
\put(1328.0,184.0){\usebox{\plotpoint}}
\put(1328.0,185.0){\usebox{\plotpoint}}
\put(1329.0,185.0){\usebox{\plotpoint}}
\put(1329.0,186.0){\usebox{\plotpoint}}
\put(1330,186.67){\rule{0.241pt}{0.400pt}}
\multiput(1330.00,186.17)(0.500,1.000){2}{\rule{0.120pt}{0.400pt}}
\put(1330.0,186.0){\usebox{\plotpoint}}
\put(1331,188){\usebox{\plotpoint}}
\put(1331,188){\usebox{\plotpoint}}
\put(1331,188){\usebox{\plotpoint}}
\put(1331,188){\usebox{\plotpoint}}
\put(1331,188){\usebox{\plotpoint}}
\put(1331.0,188.0){\usebox{\plotpoint}}
\put(1331.0,189.0){\usebox{\plotpoint}}
\put(1332.0,189.0){\usebox{\plotpoint}}
\put(1332.0,190.0){\usebox{\plotpoint}}
\put(1333.0,190.0){\usebox{\plotpoint}}
\put(1333.0,191.0){\usebox{\plotpoint}}
\put(1334,191.67){\rule{0.241pt}{0.400pt}}
\multiput(1334.00,191.17)(0.500,1.000){2}{\rule{0.120pt}{0.400pt}}
\put(1334.0,191.0){\usebox{\plotpoint}}
\put(1335,193){\usebox{\plotpoint}}
\put(1335,193){\usebox{\plotpoint}}
\put(1335,193){\usebox{\plotpoint}}
\put(1335,193){\usebox{\plotpoint}}
\put(1335,193){\usebox{\plotpoint}}
\put(1335.0,193.0){\usebox{\plotpoint}}
\put(1335.0,194.0){\usebox{\plotpoint}}
\put(1336.0,194.0){\usebox{\plotpoint}}
\put(1336.0,195.0){\usebox{\plotpoint}}
\put(1337,195.67){\rule{0.241pt}{0.400pt}}
\multiput(1337.00,195.17)(0.500,1.000){2}{\rule{0.120pt}{0.400pt}}
\put(1337.0,195.0){\usebox{\plotpoint}}
\put(1338,197){\usebox{\plotpoint}}
\put(1338,197){\usebox{\plotpoint}}
\put(1338,197){\usebox{\plotpoint}}
\put(1338,197){\usebox{\plotpoint}}
\put(1338,197){\usebox{\plotpoint}}
\put(1338.0,197.0){\usebox{\plotpoint}}
\put(1338.0,198.0){\usebox{\plotpoint}}
\put(1339.0,198.0){\usebox{\plotpoint}}
\put(1339.0,199.0){\usebox{\plotpoint}}
\put(1340.0,199.0){\rule[-0.200pt]{0.400pt}{0.482pt}}
\put(1340.0,201.0){\usebox{\plotpoint}}
\put(1341.0,201.0){\usebox{\plotpoint}}
\put(1341.0,202.0){\usebox{\plotpoint}}
\put(1342.0,202.0){\usebox{\plotpoint}}
\put(1342.0,203.0){\usebox{\plotpoint}}
\put(1343.0,203.0){\rule[-0.200pt]{0.400pt}{0.482pt}}
\put(1343.0,205.0){\usebox{\plotpoint}}
\put(1344.0,205.0){\usebox{\plotpoint}}
\put(1344.0,206.0){\usebox{\plotpoint}}
\put(1345,206.67){\rule{0.241pt}{0.400pt}}
\multiput(1345.00,206.17)(0.500,1.000){2}{\rule{0.120pt}{0.400pt}}
\put(1345.0,206.0){\usebox{\plotpoint}}
\put(1346,208){\usebox{\plotpoint}}
\put(1346,208){\usebox{\plotpoint}}
\put(1346,208){\usebox{\plotpoint}}
\put(1346,208){\usebox{\plotpoint}}
\put(1346,208){\usebox{\plotpoint}}
\put(1346.0,208.0){\usebox{\plotpoint}}
\put(1346.0,209.0){\usebox{\plotpoint}}
\put(1347.0,209.0){\usebox{\plotpoint}}
\put(1347.0,210.0){\usebox{\plotpoint}}
\put(1348.0,210.0){\rule[-0.200pt]{0.400pt}{0.482pt}}
\put(1348.0,212.0){\usebox{\plotpoint}}
\put(1349.0,212.0){\usebox{\plotpoint}}
\put(1349.0,213.0){\usebox{\plotpoint}}
\put(1350.0,213.0){\rule[-0.200pt]{0.400pt}{0.482pt}}
\put(1350.0,215.0){\usebox{\plotpoint}}
\put(1351.0,215.0){\usebox{\plotpoint}}
\put(1351.0,216.0){\usebox{\plotpoint}}
\put(1352.0,216.0){\rule[-0.200pt]{0.400pt}{0.482pt}}
\put(1352.0,218.0){\usebox{\plotpoint}}
\put(1353.0,218.0){\usebox{\plotpoint}}
\put(1353.0,219.0){\usebox{\plotpoint}}
\put(1354.0,219.0){\rule[-0.200pt]{0.400pt}{0.482pt}}
\put(1354.0,221.0){\usebox{\plotpoint}}
\put(1355.0,221.0){\usebox{\plotpoint}}
\put(1355.0,222.0){\usebox{\plotpoint}}
\put(1356.0,222.0){\rule[-0.200pt]{0.400pt}{0.482pt}}
\put(1356.0,224.0){\usebox{\plotpoint}}
\put(1357.0,224.0){\usebox{\plotpoint}}
\put(1357.0,225.0){\usebox{\plotpoint}}
\put(1358.0,225.0){\rule[-0.200pt]{0.400pt}{0.482pt}}
\put(1358.0,227.0){\usebox{\plotpoint}}
\put(1359.0,227.0){\usebox{\plotpoint}}
\put(1359.0,228.0){\usebox{\plotpoint}}
\put(1360.0,228.0){\rule[-0.200pt]{0.400pt}{0.482pt}}
\put(1360.0,230.0){\usebox{\plotpoint}}
\put(1361.0,230.0){\usebox{\plotpoint}}
\put(1361.0,231.0){\usebox{\plotpoint}}
\put(1362.0,231.0){\rule[-0.200pt]{0.400pt}{0.482pt}}
\put(1362.0,233.0){\usebox{\plotpoint}}
\put(1363,233.67){\rule{0.241pt}{0.400pt}}
\multiput(1363.00,233.17)(0.500,1.000){2}{\rule{0.120pt}{0.400pt}}
\put(1363.0,233.0){\usebox{\plotpoint}}
\put(1364,235){\usebox{\plotpoint}}
\put(1364,235){\usebox{\plotpoint}}
\put(1364,235){\usebox{\plotpoint}}
\put(1364,235){\usebox{\plotpoint}}
\put(1364.0,235.0){\usebox{\plotpoint}}
\put(1364.0,236.0){\usebox{\plotpoint}}
\put(1365.0,236.0){\rule[-0.200pt]{0.400pt}{0.482pt}}
\put(1365.0,238.0){\usebox{\plotpoint}}
\put(1366,238.67){\rule{0.241pt}{0.400pt}}
\multiput(1366.00,238.17)(0.500,1.000){2}{\rule{0.120pt}{0.400pt}}
\put(1366.0,238.0){\usebox{\plotpoint}}
\put(1367,240){\usebox{\plotpoint}}
\put(1367,240){\usebox{\plotpoint}}
\put(1367,240){\usebox{\plotpoint}}
\put(1367,240){\usebox{\plotpoint}}
\put(1367.0,240.0){\usebox{\plotpoint}}
\put(1367.0,241.0){\usebox{\plotpoint}}
\put(1368.0,241.0){\rule[-0.200pt]{0.400pt}{0.482pt}}
\put(1368.0,243.0){\usebox{\plotpoint}}
\put(1369.0,243.0){\usebox{\plotpoint}}
\put(1369.0,244.0){\usebox{\plotpoint}}
\put(1370.0,244.0){\rule[-0.200pt]{0.400pt}{0.482pt}}
\put(1370.0,246.0){\usebox{\plotpoint}}
\put(1371,246.67){\rule{0.241pt}{0.400pt}}
\multiput(1371.00,246.17)(0.500,1.000){2}{\rule{0.120pt}{0.400pt}}
\put(1371.0,246.0){\usebox{\plotpoint}}
\put(1372,248){\usebox{\plotpoint}}
\put(1372,248){\usebox{\plotpoint}}
\put(1372,248){\usebox{\plotpoint}}
\put(1372,248){\usebox{\plotpoint}}
\put(1372.0,248.0){\usebox{\plotpoint}}
\put(1372.0,249.0){\usebox{\plotpoint}}
\put(1373.0,249.0){\rule[-0.200pt]{0.400pt}{0.482pt}}
\put(1373.0,251.0){\usebox{\plotpoint}}
\put(1374,251.67){\rule{0.241pt}{0.400pt}}
\multiput(1374.00,251.17)(0.500,1.000){2}{\rule{0.120pt}{0.400pt}}
\put(1374.0,251.0){\usebox{\plotpoint}}
\put(1375,253){\usebox{\plotpoint}}
\put(1375,253){\usebox{\plotpoint}}
\put(1375,253){\usebox{\plotpoint}}
\put(1375.0,253.0){\usebox{\plotpoint}}
\put(1375.0,254.0){\usebox{\plotpoint}}
\put(1376.0,254.0){\rule[-0.200pt]{0.400pt}{0.482pt}}
\put(1376.0,256.0){\usebox{\plotpoint}}
\put(1377.0,256.0){\rule[-0.200pt]{0.400pt}{0.482pt}}
\put(1377.0,258.0){\usebox{\plotpoint}}
\put(1378.0,258.0){\usebox{\plotpoint}}
\put(1378.0,259.0){\usebox{\plotpoint}}
\put(1379.0,259.0){\rule[-0.200pt]{0.400pt}{0.482pt}}
\put(1379.0,261.0){\usebox{\plotpoint}}
\put(1380.0,261.0){\rule[-0.200pt]{0.400pt}{0.482pt}}
\put(1380.0,263.0){\usebox{\plotpoint}}
\put(1381,263.67){\rule{0.241pt}{0.400pt}}
\multiput(1381.00,263.17)(0.500,1.000){2}{\rule{0.120pt}{0.400pt}}
\put(1381.0,263.0){\usebox{\plotpoint}}
\put(1382,265){\usebox{\plotpoint}}
\put(1382,265){\usebox{\plotpoint}}
\put(1382,265){\usebox{\plotpoint}}
\put(1382,265){\usebox{\plotpoint}}
\put(1382.0,265.0){\usebox{\plotpoint}}
\put(1382.0,266.0){\usebox{\plotpoint}}
\put(1383.0,266.0){\rule[-0.200pt]{0.400pt}{0.482pt}}
\put(1383.0,268.0){\usebox{\plotpoint}}
\put(1384.0,268.0){\rule[-0.200pt]{0.400pt}{0.482pt}}
\put(1384.0,270.0){\usebox{\plotpoint}}
\put(1385,270.67){\rule{0.241pt}{0.400pt}}
\multiput(1385.00,270.17)(0.500,1.000){2}{\rule{0.120pt}{0.400pt}}
\put(1385.0,270.0){\usebox{\plotpoint}}
\put(1386,272){\usebox{\plotpoint}}
\put(1386,272){\usebox{\plotpoint}}
\put(1386,272){\usebox{\plotpoint}}
\put(1386,272){\usebox{\plotpoint}}
\put(1386.0,272.0){\usebox{\plotpoint}}
\put(1386.0,273.0){\usebox{\plotpoint}}
\put(1387.0,273.0){\rule[-0.200pt]{0.400pt}{0.482pt}}
\put(1387.0,275.0){\usebox{\plotpoint}}
\put(1388,275.67){\rule{0.241pt}{0.400pt}}
\multiput(1388.00,275.17)(0.500,1.000){2}{\rule{0.120pt}{0.400pt}}
\put(1388.0,275.0){\usebox{\plotpoint}}
\put(1389,277){\usebox{\plotpoint}}
\put(1389,277){\usebox{\plotpoint}}
\put(1389,277){\usebox{\plotpoint}}
\put(1389.0,277.0){\usebox{\plotpoint}}
\put(1389.0,278.0){\usebox{\plotpoint}}
\put(1390.0,278.0){\rule[-0.200pt]{0.400pt}{0.482pt}}
\put(1390.0,280.0){\usebox{\plotpoint}}
\put(1391.0,280.0){\rule[-0.200pt]{0.400pt}{0.482pt}}
\put(1391.0,282.0){\usebox{\plotpoint}}
\put(1392.0,282.0){\rule[-0.200pt]{0.400pt}{0.482pt}}
\put(1392.0,284.0){\usebox{\plotpoint}}
\put(1393,284.67){\rule{0.241pt}{0.400pt}}
\multiput(1393.00,284.17)(0.500,1.000){2}{\rule{0.120pt}{0.400pt}}
\put(1393.0,284.0){\usebox{\plotpoint}}
\put(1394,286){\usebox{\plotpoint}}
\put(1394,286){\usebox{\plotpoint}}
\put(1394,286){\usebox{\plotpoint}}
\put(1394,286.67){\rule{0.241pt}{0.400pt}}
\multiput(1394.00,286.17)(0.500,1.000){2}{\rule{0.120pt}{0.400pt}}
\put(1394.0,286.0){\usebox{\plotpoint}}
\put(1395,288){\usebox{\plotpoint}}
\put(1395,288){\usebox{\plotpoint}}
\put(1395,288){\usebox{\plotpoint}}
\put(1395,288){\usebox{\plotpoint}}
\put(1395.0,288.0){\usebox{\plotpoint}}
\put(1395.0,289.0){\usebox{\plotpoint}}
\put(1396.0,289.0){\rule[-0.200pt]{0.400pt}{0.482pt}}
\put(1396.0,291.0){\usebox{\plotpoint}}
\put(1397.0,291.0){\rule[-0.200pt]{0.400pt}{0.482pt}}
\put(1397.0,293.0){\usebox{\plotpoint}}
\put(1398,293.67){\rule{0.241pt}{0.400pt}}
\multiput(1398.00,293.17)(0.500,1.000){2}{\rule{0.120pt}{0.400pt}}
\put(1398.0,293.0){\usebox{\plotpoint}}
\put(1399,295){\usebox{\plotpoint}}
\put(1399,295){\usebox{\plotpoint}}
\put(1399,295){\usebox{\plotpoint}}
\put(1399,295.67){\rule{0.241pt}{0.400pt}}
\multiput(1399.00,295.17)(0.500,1.000){2}{\rule{0.120pt}{0.400pt}}
\put(1399.0,295.0){\usebox{\plotpoint}}
\put(1400,297){\usebox{\plotpoint}}
\put(1400,297){\usebox{\plotpoint}}
\put(1400,297){\usebox{\plotpoint}}
\put(1400,297){\usebox{\plotpoint}}
\put(1400.0,297.0){\usebox{\plotpoint}}
\put(1400.0,298.0){\usebox{\plotpoint}}
\put(1401.0,298.0){\rule[-0.200pt]{0.400pt}{0.482pt}}
\put(1401.0,300.0){\usebox{\plotpoint}}
\put(1402.0,300.0){\rule[-0.200pt]{0.400pt}{0.482pt}}
\put(1402.0,302.0){\usebox{\plotpoint}}
\put(1403.0,302.0){\rule[-0.200pt]{0.400pt}{0.482pt}}
\put(1403.0,304.0){\usebox{\plotpoint}}
\put(1404,304.67){\rule{0.241pt}{0.400pt}}
\multiput(1404.00,304.17)(0.500,1.000){2}{\rule{0.120pt}{0.400pt}}
\put(1404.0,304.0){\usebox{\plotpoint}}
\put(1405,306){\usebox{\plotpoint}}
\put(1405,306){\usebox{\plotpoint}}
\put(1405,306){\usebox{\plotpoint}}
\put(1405.0,306.0){\usebox{\plotpoint}}
\put(1405.0,307.0){\usebox{\plotpoint}}
\put(1406.0,307.0){\rule[-0.200pt]{0.400pt}{0.482pt}}
\put(1406.0,309.0){\usebox{\plotpoint}}
\put(1407.0,309.0){\rule[-0.200pt]{0.400pt}{0.482pt}}
\put(1407.0,311.0){\usebox{\plotpoint}}
\put(1408.0,311.0){\rule[-0.200pt]{0.400pt}{0.482pt}}
\put(1408.0,313.0){\usebox{\plotpoint}}
\put(1409.0,313.0){\rule[-0.200pt]{0.400pt}{0.482pt}}
\put(1409.0,315.0){\usebox{\plotpoint}}
\put(1410.0,315.0){\rule[-0.200pt]{0.400pt}{0.482pt}}
\put(1410.0,317.0){\usebox{\plotpoint}}
\put(1411.0,317.0){\rule[-0.200pt]{0.400pt}{0.482pt}}
\put(1411.0,319.0){\usebox{\plotpoint}}
\put(1412,319.67){\rule{0.241pt}{0.400pt}}
\multiput(1412.00,319.17)(0.500,1.000){2}{\rule{0.120pt}{0.400pt}}
\put(1412.0,319.0){\usebox{\plotpoint}}
\put(1413,321){\usebox{\plotpoint}}
\put(1413,321){\usebox{\plotpoint}}
\put(1413,321){\usebox{\plotpoint}}
\put(1413.0,321.0){\usebox{\plotpoint}}
\put(1413.0,322.0){\usebox{\plotpoint}}
\put(1414.0,322.0){\rule[-0.200pt]{0.400pt}{0.482pt}}
\put(1414.0,324.0){\usebox{\plotpoint}}
\put(1415.0,324.0){\rule[-0.200pt]{0.400pt}{0.482pt}}
\put(1415.0,326.0){\usebox{\plotpoint}}
\put(1416.0,326.0){\rule[-0.200pt]{0.400pt}{0.482pt}}
\put(1416.0,328.0){\usebox{\plotpoint}}
\put(1417.0,328.0){\rule[-0.200pt]{0.400pt}{0.482pt}}
\put(1417.0,330.0){\usebox{\plotpoint}}
\put(1418.0,330.0){\rule[-0.200pt]{0.400pt}{0.482pt}}
\put(1418.0,332.0){\usebox{\plotpoint}}
\put(1419.0,332.0){\rule[-0.200pt]{0.400pt}{0.482pt}}
\put(1419.0,334.0){\usebox{\plotpoint}}
\put(1420.0,334.0){\rule[-0.200pt]{0.400pt}{0.482pt}}
\put(1420.0,336.0){\usebox{\plotpoint}}
\put(1421,336.67){\rule{0.241pt}{0.400pt}}
\multiput(1421.00,336.17)(0.500,1.000){2}{\rule{0.120pt}{0.400pt}}
\put(1421.0,336.0){\usebox{\plotpoint}}
\put(1422,338){\usebox{\plotpoint}}
\put(1422,338){\usebox{\plotpoint}}
\put(1422,338){\usebox{\plotpoint}}
\put(1422.0,338.0){\usebox{\plotpoint}}
\put(1422.0,339.0){\usebox{\plotpoint}}
\put(1423.0,339.0){\rule[-0.200pt]{0.400pt}{0.482pt}}
\put(1423.0,341.0){\usebox{\plotpoint}}
\put(1424.0,341.0){\rule[-0.200pt]{0.400pt}{0.482pt}}
\put(1424.0,343.0){\usebox{\plotpoint}}
\put(1425.0,343.0){\rule[-0.200pt]{0.400pt}{0.482pt}}
\put(1425.0,345.0){\usebox{\plotpoint}}
\put(1426.0,345.0){\rule[-0.200pt]{0.400pt}{0.482pt}}
\put(1426.0,347.0){\usebox{\plotpoint}}
\put(1427.0,347.0){\rule[-0.200pt]{0.400pt}{0.482pt}}
\put(1427.0,349.0){\usebox{\plotpoint}}
\put(1428.0,349.0){\rule[-0.200pt]{0.400pt}{0.482pt}}
\put(1428.0,351.0){\usebox{\plotpoint}}
\put(1429.0,351.0){\rule[-0.200pt]{0.400pt}{0.482pt}}
\put(1429.0,353.0){\usebox{\plotpoint}}
\put(1430,353.67){\rule{0.241pt}{0.400pt}}
\multiput(1430.00,353.17)(0.500,1.000){2}{\rule{0.120pt}{0.400pt}}
\put(1430.0,353.0){\usebox{\plotpoint}}
\put(1431,355){\usebox{\plotpoint}}
\put(1431,355){\usebox{\plotpoint}}
\put(1431,355){\usebox{\plotpoint}}
\put(1431,355.67){\rule{0.241pt}{0.400pt}}
\multiput(1431.00,355.17)(0.500,1.000){2}{\rule{0.120pt}{0.400pt}}
\put(1431.0,355.0){\usebox{\plotpoint}}
\put(1432,357){\usebox{\plotpoint}}
\put(1432,357){\usebox{\plotpoint}}
\put(1432,357){\usebox{\plotpoint}}
\put(1432.0,357.0){\usebox{\plotpoint}}
\put(1432.0,358.0){\usebox{\plotpoint}}
\put(1433.0,358.0){\rule[-0.200pt]{0.400pt}{0.482pt}}
\put(1433.0,360.0){\usebox{\plotpoint}}
\put(1434.0,360.0){\rule[-0.200pt]{0.400pt}{0.482pt}}
\put(1434.0,362.0){\usebox{\plotpoint}}
\put(1435.0,362.0){\rule[-0.200pt]{0.400pt}{0.482pt}}
\put(1435.0,364.0){\usebox{\plotpoint}}
\put(1436.0,364.0){\rule[-0.200pt]{0.400pt}{0.482pt}}
\put(1436.0,366.0){\usebox{\plotpoint}}
\put(1437.0,366.0){\rule[-0.200pt]{0.400pt}{0.482pt}}
\put(1437.0,368.0){\usebox{\plotpoint}}
\put(1438.0,368.0){\rule[-0.200pt]{0.400pt}{0.482pt}}
\put(1438.0,370.0){\usebox{\plotpoint}}
\put(1439.0,370.0){\usebox{\plotpoint}}
\put(1279,778){\makebox(0,0)[r]{Namerné hodnoty}}
\put(505,263){\makebox(0,0){$\times$}}
\put(805,859){\makebox(0,0){$\times$}}
\put(1349,778){\makebox(0,0){$\times$}}
\put(171.0,131.0){\rule[-0.200pt]{0.400pt}{175.375pt}}
\put(171.0,131.0){\rule[-0.200pt]{305.461pt}{0.400pt}}
\put(1439.0,131.0){\rule[-0.200pt]{0.400pt}{175.375pt}}
\put(171.0,859.0){\rule[-0.200pt]{305.461pt}{0.400pt}}
\end{picture}

\caption{Difrakčný obrazec pre 2 štrbiny, v porovnaní s teoretickou závislosťou. Kde $U/U_0$ je relatívny úbytok napätia a $\theta$ je pozorovaný uhol.}  \label{G_D2}
\end{figure}

\begin{figure}
% GNUPLOT: LaTeX picture
\setlength{\unitlength}{0.240900pt}
\ifx\plotpoint\undefined\newsavebox{\plotpoint}\fi
\begin{picture}(1500,900)(0,0)
\sbox{\plotpoint}{\rule[-0.200pt]{0.400pt}{0.400pt}}%
\put(171.0,131.0){\rule[-0.200pt]{4.818pt}{0.400pt}}
\put(151,131){\makebox(0,0)[r]{ 0}}
\put(1419.0,131.0){\rule[-0.200pt]{4.818pt}{0.400pt}}
\put(171.0,204.0){\rule[-0.200pt]{4.818pt}{0.400pt}}
\put(151,204){\makebox(0,0)[r]{ 0.1}}
\put(1419.0,204.0){\rule[-0.200pt]{4.818pt}{0.400pt}}
\put(171.0,277.0){\rule[-0.200pt]{4.818pt}{0.400pt}}
\put(151,277){\makebox(0,0)[r]{ 0.2}}
\put(1419.0,277.0){\rule[-0.200pt]{4.818pt}{0.400pt}}
\put(171.0,349.0){\rule[-0.200pt]{4.818pt}{0.400pt}}
\put(151,349){\makebox(0,0)[r]{ 0.3}}
\put(1419.0,349.0){\rule[-0.200pt]{4.818pt}{0.400pt}}
\put(171.0,422.0){\rule[-0.200pt]{4.818pt}{0.400pt}}
\put(151,422){\makebox(0,0)[r]{ 0.4}}
\put(1419.0,422.0){\rule[-0.200pt]{4.818pt}{0.400pt}}
\put(171.0,495.0){\rule[-0.200pt]{4.818pt}{0.400pt}}
\put(151,495){\makebox(0,0)[r]{ 0.5}}
\put(1419.0,495.0){\rule[-0.200pt]{4.818pt}{0.400pt}}
\put(171.0,568.0){\rule[-0.200pt]{4.818pt}{0.400pt}}
\put(151,568){\makebox(0,0)[r]{ 0.6}}
\put(1419.0,568.0){\rule[-0.200pt]{4.818pt}{0.400pt}}
\put(171.0,641.0){\rule[-0.200pt]{4.818pt}{0.400pt}}
\put(151,641){\makebox(0,0)[r]{ 0.7}}
\put(1419.0,641.0){\rule[-0.200pt]{4.818pt}{0.400pt}}
\put(171.0,713.0){\rule[-0.200pt]{4.818pt}{0.400pt}}
\put(151,713){\makebox(0,0)[r]{ 0.8}}
\put(1419.0,713.0){\rule[-0.200pt]{4.818pt}{0.400pt}}
\put(171.0,786.0){\rule[-0.200pt]{4.818pt}{0.400pt}}
\put(151,786){\makebox(0,0)[r]{ 0.9}}
\put(1419.0,786.0){\rule[-0.200pt]{4.818pt}{0.400pt}}
\put(171.0,859.0){\rule[-0.200pt]{4.818pt}{0.400pt}}
\put(151,859){\makebox(0,0)[r]{ 1}}
\put(1419.0,859.0){\rule[-0.200pt]{4.818pt}{0.400pt}}
\put(171.0,131.0){\rule[-0.200pt]{0.400pt}{4.818pt}}
\put(171,90){\makebox(0,0){-0.2}}
\put(171.0,839.0){\rule[-0.200pt]{0.400pt}{4.818pt}}
\put(330.0,131.0){\rule[-0.200pt]{0.400pt}{4.818pt}}
\put(330,90){\makebox(0,0){-0.15}}
\put(330.0,839.0){\rule[-0.200pt]{0.400pt}{4.818pt}}
\put(488.0,131.0){\rule[-0.200pt]{0.400pt}{4.818pt}}
\put(488,90){\makebox(0,0){-0.1}}
\put(488.0,839.0){\rule[-0.200pt]{0.400pt}{4.818pt}}
\put(647.0,131.0){\rule[-0.200pt]{0.400pt}{4.818pt}}
\put(647,90){\makebox(0,0){-0.05}}
\put(647.0,839.0){\rule[-0.200pt]{0.400pt}{4.818pt}}
\put(805.0,131.0){\rule[-0.200pt]{0.400pt}{4.818pt}}
\put(805,90){\makebox(0,0){ 0}}
\put(805.0,839.0){\rule[-0.200pt]{0.400pt}{4.818pt}}
\put(964.0,131.0){\rule[-0.200pt]{0.400pt}{4.818pt}}
\put(964,90){\makebox(0,0){ 0.05}}
\put(964.0,839.0){\rule[-0.200pt]{0.400pt}{4.818pt}}
\put(1122.0,131.0){\rule[-0.200pt]{0.400pt}{4.818pt}}
\put(1122,90){\makebox(0,0){ 0.1}}
\put(1122.0,839.0){\rule[-0.200pt]{0.400pt}{4.818pt}}
\put(1281.0,131.0){\rule[-0.200pt]{0.400pt}{4.818pt}}
\put(1281,90){\makebox(0,0){ 0.15}}
\put(1281.0,839.0){\rule[-0.200pt]{0.400pt}{4.818pt}}
\put(1439.0,131.0){\rule[-0.200pt]{0.400pt}{4.818pt}}
\put(1439,90){\makebox(0,0){ 0.2}}
\put(1439.0,839.0){\rule[-0.200pt]{0.400pt}{4.818pt}}
\put(171.0,131.0){\rule[-0.200pt]{0.400pt}{175.375pt}}
\put(171.0,131.0){\rule[-0.200pt]{305.461pt}{0.400pt}}
\put(1439.0,131.0){\rule[-0.200pt]{0.400pt}{175.375pt}}
\put(171.0,859.0){\rule[-0.200pt]{305.461pt}{0.400pt}}
\put(30,495){\makebox(0,0){\popi{U/U_0}{-}}}
\put(805,29){\makebox(0,0){\popi{\theta}{rad}}}
\put(1279,819){\makebox(0,0)[r]{Teoretická zavislosť}}
\put(1299.0,819.0){\rule[-0.200pt]{24.090pt}{0.400pt}}
\put(171,168){\usebox{\plotpoint}}
\put(171,168){\usebox{\plotpoint}}
\put(171,168){\usebox{\plotpoint}}
\put(171,168){\usebox{\plotpoint}}
\put(171.0,168.0){\rule[-0.200pt]{0.964pt}{0.400pt}}
\put(175.0,168.0){\usebox{\plotpoint}}
\put(175.0,169.0){\rule[-0.200pt]{2.168pt}{0.400pt}}
\put(184.0,168.0){\usebox{\plotpoint}}
\put(184.0,168.0){\rule[-0.200pt]{1.445pt}{0.400pt}}
\put(190.0,167.0){\usebox{\plotpoint}}
\put(193,165.67){\rule{0.241pt}{0.400pt}}
\multiput(193.00,166.17)(0.500,-1.000){2}{\rule{0.120pt}{0.400pt}}
\put(190.0,167.0){\rule[-0.200pt]{0.723pt}{0.400pt}}
\put(194,166){\usebox{\plotpoint}}
\put(194,166){\usebox{\plotpoint}}
\put(194,166){\usebox{\plotpoint}}
\put(194,166){\usebox{\plotpoint}}
\put(194,166){\usebox{\plotpoint}}
\put(194,166){\usebox{\plotpoint}}
\put(194,166){\usebox{\plotpoint}}
\put(194.0,166.0){\rule[-0.200pt]{0.723pt}{0.400pt}}
\put(197.0,165.0){\usebox{\plotpoint}}
\put(197.0,165.0){\rule[-0.200pt]{0.482pt}{0.400pt}}
\put(199.0,164.0){\usebox{\plotpoint}}
\put(199.0,164.0){\rule[-0.200pt]{0.723pt}{0.400pt}}
\put(202.0,163.0){\usebox{\plotpoint}}
\put(202.0,163.0){\rule[-0.200pt]{0.482pt}{0.400pt}}
\put(204.0,162.0){\usebox{\plotpoint}}
\put(204.0,162.0){\rule[-0.200pt]{0.482pt}{0.400pt}}
\put(206.0,161.0){\usebox{\plotpoint}}
\put(206.0,161.0){\rule[-0.200pt]{0.482pt}{0.400pt}}
\put(208.0,160.0){\usebox{\plotpoint}}
\put(208.0,160.0){\rule[-0.200pt]{0.482pt}{0.400pt}}
\put(210.0,159.0){\usebox{\plotpoint}}
\put(210.0,159.0){\rule[-0.200pt]{0.482pt}{0.400pt}}
\put(212.0,158.0){\usebox{\plotpoint}}
\put(213,156.67){\rule{0.241pt}{0.400pt}}
\multiput(213.00,157.17)(0.500,-1.000){2}{\rule{0.120pt}{0.400pt}}
\put(212.0,158.0){\usebox{\plotpoint}}
\put(214,157){\usebox{\plotpoint}}
\put(214,157){\usebox{\plotpoint}}
\put(214,157){\usebox{\plotpoint}}
\put(214,157){\usebox{\plotpoint}}
\put(214,157){\usebox{\plotpoint}}
\put(214,157){\usebox{\plotpoint}}
\put(214,157){\usebox{\plotpoint}}
\put(214.0,157.0){\usebox{\plotpoint}}
\put(215.0,156.0){\usebox{\plotpoint}}
\put(215.0,156.0){\rule[-0.200pt]{0.482pt}{0.400pt}}
\put(217.0,155.0){\usebox{\plotpoint}}
\put(217.0,155.0){\rule[-0.200pt]{0.482pt}{0.400pt}}
\put(219.0,154.0){\usebox{\plotpoint}}
\put(219.0,154.0){\usebox{\plotpoint}}
\put(220.0,153.0){\usebox{\plotpoint}}
\put(220.0,153.0){\rule[-0.200pt]{0.482pt}{0.400pt}}
\put(222.0,152.0){\usebox{\plotpoint}}
\put(222.0,152.0){\rule[-0.200pt]{0.482pt}{0.400pt}}
\put(224.0,151.0){\usebox{\plotpoint}}
\put(224.0,151.0){\usebox{\plotpoint}}
\put(225.0,150.0){\usebox{\plotpoint}}
\put(225.0,150.0){\rule[-0.200pt]{0.482pt}{0.400pt}}
\put(227.0,149.0){\usebox{\plotpoint}}
\put(228,147.67){\rule{0.241pt}{0.400pt}}
\multiput(228.00,148.17)(0.500,-1.000){2}{\rule{0.120pt}{0.400pt}}
\put(227.0,149.0){\usebox{\plotpoint}}
\put(229,148){\usebox{\plotpoint}}
\put(229,148){\usebox{\plotpoint}}
\put(229,148){\usebox{\plotpoint}}
\put(229,148){\usebox{\plotpoint}}
\put(229,148){\usebox{\plotpoint}}
\put(229,148){\usebox{\plotpoint}}
\put(229,148){\usebox{\plotpoint}}
\put(229.0,148.0){\usebox{\plotpoint}}
\put(230.0,147.0){\usebox{\plotpoint}}
\put(230.0,147.0){\rule[-0.200pt]{0.482pt}{0.400pt}}
\put(232.0,146.0){\usebox{\plotpoint}}
\put(233,144.67){\rule{0.241pt}{0.400pt}}
\multiput(233.00,145.17)(0.500,-1.000){2}{\rule{0.120pt}{0.400pt}}
\put(232.0,146.0){\usebox{\plotpoint}}
\put(234,145){\usebox{\plotpoint}}
\put(234,145){\usebox{\plotpoint}}
\put(234,145){\usebox{\plotpoint}}
\put(234,145){\usebox{\plotpoint}}
\put(234,145){\usebox{\plotpoint}}
\put(234,145){\usebox{\plotpoint}}
\put(234,145){\usebox{\plotpoint}}
\put(234.0,145.0){\usebox{\plotpoint}}
\put(235.0,144.0){\usebox{\plotpoint}}
\put(235.0,144.0){\rule[-0.200pt]{0.482pt}{0.400pt}}
\put(237.0,143.0){\usebox{\plotpoint}}
\put(237.0,143.0){\rule[-0.200pt]{0.482pt}{0.400pt}}
\put(239.0,142.0){\usebox{\plotpoint}}
\put(240,140.67){\rule{0.241pt}{0.400pt}}
\multiput(240.00,141.17)(0.500,-1.000){2}{\rule{0.120pt}{0.400pt}}
\put(239.0,142.0){\usebox{\plotpoint}}
\put(241,141){\usebox{\plotpoint}}
\put(241,141){\usebox{\plotpoint}}
\put(241,141){\usebox{\plotpoint}}
\put(241,141){\usebox{\plotpoint}}
\put(241,141){\usebox{\plotpoint}}
\put(241,141){\usebox{\plotpoint}}
\put(242,139.67){\rule{0.241pt}{0.400pt}}
\multiput(242.00,140.17)(0.500,-1.000){2}{\rule{0.120pt}{0.400pt}}
\put(241.0,141.0){\usebox{\plotpoint}}
\put(243,140){\usebox{\plotpoint}}
\put(243,140){\usebox{\plotpoint}}
\put(243,140){\usebox{\plotpoint}}
\put(243,140){\usebox{\plotpoint}}
\put(243,140){\usebox{\plotpoint}}
\put(243,140){\usebox{\plotpoint}}
\put(243,140){\usebox{\plotpoint}}
\put(243.0,140.0){\usebox{\plotpoint}}
\put(244.0,139.0){\usebox{\plotpoint}}
\put(244.0,139.0){\rule[-0.200pt]{0.482pt}{0.400pt}}
\put(246.0,138.0){\usebox{\plotpoint}}
\put(248,136.67){\rule{0.241pt}{0.400pt}}
\multiput(248.00,137.17)(0.500,-1.000){2}{\rule{0.120pt}{0.400pt}}
\put(246.0,138.0){\rule[-0.200pt]{0.482pt}{0.400pt}}
\put(249,137){\usebox{\plotpoint}}
\put(249,137){\usebox{\plotpoint}}
\put(249,137){\usebox{\plotpoint}}
\put(249,137){\usebox{\plotpoint}}
\put(249,137){\usebox{\plotpoint}}
\put(249,137){\usebox{\plotpoint}}
\put(249,137){\usebox{\plotpoint}}
\put(249.0,137.0){\rule[-0.200pt]{0.482pt}{0.400pt}}
\put(251.0,136.0){\usebox{\plotpoint}}
\put(251.0,136.0){\rule[-0.200pt]{0.482pt}{0.400pt}}
\put(253.0,135.0){\usebox{\plotpoint}}
\put(253.0,135.0){\rule[-0.200pt]{0.723pt}{0.400pt}}
\put(256.0,134.0){\usebox{\plotpoint}}
\put(256.0,134.0){\rule[-0.200pt]{0.723pt}{0.400pt}}
\put(259.0,133.0){\usebox{\plotpoint}}
\put(259.0,133.0){\rule[-0.200pt]{0.964pt}{0.400pt}}
\put(263.0,132.0){\usebox{\plotpoint}}
\put(263.0,132.0){\rule[-0.200pt]{1.445pt}{0.400pt}}
\put(269.0,131.0){\usebox{\plotpoint}}
\put(269.0,131.0){\rule[-0.200pt]{3.613pt}{0.400pt}}
\put(284.0,131.0){\usebox{\plotpoint}}
\put(284.0,132.0){\rule[-0.200pt]{1.204pt}{0.400pt}}
\put(289.0,132.0){\usebox{\plotpoint}}
\put(293,132.67){\rule{0.241pt}{0.400pt}}
\multiput(293.00,132.17)(0.500,1.000){2}{\rule{0.120pt}{0.400pt}}
\put(289.0,133.0){\rule[-0.200pt]{0.964pt}{0.400pt}}
\put(294,134){\usebox{\plotpoint}}
\put(294,134){\usebox{\plotpoint}}
\put(294,134){\usebox{\plotpoint}}
\put(294,134){\usebox{\plotpoint}}
\put(294,134){\usebox{\plotpoint}}
\put(294,134){\usebox{\plotpoint}}
\put(294,134){\usebox{\plotpoint}}
\put(294.0,134.0){\rule[-0.200pt]{0.723pt}{0.400pt}}
\put(297.0,134.0){\usebox{\plotpoint}}
\put(297.0,135.0){\rule[-0.200pt]{0.723pt}{0.400pt}}
\put(300.0,135.0){\usebox{\plotpoint}}
\put(300.0,136.0){\rule[-0.200pt]{0.723pt}{0.400pt}}
\put(303.0,136.0){\usebox{\plotpoint}}
\put(303.0,137.0){\rule[-0.200pt]{0.482pt}{0.400pt}}
\put(305.0,137.0){\usebox{\plotpoint}}
\put(307,137.67){\rule{0.241pt}{0.400pt}}
\multiput(307.00,137.17)(0.500,1.000){2}{\rule{0.120pt}{0.400pt}}
\put(305.0,138.0){\rule[-0.200pt]{0.482pt}{0.400pt}}
\put(308,139){\usebox{\plotpoint}}
\put(308,139){\usebox{\plotpoint}}
\put(308,139){\usebox{\plotpoint}}
\put(308,139){\usebox{\plotpoint}}
\put(308,139){\usebox{\plotpoint}}
\put(308,139){\usebox{\plotpoint}}
\put(308,139){\usebox{\plotpoint}}
\put(308.0,139.0){\rule[-0.200pt]{0.482pt}{0.400pt}}
\put(310.0,139.0){\usebox{\plotpoint}}
\put(310.0,140.0){\rule[-0.200pt]{0.482pt}{0.400pt}}
\put(312.0,140.0){\usebox{\plotpoint}}
\put(312.0,141.0){\rule[-0.200pt]{0.482pt}{0.400pt}}
\put(314.0,141.0){\usebox{\plotpoint}}
\put(316,141.67){\rule{0.241pt}{0.400pt}}
\multiput(316.00,141.17)(0.500,1.000){2}{\rule{0.120pt}{0.400pt}}
\put(314.0,142.0){\rule[-0.200pt]{0.482pt}{0.400pt}}
\put(317,143){\usebox{\plotpoint}}
\put(317,143){\usebox{\plotpoint}}
\put(317,143){\usebox{\plotpoint}}
\put(317,143){\usebox{\plotpoint}}
\put(317,143){\usebox{\plotpoint}}
\put(317,143){\usebox{\plotpoint}}
\put(317,143){\usebox{\plotpoint}}
\put(317.0,143.0){\rule[-0.200pt]{0.482pt}{0.400pt}}
\put(319.0,143.0){\usebox{\plotpoint}}
\put(319.0,144.0){\rule[-0.200pt]{0.482pt}{0.400pt}}
\put(321.0,144.0){\usebox{\plotpoint}}
\put(321.0,145.0){\rule[-0.200pt]{0.482pt}{0.400pt}}
\put(323.0,145.0){\usebox{\plotpoint}}
\put(323.0,146.0){\rule[-0.200pt]{0.482pt}{0.400pt}}
\put(325.0,146.0){\usebox{\plotpoint}}
\put(325.0,147.0){\rule[-0.200pt]{0.482pt}{0.400pt}}
\put(327.0,147.0){\usebox{\plotpoint}}
\put(327.0,148.0){\rule[-0.200pt]{0.723pt}{0.400pt}}
\put(330.0,148.0){\usebox{\plotpoint}}
\put(330.0,149.0){\rule[-0.200pt]{0.482pt}{0.400pt}}
\put(332.0,149.0){\usebox{\plotpoint}}
\put(332.0,150.0){\rule[-0.200pt]{0.482pt}{0.400pt}}
\put(334.0,150.0){\usebox{\plotpoint}}
\put(334.0,151.0){\rule[-0.200pt]{0.723pt}{0.400pt}}
\put(337.0,151.0){\usebox{\plotpoint}}
\put(337.0,152.0){\rule[-0.200pt]{0.723pt}{0.400pt}}
\put(340.0,152.0){\usebox{\plotpoint}}
\put(340.0,153.0){\rule[-0.200pt]{0.482pt}{0.400pt}}
\put(342.0,153.0){\usebox{\plotpoint}}
\put(345,153.67){\rule{0.241pt}{0.400pt}}
\multiput(345.00,153.17)(0.500,1.000){2}{\rule{0.120pt}{0.400pt}}
\put(342.0,154.0){\rule[-0.200pt]{0.723pt}{0.400pt}}
\put(346,155){\usebox{\plotpoint}}
\put(346,155){\usebox{\plotpoint}}
\put(346,155){\usebox{\plotpoint}}
\put(346,155){\usebox{\plotpoint}}
\put(346,155){\usebox{\plotpoint}}
\put(346,155){\usebox{\plotpoint}}
\put(346.0,155.0){\rule[-0.200pt]{0.723pt}{0.400pt}}
\put(349.0,155.0){\usebox{\plotpoint}}
\put(349.0,156.0){\rule[-0.200pt]{1.204pt}{0.400pt}}
\put(354.0,156.0){\usebox{\plotpoint}}
\put(354.0,157.0){\rule[-0.200pt]{2.168pt}{0.400pt}}
\put(363.0,157.0){\usebox{\plotpoint}}
\put(363.0,158.0){\rule[-0.200pt]{1.204pt}{0.400pt}}
\put(368.0,157.0){\usebox{\plotpoint}}
\put(368.0,157.0){\rule[-0.200pt]{2.168pt}{0.400pt}}
\put(377.0,156.0){\usebox{\plotpoint}}
\put(377.0,156.0){\rule[-0.200pt]{1.204pt}{0.400pt}}
\put(382.0,155.0){\usebox{\plotpoint}}
\put(382.0,155.0){\rule[-0.200pt]{0.723pt}{0.400pt}}
\put(385.0,154.0){\usebox{\plotpoint}}
\put(388,152.67){\rule{0.241pt}{0.400pt}}
\multiput(388.00,153.17)(0.500,-1.000){2}{\rule{0.120pt}{0.400pt}}
\put(385.0,154.0){\rule[-0.200pt]{0.723pt}{0.400pt}}
\put(389,153){\usebox{\plotpoint}}
\put(389,153){\usebox{\plotpoint}}
\put(389,153){\usebox{\plotpoint}}
\put(389,153){\usebox{\plotpoint}}
\put(389,153){\usebox{\plotpoint}}
\put(389,153){\usebox{\plotpoint}}
\put(389,153){\usebox{\plotpoint}}
\put(391,151.67){\rule{0.241pt}{0.400pt}}
\multiput(391.00,152.17)(0.500,-1.000){2}{\rule{0.120pt}{0.400pt}}
\put(389.0,153.0){\rule[-0.200pt]{0.482pt}{0.400pt}}
\put(392,152){\usebox{\plotpoint}}
\put(392,152){\usebox{\plotpoint}}
\put(392,152){\usebox{\plotpoint}}
\put(392,152){\usebox{\plotpoint}}
\put(392,152){\usebox{\plotpoint}}
\put(392,152){\usebox{\plotpoint}}
\put(392,152){\usebox{\plotpoint}}
\put(392.0,152.0){\rule[-0.200pt]{0.482pt}{0.400pt}}
\put(394.0,151.0){\usebox{\plotpoint}}
\put(396,149.67){\rule{0.241pt}{0.400pt}}
\multiput(396.00,150.17)(0.500,-1.000){2}{\rule{0.120pt}{0.400pt}}
\put(394.0,151.0){\rule[-0.200pt]{0.482pt}{0.400pt}}
\put(397,150){\usebox{\plotpoint}}
\put(397,150){\usebox{\plotpoint}}
\put(397,150){\usebox{\plotpoint}}
\put(397,150){\usebox{\plotpoint}}
\put(397,150){\usebox{\plotpoint}}
\put(397,150){\usebox{\plotpoint}}
\put(397,150){\usebox{\plotpoint}}
\put(397.0,150.0){\rule[-0.200pt]{0.482pt}{0.400pt}}
\put(399.0,149.0){\usebox{\plotpoint}}
\put(399.0,149.0){\rule[-0.200pt]{0.482pt}{0.400pt}}
\put(401.0,148.0){\usebox{\plotpoint}}
\put(401.0,148.0){\rule[-0.200pt]{0.482pt}{0.400pt}}
\put(403.0,147.0){\usebox{\plotpoint}}
\put(405,145.67){\rule{0.241pt}{0.400pt}}
\multiput(405.00,146.17)(0.500,-1.000){2}{\rule{0.120pt}{0.400pt}}
\put(403.0,147.0){\rule[-0.200pt]{0.482pt}{0.400pt}}
\put(406,146){\usebox{\plotpoint}}
\put(406,146){\usebox{\plotpoint}}
\put(406,146){\usebox{\plotpoint}}
\put(406,146){\usebox{\plotpoint}}
\put(406,146){\usebox{\plotpoint}}
\put(406,146){\usebox{\plotpoint}}
\put(406,146){\usebox{\plotpoint}}
\put(406.0,146.0){\rule[-0.200pt]{0.482pt}{0.400pt}}
\put(408.0,145.0){\usebox{\plotpoint}}
\put(408.0,145.0){\rule[-0.200pt]{0.482pt}{0.400pt}}
\put(410.0,144.0){\usebox{\plotpoint}}
\put(410.0,144.0){\rule[-0.200pt]{0.482pt}{0.400pt}}
\put(412.0,143.0){\usebox{\plotpoint}}
\put(412.0,143.0){\rule[-0.200pt]{0.482pt}{0.400pt}}
\put(414.0,142.0){\usebox{\plotpoint}}
\put(414.0,142.0){\rule[-0.200pt]{0.482pt}{0.400pt}}
\put(416.0,141.0){\usebox{\plotpoint}}
\put(416.0,141.0){\rule[-0.200pt]{0.482pt}{0.400pt}}
\put(418.0,140.0){\usebox{\plotpoint}}
\put(418.0,140.0){\rule[-0.200pt]{0.723pt}{0.400pt}}
\put(421.0,139.0){\usebox{\plotpoint}}
\put(421.0,139.0){\rule[-0.200pt]{0.482pt}{0.400pt}}
\put(423.0,138.0){\usebox{\plotpoint}}
\put(423.0,138.0){\rule[-0.200pt]{0.482pt}{0.400pt}}
\put(425.0,137.0){\usebox{\plotpoint}}
\put(425.0,137.0){\rule[-0.200pt]{0.723pt}{0.400pt}}
\put(428.0,136.0){\usebox{\plotpoint}}
\put(428.0,136.0){\rule[-0.200pt]{0.482pt}{0.400pt}}
\put(430.0,135.0){\usebox{\plotpoint}}
\put(430.0,135.0){\rule[-0.200pt]{0.723pt}{0.400pt}}
\put(433.0,134.0){\usebox{\plotpoint}}
\put(433.0,134.0){\rule[-0.200pt]{0.964pt}{0.400pt}}
\put(437.0,133.0){\usebox{\plotpoint}}
\put(440,131.67){\rule{0.241pt}{0.400pt}}
\multiput(440.00,132.17)(0.500,-1.000){2}{\rule{0.120pt}{0.400pt}}
\put(437.0,133.0){\rule[-0.200pt]{0.723pt}{0.400pt}}
\put(441,132){\usebox{\plotpoint}}
\put(441,132){\usebox{\plotpoint}}
\put(441,132){\usebox{\plotpoint}}
\put(441,132){\usebox{\plotpoint}}
\put(441,132){\usebox{\plotpoint}}
\put(441,132){\usebox{\plotpoint}}
\put(441,132){\usebox{\plotpoint}}
\put(441.0,132.0){\rule[-0.200pt]{1.204pt}{0.400pt}}
\put(446.0,131.0){\usebox{\plotpoint}}
\put(460,130.67){\rule{0.241pt}{0.400pt}}
\multiput(460.00,130.17)(0.500,1.000){2}{\rule{0.120pt}{0.400pt}}
\put(446.0,131.0){\rule[-0.200pt]{3.373pt}{0.400pt}}
\put(461,132){\usebox{\plotpoint}}
\put(461,132){\usebox{\plotpoint}}
\put(461,132){\usebox{\plotpoint}}
\put(461,132){\usebox{\plotpoint}}
\put(461,132){\usebox{\plotpoint}}
\put(461,132){\usebox{\plotpoint}}
\put(461,132){\usebox{\plotpoint}}
\put(461.0,132.0){\rule[-0.200pt]{1.204pt}{0.400pt}}
\put(466.0,132.0){\usebox{\plotpoint}}
\put(466.0,133.0){\rule[-0.200pt]{0.723pt}{0.400pt}}
\put(469.0,133.0){\usebox{\plotpoint}}
\put(469.0,134.0){\rule[-0.200pt]{0.723pt}{0.400pt}}
\put(472.0,134.0){\usebox{\plotpoint}}
\put(472.0,135.0){\rule[-0.200pt]{0.723pt}{0.400pt}}
\put(475.0,135.0){\usebox{\plotpoint}}
\put(475.0,136.0){\rule[-0.200pt]{0.482pt}{0.400pt}}
\put(477.0,136.0){\usebox{\plotpoint}}
\put(477.0,137.0){\rule[-0.200pt]{0.482pt}{0.400pt}}
\put(479.0,137.0){\usebox{\plotpoint}}
\put(479.0,138.0){\rule[-0.200pt]{0.482pt}{0.400pt}}
\put(481.0,138.0){\usebox{\plotpoint}}
\put(481.0,139.0){\rule[-0.200pt]{0.482pt}{0.400pt}}
\put(483.0,139.0){\usebox{\plotpoint}}
\put(483.0,140.0){\rule[-0.200pt]{0.482pt}{0.400pt}}
\put(485.0,140.0){\usebox{\plotpoint}}
\put(485.0,141.0){\rule[-0.200pt]{0.482pt}{0.400pt}}
\put(487.0,141.0){\usebox{\plotpoint}}
\put(487.0,142.0){\usebox{\plotpoint}}
\put(488.0,142.0){\usebox{\plotpoint}}
\put(488.0,143.0){\rule[-0.200pt]{0.482pt}{0.400pt}}
\put(490.0,143.0){\usebox{\plotpoint}}
\put(490.0,144.0){\usebox{\plotpoint}}
\put(491.0,144.0){\usebox{\plotpoint}}
\put(491.0,145.0){\rule[-0.200pt]{0.482pt}{0.400pt}}
\put(493.0,145.0){\usebox{\plotpoint}}
\put(493.0,146.0){\usebox{\plotpoint}}
\put(494.0,146.0){\usebox{\plotpoint}}
\put(494.0,147.0){\rule[-0.200pt]{0.482pt}{0.400pt}}
\put(496.0,147.0){\usebox{\plotpoint}}
\put(496.0,148.0){\usebox{\plotpoint}}
\put(497.0,148.0){\usebox{\plotpoint}}
\put(497.0,149.0){\rule[-0.200pt]{0.482pt}{0.400pt}}
\put(499.0,149.0){\usebox{\plotpoint}}
\put(499.0,150.0){\usebox{\plotpoint}}
\put(500.0,150.0){\usebox{\plotpoint}}
\put(500.0,151.0){\rule[-0.200pt]{0.482pt}{0.400pt}}
\put(502.0,151.0){\usebox{\plotpoint}}
\put(502.0,152.0){\usebox{\plotpoint}}
\put(503.0,152.0){\usebox{\plotpoint}}
\put(504,152.67){\rule{0.241pt}{0.400pt}}
\multiput(504.00,152.17)(0.500,1.000){2}{\rule{0.120pt}{0.400pt}}
\put(503.0,153.0){\usebox{\plotpoint}}
\put(505,154){\usebox{\plotpoint}}
\put(505,154){\usebox{\plotpoint}}
\put(505,154){\usebox{\plotpoint}}
\put(505,154){\usebox{\plotpoint}}
\put(505,154){\usebox{\plotpoint}}
\put(505,154){\usebox{\plotpoint}}
\put(505,154){\usebox{\plotpoint}}
\put(505.0,154.0){\usebox{\plotpoint}}
\put(506.0,154.0){\usebox{\plotpoint}}
\put(506.0,155.0){\usebox{\plotpoint}}
\put(507.0,155.0){\usebox{\plotpoint}}
\put(507.0,156.0){\rule[-0.200pt]{0.482pt}{0.400pt}}
\put(509.0,156.0){\usebox{\plotpoint}}
\put(509.0,157.0){\usebox{\plotpoint}}
\put(510.0,157.0){\usebox{\plotpoint}}
\put(510.0,158.0){\usebox{\plotpoint}}
\put(511.0,158.0){\usebox{\plotpoint}}
\put(511.0,159.0){\rule[-0.200pt]{0.482pt}{0.400pt}}
\put(513.0,159.0){\usebox{\plotpoint}}
\put(513.0,160.0){\usebox{\plotpoint}}
\put(514.0,160.0){\usebox{\plotpoint}}
\put(514.0,161.0){\rule[-0.200pt]{0.482pt}{0.400pt}}
\put(516.0,161.0){\usebox{\plotpoint}}
\put(516.0,162.0){\usebox{\plotpoint}}
\put(517.0,162.0){\usebox{\plotpoint}}
\put(517.0,163.0){\rule[-0.200pt]{0.482pt}{0.400pt}}
\put(519.0,163.0){\usebox{\plotpoint}}
\put(519.0,164.0){\usebox{\plotpoint}}
\put(520.0,164.0){\usebox{\plotpoint}}
\put(520.0,165.0){\rule[-0.200pt]{0.482pt}{0.400pt}}
\put(522.0,165.0){\usebox{\plotpoint}}
\put(522.0,166.0){\rule[-0.200pt]{0.482pt}{0.400pt}}
\put(524.0,166.0){\usebox{\plotpoint}}
\put(524.0,167.0){\usebox{\plotpoint}}
\put(525.0,167.0){\usebox{\plotpoint}}
\put(525.0,168.0){\rule[-0.200pt]{0.482pt}{0.400pt}}
\put(527.0,168.0){\usebox{\plotpoint}}
\put(527.0,169.0){\rule[-0.200pt]{0.482pt}{0.400pt}}
\put(529.0,169.0){\usebox{\plotpoint}}
\put(529.0,170.0){\rule[-0.200pt]{0.482pt}{0.400pt}}
\put(531.0,170.0){\usebox{\plotpoint}}
\put(531.0,171.0){\rule[-0.200pt]{0.723pt}{0.400pt}}
\put(534.0,171.0){\usebox{\plotpoint}}
\put(534.0,172.0){\rule[-0.200pt]{0.482pt}{0.400pt}}
\put(536.0,172.0){\usebox{\plotpoint}}
\put(536.0,173.0){\rule[-0.200pt]{0.723pt}{0.400pt}}
\put(539.0,173.0){\usebox{\plotpoint}}
\put(539.0,174.0){\rule[-0.200pt]{1.204pt}{0.400pt}}
\put(544.0,174.0){\usebox{\plotpoint}}
\put(544.0,175.0){\rule[-0.200pt]{3.132pt}{0.400pt}}
\put(557.0,174.0){\usebox{\plotpoint}}
\put(557.0,174.0){\rule[-0.200pt]{0.964pt}{0.400pt}}
\put(561.0,173.0){\usebox{\plotpoint}}
\put(561.0,173.0){\rule[-0.200pt]{0.723pt}{0.400pt}}
\put(564.0,172.0){\usebox{\plotpoint}}
\put(564.0,172.0){\rule[-0.200pt]{0.723pt}{0.400pt}}
\put(567.0,171.0){\usebox{\plotpoint}}
\put(567.0,171.0){\rule[-0.200pt]{0.482pt}{0.400pt}}
\put(569.0,170.0){\usebox{\plotpoint}}
\put(569.0,170.0){\rule[-0.200pt]{0.482pt}{0.400pt}}
\put(571.0,169.0){\usebox{\plotpoint}}
\put(571.0,169.0){\rule[-0.200pt]{0.482pt}{0.400pt}}
\put(573.0,168.0){\usebox{\plotpoint}}
\put(573.0,168.0){\usebox{\plotpoint}}
\put(574.0,167.0){\usebox{\plotpoint}}
\put(574.0,167.0){\rule[-0.200pt]{0.482pt}{0.400pt}}
\put(576.0,166.0){\usebox{\plotpoint}}
\put(576.0,166.0){\usebox{\plotpoint}}
\put(577.0,165.0){\usebox{\plotpoint}}
\put(577.0,165.0){\rule[-0.200pt]{0.482pt}{0.400pt}}
\put(579.0,164.0){\usebox{\plotpoint}}
\put(579.0,164.0){\usebox{\plotpoint}}
\put(580.0,163.0){\usebox{\plotpoint}}
\put(581,161.67){\rule{0.241pt}{0.400pt}}
\multiput(581.00,162.17)(0.500,-1.000){2}{\rule{0.120pt}{0.400pt}}
\put(580.0,163.0){\usebox{\plotpoint}}
\put(582,162){\usebox{\plotpoint}}
\put(582,162){\usebox{\plotpoint}}
\put(582,162){\usebox{\plotpoint}}
\put(582,162){\usebox{\plotpoint}}
\put(582,162){\usebox{\plotpoint}}
\put(582,162){\usebox{\plotpoint}}
\put(582.0,162.0){\usebox{\plotpoint}}
\put(583.0,161.0){\usebox{\plotpoint}}
\put(583.0,161.0){\usebox{\plotpoint}}
\put(584.0,160.0){\usebox{\plotpoint}}
\put(584.0,160.0){\usebox{\plotpoint}}
\put(585.0,159.0){\usebox{\plotpoint}}
\put(585.0,159.0){\rule[-0.200pt]{0.482pt}{0.400pt}}
\put(587.0,158.0){\usebox{\plotpoint}}
\put(587.0,158.0){\usebox{\plotpoint}}
\put(588.0,157.0){\usebox{\plotpoint}}
\put(588.0,157.0){\usebox{\plotpoint}}
\put(589.0,156.0){\usebox{\plotpoint}}
\put(589.0,156.0){\usebox{\plotpoint}}
\put(590.0,155.0){\usebox{\plotpoint}}
\put(591,153.67){\rule{0.241pt}{0.400pt}}
\multiput(591.00,154.17)(0.500,-1.000){2}{\rule{0.120pt}{0.400pt}}
\put(590.0,155.0){\usebox{\plotpoint}}
\put(592,154){\usebox{\plotpoint}}
\put(592,154){\usebox{\plotpoint}}
\put(592,154){\usebox{\plotpoint}}
\put(592,154){\usebox{\plotpoint}}
\put(592,154){\usebox{\plotpoint}}
\put(592,154){\usebox{\plotpoint}}
\put(592,154){\usebox{\plotpoint}}
\put(592.0,154.0){\usebox{\plotpoint}}
\put(593.0,153.0){\usebox{\plotpoint}}
\put(593.0,153.0){\usebox{\plotpoint}}
\put(594.0,152.0){\usebox{\plotpoint}}
\put(594.0,152.0){\usebox{\plotpoint}}
\put(595.0,151.0){\usebox{\plotpoint}}
\put(595.0,151.0){\usebox{\plotpoint}}
\put(596.0,150.0){\usebox{\plotpoint}}
\put(596.0,150.0){\usebox{\plotpoint}}
\put(597.0,149.0){\usebox{\plotpoint}}
\put(597.0,149.0){\usebox{\plotpoint}}
\put(598.0,148.0){\usebox{\plotpoint}}
\put(599,146.67){\rule{0.241pt}{0.400pt}}
\multiput(599.00,147.17)(0.500,-1.000){2}{\rule{0.120pt}{0.400pt}}
\put(598.0,148.0){\usebox{\plotpoint}}
\put(600,147){\usebox{\plotpoint}}
\put(600,147){\usebox{\plotpoint}}
\put(600,147){\usebox{\plotpoint}}
\put(600,147){\usebox{\plotpoint}}
\put(600,147){\usebox{\plotpoint}}
\put(600,147){\usebox{\plotpoint}}
\put(600,147){\usebox{\plotpoint}}
\put(600.0,147.0){\usebox{\plotpoint}}
\put(601.0,146.0){\usebox{\plotpoint}}
\put(601.0,146.0){\usebox{\plotpoint}}
\put(602.0,145.0){\usebox{\plotpoint}}
\put(602.0,145.0){\usebox{\plotpoint}}
\put(603.0,144.0){\usebox{\plotpoint}}
\put(603.0,144.0){\usebox{\plotpoint}}
\put(604.0,143.0){\usebox{\plotpoint}}
\put(604.0,143.0){\usebox{\plotpoint}}
\put(605.0,142.0){\usebox{\plotpoint}}
\put(605.0,142.0){\rule[-0.200pt]{0.482pt}{0.400pt}}
\put(607.0,141.0){\usebox{\plotpoint}}
\put(607.0,141.0){\usebox{\plotpoint}}
\put(608.0,140.0){\usebox{\plotpoint}}
\put(608.0,140.0){\usebox{\plotpoint}}
\put(609.0,139.0){\usebox{\plotpoint}}
\put(609.0,139.0){\rule[-0.200pt]{0.482pt}{0.400pt}}
\put(611.0,138.0){\usebox{\plotpoint}}
\put(611.0,138.0){\usebox{\plotpoint}}
\put(612.0,137.0){\usebox{\plotpoint}}
\put(612.0,137.0){\rule[-0.200pt]{0.482pt}{0.400pt}}
\put(614.0,136.0){\usebox{\plotpoint}}
\put(614.0,136.0){\usebox{\plotpoint}}
\put(615.0,135.0){\usebox{\plotpoint}}
\put(615.0,135.0){\rule[-0.200pt]{0.482pt}{0.400pt}}
\put(617.0,134.0){\usebox{\plotpoint}}
\put(617.0,134.0){\rule[-0.200pt]{0.482pt}{0.400pt}}
\put(619.0,133.0){\usebox{\plotpoint}}
\put(619.0,133.0){\rule[-0.200pt]{0.723pt}{0.400pt}}
\put(622.0,132.0){\usebox{\plotpoint}}
\put(622.0,132.0){\rule[-0.200pt]{0.723pt}{0.400pt}}
\put(625.0,131.0){\usebox{\plotpoint}}
\put(625.0,131.0){\rule[-0.200pt]{2.168pt}{0.400pt}}
\put(634.0,131.0){\usebox{\plotpoint}}
\put(634.0,132.0){\rule[-0.200pt]{0.723pt}{0.400pt}}
\put(637.0,132.0){\usebox{\plotpoint}}
\put(637.0,133.0){\rule[-0.200pt]{0.482pt}{0.400pt}}
\put(639.0,133.0){\usebox{\plotpoint}}
\put(639.0,134.0){\usebox{\plotpoint}}
\put(640.0,134.0){\usebox{\plotpoint}}
\put(640.0,135.0){\rule[-0.200pt]{0.482pt}{0.400pt}}
\put(642.0,135.0){\usebox{\plotpoint}}
\put(642.0,136.0){\usebox{\plotpoint}}
\put(643.0,136.0){\usebox{\plotpoint}}
\put(643.0,137.0){\usebox{\plotpoint}}
\put(644.0,137.0){\usebox{\plotpoint}}
\put(644.0,138.0){\usebox{\plotpoint}}
\put(645.0,138.0){\usebox{\plotpoint}}
\put(645.0,139.0){\usebox{\plotpoint}}
\put(646.0,139.0){\usebox{\plotpoint}}
\put(646.0,140.0){\usebox{\plotpoint}}
\put(647.0,140.0){\usebox{\plotpoint}}
\put(647.0,141.0){\usebox{\plotpoint}}
\put(648,141.67){\rule{0.241pt}{0.400pt}}
\multiput(648.00,141.17)(0.500,1.000){2}{\rule{0.120pt}{0.400pt}}
\put(648.0,141.0){\usebox{\plotpoint}}
\put(649,143){\usebox{\plotpoint}}
\put(649,143){\usebox{\plotpoint}}
\put(649,143){\usebox{\plotpoint}}
\put(649,143){\usebox{\plotpoint}}
\put(649,143){\usebox{\plotpoint}}
\put(649.0,143.0){\usebox{\plotpoint}}
\put(649.0,144.0){\usebox{\plotpoint}}
\put(650.0,144.0){\usebox{\plotpoint}}
\put(650.0,145.0){\usebox{\plotpoint}}
\put(651.0,145.0){\rule[-0.200pt]{0.400pt}{0.482pt}}
\put(651.0,147.0){\usebox{\plotpoint}}
\put(652.0,147.0){\usebox{\plotpoint}}
\put(652.0,148.0){\usebox{\plotpoint}}
\put(653.0,148.0){\rule[-0.200pt]{0.400pt}{0.482pt}}
\put(653.0,150.0){\usebox{\plotpoint}}
\put(654,150.67){\rule{0.241pt}{0.400pt}}
\multiput(654.00,150.17)(0.500,1.000){2}{\rule{0.120pt}{0.400pt}}
\put(654.0,150.0){\usebox{\plotpoint}}
\put(655,152){\usebox{\plotpoint}}
\put(655,152){\usebox{\plotpoint}}
\put(655,152){\usebox{\plotpoint}}
\put(655,152.67){\rule{0.241pt}{0.400pt}}
\multiput(655.00,152.17)(0.500,1.000){2}{\rule{0.120pt}{0.400pt}}
\put(655.0,152.0){\usebox{\plotpoint}}
\put(656,154){\usebox{\plotpoint}}
\put(656,154){\usebox{\plotpoint}}
\put(656,154){\usebox{\plotpoint}}
\put(656,154){\usebox{\plotpoint}}
\put(656.0,154.0){\usebox{\plotpoint}}
\put(656.0,155.0){\usebox{\plotpoint}}
\put(657.0,155.0){\rule[-0.200pt]{0.400pt}{0.482pt}}
\put(657.0,157.0){\usebox{\plotpoint}}
\put(658,158.67){\rule{0.241pt}{0.400pt}}
\multiput(658.00,158.17)(0.500,1.000){2}{\rule{0.120pt}{0.400pt}}
\put(658.0,157.0){\rule[-0.200pt]{0.400pt}{0.482pt}}
\put(659,160){\usebox{\plotpoint}}
\put(659,160){\usebox{\plotpoint}}
\put(659,160){\usebox{\plotpoint}}
\put(659.0,160.0){\rule[-0.200pt]{0.400pt}{0.482pt}}
\put(659.0,162.0){\usebox{\plotpoint}}
\put(660.0,162.0){\rule[-0.200pt]{0.400pt}{0.482pt}}
\put(660.0,164.0){\usebox{\plotpoint}}
\put(661.0,164.0){\rule[-0.200pt]{0.400pt}{0.482pt}}
\put(661.0,166.0){\usebox{\plotpoint}}
\put(662,167.67){\rule{0.241pt}{0.400pt}}
\multiput(662.00,167.17)(0.500,1.000){2}{\rule{0.120pt}{0.400pt}}
\put(662.0,166.0){\rule[-0.200pt]{0.400pt}{0.482pt}}
\put(663,169){\usebox{\plotpoint}}
\put(663,169){\usebox{\plotpoint}}
\put(663.0,169.0){\rule[-0.200pt]{0.400pt}{0.482pt}}
\put(663.0,171.0){\usebox{\plotpoint}}
\put(664.0,171.0){\rule[-0.200pt]{0.400pt}{0.723pt}}
\put(664.0,174.0){\usebox{\plotpoint}}
\put(665,175.67){\rule{0.241pt}{0.400pt}}
\multiput(665.00,175.17)(0.500,1.000){2}{\rule{0.120pt}{0.400pt}}
\put(665.0,174.0){\rule[-0.200pt]{0.400pt}{0.482pt}}
\put(666,177){\usebox{\plotpoint}}
\put(666,177){\usebox{\plotpoint}}
\put(666,178.67){\rule{0.241pt}{0.400pt}}
\multiput(666.00,178.17)(0.500,1.000){2}{\rule{0.120pt}{0.400pt}}
\put(666.0,177.0){\rule[-0.200pt]{0.400pt}{0.482pt}}
\put(667,180){\usebox{\plotpoint}}
\put(667,180){\usebox{\plotpoint}}
\put(667,181.67){\rule{0.241pt}{0.400pt}}
\multiput(667.00,181.17)(0.500,1.000){2}{\rule{0.120pt}{0.400pt}}
\put(667.0,180.0){\rule[-0.200pt]{0.400pt}{0.482pt}}
\put(668,183){\usebox{\plotpoint}}
\put(668,183){\usebox{\plotpoint}}
\put(668,184.67){\rule{0.241pt}{0.400pt}}
\multiput(668.00,184.17)(0.500,1.000){2}{\rule{0.120pt}{0.400pt}}
\put(668.0,183.0){\rule[-0.200pt]{0.400pt}{0.482pt}}
\put(669,186){\usebox{\plotpoint}}
\put(669,186){\usebox{\plotpoint}}
\put(669.0,186.0){\rule[-0.200pt]{0.400pt}{0.482pt}}
\put(669.0,188.0){\usebox{\plotpoint}}
\put(670,190.67){\rule{0.241pt}{0.400pt}}
\multiput(670.00,190.17)(0.500,1.000){2}{\rule{0.120pt}{0.400pt}}
\put(670.0,188.0){\rule[-0.200pt]{0.400pt}{0.723pt}}
\put(671,192){\usebox{\plotpoint}}
\put(671,192){\usebox{\plotpoint}}
\put(671,193.67){\rule{0.241pt}{0.400pt}}
\multiput(671.00,193.17)(0.500,1.000){2}{\rule{0.120pt}{0.400pt}}
\put(671.0,192.0){\rule[-0.200pt]{0.400pt}{0.482pt}}
\put(672,195){\usebox{\plotpoint}}
\put(672.0,195.0){\rule[-0.200pt]{0.400pt}{0.723pt}}
\put(672.0,198.0){\usebox{\plotpoint}}
\put(673,200.67){\rule{0.241pt}{0.400pt}}
\multiput(673.00,200.17)(0.500,1.000){2}{\rule{0.120pt}{0.400pt}}
\put(673.0,198.0){\rule[-0.200pt]{0.400pt}{0.723pt}}
\put(674,202){\usebox{\plotpoint}}
\put(674.0,202.0){\rule[-0.200pt]{0.400pt}{0.723pt}}
\put(674.0,205.0){\usebox{\plotpoint}}
\put(675.0,205.0){\rule[-0.200pt]{0.400pt}{0.964pt}}
\put(675.0,209.0){\usebox{\plotpoint}}
\put(676,211.67){\rule{0.241pt}{0.400pt}}
\multiput(676.00,211.17)(0.500,1.000){2}{\rule{0.120pt}{0.400pt}}
\put(676.0,209.0){\rule[-0.200pt]{0.400pt}{0.723pt}}
\put(677,213){\usebox{\plotpoint}}
\put(677,215.67){\rule{0.241pt}{0.400pt}}
\multiput(677.00,215.17)(0.500,1.000){2}{\rule{0.120pt}{0.400pt}}
\put(677.0,213.0){\rule[-0.200pt]{0.400pt}{0.723pt}}
\put(678,217){\usebox{\plotpoint}}
\put(678.0,217.0){\rule[-0.200pt]{0.400pt}{0.723pt}}
\put(678.0,220.0){\usebox{\plotpoint}}
\put(679.0,220.0){\rule[-0.200pt]{0.400pt}{0.964pt}}
\put(679.0,224.0){\usebox{\plotpoint}}
\put(680,227.67){\rule{0.241pt}{0.400pt}}
\multiput(680.00,227.17)(0.500,1.000){2}{\rule{0.120pt}{0.400pt}}
\put(680.0,224.0){\rule[-0.200pt]{0.400pt}{0.964pt}}
\put(681,229){\usebox{\plotpoint}}
\put(681,231.67){\rule{0.241pt}{0.400pt}}
\multiput(681.00,231.17)(0.500,1.000){2}{\rule{0.120pt}{0.400pt}}
\put(681.0,229.0){\rule[-0.200pt]{0.400pt}{0.723pt}}
\put(682,233){\usebox{\plotpoint}}
\put(682.0,233.0){\rule[-0.200pt]{0.400pt}{0.964pt}}
\put(682.0,237.0){\usebox{\plotpoint}}
\put(683,240.67){\rule{0.241pt}{0.400pt}}
\multiput(683.00,240.17)(0.500,1.000){2}{\rule{0.120pt}{0.400pt}}
\put(683.0,237.0){\rule[-0.200pt]{0.400pt}{0.964pt}}
\put(684,242){\usebox{\plotpoint}}
\put(684.0,242.0){\rule[-0.200pt]{0.400pt}{0.964pt}}
\put(684.0,246.0){\usebox{\plotpoint}}
\put(685,249.67){\rule{0.241pt}{0.400pt}}
\multiput(685.00,249.17)(0.500,1.000){2}{\rule{0.120pt}{0.400pt}}
\put(685.0,246.0){\rule[-0.200pt]{0.400pt}{0.964pt}}
\put(686,254.67){\rule{0.241pt}{0.400pt}}
\multiput(686.00,254.17)(0.500,1.000){2}{\rule{0.120pt}{0.400pt}}
\put(686.0,251.0){\rule[-0.200pt]{0.400pt}{0.964pt}}
\put(687,256){\usebox{\plotpoint}}
\put(687,258.67){\rule{0.241pt}{0.400pt}}
\multiput(687.00,258.17)(0.500,1.000){2}{\rule{0.120pt}{0.400pt}}
\put(687.0,256.0){\rule[-0.200pt]{0.400pt}{0.723pt}}
\put(688,263.67){\rule{0.241pt}{0.400pt}}
\multiput(688.00,263.17)(0.500,1.000){2}{\rule{0.120pt}{0.400pt}}
\put(688.0,260.0){\rule[-0.200pt]{0.400pt}{0.964pt}}
\put(689,265){\usebox{\plotpoint}}
\put(689,268.67){\rule{0.241pt}{0.400pt}}
\multiput(689.00,268.17)(0.500,1.000){2}{\rule{0.120pt}{0.400pt}}
\put(689.0,265.0){\rule[-0.200pt]{0.400pt}{0.964pt}}
\put(690,270){\usebox{\plotpoint}}
\put(690,273.67){\rule{0.241pt}{0.400pt}}
\multiput(690.00,273.17)(0.500,1.000){2}{\rule{0.120pt}{0.400pt}}
\put(690.0,270.0){\rule[-0.200pt]{0.400pt}{0.964pt}}
\put(691,278.67){\rule{0.241pt}{0.400pt}}
\multiput(691.00,278.17)(0.500,1.000){2}{\rule{0.120pt}{0.400pt}}
\put(691.0,275.0){\rule[-0.200pt]{0.400pt}{0.964pt}}
\put(692.0,280.0){\rule[-0.200pt]{0.400pt}{1.204pt}}
\put(692.0,285.0){\usebox{\plotpoint}}
\put(693,289.67){\rule{0.241pt}{0.400pt}}
\multiput(693.00,289.17)(0.500,1.000){2}{\rule{0.120pt}{0.400pt}}
\put(693.0,285.0){\rule[-0.200pt]{0.400pt}{1.204pt}}
\put(694,291){\usebox{\plotpoint}}
\put(694,294.67){\rule{0.241pt}{0.400pt}}
\multiput(694.00,294.17)(0.500,1.000){2}{\rule{0.120pt}{0.400pt}}
\put(694.0,291.0){\rule[-0.200pt]{0.400pt}{0.964pt}}
\put(695,300.67){\rule{0.241pt}{0.400pt}}
\multiput(695.00,300.17)(0.500,1.000){2}{\rule{0.120pt}{0.400pt}}
\put(695.0,296.0){\rule[-0.200pt]{0.400pt}{1.204pt}}
\put(696,302){\usebox{\plotpoint}}
\put(696.0,302.0){\rule[-0.200pt]{0.400pt}{0.964pt}}
\put(696.0,306.0){\usebox{\plotpoint}}
\put(697,310.67){\rule{0.241pt}{0.400pt}}
\multiput(697.00,310.17)(0.500,1.000){2}{\rule{0.120pt}{0.400pt}}
\put(697.0,306.0){\rule[-0.200pt]{0.400pt}{1.204pt}}
\put(698,316.67){\rule{0.241pt}{0.400pt}}
\multiput(698.00,316.17)(0.500,1.000){2}{\rule{0.120pt}{0.400pt}}
\put(698.0,312.0){\rule[-0.200pt]{0.400pt}{1.204pt}}
\put(699,322.67){\rule{0.241pt}{0.400pt}}
\multiput(699.00,322.17)(0.500,1.000){2}{\rule{0.120pt}{0.400pt}}
\put(699.0,318.0){\rule[-0.200pt]{0.400pt}{1.204pt}}
\put(700,324){\usebox{\plotpoint}}
\put(700.0,324.0){\rule[-0.200pt]{0.400pt}{1.204pt}}
\put(700.0,329.0){\usebox{\plotpoint}}
\put(701.0,329.0){\rule[-0.200pt]{0.400pt}{1.445pt}}
\put(701.0,335.0){\usebox{\plotpoint}}
\put(702.0,335.0){\rule[-0.200pt]{0.400pt}{1.445pt}}
\put(702.0,341.0){\usebox{\plotpoint}}
\put(703,346.67){\rule{0.241pt}{0.400pt}}
\multiput(703.00,346.17)(0.500,1.000){2}{\rule{0.120pt}{0.400pt}}
\put(703.0,341.0){\rule[-0.200pt]{0.400pt}{1.445pt}}
\put(704,348){\usebox{\plotpoint}}
\put(704,351.67){\rule{0.241pt}{0.400pt}}
\multiput(704.00,351.17)(0.500,1.000){2}{\rule{0.120pt}{0.400pt}}
\put(704.0,348.0){\rule[-0.200pt]{0.400pt}{0.964pt}}
\put(705,357.67){\rule{0.241pt}{0.400pt}}
\multiput(705.00,357.17)(0.500,1.000){2}{\rule{0.120pt}{0.400pt}}
\put(705.0,353.0){\rule[-0.200pt]{0.400pt}{1.204pt}}
\put(706.0,359.0){\rule[-0.200pt]{0.400pt}{1.445pt}}
\put(706.0,365.0){\usebox{\plotpoint}}
\put(707,370.67){\rule{0.241pt}{0.400pt}}
\multiput(707.00,370.17)(0.500,1.000){2}{\rule{0.120pt}{0.400pt}}
\put(707.0,365.0){\rule[-0.200pt]{0.400pt}{1.445pt}}
\put(708,372){\usebox{\plotpoint}}
\put(708,376.67){\rule{0.241pt}{0.400pt}}
\multiput(708.00,376.17)(0.500,1.000){2}{\rule{0.120pt}{0.400pt}}
\put(708.0,372.0){\rule[-0.200pt]{0.400pt}{1.204pt}}
\put(709,383.67){\rule{0.241pt}{0.400pt}}
\multiput(709.00,383.17)(0.500,1.000){2}{\rule{0.120pt}{0.400pt}}
\put(709.0,378.0){\rule[-0.200pt]{0.400pt}{1.445pt}}
\put(710,385){\usebox{\plotpoint}}
\put(710,389.67){\rule{0.241pt}{0.400pt}}
\multiput(710.00,389.17)(0.500,1.000){2}{\rule{0.120pt}{0.400pt}}
\put(710.0,385.0){\rule[-0.200pt]{0.400pt}{1.204pt}}
\put(711,396.67){\rule{0.241pt}{0.400pt}}
\multiput(711.00,396.17)(0.500,1.000){2}{\rule{0.120pt}{0.400pt}}
\put(711.0,391.0){\rule[-0.200pt]{0.400pt}{1.445pt}}
\put(712,398){\usebox{\plotpoint}}
\put(712,402.67){\rule{0.241pt}{0.400pt}}
\multiput(712.00,402.17)(0.500,1.000){2}{\rule{0.120pt}{0.400pt}}
\put(712.0,398.0){\rule[-0.200pt]{0.400pt}{1.204pt}}
\put(713,408.67){\rule{0.241pt}{0.400pt}}
\multiput(713.00,408.17)(0.500,1.000){2}{\rule{0.120pt}{0.400pt}}
\put(713.0,404.0){\rule[-0.200pt]{0.400pt}{1.204pt}}
\put(714,415.67){\rule{0.241pt}{0.400pt}}
\multiput(714.00,415.17)(0.500,1.000){2}{\rule{0.120pt}{0.400pt}}
\put(714.0,410.0){\rule[-0.200pt]{0.400pt}{1.445pt}}
\put(715.0,417.0){\rule[-0.200pt]{0.400pt}{1.445pt}}
\put(715.0,423.0){\usebox{\plotpoint}}
\put(716,428.67){\rule{0.241pt}{0.400pt}}
\multiput(716.00,428.17)(0.500,1.000){2}{\rule{0.120pt}{0.400pt}}
\put(716.0,423.0){\rule[-0.200pt]{0.400pt}{1.445pt}}
\put(717,435.67){\rule{0.241pt}{0.400pt}}
\multiput(717.00,435.17)(0.500,1.000){2}{\rule{0.120pt}{0.400pt}}
\put(717.0,430.0){\rule[-0.200pt]{0.400pt}{1.445pt}}
\put(718,442.67){\rule{0.241pt}{0.400pt}}
\multiput(718.00,442.17)(0.500,1.000){2}{\rule{0.120pt}{0.400pt}}
\put(718.0,437.0){\rule[-0.200pt]{0.400pt}{1.445pt}}
\put(719,449.67){\rule{0.241pt}{0.400pt}}
\multiput(719.00,449.17)(0.500,1.000){2}{\rule{0.120pt}{0.400pt}}
\put(719.0,444.0){\rule[-0.200pt]{0.400pt}{1.445pt}}
\put(720,456.67){\rule{0.241pt}{0.400pt}}
\multiput(720.00,456.17)(0.500,1.000){2}{\rule{0.120pt}{0.400pt}}
\put(720.0,451.0){\rule[-0.200pt]{0.400pt}{1.445pt}}
\put(721,458){\usebox{\plotpoint}}
\put(721,463.67){\rule{0.241pt}{0.400pt}}
\multiput(721.00,463.17)(0.500,1.000){2}{\rule{0.120pt}{0.400pt}}
\put(721.0,458.0){\rule[-0.200pt]{0.400pt}{1.445pt}}
\put(722,465){\usebox{\plotpoint}}
\put(722,469.67){\rule{0.241pt}{0.400pt}}
\multiput(722.00,469.17)(0.500,1.000){2}{\rule{0.120pt}{0.400pt}}
\put(722.0,465.0){\rule[-0.200pt]{0.400pt}{1.204pt}}
\put(723,476.67){\rule{0.241pt}{0.400pt}}
\multiput(723.00,476.17)(0.500,1.000){2}{\rule{0.120pt}{0.400pt}}
\put(723.0,471.0){\rule[-0.200pt]{0.400pt}{1.445pt}}
\put(724,483.67){\rule{0.241pt}{0.400pt}}
\multiput(724.00,483.17)(0.500,1.000){2}{\rule{0.120pt}{0.400pt}}
\put(724.0,478.0){\rule[-0.200pt]{0.400pt}{1.445pt}}
\put(725,490.67){\rule{0.241pt}{0.400pt}}
\multiput(725.00,490.17)(0.500,1.000){2}{\rule{0.120pt}{0.400pt}}
\put(725.0,485.0){\rule[-0.200pt]{0.400pt}{1.445pt}}
\put(726,497.67){\rule{0.241pt}{0.400pt}}
\multiput(726.00,497.17)(0.500,1.000){2}{\rule{0.120pt}{0.400pt}}
\put(726.0,492.0){\rule[-0.200pt]{0.400pt}{1.445pt}}
\put(727,504.67){\rule{0.241pt}{0.400pt}}
\multiput(727.00,504.17)(0.500,1.000){2}{\rule{0.120pt}{0.400pt}}
\put(727.0,499.0){\rule[-0.200pt]{0.400pt}{1.445pt}}
\put(728,511.67){\rule{0.241pt}{0.400pt}}
\multiput(728.00,511.17)(0.500,1.000){2}{\rule{0.120pt}{0.400pt}}
\put(728.0,506.0){\rule[-0.200pt]{0.400pt}{1.445pt}}
\put(729,518.67){\rule{0.241pt}{0.400pt}}
\multiput(729.00,518.17)(0.500,1.000){2}{\rule{0.120pt}{0.400pt}}
\put(729.0,513.0){\rule[-0.200pt]{0.400pt}{1.445pt}}
\put(730,525.67){\rule{0.241pt}{0.400pt}}
\multiput(730.00,525.17)(0.500,1.000){2}{\rule{0.120pt}{0.400pt}}
\put(730.0,520.0){\rule[-0.200pt]{0.400pt}{1.445pt}}
\put(731,531.67){\rule{0.241pt}{0.400pt}}
\multiput(731.00,531.17)(0.500,1.000){2}{\rule{0.120pt}{0.400pt}}
\put(731.0,527.0){\rule[-0.200pt]{0.400pt}{1.204pt}}
\put(732,538.67){\rule{0.241pt}{0.400pt}}
\multiput(732.00,538.17)(0.500,1.000){2}{\rule{0.120pt}{0.400pt}}
\put(732.0,533.0){\rule[-0.200pt]{0.400pt}{1.445pt}}
\put(733,545.67){\rule{0.241pt}{0.400pt}}
\multiput(733.00,545.17)(0.500,1.000){2}{\rule{0.120pt}{0.400pt}}
\put(733.0,540.0){\rule[-0.200pt]{0.400pt}{1.445pt}}
\put(734,552.67){\rule{0.241pt}{0.400pt}}
\multiput(734.00,552.17)(0.500,1.000){2}{\rule{0.120pt}{0.400pt}}
\put(734.0,547.0){\rule[-0.200pt]{0.400pt}{1.445pt}}
\put(735,559.67){\rule{0.241pt}{0.400pt}}
\multiput(735.00,559.17)(0.500,1.000){2}{\rule{0.120pt}{0.400pt}}
\put(735.0,554.0){\rule[-0.200pt]{0.400pt}{1.445pt}}
\put(736,566.67){\rule{0.241pt}{0.400pt}}
\multiput(736.00,566.17)(0.500,1.000){2}{\rule{0.120pt}{0.400pt}}
\put(736.0,561.0){\rule[-0.200pt]{0.400pt}{1.445pt}}
\put(737,573.67){\rule{0.241pt}{0.400pt}}
\multiput(737.00,573.17)(0.500,1.000){2}{\rule{0.120pt}{0.400pt}}
\put(737.0,568.0){\rule[-0.200pt]{0.400pt}{1.445pt}}
\put(738,580.67){\rule{0.241pt}{0.400pt}}
\multiput(738.00,580.17)(0.500,1.000){2}{\rule{0.120pt}{0.400pt}}
\put(738.0,575.0){\rule[-0.200pt]{0.400pt}{1.445pt}}
\put(739,586.67){\rule{0.241pt}{0.400pt}}
\multiput(739.00,586.17)(0.500,1.000){2}{\rule{0.120pt}{0.400pt}}
\put(739.0,582.0){\rule[-0.200pt]{0.400pt}{1.204pt}}
\put(740,593.67){\rule{0.241pt}{0.400pt}}
\multiput(740.00,593.17)(0.500,1.000){2}{\rule{0.120pt}{0.400pt}}
\put(740.0,588.0){\rule[-0.200pt]{0.400pt}{1.445pt}}
\put(741,600.67){\rule{0.241pt}{0.400pt}}
\multiput(741.00,600.17)(0.500,1.000){2}{\rule{0.120pt}{0.400pt}}
\put(741.0,595.0){\rule[-0.200pt]{0.400pt}{1.445pt}}
\put(742,607.67){\rule{0.241pt}{0.400pt}}
\multiput(742.00,607.17)(0.500,1.000){2}{\rule{0.120pt}{0.400pt}}
\put(742.0,602.0){\rule[-0.200pt]{0.400pt}{1.445pt}}
\put(743,614.67){\rule{0.241pt}{0.400pt}}
\multiput(743.00,614.17)(0.500,1.000){2}{\rule{0.120pt}{0.400pt}}
\put(743.0,609.0){\rule[-0.200pt]{0.400pt}{1.445pt}}
\put(744,621.67){\rule{0.241pt}{0.400pt}}
\multiput(744.00,621.17)(0.500,1.000){2}{\rule{0.120pt}{0.400pt}}
\put(744.0,616.0){\rule[-0.200pt]{0.400pt}{1.445pt}}
\put(745,623){\usebox{\plotpoint}}
\put(745.0,623.0){\rule[-0.200pt]{0.400pt}{1.445pt}}
\put(745.0,629.0){\usebox{\plotpoint}}
\put(746,634.67){\rule{0.241pt}{0.400pt}}
\multiput(746.00,634.17)(0.500,1.000){2}{\rule{0.120pt}{0.400pt}}
\put(746.0,629.0){\rule[-0.200pt]{0.400pt}{1.445pt}}
\put(747,641.67){\rule{0.241pt}{0.400pt}}
\multiput(747.00,641.17)(0.500,1.000){2}{\rule{0.120pt}{0.400pt}}
\put(747.0,636.0){\rule[-0.200pt]{0.400pt}{1.445pt}}
\put(748,647.67){\rule{0.241pt}{0.400pt}}
\multiput(748.00,647.17)(0.500,1.000){2}{\rule{0.120pt}{0.400pt}}
\put(748.0,643.0){\rule[-0.200pt]{0.400pt}{1.204pt}}
\put(749,649){\usebox{\plotpoint}}
\put(749,653.67){\rule{0.241pt}{0.400pt}}
\multiput(749.00,653.17)(0.500,1.000){2}{\rule{0.120pt}{0.400pt}}
\put(749.0,649.0){\rule[-0.200pt]{0.400pt}{1.204pt}}
\put(750,660.67){\rule{0.241pt}{0.400pt}}
\multiput(750.00,660.17)(0.500,1.000){2}{\rule{0.120pt}{0.400pt}}
\put(750.0,655.0){\rule[-0.200pt]{0.400pt}{1.445pt}}
\put(751,662){\usebox{\plotpoint}}
\put(751,666.67){\rule{0.241pt}{0.400pt}}
\multiput(751.00,666.17)(0.500,1.000){2}{\rule{0.120pt}{0.400pt}}
\put(751.0,662.0){\rule[-0.200pt]{0.400pt}{1.204pt}}
\put(752.0,668.0){\rule[-0.200pt]{0.400pt}{1.445pt}}
\put(752.0,674.0){\usebox{\plotpoint}}
\put(753,679.67){\rule{0.241pt}{0.400pt}}
\multiput(753.00,679.17)(0.500,1.000){2}{\rule{0.120pt}{0.400pt}}
\put(753.0,674.0){\rule[-0.200pt]{0.400pt}{1.445pt}}
\put(754,685.67){\rule{0.241pt}{0.400pt}}
\multiput(754.00,685.17)(0.500,1.000){2}{\rule{0.120pt}{0.400pt}}
\put(754.0,681.0){\rule[-0.200pt]{0.400pt}{1.204pt}}
\put(755,691.67){\rule{0.241pt}{0.400pt}}
\multiput(755.00,691.17)(0.500,1.000){2}{\rule{0.120pt}{0.400pt}}
\put(755.0,687.0){\rule[-0.200pt]{0.400pt}{1.204pt}}
\put(756.0,693.0){\rule[-0.200pt]{0.400pt}{1.445pt}}
\put(756.0,699.0){\usebox{\plotpoint}}
\put(757,703.67){\rule{0.241pt}{0.400pt}}
\multiput(757.00,703.17)(0.500,1.000){2}{\rule{0.120pt}{0.400pt}}
\put(757.0,699.0){\rule[-0.200pt]{0.400pt}{1.204pt}}
\put(758,705){\usebox{\plotpoint}}
\put(758,709.67){\rule{0.241pt}{0.400pt}}
\multiput(758.00,709.17)(0.500,1.000){2}{\rule{0.120pt}{0.400pt}}
\put(758.0,705.0){\rule[-0.200pt]{0.400pt}{1.204pt}}
\put(759,711){\usebox{\plotpoint}}
\put(759.0,711.0){\rule[-0.200pt]{0.400pt}{1.204pt}}
\put(759.0,716.0){\usebox{\plotpoint}}
\put(760,720.67){\rule{0.241pt}{0.400pt}}
\multiput(760.00,720.17)(0.500,1.000){2}{\rule{0.120pt}{0.400pt}}
\put(760.0,716.0){\rule[-0.200pt]{0.400pt}{1.204pt}}
\put(761,726.67){\rule{0.241pt}{0.400pt}}
\multiput(761.00,726.17)(0.500,1.000){2}{\rule{0.120pt}{0.400pt}}
\put(761.0,722.0){\rule[-0.200pt]{0.400pt}{1.204pt}}
\put(762,732.67){\rule{0.241pt}{0.400pt}}
\multiput(762.00,732.17)(0.500,1.000){2}{\rule{0.120pt}{0.400pt}}
\put(762.0,728.0){\rule[-0.200pt]{0.400pt}{1.204pt}}
\put(763,734){\usebox{\plotpoint}}
\put(763,737.67){\rule{0.241pt}{0.400pt}}
\multiput(763.00,737.17)(0.500,1.000){2}{\rule{0.120pt}{0.400pt}}
\put(763.0,734.0){\rule[-0.200pt]{0.400pt}{0.964pt}}
\put(764.0,739.0){\rule[-0.200pt]{0.400pt}{1.204pt}}
\put(764.0,744.0){\usebox{\plotpoint}}
\put(765,748.67){\rule{0.241pt}{0.400pt}}
\multiput(765.00,748.17)(0.500,1.000){2}{\rule{0.120pt}{0.400pt}}
\put(765.0,744.0){\rule[-0.200pt]{0.400pt}{1.204pt}}
\put(766,750){\usebox{\plotpoint}}
\put(766.0,750.0){\rule[-0.200pt]{0.400pt}{0.964pt}}
\put(766.0,754.0){\usebox{\plotpoint}}
\put(767.0,754.0){\rule[-0.200pt]{0.400pt}{1.204pt}}
\put(767.0,759.0){\usebox{\plotpoint}}
\put(768,763.67){\rule{0.241pt}{0.400pt}}
\multiput(768.00,763.17)(0.500,1.000){2}{\rule{0.120pt}{0.400pt}}
\put(768.0,759.0){\rule[-0.200pt]{0.400pt}{1.204pt}}
\put(769,765){\usebox{\plotpoint}}
\put(769.0,765.0){\rule[-0.200pt]{0.400pt}{0.964pt}}
\put(769.0,769.0){\usebox{\plotpoint}}
\put(770.0,769.0){\rule[-0.200pt]{0.400pt}{1.204pt}}
\put(770.0,774.0){\usebox{\plotpoint}}
\put(771,777.67){\rule{0.241pt}{0.400pt}}
\multiput(771.00,777.17)(0.500,1.000){2}{\rule{0.120pt}{0.400pt}}
\put(771.0,774.0){\rule[-0.200pt]{0.400pt}{0.964pt}}
\put(772,782.67){\rule{0.241pt}{0.400pt}}
\multiput(772.00,782.17)(0.500,1.000){2}{\rule{0.120pt}{0.400pt}}
\put(772.0,779.0){\rule[-0.200pt]{0.400pt}{0.964pt}}
\put(773,784){\usebox{\plotpoint}}
\put(773,786.67){\rule{0.241pt}{0.400pt}}
\multiput(773.00,786.17)(0.500,1.000){2}{\rule{0.120pt}{0.400pt}}
\put(773.0,784.0){\rule[-0.200pt]{0.400pt}{0.723pt}}
\put(774,790.67){\rule{0.241pt}{0.400pt}}
\multiput(774.00,790.17)(0.500,1.000){2}{\rule{0.120pt}{0.400pt}}
\put(774.0,788.0){\rule[-0.200pt]{0.400pt}{0.723pt}}
\put(775,792){\usebox{\plotpoint}}
\put(775.0,792.0){\rule[-0.200pt]{0.400pt}{0.964pt}}
\put(775.0,796.0){\usebox{\plotpoint}}
\put(776.0,796.0){\rule[-0.200pt]{0.400pt}{0.964pt}}
\put(776.0,800.0){\usebox{\plotpoint}}
\put(777.0,800.0){\rule[-0.200pt]{0.400pt}{0.964pt}}
\put(777.0,804.0){\usebox{\plotpoint}}
\put(778.0,804.0){\rule[-0.200pt]{0.400pt}{0.964pt}}
\put(778.0,808.0){\usebox{\plotpoint}}
\put(779,810.67){\rule{0.241pt}{0.400pt}}
\multiput(779.00,810.17)(0.500,1.000){2}{\rule{0.120pt}{0.400pt}}
\put(779.0,808.0){\rule[-0.200pt]{0.400pt}{0.723pt}}
\put(780,812){\usebox{\plotpoint}}
\put(780.0,812.0){\rule[-0.200pt]{0.400pt}{0.723pt}}
\put(780.0,815.0){\usebox{\plotpoint}}
\put(781,817.67){\rule{0.241pt}{0.400pt}}
\multiput(781.00,817.17)(0.500,1.000){2}{\rule{0.120pt}{0.400pt}}
\put(781.0,815.0){\rule[-0.200pt]{0.400pt}{0.723pt}}
\put(782,819){\usebox{\plotpoint}}
\put(782.0,819.0){\rule[-0.200pt]{0.400pt}{0.723pt}}
\put(782.0,822.0){\usebox{\plotpoint}}
\put(783.0,822.0){\rule[-0.200pt]{0.400pt}{0.723pt}}
\put(783.0,825.0){\usebox{\plotpoint}}
\put(784.0,825.0){\rule[-0.200pt]{0.400pt}{0.723pt}}
\put(784.0,828.0){\usebox{\plotpoint}}
\put(785.0,828.0){\rule[-0.200pt]{0.400pt}{0.723pt}}
\put(785.0,831.0){\usebox{\plotpoint}}
\put(786,832.67){\rule{0.241pt}{0.400pt}}
\multiput(786.00,832.17)(0.500,1.000){2}{\rule{0.120pt}{0.400pt}}
\put(786.0,831.0){\rule[-0.200pt]{0.400pt}{0.482pt}}
\put(787,834){\usebox{\plotpoint}}
\put(787,834){\usebox{\plotpoint}}
\put(787.0,834.0){\rule[-0.200pt]{0.400pt}{0.482pt}}
\put(787.0,836.0){\usebox{\plotpoint}}
\put(788.0,836.0){\rule[-0.200pt]{0.400pt}{0.723pt}}
\put(788.0,839.0){\usebox{\plotpoint}}
\put(789.0,839.0){\rule[-0.200pt]{0.400pt}{0.482pt}}
\put(789.0,841.0){\usebox{\plotpoint}}
\put(790,842.67){\rule{0.241pt}{0.400pt}}
\multiput(790.00,842.17)(0.500,1.000){2}{\rule{0.120pt}{0.400pt}}
\put(790.0,841.0){\rule[-0.200pt]{0.400pt}{0.482pt}}
\put(791,844){\usebox{\plotpoint}}
\put(791,844){\usebox{\plotpoint}}
\put(791,844){\usebox{\plotpoint}}
\put(791,844.67){\rule{0.241pt}{0.400pt}}
\multiput(791.00,844.17)(0.500,1.000){2}{\rule{0.120pt}{0.400pt}}
\put(791.0,844.0){\usebox{\plotpoint}}
\put(792,846){\usebox{\plotpoint}}
\put(792,846){\usebox{\plotpoint}}
\put(792,846){\usebox{\plotpoint}}
\put(792.0,846.0){\usebox{\plotpoint}}
\put(792.0,847.0){\usebox{\plotpoint}}
\put(793.0,847.0){\rule[-0.200pt]{0.400pt}{0.482pt}}
\put(793.0,849.0){\usebox{\plotpoint}}
\put(794.0,849.0){\rule[-0.200pt]{0.400pt}{0.482pt}}
\put(794.0,851.0){\usebox{\plotpoint}}
\put(795.0,851.0){\usebox{\plotpoint}}
\put(795.0,852.0){\usebox{\plotpoint}}
\put(796.0,852.0){\rule[-0.200pt]{0.400pt}{0.482pt}}
\put(796.0,854.0){\usebox{\plotpoint}}
\put(797.0,854.0){\usebox{\plotpoint}}
\put(797.0,855.0){\usebox{\plotpoint}}
\put(798.0,855.0){\usebox{\plotpoint}}
\put(798.0,856.0){\usebox{\plotpoint}}
\put(799.0,856.0){\usebox{\plotpoint}}
\put(800,856.67){\rule{0.241pt}{0.400pt}}
\multiput(800.00,856.17)(0.500,1.000){2}{\rule{0.120pt}{0.400pt}}
\put(799.0,857.0){\usebox{\plotpoint}}
\put(801,858){\usebox{\plotpoint}}
\put(801,858){\usebox{\plotpoint}}
\put(801,858){\usebox{\plotpoint}}
\put(801,858){\usebox{\plotpoint}}
\put(801,858){\usebox{\plotpoint}}
\put(801,858){\usebox{\plotpoint}}
\put(802,857.67){\rule{0.241pt}{0.400pt}}
\multiput(802.00,857.17)(0.500,1.000){2}{\rule{0.120pt}{0.400pt}}
\put(801.0,858.0){\usebox{\plotpoint}}
\put(803,859){\usebox{\plotpoint}}
\put(803,859){\usebox{\plotpoint}}
\put(803,859){\usebox{\plotpoint}}
\put(803,859){\usebox{\plotpoint}}
\put(803,859){\usebox{\plotpoint}}
\put(803,859){\usebox{\plotpoint}}
\put(803,859){\usebox{\plotpoint}}
\put(807,857.67){\rule{0.241pt}{0.400pt}}
\multiput(807.00,858.17)(0.500,-1.000){2}{\rule{0.120pt}{0.400pt}}
\put(803.0,859.0){\rule[-0.200pt]{0.964pt}{0.400pt}}
\put(808,858){\usebox{\plotpoint}}
\put(808,858){\usebox{\plotpoint}}
\put(808,858){\usebox{\plotpoint}}
\put(808,858){\usebox{\plotpoint}}
\put(808,858){\usebox{\plotpoint}}
\put(808,858){\usebox{\plotpoint}}
\put(808,858){\usebox{\plotpoint}}
\put(809,856.67){\rule{0.241pt}{0.400pt}}
\multiput(809.00,857.17)(0.500,-1.000){2}{\rule{0.120pt}{0.400pt}}
\put(808.0,858.0){\usebox{\plotpoint}}
\put(810,857){\usebox{\plotpoint}}
\put(810,857){\usebox{\plotpoint}}
\put(810,857){\usebox{\plotpoint}}
\put(810,857){\usebox{\plotpoint}}
\put(810,857){\usebox{\plotpoint}}
\put(810,857){\usebox{\plotpoint}}
\put(810,857){\usebox{\plotpoint}}
\put(810.0,857.0){\usebox{\plotpoint}}
\put(811.0,856.0){\usebox{\plotpoint}}
\put(811.0,856.0){\usebox{\plotpoint}}
\put(812.0,855.0){\usebox{\plotpoint}}
\put(812.0,855.0){\usebox{\plotpoint}}
\put(813.0,854.0){\usebox{\plotpoint}}
\put(813.0,854.0){\usebox{\plotpoint}}
\put(814.0,852.0){\rule[-0.200pt]{0.400pt}{0.482pt}}
\put(814.0,852.0){\usebox{\plotpoint}}
\put(815.0,851.0){\usebox{\plotpoint}}
\put(815.0,851.0){\usebox{\plotpoint}}
\put(816.0,849.0){\rule[-0.200pt]{0.400pt}{0.482pt}}
\put(816.0,849.0){\usebox{\plotpoint}}
\put(817.0,847.0){\rule[-0.200pt]{0.400pt}{0.482pt}}
\put(817.0,847.0){\usebox{\plotpoint}}
\put(818,844.67){\rule{0.241pt}{0.400pt}}
\multiput(818.00,845.17)(0.500,-1.000){2}{\rule{0.120pt}{0.400pt}}
\put(818.0,846.0){\usebox{\plotpoint}}
\put(819,845){\usebox{\plotpoint}}
\put(819,845){\usebox{\plotpoint}}
\put(819,845){\usebox{\plotpoint}}
\put(819,842.67){\rule{0.241pt}{0.400pt}}
\multiput(819.00,843.17)(0.500,-1.000){2}{\rule{0.120pt}{0.400pt}}
\put(819.0,844.0){\usebox{\plotpoint}}
\put(820,843){\usebox{\plotpoint}}
\put(820,843){\usebox{\plotpoint}}
\put(820.0,841.0){\rule[-0.200pt]{0.400pt}{0.482pt}}
\put(820.0,841.0){\usebox{\plotpoint}}
\put(821.0,839.0){\rule[-0.200pt]{0.400pt}{0.482pt}}
\put(821.0,839.0){\usebox{\plotpoint}}
\put(822.0,836.0){\rule[-0.200pt]{0.400pt}{0.723pt}}
\put(822.0,836.0){\usebox{\plotpoint}}
\put(823,832.67){\rule{0.241pt}{0.400pt}}
\multiput(823.00,833.17)(0.500,-1.000){2}{\rule{0.120pt}{0.400pt}}
\put(823.0,834.0){\rule[-0.200pt]{0.400pt}{0.482pt}}
\put(824,833){\usebox{\plotpoint}}
\put(824,833){\usebox{\plotpoint}}
\put(824.0,831.0){\rule[-0.200pt]{0.400pt}{0.482pt}}
\put(824.0,831.0){\usebox{\plotpoint}}
\put(825.0,828.0){\rule[-0.200pt]{0.400pt}{0.723pt}}
\put(825.0,828.0){\usebox{\plotpoint}}
\put(826.0,825.0){\rule[-0.200pt]{0.400pt}{0.723pt}}
\put(826.0,825.0){\usebox{\plotpoint}}
\put(827.0,822.0){\rule[-0.200pt]{0.400pt}{0.723pt}}
\put(827.0,822.0){\usebox{\plotpoint}}
\put(828,817.67){\rule{0.241pt}{0.400pt}}
\multiput(828.00,818.17)(0.500,-1.000){2}{\rule{0.120pt}{0.400pt}}
\put(828.0,819.0){\rule[-0.200pt]{0.400pt}{0.723pt}}
\put(829,818){\usebox{\plotpoint}}
\put(829,818){\usebox{\plotpoint}}
\put(829.0,815.0){\rule[-0.200pt]{0.400pt}{0.723pt}}
\put(829.0,815.0){\usebox{\plotpoint}}
\put(830,810.67){\rule{0.241pt}{0.400pt}}
\multiput(830.00,811.17)(0.500,-1.000){2}{\rule{0.120pt}{0.400pt}}
\put(830.0,812.0){\rule[-0.200pt]{0.400pt}{0.723pt}}
\put(831,811){\usebox{\plotpoint}}
\put(831.0,808.0){\rule[-0.200pt]{0.400pt}{0.723pt}}
\put(831.0,808.0){\usebox{\plotpoint}}
\put(832.0,804.0){\rule[-0.200pt]{0.400pt}{0.964pt}}
\put(832.0,804.0){\usebox{\plotpoint}}
\put(833.0,800.0){\rule[-0.200pt]{0.400pt}{0.964pt}}
\put(833.0,800.0){\usebox{\plotpoint}}
\put(834.0,796.0){\rule[-0.200pt]{0.400pt}{0.964pt}}
\put(834.0,796.0){\usebox{\plotpoint}}
\put(835,790.67){\rule{0.241pt}{0.400pt}}
\multiput(835.00,791.17)(0.500,-1.000){2}{\rule{0.120pt}{0.400pt}}
\put(835.0,792.0){\rule[-0.200pt]{0.400pt}{0.964pt}}
\put(836,791){\usebox{\plotpoint}}
\put(836,786.67){\rule{0.241pt}{0.400pt}}
\multiput(836.00,787.17)(0.500,-1.000){2}{\rule{0.120pt}{0.400pt}}
\put(836.0,788.0){\rule[-0.200pt]{0.400pt}{0.723pt}}
\put(837,787){\usebox{\plotpoint}}
\put(837,782.67){\rule{0.241pt}{0.400pt}}
\multiput(837.00,783.17)(0.500,-1.000){2}{\rule{0.120pt}{0.400pt}}
\put(837.0,784.0){\rule[-0.200pt]{0.400pt}{0.723pt}}
\put(838,777.67){\rule{0.241pt}{0.400pt}}
\multiput(838.00,778.17)(0.500,-1.000){2}{\rule{0.120pt}{0.400pt}}
\put(838.0,779.0){\rule[-0.200pt]{0.400pt}{0.964pt}}
\put(839,778){\usebox{\plotpoint}}
\put(839.0,774.0){\rule[-0.200pt]{0.400pt}{0.964pt}}
\put(839.0,774.0){\usebox{\plotpoint}}
\put(840.0,769.0){\rule[-0.200pt]{0.400pt}{1.204pt}}
\put(840.0,769.0){\usebox{\plotpoint}}
\put(841,763.67){\rule{0.241pt}{0.400pt}}
\multiput(841.00,764.17)(0.500,-1.000){2}{\rule{0.120pt}{0.400pt}}
\put(841.0,765.0){\rule[-0.200pt]{0.400pt}{0.964pt}}
\put(842.0,759.0){\rule[-0.200pt]{0.400pt}{1.204pt}}
\put(842.0,759.0){\usebox{\plotpoint}}
\put(843.0,754.0){\rule[-0.200pt]{0.400pt}{1.204pt}}
\put(843.0,754.0){\usebox{\plotpoint}}
\put(844,748.67){\rule{0.241pt}{0.400pt}}
\multiput(844.00,749.17)(0.500,-1.000){2}{\rule{0.120pt}{0.400pt}}
\put(844.0,750.0){\rule[-0.200pt]{0.400pt}{0.964pt}}
\put(845.0,744.0){\rule[-0.200pt]{0.400pt}{1.204pt}}
\put(845.0,744.0){\usebox{\plotpoint}}
\put(846,737.67){\rule{0.241pt}{0.400pt}}
\multiput(846.00,738.17)(0.500,-1.000){2}{\rule{0.120pt}{0.400pt}}
\put(846.0,739.0){\rule[-0.200pt]{0.400pt}{1.204pt}}
\put(847,738){\usebox{\plotpoint}}
\put(847,732.67){\rule{0.241pt}{0.400pt}}
\multiput(847.00,733.17)(0.500,-1.000){2}{\rule{0.120pt}{0.400pt}}
\put(847.0,734.0){\rule[-0.200pt]{0.400pt}{0.964pt}}
\put(848,726.67){\rule{0.241pt}{0.400pt}}
\multiput(848.00,727.17)(0.500,-1.000){2}{\rule{0.120pt}{0.400pt}}
\put(848.0,728.0){\rule[-0.200pt]{0.400pt}{1.204pt}}
\put(849,720.67){\rule{0.241pt}{0.400pt}}
\multiput(849.00,721.17)(0.500,-1.000){2}{\rule{0.120pt}{0.400pt}}
\put(849.0,722.0){\rule[-0.200pt]{0.400pt}{1.204pt}}
\put(850,721){\usebox{\plotpoint}}
\put(850.0,716.0){\rule[-0.200pt]{0.400pt}{1.204pt}}
\put(850.0,716.0){\usebox{\plotpoint}}
\put(851,709.67){\rule{0.241pt}{0.400pt}}
\multiput(851.00,710.17)(0.500,-1.000){2}{\rule{0.120pt}{0.400pt}}
\put(851.0,711.0){\rule[-0.200pt]{0.400pt}{1.204pt}}
\put(852,703.67){\rule{0.241pt}{0.400pt}}
\multiput(852.00,704.17)(0.500,-1.000){2}{\rule{0.120pt}{0.400pt}}
\put(852.0,705.0){\rule[-0.200pt]{0.400pt}{1.204pt}}
\put(853.0,699.0){\rule[-0.200pt]{0.400pt}{1.204pt}}
\put(853.0,699.0){\usebox{\plotpoint}}
\put(854,691.67){\rule{0.241pt}{0.400pt}}
\multiput(854.00,692.17)(0.500,-1.000){2}{\rule{0.120pt}{0.400pt}}
\put(854.0,693.0){\rule[-0.200pt]{0.400pt}{1.445pt}}
\put(855,692){\usebox{\plotpoint}}
\put(855,685.67){\rule{0.241pt}{0.400pt}}
\multiput(855.00,686.17)(0.500,-1.000){2}{\rule{0.120pt}{0.400pt}}
\put(855.0,687.0){\rule[-0.200pt]{0.400pt}{1.204pt}}
\put(856,679.67){\rule{0.241pt}{0.400pt}}
\multiput(856.00,680.17)(0.500,-1.000){2}{\rule{0.120pt}{0.400pt}}
\put(856.0,681.0){\rule[-0.200pt]{0.400pt}{1.204pt}}
\put(857.0,674.0){\rule[-0.200pt]{0.400pt}{1.445pt}}
\put(857.0,674.0){\usebox{\plotpoint}}
\put(858,666.67){\rule{0.241pt}{0.400pt}}
\multiput(858.00,667.17)(0.500,-1.000){2}{\rule{0.120pt}{0.400pt}}
\put(858.0,668.0){\rule[-0.200pt]{0.400pt}{1.445pt}}
\put(859,660.67){\rule{0.241pt}{0.400pt}}
\multiput(859.00,661.17)(0.500,-1.000){2}{\rule{0.120pt}{0.400pt}}
\put(859.0,662.0){\rule[-0.200pt]{0.400pt}{1.204pt}}
\put(860,653.67){\rule{0.241pt}{0.400pt}}
\multiput(860.00,654.17)(0.500,-1.000){2}{\rule{0.120pt}{0.400pt}}
\put(860.0,655.0){\rule[-0.200pt]{0.400pt}{1.445pt}}
\put(861,647.67){\rule{0.241pt}{0.400pt}}
\multiput(861.00,648.17)(0.500,-1.000){2}{\rule{0.120pt}{0.400pt}}
\put(861.0,649.0){\rule[-0.200pt]{0.400pt}{1.204pt}}
\put(862,641.67){\rule{0.241pt}{0.400pt}}
\multiput(862.00,642.17)(0.500,-1.000){2}{\rule{0.120pt}{0.400pt}}
\put(862.0,643.0){\rule[-0.200pt]{0.400pt}{1.204pt}}
\put(863,634.67){\rule{0.241pt}{0.400pt}}
\multiput(863.00,635.17)(0.500,-1.000){2}{\rule{0.120pt}{0.400pt}}
\put(863.0,636.0){\rule[-0.200pt]{0.400pt}{1.445pt}}
\put(864.0,629.0){\rule[-0.200pt]{0.400pt}{1.445pt}}
\put(864.0,629.0){\usebox{\plotpoint}}
\put(865,621.67){\rule{0.241pt}{0.400pt}}
\multiput(865.00,622.17)(0.500,-1.000){2}{\rule{0.120pt}{0.400pt}}
\put(865.0,623.0){\rule[-0.200pt]{0.400pt}{1.445pt}}
\put(866,614.67){\rule{0.241pt}{0.400pt}}
\multiput(866.00,615.17)(0.500,-1.000){2}{\rule{0.120pt}{0.400pt}}
\put(866.0,616.0){\rule[-0.200pt]{0.400pt}{1.445pt}}
\put(867,607.67){\rule{0.241pt}{0.400pt}}
\multiput(867.00,608.17)(0.500,-1.000){2}{\rule{0.120pt}{0.400pt}}
\put(867.0,609.0){\rule[-0.200pt]{0.400pt}{1.445pt}}
\put(868,600.67){\rule{0.241pt}{0.400pt}}
\multiput(868.00,601.17)(0.500,-1.000){2}{\rule{0.120pt}{0.400pt}}
\put(868.0,602.0){\rule[-0.200pt]{0.400pt}{1.445pt}}
\put(869,593.67){\rule{0.241pt}{0.400pt}}
\multiput(869.00,594.17)(0.500,-1.000){2}{\rule{0.120pt}{0.400pt}}
\put(869.0,595.0){\rule[-0.200pt]{0.400pt}{1.445pt}}
\put(870,594){\usebox{\plotpoint}}
\put(870,586.67){\rule{0.241pt}{0.400pt}}
\multiput(870.00,587.17)(0.500,-1.000){2}{\rule{0.120pt}{0.400pt}}
\put(870.0,588.0){\rule[-0.200pt]{0.400pt}{1.445pt}}
\put(871,587){\usebox{\plotpoint}}
\put(871,580.67){\rule{0.241pt}{0.400pt}}
\multiput(871.00,581.17)(0.500,-1.000){2}{\rule{0.120pt}{0.400pt}}
\put(871.0,582.0){\rule[-0.200pt]{0.400pt}{1.204pt}}
\put(872,573.67){\rule{0.241pt}{0.400pt}}
\multiput(872.00,574.17)(0.500,-1.000){2}{\rule{0.120pt}{0.400pt}}
\put(872.0,575.0){\rule[-0.200pt]{0.400pt}{1.445pt}}
\put(873,574){\usebox{\plotpoint}}
\put(873,566.67){\rule{0.241pt}{0.400pt}}
\multiput(873.00,567.17)(0.500,-1.000){2}{\rule{0.120pt}{0.400pt}}
\put(873.0,568.0){\rule[-0.200pt]{0.400pt}{1.445pt}}
\put(874,567){\usebox{\plotpoint}}
\put(874,559.67){\rule{0.241pt}{0.400pt}}
\multiput(874.00,560.17)(0.500,-1.000){2}{\rule{0.120pt}{0.400pt}}
\put(874.0,561.0){\rule[-0.200pt]{0.400pt}{1.445pt}}
\put(875,552.67){\rule{0.241pt}{0.400pt}}
\multiput(875.00,553.17)(0.500,-1.000){2}{\rule{0.120pt}{0.400pt}}
\put(875.0,554.0){\rule[-0.200pt]{0.400pt}{1.445pt}}
\put(876,545.67){\rule{0.241pt}{0.400pt}}
\multiput(876.00,546.17)(0.500,-1.000){2}{\rule{0.120pt}{0.400pt}}
\put(876.0,547.0){\rule[-0.200pt]{0.400pt}{1.445pt}}
\put(877,538.67){\rule{0.241pt}{0.400pt}}
\multiput(877.00,539.17)(0.500,-1.000){2}{\rule{0.120pt}{0.400pt}}
\put(877.0,540.0){\rule[-0.200pt]{0.400pt}{1.445pt}}
\put(878,531.67){\rule{0.241pt}{0.400pt}}
\multiput(878.00,532.17)(0.500,-1.000){2}{\rule{0.120pt}{0.400pt}}
\put(878.0,533.0){\rule[-0.200pt]{0.400pt}{1.445pt}}
\put(879,525.67){\rule{0.241pt}{0.400pt}}
\multiput(879.00,526.17)(0.500,-1.000){2}{\rule{0.120pt}{0.400pt}}
\put(879.0,527.0){\rule[-0.200pt]{0.400pt}{1.204pt}}
\put(880,518.67){\rule{0.241pt}{0.400pt}}
\multiput(880.00,519.17)(0.500,-1.000){2}{\rule{0.120pt}{0.400pt}}
\put(880.0,520.0){\rule[-0.200pt]{0.400pt}{1.445pt}}
\put(881,511.67){\rule{0.241pt}{0.400pt}}
\multiput(881.00,512.17)(0.500,-1.000){2}{\rule{0.120pt}{0.400pt}}
\put(881.0,513.0){\rule[-0.200pt]{0.400pt}{1.445pt}}
\put(882,504.67){\rule{0.241pt}{0.400pt}}
\multiput(882.00,505.17)(0.500,-1.000){2}{\rule{0.120pt}{0.400pt}}
\put(882.0,506.0){\rule[-0.200pt]{0.400pt}{1.445pt}}
\put(883,497.67){\rule{0.241pt}{0.400pt}}
\multiput(883.00,498.17)(0.500,-1.000){2}{\rule{0.120pt}{0.400pt}}
\put(883.0,499.0){\rule[-0.200pt]{0.400pt}{1.445pt}}
\put(884,490.67){\rule{0.241pt}{0.400pt}}
\multiput(884.00,491.17)(0.500,-1.000){2}{\rule{0.120pt}{0.400pt}}
\put(884.0,492.0){\rule[-0.200pt]{0.400pt}{1.445pt}}
\put(885,483.67){\rule{0.241pt}{0.400pt}}
\multiput(885.00,484.17)(0.500,-1.000){2}{\rule{0.120pt}{0.400pt}}
\put(885.0,485.0){\rule[-0.200pt]{0.400pt}{1.445pt}}
\put(886,476.67){\rule{0.241pt}{0.400pt}}
\multiput(886.00,477.17)(0.500,-1.000){2}{\rule{0.120pt}{0.400pt}}
\put(886.0,478.0){\rule[-0.200pt]{0.400pt}{1.445pt}}
\put(887,469.67){\rule{0.241pt}{0.400pt}}
\multiput(887.00,470.17)(0.500,-1.000){2}{\rule{0.120pt}{0.400pt}}
\put(887.0,471.0){\rule[-0.200pt]{0.400pt}{1.445pt}}
\put(888,463.67){\rule{0.241pt}{0.400pt}}
\multiput(888.00,464.17)(0.500,-1.000){2}{\rule{0.120pt}{0.400pt}}
\put(888.0,465.0){\rule[-0.200pt]{0.400pt}{1.204pt}}
\put(889,456.67){\rule{0.241pt}{0.400pt}}
\multiput(889.00,457.17)(0.500,-1.000){2}{\rule{0.120pt}{0.400pt}}
\put(889.0,458.0){\rule[-0.200pt]{0.400pt}{1.445pt}}
\put(890,449.67){\rule{0.241pt}{0.400pt}}
\multiput(890.00,450.17)(0.500,-1.000){2}{\rule{0.120pt}{0.400pt}}
\put(890.0,451.0){\rule[-0.200pt]{0.400pt}{1.445pt}}
\put(891,442.67){\rule{0.241pt}{0.400pt}}
\multiput(891.00,443.17)(0.500,-1.000){2}{\rule{0.120pt}{0.400pt}}
\put(891.0,444.0){\rule[-0.200pt]{0.400pt}{1.445pt}}
\put(892,435.67){\rule{0.241pt}{0.400pt}}
\multiput(892.00,436.17)(0.500,-1.000){2}{\rule{0.120pt}{0.400pt}}
\put(892.0,437.0){\rule[-0.200pt]{0.400pt}{1.445pt}}
\put(893,428.67){\rule{0.241pt}{0.400pt}}
\multiput(893.00,429.17)(0.500,-1.000){2}{\rule{0.120pt}{0.400pt}}
\put(893.0,430.0){\rule[-0.200pt]{0.400pt}{1.445pt}}
\put(894.0,423.0){\rule[-0.200pt]{0.400pt}{1.445pt}}
\put(894.0,423.0){\usebox{\plotpoint}}
\put(895,415.67){\rule{0.241pt}{0.400pt}}
\multiput(895.00,416.17)(0.500,-1.000){2}{\rule{0.120pt}{0.400pt}}
\put(895.0,417.0){\rule[-0.200pt]{0.400pt}{1.445pt}}
\put(896,408.67){\rule{0.241pt}{0.400pt}}
\multiput(896.00,409.17)(0.500,-1.000){2}{\rule{0.120pt}{0.400pt}}
\put(896.0,410.0){\rule[-0.200pt]{0.400pt}{1.445pt}}
\put(897,402.67){\rule{0.241pt}{0.400pt}}
\multiput(897.00,403.17)(0.500,-1.000){2}{\rule{0.120pt}{0.400pt}}
\put(897.0,404.0){\rule[-0.200pt]{0.400pt}{1.204pt}}
\put(898,403){\usebox{\plotpoint}}
\put(898,396.67){\rule{0.241pt}{0.400pt}}
\multiput(898.00,397.17)(0.500,-1.000){2}{\rule{0.120pt}{0.400pt}}
\put(898.0,398.0){\rule[-0.200pt]{0.400pt}{1.204pt}}
\put(899,389.67){\rule{0.241pt}{0.400pt}}
\multiput(899.00,390.17)(0.500,-1.000){2}{\rule{0.120pt}{0.400pt}}
\put(899.0,391.0){\rule[-0.200pt]{0.400pt}{1.445pt}}
\put(900,383.67){\rule{0.241pt}{0.400pt}}
\multiput(900.00,384.17)(0.500,-1.000){2}{\rule{0.120pt}{0.400pt}}
\put(900.0,385.0){\rule[-0.200pt]{0.400pt}{1.204pt}}
\put(901,376.67){\rule{0.241pt}{0.400pt}}
\multiput(901.00,377.17)(0.500,-1.000){2}{\rule{0.120pt}{0.400pt}}
\put(901.0,378.0){\rule[-0.200pt]{0.400pt}{1.445pt}}
\put(902,370.67){\rule{0.241pt}{0.400pt}}
\multiput(902.00,371.17)(0.500,-1.000){2}{\rule{0.120pt}{0.400pt}}
\put(902.0,372.0){\rule[-0.200pt]{0.400pt}{1.204pt}}
\put(903.0,365.0){\rule[-0.200pt]{0.400pt}{1.445pt}}
\put(903.0,365.0){\usebox{\plotpoint}}
\put(904,357.67){\rule{0.241pt}{0.400pt}}
\multiput(904.00,358.17)(0.500,-1.000){2}{\rule{0.120pt}{0.400pt}}
\put(904.0,359.0){\rule[-0.200pt]{0.400pt}{1.445pt}}
\put(905,358){\usebox{\plotpoint}}
\put(905,351.67){\rule{0.241pt}{0.400pt}}
\multiput(905.00,352.17)(0.500,-1.000){2}{\rule{0.120pt}{0.400pt}}
\put(905.0,353.0){\rule[-0.200pt]{0.400pt}{1.204pt}}
\put(906,346.67){\rule{0.241pt}{0.400pt}}
\multiput(906.00,347.17)(0.500,-1.000){2}{\rule{0.120pt}{0.400pt}}
\put(906.0,348.0){\rule[-0.200pt]{0.400pt}{0.964pt}}
\put(907.0,341.0){\rule[-0.200pt]{0.400pt}{1.445pt}}
\put(907.0,341.0){\usebox{\plotpoint}}
\put(908.0,335.0){\rule[-0.200pt]{0.400pt}{1.445pt}}
\put(908.0,335.0){\usebox{\plotpoint}}
\put(909.0,329.0){\rule[-0.200pt]{0.400pt}{1.445pt}}
\put(909.0,329.0){\usebox{\plotpoint}}
\put(910,322.67){\rule{0.241pt}{0.400pt}}
\multiput(910.00,323.17)(0.500,-1.000){2}{\rule{0.120pt}{0.400pt}}
\put(910.0,324.0){\rule[-0.200pt]{0.400pt}{1.204pt}}
\put(911,316.67){\rule{0.241pt}{0.400pt}}
\multiput(911.00,317.17)(0.500,-1.000){2}{\rule{0.120pt}{0.400pt}}
\put(911.0,318.0){\rule[-0.200pt]{0.400pt}{1.204pt}}
\put(912,310.67){\rule{0.241pt}{0.400pt}}
\multiput(912.00,311.17)(0.500,-1.000){2}{\rule{0.120pt}{0.400pt}}
\put(912.0,312.0){\rule[-0.200pt]{0.400pt}{1.204pt}}
\put(913,311){\usebox{\plotpoint}}
\put(913.0,306.0){\rule[-0.200pt]{0.400pt}{1.204pt}}
\put(913.0,306.0){\usebox{\plotpoint}}
\put(914,300.67){\rule{0.241pt}{0.400pt}}
\multiput(914.00,301.17)(0.500,-1.000){2}{\rule{0.120pt}{0.400pt}}
\put(914.0,302.0){\rule[-0.200pt]{0.400pt}{0.964pt}}
\put(915,294.67){\rule{0.241pt}{0.400pt}}
\multiput(915.00,295.17)(0.500,-1.000){2}{\rule{0.120pt}{0.400pt}}
\put(915.0,296.0){\rule[-0.200pt]{0.400pt}{1.204pt}}
\put(916,295){\usebox{\plotpoint}}
\put(916,289.67){\rule{0.241pt}{0.400pt}}
\multiput(916.00,290.17)(0.500,-1.000){2}{\rule{0.120pt}{0.400pt}}
\put(916.0,291.0){\rule[-0.200pt]{0.400pt}{0.964pt}}
\put(917.0,285.0){\rule[-0.200pt]{0.400pt}{1.204pt}}
\put(917.0,285.0){\usebox{\plotpoint}}
\put(918,278.67){\rule{0.241pt}{0.400pt}}
\multiput(918.00,279.17)(0.500,-1.000){2}{\rule{0.120pt}{0.400pt}}
\put(918.0,280.0){\rule[-0.200pt]{0.400pt}{1.204pt}}
\put(919,279){\usebox{\plotpoint}}
\put(919,273.67){\rule{0.241pt}{0.400pt}}
\multiput(919.00,274.17)(0.500,-1.000){2}{\rule{0.120pt}{0.400pt}}
\put(919.0,275.0){\rule[-0.200pt]{0.400pt}{0.964pt}}
\put(920,274){\usebox{\plotpoint}}
\put(920,268.67){\rule{0.241pt}{0.400pt}}
\multiput(920.00,269.17)(0.500,-1.000){2}{\rule{0.120pt}{0.400pt}}
\put(920.0,270.0){\rule[-0.200pt]{0.400pt}{0.964pt}}
\put(921,269){\usebox{\plotpoint}}
\put(921,263.67){\rule{0.241pt}{0.400pt}}
\multiput(921.00,264.17)(0.500,-1.000){2}{\rule{0.120pt}{0.400pt}}
\put(921.0,265.0){\rule[-0.200pt]{0.400pt}{0.964pt}}
\put(922,264){\usebox{\plotpoint}}
\put(922,258.67){\rule{0.241pt}{0.400pt}}
\multiput(922.00,259.17)(0.500,-1.000){2}{\rule{0.120pt}{0.400pt}}
\put(922.0,260.0){\rule[-0.200pt]{0.400pt}{0.964pt}}
\put(923,259){\usebox{\plotpoint}}
\put(923,254.67){\rule{0.241pt}{0.400pt}}
\multiput(923.00,255.17)(0.500,-1.000){2}{\rule{0.120pt}{0.400pt}}
\put(923.0,256.0){\rule[-0.200pt]{0.400pt}{0.723pt}}
\put(924,249.67){\rule{0.241pt}{0.400pt}}
\multiput(924.00,250.17)(0.500,-1.000){2}{\rule{0.120pt}{0.400pt}}
\put(924.0,251.0){\rule[-0.200pt]{0.400pt}{0.964pt}}
\put(925,250){\usebox{\plotpoint}}
\put(925.0,246.0){\rule[-0.200pt]{0.400pt}{0.964pt}}
\put(925.0,246.0){\usebox{\plotpoint}}
\put(926,240.67){\rule{0.241pt}{0.400pt}}
\multiput(926.00,241.17)(0.500,-1.000){2}{\rule{0.120pt}{0.400pt}}
\put(926.0,242.0){\rule[-0.200pt]{0.400pt}{0.964pt}}
\put(927,241){\usebox{\plotpoint}}
\put(927.0,237.0){\rule[-0.200pt]{0.400pt}{0.964pt}}
\put(927.0,237.0){\usebox{\plotpoint}}
\put(928,231.67){\rule{0.241pt}{0.400pt}}
\multiput(928.00,232.17)(0.500,-1.000){2}{\rule{0.120pt}{0.400pt}}
\put(928.0,233.0){\rule[-0.200pt]{0.400pt}{0.964pt}}
\put(929,232){\usebox{\plotpoint}}
\put(929,227.67){\rule{0.241pt}{0.400pt}}
\multiput(929.00,228.17)(0.500,-1.000){2}{\rule{0.120pt}{0.400pt}}
\put(929.0,229.0){\rule[-0.200pt]{0.400pt}{0.723pt}}
\put(930,228){\usebox{\plotpoint}}
\put(930.0,224.0){\rule[-0.200pt]{0.400pt}{0.964pt}}
\put(930.0,224.0){\usebox{\plotpoint}}
\put(931.0,220.0){\rule[-0.200pt]{0.400pt}{0.964pt}}
\put(931.0,220.0){\usebox{\plotpoint}}
\put(932,215.67){\rule{0.241pt}{0.400pt}}
\multiput(932.00,216.17)(0.500,-1.000){2}{\rule{0.120pt}{0.400pt}}
\put(932.0,217.0){\rule[-0.200pt]{0.400pt}{0.723pt}}
\put(933,216){\usebox{\plotpoint}}
\put(933,211.67){\rule{0.241pt}{0.400pt}}
\multiput(933.00,212.17)(0.500,-1.000){2}{\rule{0.120pt}{0.400pt}}
\put(933.0,213.0){\rule[-0.200pt]{0.400pt}{0.723pt}}
\put(934,212){\usebox{\plotpoint}}
\put(934,212){\usebox{\plotpoint}}
\put(934.0,209.0){\rule[-0.200pt]{0.400pt}{0.723pt}}
\put(934.0,209.0){\usebox{\plotpoint}}
\put(935.0,205.0){\rule[-0.200pt]{0.400pt}{0.964pt}}
\put(935.0,205.0){\usebox{\plotpoint}}
\put(936,200.67){\rule{0.241pt}{0.400pt}}
\multiput(936.00,201.17)(0.500,-1.000){2}{\rule{0.120pt}{0.400pt}}
\put(936.0,202.0){\rule[-0.200pt]{0.400pt}{0.723pt}}
\put(937,201){\usebox{\plotpoint}}
\put(937.0,198.0){\rule[-0.200pt]{0.400pt}{0.723pt}}
\put(937.0,198.0){\usebox{\plotpoint}}
\put(938,193.67){\rule{0.241pt}{0.400pt}}
\multiput(938.00,194.17)(0.500,-1.000){2}{\rule{0.120pt}{0.400pt}}
\put(938.0,195.0){\rule[-0.200pt]{0.400pt}{0.723pt}}
\put(939,194){\usebox{\plotpoint}}
\put(939,194){\usebox{\plotpoint}}
\put(939,190.67){\rule{0.241pt}{0.400pt}}
\multiput(939.00,191.17)(0.500,-1.000){2}{\rule{0.120pt}{0.400pt}}
\put(939.0,192.0){\rule[-0.200pt]{0.400pt}{0.482pt}}
\put(940,191){\usebox{\plotpoint}}
\put(940.0,188.0){\rule[-0.200pt]{0.400pt}{0.723pt}}
\put(940.0,188.0){\usebox{\plotpoint}}
\put(941,184.67){\rule{0.241pt}{0.400pt}}
\multiput(941.00,185.17)(0.500,-1.000){2}{\rule{0.120pt}{0.400pt}}
\put(941.0,186.0){\rule[-0.200pt]{0.400pt}{0.482pt}}
\put(942,185){\usebox{\plotpoint}}
\put(942,181.67){\rule{0.241pt}{0.400pt}}
\multiput(942.00,182.17)(0.500,-1.000){2}{\rule{0.120pt}{0.400pt}}
\put(942.0,183.0){\rule[-0.200pt]{0.400pt}{0.482pt}}
\put(943,182){\usebox{\plotpoint}}
\put(943,178.67){\rule{0.241pt}{0.400pt}}
\multiput(943.00,179.17)(0.500,-1.000){2}{\rule{0.120pt}{0.400pt}}
\put(943.0,180.0){\rule[-0.200pt]{0.400pt}{0.482pt}}
\put(944,179){\usebox{\plotpoint}}
\put(944,175.67){\rule{0.241pt}{0.400pt}}
\multiput(944.00,176.17)(0.500,-1.000){2}{\rule{0.120pt}{0.400pt}}
\put(944.0,177.0){\rule[-0.200pt]{0.400pt}{0.482pt}}
\put(945,176){\usebox{\plotpoint}}
\put(945,176){\usebox{\plotpoint}}
\put(945.0,174.0){\rule[-0.200pt]{0.400pt}{0.482pt}}
\put(945.0,174.0){\usebox{\plotpoint}}
\put(946.0,171.0){\rule[-0.200pt]{0.400pt}{0.723pt}}
\put(946.0,171.0){\usebox{\plotpoint}}
\put(947,167.67){\rule{0.241pt}{0.400pt}}
\multiput(947.00,168.17)(0.500,-1.000){2}{\rule{0.120pt}{0.400pt}}
\put(947.0,169.0){\rule[-0.200pt]{0.400pt}{0.482pt}}
\put(948,168){\usebox{\plotpoint}}
\put(948,168){\usebox{\plotpoint}}
\put(948.0,166.0){\rule[-0.200pt]{0.400pt}{0.482pt}}
\put(948.0,166.0){\usebox{\plotpoint}}
\put(949.0,164.0){\rule[-0.200pt]{0.400pt}{0.482pt}}
\put(949.0,164.0){\usebox{\plotpoint}}
\put(950.0,162.0){\rule[-0.200pt]{0.400pt}{0.482pt}}
\put(950.0,162.0){\usebox{\plotpoint}}
\put(951,158.67){\rule{0.241pt}{0.400pt}}
\multiput(951.00,159.17)(0.500,-1.000){2}{\rule{0.120pt}{0.400pt}}
\put(951.0,160.0){\rule[-0.200pt]{0.400pt}{0.482pt}}
\put(952,159){\usebox{\plotpoint}}
\put(952,159){\usebox{\plotpoint}}
\put(952,159){\usebox{\plotpoint}}
\put(952.0,157.0){\rule[-0.200pt]{0.400pt}{0.482pt}}
\put(952.0,157.0){\usebox{\plotpoint}}
\put(953.0,155.0){\rule[-0.200pt]{0.400pt}{0.482pt}}
\put(953.0,155.0){\usebox{\plotpoint}}
\put(954,152.67){\rule{0.241pt}{0.400pt}}
\multiput(954.00,153.17)(0.500,-1.000){2}{\rule{0.120pt}{0.400pt}}
\put(954.0,154.0){\usebox{\plotpoint}}
\put(955,153){\usebox{\plotpoint}}
\put(955,153){\usebox{\plotpoint}}
\put(955,153){\usebox{\plotpoint}}
\put(955,150.67){\rule{0.241pt}{0.400pt}}
\multiput(955.00,151.17)(0.500,-1.000){2}{\rule{0.120pt}{0.400pt}}
\put(955.0,152.0){\usebox{\plotpoint}}
\put(956,151){\usebox{\plotpoint}}
\put(956,151){\usebox{\plotpoint}}
\put(956,151){\usebox{\plotpoint}}
\put(956,151){\usebox{\plotpoint}}
\put(956.0,150.0){\usebox{\plotpoint}}
\put(956.0,150.0){\usebox{\plotpoint}}
\put(957.0,148.0){\rule[-0.200pt]{0.400pt}{0.482pt}}
\put(957.0,148.0){\usebox{\plotpoint}}
\put(958.0,147.0){\usebox{\plotpoint}}
\put(958.0,147.0){\usebox{\plotpoint}}
\put(959.0,145.0){\rule[-0.200pt]{0.400pt}{0.482pt}}
\put(959.0,145.0){\usebox{\plotpoint}}
\put(960.0,144.0){\usebox{\plotpoint}}
\put(960.0,144.0){\usebox{\plotpoint}}
\put(961,141.67){\rule{0.241pt}{0.400pt}}
\multiput(961.00,142.17)(0.500,-1.000){2}{\rule{0.120pt}{0.400pt}}
\put(961.0,143.0){\usebox{\plotpoint}}
\put(962,142){\usebox{\plotpoint}}
\put(962,142){\usebox{\plotpoint}}
\put(962,142){\usebox{\plotpoint}}
\put(962,142){\usebox{\plotpoint}}
\put(962,142){\usebox{\plotpoint}}
\put(962.0,141.0){\usebox{\plotpoint}}
\put(962.0,141.0){\usebox{\plotpoint}}
\put(963.0,140.0){\usebox{\plotpoint}}
\put(963.0,140.0){\usebox{\plotpoint}}
\put(964.0,139.0){\usebox{\plotpoint}}
\put(964.0,139.0){\usebox{\plotpoint}}
\put(965.0,138.0){\usebox{\plotpoint}}
\put(965.0,138.0){\usebox{\plotpoint}}
\put(966.0,137.0){\usebox{\plotpoint}}
\put(966.0,137.0){\usebox{\plotpoint}}
\put(967.0,136.0){\usebox{\plotpoint}}
\put(967.0,136.0){\usebox{\plotpoint}}
\put(968.0,135.0){\usebox{\plotpoint}}
\put(968.0,135.0){\rule[-0.200pt]{0.482pt}{0.400pt}}
\put(970.0,134.0){\usebox{\plotpoint}}
\put(970.0,134.0){\usebox{\plotpoint}}
\put(971.0,133.0){\usebox{\plotpoint}}
\put(971.0,133.0){\rule[-0.200pt]{0.482pt}{0.400pt}}
\put(973.0,132.0){\usebox{\plotpoint}}
\put(973.0,132.0){\rule[-0.200pt]{0.723pt}{0.400pt}}
\put(976.0,131.0){\usebox{\plotpoint}}
\put(976.0,131.0){\rule[-0.200pt]{2.168pt}{0.400pt}}
\put(985.0,131.0){\usebox{\plotpoint}}
\put(985.0,132.0){\rule[-0.200pt]{0.723pt}{0.400pt}}
\put(988.0,132.0){\usebox{\plotpoint}}
\put(988.0,133.0){\rule[-0.200pt]{0.723pt}{0.400pt}}
\put(991.0,133.0){\usebox{\plotpoint}}
\put(991.0,134.0){\rule[-0.200pt]{0.482pt}{0.400pt}}
\put(993.0,134.0){\usebox{\plotpoint}}
\put(993.0,135.0){\rule[-0.200pt]{0.482pt}{0.400pt}}
\put(995.0,135.0){\usebox{\plotpoint}}
\put(995.0,136.0){\usebox{\plotpoint}}
\put(996.0,136.0){\usebox{\plotpoint}}
\put(996.0,137.0){\rule[-0.200pt]{0.482pt}{0.400pt}}
\put(998.0,137.0){\usebox{\plotpoint}}
\put(998.0,138.0){\usebox{\plotpoint}}
\put(999.0,138.0){\usebox{\plotpoint}}
\put(999.0,139.0){\rule[-0.200pt]{0.482pt}{0.400pt}}
\put(1001.0,139.0){\usebox{\plotpoint}}
\put(1001.0,140.0){\usebox{\plotpoint}}
\put(1002.0,140.0){\usebox{\plotpoint}}
\put(1002.0,141.0){\usebox{\plotpoint}}
\put(1003.0,141.0){\usebox{\plotpoint}}
\put(1003.0,142.0){\rule[-0.200pt]{0.482pt}{0.400pt}}
\put(1005.0,142.0){\usebox{\plotpoint}}
\put(1005.0,143.0){\usebox{\plotpoint}}
\put(1006.0,143.0){\usebox{\plotpoint}}
\put(1006.0,144.0){\usebox{\plotpoint}}
\put(1007.0,144.0){\usebox{\plotpoint}}
\put(1007.0,145.0){\usebox{\plotpoint}}
\put(1008.0,145.0){\usebox{\plotpoint}}
\put(1008.0,146.0){\usebox{\plotpoint}}
\put(1009.0,146.0){\usebox{\plotpoint}}
\put(1010,146.67){\rule{0.241pt}{0.400pt}}
\multiput(1010.00,146.17)(0.500,1.000){2}{\rule{0.120pt}{0.400pt}}
\put(1009.0,147.0){\usebox{\plotpoint}}
\put(1011,148){\usebox{\plotpoint}}
\put(1011,148){\usebox{\plotpoint}}
\put(1011,148){\usebox{\plotpoint}}
\put(1011,148){\usebox{\plotpoint}}
\put(1011,148){\usebox{\plotpoint}}
\put(1011,148){\usebox{\plotpoint}}
\put(1011.0,148.0){\usebox{\plotpoint}}
\put(1012.0,148.0){\usebox{\plotpoint}}
\put(1012.0,149.0){\usebox{\plotpoint}}
\put(1013.0,149.0){\usebox{\plotpoint}}
\put(1013.0,150.0){\usebox{\plotpoint}}
\put(1014.0,150.0){\usebox{\plotpoint}}
\put(1014.0,151.0){\usebox{\plotpoint}}
\put(1015.0,151.0){\usebox{\plotpoint}}
\put(1015.0,152.0){\usebox{\plotpoint}}
\put(1016.0,152.0){\usebox{\plotpoint}}
\put(1016.0,153.0){\usebox{\plotpoint}}
\put(1017.0,153.0){\usebox{\plotpoint}}
\put(1018,153.67){\rule{0.241pt}{0.400pt}}
\multiput(1018.00,153.17)(0.500,1.000){2}{\rule{0.120pt}{0.400pt}}
\put(1017.0,154.0){\usebox{\plotpoint}}
\put(1019,155){\usebox{\plotpoint}}
\put(1019,155){\usebox{\plotpoint}}
\put(1019,155){\usebox{\plotpoint}}
\put(1019,155){\usebox{\plotpoint}}
\put(1019,155){\usebox{\plotpoint}}
\put(1019,155){\usebox{\plotpoint}}
\put(1019.0,155.0){\usebox{\plotpoint}}
\put(1020.0,155.0){\usebox{\plotpoint}}
\put(1020.0,156.0){\usebox{\plotpoint}}
\put(1021.0,156.0){\usebox{\plotpoint}}
\put(1021.0,157.0){\usebox{\plotpoint}}
\put(1022.0,157.0){\usebox{\plotpoint}}
\put(1022.0,158.0){\usebox{\plotpoint}}
\put(1023.0,158.0){\usebox{\plotpoint}}
\put(1023.0,159.0){\rule[-0.200pt]{0.482pt}{0.400pt}}
\put(1025.0,159.0){\usebox{\plotpoint}}
\put(1025.0,160.0){\usebox{\plotpoint}}
\put(1026.0,160.0){\usebox{\plotpoint}}
\put(1026.0,161.0){\usebox{\plotpoint}}
\put(1027.0,161.0){\usebox{\plotpoint}}
\put(1028,161.67){\rule{0.241pt}{0.400pt}}
\multiput(1028.00,161.17)(0.500,1.000){2}{\rule{0.120pt}{0.400pt}}
\put(1027.0,162.0){\usebox{\plotpoint}}
\put(1029,163){\usebox{\plotpoint}}
\put(1029,163){\usebox{\plotpoint}}
\put(1029,163){\usebox{\plotpoint}}
\put(1029,163){\usebox{\plotpoint}}
\put(1029,163){\usebox{\plotpoint}}
\put(1029,163){\usebox{\plotpoint}}
\put(1029,163){\usebox{\plotpoint}}
\put(1029.0,163.0){\usebox{\plotpoint}}
\put(1030.0,163.0){\usebox{\plotpoint}}
\put(1030.0,164.0){\usebox{\plotpoint}}
\put(1031.0,164.0){\usebox{\plotpoint}}
\put(1031.0,165.0){\rule[-0.200pt]{0.482pt}{0.400pt}}
\put(1033.0,165.0){\usebox{\plotpoint}}
\put(1033.0,166.0){\usebox{\plotpoint}}
\put(1034.0,166.0){\usebox{\plotpoint}}
\put(1034.0,167.0){\rule[-0.200pt]{0.482pt}{0.400pt}}
\put(1036.0,167.0){\usebox{\plotpoint}}
\put(1036.0,168.0){\usebox{\plotpoint}}
\put(1037.0,168.0){\usebox{\plotpoint}}
\put(1037.0,169.0){\rule[-0.200pt]{0.482pt}{0.400pt}}
\put(1039.0,169.0){\usebox{\plotpoint}}
\put(1039.0,170.0){\rule[-0.200pt]{0.482pt}{0.400pt}}
\put(1041.0,170.0){\usebox{\plotpoint}}
\put(1041.0,171.0){\rule[-0.200pt]{0.482pt}{0.400pt}}
\put(1043.0,171.0){\usebox{\plotpoint}}
\put(1043.0,172.0){\rule[-0.200pt]{0.723pt}{0.400pt}}
\put(1046.0,172.0){\usebox{\plotpoint}}
\put(1046.0,173.0){\rule[-0.200pt]{0.723pt}{0.400pt}}
\put(1049.0,173.0){\usebox{\plotpoint}}
\put(1049.0,174.0){\rule[-0.200pt]{0.964pt}{0.400pt}}
\put(1053.0,174.0){\usebox{\plotpoint}}
\put(1053.0,175.0){\rule[-0.200pt]{3.132pt}{0.400pt}}
\put(1066.0,174.0){\usebox{\plotpoint}}
\put(1066.0,174.0){\rule[-0.200pt]{1.204pt}{0.400pt}}
\put(1071.0,173.0){\usebox{\plotpoint}}
\put(1071.0,173.0){\rule[-0.200pt]{0.723pt}{0.400pt}}
\put(1074.0,172.0){\usebox{\plotpoint}}
\put(1074.0,172.0){\rule[-0.200pt]{0.482pt}{0.400pt}}
\put(1076.0,171.0){\usebox{\plotpoint}}
\put(1076.0,171.0){\rule[-0.200pt]{0.723pt}{0.400pt}}
\put(1079.0,170.0){\usebox{\plotpoint}}
\put(1079.0,170.0){\rule[-0.200pt]{0.482pt}{0.400pt}}
\put(1081.0,169.0){\usebox{\plotpoint}}
\put(1081.0,169.0){\rule[-0.200pt]{0.482pt}{0.400pt}}
\put(1083.0,168.0){\usebox{\plotpoint}}
\put(1083.0,168.0){\rule[-0.200pt]{0.482pt}{0.400pt}}
\put(1085.0,167.0){\usebox{\plotpoint}}
\put(1085.0,167.0){\usebox{\plotpoint}}
\put(1086.0,166.0){\usebox{\plotpoint}}
\put(1086.0,166.0){\rule[-0.200pt]{0.482pt}{0.400pt}}
\put(1088.0,165.0){\usebox{\plotpoint}}
\put(1088.0,165.0){\rule[-0.200pt]{0.482pt}{0.400pt}}
\put(1090.0,164.0){\usebox{\plotpoint}}
\put(1090.0,164.0){\usebox{\plotpoint}}
\put(1091.0,163.0){\usebox{\plotpoint}}
\put(1091.0,163.0){\rule[-0.200pt]{0.482pt}{0.400pt}}
\put(1093.0,162.0){\usebox{\plotpoint}}
\put(1093.0,162.0){\usebox{\plotpoint}}
\put(1094.0,161.0){\usebox{\plotpoint}}
\put(1094.0,161.0){\rule[-0.200pt]{0.482pt}{0.400pt}}
\put(1096.0,160.0){\usebox{\plotpoint}}
\put(1096.0,160.0){\usebox{\plotpoint}}
\put(1097.0,159.0){\usebox{\plotpoint}}
\put(1097.0,159.0){\rule[-0.200pt]{0.482pt}{0.400pt}}
\put(1099.0,158.0){\usebox{\plotpoint}}
\put(1099.0,158.0){\usebox{\plotpoint}}
\put(1100.0,157.0){\usebox{\plotpoint}}
\put(1100.0,157.0){\usebox{\plotpoint}}
\put(1101.0,156.0){\usebox{\plotpoint}}
\put(1101.0,156.0){\rule[-0.200pt]{0.482pt}{0.400pt}}
\put(1103.0,155.0){\usebox{\plotpoint}}
\put(1103.0,155.0){\usebox{\plotpoint}}
\put(1104.0,154.0){\usebox{\plotpoint}}
\put(1105,152.67){\rule{0.241pt}{0.400pt}}
\multiput(1105.00,153.17)(0.500,-1.000){2}{\rule{0.120pt}{0.400pt}}
\put(1104.0,154.0){\usebox{\plotpoint}}
\put(1106,153){\usebox{\plotpoint}}
\put(1106,153){\usebox{\plotpoint}}
\put(1106,153){\usebox{\plotpoint}}
\put(1106,153){\usebox{\plotpoint}}
\put(1106,153){\usebox{\plotpoint}}
\put(1106,153){\usebox{\plotpoint}}
\put(1106,153){\usebox{\plotpoint}}
\put(1106.0,153.0){\usebox{\plotpoint}}
\put(1107.0,152.0){\usebox{\plotpoint}}
\put(1107.0,152.0){\usebox{\plotpoint}}
\put(1108.0,151.0){\usebox{\plotpoint}}
\put(1108.0,151.0){\rule[-0.200pt]{0.482pt}{0.400pt}}
\put(1110.0,150.0){\usebox{\plotpoint}}
\put(1110.0,150.0){\usebox{\plotpoint}}
\put(1111.0,149.0){\usebox{\plotpoint}}
\put(1111.0,149.0){\rule[-0.200pt]{0.482pt}{0.400pt}}
\put(1113.0,148.0){\usebox{\plotpoint}}
\put(1113.0,148.0){\usebox{\plotpoint}}
\put(1114.0,147.0){\usebox{\plotpoint}}
\put(1114.0,147.0){\rule[-0.200pt]{0.482pt}{0.400pt}}
\put(1116.0,146.0){\usebox{\plotpoint}}
\put(1116.0,146.0){\usebox{\plotpoint}}
\put(1117.0,145.0){\usebox{\plotpoint}}
\put(1117.0,145.0){\rule[-0.200pt]{0.482pt}{0.400pt}}
\put(1119.0,144.0){\usebox{\plotpoint}}
\put(1119.0,144.0){\usebox{\plotpoint}}
\put(1120.0,143.0){\usebox{\plotpoint}}
\put(1120.0,143.0){\rule[-0.200pt]{0.482pt}{0.400pt}}
\put(1122.0,142.0){\usebox{\plotpoint}}
\put(1122.0,142.0){\usebox{\plotpoint}}
\put(1123.0,141.0){\usebox{\plotpoint}}
\put(1123.0,141.0){\rule[-0.200pt]{0.482pt}{0.400pt}}
\put(1125.0,140.0){\usebox{\plotpoint}}
\put(1125.0,140.0){\rule[-0.200pt]{0.482pt}{0.400pt}}
\put(1127.0,139.0){\usebox{\plotpoint}}
\put(1127.0,139.0){\rule[-0.200pt]{0.482pt}{0.400pt}}
\put(1129.0,138.0){\usebox{\plotpoint}}
\put(1129.0,138.0){\rule[-0.200pt]{0.482pt}{0.400pt}}
\put(1131.0,137.0){\usebox{\plotpoint}}
\put(1131.0,137.0){\rule[-0.200pt]{0.482pt}{0.400pt}}
\put(1133.0,136.0){\usebox{\plotpoint}}
\put(1133.0,136.0){\rule[-0.200pt]{0.482pt}{0.400pt}}
\put(1135.0,135.0){\usebox{\plotpoint}}
\put(1135.0,135.0){\rule[-0.200pt]{0.723pt}{0.400pt}}
\put(1138.0,134.0){\usebox{\plotpoint}}
\put(1138.0,134.0){\rule[-0.200pt]{0.723pt}{0.400pt}}
\put(1141.0,133.0){\usebox{\plotpoint}}
\put(1141.0,133.0){\rule[-0.200pt]{0.723pt}{0.400pt}}
\put(1144.0,132.0){\usebox{\plotpoint}}
\put(1149,130.67){\rule{0.241pt}{0.400pt}}
\multiput(1149.00,131.17)(0.500,-1.000){2}{\rule{0.120pt}{0.400pt}}
\put(1144.0,132.0){\rule[-0.200pt]{1.204pt}{0.400pt}}
\put(1150,131){\usebox{\plotpoint}}
\put(1150,131){\usebox{\plotpoint}}
\put(1150,131){\usebox{\plotpoint}}
\put(1150,131){\usebox{\plotpoint}}
\put(1150,131){\usebox{\plotpoint}}
\put(1150,131){\usebox{\plotpoint}}
\put(1150.0,131.0){\rule[-0.200pt]{3.373pt}{0.400pt}}
\put(1164.0,131.0){\usebox{\plotpoint}}
\put(1169,131.67){\rule{0.241pt}{0.400pt}}
\multiput(1169.00,131.17)(0.500,1.000){2}{\rule{0.120pt}{0.400pt}}
\put(1164.0,132.0){\rule[-0.200pt]{1.204pt}{0.400pt}}
\put(1170,133){\usebox{\plotpoint}}
\put(1170,133){\usebox{\plotpoint}}
\put(1170,133){\usebox{\plotpoint}}
\put(1170,133){\usebox{\plotpoint}}
\put(1170,133){\usebox{\plotpoint}}
\put(1170,133){\usebox{\plotpoint}}
\put(1170,133){\usebox{\plotpoint}}
\put(1170.0,133.0){\rule[-0.200pt]{0.723pt}{0.400pt}}
\put(1173.0,133.0){\usebox{\plotpoint}}
\put(1173.0,134.0){\rule[-0.200pt]{0.964pt}{0.400pt}}
\put(1177.0,134.0){\usebox{\plotpoint}}
\put(1177.0,135.0){\rule[-0.200pt]{0.723pt}{0.400pt}}
\put(1180.0,135.0){\usebox{\plotpoint}}
\put(1180.0,136.0){\rule[-0.200pt]{0.482pt}{0.400pt}}
\put(1182.0,136.0){\usebox{\plotpoint}}
\put(1182.0,137.0){\rule[-0.200pt]{0.723pt}{0.400pt}}
\put(1185.0,137.0){\usebox{\plotpoint}}
\put(1185.0,138.0){\rule[-0.200pt]{0.482pt}{0.400pt}}
\put(1187.0,138.0){\usebox{\plotpoint}}
\put(1187.0,139.0){\rule[-0.200pt]{0.482pt}{0.400pt}}
\put(1189.0,139.0){\usebox{\plotpoint}}
\put(1189.0,140.0){\rule[-0.200pt]{0.723pt}{0.400pt}}
\put(1192.0,140.0){\usebox{\plotpoint}}
\put(1192.0,141.0){\rule[-0.200pt]{0.482pt}{0.400pt}}
\put(1194.0,141.0){\usebox{\plotpoint}}
\put(1194.0,142.0){\rule[-0.200pt]{0.482pt}{0.400pt}}
\put(1196.0,142.0){\usebox{\plotpoint}}
\put(1196.0,143.0){\rule[-0.200pt]{0.482pt}{0.400pt}}
\put(1198.0,143.0){\usebox{\plotpoint}}
\put(1198.0,144.0){\rule[-0.200pt]{0.482pt}{0.400pt}}
\put(1200.0,144.0){\usebox{\plotpoint}}
\put(1200.0,145.0){\rule[-0.200pt]{0.482pt}{0.400pt}}
\put(1202.0,145.0){\usebox{\plotpoint}}
\put(1204,145.67){\rule{0.241pt}{0.400pt}}
\multiput(1204.00,145.17)(0.500,1.000){2}{\rule{0.120pt}{0.400pt}}
\put(1202.0,146.0){\rule[-0.200pt]{0.482pt}{0.400pt}}
\put(1205,147){\usebox{\plotpoint}}
\put(1205,147){\usebox{\plotpoint}}
\put(1205,147){\usebox{\plotpoint}}
\put(1205,147){\usebox{\plotpoint}}
\put(1205,147){\usebox{\plotpoint}}
\put(1205,147){\usebox{\plotpoint}}
\put(1205,147){\usebox{\plotpoint}}
\put(1205.0,147.0){\rule[-0.200pt]{0.482pt}{0.400pt}}
\put(1207.0,147.0){\usebox{\plotpoint}}
\put(1207.0,148.0){\rule[-0.200pt]{0.482pt}{0.400pt}}
\put(1209.0,148.0){\usebox{\plotpoint}}
\put(1209.0,149.0){\rule[-0.200pt]{0.482pt}{0.400pt}}
\put(1211.0,149.0){\usebox{\plotpoint}}
\put(1213,149.67){\rule{0.241pt}{0.400pt}}
\multiput(1213.00,149.17)(0.500,1.000){2}{\rule{0.120pt}{0.400pt}}
\put(1211.0,150.0){\rule[-0.200pt]{0.482pt}{0.400pt}}
\put(1214,151){\usebox{\plotpoint}}
\put(1214,151){\usebox{\plotpoint}}
\put(1214,151){\usebox{\plotpoint}}
\put(1214,151){\usebox{\plotpoint}}
\put(1214,151){\usebox{\plotpoint}}
\put(1214,151){\usebox{\plotpoint}}
\put(1214,151){\usebox{\plotpoint}}
\put(1214.0,151.0){\rule[-0.200pt]{0.482pt}{0.400pt}}
\put(1216.0,151.0){\usebox{\plotpoint}}
\put(1218,151.67){\rule{0.241pt}{0.400pt}}
\multiput(1218.00,151.17)(0.500,1.000){2}{\rule{0.120pt}{0.400pt}}
\put(1216.0,152.0){\rule[-0.200pt]{0.482pt}{0.400pt}}
\put(1219,153){\usebox{\plotpoint}}
\put(1219,153){\usebox{\plotpoint}}
\put(1219,153){\usebox{\plotpoint}}
\put(1219,153){\usebox{\plotpoint}}
\put(1219,153){\usebox{\plotpoint}}
\put(1219,153){\usebox{\plotpoint}}
\put(1219,153){\usebox{\plotpoint}}
\put(1221,152.67){\rule{0.241pt}{0.400pt}}
\multiput(1221.00,152.17)(0.500,1.000){2}{\rule{0.120pt}{0.400pt}}
\put(1219.0,153.0){\rule[-0.200pt]{0.482pt}{0.400pt}}
\put(1222,154){\usebox{\plotpoint}}
\put(1222,154){\usebox{\plotpoint}}
\put(1222,154){\usebox{\plotpoint}}
\put(1222,154){\usebox{\plotpoint}}
\put(1222,154){\usebox{\plotpoint}}
\put(1222,154){\usebox{\plotpoint}}
\put(1222,154){\usebox{\plotpoint}}
\put(1222.0,154.0){\rule[-0.200pt]{0.723pt}{0.400pt}}
\put(1225.0,154.0){\usebox{\plotpoint}}
\put(1225.0,155.0){\rule[-0.200pt]{0.723pt}{0.400pt}}
\put(1228.0,155.0){\usebox{\plotpoint}}
\put(1228.0,156.0){\rule[-0.200pt]{1.204pt}{0.400pt}}
\put(1233.0,156.0){\usebox{\plotpoint}}
\put(1233.0,157.0){\rule[-0.200pt]{2.168pt}{0.400pt}}
\put(1242.0,157.0){\usebox{\plotpoint}}
\put(1242.0,158.0){\rule[-0.200pt]{1.204pt}{0.400pt}}
\put(1247.0,157.0){\usebox{\plotpoint}}
\put(1247.0,157.0){\rule[-0.200pt]{2.168pt}{0.400pt}}
\put(1256.0,156.0){\usebox{\plotpoint}}
\put(1256.0,156.0){\rule[-0.200pt]{1.204pt}{0.400pt}}
\put(1261.0,155.0){\usebox{\plotpoint}}
\put(1264,153.67){\rule{0.241pt}{0.400pt}}
\multiput(1264.00,154.17)(0.500,-1.000){2}{\rule{0.120pt}{0.400pt}}
\put(1261.0,155.0){\rule[-0.200pt]{0.723pt}{0.400pt}}
\put(1265,154){\usebox{\plotpoint}}
\put(1265,154){\usebox{\plotpoint}}
\put(1265,154){\usebox{\plotpoint}}
\put(1265,154){\usebox{\plotpoint}}
\put(1265,154){\usebox{\plotpoint}}
\put(1265,154){\usebox{\plotpoint}}
\put(1265,154){\usebox{\plotpoint}}
\put(1265.0,154.0){\rule[-0.200pt]{0.723pt}{0.400pt}}
\put(1268.0,153.0){\usebox{\plotpoint}}
\put(1268.0,153.0){\rule[-0.200pt]{0.482pt}{0.400pt}}
\put(1270.0,152.0){\usebox{\plotpoint}}
\put(1270.0,152.0){\rule[-0.200pt]{0.723pt}{0.400pt}}
\put(1273.0,151.0){\usebox{\plotpoint}}
\put(1273.0,151.0){\rule[-0.200pt]{0.723pt}{0.400pt}}
\put(1276.0,150.0){\usebox{\plotpoint}}
\put(1276.0,150.0){\rule[-0.200pt]{0.482pt}{0.400pt}}
\put(1278.0,149.0){\usebox{\plotpoint}}
\put(1278.0,149.0){\rule[-0.200pt]{0.482pt}{0.400pt}}
\put(1280.0,148.0){\usebox{\plotpoint}}
\put(1280.0,148.0){\rule[-0.200pt]{0.723pt}{0.400pt}}
\put(1283.0,147.0){\usebox{\plotpoint}}
\put(1283.0,147.0){\rule[-0.200pt]{0.482pt}{0.400pt}}
\put(1285.0,146.0){\usebox{\plotpoint}}
\put(1285.0,146.0){\rule[-0.200pt]{0.482pt}{0.400pt}}
\put(1287.0,145.0){\usebox{\plotpoint}}
\put(1287.0,145.0){\rule[-0.200pt]{0.482pt}{0.400pt}}
\put(1289.0,144.0){\usebox{\plotpoint}}
\put(1289.0,144.0){\rule[-0.200pt]{0.482pt}{0.400pt}}
\put(1291.0,143.0){\usebox{\plotpoint}}
\put(1293,141.67){\rule{0.241pt}{0.400pt}}
\multiput(1293.00,142.17)(0.500,-1.000){2}{\rule{0.120pt}{0.400pt}}
\put(1291.0,143.0){\rule[-0.200pt]{0.482pt}{0.400pt}}
\put(1294,142){\usebox{\plotpoint}}
\put(1294,142){\usebox{\plotpoint}}
\put(1294,142){\usebox{\plotpoint}}
\put(1294,142){\usebox{\plotpoint}}
\put(1294,142){\usebox{\plotpoint}}
\put(1294,142){\usebox{\plotpoint}}
\put(1294,142){\usebox{\plotpoint}}
\put(1294.0,142.0){\rule[-0.200pt]{0.482pt}{0.400pt}}
\put(1296.0,141.0){\usebox{\plotpoint}}
\put(1296.0,141.0){\rule[-0.200pt]{0.482pt}{0.400pt}}
\put(1298.0,140.0){\usebox{\plotpoint}}
\put(1298.0,140.0){\rule[-0.200pt]{0.482pt}{0.400pt}}
\put(1300.0,139.0){\usebox{\plotpoint}}
\put(1302,137.67){\rule{0.241pt}{0.400pt}}
\multiput(1302.00,138.17)(0.500,-1.000){2}{\rule{0.120pt}{0.400pt}}
\put(1300.0,139.0){\rule[-0.200pt]{0.482pt}{0.400pt}}
\put(1303,138){\usebox{\plotpoint}}
\put(1303,138){\usebox{\plotpoint}}
\put(1303,138){\usebox{\plotpoint}}
\put(1303,138){\usebox{\plotpoint}}
\put(1303,138){\usebox{\plotpoint}}
\put(1303,138){\usebox{\plotpoint}}
\put(1303,138){\usebox{\plotpoint}}
\put(1303.0,138.0){\rule[-0.200pt]{0.482pt}{0.400pt}}
\put(1305.0,137.0){\usebox{\plotpoint}}
\put(1305.0,137.0){\rule[-0.200pt]{0.482pt}{0.400pt}}
\put(1307.0,136.0){\usebox{\plotpoint}}
\put(1307.0,136.0){\rule[-0.200pt]{0.723pt}{0.400pt}}
\put(1310.0,135.0){\usebox{\plotpoint}}
\put(1310.0,135.0){\rule[-0.200pt]{0.723pt}{0.400pt}}
\put(1313.0,134.0){\usebox{\plotpoint}}
\put(1316,132.67){\rule{0.241pt}{0.400pt}}
\multiput(1316.00,133.17)(0.500,-1.000){2}{\rule{0.120pt}{0.400pt}}
\put(1313.0,134.0){\rule[-0.200pt]{0.723pt}{0.400pt}}
\put(1317,133){\usebox{\plotpoint}}
\put(1317,133){\usebox{\plotpoint}}
\put(1317,133){\usebox{\plotpoint}}
\put(1317,133){\usebox{\plotpoint}}
\put(1317,133){\usebox{\plotpoint}}
\put(1317,133){\usebox{\plotpoint}}
\put(1317.0,133.0){\rule[-0.200pt]{0.964pt}{0.400pt}}
\put(1321.0,132.0){\usebox{\plotpoint}}
\put(1321.0,132.0){\rule[-0.200pt]{1.204pt}{0.400pt}}
\put(1326.0,131.0){\usebox{\plotpoint}}
\put(1326.0,131.0){\rule[-0.200pt]{3.613pt}{0.400pt}}
\put(1341.0,131.0){\usebox{\plotpoint}}
\put(1341.0,132.0){\rule[-0.200pt]{1.445pt}{0.400pt}}
\put(1347.0,132.0){\usebox{\plotpoint}}
\put(1347.0,133.0){\rule[-0.200pt]{0.964pt}{0.400pt}}
\put(1351.0,133.0){\usebox{\plotpoint}}
\put(1351.0,134.0){\rule[-0.200pt]{0.723pt}{0.400pt}}
\put(1354.0,134.0){\usebox{\plotpoint}}
\put(1354.0,135.0){\rule[-0.200pt]{0.723pt}{0.400pt}}
\put(1357.0,135.0){\usebox{\plotpoint}}
\put(1357.0,136.0){\rule[-0.200pt]{0.482pt}{0.400pt}}
\put(1359.0,136.0){\usebox{\plotpoint}}
\put(1361,136.67){\rule{0.241pt}{0.400pt}}
\multiput(1361.00,136.17)(0.500,1.000){2}{\rule{0.120pt}{0.400pt}}
\put(1359.0,137.0){\rule[-0.200pt]{0.482pt}{0.400pt}}
\put(1362,138){\usebox{\plotpoint}}
\put(1362,138){\usebox{\plotpoint}}
\put(1362,138){\usebox{\plotpoint}}
\put(1362,138){\usebox{\plotpoint}}
\put(1362,138){\usebox{\plotpoint}}
\put(1362,138){\usebox{\plotpoint}}
\put(1362,138){\usebox{\plotpoint}}
\put(1362.0,138.0){\rule[-0.200pt]{0.482pt}{0.400pt}}
\put(1364.0,138.0){\usebox{\plotpoint}}
\put(1364.0,139.0){\rule[-0.200pt]{0.482pt}{0.400pt}}
\put(1366.0,139.0){\usebox{\plotpoint}}
\put(1367,139.67){\rule{0.241pt}{0.400pt}}
\multiput(1367.00,139.17)(0.500,1.000){2}{\rule{0.120pt}{0.400pt}}
\put(1366.0,140.0){\usebox{\plotpoint}}
\put(1368,141){\usebox{\plotpoint}}
\put(1368,141){\usebox{\plotpoint}}
\put(1368,141){\usebox{\plotpoint}}
\put(1368,141){\usebox{\plotpoint}}
\put(1368,141){\usebox{\plotpoint}}
\put(1368,141){\usebox{\plotpoint}}
\put(1368,141){\usebox{\plotpoint}}
\put(1369,140.67){\rule{0.241pt}{0.400pt}}
\multiput(1369.00,140.17)(0.500,1.000){2}{\rule{0.120pt}{0.400pt}}
\put(1368.0,141.0){\usebox{\plotpoint}}
\put(1370,142){\usebox{\plotpoint}}
\put(1370,142){\usebox{\plotpoint}}
\put(1370,142){\usebox{\plotpoint}}
\put(1370,142){\usebox{\plotpoint}}
\put(1370,142){\usebox{\plotpoint}}
\put(1370,142){\usebox{\plotpoint}}
\put(1370,142){\usebox{\plotpoint}}
\put(1370.0,142.0){\usebox{\plotpoint}}
\put(1371.0,142.0){\usebox{\plotpoint}}
\put(1371.0,143.0){\rule[-0.200pt]{0.482pt}{0.400pt}}
\put(1373.0,143.0){\usebox{\plotpoint}}
\put(1373.0,144.0){\rule[-0.200pt]{0.482pt}{0.400pt}}
\put(1375.0,144.0){\usebox{\plotpoint}}
\put(1376,144.67){\rule{0.241pt}{0.400pt}}
\multiput(1376.00,144.17)(0.500,1.000){2}{\rule{0.120pt}{0.400pt}}
\put(1375.0,145.0){\usebox{\plotpoint}}
\put(1377,146){\usebox{\plotpoint}}
\put(1377,146){\usebox{\plotpoint}}
\put(1377,146){\usebox{\plotpoint}}
\put(1377,146){\usebox{\plotpoint}}
\put(1377,146){\usebox{\plotpoint}}
\put(1377,146){\usebox{\plotpoint}}
\put(1377,146){\usebox{\plotpoint}}
\put(1377.0,146.0){\usebox{\plotpoint}}
\put(1378.0,146.0){\usebox{\plotpoint}}
\put(1378.0,147.0){\rule[-0.200pt]{0.482pt}{0.400pt}}
\put(1380.0,147.0){\usebox{\plotpoint}}
\put(1381,147.67){\rule{0.241pt}{0.400pt}}
\multiput(1381.00,147.17)(0.500,1.000){2}{\rule{0.120pt}{0.400pt}}
\put(1380.0,148.0){\usebox{\plotpoint}}
\put(1382,149){\usebox{\plotpoint}}
\put(1382,149){\usebox{\plotpoint}}
\put(1382,149){\usebox{\plotpoint}}
\put(1382,149){\usebox{\plotpoint}}
\put(1382,149){\usebox{\plotpoint}}
\put(1382,149){\usebox{\plotpoint}}
\put(1382,149){\usebox{\plotpoint}}
\put(1382.0,149.0){\usebox{\plotpoint}}
\put(1383.0,149.0){\usebox{\plotpoint}}
\put(1383.0,150.0){\rule[-0.200pt]{0.482pt}{0.400pt}}
\put(1385.0,150.0){\usebox{\plotpoint}}
\put(1385.0,151.0){\usebox{\plotpoint}}
\put(1386.0,151.0){\usebox{\plotpoint}}
\put(1386.0,152.0){\rule[-0.200pt]{0.482pt}{0.400pt}}
\put(1388.0,152.0){\usebox{\plotpoint}}
\put(1388.0,153.0){\rule[-0.200pt]{0.482pt}{0.400pt}}
\put(1390.0,153.0){\usebox{\plotpoint}}
\put(1390.0,154.0){\usebox{\plotpoint}}
\put(1391.0,154.0){\usebox{\plotpoint}}
\put(1391.0,155.0){\rule[-0.200pt]{0.482pt}{0.400pt}}
\put(1393.0,155.0){\usebox{\plotpoint}}
\put(1393.0,156.0){\rule[-0.200pt]{0.482pt}{0.400pt}}
\put(1395.0,156.0){\usebox{\plotpoint}}
\put(1396,156.67){\rule{0.241pt}{0.400pt}}
\multiput(1396.00,156.17)(0.500,1.000){2}{\rule{0.120pt}{0.400pt}}
\put(1395.0,157.0){\usebox{\plotpoint}}
\put(1397,158){\usebox{\plotpoint}}
\put(1397,158){\usebox{\plotpoint}}
\put(1397,158){\usebox{\plotpoint}}
\put(1397,158){\usebox{\plotpoint}}
\put(1397,158){\usebox{\plotpoint}}
\put(1397,158){\usebox{\plotpoint}}
\put(1397,158){\usebox{\plotpoint}}
\put(1397.0,158.0){\usebox{\plotpoint}}
\put(1398.0,158.0){\usebox{\plotpoint}}
\put(1398.0,159.0){\rule[-0.200pt]{0.482pt}{0.400pt}}
\put(1400.0,159.0){\usebox{\plotpoint}}
\put(1400.0,160.0){\rule[-0.200pt]{0.482pt}{0.400pt}}
\put(1402.0,160.0){\usebox{\plotpoint}}
\put(1402.0,161.0){\rule[-0.200pt]{0.482pt}{0.400pt}}
\put(1404.0,161.0){\usebox{\plotpoint}}
\put(1404.0,162.0){\rule[-0.200pt]{0.482pt}{0.400pt}}
\put(1406.0,162.0){\usebox{\plotpoint}}
\put(1406.0,163.0){\rule[-0.200pt]{0.482pt}{0.400pt}}
\put(1408.0,163.0){\usebox{\plotpoint}}
\put(1408.0,164.0){\rule[-0.200pt]{0.723pt}{0.400pt}}
\put(1411.0,164.0){\usebox{\plotpoint}}
\put(1411.0,165.0){\rule[-0.200pt]{0.482pt}{0.400pt}}
\put(1413.0,165.0){\usebox{\plotpoint}}
\put(1416,165.67){\rule{0.241pt}{0.400pt}}
\multiput(1416.00,165.17)(0.500,1.000){2}{\rule{0.120pt}{0.400pt}}
\put(1413.0,166.0){\rule[-0.200pt]{0.723pt}{0.400pt}}
\put(1417,167){\usebox{\plotpoint}}
\put(1417,167){\usebox{\plotpoint}}
\put(1417,167){\usebox{\plotpoint}}
\put(1417,167){\usebox{\plotpoint}}
\put(1417,167){\usebox{\plotpoint}}
\put(1417,167){\usebox{\plotpoint}}
\put(1417,167){\usebox{\plotpoint}}
\put(1417.0,167.0){\rule[-0.200pt]{0.723pt}{0.400pt}}
\put(1420.0,167.0){\usebox{\plotpoint}}
\put(1420.0,168.0){\rule[-0.200pt]{1.445pt}{0.400pt}}
\put(1426.0,168.0){\usebox{\plotpoint}}
\put(1426.0,169.0){\rule[-0.200pt]{2.168pt}{0.400pt}}
\put(1435.0,168.0){\usebox{\plotpoint}}
\put(1435.0,168.0){\rule[-0.200pt]{0.964pt}{0.400pt}}
\put(1279,778){\makebox(0,0)[r]{Namerné hodnoty}}
\put(805,859){\makebox(0,0){$\times$}}
\put(426,277){\makebox(0,0){$\times$}}
\put(1400,225){\makebox(0,0){$\times$}}
\put(1349,778){\makebox(0,0){$\times$}}
\put(171.0,131.0){\rule[-0.200pt]{0.400pt}{175.375pt}}
\put(171.0,131.0){\rule[-0.200pt]{305.461pt}{0.400pt}}
\put(1439.0,131.0){\rule[-0.200pt]{0.400pt}{175.375pt}}
\put(171.0,859.0){\rule[-0.200pt]{305.461pt}{0.400pt}}
\end{picture}

\caption{Difrakčný obrazec pre 5 štrbín, v porovnaní s teoretickou závislosťou. Kde $U/U_0$ je relatívny úbytok napätia a $\theta$ je pozorovaný uhol.}  \label{G_D5}
\end{figure}






\subsection{Úloha 5.}

\begin{table}[h]
\begin{center}
\begin{tabular}{| c | c | c |}
\hline
\popi{s}{cm} & \popi{t_v}{s} & \popi{t_z}{s} \\
\hline
$90 $&$ 3.3 $&$ 4.4 $\\
$90 $&$ 3.4 $&$ 4.8$\\
$90 $&$ 3.7 $&$ 4.6$\\
$90 $&$ 3.7 $&$ 4.8$\\
$90 $&$ 3.8 $&$ 4.4$\\
\hline
\end{tabular}
\caption{Namerané hodnoty dráhy $s$ vozíčku na čase pre jazdu vpred $t_v$ a vzad $t_z$.} \label{T_4}
\end{center}
\end{table}


V tabuľke Tab. \ref{T_4} sú zaznamenané merania rýchlosti vozíku, 
pričom sme pri výpočet považovali rýchlosť za konštantnú a teda $a=\infty$. 
Rýchlosť vozíku podľa vzťahu \ref{SCH_1} a \ref{R_6} vyšla pre pohyb 
vpred $v_p="\(0.25\pm0,02\) m/s"$. a pre pohyb vzad $v_z="\(0,20\pm0,01\) m/s"$

\begin{table}[h]
\begin{center}
\begin{tabular}{| c | c | c |}
\hline
\popi{f_s}{MHz} & \popi{f_v}{kHz} & \popi{f_z}{kHz} \\
\hline
$39,17$&$38.84$&$38.34$\\
$39.16$&$38.57$&$38.10$\\
$39.17$&$38.73$&$38.47$\\
$39.17$&$38.44$&$38.47$\\
$39.16$&$38.37$&$38.41$\\
\hline
\end{tabular}
\caption{Namerané hodnoty prejdutej dráhy $s$ vozíčku a časy pre jazdu vpred $t_v$ a vzad $t_z$.} \label{T_5}
\end{center}
\end{table}

V tabuľke Tab. \ref{T_5} sú zaznamenané hodnoty počtu impulzov pre pohyb vozíka vpred $f_v$ vzad $f_z$ a pri státí na mieste. $f_s$. 
Z týchto hodnôt boli vypočítané za pomoci vzťahu \ref{SCH_1} príšerné hodnoty
\eq[m]{
f_s = "\(39.2\pm0.005\) kHz"\,,\\
f_v = "\(38.6\pm0.19\) kHz"\,,\\
f_z = "\(38.4\pm0.15\) kHz"\,.
} 
Z rýchlosti a smeru vozíku môžeme za pomoci vzťahu \ref{R_7}
určiť predpokladané posuny frekvencií pri pohybe
pre pohyb v pred $f_v="39.2 MHz"$ a pre pohyb vzad $f_z="38.6 kHz"$.

\section{Diskusia}
\subsection{Úloha 1.}
Podľa nameraných dát jasne vidíme, že najväčšiu intenzitu sme dosiahli pre uhol $"90 \deg"$.
Však z túto hranicu sme nepokračovali a teda nemáme dáta aj pre uhly väčšie kde by sme mali pozorovať opäť znižovanie intenzity. 
\subsection{Úloha 2. a 3.}

Pri týchto častiach sme dostali priamym meraním a meraním s odrazom hodnoty rýchlosti zvuku $v_v="\(384.9\pm48.5\) m\cdot s^{-1}"$ a $v_z="\(320.1\pm78.2\) m\cdot s^{-1}"$. Oproti hodnote vypočítanej podľa kde sme sa $v="345 m\cdot s^{-1}"$ s oboma nameranými hodnotami vrámci chyby vošli. hlavný zdroj chýb, ktorý je oneskorenie prístrojov a to hlavne u merania malých vzdialeností a zároveň, ťažko určiteľný začiatok hrany signálu na osciloskope.

\subsection{Úloha 4.}
Pri tejto úlohe bolo niekoľko závažných problémov ktoré ovplyvnili presnosť experimentu. Hlavný zdroj problému bolo odhadnuť body, kde sa nachádza maximum alebo minimum. Hodnoty na osciloskope aj skákali a teda pri odčítaní sa niekedy líšili aj o pol rádu. Toto meranie by som jednoznačne nepovažoval za presné.
\subsection{Úloha 5.}
V poslednej časti jasne vidíme, že sme \uv{vyvrátili} Doplera. 
Hlavným dôvodom, tohoto výsledku je chyba čítača, ktorý nezachytával, všetky impulzy v momente keď bol vozíček ďalej od zdroja. Pri zisťovaní rýchlosti vozíku sme spravili navyše viacero aproximácii napr., že vozík zrýchľuje nekonečne rýchlo na svoju maximálnu rýchlosť alebo, že sa baterky nevybíjajú


\section{Záver}
Pomocou priameho a merania a merania odrazom bola rýchlosti zvuku určená postupne ako $v_z="\(384.9\pm48.5\) m\cdot s^{-1}"$ a $v_z="\(320.1\pm78.2\) m\cdot s^{-1}"$. 


\begin{thebibliography}{2}
\bibitem{C_1}
Sonar [cit. 8.12.2016]Dostupné po prihlásení z Kurz: Fyzikální praktikum I:\url{https://praktikum.fjfi.cvut.cz/pluginfile.php/4334/mod_resource/content/9/12-sonar-20161014.pdf}

\end{thebibliography}

\end{document}


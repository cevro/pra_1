\documentclass[a4paper,10pt]{article}
%\usepackage[IL2]{fontenc}
\usepackage[utf8x]{inputenc}
\usepackage[czech]{babel}
\usepackage{amsfonts,amsmath,amssymb,graphicx,color}
%\usepackage[total={17cm,27cm}, top=2cm, left=2cm, includefoot]{geometry}
%\usepackage{fancyhdr}
\usepackage{fkssugar}
\usepackage{hyperref}

%\usepackage{caption}
\renewcommand{\popi}[2]{$#1$[\jd{#2}]}
\renewcommand{\figurename}{Obr.}
\addto\captionsczech{\renewcommand{\figurename}{Obr.}}
\addto\captionsczech{\renewcommand{\tablename}{Tab.}}

\begin{document}
\def\mean#1{\left< #1 \right>}
\noindent
{\large Fyzikální praktikum 1.} \hfil {\large FJFI ČVUT V Praze}\\
\noindent
{\large\textbf{pracovní úkol \# 6}}
\begin{center}
{\large\textit{Kalorimetrie}}
\end{center}
\noindent
\rule{\textwidth}{1px}
\vspace{\baselineskip}

\emph{Michal Červeňák}
\par
\vspace{\baselineskip}
\begin{minipage}[l]{0.5\textwidth}%
\textit{dátum merania:}~10.10. 2016\\%
%\vspace{\baselineskip}%
\par%
\noindent%
\textit{skupina:}~4\\%
%\vspace{\baselineskip}%
\par%
\noindent%
\textit{Klasifikace:}\dotfill\\%
\end{minipage}

\section{Pracovní úkol}
\begin{enumerate}
\item Zkalibrujte teplotní čidlo a sestavte kalibrační křivku
\item Určete tepelnou kapacitu kalorimetru
\item Určete měrnou tepelnou kapacitu přiložených kovových válečků a ověřte Dulongův-Petitův zákon
\item Určete měrné skupenské teplo tání ledu 
\item Určete měrné skupenské teplo varu vody
\end{enumerate}

\section{Postup merania}


\subsection{Tepelná kapacita kalorimetru}
\begin{enumerate}
\item Do kalorimetru sa nalial známe množstvo vody o pokojovej teplote
\item Dáme sa zohrievať voda.
\item Po zohriatí sa určila jej teplota
\item Obe kvapaliny boli zmiešané a určená teplota zmesi.
\end{enumerate}
\subsection{Meraná tepelná kapacita pevné látky}
\begin{enumerate}
\item Do kalorimetru sa dal známe množstvo vody o známej teplote
\item Skúmané teleso zohrialo na $"100 \C"$
\item Teleso sa ponorilo do kvapaliny, pričom sa meral časový vývoj teploty
\item Tento postup sa opakoval pre všetky telesá
\end{enumerate}
\subsection{Meraná skupenské teplo topenia}
\begin{enumerate}
\item Pripravila sa zmes vody a ľadu o teplote $t="0 \C"$
\item Odmerala sa jej teplota --kvôli kalibrácii teplomeru
\item Do tejto zmesi sa dali kusy ľadu
\item Po vyrovnaní teplôt sa kúsky ľadu vložili do kalorimetru pričom sa opäť merala časová závislosť teploty.
\end{enumerate}

\subsection{Meraná skupenské teplo varu}
\begin{enumerate}
\item V paro-generátore zahrejeme vodu na $t=100\C$ tak aby nám začal var
\item Kalorimeter si naplníme vodou o známej teplote a známej hmotnosti
\item Po tom ako nastane var v paro-generátore vložíme jeho vývod do kalorimetra a regulujeme ohrev tak aby sa všetká para stihla skondenzovať.
\item Popritom zaznamenávame časový vývoj teploty 
\end{enumerate}

\section{Pomôcky}
Kalorimeter, digitálny teplomer, paro-generátor, váhy.

\section{Teória}
Pre výpočet tepelnej kapacity kalorimetra môžem použiť vzorec
\eq{
\kappa = \frac{\(m_z c\)\(t_z-t\)}{t-t_1} - m_1 c \,, \lbl{R_1}
}
pričom $c$ je tepelná kapacita vody, 
$m_z$ a $t_z$ je hmotnosť a teplota zohrievanej vody,
$m_1$ a $t_1$ je hmotnosť a teplota vody v kalorimetre, a $t$ je teplota zmesi po ustálení teploty. 

Na výpočet tepelnej kapacity neznámej látky použijem vzorec 
\eq{
c = \frac{m_1 c_v\(t-t_1\)}{m_d2\(t_2 - t\)}\,, \lbl{R_2}
}
kde $m_1$ a $t_1$ je teplota vody v kalorimetre $m_2$ a $t_2$ je hmotnosť a teplota vkladaného telesa. 
A $t$ je teplota zmesi. 

Prevod medzi mernou tepelnou kapacitou a molárnou tepelnou kapacitou
\eq{
c_M = M_n c\,, \lbl{R_3}
}
pričom $M_m$ je molárna hmotnosť a $c_M$ je molárna tepelná kapacita.


Na výpočet merného tepelného skupenstva topenia uvažujeme vzťah
\eq{
k_t = \frac{\(m_1 c+\kappa\)\(t_1-t\)-m c t}{m}\,, \lbl{R_4}
}
kde $m$ je hmotnosť ladu, a zvyšné veličiny sú obdobné ako u prechádzajúcich vzťahov.

K výpočtu merného skupenského tepla varu použijeme vzťah
\eq{
l_v = \frac{\(m_1 c+\kappa\)\(t-t_1\)-m c\(t_v-t\)}{m}\,, \lbl{R_5}
}
kde $t_v$ je teplota varu a $m$ je hmotnosť zkondenzovanej vodnej pary.



\section{Výsledky merania}
Hmotnosť kalorimetra~$m_k$ bola pomocou digitálnych váh určená na $m_k = "52.18 g"$.

\subsection{Tepelná kapacita kalorimetru}
Do kalorimetru bola naliatá voda o hmotnosti $m_1 = "105.66 g"$ a teplote $t_1="22.3 \C"$. 
Hmotnosť zohrievanej vody bola $m_z = "81.3 g"$.
Teplota vody po zohriatí bola $t_z = "100 \C"$.
Po zmeniešaní mala zmes teplotu $t_v = "47.9 \C" $
Z toho bola pomocou vzťahu \ref{R_1} určená tepelná kapacita kalorimetru $\kappa = "158 J \cdot K^{-1}" $\footnote{Pre ďalšie výpočty je uvažované práve s touto kapacitou.}.

\subsection{Meraná tepelná kapacita pevné látky}

Vážením boli určené hmotnosti všetkých troch valčekov 
\eq[m]{
m_{mo} = "334.56 g"\,,\\
m_{Cu} = "364.00 g"\,,\\
m_{Al} = "114.36 g"\,.
}
Všetky 3 valčeky boli zohraté na teplotu $t_2="100 \C"$.
Najskôr bolo prevedené meranie bronzového valčeku, ktorý bol ponorený do vody o teplote $t_1="24.7 \C"$ a hmotnosti $m_1 ="238.9 g"$.
Teplota sa ustálila po istom čase na $t="31.6 \C"$.

Následne bolo prevedené meranie medi ktorý bol ohriaty na $t_2="100 \C"$, a ponorený do vody o teplote $t_1="24.7 \C"$ a hmotnosti $m_1 ="220.7 g"$.
Teplota sa ustálila po istom čase na $t="33.6 \C"$.

Ako posledné bolo prevedené meranie hliníku, ktorý bol ohriaty na $t_2="100 \C"$, a ponorený do vody o teplote $t_1="24.8 \C"$ a hmotnosti $m_1 ="255.59 g"$.
Teplota sa ustálila po istom čase na $t="30.4 \C"$.

Podľa vzťahu \ref{R_2} boli vypočítané jednotlivé tepelne kapacity látok
\eq[m]{
c_{mo} = "0.35 J\cdot g^{-1} \cdot K^{-1}"\,,\\
c_{Cu} = "0.39 J\cdot g^{-1} \cdot K^{-1}"\,,\\
c_{Al} = "0.86 J\cdot g^{-1} \cdot K^{-1}"\,,
}
následne boli všetky hodnoty prepočítané na molárnu tepelnú kapacitu podľa vzťahu \ref{R_3},
a porovnané s hodnotou ktorá bola spočítaná podľa Dulonův-Petitůvého zákona.
Pričom molárne hmotnosti boli vyhľadané v periodickej tabuľke\cite{C_1}, 
pre med $M_{Cu}="63.546 g/mol"$ a hliník $M_{Al}="26.98154 g/mol"$
\eq[m]{
c_{M_{Cu}} = "24.8 J\cdot mol^{-1} \cdot K^{-1}"\,,\\
c_{M_{Al}} = "23.2 J\cdot mol^{-1} \cdot K^{-1}"\,,
}
pričom molárna hmotnosť mosadze nieje dobre definovaná preto nebola dopočítaná hodnota $c_M$ pre mosadz.

\subsection{Meraná skupenské teplo topenia}

V tomto prípadne boli uskotočnené 2 merania pre rôzne množstvá ľadu.
V pravom prípade, bola hmotnosť ľadu $m="19.44 g"$, teplota vody na začiatku $t_1="26.7 \C"$, a hmotnosť vody $m_1="247.47 g"$ a $t="19 \C"$.
Podľa vzťahu \ref{R_4} bolo určené merné skupenská teplo topenia na $l="358.76 J\cdot g^{-1}"$.

Následne bol prevedený experiment s väčším množstvom ľadu, konkrétne $m="76.82 g"$, $t_1="19.6 \C"$ a $t="3.7 \C"$. 
Pre tento prípad bolo určené merná skupenské teplo $l="251.76 J\cdot g^{-1}"$.
Obe teploty boli určené z grafov \ref{G_5} a \ref{G_6}

\subsection{Meraná skupenské teplo varu}
Hmotnosť vody o teplote $t_1="62.3 \C"$ bola $m_1="147.86 g"$. 
Hmotnosť kondenzovanej vody je $m="21.48 g"$. 
Pričom teplota vystúpila na $t="62.3 \C"$. 
Teplota $t$ numerickou integráciou, viď graf \ref{G_4} ako $t="62.3 \C"$
Z nameraných údajov bola pomocou vzťahu \ref{R_5} určené merné skupenské teplo varu $l_v = "1474 J\cdot g^{-1}"$




\subsection{určenie kalibračnej krivky}

Kalibračná krivka bola určená pomocou teploty varu a topenia vody
\eq[m]{
t_t = "99.9 \C"\,, \\
t_v = "0.6 \C" \,,
}
následne bola krivka vykreslená do grafu obr. \ref{G_6}. a spočítaná rovnica priamky
\eq{
t\_{nameraná} = 0.993 t\_{skutočná} + 0.6  \,.
}
\subsection{Chyby merania}
Chyby merania a chyby meracích prístrojov v tomto prípade uvažujeme ako zanedbateľné.
Konkrétne chyba určovanie hmotnosti $\Delta m = "0.01 g"$, teploty $\Delta t = "0.1 \C"$ sú určené ako veľkosť najmenšieho dieliku stupnice.

\section{Diskusia}

Merné skupenské teplo varu vody bolo určené ako $l_v = "1\,474 J\cdot g^{-1}"$, 
v porovnaní s tabuľkovou hodnotou $l_v = "2\,256 J\cdot g^{-1}"$ sa hodnoty líšia o pol radu. 
Tento rozdiel je v prvej rade spôsobený únikom časti vodných par von z kalorimetru. 
Ďalej výmenou tepla s okolitým prostredím a chľadnutím už aparatúre na výrobu pary.

Nami namerané merné skupenské teplo topenie vody v porovnaní s tabuľkovými 
hodnotami \cite{C_2} $t_t = "333.7 J\cdot g^{-1}"$ je v pravom prípade pri použití malého množstva ľadu veľmi podobná. 
Pri použití rádovo viac ľadu sa táto hodnota líši už o cca $"40 \%"$. 
V prvej rade je to spôsobené veľmi dlhým meraním, a teda vplyv okolia 
je už nezanedbateľný a lineárny odhad teploty je v tomto prípade nedostatočný, treba použiť exponenciálny.
Ďalším problémom je nemožnosť ochladiť ľad ako celok na $"0 \C"$ aj 
keď vonkajšok sa ochladil na nulu tak vnútro ľadových kociek malo stále nižšiu teplotu.

Overenie Dulonův-Petitůvého zákona sa pre hliník ale aj meď ukázalo ako celkom presné a nelíši sa od hodnoty $3R \doteq 25 $. 
Chyby ako u všetkých meraní sú výmena tepla s okolím.




\section{Záver}
Merné skupenské teplo varu vody bolo určené ako $l_v = "1\,474 J\cdot g^{-1}"$,
merné skupenské teplo topenia vody bolo určené ako $l_t = "358.76 J\cdot g^{-1}"$.

Pre meď a hliník bol overený Dulonův-Petitůvého zákon. A vynesená kalibračná krivka od obr. \ref{G_7}.



\begin{thebibliography}{2}
\bibitem{C_1}
Periodická tabuľka dostupná na \url{http://galerie2.sweb.cz/prvky.htm}

\bibitem{C_2}
Fyzikálne tabuľky dostupné na \url{http://physics.mff.cuni.cz/kfpp/skripta/kurz_fyziky_pro_DS/display.php/molekul/8_3}
\end{thebibliography}

\section{Prílohy}
\begin{figure}
% GNUPLOT: LaTeX picture
\setlength{\unitlength}{0.240900pt}
\ifx\plotpoint\undefined\newsavebox{\plotpoint}\fi
\begin{picture}(1500,900)(0,0)
\sbox{\plotpoint}{\rule[-0.200pt]{0.400pt}{0.400pt}}%
\put(151.0,131.0){\rule[-0.200pt]{4.818pt}{0.400pt}}
\put(131,131){\makebox(0,0)[r]{ 24}}
\put(1419.0,131.0){\rule[-0.200pt]{4.818pt}{0.400pt}}
\put(151.0,222.0){\rule[-0.200pt]{4.818pt}{0.400pt}}
\put(131,222){\makebox(0,0)[r]{ 25}}
\put(1419.0,222.0){\rule[-0.200pt]{4.818pt}{0.400pt}}
\put(151.0,313.0){\rule[-0.200pt]{4.818pt}{0.400pt}}
\put(131,313){\makebox(0,0)[r]{ 26}}
\put(1419.0,313.0){\rule[-0.200pt]{4.818pt}{0.400pt}}
\put(151.0,404.0){\rule[-0.200pt]{4.818pt}{0.400pt}}
\put(131,404){\makebox(0,0)[r]{ 27}}
\put(1419.0,404.0){\rule[-0.200pt]{4.818pt}{0.400pt}}
\put(151.0,495.0){\rule[-0.200pt]{4.818pt}{0.400pt}}
\put(131,495){\makebox(0,0)[r]{ 28}}
\put(1419.0,495.0){\rule[-0.200pt]{4.818pt}{0.400pt}}
\put(151.0,586.0){\rule[-0.200pt]{4.818pt}{0.400pt}}
\put(131,586){\makebox(0,0)[r]{ 29}}
\put(1419.0,586.0){\rule[-0.200pt]{4.818pt}{0.400pt}}
\put(151.0,677.0){\rule[-0.200pt]{4.818pt}{0.400pt}}
\put(131,677){\makebox(0,0)[r]{ 30}}
\put(1419.0,677.0){\rule[-0.200pt]{4.818pt}{0.400pt}}
\put(151.0,768.0){\rule[-0.200pt]{4.818pt}{0.400pt}}
\put(131,768){\makebox(0,0)[r]{ 31}}
\put(1419.0,768.0){\rule[-0.200pt]{4.818pt}{0.400pt}}
\put(151.0,859.0){\rule[-0.200pt]{4.818pt}{0.400pt}}
\put(131,859){\makebox(0,0)[r]{ 32}}
\put(1419.0,859.0){\rule[-0.200pt]{4.818pt}{0.400pt}}
\put(151.0,131.0){\rule[-0.200pt]{0.400pt}{4.818pt}}
\put(151,90){\makebox(0,0){ 0}}
\put(151.0,839.0){\rule[-0.200pt]{0.400pt}{4.818pt}}
\put(366.0,131.0){\rule[-0.200pt]{0.400pt}{4.818pt}}
\put(366,90){\makebox(0,0){ 100}}
\put(366.0,839.0){\rule[-0.200pt]{0.400pt}{4.818pt}}
\put(580.0,131.0){\rule[-0.200pt]{0.400pt}{4.818pt}}
\put(580,90){\makebox(0,0){ 200}}
\put(580.0,839.0){\rule[-0.200pt]{0.400pt}{4.818pt}}
\put(795.0,131.0){\rule[-0.200pt]{0.400pt}{4.818pt}}
\put(795,90){\makebox(0,0){ 300}}
\put(795.0,839.0){\rule[-0.200pt]{0.400pt}{4.818pt}}
\put(1010.0,131.0){\rule[-0.200pt]{0.400pt}{4.818pt}}
\put(1010,90){\makebox(0,0){ 400}}
\put(1010.0,839.0){\rule[-0.200pt]{0.400pt}{4.818pt}}
\put(1224.0,131.0){\rule[-0.200pt]{0.400pt}{4.818pt}}
\put(1224,90){\makebox(0,0){ 500}}
\put(1224.0,839.0){\rule[-0.200pt]{0.400pt}{4.818pt}}
\put(1439.0,131.0){\rule[-0.200pt]{0.400pt}{4.818pt}}
\put(1439,90){\makebox(0,0){ 600}}
\put(1439.0,839.0){\rule[-0.200pt]{0.400pt}{4.818pt}}
\put(151.0,131.0){\rule[-0.200pt]{0.400pt}{175.375pt}}
\put(151.0,131.0){\rule[-0.200pt]{310.279pt}{0.400pt}}
\put(1439.0,131.0){\rule[-0.200pt]{0.400pt}{175.375pt}}
\put(151.0,859.0){\rule[-0.200pt]{310.279pt}{0.400pt}}
\put(30,495){\makebox(0,0){\popi{t}{\C}}}
\put(795,29){\makebox(0,0){\popi{t}{s}}}
\put(471,536){\makebox(0,0)[r]{namerané dáta}}
\put(151,386){\makebox(0,0){$+$}}
\put(172,258){\makebox(0,0){$+$}}
\put(194,231){\makebox(0,0){$+$}}
\put(215,213){\makebox(0,0){$+$}}
\put(237,204){\makebox(0,0){$+$}}
\put(258,195){\makebox(0,0){$+$}}
\put(280,195){\makebox(0,0){$+$}}
\put(301,195){\makebox(0,0){$+$}}
\put(323,195){\makebox(0,0){$+$}}
\put(344,186){\makebox(0,0){$+$}}
\put(366,186){\makebox(0,0){$+$}}
\put(387,186){\makebox(0,0){$+$}}
\put(409,186){\makebox(0,0){$+$}}
\put(430,186){\makebox(0,0){$+$}}
\put(452,186){\makebox(0,0){$+$}}
\put(473,186){\makebox(0,0){$+$}}
\put(494,186){\makebox(0,0){$+$}}
\put(516,177){\makebox(0,0){$+$}}
\put(537,177){\makebox(0,0){$+$}}
\put(559,186){\makebox(0,0){$+$}}
\put(580,177){\makebox(0,0){$+$}}
\put(602,177){\makebox(0,0){$+$}}
\put(623,177){\makebox(0,0){$+$}}
\put(645,177){\makebox(0,0){$+$}}
\put(666,177){\makebox(0,0){$+$}}
\put(688,177){\makebox(0,0){$+$}}
\put(709,177){\makebox(0,0){$+$}}
\put(731,177){\makebox(0,0){$+$}}
\put(752,177){\makebox(0,0){$+$}}
\put(774,167){\makebox(0,0){$+$}}
\put(795,167){\makebox(0,0){$+$}}
\put(816,167){\makebox(0,0){$+$}}
\put(838,167){\makebox(0,0){$+$}}
\put(859,167){\makebox(0,0){$+$}}
\put(881,167){\makebox(0,0){$+$}}
\put(902,167){\makebox(0,0){$+$}}
\put(924,177){\makebox(0,0){$+$}}
\put(945,222){\makebox(0,0){$+$}}
\put(967,258){\makebox(0,0){$+$}}
\put(988,331){\makebox(0,0){$+$}}
\put(1010,741){\makebox(0,0){$+$}}
\put(1031,814){\makebox(0,0){$+$}}
\put(1053,823){\makebox(0,0){$+$}}
\put(1074,823){\makebox(0,0){$+$}}
\put(1096,823){\makebox(0,0){$+$}}
\put(1117,823){\makebox(0,0){$+$}}
\put(1138,823){\makebox(0,0){$+$}}
\put(1160,823){\makebox(0,0){$+$}}
\put(1181,814){\makebox(0,0){$+$}}
\put(1203,823){\makebox(0,0){$+$}}
\put(1224,823){\makebox(0,0){$+$}}
\put(1246,823){\makebox(0,0){$+$}}
\put(1267,823){\makebox(0,0){$+$}}
\put(541,536){\makebox(0,0){$+$}}
\put(471,495){\makebox(0,0)[r]{teplota \uv{po}}}
\multiput(491,495)(20.756,0.000){5}{\usebox{\plotpoint}}
\put(591,495){\usebox{\plotpoint}}
\put(151,821){\usebox{\plotpoint}}
\put(151.00,821.00){\usebox{\plotpoint}}
\put(171.76,821.00){\usebox{\plotpoint}}
\put(192.51,821.00){\usebox{\plotpoint}}
\put(213.27,821.00){\usebox{\plotpoint}}
\put(234.02,821.00){\usebox{\plotpoint}}
\put(254.78,821.00){\usebox{\plotpoint}}
\put(275.53,821.00){\usebox{\plotpoint}}
\put(296.29,821.00){\usebox{\plotpoint}}
\put(317.04,821.00){\usebox{\plotpoint}}
\put(337.80,821.00){\usebox{\plotpoint}}
\put(358.55,821.00){\usebox{\plotpoint}}
\put(379.31,821.00){\usebox{\plotpoint}}
\put(400.07,821.00){\usebox{\plotpoint}}
\put(420.82,821.00){\usebox{\plotpoint}}
\put(441.58,821.00){\usebox{\plotpoint}}
\put(462.33,821.00){\usebox{\plotpoint}}
\put(483.09,821.00){\usebox{\plotpoint}}
\put(503.84,821.00){\usebox{\plotpoint}}
\put(524.60,821.00){\usebox{\plotpoint}}
\put(545.35,821.00){\usebox{\plotpoint}}
\put(566.11,821.00){\usebox{\plotpoint}}
\put(586.87,821.00){\usebox{\plotpoint}}
\put(607.62,821.00){\usebox{\plotpoint}}
\put(628.38,821.00){\usebox{\plotpoint}}
\put(649.13,821.00){\usebox{\plotpoint}}
\put(669.89,821.00){\usebox{\plotpoint}}
\put(690.64,821.00){\usebox{\plotpoint}}
\put(711.40,821.00){\usebox{\plotpoint}}
\put(732.15,821.00){\usebox{\plotpoint}}
\put(752.91,821.00){\usebox{\plotpoint}}
\put(773.66,821.00){\usebox{\plotpoint}}
\put(794.42,821.00){\usebox{\plotpoint}}
\put(815.18,821.00){\usebox{\plotpoint}}
\put(835.93,821.00){\usebox{\plotpoint}}
\put(856.69,821.00){\usebox{\plotpoint}}
\put(877.44,821.00){\usebox{\plotpoint}}
\put(898.20,821.00){\usebox{\plotpoint}}
\put(918.95,821.00){\usebox{\plotpoint}}
\put(939.71,821.00){\usebox{\plotpoint}}
\put(960.46,821.00){\usebox{\plotpoint}}
\put(981.22,821.00){\usebox{\plotpoint}}
\put(1001.98,821.00){\usebox{\plotpoint}}
\put(1022.73,821.00){\usebox{\plotpoint}}
\put(1043.49,821.00){\usebox{\plotpoint}}
\put(1064.24,821.00){\usebox{\plotpoint}}
\put(1085.00,821.00){\usebox{\plotpoint}}
\put(1105.75,821.00){\usebox{\plotpoint}}
\put(1126.51,821.00){\usebox{\plotpoint}}
\put(1147.26,821.00){\usebox{\plotpoint}}
\put(1168.02,821.00){\usebox{\plotpoint}}
\put(1188.77,821.00){\usebox{\plotpoint}}
\put(1209.53,821.00){\usebox{\plotpoint}}
\put(1230.29,821.00){\usebox{\plotpoint}}
\put(1251.04,821.00){\usebox{\plotpoint}}
\put(1267,821){\usebox{\plotpoint}}
\sbox{\plotpoint}{\rule[-0.400pt]{0.800pt}{0.800pt}}%
\sbox{\plotpoint}{\rule[-0.200pt]{0.400pt}{0.400pt}}%
\put(471,454){\makebox(0,0)[r]{teplota \uv{pred}}}
\sbox{\plotpoint}{\rule[-0.400pt]{0.800pt}{0.800pt}}%
\put(491.0,454.0){\rule[-0.400pt]{24.090pt}{0.800pt}}
\put(151,198){\usebox{\plotpoint}}
\put(162,195.84){\rule{2.891pt}{0.800pt}}
\multiput(162.00,196.34)(6.000,-1.000){2}{\rule{1.445pt}{0.800pt}}
\put(151.0,198.0){\rule[-0.400pt]{2.650pt}{0.800pt}}
\put(185,194.84){\rule{2.650pt}{0.800pt}}
\multiput(185.00,195.34)(5.500,-1.000){2}{\rule{1.325pt}{0.800pt}}
\put(174.0,197.0){\rule[-0.400pt]{2.650pt}{0.800pt}}
\put(207,193.84){\rule{2.891pt}{0.800pt}}
\multiput(207.00,194.34)(6.000,-1.000){2}{\rule{1.445pt}{0.800pt}}
\put(196.0,196.0){\rule[-0.400pt]{2.650pt}{0.800pt}}
\put(230,192.84){\rule{2.650pt}{0.800pt}}
\multiput(230.00,193.34)(5.500,-1.000){2}{\rule{1.325pt}{0.800pt}}
\put(219.0,195.0){\rule[-0.400pt]{2.650pt}{0.800pt}}
\put(252,191.84){\rule{2.891pt}{0.800pt}}
\multiput(252.00,192.34)(6.000,-1.000){2}{\rule{1.445pt}{0.800pt}}
\put(241.0,194.0){\rule[-0.400pt]{2.650pt}{0.800pt}}
\put(275,190.84){\rule{2.650pt}{0.800pt}}
\multiput(275.00,191.34)(5.500,-1.000){2}{\rule{1.325pt}{0.800pt}}
\put(264.0,193.0){\rule[-0.400pt]{2.650pt}{0.800pt}}
\put(298,189.84){\rule{2.650pt}{0.800pt}}
\multiput(298.00,190.34)(5.500,-1.000){2}{\rule{1.325pt}{0.800pt}}
\put(286.0,192.0){\rule[-0.400pt]{2.891pt}{0.800pt}}
\put(320,188.84){\rule{2.650pt}{0.800pt}}
\multiput(320.00,189.34)(5.500,-1.000){2}{\rule{1.325pt}{0.800pt}}
\put(309.0,191.0){\rule[-0.400pt]{2.650pt}{0.800pt}}
\put(343,187.84){\rule{2.650pt}{0.800pt}}
\multiput(343.00,188.34)(5.500,-1.000){2}{\rule{1.325pt}{0.800pt}}
\put(331.0,190.0){\rule[-0.400pt]{2.891pt}{0.800pt}}
\put(365,186.84){\rule{2.891pt}{0.800pt}}
\multiput(365.00,187.34)(6.000,-1.000){2}{\rule{1.445pt}{0.800pt}}
\put(354.0,189.0){\rule[-0.400pt]{2.650pt}{0.800pt}}
\put(388,185.84){\rule{2.650pt}{0.800pt}}
\multiput(388.00,186.34)(5.500,-1.000){2}{\rule{1.325pt}{0.800pt}}
\put(377.0,188.0){\rule[-0.400pt]{2.650pt}{0.800pt}}
\put(410,184.84){\rule{2.891pt}{0.800pt}}
\multiput(410.00,185.34)(6.000,-1.000){2}{\rule{1.445pt}{0.800pt}}
\put(399.0,187.0){\rule[-0.400pt]{2.650pt}{0.800pt}}
\put(433,183.84){\rule{2.650pt}{0.800pt}}
\multiput(433.00,184.34)(5.500,-1.000){2}{\rule{1.325pt}{0.800pt}}
\put(422.0,186.0){\rule[-0.400pt]{2.650pt}{0.800pt}}
\put(455,182.84){\rule{2.891pt}{0.800pt}}
\multiput(455.00,183.34)(6.000,-1.000){2}{\rule{1.445pt}{0.800pt}}
\put(444.0,185.0){\rule[-0.400pt]{2.650pt}{0.800pt}}
\put(478,181.84){\rule{2.650pt}{0.800pt}}
\multiput(478.00,182.34)(5.500,-1.000){2}{\rule{1.325pt}{0.800pt}}
\put(467.0,184.0){\rule[-0.400pt]{2.650pt}{0.800pt}}
\put(501,180.84){\rule{2.650pt}{0.800pt}}
\multiput(501.00,181.34)(5.500,-1.000){2}{\rule{1.325pt}{0.800pt}}
\put(489.0,183.0){\rule[-0.400pt]{2.891pt}{0.800pt}}
\put(523,179.84){\rule{2.650pt}{0.800pt}}
\multiput(523.00,180.34)(5.500,-1.000){2}{\rule{1.325pt}{0.800pt}}
\put(512.0,182.0){\rule[-0.400pt]{2.650pt}{0.800pt}}
\put(546,178.84){\rule{2.650pt}{0.800pt}}
\multiput(546.00,179.34)(5.500,-1.000){2}{\rule{1.325pt}{0.800pt}}
\put(534.0,181.0){\rule[-0.400pt]{2.891pt}{0.800pt}}
\put(568,177.84){\rule{2.650pt}{0.800pt}}
\multiput(568.00,178.34)(5.500,-1.000){2}{\rule{1.325pt}{0.800pt}}
\put(557.0,180.0){\rule[-0.400pt]{2.650pt}{0.800pt}}
\put(591,176.84){\rule{2.650pt}{0.800pt}}
\multiput(591.00,177.34)(5.500,-1.000){2}{\rule{1.325pt}{0.800pt}}
\put(579.0,179.0){\rule[-0.400pt]{2.891pt}{0.800pt}}
\put(613,175.84){\rule{2.891pt}{0.800pt}}
\multiput(613.00,176.34)(6.000,-1.000){2}{\rule{1.445pt}{0.800pt}}
\put(602.0,178.0){\rule[-0.400pt]{2.650pt}{0.800pt}}
\put(636,174.84){\rule{2.650pt}{0.800pt}}
\multiput(636.00,175.34)(5.500,-1.000){2}{\rule{1.325pt}{0.800pt}}
\put(625.0,177.0){\rule[-0.400pt]{2.650pt}{0.800pt}}
\put(658,173.84){\rule{2.891pt}{0.800pt}}
\multiput(658.00,174.34)(6.000,-1.000){2}{\rule{1.445pt}{0.800pt}}
\put(647.0,176.0){\rule[-0.400pt]{2.650pt}{0.800pt}}
\put(681,172.84){\rule{2.650pt}{0.800pt}}
\multiput(681.00,173.34)(5.500,-1.000){2}{\rule{1.325pt}{0.800pt}}
\put(670.0,175.0){\rule[-0.400pt]{2.650pt}{0.800pt}}
\put(703,171.84){\rule{2.891pt}{0.800pt}}
\multiput(703.00,172.34)(6.000,-1.000){2}{\rule{1.445pt}{0.800pt}}
\put(692.0,174.0){\rule[-0.400pt]{2.650pt}{0.800pt}}
\put(726,170.84){\rule{2.650pt}{0.800pt}}
\multiput(726.00,171.34)(5.500,-1.000){2}{\rule{1.325pt}{0.800pt}}
\put(715.0,173.0){\rule[-0.400pt]{2.650pt}{0.800pt}}
\put(749,169.84){\rule{2.650pt}{0.800pt}}
\multiput(749.00,170.34)(5.500,-1.000){2}{\rule{1.325pt}{0.800pt}}
\put(737.0,172.0){\rule[-0.400pt]{2.891pt}{0.800pt}}
\put(771,168.84){\rule{2.650pt}{0.800pt}}
\multiput(771.00,169.34)(5.500,-1.000){2}{\rule{1.325pt}{0.800pt}}
\put(760.0,171.0){\rule[-0.400pt]{2.650pt}{0.800pt}}
\put(794,167.84){\rule{2.650pt}{0.800pt}}
\multiput(794.00,168.34)(5.500,-1.000){2}{\rule{1.325pt}{0.800pt}}
\put(782.0,170.0){\rule[-0.400pt]{2.891pt}{0.800pt}}
\put(816,166.84){\rule{2.891pt}{0.800pt}}
\multiput(816.00,167.34)(6.000,-1.000){2}{\rule{1.445pt}{0.800pt}}
\put(805.0,169.0){\rule[-0.400pt]{2.650pt}{0.800pt}}
\put(839,165.84){\rule{2.650pt}{0.800pt}}
\multiput(839.00,166.34)(5.500,-1.000){2}{\rule{1.325pt}{0.800pt}}
\put(828.0,168.0){\rule[-0.400pt]{2.650pt}{0.800pt}}
\put(861,164.84){\rule{2.891pt}{0.800pt}}
\multiput(861.00,165.34)(6.000,-1.000){2}{\rule{1.445pt}{0.800pt}}
\put(850.0,167.0){\rule[-0.400pt]{2.650pt}{0.800pt}}
\put(884,163.84){\rule{2.650pt}{0.800pt}}
\multiput(884.00,164.34)(5.500,-1.000){2}{\rule{1.325pt}{0.800pt}}
\put(873.0,166.0){\rule[-0.400pt]{2.650pt}{0.800pt}}
\put(906,162.84){\rule{2.891pt}{0.800pt}}
\multiput(906.00,163.34)(6.000,-1.000){2}{\rule{1.445pt}{0.800pt}}
\put(895.0,165.0){\rule[-0.400pt]{2.650pt}{0.800pt}}
\put(929,161.84){\rule{2.650pt}{0.800pt}}
\multiput(929.00,162.34)(5.500,-1.000){2}{\rule{1.325pt}{0.800pt}}
\put(918.0,164.0){\rule[-0.400pt]{2.650pt}{0.800pt}}
\put(952,160.84){\rule{2.650pt}{0.800pt}}
\multiput(952.00,161.34)(5.500,-1.000){2}{\rule{1.325pt}{0.800pt}}
\put(940.0,163.0){\rule[-0.400pt]{2.891pt}{0.800pt}}
\put(974,159.84){\rule{2.650pt}{0.800pt}}
\multiput(974.00,160.34)(5.500,-1.000){2}{\rule{1.325pt}{0.800pt}}
\put(963.0,162.0){\rule[-0.400pt]{2.650pt}{0.800pt}}
\put(997,158.84){\rule{2.650pt}{0.800pt}}
\multiput(997.00,159.34)(5.500,-1.000){2}{\rule{1.325pt}{0.800pt}}
\put(985.0,161.0){\rule[-0.400pt]{2.891pt}{0.800pt}}
\put(1019,157.84){\rule{2.650pt}{0.800pt}}
\multiput(1019.00,158.34)(5.500,-1.000){2}{\rule{1.325pt}{0.800pt}}
\put(1008.0,160.0){\rule[-0.400pt]{2.650pt}{0.800pt}}
\put(1042,156.84){\rule{2.650pt}{0.800pt}}
\multiput(1042.00,157.34)(5.500,-1.000){2}{\rule{1.325pt}{0.800pt}}
\put(1030.0,159.0){\rule[-0.400pt]{2.891pt}{0.800pt}}
\put(1064,155.84){\rule{2.891pt}{0.800pt}}
\multiput(1064.00,156.34)(6.000,-1.000){2}{\rule{1.445pt}{0.800pt}}
\put(1053.0,158.0){\rule[-0.400pt]{2.650pt}{0.800pt}}
\put(1087,154.84){\rule{2.650pt}{0.800pt}}
\multiput(1087.00,155.34)(5.500,-1.000){2}{\rule{1.325pt}{0.800pt}}
\put(1076.0,157.0){\rule[-0.400pt]{2.650pt}{0.800pt}}
\put(1109,153.84){\rule{2.891pt}{0.800pt}}
\multiput(1109.00,154.34)(6.000,-1.000){2}{\rule{1.445pt}{0.800pt}}
\put(1098.0,156.0){\rule[-0.400pt]{2.650pt}{0.800pt}}
\put(1132,152.84){\rule{2.650pt}{0.800pt}}
\multiput(1132.00,153.34)(5.500,-1.000){2}{\rule{1.325pt}{0.800pt}}
\put(1121.0,155.0){\rule[-0.400pt]{2.650pt}{0.800pt}}
\put(1155,151.84){\rule{2.650pt}{0.800pt}}
\multiput(1155.00,152.34)(5.500,-1.000){2}{\rule{1.325pt}{0.800pt}}
\put(1143.0,154.0){\rule[-0.400pt]{2.891pt}{0.800pt}}
\put(1177,150.84){\rule{2.650pt}{0.800pt}}
\multiput(1177.00,151.34)(5.500,-1.000){2}{\rule{1.325pt}{0.800pt}}
\put(1166.0,153.0){\rule[-0.400pt]{2.650pt}{0.800pt}}
\put(1200,149.84){\rule{2.650pt}{0.800pt}}
\multiput(1200.00,150.34)(5.500,-1.000){2}{\rule{1.325pt}{0.800pt}}
\put(1188.0,152.0){\rule[-0.400pt]{2.891pt}{0.800pt}}
\put(1222,148.84){\rule{2.650pt}{0.800pt}}
\multiput(1222.00,149.34)(5.500,-1.000){2}{\rule{1.325pt}{0.800pt}}
\put(1211.0,151.0){\rule[-0.400pt]{2.650pt}{0.800pt}}
\put(1245,147.84){\rule{2.650pt}{0.800pt}}
\multiput(1245.00,148.34)(5.500,-1.000){2}{\rule{1.325pt}{0.800pt}}
\put(1233.0,150.0){\rule[-0.400pt]{2.891pt}{0.800pt}}
\put(1256.0,149.0){\rule[-0.400pt]{2.650pt}{0.800pt}}
\sbox{\plotpoint}{\rule[-0.200pt]{0.400pt}{0.400pt}}%
\put(151.0,131.0){\rule[-0.200pt]{0.400pt}{175.375pt}}
\put(151.0,131.0){\rule[-0.200pt]{310.279pt}{0.400pt}}
\put(1439.0,131.0){\rule[-0.200pt]{0.400pt}{175.375pt}}
\put(151.0,859.0){\rule[-0.200pt]{310.279pt}{0.400pt}}
\end{picture}

\caption{Priebeh teploty pre mosadzné závažie}  \label{G_1}
\end{figure}

\begin{figure}
% GNUPLOT: LaTeX picture
\setlength{\unitlength}{0.240900pt}
\ifx\plotpoint\undefined\newsavebox{\plotpoint}\fi
\begin{picture}(1500,900)(0,0)
\sbox{\plotpoint}{\rule[-0.200pt]{0.400pt}{0.400pt}}%
\put(151.0,131.0){\rule[-0.200pt]{4.818pt}{0.400pt}}
\put(131,131){\makebox(0,0)[r]{ 24}}
\put(1419.0,131.0){\rule[-0.200pt]{4.818pt}{0.400pt}}
\put(151.0,197.0){\rule[-0.200pt]{4.818pt}{0.400pt}}
\put(131,197){\makebox(0,0)[r]{ 25}}
\put(1419.0,197.0){\rule[-0.200pt]{4.818pt}{0.400pt}}
\put(151.0,263.0){\rule[-0.200pt]{4.818pt}{0.400pt}}
\put(131,263){\makebox(0,0)[r]{ 26}}
\put(1419.0,263.0){\rule[-0.200pt]{4.818pt}{0.400pt}}
\put(151.0,330.0){\rule[-0.200pt]{4.818pt}{0.400pt}}
\put(131,330){\makebox(0,0)[r]{ 27}}
\put(1419.0,330.0){\rule[-0.200pt]{4.818pt}{0.400pt}}
\put(151.0,396.0){\rule[-0.200pt]{4.818pt}{0.400pt}}
\put(131,396){\makebox(0,0)[r]{ 28}}
\put(1419.0,396.0){\rule[-0.200pt]{4.818pt}{0.400pt}}
\put(151.0,462.0){\rule[-0.200pt]{4.818pt}{0.400pt}}
\put(131,462){\makebox(0,0)[r]{ 29}}
\put(1419.0,462.0){\rule[-0.200pt]{4.818pt}{0.400pt}}
\put(151.0,528.0){\rule[-0.200pt]{4.818pt}{0.400pt}}
\put(131,528){\makebox(0,0)[r]{ 30}}
\put(1419.0,528.0){\rule[-0.200pt]{4.818pt}{0.400pt}}
\put(151.0,594.0){\rule[-0.200pt]{4.818pt}{0.400pt}}
\put(131,594){\makebox(0,0)[r]{ 31}}
\put(1419.0,594.0){\rule[-0.200pt]{4.818pt}{0.400pt}}
\put(151.0,660.0){\rule[-0.200pt]{4.818pt}{0.400pt}}
\put(131,660){\makebox(0,0)[r]{ 32}}
\put(1419.0,660.0){\rule[-0.200pt]{4.818pt}{0.400pt}}
\put(151.0,727.0){\rule[-0.200pt]{4.818pt}{0.400pt}}
\put(131,727){\makebox(0,0)[r]{ 33}}
\put(1419.0,727.0){\rule[-0.200pt]{4.818pt}{0.400pt}}
\put(151.0,793.0){\rule[-0.200pt]{4.818pt}{0.400pt}}
\put(131,793){\makebox(0,0)[r]{ 34}}
\put(1419.0,793.0){\rule[-0.200pt]{4.818pt}{0.400pt}}
\put(151.0,859.0){\rule[-0.200pt]{4.818pt}{0.400pt}}
\put(131,859){\makebox(0,0)[r]{ 35}}
\put(1419.0,859.0){\rule[-0.200pt]{4.818pt}{0.400pt}}
\put(151.0,131.0){\rule[-0.200pt]{0.400pt}{4.818pt}}
\put(151,90){\makebox(0,0){ 0}}
\put(151.0,839.0){\rule[-0.200pt]{0.400pt}{4.818pt}}
\put(366.0,131.0){\rule[-0.200pt]{0.400pt}{4.818pt}}
\put(366,90){\makebox(0,0){ 50}}
\put(366.0,839.0){\rule[-0.200pt]{0.400pt}{4.818pt}}
\put(580.0,131.0){\rule[-0.200pt]{0.400pt}{4.818pt}}
\put(580,90){\makebox(0,0){ 100}}
\put(580.0,839.0){\rule[-0.200pt]{0.400pt}{4.818pt}}
\put(795.0,131.0){\rule[-0.200pt]{0.400pt}{4.818pt}}
\put(795,90){\makebox(0,0){ 150}}
\put(795.0,839.0){\rule[-0.200pt]{0.400pt}{4.818pt}}
\put(1010.0,131.0){\rule[-0.200pt]{0.400pt}{4.818pt}}
\put(1010,90){\makebox(0,0){ 200}}
\put(1010.0,839.0){\rule[-0.200pt]{0.400pt}{4.818pt}}
\put(1224.0,131.0){\rule[-0.200pt]{0.400pt}{4.818pt}}
\put(1224,90){\makebox(0,0){ 250}}
\put(1224.0,839.0){\rule[-0.200pt]{0.400pt}{4.818pt}}
\put(1439.0,131.0){\rule[-0.200pt]{0.400pt}{4.818pt}}
\put(1439,90){\makebox(0,0){ 300}}
\put(1439.0,839.0){\rule[-0.200pt]{0.400pt}{4.818pt}}
\put(151.0,131.0){\rule[-0.200pt]{0.400pt}{175.375pt}}
\put(151.0,131.0){\rule[-0.200pt]{310.279pt}{0.400pt}}
\put(1439.0,131.0){\rule[-0.200pt]{0.400pt}{175.375pt}}
\put(151.0,859.0){\rule[-0.200pt]{310.279pt}{0.400pt}}
\put(30,495){\makebox(0,0){\popi{t}{\C}}}
\put(795,29){\makebox(0,0){\popi{t}{s}}}
\put(1279,536){\makebox(0,0)[r]{namerané dáta}}
\put(151,184){\makebox(0,0){$+$}}
\put(194,184){\makebox(0,0){$+$}}
\put(237,177){\makebox(0,0){$+$}}
\put(280,177){\makebox(0,0){$+$}}
\put(323,177){\makebox(0,0){$+$}}
\put(366,177){\makebox(0,0){$+$}}
\put(409,177){\makebox(0,0){$+$}}
\put(452,177){\makebox(0,0){$+$}}
\put(494,257){\makebox(0,0){$+$}}
\put(537,244){\makebox(0,0){$+$}}
\put(580,330){\makebox(0,0){$+$}}
\put(623,515){\makebox(0,0){$+$}}
\put(666,581){\makebox(0,0){$+$}}
\put(709,733){\makebox(0,0){$+$}}
\put(752,760){\makebox(0,0){$+$}}
\put(795,766){\makebox(0,0){$+$}}
\put(838,766){\makebox(0,0){$+$}}
\put(881,766){\makebox(0,0){$+$}}
\put(924,766){\makebox(0,0){$+$}}
\put(967,766){\makebox(0,0){$+$}}
\put(1010,766){\makebox(0,0){$+$}}
\put(1053,760){\makebox(0,0){$+$}}
\put(1096,760){\makebox(0,0){$+$}}
\put(1138,760){\makebox(0,0){$+$}}
\put(1181,753){\makebox(0,0){$+$}}
\put(1224,753){\makebox(0,0){$+$}}
\put(1267,753){\makebox(0,0){$+$}}
\put(1310,746){\makebox(0,0){$+$}}
\put(1349,536){\makebox(0,0){$+$}}
\put(1279,495){\makebox(0,0)[r]{teplota \uv{po}}}
\multiput(1299,495)(20.756,0.000){5}{\usebox{\plotpoint}}
\put(1399,495){\usebox{\plotpoint}}
\put(151,795){\usebox{\plotpoint}}
\put(151.00,795.00){\usebox{\plotpoint}}
\put(171.71,794.00){\usebox{\plotpoint}}
\put(192.43,793.00){\usebox{\plotpoint}}
\put(213.14,792.00){\usebox{\plotpoint}}
\put(233.86,791.00){\usebox{\plotpoint}}
\put(254.61,791.00){\usebox{\plotpoint}}
\put(275.33,790.00){\usebox{\plotpoint}}
\put(296.04,789.00){\usebox{\plotpoint}}
\put(316.75,788.00){\usebox{\plotpoint}}
\put(337.47,787.05){\usebox{\plotpoint}}
\put(358.22,787.00){\usebox{\plotpoint}}
\put(378.93,786.00){\usebox{\plotpoint}}
\put(399.64,785.00){\usebox{\plotpoint}}
\put(420.35,784.00){\usebox{\plotpoint}}
\put(441.08,783.24){\usebox{\plotpoint}}
\put(461.82,783.00){\usebox{\plotpoint}}
\put(482.54,782.00){\usebox{\plotpoint}}
\put(503.25,781.00){\usebox{\plotpoint}}
\put(523.97,780.17){\usebox{\plotpoint}}
\put(544.69,779.36){\usebox{\plotpoint}}
\put(565.43,779.00){\usebox{\plotpoint}}
\put(586.14,778.00){\usebox{\plotpoint}}
\put(606.86,777.10){\usebox{\plotpoint}}
\put(627.58,776.29){\usebox{\plotpoint}}
\put(648.30,775.52){\usebox{\plotpoint}}
\put(669.03,775.00){\usebox{\plotpoint}}
\put(689.75,774.02){\usebox{\plotpoint}}
\put(710.47,773.21){\usebox{\plotpoint}}
\put(731.19,772.44){\usebox{\plotpoint}}
\put(751.91,771.67){\usebox{\plotpoint}}
\put(772.64,771.00){\usebox{\plotpoint}}
\put(793.36,770.14){\usebox{\plotpoint}}
\put(814.08,769.36){\usebox{\plotpoint}}
\put(834.80,768.60){\usebox{\plotpoint}}
\put(855.53,767.86){\usebox{\plotpoint}}
\put(876.24,767.00){\usebox{\plotpoint}}
\put(896.96,766.00){\usebox{\plotpoint}}
\put(917.69,765.53){\usebox{\plotpoint}}
\put(938.42,764.78){\usebox{\plotpoint}}
\put(959.14,763.99){\usebox{\plotpoint}}
\put(979.85,763.00){\usebox{\plotpoint}}
\put(1000.57,762.00){\usebox{\plotpoint}}
\put(1021.31,761.64){\usebox{\plotpoint}}
\put(1042.03,760.91){\usebox{\plotpoint}}
\put(1062.75,760.00){\usebox{\plotpoint}}
\put(1083.46,759.00){\usebox{\plotpoint}}
\put(1104.17,758.00){\usebox{\plotpoint}}
\put(1124.92,757.84){\usebox{\plotpoint}}
\put(1145.64,757.00){\usebox{\plotpoint}}
\put(1166.36,756.00){\usebox{\plotpoint}}
\put(1187.07,755.00){\usebox{\plotpoint}}
\put(1207.78,754.00){\usebox{\plotpoint}}
\put(1228.53,753.96){\usebox{\plotpoint}}
\put(1249.25,753.00){\usebox{\plotpoint}}
\put(1269.96,752.00){\usebox{\plotpoint}}
\put(1290.67,751.00){\usebox{\plotpoint}}
\put(1310,750){\usebox{\plotpoint}}
\sbox{\plotpoint}{\rule[-0.400pt]{0.800pt}{0.800pt}}%
\sbox{\plotpoint}{\rule[-0.200pt]{0.400pt}{0.400pt}}%
\put(1279,454){\makebox(0,0)[r]{teplota \uv{pred}}}
\sbox{\plotpoint}{\rule[-0.400pt]{0.800pt}{0.800pt}}%
\put(1299.0,454.0){\rule[-0.400pt]{24.090pt}{0.800pt}}
\put(151,183){\usebox{\plotpoint}}
\put(151,180.84){\rule{2.891pt}{0.800pt}}
\multiput(151.00,181.34)(6.000,-1.000){2}{\rule{1.445pt}{0.800pt}}
\put(186,179.84){\rule{2.891pt}{0.800pt}}
\multiput(186.00,180.34)(6.000,-1.000){2}{\rule{1.445pt}{0.800pt}}
\put(163.0,182.0){\rule[-0.400pt]{5.541pt}{0.800pt}}
\put(233,178.84){\rule{2.891pt}{0.800pt}}
\multiput(233.00,179.34)(6.000,-1.000){2}{\rule{1.445pt}{0.800pt}}
\put(198.0,181.0){\rule[-0.400pt]{8.431pt}{0.800pt}}
\put(268,177.84){\rule{2.891pt}{0.800pt}}
\multiput(268.00,178.34)(6.000,-1.000){2}{\rule{1.445pt}{0.800pt}}
\put(245.0,180.0){\rule[-0.400pt]{5.541pt}{0.800pt}}
\put(303,176.84){\rule{2.891pt}{0.800pt}}
\multiput(303.00,177.34)(6.000,-1.000){2}{\rule{1.445pt}{0.800pt}}
\put(280.0,179.0){\rule[-0.400pt]{5.541pt}{0.800pt}}
\put(338,175.84){\rule{2.891pt}{0.800pt}}
\multiput(338.00,176.34)(6.000,-1.000){2}{\rule{1.445pt}{0.800pt}}
\put(315.0,178.0){\rule[-0.400pt]{5.541pt}{0.800pt}}
\put(373,174.84){\rule{2.891pt}{0.800pt}}
\multiput(373.00,175.34)(6.000,-1.000){2}{\rule{1.445pt}{0.800pt}}
\put(350.0,177.0){\rule[-0.400pt]{5.541pt}{0.800pt}}
\put(409,173.84){\rule{2.650pt}{0.800pt}}
\multiput(409.00,174.34)(5.500,-1.000){2}{\rule{1.325pt}{0.800pt}}
\put(385.0,176.0){\rule[-0.400pt]{5.782pt}{0.800pt}}
\put(444,172.84){\rule{2.650pt}{0.800pt}}
\multiput(444.00,173.34)(5.500,-1.000){2}{\rule{1.325pt}{0.800pt}}
\put(420.0,175.0){\rule[-0.400pt]{5.782pt}{0.800pt}}
\put(479,171.84){\rule{2.891pt}{0.800pt}}
\multiput(479.00,172.34)(6.000,-1.000){2}{\rule{1.445pt}{0.800pt}}
\put(455.0,174.0){\rule[-0.400pt]{5.782pt}{0.800pt}}
\put(514,170.84){\rule{2.891pt}{0.800pt}}
\multiput(514.00,171.34)(6.000,-1.000){2}{\rule{1.445pt}{0.800pt}}
\put(491.0,173.0){\rule[-0.400pt]{5.541pt}{0.800pt}}
\put(549,169.84){\rule{2.891pt}{0.800pt}}
\multiput(549.00,170.34)(6.000,-1.000){2}{\rule{1.445pt}{0.800pt}}
\put(526.0,172.0){\rule[-0.400pt]{5.541pt}{0.800pt}}
\put(596,168.84){\rule{2.891pt}{0.800pt}}
\multiput(596.00,169.34)(6.000,-1.000){2}{\rule{1.445pt}{0.800pt}}
\put(561.0,171.0){\rule[-0.400pt]{8.431pt}{0.800pt}}
\put(631,167.84){\rule{2.891pt}{0.800pt}}
\multiput(631.00,168.34)(6.000,-1.000){2}{\rule{1.445pt}{0.800pt}}
\put(608.0,170.0){\rule[-0.400pt]{5.541pt}{0.800pt}}
\put(666,166.84){\rule{2.891pt}{0.800pt}}
\multiput(666.00,167.34)(6.000,-1.000){2}{\rule{1.445pt}{0.800pt}}
\put(643.0,169.0){\rule[-0.400pt]{5.541pt}{0.800pt}}
\put(701,165.84){\rule{2.891pt}{0.800pt}}
\multiput(701.00,166.34)(6.000,-1.000){2}{\rule{1.445pt}{0.800pt}}
\put(678.0,168.0){\rule[-0.400pt]{5.541pt}{0.800pt}}
\put(736,164.84){\rule{2.891pt}{0.800pt}}
\multiput(736.00,165.34)(6.000,-1.000){2}{\rule{1.445pt}{0.800pt}}
\put(713.0,167.0){\rule[-0.400pt]{5.541pt}{0.800pt}}
\put(772,163.84){\rule{2.650pt}{0.800pt}}
\multiput(772.00,164.34)(5.500,-1.000){2}{\rule{1.325pt}{0.800pt}}
\put(748.0,166.0){\rule[-0.400pt]{5.782pt}{0.800pt}}
\put(807,162.84){\rule{2.650pt}{0.800pt}}
\multiput(807.00,163.34)(5.500,-1.000){2}{\rule{1.325pt}{0.800pt}}
\put(783.0,165.0){\rule[-0.400pt]{5.782pt}{0.800pt}}
\put(842,161.84){\rule{2.891pt}{0.800pt}}
\multiput(842.00,162.34)(6.000,-1.000){2}{\rule{1.445pt}{0.800pt}}
\put(818.0,164.0){\rule[-0.400pt]{5.782pt}{0.800pt}}
\put(877,160.84){\rule{2.891pt}{0.800pt}}
\multiput(877.00,161.34)(6.000,-1.000){2}{\rule{1.445pt}{0.800pt}}
\put(854.0,163.0){\rule[-0.400pt]{5.541pt}{0.800pt}}
\put(912,159.84){\rule{2.891pt}{0.800pt}}
\multiput(912.00,160.34)(6.000,-1.000){2}{\rule{1.445pt}{0.800pt}}
\put(889.0,162.0){\rule[-0.400pt]{5.541pt}{0.800pt}}
\put(959,158.84){\rule{2.891pt}{0.800pt}}
\multiput(959.00,159.34)(6.000,-1.000){2}{\rule{1.445pt}{0.800pt}}
\put(924.0,161.0){\rule[-0.400pt]{8.431pt}{0.800pt}}
\put(994,157.84){\rule{2.891pt}{0.800pt}}
\multiput(994.00,158.34)(6.000,-1.000){2}{\rule{1.445pt}{0.800pt}}
\put(971.0,160.0){\rule[-0.400pt]{5.541pt}{0.800pt}}
\put(1029,156.84){\rule{2.891pt}{0.800pt}}
\multiput(1029.00,157.34)(6.000,-1.000){2}{\rule{1.445pt}{0.800pt}}
\put(1006.0,159.0){\rule[-0.400pt]{5.541pt}{0.800pt}}
\put(1064,155.84){\rule{2.891pt}{0.800pt}}
\multiput(1064.00,156.34)(6.000,-1.000){2}{\rule{1.445pt}{0.800pt}}
\put(1041.0,158.0){\rule[-0.400pt]{5.541pt}{0.800pt}}
\put(1099,154.84){\rule{2.891pt}{0.800pt}}
\multiput(1099.00,155.34)(6.000,-1.000){2}{\rule{1.445pt}{0.800pt}}
\put(1076.0,157.0){\rule[-0.400pt]{5.541pt}{0.800pt}}
\put(1135,153.84){\rule{2.650pt}{0.800pt}}
\multiput(1135.00,154.34)(5.500,-1.000){2}{\rule{1.325pt}{0.800pt}}
\put(1111.0,156.0){\rule[-0.400pt]{5.782pt}{0.800pt}}
\put(1170,152.84){\rule{2.650pt}{0.800pt}}
\multiput(1170.00,153.34)(5.500,-1.000){2}{\rule{1.325pt}{0.800pt}}
\put(1146.0,155.0){\rule[-0.400pt]{5.782pt}{0.800pt}}
\put(1205,151.84){\rule{2.891pt}{0.800pt}}
\multiput(1205.00,152.34)(6.000,-1.000){2}{\rule{1.445pt}{0.800pt}}
\put(1181.0,154.0){\rule[-0.400pt]{5.782pt}{0.800pt}}
\put(1240,150.84){\rule{2.891pt}{0.800pt}}
\multiput(1240.00,151.34)(6.000,-1.000){2}{\rule{1.445pt}{0.800pt}}
\put(1217.0,153.0){\rule[-0.400pt]{5.541pt}{0.800pt}}
\put(1275,149.84){\rule{2.891pt}{0.800pt}}
\multiput(1275.00,150.34)(6.000,-1.000){2}{\rule{1.445pt}{0.800pt}}
\put(1252.0,152.0){\rule[-0.400pt]{5.541pt}{0.800pt}}
\put(1287.0,151.0){\rule[-0.400pt]{5.541pt}{0.800pt}}
\sbox{\plotpoint}{\rule[-0.200pt]{0.400pt}{0.400pt}}%
\put(151.0,131.0){\rule[-0.200pt]{0.400pt}{175.375pt}}
\put(151.0,131.0){\rule[-0.200pt]{310.279pt}{0.400pt}}
\put(1439.0,131.0){\rule[-0.200pt]{0.400pt}{175.375pt}}
\put(151.0,859.0){\rule[-0.200pt]{310.279pt}{0.400pt}}
\end{picture}

\caption{Priebeh teploty pre medené závažie}  \label{G_2}
\end{figure}

\begin{figure}
% GNUPLOT: LaTeX picture
\setlength{\unitlength}{0.240900pt}
\ifx\plotpoint\undefined\newsavebox{\plotpoint}\fi
\begin{picture}(1500,900)(0,0)
\sbox{\plotpoint}{\rule[-0.200pt]{0.400pt}{0.400pt}}%
\put(151.0,131.0){\rule[-0.200pt]{4.818pt}{0.400pt}}
\put(131,131){\makebox(0,0)[r]{ 24}}
\put(1419.0,131.0){\rule[-0.200pt]{4.818pt}{0.400pt}}
\put(151.0,235.0){\rule[-0.200pt]{4.818pt}{0.400pt}}
\put(131,235){\makebox(0,0)[r]{ 25}}
\put(1419.0,235.0){\rule[-0.200pt]{4.818pt}{0.400pt}}
\put(151.0,339.0){\rule[-0.200pt]{4.818pt}{0.400pt}}
\put(131,339){\makebox(0,0)[r]{ 26}}
\put(1419.0,339.0){\rule[-0.200pt]{4.818pt}{0.400pt}}
\put(151.0,443.0){\rule[-0.200pt]{4.818pt}{0.400pt}}
\put(131,443){\makebox(0,0)[r]{ 27}}
\put(1419.0,443.0){\rule[-0.200pt]{4.818pt}{0.400pt}}
\put(151.0,547.0){\rule[-0.200pt]{4.818pt}{0.400pt}}
\put(131,547){\makebox(0,0)[r]{ 28}}
\put(1419.0,547.0){\rule[-0.200pt]{4.818pt}{0.400pt}}
\put(151.0,651.0){\rule[-0.200pt]{4.818pt}{0.400pt}}
\put(131,651){\makebox(0,0)[r]{ 29}}
\put(1419.0,651.0){\rule[-0.200pt]{4.818pt}{0.400pt}}
\put(151.0,755.0){\rule[-0.200pt]{4.818pt}{0.400pt}}
\put(131,755){\makebox(0,0)[r]{ 30}}
\put(1419.0,755.0){\rule[-0.200pt]{4.818pt}{0.400pt}}
\put(151.0,859.0){\rule[-0.200pt]{4.818pt}{0.400pt}}
\put(131,859){\makebox(0,0)[r]{ 31}}
\put(1419.0,859.0){\rule[-0.200pt]{4.818pt}{0.400pt}}
\put(151.0,131.0){\rule[-0.200pt]{0.400pt}{4.818pt}}
\put(151,90){\makebox(0,0){ 0}}
\put(151.0,839.0){\rule[-0.200pt]{0.400pt}{4.818pt}}
\put(312.0,131.0){\rule[-0.200pt]{0.400pt}{4.818pt}}
\put(312,90){\makebox(0,0){ 20}}
\put(312.0,839.0){\rule[-0.200pt]{0.400pt}{4.818pt}}
\put(473.0,131.0){\rule[-0.200pt]{0.400pt}{4.818pt}}
\put(473,90){\makebox(0,0){ 40}}
\put(473.0,839.0){\rule[-0.200pt]{0.400pt}{4.818pt}}
\put(634.0,131.0){\rule[-0.200pt]{0.400pt}{4.818pt}}
\put(634,90){\makebox(0,0){ 60}}
\put(634.0,839.0){\rule[-0.200pt]{0.400pt}{4.818pt}}
\put(795.0,131.0){\rule[-0.200pt]{0.400pt}{4.818pt}}
\put(795,90){\makebox(0,0){ 80}}
\put(795.0,839.0){\rule[-0.200pt]{0.400pt}{4.818pt}}
\put(956.0,131.0){\rule[-0.200pt]{0.400pt}{4.818pt}}
\put(956,90){\makebox(0,0){ 100}}
\put(956.0,839.0){\rule[-0.200pt]{0.400pt}{4.818pt}}
\put(1117.0,131.0){\rule[-0.200pt]{0.400pt}{4.818pt}}
\put(1117,90){\makebox(0,0){ 120}}
\put(1117.0,839.0){\rule[-0.200pt]{0.400pt}{4.818pt}}
\put(1278.0,131.0){\rule[-0.200pt]{0.400pt}{4.818pt}}
\put(1278,90){\makebox(0,0){ 140}}
\put(1278.0,839.0){\rule[-0.200pt]{0.400pt}{4.818pt}}
\put(1439.0,131.0){\rule[-0.200pt]{0.400pt}{4.818pt}}
\put(1439,90){\makebox(0,0){ 160}}
\put(1439.0,839.0){\rule[-0.200pt]{0.400pt}{4.818pt}}
\put(151.0,131.0){\rule[-0.200pt]{0.400pt}{175.375pt}}
\put(151.0,131.0){\rule[-0.200pt]{310.279pt}{0.400pt}}
\put(1439.0,131.0){\rule[-0.200pt]{0.400pt}{175.375pt}}
\put(151.0,859.0){\rule[-0.200pt]{310.279pt}{0.400pt}}
\put(30,495){\makebox(0,0){\popi{t}{\C}}}
\put(795,29){\makebox(0,0){\popi{t}{s}}}
\put(1279,536){\makebox(0,0)[r]{namerané dáta}}
\put(151,235){\makebox(0,0){$+$}}
\put(151,235){\makebox(0,0){$+$}}
\put(232,235){\makebox(0,0){$+$}}
\put(312,235){\makebox(0,0){$+$}}
\put(393,225){\makebox(0,0){$+$}}
\put(473,256){\makebox(0,0){$+$}}
\put(554,318){\makebox(0,0){$+$}}
\put(634,391){\makebox(0,0){$+$}}
\put(715,412){\makebox(0,0){$+$}}
\put(795,443){\makebox(0,0){$+$}}
\put(876,474){\makebox(0,0){$+$}}
\put(956,526){\makebox(0,0){$+$}}
\put(1037,703){\makebox(0,0){$+$}}
\put(1117,734){\makebox(0,0){$+$}}
\put(1198,745){\makebox(0,0){$+$}}
\put(1278,765){\makebox(0,0){$+$}}
\put(1359,786){\makebox(0,0){$+$}}
\put(1439,797){\makebox(0,0){$+$}}
\put(1349,536){\makebox(0,0){$+$}}
\put(1279,495){\makebox(0,0)[r]{teplota \uv{po}}}
\multiput(1299,495)(20.756,0.000){5}{\usebox{\plotpoint}}
\put(1399,495){\usebox{\plotpoint}}
\put(151,765){\usebox{\plotpoint}}
\put(151.00,765.00){\usebox{\plotpoint}}
\put(171.76,765.00){\usebox{\plotpoint}}
\put(192.51,765.00){\usebox{\plotpoint}}
\put(213.27,765.00){\usebox{\plotpoint}}
\put(234.02,765.00){\usebox{\plotpoint}}
\put(254.78,765.00){\usebox{\plotpoint}}
\put(275.53,765.00){\usebox{\plotpoint}}
\put(296.29,765.00){\usebox{\plotpoint}}
\put(317.04,765.00){\usebox{\plotpoint}}
\put(337.80,765.00){\usebox{\plotpoint}}
\put(358.55,765.00){\usebox{\plotpoint}}
\put(379.31,765.00){\usebox{\plotpoint}}
\put(400.07,765.00){\usebox{\plotpoint}}
\put(420.82,765.00){\usebox{\plotpoint}}
\put(441.58,765.00){\usebox{\plotpoint}}
\put(462.33,765.00){\usebox{\plotpoint}}
\put(483.09,765.00){\usebox{\plotpoint}}
\put(503.84,765.00){\usebox{\plotpoint}}
\put(524.60,765.00){\usebox{\plotpoint}}
\put(545.35,765.00){\usebox{\plotpoint}}
\put(566.11,765.00){\usebox{\plotpoint}}
\put(586.87,765.00){\usebox{\plotpoint}}
\put(607.62,765.00){\usebox{\plotpoint}}
\put(628.38,765.00){\usebox{\plotpoint}}
\put(649.13,765.00){\usebox{\plotpoint}}
\put(669.89,765.00){\usebox{\plotpoint}}
\put(690.64,765.00){\usebox{\plotpoint}}
\put(711.40,765.00){\usebox{\plotpoint}}
\put(732.15,765.00){\usebox{\plotpoint}}
\put(752.91,765.00){\usebox{\plotpoint}}
\put(773.66,765.00){\usebox{\plotpoint}}
\put(794.42,765.00){\usebox{\plotpoint}}
\put(815.18,765.00){\usebox{\plotpoint}}
\put(835.93,765.00){\usebox{\plotpoint}}
\put(856.69,765.00){\usebox{\plotpoint}}
\put(877.44,765.00){\usebox{\plotpoint}}
\put(898.20,765.00){\usebox{\plotpoint}}
\put(918.95,765.00){\usebox{\plotpoint}}
\put(939.71,765.00){\usebox{\plotpoint}}
\put(960.46,765.00){\usebox{\plotpoint}}
\put(981.22,765.00){\usebox{\plotpoint}}
\put(1001.98,765.00){\usebox{\plotpoint}}
\put(1022.73,765.00){\usebox{\plotpoint}}
\put(1043.49,765.00){\usebox{\plotpoint}}
\put(1064.24,765.00){\usebox{\plotpoint}}
\put(1085.00,765.00){\usebox{\plotpoint}}
\put(1105.75,765.00){\usebox{\plotpoint}}
\put(1126.51,765.00){\usebox{\plotpoint}}
\put(1147.26,765.00){\usebox{\plotpoint}}
\put(1168.02,765.00){\usebox{\plotpoint}}
\put(1188.77,765.00){\usebox{\plotpoint}}
\put(1209.53,765.00){\usebox{\plotpoint}}
\put(1230.29,765.00){\usebox{\plotpoint}}
\put(1251.04,765.00){\usebox{\plotpoint}}
\put(1271.80,765.00){\usebox{\plotpoint}}
\put(1292.55,765.00){\usebox{\plotpoint}}
\put(1313.31,765.00){\usebox{\plotpoint}}
\put(1334.06,765.00){\usebox{\plotpoint}}
\put(1354.82,765.00){\usebox{\plotpoint}}
\put(1375.57,765.00){\usebox{\plotpoint}}
\put(1396.33,765.00){\usebox{\plotpoint}}
\put(1417.09,765.00){\usebox{\plotpoint}}
\put(1437.84,765.00){\usebox{\plotpoint}}
\put(1439,765){\usebox{\plotpoint}}
\sbox{\plotpoint}{\rule[-0.400pt]{0.800pt}{0.800pt}}%
\sbox{\plotpoint}{\rule[-0.200pt]{0.400pt}{0.400pt}}%
\put(1279,454){\makebox(0,0)[r]{teplota \uv{pred}}}
\sbox{\plotpoint}{\rule[-0.400pt]{0.800pt}{0.800pt}}%
\put(1299.0,454.0){\rule[-0.400pt]{24.090pt}{0.800pt}}
\put(151,237){\usebox{\plotpoint}}
\put(151.0,237.0){\rule[-0.400pt]{310.279pt}{0.800pt}}
\sbox{\plotpoint}{\rule[-0.200pt]{0.400pt}{0.400pt}}%
\put(151.0,131.0){\rule[-0.200pt]{0.400pt}{175.375pt}}
\put(151.0,131.0){\rule[-0.200pt]{310.279pt}{0.400pt}}
\put(1439.0,131.0){\rule[-0.200pt]{0.400pt}{175.375pt}}
\put(151.0,859.0){\rule[-0.200pt]{310.279pt}{0.400pt}}
\end{picture}

\caption{Priebeh teploty pre hliníkove závažie}  \label{G_3}
\end{figure}

\begin{figure}
% GNUPLOT: LaTeX picture
\setlength{\unitlength}{0.240900pt}
\ifx\plotpoint\undefined\newsavebox{\plotpoint}\fi
\begin{picture}(1500,900)(0,0)
\sbox{\plotpoint}{\rule[-0.200pt]{0.400pt}{0.400pt}}%
\put(151.0,131.0){\rule[-0.200pt]{4.818pt}{0.400pt}}
\put(131,131){\makebox(0,0)[r]{ 20}}
\put(1419.0,131.0){\rule[-0.200pt]{4.818pt}{0.400pt}}
\put(151.0,235.0){\rule[-0.200pt]{4.818pt}{0.400pt}}
\put(131,235){\makebox(0,0)[r]{ 30}}
\put(1419.0,235.0){\rule[-0.200pt]{4.818pt}{0.400pt}}
\put(151.0,339.0){\rule[-0.200pt]{4.818pt}{0.400pt}}
\put(131,339){\makebox(0,0)[r]{ 40}}
\put(1419.0,339.0){\rule[-0.200pt]{4.818pt}{0.400pt}}
\put(151.0,443.0){\rule[-0.200pt]{4.818pt}{0.400pt}}
\put(131,443){\makebox(0,0)[r]{ 50}}
\put(1419.0,443.0){\rule[-0.200pt]{4.818pt}{0.400pt}}
\put(151.0,547.0){\rule[-0.200pt]{4.818pt}{0.400pt}}
\put(131,547){\makebox(0,0)[r]{ 60}}
\put(1419.0,547.0){\rule[-0.200pt]{4.818pt}{0.400pt}}
\put(151.0,651.0){\rule[-0.200pt]{4.818pt}{0.400pt}}
\put(131,651){\makebox(0,0)[r]{ 70}}
\put(1419.0,651.0){\rule[-0.200pt]{4.818pt}{0.400pt}}
\put(151.0,755.0){\rule[-0.200pt]{4.818pt}{0.400pt}}
\put(131,755){\makebox(0,0)[r]{ 80}}
\put(1419.0,755.0){\rule[-0.200pt]{4.818pt}{0.400pt}}
\put(151.0,859.0){\rule[-0.200pt]{4.818pt}{0.400pt}}
\put(131,859){\makebox(0,0)[r]{ 90}}
\put(1419.0,859.0){\rule[-0.200pt]{4.818pt}{0.400pt}}
\put(151.0,131.0){\rule[-0.200pt]{0.400pt}{4.818pt}}
\put(151,90){\makebox(0,0){ 0}}
\put(151.0,839.0){\rule[-0.200pt]{0.400pt}{4.818pt}}
\put(366.0,131.0){\rule[-0.200pt]{0.400pt}{4.818pt}}
\put(366,90){\makebox(0,0){ 100}}
\put(366.0,839.0){\rule[-0.200pt]{0.400pt}{4.818pt}}
\put(580.0,131.0){\rule[-0.200pt]{0.400pt}{4.818pt}}
\put(580,90){\makebox(0,0){ 200}}
\put(580.0,839.0){\rule[-0.200pt]{0.400pt}{4.818pt}}
\put(795.0,131.0){\rule[-0.200pt]{0.400pt}{4.818pt}}
\put(795,90){\makebox(0,0){ 300}}
\put(795.0,839.0){\rule[-0.200pt]{0.400pt}{4.818pt}}
\put(1010.0,131.0){\rule[-0.200pt]{0.400pt}{4.818pt}}
\put(1010,90){\makebox(0,0){ 400}}
\put(1010.0,839.0){\rule[-0.200pt]{0.400pt}{4.818pt}}
\put(1224.0,131.0){\rule[-0.200pt]{0.400pt}{4.818pt}}
\put(1224,90){\makebox(0,0){ 500}}
\put(1224.0,839.0){\rule[-0.200pt]{0.400pt}{4.818pt}}
\put(1439.0,131.0){\rule[-0.200pt]{0.400pt}{4.818pt}}
\put(1439,90){\makebox(0,0){ 600}}
\put(1439.0,839.0){\rule[-0.200pt]{0.400pt}{4.818pt}}
\put(151.0,131.0){\rule[-0.200pt]{0.400pt}{175.375pt}}
\put(151.0,131.0){\rule[-0.200pt]{310.279pt}{0.400pt}}
\put(1439.0,131.0){\rule[-0.200pt]{0.400pt}{175.375pt}}
\put(151.0,859.0){\rule[-0.200pt]{310.279pt}{0.400pt}}
\put(30,495){\makebox(0,0){\popi{t}{\C}}}
\put(795,29){\makebox(0,0){\popi{t}{s}}}
\put(1279,516){\makebox(0,0)[r]{namerané dáta}}
\put(151,240){\makebox(0,0){$+$}}
\put(151,242){\makebox(0,0){$+$}}
\put(153,239){\makebox(0,0){$+$}}
\put(155,238){\makebox(0,0){$+$}}
\put(157,238){\makebox(0,0){$+$}}
\put(160,237){\makebox(0,0){$+$}}
\put(162,236){\makebox(0,0){$+$}}
\put(164,235){\makebox(0,0){$+$}}
\put(166,234){\makebox(0,0){$+$}}
\put(168,233){\makebox(0,0){$+$}}
\put(170,232){\makebox(0,0){$+$}}
\put(172,232){\makebox(0,0){$+$}}
\put(175,231){\makebox(0,0){$+$}}
\put(177,230){\makebox(0,0){$+$}}
\put(179,230){\makebox(0,0){$+$}}
\put(181,229){\makebox(0,0){$+$}}
\put(183,228){\makebox(0,0){$+$}}
\put(185,228){\makebox(0,0){$+$}}
\put(187,227){\makebox(0,0){$+$}}
\put(190,226){\makebox(0,0){$+$}}
\put(192,219){\makebox(0,0){$+$}}
\put(194,201){\makebox(0,0){$+$}}
\put(196,188){\makebox(0,0){$+$}}
\put(198,179){\makebox(0,0){$+$}}
\put(200,172){\makebox(0,0){$+$}}
\put(203,167){\makebox(0,0){$+$}}
\put(205,163){\makebox(0,0){$+$}}
\put(207,160){\makebox(0,0){$+$}}
\put(209,157){\makebox(0,0){$+$}}
\put(211,155){\makebox(0,0){$+$}}
\put(213,153){\makebox(0,0){$+$}}
\put(215,152){\makebox(0,0){$+$}}
\put(218,151){\makebox(0,0){$+$}}
\put(220,150){\makebox(0,0){$+$}}
\put(222,149){\makebox(0,0){$+$}}
\put(224,149){\makebox(0,0){$+$}}
\put(226,148){\makebox(0,0){$+$}}
\put(228,148){\makebox(0,0){$+$}}
\put(230,147){\makebox(0,0){$+$}}
\put(233,147){\makebox(0,0){$+$}}
\put(235,147){\makebox(0,0){$+$}}
\put(237,147){\makebox(0,0){$+$}}
\put(239,147){\makebox(0,0){$+$}}
\put(241,147){\makebox(0,0){$+$}}
\put(243,147){\makebox(0,0){$+$}}
\put(245,147){\makebox(0,0){$+$}}
\put(248,147){\makebox(0,0){$+$}}
\put(250,147){\makebox(0,0){$+$}}
\put(252,147){\makebox(0,0){$+$}}
\put(254,148){\makebox(0,0){$+$}}
\put(256,149){\makebox(0,0){$+$}}
\put(258,150){\makebox(0,0){$+$}}
\put(260,151){\makebox(0,0){$+$}}
\put(263,152){\makebox(0,0){$+$}}
\put(265,153){\makebox(0,0){$+$}}
\put(267,154){\makebox(0,0){$+$}}
\put(269,154){\makebox(0,0){$+$}}
\put(271,155){\makebox(0,0){$+$}}
\put(273,155){\makebox(0,0){$+$}}
\put(276,155){\makebox(0,0){$+$}}
\put(278,158){\makebox(0,0){$+$}}
\put(280,161){\makebox(0,0){$+$}}
\put(282,164){\makebox(0,0){$+$}}
\put(284,166){\makebox(0,0){$+$}}
\put(286,167){\makebox(0,0){$+$}}
\put(288,168){\makebox(0,0){$+$}}
\put(291,171){\makebox(0,0){$+$}}
\put(293,173){\makebox(0,0){$+$}}
\put(295,174){\makebox(0,0){$+$}}
\put(297,175){\makebox(0,0){$+$}}
\put(299,176){\makebox(0,0){$+$}}
\put(301,177){\makebox(0,0){$+$}}
\put(303,177){\makebox(0,0){$+$}}
\put(306,178){\makebox(0,0){$+$}}
\put(308,179){\makebox(0,0){$+$}}
\put(310,180){\makebox(0,0){$+$}}
\put(312,181){\makebox(0,0){$+$}}
\put(314,183){\makebox(0,0){$+$}}
\put(316,184){\makebox(0,0){$+$}}
\put(318,186){\makebox(0,0){$+$}}
\put(321,188){\makebox(0,0){$+$}}
\put(323,189){\makebox(0,0){$+$}}
\put(325,189){\makebox(0,0){$+$}}
\put(327,190){\makebox(0,0){$+$}}
\put(329,191){\makebox(0,0){$+$}}
\put(331,191){\makebox(0,0){$+$}}
\put(333,193){\makebox(0,0){$+$}}
\put(336,197){\makebox(0,0){$+$}}
\put(338,200){\makebox(0,0){$+$}}
\put(340,203){\makebox(0,0){$+$}}
\put(342,205){\makebox(0,0){$+$}}
\put(344,207){\makebox(0,0){$+$}}
\put(346,207){\makebox(0,0){$+$}}
\put(348,208){\makebox(0,0){$+$}}
\put(351,210){\makebox(0,0){$+$}}
\put(353,211){\makebox(0,0){$+$}}
\put(355,212){\makebox(0,0){$+$}}
\put(357,214){\makebox(0,0){$+$}}
\put(359,215){\makebox(0,0){$+$}}
\put(361,217){\makebox(0,0){$+$}}
\put(364,218){\makebox(0,0){$+$}}
\put(366,220){\makebox(0,0){$+$}}
\put(368,221){\makebox(0,0){$+$}}
\put(370,223){\makebox(0,0){$+$}}
\put(372,223){\makebox(0,0){$+$}}
\put(374,224){\makebox(0,0){$+$}}
\put(376,224){\makebox(0,0){$+$}}
\put(379,224){\makebox(0,0){$+$}}
\put(381,225){\makebox(0,0){$+$}}
\put(383,226){\makebox(0,0){$+$}}
\put(385,227){\makebox(0,0){$+$}}
\put(387,228){\makebox(0,0){$+$}}
\put(389,229){\makebox(0,0){$+$}}
\put(391,230){\makebox(0,0){$+$}}
\put(394,231){\makebox(0,0){$+$}}
\put(396,233){\makebox(0,0){$+$}}
\put(398,235){\makebox(0,0){$+$}}
\put(400,236){\makebox(0,0){$+$}}
\put(402,237){\makebox(0,0){$+$}}
\put(404,238){\makebox(0,0){$+$}}
\put(406,240){\makebox(0,0){$+$}}
\put(409,241){\makebox(0,0){$+$}}
\put(411,242){\makebox(0,0){$+$}}
\put(413,243){\makebox(0,0){$+$}}
\put(415,244){\makebox(0,0){$+$}}
\put(417,244){\makebox(0,0){$+$}}
\put(419,246){\makebox(0,0){$+$}}
\put(421,247){\makebox(0,0){$+$}}
\put(424,249){\makebox(0,0){$+$}}
\put(426,251){\makebox(0,0){$+$}}
\put(428,252){\makebox(0,0){$+$}}
\put(430,253){\makebox(0,0){$+$}}
\put(432,254){\makebox(0,0){$+$}}
\put(434,256){\makebox(0,0){$+$}}
\put(437,258){\makebox(0,0){$+$}}
\put(439,259){\makebox(0,0){$+$}}
\put(441,260){\makebox(0,0){$+$}}
\put(443,261){\makebox(0,0){$+$}}
\put(445,262){\makebox(0,0){$+$}}
\put(447,262){\makebox(0,0){$+$}}
\put(449,263){\makebox(0,0){$+$}}
\put(452,264){\makebox(0,0){$+$}}
\put(454,265){\makebox(0,0){$+$}}
\put(456,265){\makebox(0,0){$+$}}
\put(458,266){\makebox(0,0){$+$}}
\put(460,268){\makebox(0,0){$+$}}
\put(462,270){\makebox(0,0){$+$}}
\put(464,271){\makebox(0,0){$+$}}
\put(467,272){\makebox(0,0){$+$}}
\put(469,275){\makebox(0,0){$+$}}
\put(471,276){\makebox(0,0){$+$}}
\put(473,278){\makebox(0,0){$+$}}
\put(475,279){\makebox(0,0){$+$}}
\put(477,280){\makebox(0,0){$+$}}
\put(479,282){\makebox(0,0){$+$}}
\put(482,283){\makebox(0,0){$+$}}
\put(484,284){\makebox(0,0){$+$}}
\put(486,285){\makebox(0,0){$+$}}
\put(488,286){\makebox(0,0){$+$}}
\put(490,287){\makebox(0,0){$+$}}
\put(492,288){\makebox(0,0){$+$}}
\put(494,289){\makebox(0,0){$+$}}
\put(497,289){\makebox(0,0){$+$}}
\put(499,290){\makebox(0,0){$+$}}
\put(501,292){\makebox(0,0){$+$}}
\put(503,293){\makebox(0,0){$+$}}
\put(505,295){\makebox(0,0){$+$}}
\put(507,298){\makebox(0,0){$+$}}
\put(509,301){\makebox(0,0){$+$}}
\put(512,303){\makebox(0,0){$+$}}
\put(514,304){\makebox(0,0){$+$}}
\put(516,305){\makebox(0,0){$+$}}
\put(518,306){\makebox(0,0){$+$}}
\put(520,307){\makebox(0,0){$+$}}
\put(522,308){\makebox(0,0){$+$}}
\put(525,310){\makebox(0,0){$+$}}
\put(527,312){\makebox(0,0){$+$}}
\put(529,312){\makebox(0,0){$+$}}
\put(531,313){\makebox(0,0){$+$}}
\put(533,314){\makebox(0,0){$+$}}
\put(535,314){\makebox(0,0){$+$}}
\put(537,315){\makebox(0,0){$+$}}
\put(540,316){\makebox(0,0){$+$}}
\put(542,318){\makebox(0,0){$+$}}
\put(544,321){\makebox(0,0){$+$}}
\put(546,323){\makebox(0,0){$+$}}
\put(548,325){\makebox(0,0){$+$}}
\put(550,327){\makebox(0,0){$+$}}
\put(552,329){\makebox(0,0){$+$}}
\put(555,330){\makebox(0,0){$+$}}
\put(557,331){\makebox(0,0){$+$}}
\put(559,332){\makebox(0,0){$+$}}
\put(561,332){\makebox(0,0){$+$}}
\put(563,333){\makebox(0,0){$+$}}
\put(565,334){\makebox(0,0){$+$}}
\put(567,336){\makebox(0,0){$+$}}
\put(570,338){\makebox(0,0){$+$}}
\put(572,339){\makebox(0,0){$+$}}
\put(574,340){\makebox(0,0){$+$}}
\put(576,341){\makebox(0,0){$+$}}
\put(578,342){\makebox(0,0){$+$}}
\put(580,343){\makebox(0,0){$+$}}
\put(582,344){\makebox(0,0){$+$}}
\put(585,345){\makebox(0,0){$+$}}
\put(587,346){\makebox(0,0){$+$}}
\put(589,346){\makebox(0,0){$+$}}
\put(591,346){\makebox(0,0){$+$}}
\put(593,347){\makebox(0,0){$+$}}
\put(595,348){\makebox(0,0){$+$}}
\put(598,349){\makebox(0,0){$+$}}
\put(600,349){\makebox(0,0){$+$}}
\put(602,350){\makebox(0,0){$+$}}
\put(604,351){\makebox(0,0){$+$}}
\put(606,353){\makebox(0,0){$+$}}
\put(608,354){\makebox(0,0){$+$}}
\put(610,356){\makebox(0,0){$+$}}
\put(613,357){\makebox(0,0){$+$}}
\put(615,358){\makebox(0,0){$+$}}
\put(617,359){\makebox(0,0){$+$}}
\put(619,360){\makebox(0,0){$+$}}
\put(621,361){\makebox(0,0){$+$}}
\put(623,363){\makebox(0,0){$+$}}
\put(625,364){\makebox(0,0){$+$}}
\put(628,365){\makebox(0,0){$+$}}
\put(630,366){\makebox(0,0){$+$}}
\put(632,366){\makebox(0,0){$+$}}
\put(634,367){\makebox(0,0){$+$}}
\put(636,368){\makebox(0,0){$+$}}
\put(638,370){\makebox(0,0){$+$}}
\put(640,370){\makebox(0,0){$+$}}
\put(643,371){\makebox(0,0){$+$}}
\put(645,372){\makebox(0,0){$+$}}
\put(647,372){\makebox(0,0){$+$}}
\put(649,373){\makebox(0,0){$+$}}
\put(651,374){\makebox(0,0){$+$}}
\put(653,376){\makebox(0,0){$+$}}
\put(655,376){\makebox(0,0){$+$}}
\put(658,377){\makebox(0,0){$+$}}
\put(660,379){\makebox(0,0){$+$}}
\put(662,380){\makebox(0,0){$+$}}
\put(664,380){\makebox(0,0){$+$}}
\put(666,382){\makebox(0,0){$+$}}
\put(668,383){\makebox(0,0){$+$}}
\put(670,384){\makebox(0,0){$+$}}
\put(673,385){\makebox(0,0){$+$}}
\put(675,386){\makebox(0,0){$+$}}
\put(677,387){\makebox(0,0){$+$}}
\put(679,388){\makebox(0,0){$+$}}
\put(681,389){\makebox(0,0){$+$}}
\put(683,389){\makebox(0,0){$+$}}
\put(686,390){\makebox(0,0){$+$}}
\put(688,391){\makebox(0,0){$+$}}
\put(690,392){\makebox(0,0){$+$}}
\put(692,394){\makebox(0,0){$+$}}
\put(694,395){\makebox(0,0){$+$}}
\put(696,397){\makebox(0,0){$+$}}
\put(698,398){\makebox(0,0){$+$}}
\put(701,399){\makebox(0,0){$+$}}
\put(703,401){\makebox(0,0){$+$}}
\put(705,402){\makebox(0,0){$+$}}
\put(707,405){\makebox(0,0){$+$}}
\put(709,407){\makebox(0,0){$+$}}
\put(711,407){\makebox(0,0){$+$}}
\put(713,408){\makebox(0,0){$+$}}
\put(716,409){\makebox(0,0){$+$}}
\put(718,410){\makebox(0,0){$+$}}
\put(720,410){\makebox(0,0){$+$}}
\put(722,411){\makebox(0,0){$+$}}
\put(724,412){\makebox(0,0){$+$}}
\put(726,413){\makebox(0,0){$+$}}
\put(728,414){\makebox(0,0){$+$}}
\put(731,414){\makebox(0,0){$+$}}
\put(733,415){\makebox(0,0){$+$}}
\put(735,416){\makebox(0,0){$+$}}
\put(737,417){\makebox(0,0){$+$}}
\put(739,418){\makebox(0,0){$+$}}
\put(741,418){\makebox(0,0){$+$}}
\put(743,418){\makebox(0,0){$+$}}
\put(746,420){\makebox(0,0){$+$}}
\put(748,421){\makebox(0,0){$+$}}
\put(750,422){\makebox(0,0){$+$}}
\put(752,422){\makebox(0,0){$+$}}
\put(754,423){\makebox(0,0){$+$}}
\put(756,424){\makebox(0,0){$+$}}
\put(759,425){\makebox(0,0){$+$}}
\put(761,426){\makebox(0,0){$+$}}
\put(763,427){\makebox(0,0){$+$}}
\put(765,428){\makebox(0,0){$+$}}
\put(767,428){\makebox(0,0){$+$}}
\put(769,429){\makebox(0,0){$+$}}
\put(771,429){\makebox(0,0){$+$}}
\put(774,431){\makebox(0,0){$+$}}
\put(776,432){\makebox(0,0){$+$}}
\put(778,433){\makebox(0,0){$+$}}
\put(780,435){\makebox(0,0){$+$}}
\put(782,435){\makebox(0,0){$+$}}
\put(784,436){\makebox(0,0){$+$}}
\put(786,436){\makebox(0,0){$+$}}
\put(789,437){\makebox(0,0){$+$}}
\put(791,438){\makebox(0,0){$+$}}
\put(793,439){\makebox(0,0){$+$}}
\put(795,439){\makebox(0,0){$+$}}
\put(797,440){\makebox(0,0){$+$}}
\put(799,440){\makebox(0,0){$+$}}
\put(801,440){\makebox(0,0){$+$}}
\put(804,441){\makebox(0,0){$+$}}
\put(806,442){\makebox(0,0){$+$}}
\put(808,443){\makebox(0,0){$+$}}
\put(810,443){\makebox(0,0){$+$}}
\put(812,444){\makebox(0,0){$+$}}
\put(814,445){\makebox(0,0){$+$}}
\put(816,446){\makebox(0,0){$+$}}
\put(819,447){\makebox(0,0){$+$}}
\put(821,447){\makebox(0,0){$+$}}
\put(823,448){\makebox(0,0){$+$}}
\put(825,450){\makebox(0,0){$+$}}
\put(827,452){\makebox(0,0){$+$}}
\put(829,453){\makebox(0,0){$+$}}
\put(831,453){\makebox(0,0){$+$}}
\put(834,455){\makebox(0,0){$+$}}
\put(836,457){\makebox(0,0){$+$}}
\put(838,458){\makebox(0,0){$+$}}
\put(840,458){\makebox(0,0){$+$}}
\put(842,459){\makebox(0,0){$+$}}
\put(844,460){\makebox(0,0){$+$}}
\put(847,460){\makebox(0,0){$+$}}
\put(849,461){\makebox(0,0){$+$}}
\put(851,462){\makebox(0,0){$+$}}
\put(853,463){\makebox(0,0){$+$}}
\put(855,464){\makebox(0,0){$+$}}
\put(857,465){\makebox(0,0){$+$}}
\put(859,466){\makebox(0,0){$+$}}
\put(862,467){\makebox(0,0){$+$}}
\put(864,468){\makebox(0,0){$+$}}
\put(866,469){\makebox(0,0){$+$}}
\put(868,469){\makebox(0,0){$+$}}
\put(870,470){\makebox(0,0){$+$}}
\put(872,470){\makebox(0,0){$+$}}
\put(874,471){\makebox(0,0){$+$}}
\put(877,472){\makebox(0,0){$+$}}
\put(879,473){\makebox(0,0){$+$}}
\put(881,473){\makebox(0,0){$+$}}
\put(883,474){\makebox(0,0){$+$}}
\put(885,474){\makebox(0,0){$+$}}
\put(887,474){\makebox(0,0){$+$}}
\put(889,475){\makebox(0,0){$+$}}
\put(892,476){\makebox(0,0){$+$}}
\put(894,476){\makebox(0,0){$+$}}
\put(896,477){\makebox(0,0){$+$}}
\put(898,478){\makebox(0,0){$+$}}
\put(900,478){\makebox(0,0){$+$}}
\put(902,479){\makebox(0,0){$+$}}
\put(904,480){\makebox(0,0){$+$}}
\put(907,481){\makebox(0,0){$+$}}
\put(909,481){\makebox(0,0){$+$}}
\put(911,483){\makebox(0,0){$+$}}
\put(913,483){\makebox(0,0){$+$}}
\put(915,485){\makebox(0,0){$+$}}
\put(917,486){\makebox(0,0){$+$}}
\put(920,488){\makebox(0,0){$+$}}
\put(922,488){\makebox(0,0){$+$}}
\put(924,489){\makebox(0,0){$+$}}
\put(926,490){\makebox(0,0){$+$}}
\put(928,490){\makebox(0,0){$+$}}
\put(930,491){\makebox(0,0){$+$}}
\put(932,492){\makebox(0,0){$+$}}
\put(935,492){\makebox(0,0){$+$}}
\put(937,492){\makebox(0,0){$+$}}
\put(939,492){\makebox(0,0){$+$}}
\put(941,493){\makebox(0,0){$+$}}
\put(943,494){\makebox(0,0){$+$}}
\put(945,495){\makebox(0,0){$+$}}
\put(947,495){\makebox(0,0){$+$}}
\put(950,495){\makebox(0,0){$+$}}
\put(952,496){\makebox(0,0){$+$}}
\put(954,496){\makebox(0,0){$+$}}
\put(956,497){\makebox(0,0){$+$}}
\put(958,497){\makebox(0,0){$+$}}
\put(960,498){\makebox(0,0){$+$}}
\put(962,498){\makebox(0,0){$+$}}
\put(965,499){\makebox(0,0){$+$}}
\put(967,500){\makebox(0,0){$+$}}
\put(969,501){\makebox(0,0){$+$}}
\put(971,503){\makebox(0,0){$+$}}
\put(973,504){\makebox(0,0){$+$}}
\put(975,505){\makebox(0,0){$+$}}
\put(977,506){\makebox(0,0){$+$}}
\put(980,506){\makebox(0,0){$+$}}
\put(982,507){\makebox(0,0){$+$}}
\put(984,509){\makebox(0,0){$+$}}
\put(986,509){\makebox(0,0){$+$}}
\put(988,510){\makebox(0,0){$+$}}
\put(990,510){\makebox(0,0){$+$}}
\put(992,511){\makebox(0,0){$+$}}
\put(995,512){\makebox(0,0){$+$}}
\put(997,512){\makebox(0,0){$+$}}
\put(999,512){\makebox(0,0){$+$}}
\put(1001,512){\makebox(0,0){$+$}}
\put(1003,513){\makebox(0,0){$+$}}
\put(1005,513){\makebox(0,0){$+$}}
\put(1008,513){\makebox(0,0){$+$}}
\put(1010,514){\makebox(0,0){$+$}}
\put(1012,515){\makebox(0,0){$+$}}
\put(1014,516){\makebox(0,0){$+$}}
\put(1016,517){\makebox(0,0){$+$}}
\put(1018,517){\makebox(0,0){$+$}}
\put(1020,518){\makebox(0,0){$+$}}
\put(1023,518){\makebox(0,0){$+$}}
\put(1025,518){\makebox(0,0){$+$}}
\put(1027,519){\makebox(0,0){$+$}}
\put(1029,519){\makebox(0,0){$+$}}
\put(1031,519){\makebox(0,0){$+$}}
\put(1033,521){\makebox(0,0){$+$}}
\put(1035,521){\makebox(0,0){$+$}}
\put(1038,522){\makebox(0,0){$+$}}
\put(1040,523){\makebox(0,0){$+$}}
\put(1042,523){\makebox(0,0){$+$}}
\put(1044,524){\makebox(0,0){$+$}}
\put(1046,524){\makebox(0,0){$+$}}
\put(1048,526){\makebox(0,0){$+$}}
\put(1050,526){\makebox(0,0){$+$}}
\put(1053,527){\makebox(0,0){$+$}}
\put(1055,527){\makebox(0,0){$+$}}
\put(1057,528){\makebox(0,0){$+$}}
\put(1059,528){\makebox(0,0){$+$}}
\put(1061,529){\makebox(0,0){$+$}}
\put(1063,529){\makebox(0,0){$+$}}
\put(1065,529){\makebox(0,0){$+$}}
\put(1068,530){\makebox(0,0){$+$}}
\put(1070,531){\makebox(0,0){$+$}}
\put(1072,531){\makebox(0,0){$+$}}
\put(1074,532){\makebox(0,0){$+$}}
\put(1076,533){\makebox(0,0){$+$}}
\put(1078,533){\makebox(0,0){$+$}}
\put(1081,533){\makebox(0,0){$+$}}
\put(1083,535){\makebox(0,0){$+$}}
\put(1085,535){\makebox(0,0){$+$}}
\put(1087,535){\makebox(0,0){$+$}}
\put(1089,535){\makebox(0,0){$+$}}
\put(1091,536){\makebox(0,0){$+$}}
\put(1093,536){\makebox(0,0){$+$}}
\put(1096,536){\makebox(0,0){$+$}}
\put(1098,536){\makebox(0,0){$+$}}
\put(1100,537){\makebox(0,0){$+$}}
\put(1102,537){\makebox(0,0){$+$}}
\put(1104,537){\makebox(0,0){$+$}}
\put(1106,538){\makebox(0,0){$+$}}
\put(1108,539){\makebox(0,0){$+$}}
\put(1111,540){\makebox(0,0){$+$}}
\put(1113,541){\makebox(0,0){$+$}}
\put(1115,541){\makebox(0,0){$+$}}
\put(1117,542){\makebox(0,0){$+$}}
\put(1119,543){\makebox(0,0){$+$}}
\put(1121,544){\makebox(0,0){$+$}}
\put(1123,544){\makebox(0,0){$+$}}
\put(1126,546){\makebox(0,0){$+$}}
\put(1128,546){\makebox(0,0){$+$}}
\put(1130,547){\makebox(0,0){$+$}}
\put(1132,547){\makebox(0,0){$+$}}
\put(1134,548){\makebox(0,0){$+$}}
\put(1136,549){\makebox(0,0){$+$}}
\put(1138,549){\makebox(0,0){$+$}}
\put(1141,549){\makebox(0,0){$+$}}
\put(1143,549){\makebox(0,0){$+$}}
\put(1145,550){\makebox(0,0){$+$}}
\put(1147,551){\makebox(0,0){$+$}}
\put(1149,551){\makebox(0,0){$+$}}
\put(1151,551){\makebox(0,0){$+$}}
\put(1153,551){\makebox(0,0){$+$}}
\put(1156,551){\makebox(0,0){$+$}}
\put(1158,552){\makebox(0,0){$+$}}
\put(1160,552){\makebox(0,0){$+$}}
\put(1162,553){\makebox(0,0){$+$}}
\put(1164,554){\makebox(0,0){$+$}}
\put(1166,556){\makebox(0,0){$+$}}
\put(1169,557){\makebox(0,0){$+$}}
\put(1171,557){\makebox(0,0){$+$}}
\put(1173,558){\makebox(0,0){$+$}}
\put(1175,559){\makebox(0,0){$+$}}
\put(1177,561){\makebox(0,0){$+$}}
\put(1179,562){\makebox(0,0){$+$}}
\put(1181,563){\makebox(0,0){$+$}}
\put(1184,563){\makebox(0,0){$+$}}
\put(1186,564){\makebox(0,0){$+$}}
\put(1188,564){\makebox(0,0){$+$}}
\put(1190,565){\makebox(0,0){$+$}}
\put(1192,566){\makebox(0,0){$+$}}
\put(1194,566){\makebox(0,0){$+$}}
\put(1196,567){\makebox(0,0){$+$}}
\put(1199,568){\makebox(0,0){$+$}}
\put(1201,569){\makebox(0,0){$+$}}
\put(1203,569){\makebox(0,0){$+$}}
\put(1205,569){\makebox(0,0){$+$}}
\put(1207,570){\makebox(0,0){$+$}}
\put(1209,570){\makebox(0,0){$+$}}
\put(1211,570){\makebox(0,0){$+$}}
\put(1214,570){\makebox(0,0){$+$}}
\put(1216,571){\makebox(0,0){$+$}}
\put(1218,571){\makebox(0,0){$+$}}
\put(1220,571){\makebox(0,0){$+$}}
\put(1222,571){\makebox(0,0){$+$}}
\put(1224,571){\makebox(0,0){$+$}}
\put(1226,571){\makebox(0,0){$+$}}
\put(1229,571){\makebox(0,0){$+$}}
\put(1231,571){\makebox(0,0){$+$}}
\put(1233,572){\makebox(0,0){$+$}}
\put(1235,571){\makebox(0,0){$+$}}
\put(1237,571){\makebox(0,0){$+$}}
\put(1239,571){\makebox(0,0){$+$}}
\put(1242,571){\makebox(0,0){$+$}}
\put(1244,570){\makebox(0,0){$+$}}
\put(1246,569){\makebox(0,0){$+$}}
\put(1248,568){\makebox(0,0){$+$}}
\put(1250,567){\makebox(0,0){$+$}}
\put(1252,566){\makebox(0,0){$+$}}
\put(1254,566){\makebox(0,0){$+$}}
\put(1257,565){\makebox(0,0){$+$}}
\put(1259,565){\makebox(0,0){$+$}}
\put(1261,564){\makebox(0,0){$+$}}
\put(1263,564){\makebox(0,0){$+$}}
\put(1265,563){\makebox(0,0){$+$}}
\put(1349,516){\makebox(0,0){$+$}}
\put(1279,475){\makebox(0,0)[r]{teplota \uv{po}}}
\multiput(1299,475)(20.756,0.000){5}{\usebox{\plotpoint}}
\put(1399,475){\usebox{\plotpoint}}
\put(151,824){\usebox{\plotpoint}}
\put(151.00,824.00){\usebox{\plotpoint}}
\put(171.07,818.73){\usebox{\plotpoint}}
\put(191.33,814.27){\usebox{\plotpoint}}
\put(211.45,809.26){\usebox{\plotpoint}}
\put(231.67,804.70){\usebox{\plotpoint}}
\put(251.88,800.03){\usebox{\plotpoint}}
\put(272.13,795.52){\usebox{\plotpoint}}
\put(292.21,790.31){\usebox{\plotpoint}}
\put(312.49,786.05){\usebox{\plotpoint}}
\put(332.55,780.72){\usebox{\plotpoint}}
\put(352.82,776.30){\usebox{\plotpoint}}
\put(373.06,771.80){\usebox{\plotpoint}}
\put(393.22,766.96){\usebox{\plotpoint}}
\put(413.37,762.08){\usebox{\plotpoint}}
\put(433.62,757.60){\usebox{\plotpoint}}
\put(453.89,753.20){\usebox{\plotpoint}}
\put(473.94,747.84){\usebox{\plotpoint}}
\put(494.22,743.58){\usebox{\plotpoint}}
\put(514.31,738.40){\usebox{\plotpoint}}
\put(534.56,733.90){\usebox{\plotpoint}}
\put(554.79,729.33){\usebox{\plotpoint}}
\put(574.99,724.67){\usebox{\plotpoint}}
\put(595.10,719.61){\usebox{\plotpoint}}
\put(615.36,715.16){\usebox{\plotpoint}}
\put(635.64,710.82){\usebox{\plotpoint}}
\put(655.67,705.36){\usebox{\plotpoint}}
\put(675.96,701.10){\usebox{\plotpoint}}
\put(696.08,696.08){\usebox{\plotpoint}}
\put(716.33,691.58){\usebox{\plotpoint}}
\put(736.52,686.86){\usebox{\plotpoint}}
\put(756.76,682.37){\usebox{\plotpoint}}
\put(776.83,677.14){\usebox{\plotpoint}}
\put(797.10,672.73){\usebox{\plotpoint}}
\put(817.38,668.35){\usebox{\plotpoint}}
\put(837.41,662.93){\usebox{\plotpoint}}
\put(857.69,658.63){\usebox{\plotpoint}}
\put(877.84,653.76){\usebox{\plotpoint}}
\put(898.02,648.90){\usebox{\plotpoint}}
\put(918.27,644.43){\usebox{\plotpoint}}
\put(938.55,640.08){\usebox{\plotpoint}}
\put(958.58,634.66){\usebox{\plotpoint}}
\put(978.87,630.40){\usebox{\plotpoint}}
\put(998.97,625.28){\usebox{\plotpoint}}
\put(1019.22,620.78){\usebox{\plotpoint}}
\put(1039.44,616.15){\usebox{\plotpoint}}
\put(1059.65,611.56){\usebox{\plotpoint}}
\put(1079.75,606.43){\usebox{\plotpoint}}
\put(1100.01,602.00){\usebox{\plotpoint}}
\put(1120.29,597.65){\usebox{\plotpoint}}
\put(1140.31,592.19){\usebox{\plotpoint}}
\put(1160.60,587.93){\usebox{\plotpoint}}
\put(1180.74,582.96){\usebox{\plotpoint}}
\put(1200.99,578.46){\usebox{\plotpoint}}
\put(1221.17,573.68){\usebox{\plotpoint}}
\put(1241.42,569.26){\usebox{\plotpoint}}
\put(1261.48,563.96){\usebox{\plotpoint}}
\put(1265,563){\usebox{\plotpoint}}
\put(151.0,131.0){\rule[-0.200pt]{0.400pt}{175.375pt}}
\put(151.0,131.0){\rule[-0.200pt]{310.279pt}{0.400pt}}
\put(1439.0,131.0){\rule[-0.200pt]{0.400pt}{175.375pt}}
\put(151.0,859.0){\rule[-0.200pt]{310.279pt}{0.400pt}}
\end{picture}

\caption{Priebeh teploty pre var vody}  \label{G_4}
\end{figure}

\begin{figure}
% GNUPLOT: LaTeX picture
\setlength{\unitlength}{0.240900pt}
\ifx\plotpoint\undefined\newsavebox{\plotpoint}\fi
\sbox{\plotpoint}{\rule[-0.200pt]{0.400pt}{0.400pt}}%
\begin{picture}(1500,900)(0,0)
\sbox{\plotpoint}{\rule[-0.200pt]{0.400pt}{0.400pt}}%
\put(151.0,131.0){\rule[-0.200pt]{4.818pt}{0.400pt}}
\put(131,131){\makebox(0,0)[r]{ 18}}
\put(1419.0,131.0){\rule[-0.200pt]{4.818pt}{0.400pt}}
\put(151.0,212.0){\rule[-0.200pt]{4.818pt}{0.400pt}}
\put(131,212){\makebox(0,0)[r]{ 19}}
\put(1419.0,212.0){\rule[-0.200pt]{4.818pt}{0.400pt}}
\put(151.0,293.0){\rule[-0.200pt]{4.818pt}{0.400pt}}
\put(131,293){\makebox(0,0)[r]{ 20}}
\put(1419.0,293.0){\rule[-0.200pt]{4.818pt}{0.400pt}}
\put(151.0,374.0){\rule[-0.200pt]{4.818pt}{0.400pt}}
\put(131,374){\makebox(0,0)[r]{ 21}}
\put(1419.0,374.0){\rule[-0.200pt]{4.818pt}{0.400pt}}
\put(151.0,455.0){\rule[-0.200pt]{4.818pt}{0.400pt}}
\put(131,455){\makebox(0,0)[r]{ 22}}
\put(1419.0,455.0){\rule[-0.200pt]{4.818pt}{0.400pt}}
\put(151.0,535.0){\rule[-0.200pt]{4.818pt}{0.400pt}}
\put(131,535){\makebox(0,0)[r]{ 23}}
\put(1419.0,535.0){\rule[-0.200pt]{4.818pt}{0.400pt}}
\put(151.0,616.0){\rule[-0.200pt]{4.818pt}{0.400pt}}
\put(131,616){\makebox(0,0)[r]{ 24}}
\put(1419.0,616.0){\rule[-0.200pt]{4.818pt}{0.400pt}}
\put(151.0,697.0){\rule[-0.200pt]{4.818pt}{0.400pt}}
\put(131,697){\makebox(0,0)[r]{ 25}}
\put(1419.0,697.0){\rule[-0.200pt]{4.818pt}{0.400pt}}
\put(151.0,778.0){\rule[-0.200pt]{4.818pt}{0.400pt}}
\put(131,778){\makebox(0,0)[r]{ 26}}
\put(1419.0,778.0){\rule[-0.200pt]{4.818pt}{0.400pt}}
\put(151.0,859.0){\rule[-0.200pt]{4.818pt}{0.400pt}}
\put(131,859){\makebox(0,0)[r]{ 27}}
\put(1419.0,859.0){\rule[-0.200pt]{4.818pt}{0.400pt}}
\put(151.0,131.0){\rule[-0.200pt]{0.400pt}{4.818pt}}
\put(151,90){\makebox(0,0){ 0}}
\put(151.0,839.0){\rule[-0.200pt]{0.400pt}{4.818pt}}
\put(312.0,131.0){\rule[-0.200pt]{0.400pt}{4.818pt}}
\put(312,90){\makebox(0,0){ 50}}
\put(312.0,839.0){\rule[-0.200pt]{0.400pt}{4.818pt}}
\put(473.0,131.0){\rule[-0.200pt]{0.400pt}{4.818pt}}
\put(473,90){\makebox(0,0){ 100}}
\put(473.0,839.0){\rule[-0.200pt]{0.400pt}{4.818pt}}
\put(634.0,131.0){\rule[-0.200pt]{0.400pt}{4.818pt}}
\put(634,90){\makebox(0,0){ 150}}
\put(634.0,839.0){\rule[-0.200pt]{0.400pt}{4.818pt}}
\put(795.0,131.0){\rule[-0.200pt]{0.400pt}{4.818pt}}
\put(795,90){\makebox(0,0){ 200}}
\put(795.0,839.0){\rule[-0.200pt]{0.400pt}{4.818pt}}
\put(956.0,131.0){\rule[-0.200pt]{0.400pt}{4.818pt}}
\put(956,90){\makebox(0,0){ 250}}
\put(956.0,839.0){\rule[-0.200pt]{0.400pt}{4.818pt}}
\put(1117.0,131.0){\rule[-0.200pt]{0.400pt}{4.818pt}}
\put(1117,90){\makebox(0,0){ 300}}
\put(1117.0,839.0){\rule[-0.200pt]{0.400pt}{4.818pt}}
\put(1278.0,131.0){\rule[-0.200pt]{0.400pt}{4.818pt}}
\put(1278,90){\makebox(0,0){ 350}}
\put(1278.0,839.0){\rule[-0.200pt]{0.400pt}{4.818pt}}
\put(1439.0,131.0){\rule[-0.200pt]{0.400pt}{4.818pt}}
\put(1439,90){\makebox(0,0){ 400}}
\put(1439.0,839.0){\rule[-0.200pt]{0.400pt}{4.818pt}}
\put(151.0,131.0){\rule[-0.200pt]{0.400pt}{175.375pt}}
\put(151.0,131.0){\rule[-0.200pt]{310.279pt}{0.400pt}}
\put(1439.0,131.0){\rule[-0.200pt]{0.400pt}{175.375pt}}
\put(151.0,859.0){\rule[-0.200pt]{310.279pt}{0.400pt}}
\put(30,495){\makebox(0,0){\popi{t}{\C}}}
\put(795,29){\makebox(0,0){\popi{t}{s}}}
\put(1279,536){\makebox(0,0)[r]{namerané dáta}}
\put(151,746){\makebox(0,0){$+$}}
\put(151,657){\makebox(0,0){$+$}}
\put(157,770){\makebox(0,0){$+$}}
\put(164,786){\makebox(0,0){$+$}}
\put(170,794){\makebox(0,0){$+$}}
\put(177,802){\makebox(0,0){$+$}}
\put(183,810){\makebox(0,0){$+$}}
\put(190,819){\makebox(0,0){$+$}}
\put(196,827){\makebox(0,0){$+$}}
\put(203,827){\makebox(0,0){$+$}}
\put(209,827){\makebox(0,0){$+$}}
\put(215,827){\makebox(0,0){$+$}}
\put(222,827){\makebox(0,0){$+$}}
\put(228,835){\makebox(0,0){$+$}}
\put(235,835){\makebox(0,0){$+$}}
\put(241,835){\makebox(0,0){$+$}}
\put(248,835){\makebox(0,0){$+$}}
\put(254,835){\makebox(0,0){$+$}}
\put(260,835){\makebox(0,0){$+$}}
\put(267,835){\makebox(0,0){$+$}}
\put(273,835){\makebox(0,0){$+$}}
\put(280,835){\makebox(0,0){$+$}}
\put(286,835){\makebox(0,0){$+$}}
\put(293,835){\makebox(0,0){$+$}}
\put(299,835){\makebox(0,0){$+$}}
\put(306,835){\makebox(0,0){$+$}}
\put(312,835){\makebox(0,0){$+$}}
\put(318,835){\makebox(0,0){$+$}}
\put(325,835){\makebox(0,0){$+$}}
\put(331,835){\makebox(0,0){$+$}}
\put(338,835){\makebox(0,0){$+$}}
\put(344,835){\makebox(0,0){$+$}}
\put(351,835){\makebox(0,0){$+$}}
\put(357,835){\makebox(0,0){$+$}}
\put(364,835){\makebox(0,0){$+$}}
\put(370,835){\makebox(0,0){$+$}}
\put(376,835){\makebox(0,0){$+$}}
\put(383,835){\makebox(0,0){$+$}}
\put(389,835){\makebox(0,0){$+$}}
\put(396,835){\makebox(0,0){$+$}}
\put(402,835){\makebox(0,0){$+$}}
\put(409,835){\makebox(0,0){$+$}}
\put(415,835){\makebox(0,0){$+$}}
\put(421,835){\makebox(0,0){$+$}}
\put(428,835){\makebox(0,0){$+$}}
\put(434,835){\makebox(0,0){$+$}}
\put(441,835){\makebox(0,0){$+$}}
\put(447,835){\makebox(0,0){$+$}}
\put(454,827){\makebox(0,0){$+$}}
\put(460,827){\makebox(0,0){$+$}}
\put(467,827){\makebox(0,0){$+$}}
\put(473,819){\makebox(0,0){$+$}}
\put(479,819){\makebox(0,0){$+$}}
\put(486,810){\makebox(0,0){$+$}}
\put(492,786){\makebox(0,0){$+$}}
\put(499,754){\makebox(0,0){$+$}}
\put(505,730){\makebox(0,0){$+$}}
\put(512,713){\makebox(0,0){$+$}}
\put(518,697){\makebox(0,0){$+$}}
\put(525,681){\makebox(0,0){$+$}}
\put(531,673){\makebox(0,0){$+$}}
\put(537,665){\makebox(0,0){$+$}}
\put(544,649){\makebox(0,0){$+$}}
\put(550,633){\makebox(0,0){$+$}}
\put(557,624){\makebox(0,0){$+$}}
\put(563,608){\makebox(0,0){$+$}}
\put(570,600){\makebox(0,0){$+$}}
\put(576,584){\makebox(0,0){$+$}}
\put(582,568){\makebox(0,0){$+$}}
\put(589,552){\makebox(0,0){$+$}}
\put(595,535){\makebox(0,0){$+$}}
\put(602,527){\makebox(0,0){$+$}}
\put(608,511){\makebox(0,0){$+$}}
\put(615,495){\makebox(0,0){$+$}}
\put(621,479){\makebox(0,0){$+$}}
\put(628,471){\makebox(0,0){$+$}}
\put(634,463){\makebox(0,0){$+$}}
\put(640,455){\makebox(0,0){$+$}}
\put(647,438){\makebox(0,0){$+$}}
\put(653,430){\makebox(0,0){$+$}}
\put(660,422){\makebox(0,0){$+$}}
\put(666,422){\makebox(0,0){$+$}}
\put(673,406){\makebox(0,0){$+$}}
\put(679,406){\makebox(0,0){$+$}}
\put(686,398){\makebox(0,0){$+$}}
\put(692,382){\makebox(0,0){$+$}}
\put(698,374){\makebox(0,0){$+$}}
\put(705,366){\makebox(0,0){$+$}}
\put(711,357){\makebox(0,0){$+$}}
\put(718,349){\makebox(0,0){$+$}}
\put(724,341){\makebox(0,0){$+$}}
\put(731,341){\makebox(0,0){$+$}}
\put(737,333){\makebox(0,0){$+$}}
\put(743,333){\makebox(0,0){$+$}}
\put(750,325){\makebox(0,0){$+$}}
\put(756,317){\makebox(0,0){$+$}}
\put(763,317){\makebox(0,0){$+$}}
\put(769,309){\makebox(0,0){$+$}}
\put(776,301){\makebox(0,0){$+$}}
\put(782,301){\makebox(0,0){$+$}}
\put(789,293){\makebox(0,0){$+$}}
\put(795,285){\makebox(0,0){$+$}}
\put(801,285){\makebox(0,0){$+$}}
\put(808,277){\makebox(0,0){$+$}}
\put(814,269){\makebox(0,0){$+$}}
\put(821,269){\makebox(0,0){$+$}}
\put(827,260){\makebox(0,0){$+$}}
\put(834,260){\makebox(0,0){$+$}}
\put(840,252){\makebox(0,0){$+$}}
\put(847,252){\makebox(0,0){$+$}}
\put(853,252){\makebox(0,0){$+$}}
\put(859,244){\makebox(0,0){$+$}}
\put(866,244){\makebox(0,0){$+$}}
\put(872,244){\makebox(0,0){$+$}}
\put(879,244){\makebox(0,0){$+$}}
\put(885,236){\makebox(0,0){$+$}}
\put(892,236){\makebox(0,0){$+$}}
\put(898,236){\makebox(0,0){$+$}}
\put(904,236){\makebox(0,0){$+$}}
\put(911,236){\makebox(0,0){$+$}}
\put(917,236){\makebox(0,0){$+$}}
\put(924,228){\makebox(0,0){$+$}}
\put(930,228){\makebox(0,0){$+$}}
\put(937,228){\makebox(0,0){$+$}}
\put(943,228){\makebox(0,0){$+$}}
\put(950,228){\makebox(0,0){$+$}}
\put(956,220){\makebox(0,0){$+$}}
\put(962,220){\makebox(0,0){$+$}}
\put(969,220){\makebox(0,0){$+$}}
\put(975,220){\makebox(0,0){$+$}}
\put(982,220){\makebox(0,0){$+$}}
\put(988,220){\makebox(0,0){$+$}}
\put(995,220){\makebox(0,0){$+$}}
\put(1001,220){\makebox(0,0){$+$}}
\put(1008,220){\makebox(0,0){$+$}}
\put(1014,220){\makebox(0,0){$+$}}
\put(1020,212){\makebox(0,0){$+$}}
\put(1027,220){\makebox(0,0){$+$}}
\put(1033,212){\makebox(0,0){$+$}}
\put(1040,212){\makebox(0,0){$+$}}
\put(1046,212){\makebox(0,0){$+$}}
\put(1053,212){\makebox(0,0){$+$}}
\put(1059,212){\makebox(0,0){$+$}}
\put(1065,212){\makebox(0,0){$+$}}
\put(1072,212){\makebox(0,0){$+$}}
\put(1078,212){\makebox(0,0){$+$}}
\put(1085,212){\makebox(0,0){$+$}}
\put(1091,212){\makebox(0,0){$+$}}
\put(1098,212){\makebox(0,0){$+$}}
\put(1104,212){\makebox(0,0){$+$}}
\put(1111,212){\makebox(0,0){$+$}}
\put(1117,212){\makebox(0,0){$+$}}
\put(1123,212){\makebox(0,0){$+$}}
\put(1130,212){\makebox(0,0){$+$}}
\put(1136,212){\makebox(0,0){$+$}}
\put(1143,212){\makebox(0,0){$+$}}
\put(1149,212){\makebox(0,0){$+$}}
\put(1156,212){\makebox(0,0){$+$}}
\put(1162,212){\makebox(0,0){$+$}}
\put(1169,212){\makebox(0,0){$+$}}
\put(1175,212){\makebox(0,0){$+$}}
\put(1181,212){\makebox(0,0){$+$}}
\put(1188,212){\makebox(0,0){$+$}}
\put(1194,212){\makebox(0,0){$+$}}
\put(1201,212){\makebox(0,0){$+$}}
\put(1207,212){\makebox(0,0){$+$}}
\put(1214,212){\makebox(0,0){$+$}}
\put(1220,212){\makebox(0,0){$+$}}
\put(1226,252){\makebox(0,0){$+$}}
\put(1233,269){\makebox(0,0){$+$}}
\put(1239,285){\makebox(0,0){$+$}}
\put(1246,285){\makebox(0,0){$+$}}
\put(1252,285){\makebox(0,0){$+$}}
\put(1259,293){\makebox(0,0){$+$}}
\put(1265,293){\makebox(0,0){$+$}}
\put(1272,293){\makebox(0,0){$+$}}
\put(1278,293){\makebox(0,0){$+$}}
\put(1284,293){\makebox(0,0){$+$}}
\put(1291,293){\makebox(0,0){$+$}}
\put(1297,293){\makebox(0,0){$+$}}
\put(1304,285){\makebox(0,0){$+$}}
\put(1310,285){\makebox(0,0){$+$}}
\put(1317,285){\makebox(0,0){$+$}}
\put(1323,285){\makebox(0,0){$+$}}
\put(1330,285){\makebox(0,0){$+$}}
\put(1336,285){\makebox(0,0){$+$}}
\put(1342,285){\makebox(0,0){$+$}}
\put(1349,285){\makebox(0,0){$+$}}
\put(1355,277){\makebox(0,0){$+$}}
\put(1362,277){\makebox(0,0){$+$}}
\put(1349,536){\makebox(0,0){$+$}}
\put(1279,495){\makebox(0,0)[r]{teplota \uv{po}}}
\multiput(1299,495)(20.756,0.000){5}{\usebox{\plotpoint}}
\put(1399,495){\usebox{\plotpoint}}
\put(151,277){\usebox{\plotpoint}}
\put(151.00,277.00){\usebox{\plotpoint}}
\put(171.71,276.00){\usebox{\plotpoint}}
\put(192.42,274.63){\usebox{\plotpoint}}
\put(213.10,272.91){\usebox{\plotpoint}}
\put(233.79,271.25){\usebox{\plotpoint}}
\put(254.49,270.00){\usebox{\plotpoint}}
\put(275.20,268.83){\usebox{\plotpoint}}
\put(295.89,267.18){\usebox{\plotpoint}}
\put(316.57,265.45){\usebox{\plotpoint}}
\put(337.27,264.00){\usebox{\plotpoint}}
\put(357.99,263.08){\usebox{\plotpoint}}
\put(378.67,261.36){\usebox{\plotpoint}}
\put(399.36,259.72){\usebox{\plotpoint}}
\put(420.04,258.00){\usebox{\plotpoint}}
\put(440.77,257.33){\usebox{\plotpoint}}
\put(461.46,255.63){\usebox{\plotpoint}}
\put(482.14,253.90){\usebox{\plotpoint}}
\put(502.83,252.24){\usebox{\plotpoint}}
\put(523.56,251.54){\usebox{\plotpoint}}
\put(544.24,249.83){\usebox{\plotpoint}}
\put(564.93,248.17){\usebox{\plotpoint}}
\put(585.62,246.45){\usebox{\plotpoint}}
\put(606.34,245.74){\usebox{\plotpoint}}
\put(627.03,244.08){\usebox{\plotpoint}}
\put(647.72,242.36){\usebox{\plotpoint}}
\put(668.41,240.72){\usebox{\plotpoint}}
\put(689.13,239.99){\usebox{\plotpoint}}
\put(709.82,238.32){\usebox{\plotpoint}}
\put(730.51,236.62){\usebox{\plotpoint}}
\put(751.19,234.90){\usebox{\plotpoint}}
\put(771.91,234.00){\usebox{\plotpoint}}
\put(792.60,232.53){\usebox{\plotpoint}}
\put(813.29,230.82){\usebox{\plotpoint}}
\put(833.98,229.17){\usebox{\plotpoint}}
\put(854.68,228.00){\usebox{\plotpoint}}
\put(875.39,226.80){\usebox{\plotpoint}}
\put(896.08,225.08){\usebox{\plotpoint}}
\put(916.76,223.35){\usebox{\plotpoint}}
\put(937.49,222.71){\usebox{\plotpoint}}
\put(958.17,220.99){\usebox{\plotpoint}}
\put(978.86,219.32){\usebox{\plotpoint}}
\put(999.55,217.62){\usebox{\plotpoint}}
\put(1020.27,216.90){\usebox{\plotpoint}}
\put(1040.96,215.25){\usebox{\plotpoint}}
\put(1061.65,213.53){\usebox{\plotpoint}}
\put(1082.33,211.82){\usebox{\plotpoint}}
\put(1103.06,211.00){\usebox{\plotpoint}}
\put(1123.75,209.44){\usebox{\plotpoint}}
\put(1144.44,207.80){\usebox{\plotpoint}}
\put(1165.12,206.07){\usebox{\plotpoint}}
\put(1185.83,205.00){\usebox{\plotpoint}}
\put(1206.53,203.71){\usebox{\plotpoint}}
\put(1227.22,201.98){\usebox{\plotpoint}}
\put(1247.91,200.31){\usebox{\plotpoint}}
\put(1268.61,199.00){\usebox{\plotpoint}}
\put(1289.32,197.90){\usebox{\plotpoint}}
\put(1310.01,196.25){\usebox{\plotpoint}}
\put(1330.69,194.53){\usebox{\plotpoint}}
\put(1351.38,193.00){\usebox{\plotpoint}}
\put(1362,193){\usebox{\plotpoint}}
\sbox{\plotpoint}{\rule[-0.400pt]{0.800pt}{0.800pt}}%
\sbox{\plotpoint}{\rule[-0.200pt]{0.400pt}{0.400pt}}%
\put(1279,454){\makebox(0,0)[r]{teplota \uv{pred}}}
\sbox{\plotpoint}{\rule[-0.400pt]{0.800pt}{0.800pt}}%
\put(1299.0,454.0){\rule[-0.400pt]{24.090pt}{0.800pt}}
\put(151,834){\usebox{\plotpoint}}
\put(151.0,834.0){\rule[-0.400pt]{291.730pt}{0.800pt}}
\sbox{\plotpoint}{\rule[-0.200pt]{0.400pt}{0.400pt}}%
\put(151.0,131.0){\rule[-0.200pt]{0.400pt}{175.375pt}}
\put(151.0,131.0){\rule[-0.200pt]{310.279pt}{0.400pt}}
\put(1439.0,131.0){\rule[-0.200pt]{0.400pt}{175.375pt}}
\put(151.0,859.0){\rule[-0.200pt]{310.279pt}{0.400pt}}
\end{picture}

\caption{Priebeh teploty topenie ľadu}  \label{G_5}
\end{figure}

\begin{figure}
% GNUPLOT: LaTeX picture
\setlength{\unitlength}{0.240900pt}
\ifx\plotpoint\undefined\newsavebox{\plotpoint}\fi
\begin{picture}(1500,900)(0,0)
\sbox{\plotpoint}{\rule[-0.200pt]{0.400pt}{0.400pt}}%
\put(151.0,131.0){\rule[-0.200pt]{4.818pt}{0.400pt}}
\put(131,131){\makebox(0,0)[r]{ 24}}
\put(1419.0,131.0){\rule[-0.200pt]{4.818pt}{0.400pt}}
\put(151.0,222.0){\rule[-0.200pt]{4.818pt}{0.400pt}}
\put(131,222){\makebox(0,0)[r]{ 25}}
\put(1419.0,222.0){\rule[-0.200pt]{4.818pt}{0.400pt}}
\put(151.0,313.0){\rule[-0.200pt]{4.818pt}{0.400pt}}
\put(131,313){\makebox(0,0)[r]{ 26}}
\put(1419.0,313.0){\rule[-0.200pt]{4.818pt}{0.400pt}}
\put(151.0,404.0){\rule[-0.200pt]{4.818pt}{0.400pt}}
\put(131,404){\makebox(0,0)[r]{ 27}}
\put(1419.0,404.0){\rule[-0.200pt]{4.818pt}{0.400pt}}
\put(151.0,495.0){\rule[-0.200pt]{4.818pt}{0.400pt}}
\put(131,495){\makebox(0,0)[r]{ 28}}
\put(1419.0,495.0){\rule[-0.200pt]{4.818pt}{0.400pt}}
\put(151.0,586.0){\rule[-0.200pt]{4.818pt}{0.400pt}}
\put(131,586){\makebox(0,0)[r]{ 29}}
\put(1419.0,586.0){\rule[-0.200pt]{4.818pt}{0.400pt}}
\put(151.0,677.0){\rule[-0.200pt]{4.818pt}{0.400pt}}
\put(131,677){\makebox(0,0)[r]{ 30}}
\put(1419.0,677.0){\rule[-0.200pt]{4.818pt}{0.400pt}}
\put(151.0,768.0){\rule[-0.200pt]{4.818pt}{0.400pt}}
\put(131,768){\makebox(0,0)[r]{ 31}}
\put(1419.0,768.0){\rule[-0.200pt]{4.818pt}{0.400pt}}
\put(151.0,859.0){\rule[-0.200pt]{4.818pt}{0.400pt}}
\put(131,859){\makebox(0,0)[r]{ 32}}
\put(1419.0,859.0){\rule[-0.200pt]{4.818pt}{0.400pt}}
\put(151.0,131.0){\rule[-0.200pt]{0.400pt}{4.818pt}}
\put(151,90){\makebox(0,0){ 0}}
\put(151.0,839.0){\rule[-0.200pt]{0.400pt}{4.818pt}}
\put(366.0,131.0){\rule[-0.200pt]{0.400pt}{4.818pt}}
\put(366,90){\makebox(0,0){ 100}}
\put(366.0,839.0){\rule[-0.200pt]{0.400pt}{4.818pt}}
\put(580.0,131.0){\rule[-0.200pt]{0.400pt}{4.818pt}}
\put(580,90){\makebox(0,0){ 200}}
\put(580.0,839.0){\rule[-0.200pt]{0.400pt}{4.818pt}}
\put(795.0,131.0){\rule[-0.200pt]{0.400pt}{4.818pt}}
\put(795,90){\makebox(0,0){ 300}}
\put(795.0,839.0){\rule[-0.200pt]{0.400pt}{4.818pt}}
\put(1010.0,131.0){\rule[-0.200pt]{0.400pt}{4.818pt}}
\put(1010,90){\makebox(0,0){ 400}}
\put(1010.0,839.0){\rule[-0.200pt]{0.400pt}{4.818pt}}
\put(1224.0,131.0){\rule[-0.200pt]{0.400pt}{4.818pt}}
\put(1224,90){\makebox(0,0){ 500}}
\put(1224.0,839.0){\rule[-0.200pt]{0.400pt}{4.818pt}}
\put(1439.0,131.0){\rule[-0.200pt]{0.400pt}{4.818pt}}
\put(1439,90){\makebox(0,0){ 600}}
\put(1439.0,839.0){\rule[-0.200pt]{0.400pt}{4.818pt}}
\put(151.0,131.0){\rule[-0.200pt]{0.400pt}{175.375pt}}
\put(151.0,131.0){\rule[-0.200pt]{310.279pt}{0.400pt}}
\put(1439.0,131.0){\rule[-0.200pt]{0.400pt}{175.375pt}}
\put(151.0,859.0){\rule[-0.200pt]{310.279pt}{0.400pt}}
\put(30,495){\makebox(0,0){\popi{t}{\C}}}
\put(795,29){\makebox(0,0){\popi{t}{s}}}
\put(471,536){\makebox(0,0)[r]{namerané dáta}}
\put(151,386){\makebox(0,0){$+$}}
\put(172,258){\makebox(0,0){$+$}}
\put(194,231){\makebox(0,0){$+$}}
\put(215,213){\makebox(0,0){$+$}}
\put(237,204){\makebox(0,0){$+$}}
\put(258,195){\makebox(0,0){$+$}}
\put(280,195){\makebox(0,0){$+$}}
\put(301,195){\makebox(0,0){$+$}}
\put(323,195){\makebox(0,0){$+$}}
\put(344,186){\makebox(0,0){$+$}}
\put(366,186){\makebox(0,0){$+$}}
\put(387,186){\makebox(0,0){$+$}}
\put(409,186){\makebox(0,0){$+$}}
\put(430,186){\makebox(0,0){$+$}}
\put(452,186){\makebox(0,0){$+$}}
\put(473,186){\makebox(0,0){$+$}}
\put(494,186){\makebox(0,0){$+$}}
\put(516,177){\makebox(0,0){$+$}}
\put(537,177){\makebox(0,0){$+$}}
\put(559,186){\makebox(0,0){$+$}}
\put(580,177){\makebox(0,0){$+$}}
\put(602,177){\makebox(0,0){$+$}}
\put(623,177){\makebox(0,0){$+$}}
\put(645,177){\makebox(0,0){$+$}}
\put(666,177){\makebox(0,0){$+$}}
\put(688,177){\makebox(0,0){$+$}}
\put(709,177){\makebox(0,0){$+$}}
\put(731,177){\makebox(0,0){$+$}}
\put(752,177){\makebox(0,0){$+$}}
\put(774,167){\makebox(0,0){$+$}}
\put(795,167){\makebox(0,0){$+$}}
\put(816,167){\makebox(0,0){$+$}}
\put(838,167){\makebox(0,0){$+$}}
\put(859,167){\makebox(0,0){$+$}}
\put(881,167){\makebox(0,0){$+$}}
\put(902,167){\makebox(0,0){$+$}}
\put(924,177){\makebox(0,0){$+$}}
\put(945,222){\makebox(0,0){$+$}}
\put(967,258){\makebox(0,0){$+$}}
\put(988,331){\makebox(0,0){$+$}}
\put(1010,741){\makebox(0,0){$+$}}
\put(1031,814){\makebox(0,0){$+$}}
\put(1053,823){\makebox(0,0){$+$}}
\put(1074,823){\makebox(0,0){$+$}}
\put(1096,823){\makebox(0,0){$+$}}
\put(1117,823){\makebox(0,0){$+$}}
\put(1138,823){\makebox(0,0){$+$}}
\put(1160,823){\makebox(0,0){$+$}}
\put(1181,814){\makebox(0,0){$+$}}
\put(1203,823){\makebox(0,0){$+$}}
\put(1224,823){\makebox(0,0){$+$}}
\put(1246,823){\makebox(0,0){$+$}}
\put(1267,823){\makebox(0,0){$+$}}
\put(541,536){\makebox(0,0){$+$}}
\put(471,495){\makebox(0,0)[r]{teplota \uv{po}}}
\multiput(491,495)(20.756,0.000){5}{\usebox{\plotpoint}}
\put(591,495){\usebox{\plotpoint}}
\put(151,821){\usebox{\plotpoint}}
\put(151.00,821.00){\usebox{\plotpoint}}
\put(171.76,821.00){\usebox{\plotpoint}}
\put(192.51,821.00){\usebox{\plotpoint}}
\put(213.27,821.00){\usebox{\plotpoint}}
\put(234.02,821.00){\usebox{\plotpoint}}
\put(254.78,821.00){\usebox{\plotpoint}}
\put(275.53,821.00){\usebox{\plotpoint}}
\put(296.29,821.00){\usebox{\plotpoint}}
\put(317.04,821.00){\usebox{\plotpoint}}
\put(337.80,821.00){\usebox{\plotpoint}}
\put(358.55,821.00){\usebox{\plotpoint}}
\put(379.31,821.00){\usebox{\plotpoint}}
\put(400.07,821.00){\usebox{\plotpoint}}
\put(420.82,821.00){\usebox{\plotpoint}}
\put(441.58,821.00){\usebox{\plotpoint}}
\put(462.33,821.00){\usebox{\plotpoint}}
\put(483.09,821.00){\usebox{\plotpoint}}
\put(503.84,821.00){\usebox{\plotpoint}}
\put(524.60,821.00){\usebox{\plotpoint}}
\put(545.35,821.00){\usebox{\plotpoint}}
\put(566.11,821.00){\usebox{\plotpoint}}
\put(586.87,821.00){\usebox{\plotpoint}}
\put(607.62,821.00){\usebox{\plotpoint}}
\put(628.38,821.00){\usebox{\plotpoint}}
\put(649.13,821.00){\usebox{\plotpoint}}
\put(669.89,821.00){\usebox{\plotpoint}}
\put(690.64,821.00){\usebox{\plotpoint}}
\put(711.40,821.00){\usebox{\plotpoint}}
\put(732.15,821.00){\usebox{\plotpoint}}
\put(752.91,821.00){\usebox{\plotpoint}}
\put(773.66,821.00){\usebox{\plotpoint}}
\put(794.42,821.00){\usebox{\plotpoint}}
\put(815.18,821.00){\usebox{\plotpoint}}
\put(835.93,821.00){\usebox{\plotpoint}}
\put(856.69,821.00){\usebox{\plotpoint}}
\put(877.44,821.00){\usebox{\plotpoint}}
\put(898.20,821.00){\usebox{\plotpoint}}
\put(918.95,821.00){\usebox{\plotpoint}}
\put(939.71,821.00){\usebox{\plotpoint}}
\put(960.46,821.00){\usebox{\plotpoint}}
\put(981.22,821.00){\usebox{\plotpoint}}
\put(1001.98,821.00){\usebox{\plotpoint}}
\put(1022.73,821.00){\usebox{\plotpoint}}
\put(1043.49,821.00){\usebox{\plotpoint}}
\put(1064.24,821.00){\usebox{\plotpoint}}
\put(1085.00,821.00){\usebox{\plotpoint}}
\put(1105.75,821.00){\usebox{\plotpoint}}
\put(1126.51,821.00){\usebox{\plotpoint}}
\put(1147.26,821.00){\usebox{\plotpoint}}
\put(1168.02,821.00){\usebox{\plotpoint}}
\put(1188.77,821.00){\usebox{\plotpoint}}
\put(1209.53,821.00){\usebox{\plotpoint}}
\put(1230.29,821.00){\usebox{\plotpoint}}
\put(1251.04,821.00){\usebox{\plotpoint}}
\put(1267,821){\usebox{\plotpoint}}
\sbox{\plotpoint}{\rule[-0.400pt]{0.800pt}{0.800pt}}%
\sbox{\plotpoint}{\rule[-0.200pt]{0.400pt}{0.400pt}}%
\put(471,454){\makebox(0,0)[r]{teplota \uv{pred}}}
\sbox{\plotpoint}{\rule[-0.400pt]{0.800pt}{0.800pt}}%
\put(491.0,454.0){\rule[-0.400pt]{24.090pt}{0.800pt}}
\put(151,198){\usebox{\plotpoint}}
\put(162,195.84){\rule{2.891pt}{0.800pt}}
\multiput(162.00,196.34)(6.000,-1.000){2}{\rule{1.445pt}{0.800pt}}
\put(151.0,198.0){\rule[-0.400pt]{2.650pt}{0.800pt}}
\put(185,194.84){\rule{2.650pt}{0.800pt}}
\multiput(185.00,195.34)(5.500,-1.000){2}{\rule{1.325pt}{0.800pt}}
\put(174.0,197.0){\rule[-0.400pt]{2.650pt}{0.800pt}}
\put(207,193.84){\rule{2.891pt}{0.800pt}}
\multiput(207.00,194.34)(6.000,-1.000){2}{\rule{1.445pt}{0.800pt}}
\put(196.0,196.0){\rule[-0.400pt]{2.650pt}{0.800pt}}
\put(230,192.84){\rule{2.650pt}{0.800pt}}
\multiput(230.00,193.34)(5.500,-1.000){2}{\rule{1.325pt}{0.800pt}}
\put(219.0,195.0){\rule[-0.400pt]{2.650pt}{0.800pt}}
\put(252,191.84){\rule{2.891pt}{0.800pt}}
\multiput(252.00,192.34)(6.000,-1.000){2}{\rule{1.445pt}{0.800pt}}
\put(241.0,194.0){\rule[-0.400pt]{2.650pt}{0.800pt}}
\put(275,190.84){\rule{2.650pt}{0.800pt}}
\multiput(275.00,191.34)(5.500,-1.000){2}{\rule{1.325pt}{0.800pt}}
\put(264.0,193.0){\rule[-0.400pt]{2.650pt}{0.800pt}}
\put(298,189.84){\rule{2.650pt}{0.800pt}}
\multiput(298.00,190.34)(5.500,-1.000){2}{\rule{1.325pt}{0.800pt}}
\put(286.0,192.0){\rule[-0.400pt]{2.891pt}{0.800pt}}
\put(320,188.84){\rule{2.650pt}{0.800pt}}
\multiput(320.00,189.34)(5.500,-1.000){2}{\rule{1.325pt}{0.800pt}}
\put(309.0,191.0){\rule[-0.400pt]{2.650pt}{0.800pt}}
\put(343,187.84){\rule{2.650pt}{0.800pt}}
\multiput(343.00,188.34)(5.500,-1.000){2}{\rule{1.325pt}{0.800pt}}
\put(331.0,190.0){\rule[-0.400pt]{2.891pt}{0.800pt}}
\put(365,186.84){\rule{2.891pt}{0.800pt}}
\multiput(365.00,187.34)(6.000,-1.000){2}{\rule{1.445pt}{0.800pt}}
\put(354.0,189.0){\rule[-0.400pt]{2.650pt}{0.800pt}}
\put(388,185.84){\rule{2.650pt}{0.800pt}}
\multiput(388.00,186.34)(5.500,-1.000){2}{\rule{1.325pt}{0.800pt}}
\put(377.0,188.0){\rule[-0.400pt]{2.650pt}{0.800pt}}
\put(410,184.84){\rule{2.891pt}{0.800pt}}
\multiput(410.00,185.34)(6.000,-1.000){2}{\rule{1.445pt}{0.800pt}}
\put(399.0,187.0){\rule[-0.400pt]{2.650pt}{0.800pt}}
\put(433,183.84){\rule{2.650pt}{0.800pt}}
\multiput(433.00,184.34)(5.500,-1.000){2}{\rule{1.325pt}{0.800pt}}
\put(422.0,186.0){\rule[-0.400pt]{2.650pt}{0.800pt}}
\put(455,182.84){\rule{2.891pt}{0.800pt}}
\multiput(455.00,183.34)(6.000,-1.000){2}{\rule{1.445pt}{0.800pt}}
\put(444.0,185.0){\rule[-0.400pt]{2.650pt}{0.800pt}}
\put(478,181.84){\rule{2.650pt}{0.800pt}}
\multiput(478.00,182.34)(5.500,-1.000){2}{\rule{1.325pt}{0.800pt}}
\put(467.0,184.0){\rule[-0.400pt]{2.650pt}{0.800pt}}
\put(501,180.84){\rule{2.650pt}{0.800pt}}
\multiput(501.00,181.34)(5.500,-1.000){2}{\rule{1.325pt}{0.800pt}}
\put(489.0,183.0){\rule[-0.400pt]{2.891pt}{0.800pt}}
\put(523,179.84){\rule{2.650pt}{0.800pt}}
\multiput(523.00,180.34)(5.500,-1.000){2}{\rule{1.325pt}{0.800pt}}
\put(512.0,182.0){\rule[-0.400pt]{2.650pt}{0.800pt}}
\put(546,178.84){\rule{2.650pt}{0.800pt}}
\multiput(546.00,179.34)(5.500,-1.000){2}{\rule{1.325pt}{0.800pt}}
\put(534.0,181.0){\rule[-0.400pt]{2.891pt}{0.800pt}}
\put(568,177.84){\rule{2.650pt}{0.800pt}}
\multiput(568.00,178.34)(5.500,-1.000){2}{\rule{1.325pt}{0.800pt}}
\put(557.0,180.0){\rule[-0.400pt]{2.650pt}{0.800pt}}
\put(591,176.84){\rule{2.650pt}{0.800pt}}
\multiput(591.00,177.34)(5.500,-1.000){2}{\rule{1.325pt}{0.800pt}}
\put(579.0,179.0){\rule[-0.400pt]{2.891pt}{0.800pt}}
\put(613,175.84){\rule{2.891pt}{0.800pt}}
\multiput(613.00,176.34)(6.000,-1.000){2}{\rule{1.445pt}{0.800pt}}
\put(602.0,178.0){\rule[-0.400pt]{2.650pt}{0.800pt}}
\put(636,174.84){\rule{2.650pt}{0.800pt}}
\multiput(636.00,175.34)(5.500,-1.000){2}{\rule{1.325pt}{0.800pt}}
\put(625.0,177.0){\rule[-0.400pt]{2.650pt}{0.800pt}}
\put(658,173.84){\rule{2.891pt}{0.800pt}}
\multiput(658.00,174.34)(6.000,-1.000){2}{\rule{1.445pt}{0.800pt}}
\put(647.0,176.0){\rule[-0.400pt]{2.650pt}{0.800pt}}
\put(681,172.84){\rule{2.650pt}{0.800pt}}
\multiput(681.00,173.34)(5.500,-1.000){2}{\rule{1.325pt}{0.800pt}}
\put(670.0,175.0){\rule[-0.400pt]{2.650pt}{0.800pt}}
\put(703,171.84){\rule{2.891pt}{0.800pt}}
\multiput(703.00,172.34)(6.000,-1.000){2}{\rule{1.445pt}{0.800pt}}
\put(692.0,174.0){\rule[-0.400pt]{2.650pt}{0.800pt}}
\put(726,170.84){\rule{2.650pt}{0.800pt}}
\multiput(726.00,171.34)(5.500,-1.000){2}{\rule{1.325pt}{0.800pt}}
\put(715.0,173.0){\rule[-0.400pt]{2.650pt}{0.800pt}}
\put(749,169.84){\rule{2.650pt}{0.800pt}}
\multiput(749.00,170.34)(5.500,-1.000){2}{\rule{1.325pt}{0.800pt}}
\put(737.0,172.0){\rule[-0.400pt]{2.891pt}{0.800pt}}
\put(771,168.84){\rule{2.650pt}{0.800pt}}
\multiput(771.00,169.34)(5.500,-1.000){2}{\rule{1.325pt}{0.800pt}}
\put(760.0,171.0){\rule[-0.400pt]{2.650pt}{0.800pt}}
\put(794,167.84){\rule{2.650pt}{0.800pt}}
\multiput(794.00,168.34)(5.500,-1.000){2}{\rule{1.325pt}{0.800pt}}
\put(782.0,170.0){\rule[-0.400pt]{2.891pt}{0.800pt}}
\put(816,166.84){\rule{2.891pt}{0.800pt}}
\multiput(816.00,167.34)(6.000,-1.000){2}{\rule{1.445pt}{0.800pt}}
\put(805.0,169.0){\rule[-0.400pt]{2.650pt}{0.800pt}}
\put(839,165.84){\rule{2.650pt}{0.800pt}}
\multiput(839.00,166.34)(5.500,-1.000){2}{\rule{1.325pt}{0.800pt}}
\put(828.0,168.0){\rule[-0.400pt]{2.650pt}{0.800pt}}
\put(861,164.84){\rule{2.891pt}{0.800pt}}
\multiput(861.00,165.34)(6.000,-1.000){2}{\rule{1.445pt}{0.800pt}}
\put(850.0,167.0){\rule[-0.400pt]{2.650pt}{0.800pt}}
\put(884,163.84){\rule{2.650pt}{0.800pt}}
\multiput(884.00,164.34)(5.500,-1.000){2}{\rule{1.325pt}{0.800pt}}
\put(873.0,166.0){\rule[-0.400pt]{2.650pt}{0.800pt}}
\put(906,162.84){\rule{2.891pt}{0.800pt}}
\multiput(906.00,163.34)(6.000,-1.000){2}{\rule{1.445pt}{0.800pt}}
\put(895.0,165.0){\rule[-0.400pt]{2.650pt}{0.800pt}}
\put(929,161.84){\rule{2.650pt}{0.800pt}}
\multiput(929.00,162.34)(5.500,-1.000){2}{\rule{1.325pt}{0.800pt}}
\put(918.0,164.0){\rule[-0.400pt]{2.650pt}{0.800pt}}
\put(952,160.84){\rule{2.650pt}{0.800pt}}
\multiput(952.00,161.34)(5.500,-1.000){2}{\rule{1.325pt}{0.800pt}}
\put(940.0,163.0){\rule[-0.400pt]{2.891pt}{0.800pt}}
\put(974,159.84){\rule{2.650pt}{0.800pt}}
\multiput(974.00,160.34)(5.500,-1.000){2}{\rule{1.325pt}{0.800pt}}
\put(963.0,162.0){\rule[-0.400pt]{2.650pt}{0.800pt}}
\put(997,158.84){\rule{2.650pt}{0.800pt}}
\multiput(997.00,159.34)(5.500,-1.000){2}{\rule{1.325pt}{0.800pt}}
\put(985.0,161.0){\rule[-0.400pt]{2.891pt}{0.800pt}}
\put(1019,157.84){\rule{2.650pt}{0.800pt}}
\multiput(1019.00,158.34)(5.500,-1.000){2}{\rule{1.325pt}{0.800pt}}
\put(1008.0,160.0){\rule[-0.400pt]{2.650pt}{0.800pt}}
\put(1042,156.84){\rule{2.650pt}{0.800pt}}
\multiput(1042.00,157.34)(5.500,-1.000){2}{\rule{1.325pt}{0.800pt}}
\put(1030.0,159.0){\rule[-0.400pt]{2.891pt}{0.800pt}}
\put(1064,155.84){\rule{2.891pt}{0.800pt}}
\multiput(1064.00,156.34)(6.000,-1.000){2}{\rule{1.445pt}{0.800pt}}
\put(1053.0,158.0){\rule[-0.400pt]{2.650pt}{0.800pt}}
\put(1087,154.84){\rule{2.650pt}{0.800pt}}
\multiput(1087.00,155.34)(5.500,-1.000){2}{\rule{1.325pt}{0.800pt}}
\put(1076.0,157.0){\rule[-0.400pt]{2.650pt}{0.800pt}}
\put(1109,153.84){\rule{2.891pt}{0.800pt}}
\multiput(1109.00,154.34)(6.000,-1.000){2}{\rule{1.445pt}{0.800pt}}
\put(1098.0,156.0){\rule[-0.400pt]{2.650pt}{0.800pt}}
\put(1132,152.84){\rule{2.650pt}{0.800pt}}
\multiput(1132.00,153.34)(5.500,-1.000){2}{\rule{1.325pt}{0.800pt}}
\put(1121.0,155.0){\rule[-0.400pt]{2.650pt}{0.800pt}}
\put(1155,151.84){\rule{2.650pt}{0.800pt}}
\multiput(1155.00,152.34)(5.500,-1.000){2}{\rule{1.325pt}{0.800pt}}
\put(1143.0,154.0){\rule[-0.400pt]{2.891pt}{0.800pt}}
\put(1177,150.84){\rule{2.650pt}{0.800pt}}
\multiput(1177.00,151.34)(5.500,-1.000){2}{\rule{1.325pt}{0.800pt}}
\put(1166.0,153.0){\rule[-0.400pt]{2.650pt}{0.800pt}}
\put(1200,149.84){\rule{2.650pt}{0.800pt}}
\multiput(1200.00,150.34)(5.500,-1.000){2}{\rule{1.325pt}{0.800pt}}
\put(1188.0,152.0){\rule[-0.400pt]{2.891pt}{0.800pt}}
\put(1222,148.84){\rule{2.650pt}{0.800pt}}
\multiput(1222.00,149.34)(5.500,-1.000){2}{\rule{1.325pt}{0.800pt}}
\put(1211.0,151.0){\rule[-0.400pt]{2.650pt}{0.800pt}}
\put(1245,147.84){\rule{2.650pt}{0.800pt}}
\multiput(1245.00,148.34)(5.500,-1.000){2}{\rule{1.325pt}{0.800pt}}
\put(1233.0,150.0){\rule[-0.400pt]{2.891pt}{0.800pt}}
\put(1256.0,149.0){\rule[-0.400pt]{2.650pt}{0.800pt}}
\sbox{\plotpoint}{\rule[-0.200pt]{0.400pt}{0.400pt}}%
\put(151.0,131.0){\rule[-0.200pt]{0.400pt}{175.375pt}}
\put(151.0,131.0){\rule[-0.200pt]{310.279pt}{0.400pt}}
\put(1439.0,131.0){\rule[-0.200pt]{0.400pt}{175.375pt}}
\put(151.0,859.0){\rule[-0.200pt]{310.279pt}{0.400pt}}
\end{picture}

\caption{Priebeh teploty topenie ľadu}  \label{G_6}
\end{figure}

\begin{figure}
% GNUPLOT: LaTeX picture
\setlength{\unitlength}{0.240900pt}
\ifx\plotpoint\undefined\newsavebox{\plotpoint}\fi
\begin{picture}(1500,900)(0,0)
\sbox{\plotpoint}{\rule[-0.200pt]{0.400pt}{0.400pt}}%
\put(171.0,131.0){\rule[-0.200pt]{4.818pt}{0.400pt}}
\put(151,131){\makebox(0,0)[r]{ 0}}
\put(1419.0,131.0){\rule[-0.200pt]{4.818pt}{0.400pt}}
\put(171.0,204.0){\rule[-0.200pt]{4.818pt}{0.400pt}}
\put(151,204){\makebox(0,0)[r]{ 10}}
\put(1419.0,204.0){\rule[-0.200pt]{4.818pt}{0.400pt}}
\put(171.0,277.0){\rule[-0.200pt]{4.818pt}{0.400pt}}
\put(151,277){\makebox(0,0)[r]{ 20}}
\put(1419.0,277.0){\rule[-0.200pt]{4.818pt}{0.400pt}}
\put(171.0,349.0){\rule[-0.200pt]{4.818pt}{0.400pt}}
\put(151,349){\makebox(0,0)[r]{ 30}}
\put(1419.0,349.0){\rule[-0.200pt]{4.818pt}{0.400pt}}
\put(171.0,422.0){\rule[-0.200pt]{4.818pt}{0.400pt}}
\put(151,422){\makebox(0,0)[r]{ 40}}
\put(1419.0,422.0){\rule[-0.200pt]{4.818pt}{0.400pt}}
\put(171.0,495.0){\rule[-0.200pt]{4.818pt}{0.400pt}}
\put(151,495){\makebox(0,0)[r]{ 50}}
\put(1419.0,495.0){\rule[-0.200pt]{4.818pt}{0.400pt}}
\put(171.0,568.0){\rule[-0.200pt]{4.818pt}{0.400pt}}
\put(151,568){\makebox(0,0)[r]{ 60}}
\put(1419.0,568.0){\rule[-0.200pt]{4.818pt}{0.400pt}}
\put(171.0,641.0){\rule[-0.200pt]{4.818pt}{0.400pt}}
\put(151,641){\makebox(0,0)[r]{ 70}}
\put(1419.0,641.0){\rule[-0.200pt]{4.818pt}{0.400pt}}
\put(171.0,713.0){\rule[-0.200pt]{4.818pt}{0.400pt}}
\put(151,713){\makebox(0,0)[r]{ 80}}
\put(1419.0,713.0){\rule[-0.200pt]{4.818pt}{0.400pt}}
\put(171.0,786.0){\rule[-0.200pt]{4.818pt}{0.400pt}}
\put(151,786){\makebox(0,0)[r]{ 90}}
\put(1419.0,786.0){\rule[-0.200pt]{4.818pt}{0.400pt}}
\put(171.0,859.0){\rule[-0.200pt]{4.818pt}{0.400pt}}
\put(151,859){\makebox(0,0)[r]{ 100}}
\put(1419.0,859.0){\rule[-0.200pt]{4.818pt}{0.400pt}}
\put(171.0,131.0){\rule[-0.200pt]{0.400pt}{4.818pt}}
\put(171,90){\makebox(0,0){ 0}}
\put(171.0,839.0){\rule[-0.200pt]{0.400pt}{4.818pt}}
\put(425.0,131.0){\rule[-0.200pt]{0.400pt}{4.818pt}}
\put(425,90){\makebox(0,0){ 20}}
\put(425.0,839.0){\rule[-0.200pt]{0.400pt}{4.818pt}}
\put(678.0,131.0){\rule[-0.200pt]{0.400pt}{4.818pt}}
\put(678,90){\makebox(0,0){ 40}}
\put(678.0,839.0){\rule[-0.200pt]{0.400pt}{4.818pt}}
\put(932.0,131.0){\rule[-0.200pt]{0.400pt}{4.818pt}}
\put(932,90){\makebox(0,0){ 60}}
\put(932.0,839.0){\rule[-0.200pt]{0.400pt}{4.818pt}}
\put(1185.0,131.0){\rule[-0.200pt]{0.400pt}{4.818pt}}
\put(1185,90){\makebox(0,0){ 80}}
\put(1185.0,839.0){\rule[-0.200pt]{0.400pt}{4.818pt}}
\put(1439.0,131.0){\rule[-0.200pt]{0.400pt}{4.818pt}}
\put(1439,90){\makebox(0,0){ 100}}
\put(1439.0,839.0){\rule[-0.200pt]{0.400pt}{4.818pt}}
\put(171.0,131.0){\rule[-0.200pt]{0.400pt}{175.375pt}}
\put(171.0,131.0){\rule[-0.200pt]{305.461pt}{0.400pt}}
\put(1439.0,131.0){\rule[-0.200pt]{0.400pt}{175.375pt}}
\put(171.0,859.0){\rule[-0.200pt]{305.461pt}{0.400pt}}
\put(30,495){\makebox(0,0){\popi{t\_{nameraná}}{\C}}}
\put(805,29){\makebox(0,0){\popi{t\_{skutočná}}{\C}}}
\put(1279,213){\makebox(0,0)[r]{Namerná body}}
\put(171,135){\makebox(0,0){$+$}}
\put(1439,858){\makebox(0,0){$+$}}
\put(1349,213){\makebox(0,0){$+$}}
\put(1279,172){\makebox(0,0)[r]{kalibračná krivka}}
\multiput(1299,172)(20.756,0.000){5}{\usebox{\plotpoint}}
\put(1399,172){\usebox{\plotpoint}}
\put(171,135){\usebox{\plotpoint}}
\put(171.00,135.00){\usebox{\plotpoint}}
\put(188.83,145.60){\usebox{\plotpoint}}
\put(206.92,155.79){\usebox{\plotpoint}}
\put(224.71,166.46){\usebox{\plotpoint}}
\put(242.99,176.30){\usebox{\plotpoint}}
\put(261.25,186.16){\usebox{\plotpoint}}
\put(278.84,197.14){\usebox{\plotpoint}}
\put(297.11,206.98){\usebox{\plotpoint}}
\put(314.95,217.59){\usebox{\plotpoint}}
\put(333.22,227.43){\usebox{\plotpoint}}
\put(350.80,238.43){\usebox{\plotpoint}}
\put(369.07,248.27){\usebox{\plotpoint}}
\put(386.98,258.75){\usebox{\plotpoint}}
\put(405.12,268.82){\usebox{\plotpoint}}
\put(423.23,278.97){\usebox{\plotpoint}}
\put(441.06,289.57){\usebox{\plotpoint}}
\put(459.33,299.41){\usebox{\plotpoint}}
\put(476.97,310.31){\usebox{\plotpoint}}
\put(495.19,320.25){\usebox{\plotpoint}}
\put(513.15,330.63){\usebox{\plotpoint}}
\put(531.27,340.74){\usebox{\plotpoint}}
\put(549.34,350.95){\usebox{\plotpoint}}
\put(567.20,361.51){\usebox{\plotpoint}}
\put(585.45,371.39){\usebox{\plotpoint}}
\put(603.19,382.12){\usebox{\plotpoint}}
\put(621.30,392.24){\usebox{\plotpoint}}
\put(639.32,402.51){\usebox{\plotpoint}}
\put(657.41,412.68){\usebox{\plotpoint}}
\put(675.59,422.68){\usebox{\plotpoint}}
\put(693.37,433.38){\usebox{\plotpoint}}
\put(711.56,443.38){\usebox{\plotpoint}}
\put(729.58,453.66){\usebox{\plotpoint}}
\put(747.44,464.24){\usebox{\plotpoint}}
\put(765.52,474.40){\usebox{\plotpoint}}
\put(783.55,484.68){\usebox{\plotpoint}}
\put(801.77,494.61){\usebox{\plotpoint}}
\put(819.57,505.28){\usebox{\plotpoint}}
\put(837.70,515.38){\usebox{\plotpoint}}
\put(855.78,525.56){\usebox{\plotpoint}}
\put(873.60,536.19){\usebox{\plotpoint}}
\put(891.72,546.29){\usebox{\plotpoint}}
\put(909.69,556.68){\usebox{\plotpoint}}
\put(927.94,566.55){\usebox{\plotpoint}}
\put(945.77,577.17){\usebox{\plotpoint}}
\put(963.84,587.37){\usebox{\plotpoint}}
\put(981.98,597.45){\usebox{\plotpoint}}
\put(999.95,607.82){\usebox{\plotpoint}}
\put(1017.92,618.18){\usebox{\plotpoint}}
\put(1035.82,628.67){\usebox{\plotpoint}}
\put(1054.10,638.51){\usebox{\plotpoint}}
\put(1071.86,649.25){\usebox{\plotpoint}}
\put(1089.98,659.37){\usebox{\plotpoint}}
\put(1108.18,669.34){\usebox{\plotpoint}}
\put(1126.09,679.82){\usebox{\plotpoint}}
\put(1144.12,690.08){\usebox{\plotpoint}}
\put(1161.96,700.67){\usebox{\plotpoint}}
\put(1180.24,710.51){\usebox{\plotpoint}}
\put(1198.03,721.19){\usebox{\plotpoint}}
\put(1216.12,731.37){\usebox{\plotpoint}}
\put(1234.38,741.23){\usebox{\plotpoint}}
\put(1252.23,751.81){\usebox{\plotpoint}}
\put(1270.30,762.01){\usebox{\plotpoint}}
\put(1288.10,772.67){\usebox{\plotpoint}}
\put(1306.38,782.51){\usebox{\plotpoint}}
\put(1324.21,793.11){\usebox{\plotpoint}}
\put(1342.38,803.14){\usebox{\plotpoint}}
\put(1360.53,813.21){\usebox{\plotpoint}}
\put(1378.36,823.81){\usebox{\plotpoint}}
\put(1396.64,833.65){\usebox{\plotpoint}}
\put(1414.22,844.65){\usebox{\plotpoint}}
\put(1432.49,854.49){\usebox{\plotpoint}}
\put(1439,858){\usebox{\plotpoint}}
\put(171.0,131.0){\rule[-0.200pt]{0.400pt}{175.375pt}}
\put(171.0,131.0){\rule[-0.200pt]{305.461pt}{0.400pt}}
\put(1439.0,131.0){\rule[-0.200pt]{0.400pt}{175.375pt}}
\put(171.0,859.0){\rule[-0.200pt]{305.461pt}{0.400pt}}
\end{picture}

\caption{Priebeh teploty topenie ľadu}  \label{G_6}
\end{figure}


\end{document}






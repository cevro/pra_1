\documentclass[a4paper,10pt]{article}
%\usepackage[IL2]{fontenc}
\usepackage[utf8x]{inputenc}
\usepackage[czech]{babel}
\usepackage{amsfonts,amsmath,amssymb,graphicx,color}
%\usepackage[total={17cm,27cm}, top=2cm, left=2cm, includefoot]{geometry}
%\usepackage{fancyhdr}
\usepackage{fkssugar}
\usepackage{hyperref}

%\usepackage{caption}
\renewcommand{\popi}[2]{$#1$[\jd{#2}]}
\renewcommand{\figurename}{Obr.}
\addto\captionsczech{\renewcommand{\figurename}{Obr.}}
\addto\captionsczech{\renewcommand{\tablename}{Tab.}}

\begin{document}
\def\mean#1{\left< #1 \right>}
\noindent
{\large Fyzikální praktikum 1.} \hfil {\large FJFI ČVUT V Praze}\\
\noindent
{\large\textbf{pracovní úkol \# 9}}
\begin{center}
{\large\textit{Rozšířená rozsahu miliampármetru a voltmetru,cejchováná kompenzátorem}}
\end{center}
\noindent
\rule{\textwidth}{1px}
\vspace{\baselineskip}

\emph{Michal Červeňák}
\par
\vspace{\baselineskip}
\begin{minipage}[l]{0.5\textwidth}%
\textit{dátum merania:}~10.10. 2016\\%
%\vspace{\baselineskip}%
\par%
\noindent%
\textit{skupina:}~4\\%
%\vspace{\baselineskip}%
\par%
\noindent%
\textit{Klasifikace:}\dotfill\\%
\end{minipage}

\section{Pracovní úkol}

\begin{enumerate}
\item DU: V přípravě odvoď ťe vztah (7)\footnote{Wrong reference by definition.}.
\item Pomocí kompenzátoru ocejchujte stupnici voltmetru (cejchujte v celém rozsahu stupnice). Pro 10 naměřených hodnot sestrojte kalibrační křivku a vyneste ji do grafu.
\item Pomocí kompenzátoru ocejchujte stupnici miliampérmetru (cejchujte v celém rozsahu stupnice). Pro 10 naměřených hodnot sestrojte kalibrační křivku a vyneste ji do grafu.
\item Pomocí kompenzátoru ocejchujte odporovou dekádu. Měření provedťe pro 10 hodnot v rozsahu $"100-1000 \Omega"$. Pro 10 naměřených hodnot sestrojte kalibrační křivku a vyneste ji do grafu.
\item Rozšiřte rozsah miliampérmetru dvakrát a určete jeho vnitřní odpor $R_0$. Měření provedťe pro
10 různých nastavení obvodu, t.j. pro 10 různých proudů.
\item Rozšiřte rozsah voltmetru dvakrát a určete jeho vnitřní odpor. Měření provedťe pro 10 různých
nastavení obvodu, t.j. pro 10 různých napětí.
\item Při zpracování výsledků z měření vnitřních odporů vezměte v úvahu výsledky získané cejchováním
stupnic voltmetru a miliampérmetru a provedťe korekci naměřených hodnot. Diskutujte
rozdíl mezi výsledkem získaným bez korekce a s korekcí.
\end{enumerate}

\section{Postup merania}
\begin{enumerate}
\item Pomocou Westnového normálneho članku sa skalibruje kompenzátor.
\item Podľa schémy Obr. (6) \cite{C_1} sa zostavý obvod
\item Pomocou reostatu sa reguluje napätie aby sa nameral celý rozsah ciachovaného voltmetru.
\item Podľa schémy Obr. (7) \cite{C_1} sa zostaví obvod
\item Pomocou reostatu sa reguluje prúd aby sa nameral celý rozsah ciachovaného ampérmetru.
\item Podľa schémy Obr. (8) \cite{C_1} sa zostaví obvod
\item Na odporovej dekáde sa reguluje mení odpor v rozsahu $R = "100-1000 \Omega"$ a pre každú hodnotu sa odmeria hodnota napätia na normálovom rezistore a odporovej dekáde.
\item Zostaví sa obvod podľa schémy Obr. (5) \cite{C_1}.
\item Pomocou odporovej dekády sa mení rozsah tak aby zväčšil rozsah voltmetru presne 2 krát.
\item Zostaví sa obvod podľa schémy Obr. (4) \cite{C_1}.
\item Pomocou odporovej dekády sa mení rozsah tak aby zväčšil rozsah ampérmetru presne 2 krát.
\end{enumerate}

\section{Pomôcky}
Miliampérmetr, voltmetr, zdroj $"0-20 V"$, batéria $"1.5 V"$, odporová dekáda, reostaty $"115 \Omega"$ a $"6000 \Omega"$, 
dva vypínače, multimetr, odporové normály $"100 \Omega"$ a $"1000 \Omega"$ a $"10\,000 \Omega"$ , technický kompenzátor, 
Westonův normálná článek, vodiče

\section{Teória}
Závislosť napätia Westnového normálneho článku na teplote udáva vzťah
\eq{
U_t = U_{20}-"4.06" \cdot 10^{−5}\(t − 20\)-"0.95" \cdot 10^{−6}\(t − 20\)^2+1 \cdot 10^{−8}\(t − 20\)^3" V"\,,
}
pričom $t$ je teplota v $\jd{\C}$ a napätie pri teplote $t = "20 \C"$ je $U_{20} = "1.01865 \C"$.

Pre prepočet napätia $U$ na rezistore s odporom $R$ na prúd $I$ prechádzajúci rezistorom môžeme použiť vzťah
\eq{
I = \frac{U}{R} \,. \lbl{R_2}
}


Pre výpočet odporu rezistrou porovnávaním pomerov napätí na dvoch rezistoroch platí vzťah
\eq{
R_k = \frac{U_x}{U_n} R_n\,, \lbl{R_3}
}
kde $R_k$ je neznámy odpor s úbytkom napätia $U_x$ a $R_n$ rezistor so známym odporom a úbytkom napätia na ňom $U_n$.


Pre zväčšovanie rozsahu voltmetru platí vzťah
\eq{
R_V = \frac{R_p}{n-1} \,, \lbl{R_6}
}
kde $R_v$ je vnútorný odpor voltmetru, a $R_p$ je odpor predradeného odporu, a $n$ udáva koľkokrát je rozsah zväčšení v našom pripade $n = U_1/U^\prime$.

Pre zväčšenie rozsahu ampérmetru platí vzťah
\eq{
R_0 = \(n-1\)R_b\,, \lbl{R_5}
}
kde $n$ je koľkokrát sa zväčši rozsah ampérmetru, $R_b$ je odpor bočníku\footnote{Odporu zaradeného paralelne k ampérmetru}, a $R_0$ je vnútorný odpor ampérmetru. 




\subsubsection{Spracovanie chýb merania}

Označme $\mean{t}$ aritmetický priemer nameraných hodnôt $t_i$, a $\Delta t$ hodnotu $\mean{t}-t$, pričom 
\eq{
\mean{t} = \frac{1}{n}\sum_{i=1}^n t_i \,, \lbl{SCH_1}
}  
a chybu aritmetického priemeru 
\eq{
  \sigma_0=\sqrt{\frac{\sum_{i=1}^n \(t_i - \mean{t}\)^2}{n\(n-1\)}}\,, \lbl{SCH_2}
}
pričom $n$ je počet meraní.


\section{Výsledky merania}


\subsection{Ciachovanie voltmetru}

Odčítané hodnoty na stupnici ciachovaného voltmetru označme $U_x$ a hodnoty odčítané z kompenzátoru
ako $U_k$. Namerané hodnoty boli vynesené do grafu Obr. \ref{G_1} a Tab. \ref{T_1}.

\begin{table}[h]

\begin{center}
\begin{tabular}{| c | c |}
\hline
 \popi{U_x}{V} & \popi{U_k}{V}  \\
\hline
$9  \pm0.1$ &$9.549\pm0.001$ \\
$8.6\pm0.1$ &$9.236\pm0.001$ \\
$8.0\pm0.1$ &$8.613\pm0.001$ \\
$7.0\pm0.1$ &$7.623\pm0.001$ \\
$6.6\pm0.1$ &$7.133\pm0.001$ \\
$6.0\pm0.1$ &$6.594\pm0.001$ \\
$5.4\pm0.1$ &$5.952\pm0.001$ \\
$5.0\pm0.1$ &$5.494\pm0.001$ \\
$3  \pm0.1$&  $3.292\pm0.001$ \\
$1.5\pm0.1$ &$1.746\pm0.001$ \\
\hline

\end{tabular}
\caption{Namerané hodnoty ciachovaného voltmetru $U_x$ v závislosti od hodnôt odčítaných z kompenzátoru 
$U_k$} \label{T_1}
\end{center}
\end{table}



\begin{figure}
% GNUPLOT: LaTeX picture
\setlength{\unitlength}{0.240900pt}
\ifx\plotpoint\undefined\newsavebox{\plotpoint}\fi
\begin{picture}(1500,900)(0,0)
\sbox{\plotpoint}{\rule[-0.200pt]{0.400pt}{0.400pt}}%
\put(151.0,131.0){\rule[-0.200pt]{4.818pt}{0.400pt}}
\put(131,131){\makebox(0,0)[r]{ 1}}
\put(1419.0,131.0){\rule[-0.200pt]{4.818pt}{0.400pt}}
\put(151.0,212.0){\rule[-0.200pt]{4.818pt}{0.400pt}}
\put(131,212){\makebox(0,0)[r]{ 2}}
\put(1419.0,212.0){\rule[-0.200pt]{4.818pt}{0.400pt}}
\put(151.0,293.0){\rule[-0.200pt]{4.818pt}{0.400pt}}
\put(131,293){\makebox(0,0)[r]{ 3}}
\put(1419.0,293.0){\rule[-0.200pt]{4.818pt}{0.400pt}}
\put(151.0,374.0){\rule[-0.200pt]{4.818pt}{0.400pt}}
\put(131,374){\makebox(0,0)[r]{ 4}}
\put(1419.0,374.0){\rule[-0.200pt]{4.818pt}{0.400pt}}
\put(151.0,455.0){\rule[-0.200pt]{4.818pt}{0.400pt}}
\put(131,455){\makebox(0,0)[r]{ 5}}
\put(1419.0,455.0){\rule[-0.200pt]{4.818pt}{0.400pt}}
\put(151.0,535.0){\rule[-0.200pt]{4.818pt}{0.400pt}}
\put(131,535){\makebox(0,0)[r]{ 6}}
\put(1419.0,535.0){\rule[-0.200pt]{4.818pt}{0.400pt}}
\put(151.0,616.0){\rule[-0.200pt]{4.818pt}{0.400pt}}
\put(131,616){\makebox(0,0)[r]{ 7}}
\put(1419.0,616.0){\rule[-0.200pt]{4.818pt}{0.400pt}}
\put(151.0,697.0){\rule[-0.200pt]{4.818pt}{0.400pt}}
\put(131,697){\makebox(0,0)[r]{ 8}}
\put(1419.0,697.0){\rule[-0.200pt]{4.818pt}{0.400pt}}
\put(151.0,778.0){\rule[-0.200pt]{4.818pt}{0.400pt}}
\put(131,778){\makebox(0,0)[r]{ 9}}
\put(1419.0,778.0){\rule[-0.200pt]{4.818pt}{0.400pt}}
\put(151.0,859.0){\rule[-0.200pt]{4.818pt}{0.400pt}}
\put(131,859){\makebox(0,0)[r]{ 10}}
\put(1419.0,859.0){\rule[-0.200pt]{4.818pt}{0.400pt}}
\put(151.0,131.0){\rule[-0.200pt]{0.400pt}{4.818pt}}
\put(151,90){\makebox(0,0){ 1}}
\put(151.0,839.0){\rule[-0.200pt]{0.400pt}{4.818pt}}
\put(294.0,131.0){\rule[-0.200pt]{0.400pt}{4.818pt}}
\put(294,90){\makebox(0,0){ 2}}
\put(294.0,839.0){\rule[-0.200pt]{0.400pt}{4.818pt}}
\put(437.0,131.0){\rule[-0.200pt]{0.400pt}{4.818pt}}
\put(437,90){\makebox(0,0){ 3}}
\put(437.0,839.0){\rule[-0.200pt]{0.400pt}{4.818pt}}
\put(580.0,131.0){\rule[-0.200pt]{0.400pt}{4.818pt}}
\put(580,90){\makebox(0,0){ 4}}
\put(580.0,839.0){\rule[-0.200pt]{0.400pt}{4.818pt}}
\put(723.0,131.0){\rule[-0.200pt]{0.400pt}{4.818pt}}
\put(723,90){\makebox(0,0){ 5}}
\put(723.0,839.0){\rule[-0.200pt]{0.400pt}{4.818pt}}
\put(867.0,131.0){\rule[-0.200pt]{0.400pt}{4.818pt}}
\put(867,90){\makebox(0,0){ 6}}
\put(867.0,839.0){\rule[-0.200pt]{0.400pt}{4.818pt}}
\put(1010.0,131.0){\rule[-0.200pt]{0.400pt}{4.818pt}}
\put(1010,90){\makebox(0,0){ 7}}
\put(1010.0,839.0){\rule[-0.200pt]{0.400pt}{4.818pt}}
\put(1153.0,131.0){\rule[-0.200pt]{0.400pt}{4.818pt}}
\put(1153,90){\makebox(0,0){ 8}}
\put(1153.0,839.0){\rule[-0.200pt]{0.400pt}{4.818pt}}
\put(1296.0,131.0){\rule[-0.200pt]{0.400pt}{4.818pt}}
\put(1296,90){\makebox(0,0){ 9}}
\put(1296.0,839.0){\rule[-0.200pt]{0.400pt}{4.818pt}}
\put(1439.0,131.0){\rule[-0.200pt]{0.400pt}{4.818pt}}
\put(1439,90){\makebox(0,0){ 10}}
\put(1439.0,839.0){\rule[-0.200pt]{0.400pt}{4.818pt}}
\put(151.0,131.0){\rule[-0.200pt]{0.400pt}{175.375pt}}
\put(151.0,131.0){\rule[-0.200pt]{310.279pt}{0.400pt}}
\put(1439.0,131.0){\rule[-0.200pt]{0.400pt}{175.375pt}}
\put(151.0,859.0){\rule[-0.200pt]{310.279pt}{0.400pt}}
\put(30,495){\makebox(0,0){\popi{U_k}{V}}}
\put(795,29){\makebox(0,0){\popi{U_x}{V}}}
\put(1171,819){\makebox(0,0)[r]{Namerané hodnoty}}
\put(1191.0,819.0){\rule[-0.200pt]{24.090pt}{0.400pt}}
\put(1191.0,809.0){\rule[-0.200pt]{0.400pt}{4.818pt}}
\put(1291.0,809.0){\rule[-0.200pt]{0.400pt}{4.818pt}}
\put(1282.0,823.0){\rule[-0.200pt]{6.745pt}{0.400pt}}
\put(1282.0,813.0){\rule[-0.200pt]{0.400pt}{4.818pt}}
\put(1310.0,813.0){\rule[-0.200pt]{0.400pt}{4.818pt}}
\put(1224.0,797.0){\rule[-0.200pt]{6.986pt}{0.400pt}}
\put(1224.0,787.0){\rule[-0.200pt]{0.400pt}{4.818pt}}
\put(1253.0,787.0){\rule[-0.200pt]{0.400pt}{4.818pt}}
\put(1138.0,747.0){\rule[-0.200pt]{6.986pt}{0.400pt}}
\put(1138.0,737.0){\rule[-0.200pt]{0.400pt}{4.818pt}}
\put(1167.0,737.0){\rule[-0.200pt]{0.400pt}{4.818pt}}
\put(995.0,667.0){\rule[-0.200pt]{6.986pt}{0.400pt}}
\put(995.0,657.0){\rule[-0.200pt]{0.400pt}{4.818pt}}
\put(1024.0,657.0){\rule[-0.200pt]{0.400pt}{4.818pt}}
\put(938.0,627.0){\rule[-0.200pt]{6.986pt}{0.400pt}}
\put(938.0,617.0){\rule[-0.200pt]{0.400pt}{4.818pt}}
\put(967.0,617.0){\rule[-0.200pt]{0.400pt}{4.818pt}}
\put(852.0,583.0){\rule[-0.200pt]{6.986pt}{0.400pt}}
\put(852.0,573.0){\rule[-0.200pt]{0.400pt}{4.818pt}}
\put(881.0,573.0){\rule[-0.200pt]{0.400pt}{4.818pt}}
\put(766.0,532.0){\rule[-0.200pt]{6.986pt}{0.400pt}}
\put(766.0,522.0){\rule[-0.200pt]{0.400pt}{4.818pt}}
\put(795.0,522.0){\rule[-0.200pt]{0.400pt}{4.818pt}}
\put(709.0,495.0){\rule[-0.200pt]{6.986pt}{0.400pt}}
\put(709.0,485.0){\rule[-0.200pt]{0.400pt}{4.818pt}}
\put(738.0,485.0){\rule[-0.200pt]{0.400pt}{4.818pt}}
\put(423.0,316.0){\rule[-0.200pt]{6.986pt}{0.400pt}}
\put(423.0,306.0){\rule[-0.200pt]{0.400pt}{4.818pt}}
\put(452.0,306.0){\rule[-0.200pt]{0.400pt}{4.818pt}}
\put(208.0,191.0){\rule[-0.200pt]{6.986pt}{0.400pt}}
\put(208.0,181.0){\rule[-0.200pt]{0.400pt}{4.818pt}}
\put(1296,823){\makebox(0,0){$+$}}
\put(1239,797){\makebox(0,0){$+$}}
\put(1153,747){\makebox(0,0){$+$}}
\put(1010,667){\makebox(0,0){$+$}}
\put(952,627){\makebox(0,0){$+$}}
\put(867,583){\makebox(0,0){$+$}}
\put(781,532){\makebox(0,0){$+$}}
\put(723,495){\makebox(0,0){$+$}}
\put(437,316){\makebox(0,0){$+$}}
\put(223,191){\makebox(0,0){$+$}}
\put(1241,819){\makebox(0,0){$+$}}
\put(237.0,181.0){\rule[-0.200pt]{0.400pt}{4.818pt}}
\put(1171,778){\makebox(0,0)[r]{Preložená závyslosť $U_k = "1.05"\cdot U_x + "0.21 V"$}}
\multiput(1191,778)(20.756,0.000){5}{\usebox{\plotpoint}}
\put(1291,778){\usebox{\plotpoint}}
\put(208,186){\usebox{\plotpoint}}
\put(208.00,186.00){\usebox{\plotpoint}}
\put(225.90,196.45){\usebox{\plotpoint}}
\put(243.77,206.97){\usebox{\plotpoint}}
\put(261.64,217.50){\usebox{\plotpoint}}
\put(279.32,228.36){\usebox{\plotpoint}}
\put(297.09,239.06){\usebox{\plotpoint}}
\put(315.00,249.50){\usebox{\plotpoint}}
\put(332.87,260.02){\usebox{\plotpoint}}
\put(350.73,270.56){\usebox{\plotpoint}}
\put(368.42,281.41){\usebox{\plotpoint}}
\put(386.19,292.10){\usebox{\plotpoint}}
\put(404.29,302.25){\usebox{\plotpoint}}
\put(421.99,313.09){\usebox{\plotpoint}}
\put(439.85,323.63){\usebox{\plotpoint}}
\put(457.79,334.05){\usebox{\plotpoint}}
\put(475.32,345.16){\usebox{\plotpoint}}
\put(493.50,355.13){\usebox{\plotpoint}}
\put(511.43,365.55){\usebox{\plotpoint}}
\put(528.95,376.69){\usebox{\plotpoint}}
\put(546.89,387.11){\usebox{\plotpoint}}
\put(564.42,398.21){\usebox{\plotpoint}}
\put(582.59,408.19){\usebox{\plotpoint}}
\put(600.53,418.61){\usebox{\plotpoint}}
\put(618.04,429.75){\usebox{\plotpoint}}
\put(635.98,440.17){\usebox{\plotpoint}}
\put(653.94,450.55){\usebox{\plotpoint}}
\put(671.71,461.27){\usebox{\plotpoint}}
\put(689.65,471.68){\usebox{\plotpoint}}
\put(707.53,482.20){\usebox{\plotpoint}}
\put(725.10,493.24){\usebox{\plotpoint}}
\put(743.29,503.18){\usebox{\plotpoint}}
\put(761.07,513.86){\usebox{\plotpoint}}
\put(778.74,524.74){\usebox{\plotpoint}}
\put(796.62,535.25){\usebox{\plotpoint}}
\put(814.19,546.30){\usebox{\plotpoint}}
\put(832.38,556.24){\usebox{\plotpoint}}
\put(850.17,566.91){\usebox{\plotpoint}}
\put(867.83,577.80){\usebox{\plotpoint}}
\put(885.72,588.30){\usebox{\plotpoint}}
\put(903.49,599.00){\usebox{\plotpoint}}
\put(921.50,609.32){\usebox{\plotpoint}}
\put(939.29,619.98){\usebox{\plotpoint}}
\put(957.07,630.67){\usebox{\plotpoint}}
\put(974.89,641.29){\usebox{\plotpoint}}
\put(992.73,651.86){\usebox{\plotpoint}}
\put(1010.62,662.34){\usebox{\plotpoint}}
\put(1028.53,672.79){\usebox{\plotpoint}}
\put(1046.17,683.73){\usebox{\plotpoint}}
\put(1063.98,694.35){\usebox{\plotpoint}}
\put(1081.82,704.91){\usebox{\plotpoint}}
\put(1099.71,715.39){\usebox{\plotpoint}}
\put(1117.62,725.85){\usebox{\plotpoint}}
\put(1135.26,736.78){\usebox{\plotpoint}}
\put(1153.07,747.41){\usebox{\plotpoint}}
\put(1171.16,757.59){\usebox{\plotpoint}}
\put(1188.84,768.46){\usebox{\plotpoint}}
\put(1206.74,778.93){\usebox{\plotpoint}}
\put(1224.68,789.34){\usebox{\plotpoint}}
\put(1242.19,800.49){\usebox{\plotpoint}}
\put(1260.39,810.43){\usebox{\plotpoint}}
\put(1278.32,820.84){\usebox{\plotpoint}}
\put(1295.84,831.99){\usebox{\plotpoint}}
\put(1310,840){\usebox{\plotpoint}}
\put(151.0,131.0){\rule[-0.200pt]{0.400pt}{175.375pt}}
\put(151.0,131.0){\rule[-0.200pt]{310.279pt}{0.400pt}}
\put(1439.0,131.0){\rule[-0.200pt]{0.400pt}{175.375pt}}
\put(151.0,859.0){\rule[-0.200pt]{310.279pt}{0.400pt}}
\end{picture}

\caption{Namerané hodnoty ciachovaného voltmetru $U_v$ v závislosti od hodnôt odčítaných z kompenzátoru 
$U_k$, preložené závislosťou $U_k = "1.05" \cdot U_x + "0.21 V"$ }  \label{G_1}
\end{figure}

Z lineárneho fitu dostávame výsledný vzťah
\eq{
U_k = "1.05" \cdot U_x + "0.21 V" \,. \lbl{R_10}
}




\subsection{Ciachovanie ampérmetru}

Odčítané hodnoty na stupnici ciachovaného ampérmetru označme $I_x$ a hodnoty odčítané z kompenzátoru
ako $U_k$. Normálový rezistor bol použitý $R_b="1000 \Omega"$, ktoré boli pomocou \ref{R_2} prepočítané na $I_k$. 
Namerané hodnoty boli vynesené do grafu Obr. \ref{G_2} a Tab. \ref{T_2}.

\begin{table}[h]

\begin{center}
\begin{tabular}{| c | c |}
\hline
 \popi{I_x}{mA} & \popi{I_k}{mA}  \\
\hline
$0.96\pm0.1$ & $0.92\pm0.001$\\
$0.90\pm0.1$ & $0.87\pm0.001$\\
$0.84\pm0.1$ & $0.81\pm0.001$\\
$0.80\pm0.1$ & $0.78\pm0.001$\\
$0.7\pm0.1$  & $0.69\pm0.001$\\
$0.68\pm0.1$ & $0.67\pm0.001$\\
$0.62\pm0.1$ & $0.62\pm0.001$\\
$0.40\pm0.1$ & $0.39\pm0.001$\\
$0.30\pm0.1$& $0.30\pm0.001$\\
$0.2\pm0.1$& $0.20\pm0.001$\\
\hline

\end{tabular}
\caption{Namerané hodnoty ciachovaného ampérmetru $I_x$ v závislosti od hodnôt odčítaných z kompenzátoru
a prepočítaných na $\jd{mA}$ ako $I_k$} \label{T_2}
\end{center}
\end{table}



\begin{figure}
% GNUPLOT: LaTeX picture
\setlength{\unitlength}{0.240900pt}
\ifx\plotpoint\undefined\newsavebox{\plotpoint}\fi
\begin{picture}(1500,900)(0,0)
\sbox{\plotpoint}{\rule[-0.200pt]{0.400pt}{0.400pt}}%
\put(191.0,131.0){\rule[-0.200pt]{4.818pt}{0.400pt}}
\put(171,131){\makebox(0,0)[r]{ 0.05}}
\put(1419.0,131.0){\rule[-0.200pt]{4.818pt}{0.400pt}}
\put(191.0,212.0){\rule[-0.200pt]{4.818pt}{0.400pt}}
\put(171,212){\makebox(0,0)[r]{ 0.1}}
\put(1419.0,212.0){\rule[-0.200pt]{4.818pt}{0.400pt}}
\put(191.0,293.0){\rule[-0.200pt]{4.818pt}{0.400pt}}
\put(171,293){\makebox(0,0)[r]{ 0.15}}
\put(1419.0,293.0){\rule[-0.200pt]{4.818pt}{0.400pt}}
\put(191.0,374.0){\rule[-0.200pt]{4.818pt}{0.400pt}}
\put(171,374){\makebox(0,0)[r]{ 0.2}}
\put(1419.0,374.0){\rule[-0.200pt]{4.818pt}{0.400pt}}
\put(191.0,455.0){\rule[-0.200pt]{4.818pt}{0.400pt}}
\put(171,455){\makebox(0,0)[r]{ 0.25}}
\put(1419.0,455.0){\rule[-0.200pt]{4.818pt}{0.400pt}}
\put(191.0,535.0){\rule[-0.200pt]{4.818pt}{0.400pt}}
\put(171,535){\makebox(0,0)[r]{ 0.3}}
\put(1419.0,535.0){\rule[-0.200pt]{4.818pt}{0.400pt}}
\put(191.0,616.0){\rule[-0.200pt]{4.818pt}{0.400pt}}
\put(171,616){\makebox(0,0)[r]{ 0.35}}
\put(1419.0,616.0){\rule[-0.200pt]{4.818pt}{0.400pt}}
\put(191.0,697.0){\rule[-0.200pt]{4.818pt}{0.400pt}}
\put(171,697){\makebox(0,0)[r]{ 0.4}}
\put(1419.0,697.0){\rule[-0.200pt]{4.818pt}{0.400pt}}
\put(191.0,778.0){\rule[-0.200pt]{4.818pt}{0.400pt}}
\put(171,778){\makebox(0,0)[r]{ 0.45}}
\put(1419.0,778.0){\rule[-0.200pt]{4.818pt}{0.400pt}}
\put(191.0,859.0){\rule[-0.200pt]{4.818pt}{0.400pt}}
\put(171,859){\makebox(0,0)[r]{ 0.5}}
\put(1419.0,859.0){\rule[-0.200pt]{4.818pt}{0.400pt}}
\put(191.0,131.0){\rule[-0.200pt]{0.400pt}{4.818pt}}
\put(191,90){\makebox(0,0){ 0.05}}
\put(191.0,839.0){\rule[-0.200pt]{0.400pt}{4.818pt}}
\put(369.0,131.0){\rule[-0.200pt]{0.400pt}{4.818pt}}
\put(369,90){\makebox(0,0){ 0.1}}
\put(369.0,839.0){\rule[-0.200pt]{0.400pt}{4.818pt}}
\put(548.0,131.0){\rule[-0.200pt]{0.400pt}{4.818pt}}
\put(548,90){\makebox(0,0){ 0.15}}
\put(548.0,839.0){\rule[-0.200pt]{0.400pt}{4.818pt}}
\put(726.0,131.0){\rule[-0.200pt]{0.400pt}{4.818pt}}
\put(726,90){\makebox(0,0){ 0.2}}
\put(726.0,839.0){\rule[-0.200pt]{0.400pt}{4.818pt}}
\put(904.0,131.0){\rule[-0.200pt]{0.400pt}{4.818pt}}
\put(904,90){\makebox(0,0){ 0.25}}
\put(904.0,839.0){\rule[-0.200pt]{0.400pt}{4.818pt}}
\put(1082.0,131.0){\rule[-0.200pt]{0.400pt}{4.818pt}}
\put(1082,90){\makebox(0,0){ 0.3}}
\put(1082.0,839.0){\rule[-0.200pt]{0.400pt}{4.818pt}}
\put(1261.0,131.0){\rule[-0.200pt]{0.400pt}{4.818pt}}
\put(1261,90){\makebox(0,0){ 0.35}}
\put(1261.0,839.0){\rule[-0.200pt]{0.400pt}{4.818pt}}
\put(1439.0,131.0){\rule[-0.200pt]{0.400pt}{4.818pt}}
\put(1439,90){\makebox(0,0){ 0.4}}
\put(1439.0,839.0){\rule[-0.200pt]{0.400pt}{4.818pt}}
\put(191.0,131.0){\rule[-0.200pt]{0.400pt}{175.375pt}}
\put(191.0,131.0){\rule[-0.200pt]{300.643pt}{0.400pt}}
\put(1439.0,131.0){\rule[-0.200pt]{0.400pt}{175.375pt}}
\put(191.0,859.0){\rule[-0.200pt]{300.643pt}{0.400pt}}
\put(30,495){\makebox(0,0){\popi{I}{N\cdot s}}}
\put(815,29){\makebox(0,0){\popi{p}{N\cdot s}}}
\put(1111,819){\makebox(0,0)[r]{namerané dáta}}
\put(1131.0,819.0){\rule[-0.200pt]{24.090pt}{0.400pt}}
\put(1131.0,809.0){\rule[-0.200pt]{0.400pt}{4.818pt}}
\put(1231.0,809.0){\rule[-0.200pt]{0.400pt}{4.818pt}}
\put(979.0,669.0){\rule[-0.200pt]{86.001pt}{0.400pt}}
\put(979.0,659.0){\rule[-0.200pt]{0.400pt}{4.818pt}}
\put(1336.0,659.0){\rule[-0.200pt]{0.400pt}{4.818pt}}
\put(922.0,661.0){\rule[-0.200pt]{86.001pt}{0.400pt}}
\put(922.0,651.0){\rule[-0.200pt]{0.400pt}{4.818pt}}
\put(1279.0,651.0){\rule[-0.200pt]{0.400pt}{4.818pt}}
\put(819.0,669.0){\rule[-0.200pt]{85.760pt}{0.400pt}}
\put(819.0,659.0){\rule[-0.200pt]{0.400pt}{4.818pt}}
\put(1175.0,659.0){\rule[-0.200pt]{0.400pt}{4.818pt}}
\put(815.0,645.0){\rule[-0.200pt]{86.001pt}{0.400pt}}
\put(815.0,635.0){\rule[-0.200pt]{0.400pt}{4.818pt}}
\put(1172.0,635.0){\rule[-0.200pt]{0.400pt}{4.818pt}}
\put(876.0,678.0){\rule[-0.200pt]{85.760pt}{0.400pt}}
\put(876.0,668.0){\rule[-0.200pt]{0.400pt}{4.818pt}}
\put(1232.0,668.0){\rule[-0.200pt]{0.400pt}{4.818pt}}
\put(608.0,491.0){\rule[-0.200pt]{86.001pt}{0.400pt}}
\put(608.0,481.0){\rule[-0.200pt]{0.400pt}{4.818pt}}
\put(965.0,481.0){\rule[-0.200pt]{0.400pt}{4.818pt}}
\put(615.0,522.0){\rule[-0.200pt]{86.001pt}{0.400pt}}
\put(615.0,512.0){\rule[-0.200pt]{0.400pt}{4.818pt}}
\put(972.0,512.0){\rule[-0.200pt]{0.400pt}{4.818pt}}
\put(580.0,506.0){\rule[-0.200pt]{85.760pt}{0.400pt}}
\put(580.0,496.0){\rule[-0.200pt]{0.400pt}{4.818pt}}
\put(936.0,496.0){\rule[-0.200pt]{0.400pt}{4.818pt}}
\put(523.0,438.0){\rule[-0.200pt]{85.760pt}{0.400pt}}
\put(523.0,428.0){\rule[-0.200pt]{0.400pt}{4.818pt}}
\put(879.0,428.0){\rule[-0.200pt]{0.400pt}{4.818pt}}
\put(633.0,511.0){\rule[-0.200pt]{86.001pt}{0.400pt}}
\put(633.0,501.0){\rule[-0.200pt]{0.400pt}{4.818pt}}
\put(990.0,501.0){\rule[-0.200pt]{0.400pt}{4.818pt}}
\put(337.0,292.0){\rule[-0.200pt]{86.001pt}{0.400pt}}
\put(337.0,282.0){\rule[-0.200pt]{0.400pt}{4.818pt}}
\put(694.0,282.0){\rule[-0.200pt]{0.400pt}{4.818pt}}
\put(341.0,301.0){\rule[-0.200pt]{85.760pt}{0.400pt}}
\put(341.0,291.0){\rule[-0.200pt]{0.400pt}{4.818pt}}
\put(697.0,291.0){\rule[-0.200pt]{0.400pt}{4.818pt}}
\put(273.0,276.0){\rule[-0.200pt]{86.001pt}{0.400pt}}
\put(273.0,266.0){\rule[-0.200pt]{0.400pt}{4.818pt}}
\put(630.0,266.0){\rule[-0.200pt]{0.400pt}{4.818pt}}
\put(373.0,323.0){\rule[-0.200pt]{85.760pt}{0.400pt}}
\put(373.0,313.0){\rule[-0.200pt]{0.400pt}{4.818pt}}
\put(729.0,313.0){\rule[-0.200pt]{0.400pt}{4.818pt}}
\put(280.0,261.0){\rule[-0.200pt]{86.001pt}{0.400pt}}
\put(280.0,251.0){\rule[-0.200pt]{0.400pt}{4.818pt}}
\put(1157,669){\makebox(0,0){$+$}}
\put(1100,661){\makebox(0,0){$+$}}
\put(997,669){\makebox(0,0){$+$}}
\put(993,645){\makebox(0,0){$+$}}
\put(1054,678){\makebox(0,0){$+$}}
\put(786,491){\makebox(0,0){$+$}}
\put(794,522){\makebox(0,0){$+$}}
\put(758,506){\makebox(0,0){$+$}}
\put(701,438){\makebox(0,0){$+$}}
\put(811,511){\makebox(0,0){$+$}}
\put(515,292){\makebox(0,0){$+$}}
\put(519,301){\makebox(0,0){$+$}}
\put(451,276){\makebox(0,0){$+$}}
\put(551,323){\makebox(0,0){$+$}}
\put(458,261){\makebox(0,0){$+$}}
\put(1181,819){\makebox(0,0){$+$}}
\put(637.0,251.0){\rule[-0.200pt]{0.400pt}{4.818pt}}
\put(1111,778){\makebox(0,0)[r]{teoretická závyslosť $y=x$}}
\multiput(1131,778)(20.756,0.000){5}{\usebox{\plotpoint}}
\put(1231,778){\usebox{\plotpoint}}
\put(273,168){\usebox{\plotpoint}}
\put(273.00,168.00){\usebox{\plotpoint}}
\put(291.76,176.88){\usebox{\plotpoint}}
\put(310.61,185.55){\usebox{\plotpoint}}
\put(329.56,194.02){\usebox{\plotpoint}}
\put(348.60,202.27){\usebox{\plotpoint}}
\put(367.49,210.86){\usebox{\plotpoint}}
\put(386.21,219.82){\usebox{\plotpoint}}
\put(405.10,228.41){\usebox{\plotpoint}}
\put(424.19,236.54){\usebox{\plotpoint}}
\put(443.09,245.13){\usebox{\plotpoint}}
\put(461.81,254.09){\usebox{\plotpoint}}
\put(480.70,262.68){\usebox{\plotpoint}}
\put(499.79,270.81){\usebox{\plotpoint}}
\put(518.69,279.40){\usebox{\plotpoint}}
\put(537.47,288.23){\usebox{\plotpoint}}
\put(556.30,296.95){\usebox{\plotpoint}}
\put(575.22,305.49){\usebox{\plotpoint}}
\put(594.28,313.67){\usebox{\plotpoint}}
\put(613.05,322.53){\usebox{\plotpoint}}
\put(631.90,331.23){\usebox{\plotpoint}}
\put(650.76,339.88){\usebox{\plotpoint}}
\put(669.85,347.94){\usebox{\plotpoint}}
\put(688.75,356.52){\usebox{\plotpoint}}
\put(707.46,365.48){\usebox{\plotpoint}}
\put(726.36,374.07){\usebox{\plotpoint}}
\put(745.45,382.20){\usebox{\plotpoint}}
\put(764.35,390.79){\usebox{\plotpoint}}
\put(783.06,399.76){\usebox{\plotpoint}}
\put(801.96,408.34){\usebox{\plotpoint}}
\put(820.70,417.25){\usebox{\plotpoint}}
\put(839.91,425.05){\usebox{\plotpoint}}
\put(858.71,433.85){\usebox{\plotpoint}}
\put(877.53,442.60){\usebox{\plotpoint}}
\put(896.41,451.21){\usebox{\plotpoint}}
\put(915.43,459.43){\usebox{\plotpoint}}
\put(934.27,468.13){\usebox{\plotpoint}}
\put(953.09,476.86){\usebox{\plotpoint}}
\put(971.97,485.49){\usebox{\plotpoint}}
\put(991.02,493.64){\usebox{\plotpoint}}
\put(1009.95,502.16){\usebox{\plotpoint}}
\put(1028.66,511.12){\usebox{\plotpoint}}
\put(1047.56,519.71){\usebox{\plotpoint}}
\put(1066.29,528.64){\usebox{\plotpoint}}
\put(1085.51,536.42){\usebox{\plotpoint}}
\put(1104.23,545.38){\usebox{\plotpoint}}
\put(1123.13,553.97){\usebox{\plotpoint}}
\put(1141.85,562.92){\usebox{\plotpoint}}
\put(1160.99,570.90){\usebox{\plotpoint}}
\put(1179.89,579.45){\usebox{\plotpoint}}
\put(1198.69,588.22){\usebox{\plotpoint}}
\put(1217.59,596.81){\usebox{\plotpoint}}
\put(1236.58,605.12){\usebox{\plotpoint}}
\put(1255.45,613.72){\usebox{\plotpoint}}
\put(1274.26,622.48){\usebox{\plotpoint}}
\put(1293.16,631.08){\usebox{\plotpoint}}
\put(1311.88,640.03){\usebox{\plotpoint}}
\put(1331.11,647.78){\usebox{\plotpoint}}
\put(1336,650){\usebox{\plotpoint}}
\sbox{\plotpoint}{\rule[-0.400pt]{0.800pt}{0.800pt}}%
\sbox{\plotpoint}{\rule[-0.200pt]{0.400pt}{0.400pt}}%
\put(1111,737){\makebox(0,0)[r]{linearny fit $\(1.43\pm0.07\)x -\(0.039\pm0.015\) $}}
\sbox{\plotpoint}{\rule[-0.400pt]{0.800pt}{0.800pt}}%
\put(1131.0,737.0){\rule[-0.400pt]{24.090pt}{0.800pt}}
\put(273,156){\usebox{\plotpoint}}
\multiput(273.00,157.40)(0.825,0.526){7}{\rule{1.457pt}{0.127pt}}
\multiput(273.00,154.34)(7.976,7.000){2}{\rule{0.729pt}{0.800pt}}
\multiput(284.00,164.40)(0.738,0.526){7}{\rule{1.343pt}{0.127pt}}
\multiput(284.00,161.34)(7.213,7.000){2}{\rule{0.671pt}{0.800pt}}
\multiput(294.00,171.40)(0.825,0.526){7}{\rule{1.457pt}{0.127pt}}
\multiput(294.00,168.34)(7.976,7.000){2}{\rule{0.729pt}{0.800pt}}
\multiput(305.00,178.40)(0.825,0.526){7}{\rule{1.457pt}{0.127pt}}
\multiput(305.00,175.34)(7.976,7.000){2}{\rule{0.729pt}{0.800pt}}
\multiput(316.00,185.40)(0.825,0.526){7}{\rule{1.457pt}{0.127pt}}
\multiput(316.00,182.34)(7.976,7.000){2}{\rule{0.729pt}{0.800pt}}
\multiput(327.00,192.40)(0.738,0.526){7}{\rule{1.343pt}{0.127pt}}
\multiput(327.00,189.34)(7.213,7.000){2}{\rule{0.671pt}{0.800pt}}
\multiput(337.00,199.40)(0.825,0.526){7}{\rule{1.457pt}{0.127pt}}
\multiput(337.00,196.34)(7.976,7.000){2}{\rule{0.729pt}{0.800pt}}
\multiput(348.00,206.40)(0.825,0.526){7}{\rule{1.457pt}{0.127pt}}
\multiput(348.00,203.34)(7.976,7.000){2}{\rule{0.729pt}{0.800pt}}
\multiput(359.00,213.40)(0.825,0.526){7}{\rule{1.457pt}{0.127pt}}
\multiput(359.00,210.34)(7.976,7.000){2}{\rule{0.729pt}{0.800pt}}
\multiput(370.00,220.40)(0.738,0.526){7}{\rule{1.343pt}{0.127pt}}
\multiput(370.00,217.34)(7.213,7.000){2}{\rule{0.671pt}{0.800pt}}
\multiput(380.00,227.40)(0.825,0.526){7}{\rule{1.457pt}{0.127pt}}
\multiput(380.00,224.34)(7.976,7.000){2}{\rule{0.729pt}{0.800pt}}
\multiput(391.00,234.40)(0.825,0.526){7}{\rule{1.457pt}{0.127pt}}
\multiput(391.00,231.34)(7.976,7.000){2}{\rule{0.729pt}{0.800pt}}
\multiput(402.00,241.40)(0.825,0.526){7}{\rule{1.457pt}{0.127pt}}
\multiput(402.00,238.34)(7.976,7.000){2}{\rule{0.729pt}{0.800pt}}
\multiput(413.00,248.40)(0.738,0.526){7}{\rule{1.343pt}{0.127pt}}
\multiput(413.00,245.34)(7.213,7.000){2}{\rule{0.671pt}{0.800pt}}
\multiput(423.00,255.40)(0.825,0.526){7}{\rule{1.457pt}{0.127pt}}
\multiput(423.00,252.34)(7.976,7.000){2}{\rule{0.729pt}{0.800pt}}
\multiput(434.00,262.40)(0.825,0.526){7}{\rule{1.457pt}{0.127pt}}
\multiput(434.00,259.34)(7.976,7.000){2}{\rule{0.729pt}{0.800pt}}
\multiput(445.00,269.40)(0.738,0.526){7}{\rule{1.343pt}{0.127pt}}
\multiput(445.00,266.34)(7.213,7.000){2}{\rule{0.671pt}{0.800pt}}
\multiput(455.00,276.40)(0.825,0.526){7}{\rule{1.457pt}{0.127pt}}
\multiput(455.00,273.34)(7.976,7.000){2}{\rule{0.729pt}{0.800pt}}
\multiput(466.00,283.39)(1.020,0.536){5}{\rule{1.667pt}{0.129pt}}
\multiput(466.00,280.34)(7.541,6.000){2}{\rule{0.833pt}{0.800pt}}
\multiput(477.00,289.40)(0.825,0.526){7}{\rule{1.457pt}{0.127pt}}
\multiput(477.00,286.34)(7.976,7.000){2}{\rule{0.729pt}{0.800pt}}
\multiput(488.00,296.40)(0.738,0.526){7}{\rule{1.343pt}{0.127pt}}
\multiput(488.00,293.34)(7.213,7.000){2}{\rule{0.671pt}{0.800pt}}
\multiput(498.00,303.40)(0.825,0.526){7}{\rule{1.457pt}{0.127pt}}
\multiput(498.00,300.34)(7.976,7.000){2}{\rule{0.729pt}{0.800pt}}
\multiput(509.00,310.40)(0.825,0.526){7}{\rule{1.457pt}{0.127pt}}
\multiput(509.00,307.34)(7.976,7.000){2}{\rule{0.729pt}{0.800pt}}
\multiput(520.00,317.40)(0.825,0.526){7}{\rule{1.457pt}{0.127pt}}
\multiput(520.00,314.34)(7.976,7.000){2}{\rule{0.729pt}{0.800pt}}
\multiput(531.00,324.40)(0.738,0.526){7}{\rule{1.343pt}{0.127pt}}
\multiput(531.00,321.34)(7.213,7.000){2}{\rule{0.671pt}{0.800pt}}
\multiput(541.00,331.40)(0.825,0.526){7}{\rule{1.457pt}{0.127pt}}
\multiput(541.00,328.34)(7.976,7.000){2}{\rule{0.729pt}{0.800pt}}
\multiput(552.00,338.40)(0.825,0.526){7}{\rule{1.457pt}{0.127pt}}
\multiput(552.00,335.34)(7.976,7.000){2}{\rule{0.729pt}{0.800pt}}
\multiput(563.00,345.40)(0.825,0.526){7}{\rule{1.457pt}{0.127pt}}
\multiput(563.00,342.34)(7.976,7.000){2}{\rule{0.729pt}{0.800pt}}
\multiput(574.00,352.40)(0.738,0.526){7}{\rule{1.343pt}{0.127pt}}
\multiput(574.00,349.34)(7.213,7.000){2}{\rule{0.671pt}{0.800pt}}
\multiput(584.00,359.40)(0.825,0.526){7}{\rule{1.457pt}{0.127pt}}
\multiput(584.00,356.34)(7.976,7.000){2}{\rule{0.729pt}{0.800pt}}
\multiput(595.00,366.40)(0.825,0.526){7}{\rule{1.457pt}{0.127pt}}
\multiput(595.00,363.34)(7.976,7.000){2}{\rule{0.729pt}{0.800pt}}
\multiput(606.00,373.40)(0.738,0.526){7}{\rule{1.343pt}{0.127pt}}
\multiput(606.00,370.34)(7.213,7.000){2}{\rule{0.671pt}{0.800pt}}
\multiput(616.00,380.40)(0.825,0.526){7}{\rule{1.457pt}{0.127pt}}
\multiput(616.00,377.34)(7.976,7.000){2}{\rule{0.729pt}{0.800pt}}
\multiput(627.00,387.40)(0.825,0.526){7}{\rule{1.457pt}{0.127pt}}
\multiput(627.00,384.34)(7.976,7.000){2}{\rule{0.729pt}{0.800pt}}
\multiput(638.00,394.40)(0.825,0.526){7}{\rule{1.457pt}{0.127pt}}
\multiput(638.00,391.34)(7.976,7.000){2}{\rule{0.729pt}{0.800pt}}
\multiput(649.00,401.40)(0.738,0.526){7}{\rule{1.343pt}{0.127pt}}
\multiput(649.00,398.34)(7.213,7.000){2}{\rule{0.671pt}{0.800pt}}
\multiput(659.00,408.40)(0.825,0.526){7}{\rule{1.457pt}{0.127pt}}
\multiput(659.00,405.34)(7.976,7.000){2}{\rule{0.729pt}{0.800pt}}
\multiput(670.00,415.40)(0.825,0.526){7}{\rule{1.457pt}{0.127pt}}
\multiput(670.00,412.34)(7.976,7.000){2}{\rule{0.729pt}{0.800pt}}
\multiput(681.00,422.40)(0.825,0.526){7}{\rule{1.457pt}{0.127pt}}
\multiput(681.00,419.34)(7.976,7.000){2}{\rule{0.729pt}{0.800pt}}
\multiput(692.00,429.40)(0.738,0.526){7}{\rule{1.343pt}{0.127pt}}
\multiput(692.00,426.34)(7.213,7.000){2}{\rule{0.671pt}{0.800pt}}
\multiput(702.00,436.40)(0.825,0.526){7}{\rule{1.457pt}{0.127pt}}
\multiput(702.00,433.34)(7.976,7.000){2}{\rule{0.729pt}{0.800pt}}
\multiput(713.00,443.40)(0.825,0.526){7}{\rule{1.457pt}{0.127pt}}
\multiput(713.00,440.34)(7.976,7.000){2}{\rule{0.729pt}{0.800pt}}
\multiput(724.00,450.40)(0.825,0.526){7}{\rule{1.457pt}{0.127pt}}
\multiput(724.00,447.34)(7.976,7.000){2}{\rule{0.729pt}{0.800pt}}
\multiput(735.00,457.40)(0.738,0.526){7}{\rule{1.343pt}{0.127pt}}
\multiput(735.00,454.34)(7.213,7.000){2}{\rule{0.671pt}{0.800pt}}
\multiput(745.00,464.40)(0.825,0.526){7}{\rule{1.457pt}{0.127pt}}
\multiput(745.00,461.34)(7.976,7.000){2}{\rule{0.729pt}{0.800pt}}
\multiput(756.00,471.39)(1.020,0.536){5}{\rule{1.667pt}{0.129pt}}
\multiput(756.00,468.34)(7.541,6.000){2}{\rule{0.833pt}{0.800pt}}
\multiput(767.00,477.40)(0.738,0.526){7}{\rule{1.343pt}{0.127pt}}
\multiput(767.00,474.34)(7.213,7.000){2}{\rule{0.671pt}{0.800pt}}
\multiput(777.00,484.40)(0.825,0.526){7}{\rule{1.457pt}{0.127pt}}
\multiput(777.00,481.34)(7.976,7.000){2}{\rule{0.729pt}{0.800pt}}
\multiput(788.00,491.40)(0.825,0.526){7}{\rule{1.457pt}{0.127pt}}
\multiput(788.00,488.34)(7.976,7.000){2}{\rule{0.729pt}{0.800pt}}
\multiput(799.00,498.40)(0.825,0.526){7}{\rule{1.457pt}{0.127pt}}
\multiput(799.00,495.34)(7.976,7.000){2}{\rule{0.729pt}{0.800pt}}
\multiput(810.00,505.40)(0.738,0.526){7}{\rule{1.343pt}{0.127pt}}
\multiput(810.00,502.34)(7.213,7.000){2}{\rule{0.671pt}{0.800pt}}
\multiput(820.00,512.40)(0.825,0.526){7}{\rule{1.457pt}{0.127pt}}
\multiput(820.00,509.34)(7.976,7.000){2}{\rule{0.729pt}{0.800pt}}
\multiput(831.00,519.40)(0.825,0.526){7}{\rule{1.457pt}{0.127pt}}
\multiput(831.00,516.34)(7.976,7.000){2}{\rule{0.729pt}{0.800pt}}
\multiput(842.00,526.40)(0.825,0.526){7}{\rule{1.457pt}{0.127pt}}
\multiput(842.00,523.34)(7.976,7.000){2}{\rule{0.729pt}{0.800pt}}
\multiput(853.00,533.40)(0.738,0.526){7}{\rule{1.343pt}{0.127pt}}
\multiput(853.00,530.34)(7.213,7.000){2}{\rule{0.671pt}{0.800pt}}
\multiput(863.00,540.40)(0.825,0.526){7}{\rule{1.457pt}{0.127pt}}
\multiput(863.00,537.34)(7.976,7.000){2}{\rule{0.729pt}{0.800pt}}
\multiput(874.00,547.40)(0.825,0.526){7}{\rule{1.457pt}{0.127pt}}
\multiput(874.00,544.34)(7.976,7.000){2}{\rule{0.729pt}{0.800pt}}
\multiput(885.00,554.40)(0.825,0.526){7}{\rule{1.457pt}{0.127pt}}
\multiput(885.00,551.34)(7.976,7.000){2}{\rule{0.729pt}{0.800pt}}
\multiput(896.00,561.40)(0.738,0.526){7}{\rule{1.343pt}{0.127pt}}
\multiput(896.00,558.34)(7.213,7.000){2}{\rule{0.671pt}{0.800pt}}
\multiput(906.00,568.40)(0.825,0.526){7}{\rule{1.457pt}{0.127pt}}
\multiput(906.00,565.34)(7.976,7.000){2}{\rule{0.729pt}{0.800pt}}
\multiput(917.00,575.40)(0.825,0.526){7}{\rule{1.457pt}{0.127pt}}
\multiput(917.00,572.34)(7.976,7.000){2}{\rule{0.729pt}{0.800pt}}
\multiput(928.00,582.40)(0.738,0.526){7}{\rule{1.343pt}{0.127pt}}
\multiput(928.00,579.34)(7.213,7.000){2}{\rule{0.671pt}{0.800pt}}
\multiput(938.00,589.40)(0.825,0.526){7}{\rule{1.457pt}{0.127pt}}
\multiput(938.00,586.34)(7.976,7.000){2}{\rule{0.729pt}{0.800pt}}
\multiput(949.00,596.40)(0.825,0.526){7}{\rule{1.457pt}{0.127pt}}
\multiput(949.00,593.34)(7.976,7.000){2}{\rule{0.729pt}{0.800pt}}
\multiput(960.00,603.40)(0.825,0.526){7}{\rule{1.457pt}{0.127pt}}
\multiput(960.00,600.34)(7.976,7.000){2}{\rule{0.729pt}{0.800pt}}
\multiput(971.00,610.40)(0.738,0.526){7}{\rule{1.343pt}{0.127pt}}
\multiput(971.00,607.34)(7.213,7.000){2}{\rule{0.671pt}{0.800pt}}
\multiput(981.00,617.40)(0.825,0.526){7}{\rule{1.457pt}{0.127pt}}
\multiput(981.00,614.34)(7.976,7.000){2}{\rule{0.729pt}{0.800pt}}
\multiput(992.00,624.40)(0.825,0.526){7}{\rule{1.457pt}{0.127pt}}
\multiput(992.00,621.34)(7.976,7.000){2}{\rule{0.729pt}{0.800pt}}
\multiput(1003.00,631.40)(0.825,0.526){7}{\rule{1.457pt}{0.127pt}}
\multiput(1003.00,628.34)(7.976,7.000){2}{\rule{0.729pt}{0.800pt}}
\multiput(1014.00,638.40)(0.738,0.526){7}{\rule{1.343pt}{0.127pt}}
\multiput(1014.00,635.34)(7.213,7.000){2}{\rule{0.671pt}{0.800pt}}
\multiput(1024.00,645.40)(0.825,0.526){7}{\rule{1.457pt}{0.127pt}}
\multiput(1024.00,642.34)(7.976,7.000){2}{\rule{0.729pt}{0.800pt}}
\multiput(1035.00,652.39)(1.020,0.536){5}{\rule{1.667pt}{0.129pt}}
\multiput(1035.00,649.34)(7.541,6.000){2}{\rule{0.833pt}{0.800pt}}
\multiput(1046.00,658.40)(0.825,0.526){7}{\rule{1.457pt}{0.127pt}}
\multiput(1046.00,655.34)(7.976,7.000){2}{\rule{0.729pt}{0.800pt}}
\multiput(1057.00,665.40)(0.738,0.526){7}{\rule{1.343pt}{0.127pt}}
\multiput(1057.00,662.34)(7.213,7.000){2}{\rule{0.671pt}{0.800pt}}
\multiput(1067.00,672.40)(0.825,0.526){7}{\rule{1.457pt}{0.127pt}}
\multiput(1067.00,669.34)(7.976,7.000){2}{\rule{0.729pt}{0.800pt}}
\multiput(1078.00,679.40)(0.825,0.526){7}{\rule{1.457pt}{0.127pt}}
\multiput(1078.00,676.34)(7.976,7.000){2}{\rule{0.729pt}{0.800pt}}
\multiput(1089.00,686.40)(0.738,0.526){7}{\rule{1.343pt}{0.127pt}}
\multiput(1089.00,683.34)(7.213,7.000){2}{\rule{0.671pt}{0.800pt}}
\multiput(1099.00,693.40)(0.825,0.526){7}{\rule{1.457pt}{0.127pt}}
\multiput(1099.00,690.34)(7.976,7.000){2}{\rule{0.729pt}{0.800pt}}
\multiput(1110.00,700.40)(0.825,0.526){7}{\rule{1.457pt}{0.127pt}}
\multiput(1110.00,697.34)(7.976,7.000){2}{\rule{0.729pt}{0.800pt}}
\multiput(1121.00,707.40)(0.825,0.526){7}{\rule{1.457pt}{0.127pt}}
\multiput(1121.00,704.34)(7.976,7.000){2}{\rule{0.729pt}{0.800pt}}
\multiput(1132.00,714.40)(0.738,0.526){7}{\rule{1.343pt}{0.127pt}}
\multiput(1132.00,711.34)(7.213,7.000){2}{\rule{0.671pt}{0.800pt}}
\multiput(1142.00,721.40)(0.825,0.526){7}{\rule{1.457pt}{0.127pt}}
\multiput(1142.00,718.34)(7.976,7.000){2}{\rule{0.729pt}{0.800pt}}
\multiput(1153.00,728.40)(0.825,0.526){7}{\rule{1.457pt}{0.127pt}}
\multiput(1153.00,725.34)(7.976,7.000){2}{\rule{0.729pt}{0.800pt}}
\multiput(1164.00,735.40)(0.825,0.526){7}{\rule{1.457pt}{0.127pt}}
\multiput(1164.00,732.34)(7.976,7.000){2}{\rule{0.729pt}{0.800pt}}
\multiput(1175.00,742.40)(0.738,0.526){7}{\rule{1.343pt}{0.127pt}}
\multiput(1175.00,739.34)(7.213,7.000){2}{\rule{0.671pt}{0.800pt}}
\multiput(1185.00,749.40)(0.825,0.526){7}{\rule{1.457pt}{0.127pt}}
\multiput(1185.00,746.34)(7.976,7.000){2}{\rule{0.729pt}{0.800pt}}
\multiput(1196.00,756.40)(0.825,0.526){7}{\rule{1.457pt}{0.127pt}}
\multiput(1196.00,753.34)(7.976,7.000){2}{\rule{0.729pt}{0.800pt}}
\multiput(1207.00,763.40)(0.825,0.526){7}{\rule{1.457pt}{0.127pt}}
\multiput(1207.00,760.34)(7.976,7.000){2}{\rule{0.729pt}{0.800pt}}
\multiput(1218.00,770.40)(0.738,0.526){7}{\rule{1.343pt}{0.127pt}}
\multiput(1218.00,767.34)(7.213,7.000){2}{\rule{0.671pt}{0.800pt}}
\multiput(1228.00,777.40)(0.825,0.526){7}{\rule{1.457pt}{0.127pt}}
\multiput(1228.00,774.34)(7.976,7.000){2}{\rule{0.729pt}{0.800pt}}
\multiput(1239.00,784.40)(0.825,0.526){7}{\rule{1.457pt}{0.127pt}}
\multiput(1239.00,781.34)(7.976,7.000){2}{\rule{0.729pt}{0.800pt}}
\multiput(1250.00,791.40)(0.738,0.526){7}{\rule{1.343pt}{0.127pt}}
\multiput(1250.00,788.34)(7.213,7.000){2}{\rule{0.671pt}{0.800pt}}
\multiput(1260.00,798.40)(0.825,0.526){7}{\rule{1.457pt}{0.127pt}}
\multiput(1260.00,795.34)(7.976,7.000){2}{\rule{0.729pt}{0.800pt}}
\multiput(1271.00,805.40)(0.825,0.526){7}{\rule{1.457pt}{0.127pt}}
\multiput(1271.00,802.34)(7.976,7.000){2}{\rule{0.729pt}{0.800pt}}
\multiput(1282.00,812.40)(0.825,0.526){7}{\rule{1.457pt}{0.127pt}}
\multiput(1282.00,809.34)(7.976,7.000){2}{\rule{0.729pt}{0.800pt}}
\multiput(1293.00,819.40)(0.738,0.526){7}{\rule{1.343pt}{0.127pt}}
\multiput(1293.00,816.34)(7.213,7.000){2}{\rule{0.671pt}{0.800pt}}
\multiput(1303.00,826.40)(0.825,0.526){7}{\rule{1.457pt}{0.127pt}}
\multiput(1303.00,823.34)(7.976,7.000){2}{\rule{0.729pt}{0.800pt}}
\multiput(1314.00,833.39)(1.020,0.536){5}{\rule{1.667pt}{0.129pt}}
\multiput(1314.00,830.34)(7.541,6.000){2}{\rule{0.833pt}{0.800pt}}
\multiput(1325.00,839.40)(0.825,0.526){7}{\rule{1.457pt}{0.127pt}}
\multiput(1325.00,836.34)(7.976,7.000){2}{\rule{0.729pt}{0.800pt}}
\sbox{\plotpoint}{\rule[-0.200pt]{0.400pt}{0.400pt}}%
\put(191.0,131.0){\rule[-0.200pt]{0.400pt}{175.375pt}}
\put(191.0,131.0){\rule[-0.200pt]{300.643pt}{0.400pt}}
\put(1439.0,131.0){\rule[-0.200pt]{0.400pt}{175.375pt}}
\put(191.0,859.0){\rule[-0.200pt]{300.643pt}{0.400pt}}
\end{picture}

\caption{Namerané hodnoty ciachovaného ampérmetru $I_x$ v závislosti od hodnôt odčítaných z kompenzátoru
a prepočítaných na prúd $I_k$, preložené závislosťou $I_k = "0.95" \cdot I_x + "0.017 mA"$ }  \label{G_2}
\end{figure}

Z lineárneho fitu dostávame výsledný vzťah
\eq{
I_k = "0.95" \cdot I_x + "0.017 mA" \,. \lbl{R_11}
}


\subsection{Ciachovanie odporové dekády}

Pre hodnoty odporu na odporovej dekáde v rozsahu $R = "1000-500 \Omega"$ 
bol použitý normálový rezistor o hodnote $R_n = "1000 \Omega"$ a 
pre hodnoty nižšie ako $R="500 \Omega"$ bol použitý normálový rezistor $R_n = "100 \Omega"$.

Napätie na odporovej dekáde označme ako $U_x$ a na normálovom odpore $U_n$, pričom $R_n$ je odpor normálového rezistoru a $R_x$ hodnota odporu nevoleného na odporovej dekáde.

Namerané hodnoty boli vynesené do tabuľky \ref{T_3} a podľa vzťahu \ref{R_3} 
boli vypočítané hodnoty $R_k$.

Závislosť $R_k$ na $R_x$ bola vynesená do grafu Obr. \ref{G_3} a z výsledku fitu bol určený vzťah $R_k$ na $R_x$
\eq{
R_k = "1.00" \cdot R_x +"1.83 \Omega" \,.
}
\begin{table}[h]

\begin{center}
\begin{tabular}{| c | c | c | c | c | c |}
\hline
 \popi{U_x}{V} & \popi{I}{mA} &\popi{U_n}{V} & \popi{R_c}{\Omega} &\popi{R_x}{\Omega} & \popi{R_k}{\Omega}\\
\hline
$0.6195\pm0.001$ & $0.6$ & $0.6158\pm0.001$ & $1000$ & $1000$ & $1006.00$\\
$0.5792\pm0.001$ & $0.64$ & $0.6431\pm0.001$ & $1000$ & $900$ & $900.64$\\
$0.5373\pm0.001$ & $0.66$ & $0.6709\pm0.001$ & $1000$ & $800$ & $800.86$\\
$0.4915\pm0.001$ & $0.70$ & $0.7021\pm0.001$ & $1000$ & $700$ & $700.04$\\
$0.4418\pm0.001$ & $0.72$ & $0.7364\pm0.001$ & $1000$ & $600$ & $599.95$\\
$0.3874\pm0.001$ & $0.76$ & $0.7745\pm0.001$ & $1000$ & $500$ & $500.19$\\
$0.6450\pm0.001$ & $1.62$ & $0.163\pm0.001$ & $100$ & $400$ & $395.71$\\
$0.5482\pm0.001$ & $1.82$ & $0.183\pm0.001$ & $100$ & $300$ & $299.56$\\
$0.4170\pm0.001$ & $2.06$ & $0.2088\pm0.001$ & $100$ & $200$ & $199.71$\\
$0.2426\pm0.001$ & $2.4$ & $0.2429\pm0.001$ & $100$ & $100$ & $99.88$\\
$0.2004\pm0.001$ & $2.48$ & $0.2504\pm0.001$ & $100$ & $80$ & $80.032$\\
\hline

\end{tabular}
\caption{Namerané hodnoty ciachovania odporovej dekády $R_x$ v závislosti vypočítanej hodnote odporu $R_k$} \label{T_3}
\end{center}
\end{table}



\begin{figure}
% GNUPLOT: LaTeX picture
\setlength{\unitlength}{0.240900pt}
\ifx\plotpoint\undefined\newsavebox{\plotpoint}\fi
\begin{picture}(1500,900)(0,0)
\sbox{\plotpoint}{\rule[-0.200pt]{0.400pt}{0.400pt}}%
\put(191.0,131.0){\rule[-0.200pt]{4.818pt}{0.400pt}}
\put(171,131){\makebox(0,0)[r]{ 0}}
\put(1419.0,131.0){\rule[-0.200pt]{4.818pt}{0.400pt}}
\put(191.0,197.0){\rule[-0.200pt]{4.818pt}{0.400pt}}
\put(171,197){\makebox(0,0)[r]{ 100}}
\put(1419.0,197.0){\rule[-0.200pt]{4.818pt}{0.400pt}}
\put(191.0,263.0){\rule[-0.200pt]{4.818pt}{0.400pt}}
\put(171,263){\makebox(0,0)[r]{ 200}}
\put(1419.0,263.0){\rule[-0.200pt]{4.818pt}{0.400pt}}
\put(191.0,330.0){\rule[-0.200pt]{4.818pt}{0.400pt}}
\put(171,330){\makebox(0,0)[r]{ 300}}
\put(1419.0,330.0){\rule[-0.200pt]{4.818pt}{0.400pt}}
\put(191.0,396.0){\rule[-0.200pt]{4.818pt}{0.400pt}}
\put(171,396){\makebox(0,0)[r]{ 400}}
\put(1419.0,396.0){\rule[-0.200pt]{4.818pt}{0.400pt}}
\put(191.0,462.0){\rule[-0.200pt]{4.818pt}{0.400pt}}
\put(171,462){\makebox(0,0)[r]{ 500}}
\put(1419.0,462.0){\rule[-0.200pt]{4.818pt}{0.400pt}}
\put(191.0,528.0){\rule[-0.200pt]{4.818pt}{0.400pt}}
\put(171,528){\makebox(0,0)[r]{ 600}}
\put(1419.0,528.0){\rule[-0.200pt]{4.818pt}{0.400pt}}
\put(191.0,594.0){\rule[-0.200pt]{4.818pt}{0.400pt}}
\put(171,594){\makebox(0,0)[r]{ 700}}
\put(1419.0,594.0){\rule[-0.200pt]{4.818pt}{0.400pt}}
\put(191.0,660.0){\rule[-0.200pt]{4.818pt}{0.400pt}}
\put(171,660){\makebox(0,0)[r]{ 800}}
\put(1419.0,660.0){\rule[-0.200pt]{4.818pt}{0.400pt}}
\put(191.0,727.0){\rule[-0.200pt]{4.818pt}{0.400pt}}
\put(171,727){\makebox(0,0)[r]{ 900}}
\put(1419.0,727.0){\rule[-0.200pt]{4.818pt}{0.400pt}}
\put(191.0,793.0){\rule[-0.200pt]{4.818pt}{0.400pt}}
\put(171,793){\makebox(0,0)[r]{ 1000}}
\put(1419.0,793.0){\rule[-0.200pt]{4.818pt}{0.400pt}}
\put(191.0,859.0){\rule[-0.200pt]{4.818pt}{0.400pt}}
\put(171,859){\makebox(0,0)[r]{ 1100}}
\put(1419.0,859.0){\rule[-0.200pt]{4.818pt}{0.400pt}}
\put(191.0,131.0){\rule[-0.200pt]{0.400pt}{4.818pt}}
\put(191,90){\makebox(0,0){ 0}}
\put(191.0,839.0){\rule[-0.200pt]{0.400pt}{4.818pt}}
\put(304.0,131.0){\rule[-0.200pt]{0.400pt}{4.818pt}}
\put(304,90){\makebox(0,0){ 100}}
\put(304.0,839.0){\rule[-0.200pt]{0.400pt}{4.818pt}}
\put(418.0,131.0){\rule[-0.200pt]{0.400pt}{4.818pt}}
\put(418,90){\makebox(0,0){ 200}}
\put(418.0,839.0){\rule[-0.200pt]{0.400pt}{4.818pt}}
\put(531.0,131.0){\rule[-0.200pt]{0.400pt}{4.818pt}}
\put(531,90){\makebox(0,0){ 300}}
\put(531.0,839.0){\rule[-0.200pt]{0.400pt}{4.818pt}}
\put(645.0,131.0){\rule[-0.200pt]{0.400pt}{4.818pt}}
\put(645,90){\makebox(0,0){ 400}}
\put(645.0,839.0){\rule[-0.200pt]{0.400pt}{4.818pt}}
\put(758.0,131.0){\rule[-0.200pt]{0.400pt}{4.818pt}}
\put(758,90){\makebox(0,0){ 500}}
\put(758.0,839.0){\rule[-0.200pt]{0.400pt}{4.818pt}}
\put(872.0,131.0){\rule[-0.200pt]{0.400pt}{4.818pt}}
\put(872,90){\makebox(0,0){ 600}}
\put(872.0,839.0){\rule[-0.200pt]{0.400pt}{4.818pt}}
\put(985.0,131.0){\rule[-0.200pt]{0.400pt}{4.818pt}}
\put(985,90){\makebox(0,0){ 700}}
\put(985.0,839.0){\rule[-0.200pt]{0.400pt}{4.818pt}}
\put(1099.0,131.0){\rule[-0.200pt]{0.400pt}{4.818pt}}
\put(1099,90){\makebox(0,0){ 800}}
\put(1099.0,839.0){\rule[-0.200pt]{0.400pt}{4.818pt}}
\put(1212.0,131.0){\rule[-0.200pt]{0.400pt}{4.818pt}}
\put(1212,90){\makebox(0,0){ 900}}
\put(1212.0,839.0){\rule[-0.200pt]{0.400pt}{4.818pt}}
\put(1326.0,131.0){\rule[-0.200pt]{0.400pt}{4.818pt}}
\put(1326,90){\makebox(0,0){ 1000}}
\put(1326.0,839.0){\rule[-0.200pt]{0.400pt}{4.818pt}}
\put(1439.0,131.0){\rule[-0.200pt]{0.400pt}{4.818pt}}
\put(1439,90){\makebox(0,0){ 1100}}
\put(1439.0,839.0){\rule[-0.200pt]{0.400pt}{4.818pt}}
\put(191.0,131.0){\rule[-0.200pt]{0.400pt}{175.375pt}}
\put(191.0,131.0){\rule[-0.200pt]{300.643pt}{0.400pt}}
\put(1439.0,131.0){\rule[-0.200pt]{0.400pt}{175.375pt}}
\put(191.0,859.0){\rule[-0.200pt]{300.643pt}{0.400pt}}
\put(30,495){\makebox(0,0){\popi{R_d}{\Omega}}}
\put(815,29){\makebox(0,0){\popi{R_x}{\Omega}}}
\put(1211,819){\makebox(0,0)[r]{Namerané hodnoty}}
\put(1332,793){\makebox(0,0){$+$}}
\put(1213,727){\makebox(0,0){$+$}}
\put(1100,660){\makebox(0,0){$+$}}
\put(985,594){\makebox(0,0){$+$}}
\put(872,528){\makebox(0,0){$+$}}
\put(758,462){\makebox(0,0){$+$}}
\put(640,396){\makebox(0,0){$+$}}
\put(531,330){\makebox(0,0){$+$}}
\put(418,263){\makebox(0,0){$+$}}
\put(304,197){\makebox(0,0){$+$}}
\put(282,184){\makebox(0,0){$+$}}
\put(1281,819){\makebox(0,0){$+$}}
\put(1211,778){\makebox(0,0)[r]{Preložená závyslosť $R_x = "1.00" \cdot R_d +"1.83 \Omega"$}}
\multiput(1231,778)(20.756,0.000){5}{\usebox{\plotpoint}}
\put(1331,778){\usebox{\plotpoint}}
\put(282,185){\usebox{\plotpoint}}
\put(282.00,185.00){\usebox{\plotpoint}}
\put(299.98,195.35){\usebox{\plotpoint}}
\put(317.92,205.75){\usebox{\plotpoint}}
\put(335.69,216.42){\usebox{\plotpoint}}
\put(353.69,226.74){\usebox{\plotpoint}}
\put(371.80,236.88){\usebox{\plotpoint}}
\put(389.45,247.79){\usebox{\plotpoint}}
\put(407.47,258.08){\usebox{\plotpoint}}
\put(425.52,268.31){\usebox{\plotpoint}}
\put(443.54,278.61){\usebox{\plotpoint}}
\put(461.19,289.52){\usebox{\plotpoint}}
\put(479.25,299.75){\usebox{\plotpoint}}
\put(497.38,309.84){\usebox{\plotpoint}}
\put(514.89,320.92){\usebox{\plotpoint}}
\put(532.94,331.16){\usebox{\plotpoint}}
\put(551.09,341.23){\usebox{\plotpoint}}
\put(569.03,351.66){\usebox{\plotpoint}}
\put(586.85,362.28){\usebox{\plotpoint}}
\put(604.83,372.63){\usebox{\plotpoint}}
\put(622.81,382.99){\usebox{\plotpoint}}
\put(640.68,393.53){\usebox{\plotpoint}}
\put(658.57,404.04){\usebox{\plotpoint}}
\put(676.55,414.39){\usebox{\plotpoint}}
\put(694.78,424.33){\usebox{\plotpoint}}
\put(712.28,435.43){\usebox{\plotpoint}}
\put(730.26,445.78){\usebox{\plotpoint}}
\put(748.49,455.72){\usebox{\plotpoint}}
\put(766.18,466.57){\usebox{\plotpoint}}
\put(784.17,476.90){\usebox{\plotpoint}}
\put(802.22,487.13){\usebox{\plotpoint}}
\put(820.21,497.48){\usebox{\plotpoint}}
\put(837.89,508.34){\usebox{\plotpoint}}
\put(855.95,518.57){\usebox{\plotpoint}}
\put(873.95,528.88){\usebox{\plotpoint}}
\put(891.83,539.38){\usebox{\plotpoint}}
\put(909.68,549.91){\usebox{\plotpoint}}
\put(927.69,560.22){\usebox{\plotpoint}}
\put(945.75,570.45){\usebox{\plotpoint}}
\put(963.42,581.32){\usebox{\plotpoint}}
\put(981.42,591.65){\usebox{\plotpoint}}
\put(999.47,601.88){\usebox{\plotpoint}}
\put(1017.47,612.21){\usebox{\plotpoint}}
\put(1035.14,623.08){\usebox{\plotpoint}}
\put(1053.19,633.32){\usebox{\plotpoint}}
\put(1071.35,643.37){\usebox{\plotpoint}}
\put(1088.85,654.47){\usebox{\plotpoint}}
\put(1107.07,664.40){\usebox{\plotpoint}}
\put(1125.06,674.76){\usebox{\plotpoint}}
\put(1143.04,685.11){\usebox{\plotpoint}}
\put(1160.90,695.66){\usebox{\plotpoint}}
\put(1178.80,706.16){\usebox{\plotpoint}}
\put(1196.78,716.52){\usebox{\plotpoint}}
\put(1215.00,726.46){\usebox{\plotpoint}}
\put(1232.51,737.55){\usebox{\plotpoint}}
\put(1250.49,747.90){\usebox{\plotpoint}}
\put(1268.71,757.84){\usebox{\plotpoint}}
\put(1286.40,768.71){\usebox{\plotpoint}}
\put(1304.39,779.03){\usebox{\plotpoint}}
\put(1322.44,789.27){\usebox{\plotpoint}}
\put(1332,795){\usebox{\plotpoint}}
\put(191.0,131.0){\rule[-0.200pt]{0.400pt}{175.375pt}}
\put(191.0,131.0){\rule[-0.200pt]{300.643pt}{0.400pt}}
\put(1439.0,131.0){\rule[-0.200pt]{0.400pt}{175.375pt}}
\put(191.0,859.0){\rule[-0.200pt]{300.643pt}{0.400pt}}
\end{picture}

\caption{Namerané hodnoty ciachovania odporovej dekády $R_x$ v závislosti vypočítanej hodnote odporu $R_k$, preložené závislosťou $R_k = "1.00" \cdot R_x +"1.83 \Omega"$ }  \label{G_3}
\end{figure}





\subsection{Rozšírenie rozsahu ampérmetru}



Namerané hodnoty prúdu na ampérmetri s bočníkom onačme ako $I_2$ a prúd pretekajúci častou s nezaradením bočníkom ako $T_1$. Odpor bočníka označme $R_b$.
Na meranie prúdu $T_2$ bol použitý ampérmeter ktorý bol v predchádzajúcej časti ciachovaný. Teda skutočné hodnoty prúdu po prepočet vzťahom \ref{R_11} boli označené ako $I^\prime_2$.
Tieto hodnoty boli vynesené do Tab. \ref{T_4}

\begin{table}[h]
\begin{center}
\begin{tabular}{| c | c | c | c | c | }
\hline
 \popi{R_b}{\Omega} & \popi{I_2}{mA} &\popi{I_1}{mA} & \popi{I^\prime_2}{mA}  & \popi{R_0}{\Omega}\\
\hline
$\infty$ & $0.22$ & $0.22$ & $0.22$ & $-$\\
$200$ & $0.16$ & $0.22$ & $0.17$    &  $60.72$\\
$100$ & $0.12$ & $0.22$ & $0.13$   &  $68.25$\\
$70$ & $0.1$ & $0.22$ & $0.11$      &  $67.80$\\
$90$ & $0.11$ & $0.22$ & $0.12$ & $73.30$\\
$300$ & $0.18$ & $0.22$ & $0.19$ & $51.50$\\
$400$ & $0.2$ & $0.22$ & $0.21$ &  $25.60$\\
$20$ & $0.04$ & $0.22$ & $0.055$ &  $60.38$\\
$50$ & $0.08$ & $0.22$ & $0.093$ & $68.60$\\
$500$ & $0.21$ & $0.22$ & $0.21$ &  $8.63$\\
$800$ & $0.21$ & $0.22$ & $0.21$ &  $13.80$\\

\hline

\end{tabular}
\caption{Namerané hodnoty prúdov $I_1$ a $I_2$, hodnota odporu bočníku, prepočítaný prúd kalibračnou rovnicou pre ampérmeter $I^\prime_2$ a vypočítaný vnútorný odpor ampérmetru $R_0$} \label{T_4}
\end{center}
\end{table}

Podľa vzťahu \ref{R_5} bola vypočítaná hodnota $R_0$pre jednotlivé merania. 
Pričom pri štatistickom spracovaní boli vynechané hodnoty $R_b>"300 \Omega"$, 
kde sa meranie ukazuje ako veľmi nepresné z dôvodu malého rozdielu prúdov oproti ich hodnote.
Teda bola pomocou vzťahu \ref{SCH_1} a vzťahu \ref{SCH_2} určená hodnota vnútorného odporu ampérmetru 
\eq{
R_0= "\(52.5\pm24.3\) \Omega"\,.
} 

\subsection{Rozšírenie rozsahu voltmetru}

Namerané hodnoty na voltmetri s preradeným odporom (ďalej len predradník), 
sú označené $U$, po prepočte vzťahom \ref{R_10} označené ako $U_1$ a napätie na voltmetri spolu predradníkom  ako $U^\prime$ a vnútorný odpor voltmetru $R_V$.
Tieto hodnoty boli zanesené do tabuľky \ref{T_5}


\begin{table}[h]
\begin{center}
\begin{tabular}{| c | c | c | c | c | }
\hline
 \popi{U}{V} & \popi{U^\prime}{V} &\popi{R}{\Omega} & \popi{U_1}{V}  & \popi{R_V}{\Omega}\\
\hline
$5.2$ & $5.93$ & $0$ & $5.672318$ &  $0$ \\
$4.2$ & $5.93$ & $1000$ & $4.621868$ & $3533.181667$ \\
$3.6$ & $5.93$ & $2000$ & $3.991598$  & $4118.441892$ \\
$3.2$ & $5.93$ & $3000$ & $3.571418$  & $4542.667586$ \\
$2.8$ & $5.93$ & $4000$ & $3.151238$  & $4536.175462$ \\
$2.4$ & $5.93$ & $5000$ & $2.731058$ & $4268.68946$ \\
$2.2$ & $5.93$ & $6000$ & $2.520968$ & $4436.980351$ \\
$2$ & $5.93$ & $7000$ & $2.310878$ &  $4469.63269$ \\
$1.6$ & $5.93$ & $10000$ & $1.890698$  & $4680.754249$ \\
$0.4$ & $5.93$ & $50000$ & $0.630158$  & $5945.064023$ \\
$2.6$ & $5.93$ & $4600$ & $2.941148$ & $4526.581042$\\
\hline

\end{tabular}
\caption{Namerané hodnoty napätie $U$ a $U^\prime$, hodnota odporu predradníku, prepočítané napätie kalibračnou rovnicou pre voltmeter $U_1$ a vypočítaný vnútorný odpor ampérmetru $R_V$} \label{T_5}
\end{center}
\end{table}

Podľa vzťahu \ref{R_6} bola vypočítaná hodnota $R_V$ a z nej podľa vzťahu \ref{SCH_1} a vzťahu \ref{SCH_2} bola určená výsledná hodnota odporu
\eq{
R_V = "4505.8\pm602.1 \Omega"\,.
}


\section{Diskusia}
Pri kalibrácii voltmetru ale i ampérmetru, sa ukázalo ako najväčší nedostatok veľká veľkosť dielika. 
Pri voltmetri $\Delta U = "0.2 V"$ a  $\Delta I = "0.2 mA"$. 
S tým súvisí aj neskorší problém pri rozširovaní, 
rozsahov kde pri zmene odporu nedokázal merák dostatočne presne reagovať 
a ručička sa vychýlila o menej ako pol dielika, čo spôsobovalo veľké nepresnosti.
Tieto nepresnosti sa najviac prejavili u merania ampérmetru kde hodnoty 
pre väčšie hodnoty odporu sa pohybovali ďaleko za hranicou $\sigma_0$.

Samotné vzťahy \ref{R_11} a \ref{R_10} sú zaťažené chybou fitu, ktorá 
sa pohybuje u lineárneho členu v rádoch desatín \% a u absolútneho členu 
v ráde jednotiek \%. Bohužiaľ nebola určená hodnota v 0, čo by fit výrazne spresnilo.

Pri meraní ampérmetru, zároveň vidíme, že v oblasti kde sa $n\doteq2$ sa hodnota $R_0\doteq "70 \Omega"$. 
Čo je podľa môjho subjektívneho názoru správna hodnota, bez zaťažená meraniami v oblasti rádovo rozdielnej od nameranej hodnoty.


%Pri kalibrácii či už voltmetru alebo ampermetru, sa ukazuje ako \uv{magic} pristroj reostat, 
%ktorého sa stačí dotknúť a voltmeter resp. ampermeter ukazuje o rád iné hodnoty.






\section{Záver}
Kalibračná rovnica ampérmetru bola určená ako
\eq{
I_k = "0.95" \cdot I_x + "0.017 mA" \,,
} 
pre voltmeter 
\eq{
U_k = "1.05" \cdot U_x + "0.21 V" \,,
}
a pre odporovú dekádu
\eq{
R_k = "1.00" \cdot R_x +"1.83 \Omega" \,.
}

Vnútorný odpor ampérmetru bol určený ako 
\eq{
R_0= "\(52.5\pm24.3\) \Omega"\,,
} a pre voltmeter
\eq{
R_V = "4505.8\pm602.1 \Omega"\,.
}

\begin{thebibliography}{2}
\bibitem{C_1}
Rozšířená rozsahu miliampármetru a voltmetru,cejchováná kompenzátorem [cit. 21.11.2016]Dostupné po prihlásení z Kurz: Fyzikální praktikum I:\url{https://praktikum.fjfi.cvut.cz/pluginfile.php/119/mod_resource/content/12/161010.pdf}

\end{thebibliography}

\section{Prílohy}






\end{document}






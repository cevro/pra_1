\documentclass[a4paper,10pt]{article}
%\usepackage[IL2]{fontenc}
\usepackage[utf8x]{inputenc}
\usepackage[czech]{babel}
\usepackage{listings}  
\usepackage{amsfonts,amsmath,amssymb,graphicx,color}
%\usepackage[total={17cm,27cm}, top=2cm, left=2cm, includefoot]{geometry}
%\usepackage{fancyhdr}
\usepackage{fkssugar}
\usepackage{hyperref}

%\usepackage{caption}
\renewcommand{\popi}[2]{$\frac{#1}{[\jd{#2}]}$}
\renewcommand{\figurename}{Obr.}
\addto\captionsczech{\renewcommand{\figurename}{Obr.}}
\addto\captionsczech{\renewcommand{\tablename}{Tab.}}

\begin{document}
\def\mean#1{\left< #1 \right>}
\noindent
{\large Fyzikální praktikum 1.} \hfil {\large FJFI ČVUT V Praze}\\
\noindent
{\large\textbf{pracovní úkol \# 2}}
\begin{center}
{\large\textit{Dynamika rotačního pohybu}}
\end{center}
\noindent
\rule{\textwidth}{1px}
\vspace{\baselineskip}

\emph{Michal Červeňák}
\par
\vspace{\baselineskip}
\begin{minipage}[l]{0.5\textwidth}%
\textit{dátum merania:}~16.12. 2016\\%
%\vspace{\baselineskip}%
\par%
\noindent%
\textit{skupina:}~4\\%
%\vspace{\baselineskip}%
\par%
\noindent%
\textit{Klasifikace:}\dotfill\\%
\end{minipage}

\section{Pracovní úkol}

\begin{enumerate}
\item DU: Odvoďte kapacitu C deskového kondenzátoru.
\item DU: Pro deskový kondenzátor stanovte závislost poměru S/d plochy desek S a
vzdálenosti d mezi nimi jako funkci náboje Q a napětí U. Následně spočítejte hodnotu poměru S/d
pro vzduchový deskový kondenzátor s volbou Q = Qmax = 50 µC a U = Umax = 100 kV.
\item Změřte průrazné napětí U mezi deskami kondenzátoru pro deset různých vzdáleností desek
d. Náboj přivádějte až do průrazu mezi deskami kondenzátoru. Průrazné napětí U určete
prostřednictvím silového působení na vahách ve chvíli průrazu a vztahu (13). Z naměřených hodnot
průrazného napětí U pro různé vzdálenosti d určete následně dielektrickou pevnost vzduchu
a porovnejte ji s tabulkovou hodnotou pro suchý vzduch. Diskutujte důvod případné odlišnosti
hodnot.
\item Změřte přitažlivé síly mezi deskami kondenzátoru v závislosti na doskoku jiskřiště s 1 pro tři
různé vzdálenosti desek d. Náboj přivádějte až do průrazu na kulovém jiskřišti s mikrometrickým
šroubem, paralelně připojenému k deskovému kondenzátoru. Ze silového působení spočítejte
napětí a ze vztahu v poznámce se pokuste určit neznámou funkci f(s/D) ze vztahu (14) vzhledem
k podmínce (15) a monoťonnosti funkce u doskoku s. Experimentální data a nalezenou funkci
zpracujte do grafu.
\item Zvolte si dvě konfigurace elektrod, nastavte na nich napětí cca 10 V a zmapujte potenciál v
síti 12 × 12 bodů. V domácím vyhodnocení prověďte grafické zpracovnání naměřených dat.
\end{enumerate}



\section{Pomôcky}
Wimshurstova elektrika, váhy, doskový kondenzátor, kolové iskrisko, podstavec,
sada vodičov, zkratovač, regulovatelný zdroj 12V, souprava pro mapování elektrostatického pole,
voltmetr.

\section{Teória}
Pre výpočet napätia $U$ zo sily $F = m g$, ktorá pôsobí na dosky kondenzátora s plochou $S$ použijeme vzťah
\eq{
U = \sqrt{\frac{2 m g d^2}{\epsilon S}}\,, \lbl{R_1}
}
kde $\epsilon$ je primitivita vákua (vzduchu).

K výpočtu dielektrickej pevnosti vzduchu $E_p$ z napätia medzi doskami a ich vzdialenosťami použijeme vzťah
\eq{
E_p = \frac{V}{d}\,. \lbl{R_2}
}

Na spočítanie prierazu použije vzťah
\eq[m]{
U = 27.75\(1+\frac{0.757}{\sqrt{\gamma D}}\)\frac{\gamma s}{f\(\frac{s}{D}\)}\,,\lbl{R_3}\\
\gamma = \frac{p}{p_a}\cdot \frac{293.16}{273.16+t} \,,
}
pričom $D$ je priemer guľôčky iskriska a $s$ ich vzdialenosť, $p$ je atmosferický tlak a $p_a$ je normálny atmosferický tlak a $t$ je teplota v $[\C]$

\subsubsection{Spracovanie chýb merania}

Označme $\mean{t}$ aritmetický priemer nameraných hodnôt $t_i$, a $\Delta t$ hodnotu $\mean{t}-t$, pričom 
\eq{
\mean{t} = \frac{1}{n}\sum_{i=1}^n t_i \,, \lbl{SCH_1}
}  
a chybu aritmetického priemeru 
\eq{
  \sigma_0=\sqrt{\frac{\sum_{i=1}^n \(t_i - \mean{t}\)^2}{n\(n-1\)}}\,, \lbl{SCH_2}
}
pričom $n$ je počet meraní.

\section{Postup merania}
\begin{enumerate}
\item Najskôr bol odmeraný priemer  doskového kondenzátora.
\item Následne bola pripojená Wimshurstova elektrika ku doskovému kondenzátoru
\item Kondenzáto bol vyskratovaný a bola pomocou šubléry odmeraná vzdialenosť dosiek, a pomocou elektriky nabíjaný až do prierazu 
\item Pri prieraze bola odčítaná hmotnosť.
\item Posledné dva body boli opakované pre rôzne vzdialenosti dosiek kondenzátoru.
\end{enumerate}

\begin{enumerate}
\item K obvodu z predchádzajúceho pokusu bolo pripojené paralelne iskrisko
\item Pomocou šuplery bol odmeraný priemer guličoek iskriska.
\item Kondenzátor bol vyskratovaný a bola pomocou šuplery odmeraná vzdialenosť dosiek, a vzájomná vzdialenosť guľôčok iskriska a pomocou elektriky nabíjaný až do prierazu na iskrisku.
\item Pri prieraze bola odčítaná hmotnosť.
\item Posledné dva body boli opakované pre roznes vzdialenosti dosiek kondenzátoru a vzdialenosti iskriska.
\end{enumerate}

\begin{enumerate}
\item Do Petriho misky bola naliatia destilovaná voda
\item Do Petriho misky boli vložené elektródy.
\item Elektródy boli pripojené k zdroju napätia
\item Na každom z 140 bol zmeraný pomocou voltmetru potenciál voči zemi zdroja.
\item Postup sa opakoval pre oba typy elektród.
\end{enumerate}

\section{Výsledky merania}

Teplota vzduchu bola určená na $t="22.3 \C"$, tlak vzduchu $p = "991 Hpa"$, priemer guľôčok iskriska na $D = "\(1.465\pm0.005\) cm"$. Plocha kondenzátora na $S="0.022 m^2"$

Chyba pri meraní hmotnosti sa pohybuje v okolo $40\%$, hlavne pretože som nestíhal sledovať pristroj aj nabíjať kondenzátor. 

Do tabuľky Tab. \ref{T_1} boli zaznamenané namerané hodnoty vzdialenosti dosiek kondenzátoru a hmotnosť, teda sila, ktorou sa dosky priťahovali.
Z nich boli vypočítané podľa \ref{R_1} hodnoty napätia $U$ a pomocou \ref{R_2} hodnota dielektrickej pevnosti vzduchu $E_p$. 
Pomocou \ref{SCH_1} a \ref{SCH_2} bola určená priemerná hodnota $E_p = "\(1.02\pm0.60\text{st.}\pm0.4\text{sys.}\) MV\cdot m^{-1}"$.

\begin{table}[h]
\begin{center}
\begin{tabular}{| c | c | c | c |}
\hline
\popi{d}{cm} & \popi{m}{g} & \popi{U}{kV} & \popi{E_p}{kV\cdot m^{-1}}\\
\hline
$1.1$ & $38.0$ & $21.59$ & $1962$\\
$2.0$ & $35.0$ & $37.67$ & $1883$\\
$3.0$ & $10.0$ & $30.20$ & $1006$\\
$4.0$ & $6.0$ & $31.19$ &  $779$\\
$4.5$ & $4.5$ & $30.39$ &  $675$\\
$5.0$ & $3.5$ & $29.78$ &  $595$\\
$5.5$ & $2.0$ & $24.76$ &  $450$\\
$6.0$ & $0.5$ & $13.51$ &  $225$\\
$3.5$ & $9.5$ & $34.34$ &  $981$\\
$2.5$ & $25.0$ & $39.79$ & $1591$\\
\hline
\end{tabular}
\caption{
Namerané hodnoty vzdialenosti dosiek $d$ kondenzátoru, hmotnosť $m$,vypočítané hodnoty napätia $U$ a dielektrickej pevnosti vzduchu $E_p$. 
} \label{T_1}
\end{center}
\end{table}


V tabuľke Tab \ref{T_2} sú namerané hodnoty vzdialenosti dosiek $d$, vzdialenosti guľôčok iskriska $s$, a hmotnosť $m$\footnote{Rovnako ako v prvom prípade}.
Ďalej vypočítané hodnoty napätia medzi doskami kondenzátoru $U$ podľa \ref{R_1}, a neznámej funkcie $f\(s/D\)$ podľa \ref{R_3}.

\begin{table}[h]
\begin{center}
\begin{tabular}{| c | c | c | c | c | c |}
\hline
\popi{d}{cm} & \popi{s}{mm} & \popi{m}{g} & \popi{U}{kV} & \popi{\frac{s}{D}}{-} & \popi{f\(\frac{s}{D}\)\cdot 10^{-3}}{-}\\
\hline
$1.50$ & $5.00$ & $70$  & $39.95$ & $0.34$ & $0.025$\\
$1.50$ & $6.00$ & $190$ & $65.82$ & $0.41$ & $0.018$\\
$1.50$ & $7.00$ & $205$ & $68.37$ & $0.48$ & $0.021$\\
$1.50$ & $4.00$ & $60$  & $36.99$ & $0.27$ & $0.022$\\
$1.50$ & $3.00$ & $35$  & $28.25$ & $0.21$ & $0.021$\\
$1.50$ & $5.50$ & $170$ & $62.26$ & $0.38$ & $0.018$\\
$1.50$ & $6.50$ & $270$ & $78.46$ & $0.45$ & $0.017$\\
$1.50$ & $7.50$ & $-$   & $-$     & $0.51$ & $-$\\
$1.50$ & $4.50$ & $85$  & $44.02$ & $0.31$ & $0.021$\\
$1.50$ & $3.50$ & $45$  & $32.03$ & $0.24$ & $0.022$\\
$2.00$ & $5.00$ & $10$  & $20.13$ & $0.34$ & $0.050$\\
$2.00$ & $6.00$ & $55$  & $47.22$ & $0.41$ & $0.023$\\
$2.00$ & $6.50$ & $80$  & $56.95$ & $0.45$ & $0.023$\\
$2.00$ & $7.00$ & $95$  & $62.06$ & $0.48$ & $0.023$\\
$2.00$ & $7.50$ & $110$ & $66.78$ & $0.51$ & $0.023$\\
$2.00$ & $8.00$ & $135$ & $73.98$ & $0.55$ & $0.022$\\
$2.00$ & $8.50$ & $150$ & $77.98$ & $0.58$ & $0.022$\\
$2.00$ & $9.00$ & $175$ & $84.23$ & $0.62$ & $0.021$\\
$2.00$ & $9.50$ & $195$ & $88.91$ & $0.65$ & $0.021$\\
$2.00$ & $10.00$ & $220$& $94.44$ & $0.68$ & $0.021$\\
$1.00$ & $4.00$ & $225$ & $47.75$ & $0.27$ & $0.017$\\
$1.00$ & $3.50$ & $185$ & $43.30$ & $0.24$ & $0.016$\\
$1.00$ & $3.00$ & $170$ & $41.51$ & $0.21$ & $0.015$\\
$1.00$ & $2.50$ & $105$ & $32.62$ & $0.17$ & $0.015$\\
$1.00$ & $2.00$ & $50$  & $22.51$ & $0.14$ & $0.018$\\
$1.00$ & $1.50$ & $5$   & $7.12$ & $0.10$ & $0.042$\\
\hline
\end{tabular}
\caption{
Namerané hodnoty vzdialenosti dosiek $d$, vzdialenosti guľôčok iskriska $s$, a hmotnosť $m$,
vypočítané hodnoty napätia $U$, a neznámej funkcie $f\(s/D\)$.
} \label{T_2}
\end{center}
\end{table}


Na obrázkoch Obr. \ref{G_1} a \ref{G_2} sú vyobrazené 2 rozmerné mapy elektrického pola v okolí elektród, na z-tovú osu je vynesená hodnota napätia $U$ voči zemi zdroja.



\begin{figure}
% GNUPLOT: LaTeX picture
\setlength{\unitlength}{0.240900pt}
\ifx\plotpoint\undefined\newsavebox{\plotpoint}\fi
\begin{picture}(1500,900)(0,0)
\sbox{\plotpoint}{\rule[-0.200pt]{0.400pt}{0.400pt}}%
\multiput(246.00,230.58)(0.534,0.500){501}{\rule{0.527pt}{0.120pt}}
\multiput(246.00,229.17)(267.906,252.000){2}{\rule{0.263pt}{0.400pt}}
\multiput(1240.25,390.58)(-4.029,0.499){181}{\rule{3.313pt}{0.120pt}}
\multiput(1247.12,389.17)(-732.124,92.000){2}{\rule{1.657pt}{0.400pt}}
\put(246.0,230.0){\rule[-0.200pt]{0.400pt}{77.088pt}}
\multiput(246.00,230.58)(0.496,0.492){21}{\rule{0.500pt}{0.119pt}}
\multiput(246.00,229.17)(10.962,12.000){2}{\rule{0.250pt}{0.400pt}}
\put(233,205){\makebox(0,0){ 0}}
\multiput(512.79,480.92)(-0.539,-0.492){21}{\rule{0.533pt}{0.119pt}}
\multiput(513.89,481.17)(-11.893,-12.000){2}{\rule{0.267pt}{0.400pt}}
\multiput(369.00,214.58)(0.539,0.492){21}{\rule{0.533pt}{0.119pt}}
\multiput(369.00,213.17)(11.893,12.000){2}{\rule{0.267pt}{0.400pt}}
\put(356,189){\makebox(0,0){ 2}}
\multiput(635.92,465.92)(-0.496,-0.492){21}{\rule{0.500pt}{0.119pt}}
\multiput(636.96,466.17)(-10.962,-12.000){2}{\rule{0.250pt}{0.400pt}}
\multiput(492.00,199.58)(0.539,0.492){21}{\rule{0.533pt}{0.119pt}}
\multiput(492.00,198.17)(11.893,12.000){2}{\rule{0.267pt}{0.400pt}}
\put(479,174){\makebox(0,0){ 4}}
\multiput(758.92,450.92)(-0.496,-0.492){21}{\rule{0.500pt}{0.119pt}}
\multiput(759.96,451.17)(-10.962,-12.000){2}{\rule{0.250pt}{0.400pt}}
\multiput(616.00,183.58)(0.497,0.493){23}{\rule{0.500pt}{0.119pt}}
\multiput(616.00,182.17)(11.962,13.000){2}{\rule{0.250pt}{0.400pt}}
\put(603,159){\makebox(0,0){ 6}}
\multiput(881.92,435.92)(-0.497,-0.493){23}{\rule{0.500pt}{0.119pt}}
\multiput(882.96,436.17)(-11.962,-13.000){2}{\rule{0.250pt}{0.400pt}}
\multiput(739.00,168.58)(0.496,0.492){21}{\rule{0.500pt}{0.119pt}}
\multiput(739.00,167.17)(10.962,12.000){2}{\rule{0.250pt}{0.400pt}}
\put(726,143){\makebox(0,0){ 8}}
\multiput(1005.79,419.92)(-0.539,-0.492){21}{\rule{0.533pt}{0.119pt}}
\multiput(1006.89,420.17)(-11.893,-12.000){2}{\rule{0.267pt}{0.400pt}}
\multiput(862.00,153.58)(0.496,0.492){21}{\rule{0.500pt}{0.119pt}}
\multiput(862.00,152.17)(10.962,12.000){2}{\rule{0.250pt}{0.400pt}}
\put(849,128){\makebox(0,0){ 10}}
\multiput(1128.79,404.92)(-0.539,-0.492){21}{\rule{0.533pt}{0.119pt}}
\multiput(1129.89,405.17)(-11.893,-12.000){2}{\rule{0.267pt}{0.400pt}}
\multiput(985.00,137.58)(0.539,0.492){21}{\rule{0.533pt}{0.119pt}}
\multiput(985.00,136.17)(11.893,12.000){2}{\rule{0.267pt}{0.400pt}}
\put(972,113){\makebox(0,0){ 12}}
\multiput(1251.92,388.92)(-0.496,-0.492){21}{\rule{0.500pt}{0.119pt}}
\multiput(1252.96,389.17)(-10.962,-12.000){2}{\rule{0.250pt}{0.400pt}}
\multiput(974.07,137.61)(-4.034,0.447){3}{\rule{2.633pt}{0.108pt}}
\multiput(979.53,136.17)(-13.534,3.000){2}{\rule{1.317pt}{0.400pt}}
\put(1004,133){\makebox(0,0)[l]{ 0}}
\multiput(246.00,228.95)(3.811,-0.447){3}{\rule{2.500pt}{0.108pt}}
\multiput(246.00,229.17)(12.811,-3.000){2}{\rule{1.250pt}{0.400pt}}
\put(1011,180.17){\rule{3.900pt}{0.400pt}}
\multiput(1021.91,179.17)(-10.905,2.000){2}{\rule{1.950pt}{0.400pt}}
\put(1049,175){\makebox(0,0)[l]{ 2}}
\multiput(290.00,270.95)(4.034,-0.447){3}{\rule{2.633pt}{0.108pt}}
\multiput(290.00,271.17)(13.534,-3.000){2}{\rule{1.317pt}{0.400pt}}
\put(1056,222.17){\rule{3.900pt}{0.400pt}}
\multiput(1066.91,221.17)(-10.905,2.000){2}{\rule{1.950pt}{0.400pt}}
\put(1094,217){\makebox(0,0)[l]{ 4}}
\put(335,312.17){\rule{3.900pt}{0.400pt}}
\multiput(335.00,313.17)(10.905,-2.000){2}{\rule{1.950pt}{0.400pt}}
\put(1101,264.17){\rule{3.900pt}{0.400pt}}
\multiput(1111.91,263.17)(-10.905,2.000){2}{\rule{1.950pt}{0.400pt}}
\put(1139,259){\makebox(0,0)[l]{ 6}}
\put(380,354.17){\rule{3.900pt}{0.400pt}}
\multiput(380.00,355.17)(10.905,-2.000){2}{\rule{1.950pt}{0.400pt}}
\put(1146,306.17){\rule{3.900pt}{0.400pt}}
\multiput(1156.91,305.17)(-10.905,2.000){2}{\rule{1.950pt}{0.400pt}}
\put(1183,301){\makebox(0,0)[l]{ 8}}
\put(425,396.17){\rule{3.900pt}{0.400pt}}
\multiput(425.00,397.17)(10.905,-2.000){2}{\rule{1.950pt}{0.400pt}}
\multiput(1199.07,348.61)(-4.034,0.447){3}{\rule{2.633pt}{0.108pt}}
\multiput(1204.53,347.17)(-13.534,3.000){2}{\rule{1.317pt}{0.400pt}}
\put(1228,344){\makebox(0,0)[l]{ 10}}
\put(470,438.17){\rule{3.900pt}{0.400pt}}
\multiput(470.00,439.17)(10.905,-2.000){2}{\rule{1.950pt}{0.400pt}}
\multiput(1243.62,390.61)(-3.811,0.447){3}{\rule{2.500pt}{0.108pt}}
\multiput(1248.81,389.17)(-12.811,3.000){2}{\rule{1.250pt}{0.400pt}}
\put(1273,386){\makebox(0,0)[l]{ 12}}
\multiput(515.00,480.95)(4.034,-0.447){3}{\rule{2.633pt}{0.108pt}}
\multiput(515.00,481.17)(13.534,-3.000){2}{\rule{1.317pt}{0.400pt}}
\put(206,230){\makebox(0,0)[r]{ 1}}
\put(246.0,230.0){\rule[-0.200pt]{4.818pt}{0.400pt}}
\put(206,265){\makebox(0,0)[r]{ 2}}
\put(246.0,265.0){\rule[-0.200pt]{4.818pt}{0.400pt}}
\put(206,301){\makebox(0,0)[r]{ 3}}
\put(246.0,301.0){\rule[-0.200pt]{4.818pt}{0.400pt}}
\put(206,337){\makebox(0,0)[r]{ 4}}
\put(246.0,337.0){\rule[-0.200pt]{4.818pt}{0.400pt}}
\put(206,372){\makebox(0,0)[r]{ 5}}
\put(246.0,372.0){\rule[-0.200pt]{4.818pt}{0.400pt}}
\put(206,408){\makebox(0,0)[r]{ 6}}
\put(246.0,408.0){\rule[-0.200pt]{4.818pt}{0.400pt}}
\put(206,444){\makebox(0,0)[r]{ 7}}
\put(246.0,444.0){\rule[-0.200pt]{4.818pt}{0.400pt}}
\put(206,478){\makebox(0,0)[r]{ 8}}
\put(246.0,478.0){\rule[-0.200pt]{4.818pt}{0.400pt}}
\put(206,514){\makebox(0,0)[r]{ 9}}
\put(246.0,514.0){\rule[-0.200pt]{4.818pt}{0.400pt}}
\put(206,550){\makebox(0,0)[r]{ 10}}
\put(246.0,550.0){\rule[-0.200pt]{4.818pt}{0.400pt}}
\put(106,390){\makebox(0,0){\popi{U}{V}}}
\multiput(493.00,475.59)(5.463,0.482){9}{\rule{4.167pt}{0.116pt}}
\multiput(493.00,474.17)(52.352,6.000){2}{\rule{2.083pt}{0.400pt}}
\multiput(554.00,481.59)(4.688,0.485){11}{\rule{3.643pt}{0.117pt}}
\multiput(554.00,480.17)(54.439,7.000){2}{\rule{1.821pt}{0.400pt}}
\multiput(616.00,486.94)(8.962,-0.468){5}{\rule{6.300pt}{0.113pt}}
\multiput(616.00,487.17)(48.924,-4.000){2}{\rule{3.150pt}{0.400pt}}
\multiput(678.00,482.93)(3.994,-0.488){13}{\rule{3.150pt}{0.117pt}}
\multiput(678.00,483.17)(54.462,-8.000){2}{\rule{1.575pt}{0.400pt}}
\multiput(800.00,474.92)(2.260,-0.494){25}{\rule{1.871pt}{0.119pt}}
\multiput(800.00,475.17)(58.116,-14.000){2}{\rule{0.936pt}{0.400pt}}
\multiput(862.00,460.92)(2.901,-0.492){19}{\rule{2.355pt}{0.118pt}}
\multiput(862.00,461.17)(57.113,-11.000){2}{\rule{1.177pt}{0.400pt}}
\multiput(924.00,449.92)(2.070,-0.494){27}{\rule{1.727pt}{0.119pt}}
\multiput(924.00,450.17)(57.416,-15.000){2}{\rule{0.863pt}{0.400pt}}
\multiput(985.00,434.92)(2.105,-0.494){27}{\rule{1.753pt}{0.119pt}}
\multiput(985.00,435.17)(58.361,-15.000){2}{\rule{0.877pt}{0.400pt}}
\multiput(1047.00,419.92)(1.201,-0.497){49}{\rule{1.054pt}{0.120pt}}
\multiput(1047.00,420.17)(59.813,-26.000){2}{\rule{0.527pt}{0.400pt}}
\multiput(1109.00,393.92)(1.717,-0.495){33}{\rule{1.456pt}{0.119pt}}
\multiput(1109.00,394.17)(57.979,-18.000){2}{\rule{0.728pt}{0.400pt}}
\multiput(470.00,462.59)(5.553,0.482){9}{\rule{4.233pt}{0.116pt}}
\multiput(470.00,461.17)(53.214,6.000){2}{\rule{2.117pt}{0.400pt}}
\multiput(532.00,468.58)(2.439,0.493){23}{\rule{2.008pt}{0.119pt}}
\multiput(532.00,467.17)(57.833,13.000){2}{\rule{1.004pt}{0.400pt}}
\put(739.0,476.0){\rule[-0.200pt]{14.695pt}{0.400pt}}
\multiput(655.00,479.93)(4.060,-0.488){13}{\rule{3.200pt}{0.117pt}}
\multiput(655.00,480.17)(55.358,-8.000){2}{\rule{1.600pt}{0.400pt}}
\multiput(717.00,473.61)(13.411,0.447){3}{\rule{8.233pt}{0.108pt}}
\multiput(717.00,472.17)(43.911,3.000){2}{\rule{4.117pt}{0.400pt}}
\multiput(778.00,474.92)(2.223,-0.494){25}{\rule{1.843pt}{0.119pt}}
\multiput(778.00,475.17)(57.175,-14.000){2}{\rule{0.921pt}{0.400pt}}
\multiput(839.00,460.93)(4.060,-0.488){13}{\rule{3.200pt}{0.117pt}}
\multiput(839.00,461.17)(55.358,-8.000){2}{\rule{1.600pt}{0.400pt}}
\multiput(901.00,452.92)(2.901,-0.492){19}{\rule{2.355pt}{0.118pt}}
\multiput(901.00,453.17)(57.113,-11.000){2}{\rule{1.177pt}{0.400pt}}
\multiput(963.00,441.92)(1.717,-0.495){33}{\rule{1.456pt}{0.119pt}}
\multiput(963.00,442.17)(57.979,-18.000){2}{\rule{0.728pt}{0.400pt}}
\multiput(1024.00,423.92)(0.943,-0.497){63}{\rule{0.852pt}{0.120pt}}
\multiput(1024.00,424.17)(60.233,-33.000){2}{\rule{0.426pt}{0.400pt}}
\multiput(1086.00,390.92)(0.973,-0.497){61}{\rule{0.875pt}{0.120pt}}
\multiput(1086.00,391.17)(60.184,-32.000){2}{\rule{0.438pt}{0.400pt}}
\multiput(448.00,448.58)(3.154,0.491){17}{\rule{2.540pt}{0.118pt}}
\multiput(448.00,447.17)(55.728,10.000){2}{\rule{1.270pt}{0.400pt}}
\multiput(509.00,458.58)(1.360,0.496){43}{\rule{1.178pt}{0.120pt}}
\multiput(509.00,457.17)(59.554,23.000){2}{\rule{0.589pt}{0.400pt}}
\multiput(571.00,479.94)(8.962,-0.468){5}{\rule{6.300pt}{0.113pt}}
\multiput(571.00,480.17)(48.924,-4.000){2}{\rule{3.150pt}{0.400pt}}
\multiput(633.00,477.61)(13.411,0.447){3}{\rule{8.233pt}{0.108pt}}
\multiput(633.00,476.17)(43.911,3.000){2}{\rule{4.117pt}{0.400pt}}
\multiput(694.00,480.61)(13.411,0.447){3}{\rule{8.233pt}{0.108pt}}
\multiput(694.00,479.17)(43.911,3.000){2}{\rule{4.117pt}{0.400pt}}
\multiput(755.00,481.92)(2.439,-0.493){23}{\rule{2.008pt}{0.119pt}}
\multiput(755.00,482.17)(57.833,-13.000){2}{\rule{1.004pt}{0.400pt}}
\multiput(817.00,468.92)(2.650,-0.492){21}{\rule{2.167pt}{0.119pt}}
\multiput(817.00,469.17)(57.503,-12.000){2}{\rule{1.083pt}{0.400pt}}
\put(594.0,481.0){\rule[-0.200pt]{14.695pt}{0.400pt}}
\multiput(940.00,456.92)(1.201,-0.497){49}{\rule{1.054pt}{0.120pt}}
\multiput(940.00,457.17)(59.813,-26.000){2}{\rule{0.527pt}{0.400pt}}
\multiput(1002.00,430.92)(0.722,-0.498){83}{\rule{0.677pt}{0.120pt}}
\multiput(1002.00,431.17)(60.595,-43.000){2}{\rule{0.338pt}{0.400pt}}
\multiput(1064.00,387.92)(0.850,-0.498){69}{\rule{0.778pt}{0.120pt}}
\multiput(1064.00,388.17)(59.386,-36.000){2}{\rule{0.389pt}{0.400pt}}
\multiput(425.00,434.58)(2.260,0.494){25}{\rule{1.871pt}{0.119pt}}
\multiput(425.00,433.17)(58.116,14.000){2}{\rule{0.936pt}{0.400pt}}
\multiput(487.00,448.58)(0.915,0.498){65}{\rule{0.829pt}{0.120pt}}
\multiput(487.00,447.17)(60.279,34.000){2}{\rule{0.415pt}{0.400pt}}
\multiput(549.00,480.93)(3.994,-0.488){13}{\rule{3.150pt}{0.117pt}}
\multiput(549.00,481.17)(54.462,-8.000){2}{\rule{1.575pt}{0.400pt}}
\multiput(610.00,474.59)(4.688,0.485){11}{\rule{3.643pt}{0.117pt}}
\multiput(610.00,473.17)(54.439,7.000){2}{\rule{1.821pt}{0.400pt}}
\multiput(672.00,481.58)(2.439,0.493){23}{\rule{2.008pt}{0.119pt}}
\multiput(672.00,480.17)(57.833,13.000){2}{\rule{1.004pt}{0.400pt}}
\multiput(734.00,492.92)(1.077,-0.497){53}{\rule{0.957pt}{0.120pt}}
\multiput(734.00,493.17)(58.013,-28.000){2}{\rule{0.479pt}{0.400pt}}
\multiput(794.00,466.59)(3.582,0.489){15}{\rule{2.856pt}{0.118pt}}
\multiput(794.00,465.17)(56.073,9.000){2}{\rule{1.428pt}{0.400pt}}
\multiput(856.00,473.92)(1.851,-0.495){31}{\rule{1.559pt}{0.119pt}}
\multiput(856.00,474.17)(58.765,-17.000){2}{\rule{0.779pt}{0.400pt}}
\multiput(918.00,456.92)(1.249,-0.497){47}{\rule{1.092pt}{0.120pt}}
\multiput(918.00,457.17)(59.733,-25.000){2}{\rule{0.546pt}{0.400pt}}
\multiput(980.00,431.92)(0.649,-0.498){91}{\rule{0.619pt}{0.120pt}}
\multiput(980.00,432.17)(59.715,-47.000){2}{\rule{0.310pt}{0.400pt}}
\multiput(1041.00,384.92)(0.705,-0.498){85}{\rule{0.664pt}{0.120pt}}
\multiput(1041.00,385.17)(60.623,-44.000){2}{\rule{0.332pt}{0.400pt}}
\multiput(403.00,420.58)(1.821,0.495){31}{\rule{1.535pt}{0.119pt}}
\multiput(403.00,419.17)(57.813,17.000){2}{\rule{0.768pt}{0.400pt}}
\multiput(464.00,437.58)(0.633,0.498){95}{\rule{0.606pt}{0.120pt}}
\multiput(464.00,436.17)(60.742,49.000){2}{\rule{0.303pt}{0.400pt}}
\multiput(526.00,484.92)(2.105,-0.494){27}{\rule{1.753pt}{0.119pt}}
\multiput(526.00,485.17)(58.361,-15.000){2}{\rule{0.877pt}{0.400pt}}
\multiput(588.00,471.58)(1.851,0.495){31}{\rule{1.559pt}{0.119pt}}
\multiput(588.00,470.17)(58.765,17.000){2}{\rule{0.779pt}{0.400pt}}
\multiput(650.00,488.58)(1.229,0.497){47}{\rule{1.076pt}{0.120pt}}
\multiput(650.00,487.17)(58.767,25.000){2}{\rule{0.538pt}{0.400pt}}
\multiput(711.00,511.92)(0.826,-0.498){71}{\rule{0.759pt}{0.120pt}}
\multiput(711.00,512.17)(59.424,-37.000){2}{\rule{0.380pt}{0.400pt}}
\multiput(772.00,476.58)(2.260,0.494){25}{\rule{1.871pt}{0.119pt}}
\multiput(772.00,475.17)(58.116,14.000){2}{\rule{0.936pt}{0.400pt}}
\multiput(834.00,488.92)(2.854,-0.492){19}{\rule{2.318pt}{0.118pt}}
\multiput(834.00,489.17)(56.188,-11.000){2}{\rule{1.159pt}{0.400pt}}
\multiput(895.00,477.92)(0.674,-0.498){89}{\rule{0.639pt}{0.120pt}}
\multiput(895.00,478.17)(60.673,-46.000){2}{\rule{0.320pt}{0.400pt}}
\multiput(957.00,431.92)(0.608,-0.498){99}{\rule{0.586pt}{0.120pt}}
\multiput(957.00,432.17)(60.783,-51.000){2}{\rule{0.293pt}{0.400pt}}
\multiput(1019.00,380.92)(0.663,-0.498){89}{\rule{0.630pt}{0.120pt}}
\multiput(1019.00,381.17)(59.692,-46.000){2}{\rule{0.315pt}{0.400pt}}
\multiput(380.00,402.58)(1.249,0.497){47}{\rule{1.092pt}{0.120pt}}
\multiput(380.00,401.17)(59.733,25.000){2}{\rule{0.546pt}{0.400pt}}
\multiput(442.00,427.58)(1.005,0.497){59}{\rule{0.900pt}{0.120pt}}
\multiput(442.00,426.17)(60.132,31.000){2}{\rule{0.450pt}{0.400pt}}
\multiput(504.00,458.58)(1.542,0.496){37}{\rule{1.320pt}{0.119pt}}
\multiput(504.00,457.17)(58.260,20.000){2}{\rule{0.660pt}{0.400pt}}
\multiput(565.00,478.58)(0.620,0.498){97}{\rule{0.596pt}{0.120pt}}
\multiput(565.00,477.17)(60.763,50.000){2}{\rule{0.298pt}{0.400pt}}
\multiput(627.58,528.00)(0.499,0.889){121}{\rule{0.120pt}{0.810pt}}
\multiput(626.17,528.00)(62.000,108.319){2}{\rule{0.400pt}{0.405pt}}
\multiput(689.58,633.77)(0.499,-1.151){119}{\rule{0.120pt}{1.018pt}}
\multiput(688.17,635.89)(61.000,-137.887){2}{\rule{0.400pt}{0.509pt}}
\multiput(750.00,498.59)(4.612,0.485){11}{\rule{3.586pt}{0.117pt}}
\multiput(750.00,497.17)(53.558,7.000){2}{\rule{1.793pt}{0.400pt}}
\multiput(811.00,503.92)(0.797,-0.498){75}{\rule{0.736pt}{0.120pt}}
\multiput(811.00,504.17)(60.473,-39.000){2}{\rule{0.368pt}{0.400pt}}
\multiput(873.00,464.92)(0.722,-0.498){83}{\rule{0.677pt}{0.120pt}}
\multiput(873.00,465.17)(60.595,-43.000){2}{\rule{0.338pt}{0.400pt}}
\multiput(935.00,421.92)(0.554,-0.499){107}{\rule{0.544pt}{0.120pt}}
\multiput(935.00,422.17)(59.872,-55.000){2}{\rule{0.272pt}{0.400pt}}
\multiput(996.00,366.92)(0.797,-0.498){75}{\rule{0.736pt}{0.120pt}}
\multiput(996.00,367.17)(60.473,-39.000){2}{\rule{0.368pt}{0.400pt}}
\multiput(358.00,392.58)(1.851,0.495){31}{\rule{1.559pt}{0.119pt}}
\multiput(358.00,391.17)(58.765,17.000){2}{\rule{0.779pt}{0.400pt}}
\multiput(420.00,409.58)(0.710,0.498){83}{\rule{0.667pt}{0.120pt}}
\multiput(420.00,408.17)(59.615,43.000){2}{\rule{0.334pt}{0.400pt}}
\multiput(481.00,452.58)(1.038,0.497){57}{\rule{0.927pt}{0.120pt}}
\multiput(481.00,451.17)(60.077,30.000){2}{\rule{0.463pt}{0.400pt}}
\multiput(543.00,482.58)(0.889,0.498){67}{\rule{0.809pt}{0.120pt}}
\multiput(543.00,481.17)(60.322,35.000){2}{\rule{0.404pt}{0.400pt}}
\multiput(605.58,517.00)(0.499,0.821){119}{\rule{0.120pt}{0.756pt}}
\multiput(604.17,517.00)(61.000,98.431){2}{\rule{0.400pt}{0.378pt}}
\multiput(666.58,613.80)(0.499,-0.840){121}{\rule{0.120pt}{0.771pt}}
\multiput(665.17,615.40)(62.000,-102.400){2}{\rule{0.400pt}{0.385pt}}
\multiput(728.00,511.92)(0.710,-0.498){83}{\rule{0.667pt}{0.120pt}}
\multiput(728.00,512.17)(59.615,-43.000){2}{\rule{0.334pt}{0.400pt}}
\multiput(789.00,468.92)(0.957,-0.497){61}{\rule{0.863pt}{0.120pt}}
\multiput(789.00,469.17)(59.210,-32.000){2}{\rule{0.431pt}{0.400pt}}
\multiput(850.00,436.92)(0.776,-0.498){77}{\rule{0.720pt}{0.120pt}}
\multiput(850.00,437.17)(60.506,-40.000){2}{\rule{0.360pt}{0.400pt}}
\multiput(912.00,396.92)(0.660,-0.498){91}{\rule{0.628pt}{0.120pt}}
\multiput(912.00,397.17)(60.697,-47.000){2}{\rule{0.314pt}{0.400pt}}
\multiput(974.00,349.92)(0.660,-0.498){91}{\rule{0.628pt}{0.120pt}}
\multiput(974.00,350.17)(60.697,-47.000){2}{\rule{0.314pt}{0.400pt}}
\multiput(335.00,360.58)(1.249,0.497){47}{\rule{1.092pt}{0.120pt}}
\multiput(335.00,359.17)(59.733,25.000){2}{\rule{0.546pt}{0.400pt}}
\multiput(397.00,385.58)(0.818,0.498){73}{\rule{0.753pt}{0.120pt}}
\multiput(397.00,384.17)(60.438,38.000){2}{\rule{0.376pt}{0.400pt}}
\multiput(459.00,423.58)(1.229,0.497){47}{\rule{1.076pt}{0.120pt}}
\multiput(459.00,422.17)(58.767,25.000){2}{\rule{0.538pt}{0.400pt}}
\multiput(520.00,448.58)(1.155,0.497){51}{\rule{1.019pt}{0.120pt}}
\multiput(520.00,447.17)(59.886,27.000){2}{\rule{0.509pt}{0.400pt}}
\multiput(582.00,475.58)(1.302,0.496){45}{\rule{1.133pt}{0.120pt}}
\multiput(582.00,474.17)(59.648,24.000){2}{\rule{0.567pt}{0.400pt}}
\multiput(644.00,497.92)(0.739,-0.498){81}{\rule{0.690pt}{0.120pt}}
\multiput(644.00,498.17)(60.567,-42.000){2}{\rule{0.345pt}{0.400pt}}
\multiput(706.00,455.92)(0.751,-0.498){77}{\rule{0.700pt}{0.120pt}}
\multiput(706.00,456.17)(58.547,-40.000){2}{\rule{0.350pt}{0.400pt}}
\multiput(766.00,415.92)(1.423,-0.496){41}{\rule{1.227pt}{0.120pt}}
\multiput(766.00,416.17)(59.453,-22.000){2}{\rule{0.614pt}{0.400pt}}
\multiput(828.00,393.92)(0.864,-0.498){69}{\rule{0.789pt}{0.120pt}}
\multiput(828.00,394.17)(60.363,-36.000){2}{\rule{0.394pt}{0.400pt}}
\multiput(890.00,357.92)(0.850,-0.498){69}{\rule{0.778pt}{0.120pt}}
\multiput(890.00,358.17)(59.386,-36.000){2}{\rule{0.389pt}{0.400pt}}
\multiput(951.00,321.92)(0.705,-0.498){85}{\rule{0.664pt}{0.120pt}}
\multiput(951.00,322.17)(60.623,-44.000){2}{\rule{0.332pt}{0.400pt}}
\multiput(313.00,336.58)(1.568,0.496){37}{\rule{1.340pt}{0.119pt}}
\multiput(313.00,335.17)(59.219,20.000){2}{\rule{0.670pt}{0.400pt}}
\multiput(375.00,356.58)(1.096,0.497){53}{\rule{0.971pt}{0.120pt}}
\multiput(375.00,355.17)(58.984,28.000){2}{\rule{0.486pt}{0.400pt}}
\multiput(436.00,384.58)(1.492,0.496){39}{\rule{1.281pt}{0.119pt}}
\multiput(436.00,383.17)(59.341,21.000){2}{\rule{0.640pt}{0.400pt}}
\multiput(498.00,405.58)(1.492,0.496){39}{\rule{1.281pt}{0.119pt}}
\multiput(498.00,404.17)(59.341,21.000){2}{\rule{0.640pt}{0.400pt}}
\multiput(560.00,426.59)(4.612,0.485){11}{\rule{3.586pt}{0.117pt}}
\multiput(560.00,425.17)(53.558,7.000){2}{\rule{1.793pt}{0.400pt}}
\multiput(621.00,431.92)(0.943,-0.497){63}{\rule{0.852pt}{0.120pt}}
\multiput(621.00,432.17)(60.233,-33.000){2}{\rule{0.426pt}{0.400pt}}
\multiput(683.00,398.92)(1.075,-0.497){55}{\rule{0.955pt}{0.120pt}}
\multiput(683.00,399.17)(60.017,-29.000){2}{\rule{0.478pt}{0.400pt}}
\multiput(745.00,369.92)(1.717,-0.495){33}{\rule{1.456pt}{0.119pt}}
\multiput(745.00,370.17)(57.979,-18.000){2}{\rule{0.728pt}{0.400pt}}
\multiput(806.00,351.92)(0.826,-0.498){71}{\rule{0.759pt}{0.120pt}}
\multiput(806.00,352.17)(59.424,-37.000){2}{\rule{0.380pt}{0.400pt}}
\multiput(867.00,314.92)(1.249,-0.497){47}{\rule{1.092pt}{0.120pt}}
\multiput(867.00,315.17)(59.733,-25.000){2}{\rule{0.546pt}{0.400pt}}
\multiput(929.00,289.92)(0.776,-0.498){77}{\rule{0.720pt}{0.120pt}}
\multiput(929.00,290.17)(60.506,-40.000){2}{\rule{0.360pt}{0.400pt}}
\multiput(290.00,311.58)(3.206,0.491){17}{\rule{2.580pt}{0.118pt}}
\multiput(290.00,310.17)(56.645,10.000){2}{\rule{1.290pt}{0.400pt}}
\multiput(352.00,321.58)(1.249,0.497){47}{\rule{1.092pt}{0.120pt}}
\multiput(352.00,320.17)(59.733,25.000){2}{\rule{0.546pt}{0.400pt}}
\multiput(414.00,346.58)(2.439,0.493){23}{\rule{2.008pt}{0.119pt}}
\multiput(414.00,345.17)(57.833,13.000){2}{\rule{1.004pt}{0.400pt}}
\multiput(476.00,359.58)(2.223,0.494){25}{\rule{1.843pt}{0.119pt}}
\multiput(476.00,358.17)(57.175,14.000){2}{\rule{0.921pt}{0.400pt}}
\multiput(537.00,373.61)(13.635,0.447){3}{\rule{8.367pt}{0.108pt}}
\multiput(537.00,372.17)(44.635,3.000){2}{\rule{4.183pt}{0.400pt}}
\multiput(599.00,374.92)(1.075,-0.497){55}{\rule{0.955pt}{0.120pt}}
\multiput(599.00,375.17)(60.017,-29.000){2}{\rule{0.478pt}{0.400pt}}
\multiput(661.00,345.92)(1.181,-0.497){49}{\rule{1.038pt}{0.120pt}}
\multiput(661.00,346.17)(58.845,-26.000){2}{\rule{0.519pt}{0.400pt}}
\multiput(722.00,319.92)(2.223,-0.494){25}{\rule{1.843pt}{0.119pt}}
\multiput(722.00,320.17)(57.175,-14.000){2}{\rule{0.921pt}{0.400pt}}
\multiput(783.00,305.92)(0.943,-0.497){63}{\rule{0.852pt}{0.120pt}}
\multiput(783.00,306.17)(60.233,-33.000){2}{\rule{0.426pt}{0.400pt}}
\multiput(845.00,272.92)(1.400,-0.496){41}{\rule{1.209pt}{0.120pt}}
\multiput(845.00,273.17)(58.490,-22.000){2}{\rule{0.605pt}{0.400pt}}
\multiput(906.00,250.92)(1.075,-0.497){55}{\rule{0.955pt}{0.120pt}}
\multiput(906.00,251.17)(60.017,-29.000){2}{\rule{0.478pt}{0.400pt}}
\multiput(268.00,283.59)(5.553,0.482){9}{\rule{4.233pt}{0.116pt}}
\multiput(268.00,282.17)(53.214,6.000){2}{\rule{2.117pt}{0.400pt}}
\multiput(330.00,289.58)(2.223,0.494){25}{\rule{1.843pt}{0.119pt}}
\multiput(330.00,288.17)(57.175,14.000){2}{\rule{0.921pt}{0.400pt}}
\multiput(391.00,303.58)(2.260,0.494){25}{\rule{1.871pt}{0.119pt}}
\multiput(391.00,302.17)(58.116,14.000){2}{\rule{0.936pt}{0.400pt}}
\multiput(453.00,317.59)(5.553,0.482){9}{\rule{4.233pt}{0.116pt}}
\multiput(453.00,316.17)(53.214,6.000){2}{\rule{2.117pt}{0.400pt}}
\multiput(515.00,321.94)(8.816,-0.468){5}{\rule{6.200pt}{0.113pt}}
\multiput(515.00,322.17)(48.132,-4.000){2}{\rule{3.100pt}{0.400pt}}
\multiput(576.00,317.92)(1.423,-0.496){41}{\rule{1.227pt}{0.120pt}}
\multiput(576.00,318.17)(59.453,-22.000){2}{\rule{0.614pt}{0.400pt}}
\multiput(638.00,295.92)(1.423,-0.496){41}{\rule{1.227pt}{0.120pt}}
\multiput(638.00,296.17)(59.453,-22.000){2}{\rule{0.614pt}{0.400pt}}
\multiput(700.00,273.92)(2.223,-0.494){25}{\rule{1.843pt}{0.119pt}}
\multiput(700.00,274.17)(57.175,-14.000){2}{\rule{0.921pt}{0.400pt}}
\multiput(761.00,259.92)(0.928,-0.497){63}{\rule{0.839pt}{0.120pt}}
\multiput(761.00,260.17)(59.258,-33.000){2}{\rule{0.420pt}{0.400pt}}
\multiput(822.00,226.92)(2.105,-0.494){27}{\rule{1.753pt}{0.119pt}}
\multiput(822.00,227.17)(58.361,-15.000){2}{\rule{0.877pt}{0.400pt}}
\multiput(884.00,211.92)(1.075,-0.497){55}{\rule{0.955pt}{0.120pt}}
\multiput(884.00,212.17)(60.017,-29.000){2}{\rule{0.478pt}{0.400pt}}
\put(246,249.67){\rule{14.695pt}{0.400pt}}
\multiput(246.00,250.17)(30.500,-1.000){2}{\rule{7.347pt}{0.400pt}}
\multiput(307.00,250.58)(1.492,0.496){39}{\rule{1.281pt}{0.119pt}}
\multiput(307.00,249.17)(59.341,21.000){2}{\rule{0.640pt}{0.400pt}}
\multiput(369.00,269.94)(8.962,-0.468){5}{\rule{6.300pt}{0.113pt}}
\multiput(369.00,270.17)(48.924,-4.000){2}{\rule{3.150pt}{0.400pt}}
\multiput(431.00,267.59)(4.612,0.485){11}{\rule{3.586pt}{0.117pt}}
\multiput(431.00,266.17)(53.558,7.000){2}{\rule{1.793pt}{0.400pt}}
\multiput(492.00,272.92)(2.650,-0.492){21}{\rule{2.167pt}{0.119pt}}
\multiput(492.00,273.17)(57.503,-12.000){2}{\rule{1.083pt}{0.400pt}}
\multiput(554.00,260.92)(2.260,-0.494){25}{\rule{1.871pt}{0.119pt}}
\multiput(554.00,261.17)(58.116,-14.000){2}{\rule{0.936pt}{0.400pt}}
\multiput(616.00,246.92)(1.400,-0.496){41}{\rule{1.209pt}{0.120pt}}
\multiput(616.00,247.17)(58.490,-22.000){2}{\rule{0.605pt}{0.400pt}}
\multiput(677.00,224.92)(2.105,-0.494){27}{\rule{1.753pt}{0.119pt}}
\multiput(677.00,225.17)(58.361,-15.000){2}{\rule{0.877pt}{0.400pt}}
\multiput(739.00,209.92)(1.717,-0.495){33}{\rule{1.456pt}{0.119pt}}
\multiput(739.00,210.17)(57.979,-18.000){2}{\rule{0.728pt}{0.400pt}}
\multiput(800.00,191.92)(1.652,-0.495){35}{\rule{1.405pt}{0.119pt}}
\multiput(800.00,192.17)(59.083,-19.000){2}{\rule{0.703pt}{0.400pt}}
\multiput(862.00,172.92)(1.229,-0.497){47}{\rule{1.076pt}{0.120pt}}
\multiput(862.00,173.17)(58.767,-25.000){2}{\rule{0.538pt}{0.400pt}}
\multiput(1167.44,375.92)(-0.648,-0.495){31}{\rule{0.618pt}{0.119pt}}
\multiput(1168.72,376.17)(-20.718,-17.000){2}{\rule{0.309pt}{0.400pt}}
\multiput(1142.13,358.93)(-1.713,-0.485){11}{\rule{1.414pt}{0.117pt}}
\multiput(1145.06,359.17)(-20.065,-7.000){2}{\rule{0.707pt}{0.400pt}}
\multiput(1121.26,351.92)(-1.015,-0.492){19}{\rule{0.900pt}{0.118pt}}
\multiput(1123.13,352.17)(-20.132,-11.000){2}{\rule{0.450pt}{0.400pt}}
\multiput(1096.22,340.93)(-2.027,-0.482){9}{\rule{1.633pt}{0.116pt}}
\multiput(1099.61,341.17)(-19.610,-6.000){2}{\rule{0.817pt}{0.400pt}}
\multiput(1074.37,334.93)(-1.637,-0.485){11}{\rule{1.357pt}{0.117pt}}
\multiput(1077.18,335.17)(-19.183,-7.000){2}{\rule{0.679pt}{0.400pt}}
\multiput(1056.92,326.70)(-0.496,-0.567){41}{\rule{0.120pt}{0.555pt}}
\multiput(1057.17,327.85)(-22.000,-23.849){2}{\rule{0.400pt}{0.277pt}}
\multiput(1034.92,301.78)(-0.496,-0.542){43}{\rule{0.120pt}{0.535pt}}
\multiput(1035.17,302.89)(-23.000,-23.890){2}{\rule{0.400pt}{0.267pt}}
\multiput(1011.92,276.47)(-0.496,-0.637){41}{\rule{0.120pt}{0.609pt}}
\multiput(1012.17,277.74)(-22.000,-26.736){2}{\rule{0.400pt}{0.305pt}}
\multiput(989.92,248.56)(-0.496,-0.609){43}{\rule{0.120pt}{0.587pt}}
\multiput(990.17,249.78)(-23.000,-26.782){2}{\rule{0.400pt}{0.293pt}}
\multiput(966.92,219.64)(-0.496,-0.891){41}{\rule{0.120pt}{0.809pt}}
\multiput(967.17,221.32)(-22.000,-37.321){2}{\rule{0.400pt}{0.405pt}}
\multiput(944.92,181.06)(-0.496,-0.763){43}{\rule{0.120pt}{0.709pt}}
\multiput(945.17,182.53)(-23.000,-33.529){2}{\rule{0.400pt}{0.354pt}}
\multiput(1095.85,393.95)(-4.927,-0.447){3}{\rule{3.167pt}{0.108pt}}
\multiput(1102.43,394.17)(-16.427,-3.000){2}{\rule{1.583pt}{0.400pt}}
\multiput(1073.41,390.95)(-4.704,-0.447){3}{\rule{3.033pt}{0.108pt}}
\multiput(1079.70,391.17)(-15.704,-3.000){2}{\rule{1.517pt}{0.400pt}}
\multiput(1050.85,387.95)(-4.927,-0.447){3}{\rule{3.167pt}{0.108pt}}
\multiput(1057.43,388.17)(-16.427,-3.000){2}{\rule{1.583pt}{0.400pt}}
\multiput(1031.45,384.94)(-3.113,-0.468){5}{\rule{2.300pt}{0.113pt}}
\multiput(1036.23,385.17)(-17.226,-4.000){2}{\rule{1.150pt}{0.400pt}}
\multiput(1015.86,380.92)(-0.827,-0.494){25}{\rule{0.757pt}{0.119pt}}
\multiput(1017.43,381.17)(-21.429,-14.000){2}{\rule{0.379pt}{0.400pt}}
\multiput(993.44,366.92)(-0.648,-0.495){31}{\rule{0.618pt}{0.119pt}}
\multiput(994.72,367.17)(-20.718,-17.000){2}{\rule{0.309pt}{0.400pt}}
\multiput(972.92,348.56)(-0.496,-0.609){43}{\rule{0.120pt}{0.587pt}}
\multiput(973.17,349.78)(-23.000,-26.782){2}{\rule{0.400pt}{0.293pt}}
\multiput(949.92,320.17)(-0.496,-0.729){41}{\rule{0.120pt}{0.682pt}}
\multiput(950.17,321.58)(-22.000,-30.585){2}{\rule{0.400pt}{0.341pt}}
\multiput(927.92,287.77)(-0.496,-0.852){43}{\rule{0.120pt}{0.778pt}}
\multiput(928.17,289.38)(-23.000,-37.385){2}{\rule{0.400pt}{0.389pt}}
\multiput(904.92,248.64)(-0.496,-0.891){41}{\rule{0.120pt}{0.809pt}}
\multiput(905.17,250.32)(-22.000,-37.321){2}{\rule{0.400pt}{0.405pt}}
\multiput(882.92,209.64)(-0.496,-0.891){41}{\rule{0.120pt}{0.809pt}}
\multiput(883.17,211.32)(-22.000,-37.321){2}{\rule{0.400pt}{0.405pt}}
\multiput(1037.04,421.60)(-3.259,0.468){5}{\rule{2.400pt}{0.113pt}}
\multiput(1042.02,420.17)(-18.019,4.000){2}{\rule{1.200pt}{0.400pt}}
\multiput(1018.37,425.59)(-1.637,0.485){11}{\rule{1.357pt}{0.117pt}}
\multiput(1021.18,424.17)(-19.183,7.000){2}{\rule{0.679pt}{0.400pt}}
\put(980,431.67){\rule{5.300pt}{0.400pt}}
\multiput(991.00,431.17)(-11.000,1.000){2}{\rule{2.650pt}{0.400pt}}
\put(879.0,458.0){\rule[-0.200pt]{14.695pt}{0.400pt}}
\multiput(952.93,431.92)(-1.121,-0.491){17}{\rule{0.980pt}{0.118pt}}
\multiput(954.97,432.17)(-19.966,-10.000){2}{\rule{0.490pt}{0.400pt}}
\multiput(933.92,420.78)(-0.496,-0.542){43}{\rule{0.120pt}{0.535pt}}
\multiput(934.17,421.89)(-23.000,-23.890){2}{\rule{0.400pt}{0.267pt}}
\multiput(910.92,394.64)(-0.496,-0.891){41}{\rule{0.120pt}{0.809pt}}
\multiput(911.17,396.32)(-22.000,-37.321){2}{\rule{0.400pt}{0.405pt}}
\multiput(888.92,355.48)(-0.496,-0.940){43}{\rule{0.120pt}{0.848pt}}
\multiput(889.17,357.24)(-23.000,-41.240){2}{\rule{0.400pt}{0.424pt}}
\multiput(865.92,312.41)(-0.496,-0.960){41}{\rule{0.120pt}{0.864pt}}
\multiput(866.17,314.21)(-22.000,-40.207){2}{\rule{0.400pt}{0.432pt}}
\multiput(843.92,270.26)(-0.496,-1.006){43}{\rule{0.120pt}{0.900pt}}
\multiput(844.17,272.13)(-23.000,-44.132){2}{\rule{0.400pt}{0.450pt}}
\multiput(820.92,224.94)(-0.496,-0.799){41}{\rule{0.120pt}{0.736pt}}
\multiput(821.17,226.47)(-22.000,-33.472){2}{\rule{0.400pt}{0.368pt}}
\multiput(979.37,436.59)(-1.637,0.485){11}{\rule{1.357pt}{0.117pt}}
\multiput(982.18,435.17)(-19.183,7.000){2}{\rule{0.679pt}{0.400pt}}
\multiput(960.04,443.58)(-0.771,0.494){27}{\rule{0.713pt}{0.119pt}}
\multiput(961.52,442.17)(-21.519,15.000){2}{\rule{0.357pt}{0.400pt}}
\put(957.0,433.0){\rule[-0.200pt]{5.541pt}{0.400pt}}
\multiput(915.77,458.58)(-0.546,0.496){39}{\rule{0.538pt}{0.119pt}}
\multiput(916.88,457.17)(-21.883,21.000){2}{\rule{0.269pt}{0.400pt}}
\multiput(891.77,477.92)(-0.853,-0.493){23}{\rule{0.777pt}{0.119pt}}
\multiput(893.39,478.17)(-20.387,-13.000){2}{\rule{0.388pt}{0.400pt}}
\multiput(871.92,463.56)(-0.496,-0.609){43}{\rule{0.120pt}{0.587pt}}
\multiput(872.17,464.78)(-23.000,-26.782){2}{\rule{0.400pt}{0.293pt}}
\multiput(848.92,434.34)(-0.496,-0.983){41}{\rule{0.120pt}{0.882pt}}
\multiput(849.17,436.17)(-22.000,-41.170){2}{\rule{0.400pt}{0.441pt}}
\multiput(826.92,391.41)(-0.496,-0.960){41}{\rule{0.120pt}{0.864pt}}
\multiput(827.17,393.21)(-22.000,-40.207){2}{\rule{0.400pt}{0.432pt}}
\multiput(804.92,349.26)(-0.496,-1.006){43}{\rule{0.120pt}{0.900pt}}
\multiput(805.17,351.13)(-23.000,-44.132){2}{\rule{0.400pt}{0.450pt}}
\multiput(781.92,303.11)(-0.496,-1.053){41}{\rule{0.120pt}{0.936pt}}
\multiput(782.17,305.06)(-22.000,-44.057){2}{\rule{0.400pt}{0.468pt}}
\multiput(759.92,256.81)(-0.496,-1.145){41}{\rule{0.120pt}{1.009pt}}
\multiput(760.17,258.91)(-22.000,-47.906){2}{\rule{0.400pt}{0.505pt}}
\multiput(910.85,451.61)(-4.927,0.447){3}{\rule{3.167pt}{0.108pt}}
\multiput(917.43,450.17)(-16.427,3.000){2}{\rule{1.583pt}{0.400pt}}
\multiput(891.45,454.60)(-3.113,0.468){5}{\rule{2.300pt}{0.113pt}}
\multiput(896.23,453.17)(-17.226,4.000){2}{\rule{1.150pt}{0.400pt}}
\multiput(876.34,458.58)(-0.678,0.495){31}{\rule{0.641pt}{0.119pt}}
\multiput(877.67,457.17)(-21.669,17.000){2}{\rule{0.321pt}{0.400pt}}
\multiput(853.15,475.58)(-0.737,0.494){27}{\rule{0.687pt}{0.119pt}}
\multiput(854.57,474.17)(-20.575,15.000){2}{\rule{0.343pt}{0.400pt}}
\multiput(831.04,490.58)(-0.771,0.494){27}{\rule{0.713pt}{0.119pt}}
\multiput(832.52,489.17)(-21.519,15.000){2}{\rule{0.357pt}{0.400pt}}
\multiput(809.92,501.94)(-0.496,-0.799){41}{\rule{0.120pt}{0.736pt}}
\multiput(810.17,503.47)(-22.000,-33.472){2}{\rule{0.400pt}{0.368pt}}
\multiput(787.92,465.76)(-0.496,-1.161){43}{\rule{0.120pt}{1.022pt}}
\multiput(788.17,467.88)(-23.000,-50.879){2}{\rule{0.400pt}{0.511pt}}
\multiput(764.92,412.95)(-0.496,-1.104){39}{\rule{0.119pt}{0.976pt}}
\multiput(765.17,414.97)(-21.000,-43.974){2}{\rule{0.400pt}{0.488pt}}
\multiput(743.92,366.98)(-0.496,-1.095){43}{\rule{0.120pt}{0.970pt}}
\multiput(744.17,368.99)(-23.000,-47.988){2}{\rule{0.400pt}{0.485pt}}
\multiput(720.92,317.11)(-0.496,-1.053){41}{\rule{0.120pt}{0.936pt}}
\multiput(721.17,319.06)(-22.000,-44.057){2}{\rule{0.400pt}{0.468pt}}
\multiput(698.92,271.05)(-0.496,-1.073){43}{\rule{0.120pt}{0.952pt}}
\multiput(699.17,273.02)(-23.000,-47.024){2}{\rule{0.400pt}{0.476pt}}
\put(918.0,458.0){\rule[-0.200pt]{5.300pt}{0.400pt}}
\multiput(834.02,462.59)(-1.418,0.488){13}{\rule{1.200pt}{0.117pt}}
\multiput(836.51,461.17)(-19.509,8.000){2}{\rule{0.600pt}{0.400pt}}
\multiput(807.04,468.94)(-3.259,-0.468){5}{\rule{2.400pt}{0.113pt}}
\multiput(812.02,469.17)(-18.019,-4.000){2}{\rule{1.200pt}{0.400pt}}
\multiput(789.93,466.58)(-1.121,0.491){17}{\rule{0.980pt}{0.118pt}}
\multiput(791.97,465.17)(-19.966,10.000){2}{\rule{0.490pt}{0.400pt}}
\multiput(769.92,476.58)(-0.498,0.496){41}{\rule{0.500pt}{0.120pt}}
\multiput(770.96,475.17)(-20.962,22.000){2}{\rule{0.250pt}{0.400pt}}
\multiput(747.15,498.58)(-0.737,0.494){27}{\rule{0.687pt}{0.119pt}}
\multiput(748.57,497.17)(-20.575,15.000){2}{\rule{0.343pt}{0.400pt}}
\multiput(726.92,508.36)(-0.496,-1.284){41}{\rule{0.120pt}{1.118pt}}
\multiput(727.17,510.68)(-22.000,-53.679){2}{\rule{0.400pt}{0.559pt}}
\multiput(704.92,452.47)(-0.496,-1.249){43}{\rule{0.120pt}{1.091pt}}
\multiput(705.17,454.73)(-23.000,-54.735){2}{\rule{0.400pt}{0.546pt}}
\multiput(681.92,395.58)(-0.496,-1.215){41}{\rule{0.120pt}{1.064pt}}
\multiput(682.17,397.79)(-22.000,-50.792){2}{\rule{0.400pt}{0.532pt}}
\multiput(659.92,342.98)(-0.496,-1.095){43}{\rule{0.120pt}{0.970pt}}
\multiput(660.17,344.99)(-23.000,-47.988){2}{\rule{0.400pt}{0.485pt}}
\multiput(636.92,292.89)(-0.496,-1.122){41}{\rule{0.120pt}{0.991pt}}
\multiput(637.17,294.94)(-22.000,-46.943){2}{\rule{0.400pt}{0.495pt}}
\put(839.0,462.0){\rule[-0.200pt]{5.541pt}{0.400pt}}
\multiput(772.13,476.59)(-1.713,0.485){11}{\rule{1.414pt}{0.117pt}}
\multiput(775.06,475.17)(-20.065,7.000){2}{\rule{0.707pt}{0.400pt}}
\multiput(751.41,483.58)(-0.967,0.492){19}{\rule{0.864pt}{0.118pt}}
\multiput(753.21,482.17)(-19.207,11.000){2}{\rule{0.432pt}{0.400pt}}
\multiput(731.57,494.58)(-0.605,0.495){35}{\rule{0.584pt}{0.119pt}}
\multiput(732.79,493.17)(-21.787,19.000){2}{\rule{0.292pt}{0.400pt}}
\multiput(709.92,513.00)(-0.496,2.879){41}{\rule{0.120pt}{2.373pt}}
\multiput(710.17,513.00)(-22.000,120.075){2}{\rule{0.400pt}{1.186pt}}
\multiput(686.77,636.92)(-0.546,-0.496){39}{\rule{0.538pt}{0.119pt}}
\multiput(687.88,637.17)(-21.883,-21.000){2}{\rule{0.269pt}{0.400pt}}
\multiput(664.92,607.68)(-0.496,-2.717){41}{\rule{0.120pt}{2.245pt}}
\multiput(665.17,612.34)(-22.000,-113.339){2}{\rule{0.400pt}{1.123pt}}
\multiput(642.92,493.82)(-0.496,-1.448){43}{\rule{0.120pt}{1.248pt}}
\multiput(643.17,496.41)(-23.000,-63.410){2}{\rule{0.400pt}{0.624pt}}
\multiput(619.92,428.28)(-0.496,-1.307){41}{\rule{0.120pt}{1.136pt}}
\multiput(620.17,430.64)(-22.000,-54.641){2}{\rule{0.400pt}{0.568pt}}
\multiput(597.92,371.47)(-0.496,-1.249){43}{\rule{0.120pt}{1.091pt}}
\multiput(598.17,373.73)(-23.000,-54.735){2}{\rule{0.400pt}{0.546pt}}
\multiput(574.92,314.28)(-0.496,-1.307){41}{\rule{0.120pt}{1.136pt}}
\multiput(575.17,316.64)(-22.000,-54.641){2}{\rule{0.400pt}{0.568pt}}
\multiput(726.41,474.95)(-4.704,-0.447){3}{\rule{3.033pt}{0.108pt}}
\multiput(732.70,475.17)(-15.704,-3.000){2}{\rule{1.517pt}{0.400pt}}
\multiput(711.13,473.59)(-1.713,0.485){11}{\rule{1.414pt}{0.117pt}}
\multiput(714.06,472.17)(-20.065,7.000){2}{\rule{0.707pt}{0.400pt}}
\put(672,479.67){\rule{5.300pt}{0.400pt}}
\multiput(683.00,479.17)(-11.000,1.000){2}{\rule{2.650pt}{0.400pt}}
\multiput(666.37,481.59)(-1.637,0.485){11}{\rule{1.357pt}{0.117pt}}
\multiput(669.18,480.17)(-19.183,7.000){2}{\rule{0.679pt}{0.400pt}}
\multiput(648.92,488.00)(-0.496,0.874){43}{\rule{0.120pt}{0.796pt}}
\multiput(649.17,488.00)(-23.000,38.349){2}{\rule{0.400pt}{0.398pt}}
\multiput(623.26,526.92)(-1.015,-0.492){19}{\rule{0.900pt}{0.118pt}}
\multiput(625.13,527.17)(-20.132,-11.000){2}{\rule{0.450pt}{0.400pt}}
\multiput(603.92,513.55)(-0.496,-0.918){43}{\rule{0.120pt}{0.830pt}}
\multiput(604.17,515.28)(-23.000,-40.276){2}{\rule{0.400pt}{0.415pt}}
\multiput(580.92,470.89)(-0.496,-1.122){41}{\rule{0.120pt}{0.991pt}}
\multiput(581.17,472.94)(-22.000,-46.943){2}{\rule{0.400pt}{0.495pt}}
\multiput(558.92,421.76)(-0.496,-1.161){43}{\rule{0.120pt}{1.022pt}}
\multiput(559.17,423.88)(-23.000,-50.879){2}{\rule{0.400pt}{0.511pt}}
\multiput(535.92,368.81)(-0.496,-1.145){41}{\rule{0.120pt}{1.009pt}}
\multiput(536.17,370.91)(-22.000,-47.906){2}{\rule{0.400pt}{0.505pt}}
\multiput(513.92,319.05)(-0.496,-1.073){43}{\rule{0.120pt}{0.952pt}}
\multiput(514.17,321.02)(-23.000,-47.024){2}{\rule{0.400pt}{0.476pt}}
\multiput(664.85,482.95)(-4.927,-0.447){3}{\rule{3.167pt}{0.108pt}}
\multiput(671.43,483.17)(-16.427,-3.000){2}{\rule{1.583pt}{0.400pt}}
\multiput(645.45,479.94)(-3.113,-0.468){5}{\rule{2.300pt}{0.113pt}}
\multiput(650.23,480.17)(-17.226,-4.000){2}{\rule{1.150pt}{0.400pt}}
\multiput(619.85,475.95)(-4.927,-0.447){3}{\rule{3.167pt}{0.108pt}}
\multiput(626.43,476.17)(-16.427,-3.000){2}{\rule{1.583pt}{0.400pt}}
\multiput(597.41,472.95)(-4.704,-0.447){3}{\rule{3.033pt}{0.108pt}}
\multiput(603.70,473.17)(-15.704,-3.000){2}{\rule{1.517pt}{0.400pt}}
\multiput(582.13,471.59)(-1.713,0.485){11}{\rule{1.414pt}{0.117pt}}
\multiput(585.06,470.17)(-20.065,7.000){2}{\rule{0.707pt}{0.400pt}}
\multiput(555.45,478.60)(-3.113,0.468){5}{\rule{2.300pt}{0.113pt}}
\multiput(560.23,477.17)(-17.226,4.000){2}{\rule{1.150pt}{0.400pt}}
\multiput(541.92,479.13)(-0.496,-0.741){43}{\rule{0.120pt}{0.691pt}}
\multiput(542.17,480.57)(-23.000,-32.565){2}{\rule{0.400pt}{0.346pt}}
\multiput(518.92,444.34)(-0.496,-0.983){41}{\rule{0.120pt}{0.882pt}}
\multiput(519.17,446.17)(-22.000,-41.170){2}{\rule{0.400pt}{0.441pt}}
\multiput(496.92,401.11)(-0.496,-1.053){41}{\rule{0.120pt}{0.936pt}}
\multiput(497.17,403.06)(-22.000,-44.057){2}{\rule{0.400pt}{0.468pt}}
\multiput(474.92,355.55)(-0.496,-0.918){43}{\rule{0.120pt}{0.830pt}}
\multiput(475.17,357.28)(-23.000,-40.276){2}{\rule{0.400pt}{0.415pt}}
\multiput(451.92,312.81)(-0.496,-1.145){41}{\rule{0.120pt}{1.009pt}}
\multiput(452.17,314.91)(-22.000,-47.906){2}{\rule{0.400pt}{0.505pt}}
\multiput(610.37,486.93)(-1.637,-0.485){11}{\rule{1.357pt}{0.117pt}}
\multiput(613.18,487.17)(-19.183,-7.000){2}{\rule{0.679pt}{0.400pt}}
\put(778.0,476.0){\rule[-0.200pt]{5.300pt}{0.400pt}}
\put(549,480.67){\rule{5.300pt}{0.400pt}}
\multiput(560.00,480.17)(-11.000,1.000){2}{\rule{2.650pt}{0.400pt}}
\multiput(539.04,482.60)(-3.259,0.468){5}{\rule{2.400pt}{0.113pt}}
\multiput(544.02,481.17)(-18.019,4.000){2}{\rule{1.200pt}{0.400pt}}
\multiput(524.92,483.47)(-0.496,-0.637){41}{\rule{0.120pt}{0.609pt}}
\multiput(525.17,484.74)(-22.000,-26.736){2}{\rule{0.400pt}{0.305pt}}
\multiput(497.22,456.93)(-2.027,-0.482){9}{\rule{1.633pt}{0.116pt}}
\multiput(500.61,457.17)(-19.610,-6.000){2}{\rule{0.817pt}{0.400pt}}
\multiput(479.92,449.40)(-0.496,-0.660){41}{\rule{0.120pt}{0.627pt}}
\multiput(480.17,450.70)(-22.000,-27.698){2}{\rule{0.400pt}{0.314pt}}
\multiput(457.92,419.77)(-0.496,-0.852){43}{\rule{0.120pt}{0.778pt}}
\multiput(458.17,421.38)(-23.000,-37.385){2}{\rule{0.400pt}{0.389pt}}
\multiput(434.92,380.72)(-0.496,-0.868){41}{\rule{0.120pt}{0.791pt}}
\multiput(435.17,382.36)(-22.000,-36.358){2}{\rule{0.400pt}{0.395pt}}
\multiput(412.92,342.48)(-0.496,-0.940){43}{\rule{0.120pt}{0.848pt}}
\multiput(413.17,344.24)(-23.000,-41.240){2}{\rule{0.400pt}{0.424pt}}
\multiput(389.92,300.17)(-0.496,-0.729){41}{\rule{0.120pt}{0.682pt}}
\multiput(390.17,301.58)(-22.000,-30.585){2}{\rule{0.400pt}{0.341pt}}
\multiput(550.77,479.92)(-0.853,-0.493){23}{\rule{0.777pt}{0.119pt}}
\multiput(552.39,480.17)(-20.387,-13.000){2}{\rule{0.388pt}{0.400pt}}
\multiput(527.77,466.92)(-1.173,-0.491){17}{\rule{1.020pt}{0.118pt}}
\multiput(529.88,467.17)(-20.883,-10.000){2}{\rule{0.510pt}{0.400pt}}
\multiput(504.93,456.92)(-1.121,-0.491){17}{\rule{0.980pt}{0.118pt}}
\multiput(506.97,457.17)(-19.966,-10.000){2}{\rule{0.490pt}{0.400pt}}
\multiput(483.11,446.92)(-1.062,-0.492){19}{\rule{0.936pt}{0.118pt}}
\multiput(485.06,447.17)(-21.057,-11.000){2}{\rule{0.468pt}{0.400pt}}
\multiput(459.93,435.92)(-1.121,-0.491){17}{\rule{0.980pt}{0.118pt}}
\multiput(461.97,436.17)(-19.966,-10.000){2}{\rule{0.490pt}{0.400pt}}
\multiput(439.56,425.92)(-0.611,-0.495){33}{\rule{0.589pt}{0.119pt}}
\multiput(440.78,426.17)(-20.778,-18.000){2}{\rule{0.294pt}{0.400pt}}
\multiput(418.92,406.85)(-0.496,-0.520){43}{\rule{0.120pt}{0.517pt}}
\multiput(419.17,407.93)(-23.000,-22.926){2}{\rule{0.400pt}{0.259pt}}
\multiput(395.92,382.40)(-0.496,-0.660){41}{\rule{0.120pt}{0.627pt}}
\multiput(396.17,383.70)(-22.000,-27.698){2}{\rule{0.400pt}{0.314pt}}
\multiput(373.92,353.06)(-0.496,-0.763){43}{\rule{0.120pt}{0.709pt}}
\multiput(374.17,354.53)(-23.000,-33.529){2}{\rule{0.400pt}{0.354pt}}
\multiput(350.92,318.17)(-0.496,-0.729){41}{\rule{0.120pt}{0.682pt}}
\multiput(351.17,319.58)(-22.000,-30.585){2}{\rule{0.400pt}{0.341pt}}
\multiput(328.92,285.77)(-0.496,-0.852){43}{\rule{0.120pt}{0.778pt}}
\multiput(329.17,287.38)(-23.000,-37.385){2}{\rule{0.400pt}{0.389pt}}
\multiput(489.65,473.92)(-0.893,-0.493){23}{\rule{0.808pt}{0.119pt}}
\multiput(491.32,474.17)(-21.324,-13.000){2}{\rule{0.404pt}{0.400pt}}
\multiput(466.98,460.92)(-0.791,-0.494){25}{\rule{0.729pt}{0.119pt}}
\multiput(468.49,461.17)(-20.488,-14.000){2}{\rule{0.364pt}{0.400pt}}
\multiput(444.86,446.92)(-0.827,-0.494){25}{\rule{0.757pt}{0.119pt}}
\multiput(446.43,447.17)(-21.429,-14.000){2}{\rule{0.379pt}{0.400pt}}
\multiput(421.98,432.92)(-0.791,-0.494){25}{\rule{0.729pt}{0.119pt}}
\multiput(423.49,433.17)(-20.488,-14.000){2}{\rule{0.364pt}{0.400pt}}
\multiput(400.46,418.92)(-0.639,-0.495){33}{\rule{0.611pt}{0.119pt}}
\multiput(401.73,419.17)(-21.732,-18.000){2}{\rule{0.306pt}{0.400pt}}
\multiput(375.93,400.92)(-1.121,-0.491){17}{\rule{0.980pt}{0.118pt}}
\multiput(377.97,401.17)(-19.966,-10.000){2}{\rule{0.490pt}{0.400pt}}
\multiput(356.92,389.27)(-0.496,-0.697){43}{\rule{0.120pt}{0.657pt}}
\multiput(357.17,390.64)(-23.000,-30.637){2}{\rule{0.400pt}{0.328pt}}
\multiput(333.92,357.77)(-0.496,-0.544){41}{\rule{0.120pt}{0.536pt}}
\multiput(334.17,358.89)(-22.000,-22.887){2}{\rule{0.400pt}{0.268pt}}
\multiput(311.92,333.78)(-0.496,-0.542){43}{\rule{0.120pt}{0.535pt}}
\multiput(312.17,334.89)(-23.000,-23.890){2}{\rule{0.400pt}{0.267pt}}
\multiput(288.92,308.47)(-0.496,-0.637){41}{\rule{0.120pt}{0.609pt}}
\multiput(289.17,309.74)(-22.000,-26.736){2}{\rule{0.400pt}{0.305pt}}
\multiput(266.92,280.17)(-0.496,-0.729){41}{\rule{0.120pt}{0.682pt}}
\multiput(267.17,281.58)(-22.000,-30.585){2}{\rule{0.400pt}{0.341pt}}
\put(571.0,481.0){\rule[-0.200pt]{5.541pt}{0.400pt}}
\multiput(1251.82,388.92)(-0.532,-0.500){503}{\rule{0.525pt}{0.120pt}}
\multiput(1252.91,389.17)(-267.910,-253.000){2}{\rule{0.263pt}{0.400pt}}
\multiput(246.00,228.92)(3.986,-0.499){183}{\rule{3.278pt}{0.120pt}}
\multiput(246.00,229.17)(732.195,-93.000){2}{\rule{1.639pt}{0.400pt}}
\put(548,120){\makebox(0,0){\popi{x}{-}}}
\put(1305,241){\makebox(0,0){\popi{y}{-}}}
\put(106,390){\makebox(0,0){\popi{U}{V}}}
\end{picture}

\caption{Grafické znázornenie závislosť potenciálu $U$ na polohe $x$ a $y$}  \label{G_1}
\end{figure}


\begin{figure}
% GNUPLOT: LaTeX picture
\setlength{\unitlength}{0.240900pt}
\ifx\plotpoint\undefined\newsavebox{\plotpoint}\fi
\begin{picture}(1500,900)(0,0)
\sbox{\plotpoint}{\rule[-0.200pt]{0.400pt}{0.400pt}}%
\multiput(246.00,230.58)(0.534,0.500){501}{\rule{0.527pt}{0.120pt}}
\multiput(246.00,229.17)(267.906,252.000){2}{\rule{0.263pt}{0.400pt}}
\multiput(1240.25,390.58)(-4.029,0.499){181}{\rule{3.313pt}{0.120pt}}
\multiput(1247.12,389.17)(-732.124,92.000){2}{\rule{1.657pt}{0.400pt}}
\put(246.0,230.0){\rule[-0.200pt]{0.400pt}{77.088pt}}
\multiput(246.00,230.58)(0.496,0.492){21}{\rule{0.500pt}{0.119pt}}
\multiput(246.00,229.17)(10.962,12.000){2}{\rule{0.250pt}{0.400pt}}
\put(233,205){\makebox(0,0){ 0}}
\multiput(512.79,480.92)(-0.539,-0.492){21}{\rule{0.533pt}{0.119pt}}
\multiput(513.89,481.17)(-11.893,-12.000){2}{\rule{0.267pt}{0.400pt}}
\multiput(369.00,214.58)(0.539,0.492){21}{\rule{0.533pt}{0.119pt}}
\multiput(369.00,213.17)(11.893,12.000){2}{\rule{0.267pt}{0.400pt}}
\put(356,189){\makebox(0,0){ 2}}
\multiput(635.92,465.92)(-0.496,-0.492){21}{\rule{0.500pt}{0.119pt}}
\multiput(636.96,466.17)(-10.962,-12.000){2}{\rule{0.250pt}{0.400pt}}
\multiput(492.00,199.58)(0.539,0.492){21}{\rule{0.533pt}{0.119pt}}
\multiput(492.00,198.17)(11.893,12.000){2}{\rule{0.267pt}{0.400pt}}
\put(479,174){\makebox(0,0){ 4}}
\multiput(758.92,450.92)(-0.496,-0.492){21}{\rule{0.500pt}{0.119pt}}
\multiput(759.96,451.17)(-10.962,-12.000){2}{\rule{0.250pt}{0.400pt}}
\multiput(616.00,183.58)(0.497,0.493){23}{\rule{0.500pt}{0.119pt}}
\multiput(616.00,182.17)(11.962,13.000){2}{\rule{0.250pt}{0.400pt}}
\put(603,159){\makebox(0,0){ 6}}
\multiput(881.92,435.92)(-0.497,-0.493){23}{\rule{0.500pt}{0.119pt}}
\multiput(882.96,436.17)(-11.962,-13.000){2}{\rule{0.250pt}{0.400pt}}
\multiput(739.00,168.58)(0.496,0.492){21}{\rule{0.500pt}{0.119pt}}
\multiput(739.00,167.17)(10.962,12.000){2}{\rule{0.250pt}{0.400pt}}
\put(726,143){\makebox(0,0){ 8}}
\multiput(1005.79,419.92)(-0.539,-0.492){21}{\rule{0.533pt}{0.119pt}}
\multiput(1006.89,420.17)(-11.893,-12.000){2}{\rule{0.267pt}{0.400pt}}
\multiput(862.00,153.58)(0.496,0.492){21}{\rule{0.500pt}{0.119pt}}
\multiput(862.00,152.17)(10.962,12.000){2}{\rule{0.250pt}{0.400pt}}
\put(849,128){\makebox(0,0){ 10}}
\multiput(1128.79,404.92)(-0.539,-0.492){21}{\rule{0.533pt}{0.119pt}}
\multiput(1129.89,405.17)(-11.893,-12.000){2}{\rule{0.267pt}{0.400pt}}
\multiput(985.00,137.58)(0.539,0.492){21}{\rule{0.533pt}{0.119pt}}
\multiput(985.00,136.17)(11.893,12.000){2}{\rule{0.267pt}{0.400pt}}
\put(972,113){\makebox(0,0){ 12}}
\multiput(1251.92,388.92)(-0.496,-0.492){21}{\rule{0.500pt}{0.119pt}}
\multiput(1252.96,389.17)(-10.962,-12.000){2}{\rule{0.250pt}{0.400pt}}
\multiput(974.07,137.61)(-4.034,0.447){3}{\rule{2.633pt}{0.108pt}}
\multiput(979.53,136.17)(-13.534,3.000){2}{\rule{1.317pt}{0.400pt}}
\put(1004,133){\makebox(0,0)[l]{ 0}}
\multiput(246.00,228.95)(3.811,-0.447){3}{\rule{2.500pt}{0.108pt}}
\multiput(246.00,229.17)(12.811,-3.000){2}{\rule{1.250pt}{0.400pt}}
\put(1011,180.17){\rule{3.900pt}{0.400pt}}
\multiput(1021.91,179.17)(-10.905,2.000){2}{\rule{1.950pt}{0.400pt}}
\put(1049,175){\makebox(0,0)[l]{ 2}}
\multiput(290.00,270.95)(4.034,-0.447){3}{\rule{2.633pt}{0.108pt}}
\multiput(290.00,271.17)(13.534,-3.000){2}{\rule{1.317pt}{0.400pt}}
\put(1056,222.17){\rule{3.900pt}{0.400pt}}
\multiput(1066.91,221.17)(-10.905,2.000){2}{\rule{1.950pt}{0.400pt}}
\put(1094,217){\makebox(0,0)[l]{ 4}}
\put(335,312.17){\rule{3.900pt}{0.400pt}}
\multiput(335.00,313.17)(10.905,-2.000){2}{\rule{1.950pt}{0.400pt}}
\put(1101,264.17){\rule{3.900pt}{0.400pt}}
\multiput(1111.91,263.17)(-10.905,2.000){2}{\rule{1.950pt}{0.400pt}}
\put(1139,259){\makebox(0,0)[l]{ 6}}
\put(380,354.17){\rule{3.900pt}{0.400pt}}
\multiput(380.00,355.17)(10.905,-2.000){2}{\rule{1.950pt}{0.400pt}}
\put(1146,306.17){\rule{3.900pt}{0.400pt}}
\multiput(1156.91,305.17)(-10.905,2.000){2}{\rule{1.950pt}{0.400pt}}
\put(1183,301){\makebox(0,0)[l]{ 8}}
\put(425,396.17){\rule{3.900pt}{0.400pt}}
\multiput(425.00,397.17)(10.905,-2.000){2}{\rule{1.950pt}{0.400pt}}
\multiput(1199.07,348.61)(-4.034,0.447){3}{\rule{2.633pt}{0.108pt}}
\multiput(1204.53,347.17)(-13.534,3.000){2}{\rule{1.317pt}{0.400pt}}
\put(1228,344){\makebox(0,0)[l]{ 10}}
\put(470,438.17){\rule{3.900pt}{0.400pt}}
\multiput(470.00,439.17)(10.905,-2.000){2}{\rule{1.950pt}{0.400pt}}
\multiput(1243.62,390.61)(-3.811,0.447){3}{\rule{2.500pt}{0.108pt}}
\multiput(1248.81,389.17)(-12.811,3.000){2}{\rule{1.250pt}{0.400pt}}
\put(1273,386){\makebox(0,0)[l]{ 12}}
\multiput(515.00,480.95)(4.034,-0.447){3}{\rule{2.633pt}{0.108pt}}
\multiput(515.00,481.17)(13.534,-3.000){2}{\rule{1.317pt}{0.400pt}}
\put(206,230){\makebox(0,0)[r]{ 0}}
\put(246.0,230.0){\rule[-0.200pt]{4.818pt}{0.400pt}}
\put(206,262){\makebox(0,0)[r]{ 1}}
\put(246.0,262.0){\rule[-0.200pt]{4.818pt}{0.400pt}}
\put(206,294){\makebox(0,0)[r]{ 2}}
\put(246.0,294.0){\rule[-0.200pt]{4.818pt}{0.400pt}}
\put(206,326){\makebox(0,0)[r]{ 3}}
\put(246.0,326.0){\rule[-0.200pt]{4.818pt}{0.400pt}}
\put(206,358){\makebox(0,0)[r]{ 4}}
\put(246.0,358.0){\rule[-0.200pt]{4.818pt}{0.400pt}}
\put(206,390){\makebox(0,0)[r]{ 5}}
\put(246.0,390.0){\rule[-0.200pt]{4.818pt}{0.400pt}}
\put(206,422){\makebox(0,0)[r]{ 6}}
\put(246.0,422.0){\rule[-0.200pt]{4.818pt}{0.400pt}}
\put(206,454){\makebox(0,0)[r]{ 7}}
\put(246.0,454.0){\rule[-0.200pt]{4.818pt}{0.400pt}}
\put(206,485){\makebox(0,0)[r]{ 8}}
\put(246.0,485.0){\rule[-0.200pt]{4.818pt}{0.400pt}}
\put(206,517){\makebox(0,0)[r]{ 9}}
\put(246.0,517.0){\rule[-0.200pt]{4.818pt}{0.400pt}}
\put(206,550){\makebox(0,0)[r]{ 10}}
\put(246.0,550.0){\rule[-0.200pt]{4.818pt}{0.400pt}}
\put(106,390){\makebox(0,0){\popi{U}{V}}}
\multiput(493.00,715.92)(1.821,-0.495){31}{\rule{1.535pt}{0.119pt}}
\multiput(493.00,716.17)(57.813,-17.000){2}{\rule{0.768pt}{0.400pt}}
\multiput(554.00,698.92)(2.260,-0.494){25}{\rule{1.871pt}{0.119pt}}
\multiput(554.00,699.17)(58.116,-14.000){2}{\rule{0.936pt}{0.400pt}}
\multiput(616.00,684.92)(0.943,-0.497){63}{\rule{0.852pt}{0.120pt}}
\multiput(616.00,685.17)(60.233,-33.000){2}{\rule{0.426pt}{0.400pt}}
\multiput(678.00,651.92)(2.070,-0.494){27}{\rule{1.727pt}{0.119pt}}
\multiput(678.00,652.17)(57.416,-15.000){2}{\rule{0.863pt}{0.400pt}}
\multiput(739.00,636.92)(0.623,-0.498){95}{\rule{0.598pt}{0.120pt}}
\multiput(739.00,637.17)(59.759,-49.000){2}{\rule{0.299pt}{0.400pt}}
\multiput(800.00,587.92)(0.840,-0.498){71}{\rule{0.770pt}{0.120pt}}
\multiput(800.00,588.17)(60.401,-37.000){2}{\rule{0.385pt}{0.400pt}}
\multiput(862.00,550.92)(0.739,-0.498){81}{\rule{0.690pt}{0.120pt}}
\multiput(862.00,551.17)(60.567,-42.000){2}{\rule{0.345pt}{0.400pt}}
\multiput(924.00,508.92)(0.727,-0.498){81}{\rule{0.681pt}{0.120pt}}
\multiput(924.00,509.17)(59.587,-42.000){2}{\rule{0.340pt}{0.400pt}}
\multiput(985.00,466.92)(1.492,-0.496){39}{\rule{1.281pt}{0.119pt}}
\multiput(985.00,467.17)(59.341,-21.000){2}{\rule{0.640pt}{0.400pt}}
\multiput(1047.00,445.92)(1.302,-0.496){45}{\rule{1.133pt}{0.120pt}}
\multiput(1047.00,446.17)(59.648,-24.000){2}{\rule{0.567pt}{0.400pt}}
\multiput(1109.00,421.93)(4.612,-0.485){11}{\rule{3.586pt}{0.117pt}}
\multiput(1109.00,422.17)(53.558,-7.000){2}{\rule{1.793pt}{0.400pt}}
\multiput(470.00,704.92)(2.901,-0.492){19}{\rule{2.355pt}{0.118pt}}
\multiput(470.00,705.17)(57.113,-11.000){2}{\rule{1.177pt}{0.400pt}}
\multiput(532.00,693.92)(2.901,-0.492){19}{\rule{2.355pt}{0.118pt}}
\multiput(532.00,694.17)(57.113,-11.000){2}{\rule{1.177pt}{0.400pt}}
\multiput(594.00,682.92)(0.764,-0.498){77}{\rule{0.710pt}{0.120pt}}
\multiput(594.00,683.17)(59.526,-40.000){2}{\rule{0.355pt}{0.400pt}}
\multiput(655.00,642.92)(1.851,-0.495){31}{\rule{1.559pt}{0.119pt}}
\multiput(655.00,643.17)(58.765,-17.000){2}{\rule{0.779pt}{0.400pt}}
\multiput(717.00,625.92)(0.575,-0.498){103}{\rule{0.560pt}{0.120pt}}
\multiput(717.00,626.17)(59.837,-53.000){2}{\rule{0.280pt}{0.400pt}}
\multiput(778.00,572.92)(0.710,-0.498){83}{\rule{0.667pt}{0.120pt}}
\multiput(778.00,573.17)(59.615,-43.000){2}{\rule{0.334pt}{0.400pt}}
\multiput(839.00,529.92)(0.525,-0.499){115}{\rule{0.520pt}{0.120pt}}
\multiput(839.00,530.17)(60.920,-59.000){2}{\rule{0.260pt}{0.400pt}}
\multiput(901.00,470.92)(0.608,-0.498){99}{\rule{0.586pt}{0.120pt}}
\multiput(901.00,471.17)(60.783,-51.000){2}{\rule{0.293pt}{0.400pt}}
\multiput(963.00,419.92)(0.663,-0.498){89}{\rule{0.630pt}{0.120pt}}
\multiput(963.00,420.17)(59.692,-46.000){2}{\rule{0.315pt}{0.400pt}}
\multiput(1024.00,375.58)(1.746,0.495){33}{\rule{1.478pt}{0.119pt}}
\multiput(1024.00,374.17)(58.933,18.000){2}{\rule{0.739pt}{0.400pt}}
\multiput(1086.00,391.92)(2.901,-0.492){19}{\rule{2.355pt}{0.118pt}}
\multiput(1086.00,392.17)(57.113,-11.000){2}{\rule{1.177pt}{0.400pt}}
\multiput(448.00,694.58)(2.607,0.492){21}{\rule{2.133pt}{0.119pt}}
\multiput(448.00,693.17)(56.572,12.000){2}{\rule{1.067pt}{0.400pt}}
\put(509,706.17){\rule{12.500pt}{0.400pt}}
\multiput(509.00,705.17)(36.056,2.000){2}{\rule{6.250pt}{0.400pt}}
\multiput(571.00,706.92)(0.674,-0.498){89}{\rule{0.639pt}{0.120pt}}
\multiput(571.00,707.17)(60.673,-46.000){2}{\rule{0.320pt}{0.400pt}}
\multiput(633.00,660.92)(0.710,-0.498){83}{\rule{0.667pt}{0.120pt}}
\multiput(633.00,661.17)(59.615,-43.000){2}{\rule{0.334pt}{0.400pt}}
\multiput(694.58,616.79)(0.499,-0.541){119}{\rule{0.120pt}{0.533pt}}
\multiput(693.17,617.89)(61.000,-64.894){2}{\rule{0.400pt}{0.266pt}}
\multiput(755.00,551.92)(0.553,-0.499){109}{\rule{0.543pt}{0.120pt}}
\multiput(755.00,552.17)(60.873,-56.000){2}{\rule{0.271pt}{0.400pt}}
\multiput(817.58,494.87)(0.499,-0.516){121}{\rule{0.120pt}{0.513pt}}
\multiput(816.17,495.94)(62.000,-62.935){2}{\rule{0.400pt}{0.256pt}}
\multiput(879.00,431.92)(0.516,-0.499){115}{\rule{0.514pt}{0.120pt}}
\multiput(879.00,432.17)(59.934,-59.000){2}{\rule{0.257pt}{0.400pt}}
\multiput(940.00,372.92)(1.492,-0.496){39}{\rule{1.281pt}{0.119pt}}
\multiput(940.00,373.17)(59.341,-21.000){2}{\rule{0.640pt}{0.400pt}}
\put(1002,353.17){\rule{12.500pt}{0.400pt}}
\multiput(1002.00,352.17)(36.056,2.000){2}{\rule{6.250pt}{0.400pt}}
\put(1064,353.67){\rule{14.695pt}{0.400pt}}
\multiput(1064.00,354.17)(30.500,-1.000){2}{\rule{7.347pt}{0.400pt}}
\multiput(425.00,678.93)(6.833,-0.477){7}{\rule{5.060pt}{0.115pt}}
\multiput(425.00,679.17)(51.498,-5.000){2}{\rule{2.530pt}{0.400pt}}
\multiput(487.00,675.58)(2.650,0.492){21}{\rule{2.167pt}{0.119pt}}
\multiput(487.00,674.17)(57.503,12.000){2}{\rule{1.083pt}{0.400pt}}
\multiput(549.00,685.92)(0.764,-0.498){77}{\rule{0.710pt}{0.120pt}}
\multiput(549.00,686.17)(59.526,-40.000){2}{\rule{0.355pt}{0.400pt}}
\multiput(610.00,645.92)(0.633,-0.498){95}{\rule{0.606pt}{0.120pt}}
\multiput(610.00,646.17)(60.742,-49.000){2}{\rule{0.303pt}{0.400pt}}
\multiput(672.58,595.82)(0.499,-0.532){121}{\rule{0.120pt}{0.526pt}}
\multiput(671.17,596.91)(62.000,-64.909){2}{\rule{0.400pt}{0.263pt}}
\multiput(734.58,529.87)(0.499,-0.516){117}{\rule{0.120pt}{0.513pt}}
\multiput(733.17,530.93)(60.000,-60.935){2}{\rule{0.400pt}{0.257pt}}
\multiput(794.58,467.60)(0.499,-0.597){121}{\rule{0.120pt}{0.577pt}}
\multiput(793.17,468.80)(62.000,-72.802){2}{\rule{0.400pt}{0.289pt}}
\multiput(856.00,394.92)(0.585,-0.498){103}{\rule{0.568pt}{0.120pt}}
\multiput(856.00,395.17)(60.821,-53.000){2}{\rule{0.284pt}{0.400pt}}
\multiput(918.00,341.92)(2.260,-0.494){25}{\rule{1.871pt}{0.119pt}}
\multiput(918.00,342.17)(58.116,-14.000){2}{\rule{0.936pt}{0.400pt}}
\multiput(980.00,327.94)(8.816,-0.468){5}{\rule{6.200pt}{0.113pt}}
\multiput(980.00,328.17)(48.132,-4.000){2}{\rule{3.100pt}{0.400pt}}
\multiput(1041.00,323.93)(4.060,-0.488){13}{\rule{3.200pt}{0.117pt}}
\multiput(1041.00,324.17)(55.358,-8.000){2}{\rule{1.600pt}{0.400pt}}
\put(403,657.17){\rule{12.300pt}{0.400pt}}
\multiput(403.00,658.17)(35.471,-2.000){2}{\rule{6.150pt}{0.400pt}}
\multiput(464.00,657.59)(3.582,0.489){15}{\rule{2.856pt}{0.118pt}}
\multiput(464.00,656.17)(56.073,9.000){2}{\rule{1.428pt}{0.400pt}}
\multiput(526.00,664.92)(0.776,-0.498){77}{\rule{0.720pt}{0.120pt}}
\multiput(526.00,665.17)(60.506,-40.000){2}{\rule{0.360pt}{0.400pt}}
\multiput(588.00,624.92)(0.525,-0.499){115}{\rule{0.520pt}{0.120pt}}
\multiput(588.00,625.17)(60.920,-59.000){2}{\rule{0.260pt}{0.400pt}}
\multiput(650.00,565.92)(0.544,-0.499){109}{\rule{0.536pt}{0.120pt}}
\multiput(650.00,566.17)(59.888,-56.000){2}{\rule{0.268pt}{0.400pt}}
\multiput(711.58,508.84)(0.499,-0.524){119}{\rule{0.120pt}{0.520pt}}
\multiput(710.17,509.92)(61.000,-62.921){2}{\rule{0.400pt}{0.260pt}}
\multiput(772.58,444.47)(0.499,-0.637){121}{\rule{0.120pt}{0.610pt}}
\multiput(771.17,445.73)(62.000,-77.735){2}{\rule{0.400pt}{0.305pt}}
\multiput(834.00,366.92)(0.663,-0.498){89}{\rule{0.630pt}{0.120pt}}
\multiput(834.00,367.17)(59.692,-46.000){2}{\rule{0.315pt}{0.400pt}}
\multiput(895.00,320.92)(2.260,-0.494){25}{\rule{1.871pt}{0.119pt}}
\multiput(895.00,321.17)(58.116,-14.000){2}{\rule{0.936pt}{0.400pt}}
\multiput(957.00,306.93)(4.060,-0.488){13}{\rule{3.200pt}{0.117pt}}
\multiput(957.00,307.17)(55.358,-8.000){2}{\rule{1.600pt}{0.400pt}}
\multiput(1019.00,298.93)(4.612,-0.485){11}{\rule{3.586pt}{0.117pt}}
\multiput(1019.00,299.17)(53.558,-7.000){2}{\rule{1.793pt}{0.400pt}}
\multiput(380.00,642.92)(2.901,-0.492){19}{\rule{2.355pt}{0.118pt}}
\multiput(380.00,643.17)(57.113,-11.000){2}{\rule{1.177pt}{0.400pt}}
\multiput(442.00,633.58)(2.650,0.492){21}{\rule{2.167pt}{0.119pt}}
\multiput(442.00,632.17)(57.503,12.000){2}{\rule{1.083pt}{0.400pt}}
\multiput(504.00,643.92)(0.610,-0.498){97}{\rule{0.588pt}{0.120pt}}
\multiput(504.00,644.17)(59.780,-50.000){2}{\rule{0.294pt}{0.400pt}}
\multiput(565.00,593.92)(0.596,-0.498){101}{\rule{0.577pt}{0.120pt}}
\multiput(565.00,594.17)(60.803,-52.000){2}{\rule{0.288pt}{0.400pt}}
\multiput(627.00,541.92)(0.525,-0.499){115}{\rule{0.520pt}{0.120pt}}
\multiput(627.00,542.17)(60.920,-59.000){2}{\rule{0.260pt}{0.400pt}}
\multiput(689.00,482.92)(0.554,-0.499){107}{\rule{0.544pt}{0.120pt}}
\multiput(689.00,483.17)(59.872,-55.000){2}{\rule{0.272pt}{0.400pt}}
\multiput(750.58,426.54)(0.499,-0.615){119}{\rule{0.120pt}{0.592pt}}
\multiput(749.17,427.77)(61.000,-73.772){2}{\rule{0.400pt}{0.296pt}}
\multiput(811.00,352.92)(0.620,-0.498){97}{\rule{0.596pt}{0.120pt}}
\multiput(811.00,353.17)(60.763,-50.000){2}{\rule{0.298pt}{0.400pt}}
\multiput(873.00,302.92)(1.851,-0.495){31}{\rule{1.559pt}{0.119pt}}
\multiput(873.00,303.17)(58.765,-17.000){2}{\rule{0.779pt}{0.400pt}}
\multiput(935.00,285.94)(8.816,-0.468){5}{\rule{6.200pt}{0.113pt}}
\multiput(935.00,286.17)(48.132,-4.000){2}{\rule{3.100pt}{0.400pt}}
\multiput(996.00,281.92)(2.901,-0.492){19}{\rule{2.355pt}{0.118pt}}
\multiput(996.00,282.17)(57.113,-11.000){2}{\rule{1.177pt}{0.400pt}}
\multiput(358.00,621.92)(2.901,-0.492){19}{\rule{2.355pt}{0.118pt}}
\multiput(358.00,622.17)(57.113,-11.000){2}{\rule{1.177pt}{0.400pt}}
\multiput(420.00,612.58)(2.607,0.492){21}{\rule{2.133pt}{0.119pt}}
\multiput(420.00,611.17)(56.572,12.000){2}{\rule{1.067pt}{0.400pt}}
\multiput(481.00,622.92)(0.620,-0.498){97}{\rule{0.596pt}{0.120pt}}
\multiput(481.00,623.17)(60.763,-50.000){2}{\rule{0.298pt}{0.400pt}}
\multiput(543.00,572.92)(0.674,-0.498){89}{\rule{0.639pt}{0.120pt}}
\multiput(543.00,573.17)(60.673,-46.000){2}{\rule{0.320pt}{0.400pt}}
\multiput(605.58,525.84)(0.499,-0.524){119}{\rule{0.120pt}{0.520pt}}
\multiput(604.17,526.92)(61.000,-62.921){2}{\rule{0.400pt}{0.260pt}}
\multiput(666.58,461.82)(0.499,-0.532){121}{\rule{0.120pt}{0.526pt}}
\multiput(665.17,462.91)(62.000,-64.909){2}{\rule{0.400pt}{0.263pt}}
\multiput(728.58,395.71)(0.499,-0.565){119}{\rule{0.120pt}{0.552pt}}
\multiput(727.17,396.85)(61.000,-67.853){2}{\rule{0.400pt}{0.276pt}}
\multiput(789.00,327.92)(0.623,-0.498){95}{\rule{0.598pt}{0.120pt}}
\multiput(789.00,328.17)(59.759,-49.000){2}{\rule{0.299pt}{0.400pt}}
\multiput(850.00,278.92)(2.260,-0.494){25}{\rule{1.871pt}{0.119pt}}
\multiput(850.00,279.17)(58.116,-14.000){2}{\rule{0.936pt}{0.400pt}}
\multiput(912.00,264.93)(4.060,-0.488){13}{\rule{3.200pt}{0.117pt}}
\multiput(912.00,265.17)(55.358,-8.000){2}{\rule{1.600pt}{0.400pt}}
\multiput(974.00,256.93)(4.688,-0.485){11}{\rule{3.643pt}{0.117pt}}
\multiput(974.00,257.17)(54.439,-7.000){2}{\rule{1.821pt}{0.400pt}}
\multiput(335.00,600.92)(2.901,-0.492){19}{\rule{2.355pt}{0.118pt}}
\multiput(335.00,601.17)(57.113,-11.000){2}{\rule{1.177pt}{0.400pt}}
\multiput(397.00,591.58)(2.901,0.492){19}{\rule{2.355pt}{0.118pt}}
\multiput(397.00,590.17)(57.113,11.000){2}{\rule{1.177pt}{0.400pt}}
\multiput(459.00,600.92)(0.663,-0.498){89}{\rule{0.630pt}{0.120pt}}
\multiput(459.00,601.17)(59.692,-46.000){2}{\rule{0.315pt}{0.400pt}}
\multiput(520.00,554.92)(0.633,-0.498){95}{\rule{0.606pt}{0.120pt}}
\multiput(520.00,555.17)(60.742,-49.000){2}{\rule{0.303pt}{0.400pt}}
\multiput(582.00,505.92)(0.534,-0.499){113}{\rule{0.528pt}{0.120pt}}
\multiput(582.00,506.17)(60.905,-58.000){2}{\rule{0.264pt}{0.400pt}}
\multiput(644.00,447.92)(0.499,-0.499){121}{\rule{0.500pt}{0.120pt}}
\multiput(644.00,448.17)(60.962,-62.000){2}{\rule{0.250pt}{0.400pt}}
\multiput(706.58,384.51)(0.499,-0.625){117}{\rule{0.120pt}{0.600pt}}
\multiput(705.17,385.75)(60.000,-73.755){2}{\rule{0.400pt}{0.300pt}}
\multiput(766.00,310.92)(0.585,-0.498){103}{\rule{0.568pt}{0.120pt}}
\multiput(766.00,311.17)(60.821,-53.000){2}{\rule{0.284pt}{0.400pt}}
\multiput(828.00,257.92)(2.260,-0.494){25}{\rule{1.871pt}{0.119pt}}
\multiput(828.00,258.17)(58.116,-14.000){2}{\rule{0.936pt}{0.400pt}}
\multiput(890.00,243.93)(3.994,-0.488){13}{\rule{3.150pt}{0.117pt}}
\multiput(890.00,244.17)(54.462,-8.000){2}{\rule{1.575pt}{0.400pt}}
\multiput(951.00,235.93)(4.060,-0.488){13}{\rule{3.200pt}{0.117pt}}
\multiput(951.00,236.17)(55.358,-8.000){2}{\rule{1.600pt}{0.400pt}}
\multiput(313.00,575.92)(3.206,-0.491){17}{\rule{2.580pt}{0.118pt}}
\multiput(313.00,576.17)(56.645,-10.000){2}{\rule{1.290pt}{0.400pt}}
\multiput(375.00,567.58)(2.223,0.494){25}{\rule{1.843pt}{0.119pt}}
\multiput(375.00,566.17)(57.175,14.000){2}{\rule{0.921pt}{0.400pt}}
\multiput(436.00,579.92)(0.674,-0.498){89}{\rule{0.639pt}{0.120pt}}
\multiput(436.00,580.17)(60.673,-46.000){2}{\rule{0.320pt}{0.400pt}}
\multiput(498.00,533.92)(0.553,-0.499){109}{\rule{0.543pt}{0.120pt}}
\multiput(498.00,534.17)(60.873,-56.000){2}{\rule{0.271pt}{0.400pt}}
\multiput(560.00,477.92)(0.565,-0.498){105}{\rule{0.552pt}{0.120pt}}
\multiput(560.00,478.17)(59.855,-54.000){2}{\rule{0.276pt}{0.400pt}}
\multiput(621.58,422.82)(0.499,-0.532){121}{\rule{0.120pt}{0.526pt}}
\multiput(620.17,423.91)(62.000,-64.909){2}{\rule{0.400pt}{0.263pt}}
\multiput(683.58,356.84)(0.499,-0.524){121}{\rule{0.120pt}{0.519pt}}
\multiput(682.17,357.92)(62.000,-63.922){2}{\rule{0.400pt}{0.260pt}}
\multiput(745.00,292.92)(0.575,-0.498){103}{\rule{0.560pt}{0.120pt}}
\multiput(745.00,293.17)(59.837,-53.000){2}{\rule{0.280pt}{0.400pt}}
\multiput(806.00,239.92)(2.223,-0.494){25}{\rule{1.843pt}{0.119pt}}
\multiput(806.00,240.17)(57.175,-14.000){2}{\rule{0.921pt}{0.400pt}}
\multiput(867.00,225.93)(4.060,-0.488){13}{\rule{3.200pt}{0.117pt}}
\multiput(867.00,226.17)(55.358,-8.000){2}{\rule{1.600pt}{0.400pt}}
\multiput(929.00,217.93)(4.688,-0.485){11}{\rule{3.643pt}{0.117pt}}
\multiput(929.00,218.17)(54.439,-7.000){2}{\rule{1.821pt}{0.400pt}}
\multiput(290.00,551.93)(4.060,-0.488){13}{\rule{3.200pt}{0.117pt}}
\multiput(290.00,552.17)(55.358,-8.000){2}{\rule{1.600pt}{0.400pt}}
\multiput(352.00,545.58)(2.105,0.494){27}{\rule{1.753pt}{0.119pt}}
\multiput(352.00,544.17)(58.361,15.000){2}{\rule{0.877pt}{0.400pt}}
\multiput(414.00,558.92)(0.797,-0.498){75}{\rule{0.736pt}{0.120pt}}
\multiput(414.00,559.17)(60.473,-39.000){2}{\rule{0.368pt}{0.400pt}}
\multiput(476.00,519.92)(0.516,-0.499){115}{\rule{0.514pt}{0.120pt}}
\multiput(476.00,520.17)(59.934,-59.000){2}{\rule{0.257pt}{0.400pt}}
\multiput(537.58,459.76)(0.499,-0.548){121}{\rule{0.120pt}{0.539pt}}
\multiput(536.17,460.88)(62.000,-66.882){2}{\rule{0.400pt}{0.269pt}}
\multiput(599.00,392.92)(0.620,-0.498){97}{\rule{0.596pt}{0.120pt}}
\multiput(599.00,393.17)(60.763,-50.000){2}{\rule{0.298pt}{0.400pt}}
\multiput(661.58,341.73)(0.499,-0.557){119}{\rule{0.120pt}{0.546pt}}
\multiput(660.17,342.87)(61.000,-66.867){2}{\rule{0.400pt}{0.273pt}}
\multiput(722.00,274.92)(0.516,-0.499){115}{\rule{0.514pt}{0.120pt}}
\multiput(722.00,275.17)(59.934,-59.000){2}{\rule{0.257pt}{0.400pt}}
\multiput(783.00,215.92)(2.901,-0.492){19}{\rule{2.355pt}{0.118pt}}
\multiput(783.00,216.17)(57.113,-11.000){2}{\rule{1.177pt}{0.400pt}}
\multiput(845.00,204.93)(3.994,-0.488){13}{\rule{3.150pt}{0.117pt}}
\multiput(845.00,205.17)(54.462,-8.000){2}{\rule{1.575pt}{0.400pt}}
\multiput(906.00,196.94)(8.962,-0.468){5}{\rule{6.300pt}{0.113pt}}
\multiput(906.00,197.17)(48.924,-4.000){2}{\rule{3.150pt}{0.400pt}}
\multiput(268.00,527.93)(4.060,-0.488){13}{\rule{3.200pt}{0.117pt}}
\multiput(268.00,528.17)(55.358,-8.000){2}{\rule{1.600pt}{0.400pt}}
\multiput(330.00,521.58)(1.717,0.495){33}{\rule{1.456pt}{0.119pt}}
\multiput(330.00,520.17)(57.979,18.000){2}{\rule{0.728pt}{0.400pt}}
\multiput(391.00,537.92)(0.722,-0.498){83}{\rule{0.677pt}{0.120pt}}
\multiput(391.00,538.17)(60.595,-43.000){2}{\rule{0.338pt}{0.400pt}}
\multiput(453.58,493.79)(0.499,-0.540){121}{\rule{0.120pt}{0.532pt}}
\multiput(452.17,494.90)(62.000,-65.895){2}{\rule{0.400pt}{0.266pt}}
\multiput(515.00,427.92)(0.575,-0.498){103}{\rule{0.560pt}{0.120pt}}
\multiput(515.00,428.17)(59.837,-53.000){2}{\rule{0.280pt}{0.400pt}}
\multiput(576.00,374.92)(0.525,-0.499){115}{\rule{0.520pt}{0.120pt}}
\multiput(576.00,375.17)(60.920,-59.000){2}{\rule{0.260pt}{0.400pt}}
\multiput(638.00,315.92)(0.525,-0.499){115}{\rule{0.520pt}{0.120pt}}
\multiput(638.00,316.17)(60.920,-59.000){2}{\rule{0.260pt}{0.400pt}}
\multiput(700.00,256.92)(0.516,-0.499){115}{\rule{0.514pt}{0.120pt}}
\multiput(700.00,257.17)(59.934,-59.000){2}{\rule{0.257pt}{0.400pt}}
\multiput(761.00,197.92)(1.821,-0.495){31}{\rule{1.535pt}{0.119pt}}
\multiput(761.00,198.17)(57.813,-17.000){2}{\rule{0.768pt}{0.400pt}}
\multiput(822.00,180.93)(6.833,-0.477){7}{\rule{5.060pt}{0.115pt}}
\multiput(822.00,181.17)(51.498,-5.000){2}{\rule{2.530pt}{0.400pt}}
\multiput(884.00,175.94)(8.962,-0.468){5}{\rule{6.300pt}{0.113pt}}
\multiput(884.00,176.17)(48.924,-4.000){2}{\rule{3.150pt}{0.400pt}}
\multiput(246.00,499.93)(4.612,-0.485){11}{\rule{3.586pt}{0.117pt}}
\multiput(246.00,500.17)(53.558,-7.000){2}{\rule{1.793pt}{0.400pt}}
\put(307,492.17){\rule{12.500pt}{0.400pt}}
\multiput(307.00,493.17)(36.056,-2.000){2}{\rule{6.250pt}{0.400pt}}
\multiput(369.00,490.92)(1.423,-0.496){41}{\rule{1.227pt}{0.120pt}}
\multiput(369.00,491.17)(59.453,-22.000){2}{\rule{0.614pt}{0.400pt}}
\multiput(431.58,467.79)(0.499,-0.541){119}{\rule{0.120pt}{0.533pt}}
\multiput(430.17,468.89)(61.000,-64.894){2}{\rule{0.400pt}{0.266pt}}
\multiput(492.00,402.92)(0.596,-0.498){101}{\rule{0.577pt}{0.120pt}}
\multiput(492.00,403.17)(60.803,-52.000){2}{\rule{0.288pt}{0.400pt}}
\multiput(554.00,350.92)(0.585,-0.498){103}{\rule{0.568pt}{0.120pt}}
\multiput(554.00,351.17)(60.821,-53.000){2}{\rule{0.284pt}{0.400pt}}
\multiput(616.00,297.92)(0.516,-0.499){115}{\rule{0.514pt}{0.120pt}}
\multiput(616.00,298.17)(59.934,-59.000){2}{\rule{0.257pt}{0.400pt}}
\multiput(677.00,238.92)(0.553,-0.499){109}{\rule{0.543pt}{0.120pt}}
\multiput(677.00,239.17)(60.873,-56.000){2}{\rule{0.271pt}{0.400pt}}
\multiput(739.00,182.92)(1.542,-0.496){37}{\rule{1.320pt}{0.119pt}}
\multiput(739.00,183.17)(58.260,-20.000){2}{\rule{0.660pt}{0.400pt}}
\multiput(800.00,162.93)(4.060,-0.488){13}{\rule{3.200pt}{0.117pt}}
\multiput(800.00,163.17)(55.358,-8.000){2}{\rule{1.600pt}{0.400pt}}
\multiput(862.00,154.92)(2.854,-0.492){19}{\rule{2.318pt}{0.118pt}}
\multiput(862.00,155.17)(56.188,-11.000){2}{\rule{1.159pt}{0.400pt}}
\multiput(1168.92,413.02)(-0.496,-0.775){41}{\rule{0.120pt}{0.718pt}}
\multiput(1169.17,414.51)(-22.000,-32.509){2}{\rule{0.400pt}{0.359pt}}
\multiput(1146.92,379.56)(-0.496,-0.609){43}{\rule{0.120pt}{0.587pt}}
\multiput(1147.17,380.78)(-23.000,-26.782){2}{\rule{0.400pt}{0.293pt}}
\multiput(1123.92,350.79)(-0.496,-0.845){41}{\rule{0.120pt}{0.773pt}}
\multiput(1124.17,352.40)(-22.000,-35.396){2}{\rule{0.400pt}{0.386pt}}
\multiput(1101.92,314.85)(-0.496,-0.520){43}{\rule{0.120pt}{0.517pt}}
\multiput(1102.17,315.93)(-23.000,-22.926){2}{\rule{0.400pt}{0.259pt}}
\multiput(1077.85,291.92)(-0.522,-0.496){39}{\rule{0.519pt}{0.119pt}}
\multiput(1078.92,292.17)(-20.923,-21.000){2}{\rule{0.260pt}{0.400pt}}
\multiput(1055.85,270.92)(-0.522,-0.496){39}{\rule{0.519pt}{0.119pt}}
\multiput(1056.92,271.17)(-20.923,-21.000){2}{\rule{0.260pt}{0.400pt}}
\multiput(1033.85,249.92)(-0.521,-0.496){41}{\rule{0.518pt}{0.120pt}}
\multiput(1034.92,250.17)(-21.924,-22.000){2}{\rule{0.259pt}{0.400pt}}
\multiput(1010.44,227.92)(-0.648,-0.495){31}{\rule{0.618pt}{0.119pt}}
\multiput(1011.72,228.17)(-20.718,-17.000){2}{\rule{0.309pt}{0.400pt}}
\multiput(988.46,210.92)(-0.639,-0.495){33}{\rule{0.611pt}{0.119pt}}
\multiput(989.73,211.17)(-21.732,-18.000){2}{\rule{0.306pt}{0.400pt}}
\multiput(965.85,192.92)(-0.522,-0.496){39}{\rule{0.519pt}{0.119pt}}
\multiput(966.92,193.17)(-20.923,-21.000){2}{\rule{0.260pt}{0.400pt}}
\multiput(944.92,170.56)(-0.496,-0.609){43}{\rule{0.120pt}{0.587pt}}
\multiput(945.17,171.78)(-23.000,-26.782){2}{\rule{0.400pt}{0.293pt}}
\multiput(1107.92,420.42)(-0.496,-0.653){43}{\rule{0.120pt}{0.622pt}}
\multiput(1108.17,421.71)(-23.000,-28.710){2}{\rule{0.400pt}{0.311pt}}
\multiput(1084.92,389.72)(-0.496,-0.868){41}{\rule{0.120pt}{0.791pt}}
\multiput(1085.17,391.36)(-22.000,-36.358){2}{\rule{0.400pt}{0.395pt}}
\multiput(1062.92,352.42)(-0.496,-0.653){43}{\rule{0.120pt}{0.622pt}}
\multiput(1063.17,353.71)(-23.000,-28.710){2}{\rule{0.400pt}{0.311pt}}
\multiput(1039.92,322.70)(-0.496,-0.567){41}{\rule{0.120pt}{0.555pt}}
\multiput(1040.17,323.85)(-22.000,-23.849){2}{\rule{0.400pt}{0.277pt}}
\multiput(1016.34,298.92)(-0.678,-0.495){31}{\rule{0.641pt}{0.119pt}}
\multiput(1017.67,299.17)(-21.669,-17.000){2}{\rule{0.321pt}{0.400pt}}
\multiput(994.92,280.70)(-0.496,-0.567){41}{\rule{0.120pt}{0.555pt}}
\multiput(995.17,281.85)(-22.000,-23.849){2}{\rule{0.400pt}{0.277pt}}
\multiput(971.77,256.92)(-0.546,-0.496){39}{\rule{0.538pt}{0.119pt}}
\multiput(972.88,257.17)(-21.883,-21.000){2}{\rule{0.269pt}{0.400pt}}
\multiput(948.56,235.92)(-0.611,-0.495){33}{\rule{0.589pt}{0.119pt}}
\multiput(949.78,236.17)(-20.778,-18.000){2}{\rule{0.294pt}{0.400pt}}
\multiput(926.77,217.92)(-0.546,-0.496){39}{\rule{0.538pt}{0.119pt}}
\multiput(927.88,218.17)(-21.883,-21.000){2}{\rule{0.269pt}{0.400pt}}
\multiput(903.85,196.92)(-0.522,-0.496){39}{\rule{0.519pt}{0.119pt}}
\multiput(904.92,197.17)(-20.923,-21.000){2}{\rule{0.260pt}{0.400pt}}
\multiput(881.85,175.92)(-0.522,-0.496){39}{\rule{0.519pt}{0.119pt}}
\multiput(882.92,176.17)(-20.923,-21.000){2}{\rule{0.260pt}{0.400pt}}
\multiput(1045.92,441.39)(-0.496,-1.581){43}{\rule{0.120pt}{1.352pt}}
\multiput(1046.17,444.19)(-23.000,-69.193){2}{\rule{0.400pt}{0.676pt}}
\multiput(1021.92,373.92)(-0.498,-0.496){41}{\rule{0.500pt}{0.120pt}}
\multiput(1022.96,374.17)(-20.962,-22.000){2}{\rule{0.250pt}{0.400pt}}
\multiput(1000.92,350.77)(-0.496,-0.544){41}{\rule{0.120pt}{0.536pt}}
\multiput(1001.17,351.89)(-22.000,-22.887){2}{\rule{0.400pt}{0.268pt}}
\multiput(977.77,327.92)(-0.546,-0.496){39}{\rule{0.538pt}{0.119pt}}
\multiput(978.88,328.17)(-21.883,-21.000){2}{\rule{0.269pt}{0.400pt}}
\multiput(954.85,306.92)(-0.522,-0.496){39}{\rule{0.519pt}{0.119pt}}
\multiput(955.92,307.17)(-20.923,-21.000){2}{\rule{0.260pt}{0.400pt}}
\multiput(932.77,285.92)(-0.546,-0.496){39}{\rule{0.538pt}{0.119pt}}
\multiput(933.88,286.17)(-21.883,-21.000){2}{\rule{0.269pt}{0.400pt}}
\multiput(909.85,264.92)(-0.522,-0.496){39}{\rule{0.519pt}{0.119pt}}
\multiput(910.92,265.17)(-20.923,-21.000){2}{\rule{0.260pt}{0.400pt}}
\multiput(887.46,243.92)(-0.639,-0.495){33}{\rule{0.611pt}{0.119pt}}
\multiput(888.73,244.17)(-21.732,-18.000){2}{\rule{0.306pt}{0.400pt}}
\multiput(864.85,225.92)(-0.522,-0.496){39}{\rule{0.519pt}{0.119pt}}
\multiput(865.92,226.17)(-20.923,-21.000){2}{\rule{0.260pt}{0.400pt}}
\multiput(843.92,203.85)(-0.496,-0.520){43}{\rule{0.120pt}{0.517pt}}
\multiput(844.17,204.93)(-23.000,-22.926){2}{\rule{0.400pt}{0.259pt}}
\multiput(819.56,180.92)(-0.611,-0.495){33}{\rule{0.589pt}{0.119pt}}
\multiput(820.78,181.17)(-20.778,-18.000){2}{\rule{0.294pt}{0.400pt}}
\multiput(983.92,464.04)(-0.496,-1.076){41}{\rule{0.120pt}{0.955pt}}
\multiput(984.17,466.02)(-22.000,-45.019){2}{\rule{0.400pt}{0.477pt}}
\multiput(961.92,417.19)(-0.496,-1.028){43}{\rule{0.120pt}{0.917pt}}
\multiput(962.17,419.10)(-23.000,-45.096){2}{\rule{0.400pt}{0.459pt}}
\multiput(938.92,371.25)(-0.496,-0.706){41}{\rule{0.120pt}{0.664pt}}
\multiput(939.17,372.62)(-22.000,-29.623){2}{\rule{0.400pt}{0.332pt}}
\multiput(915.77,341.92)(-0.546,-0.496){39}{\rule{0.538pt}{0.119pt}}
\multiput(916.88,342.17)(-21.883,-21.000){2}{\rule{0.269pt}{0.400pt}}
\multiput(892.56,320.92)(-0.611,-0.495){33}{\rule{0.589pt}{0.119pt}}
\multiput(893.78,321.17)(-20.778,-18.000){2}{\rule{0.294pt}{0.400pt}}
\multiput(871.92,301.85)(-0.496,-0.520){43}{\rule{0.120pt}{0.517pt}}
\multiput(872.17,302.93)(-23.000,-22.926){2}{\rule{0.400pt}{0.259pt}}
\multiput(847.85,278.92)(-0.522,-0.496){39}{\rule{0.519pt}{0.119pt}}
\multiput(848.92,279.17)(-20.923,-21.000){2}{\rule{0.260pt}{0.400pt}}
\multiput(825.56,257.92)(-0.611,-0.495){33}{\rule{0.589pt}{0.119pt}}
\multiput(826.78,258.17)(-20.778,-18.000){2}{\rule{0.294pt}{0.400pt}}
\multiput(804.92,238.85)(-0.496,-0.520){43}{\rule{0.120pt}{0.517pt}}
\multiput(805.17,239.93)(-23.000,-22.926){2}{\rule{0.400pt}{0.259pt}}
\multiput(780.56,215.92)(-0.611,-0.495){33}{\rule{0.589pt}{0.119pt}}
\multiput(781.78,216.17)(-20.778,-18.000){2}{\rule{0.294pt}{0.400pt}}
\multiput(758.15,197.92)(-0.737,-0.494){27}{\rule{0.687pt}{0.119pt}}
\multiput(759.57,198.17)(-20.575,-15.000){2}{\rule{0.343pt}{0.400pt}}
\multiput(922.92,506.84)(-0.496,-0.830){43}{\rule{0.120pt}{0.761pt}}
\multiput(923.17,508.42)(-23.000,-36.421){2}{\rule{0.400pt}{0.380pt}}
\multiput(899.92,468.64)(-0.496,-0.891){41}{\rule{0.120pt}{0.809pt}}
\multiput(900.17,470.32)(-22.000,-37.321){2}{\rule{0.400pt}{0.405pt}}
\multiput(877.92,429.91)(-0.496,-0.807){43}{\rule{0.120pt}{0.743pt}}
\multiput(878.17,431.46)(-23.000,-35.457){2}{\rule{0.400pt}{0.372pt}}
\multiput(854.92,393.47)(-0.496,-0.637){41}{\rule{0.120pt}{0.609pt}}
\multiput(855.17,394.74)(-22.000,-26.736){2}{\rule{0.400pt}{0.305pt}}
\multiput(830.86,366.92)(-0.827,-0.494){25}{\rule{0.757pt}{0.119pt}}
\multiput(832.43,367.17)(-21.429,-14.000){2}{\rule{0.379pt}{0.400pt}}
\multiput(809.92,351.70)(-0.496,-0.567){41}{\rule{0.120pt}{0.555pt}}
\multiput(810.17,352.85)(-22.000,-23.849){2}{\rule{0.400pt}{0.277pt}}
\multiput(786.34,327.92)(-0.678,-0.495){31}{\rule{0.641pt}{0.119pt}}
\multiput(787.67,328.17)(-21.669,-17.000){2}{\rule{0.321pt}{0.400pt}}
\multiput(763.65,310.92)(-0.583,-0.495){33}{\rule{0.567pt}{0.119pt}}
\multiput(764.82,311.17)(-19.824,-18.000){2}{\rule{0.283pt}{0.400pt}}
\multiput(742.46,292.92)(-0.639,-0.495){33}{\rule{0.611pt}{0.119pt}}
\multiput(743.73,293.17)(-21.732,-18.000){2}{\rule{0.306pt}{0.400pt}}
\multiput(719.56,274.92)(-0.611,-0.495){33}{\rule{0.589pt}{0.119pt}}
\multiput(720.78,275.17)(-20.778,-18.000){2}{\rule{0.294pt}{0.400pt}}
\multiput(697.46,256.92)(-0.639,-0.495){33}{\rule{0.611pt}{0.119pt}}
\multiput(698.73,257.17)(-21.732,-18.000){2}{\rule{0.306pt}{0.400pt}}
\multiput(859.77,550.92)(-0.546,-0.496){39}{\rule{0.538pt}{0.119pt}}
\multiput(860.88,551.17)(-21.883,-21.000){2}{\rule{0.269pt}{0.400pt}}
\multiput(837.92,528.02)(-0.496,-0.775){41}{\rule{0.120pt}{0.718pt}}
\multiput(838.17,529.51)(-22.000,-32.509){2}{\rule{0.400pt}{0.359pt}}
\multiput(815.92,494.64)(-0.496,-0.587){43}{\rule{0.120pt}{0.570pt}}
\multiput(816.17,495.82)(-23.000,-25.818){2}{\rule{0.400pt}{0.285pt}}
\multiput(792.92,467.85)(-0.496,-0.521){41}{\rule{0.120pt}{0.518pt}}
\multiput(793.17,468.92)(-22.000,-21.924){2}{\rule{0.400pt}{0.259pt}}
\multiput(769.56,445.92)(-0.611,-0.495){33}{\rule{0.589pt}{0.119pt}}
\multiput(770.78,446.17)(-20.778,-18.000){2}{\rule{0.294pt}{0.400pt}}
\multiput(748.92,426.25)(-0.496,-0.706){41}{\rule{0.120pt}{0.664pt}}
\multiput(749.17,427.62)(-22.000,-29.623){2}{\rule{0.400pt}{0.332pt}}
\multiput(724.26,396.92)(-1.015,-0.492){19}{\rule{0.900pt}{0.118pt}}
\multiput(726.13,397.17)(-20.132,-11.000){2}{\rule{0.450pt}{0.400pt}}
\multiput(704.92,384.56)(-0.496,-0.609){43}{\rule{0.120pt}{0.587pt}}
\multiput(705.17,385.78)(-23.000,-26.782){2}{\rule{0.400pt}{0.293pt}}
\multiput(680.15,357.92)(-0.737,-0.494){27}{\rule{0.687pt}{0.119pt}}
\multiput(681.57,358.17)(-20.575,-15.000){2}{\rule{0.343pt}{0.400pt}}
\multiput(659.92,341.64)(-0.496,-0.587){43}{\rule{0.120pt}{0.570pt}}
\multiput(660.17,342.82)(-23.000,-25.818){2}{\rule{0.400pt}{0.285pt}}
\multiput(635.56,315.92)(-0.611,-0.495){33}{\rule{0.589pt}{0.119pt}}
\multiput(636.78,316.17)(-20.778,-18.000){2}{\rule{0.294pt}{0.400pt}}
\multiput(797.15,587.92)(-0.737,-0.494){27}{\rule{0.687pt}{0.119pt}}
\multiput(798.57,588.17)(-20.575,-15.000){2}{\rule{0.343pt}{0.400pt}}
\multiput(775.77,572.92)(-0.546,-0.496){39}{\rule{0.538pt}{0.119pt}}
\multiput(776.88,573.17)(-21.883,-21.000){2}{\rule{0.269pt}{0.400pt}}
\multiput(752.92,551.92)(-0.498,-0.496){39}{\rule{0.500pt}{0.119pt}}
\multiput(753.96,552.17)(-19.962,-21.000){2}{\rule{0.250pt}{0.400pt}}
\multiput(731.77,530.92)(-0.546,-0.496){39}{\rule{0.538pt}{0.119pt}}
\multiput(732.88,531.17)(-21.883,-21.000){2}{\rule{0.269pt}{0.400pt}}
\multiput(709.92,508.55)(-0.496,-0.614){41}{\rule{0.120pt}{0.591pt}}
\multiput(710.17,509.77)(-22.000,-25.774){2}{\rule{0.400pt}{0.295pt}}
\multiput(686.68,482.92)(-0.574,-0.496){37}{\rule{0.560pt}{0.119pt}}
\multiput(687.84,483.17)(-21.838,-20.000){2}{\rule{0.280pt}{0.400pt}}
\multiput(663.15,462.92)(-0.737,-0.494){27}{\rule{0.687pt}{0.119pt}}
\multiput(664.57,463.17)(-20.575,-15.000){2}{\rule{0.343pt}{0.400pt}}
\multiput(642.92,446.85)(-0.496,-0.520){43}{\rule{0.120pt}{0.517pt}}
\multiput(643.17,447.93)(-23.000,-22.926){2}{\rule{0.400pt}{0.259pt}}
\multiput(619.92,422.25)(-0.496,-0.706){41}{\rule{0.120pt}{0.664pt}}
\multiput(620.17,423.62)(-22.000,-29.623){2}{\rule{0.400pt}{0.332pt}}
\multiput(596.46,392.92)(-0.639,-0.495){33}{\rule{0.611pt}{0.119pt}}
\multiput(597.73,393.17)(-21.732,-18.000){2}{\rule{0.306pt}{0.400pt}}
\multiput(574.92,373.77)(-0.496,-0.544){41}{\rule{0.120pt}{0.536pt}}
\multiput(575.17,374.89)(-22.000,-22.887){2}{\rule{0.400pt}{0.268pt}}
\multiput(735.26,636.92)(-1.015,-0.492){19}{\rule{0.900pt}{0.118pt}}
\multiput(737.13,637.17)(-20.132,-11.000){2}{\rule{0.450pt}{0.400pt}}
\multiput(711.81,625.93)(-1.484,-0.488){13}{\rule{1.250pt}{0.117pt}}
\multiput(714.41,626.17)(-20.406,-8.000){2}{\rule{0.625pt}{0.400pt}}
\multiput(691.85,617.92)(-0.522,-0.496){39}{\rule{0.519pt}{0.119pt}}
\multiput(692.92,618.17)(-20.923,-21.000){2}{\rule{0.260pt}{0.400pt}}
\multiput(670.92,595.25)(-0.496,-0.706){41}{\rule{0.120pt}{0.664pt}}
\multiput(671.17,596.62)(-22.000,-29.623){2}{\rule{0.400pt}{0.332pt}}
\multiput(648.92,564.85)(-0.496,-0.520){43}{\rule{0.120pt}{0.517pt}}
\multiput(649.17,565.93)(-23.000,-22.926){2}{\rule{0.400pt}{0.259pt}}
\multiput(624.15,541.92)(-0.737,-0.494){27}{\rule{0.687pt}{0.119pt}}
\multiput(625.57,542.17)(-20.575,-15.000){2}{\rule{0.343pt}{0.400pt}}
\multiput(602.77,526.92)(-0.546,-0.496){39}{\rule{0.538pt}{0.119pt}}
\multiput(603.88,527.17)(-21.883,-21.000){2}{\rule{0.269pt}{0.400pt}}
\multiput(580.92,504.47)(-0.496,-0.637){41}{\rule{0.120pt}{0.609pt}}
\multiput(581.17,505.74)(-22.000,-26.736){2}{\rule{0.400pt}{0.305pt}}
\multiput(557.34,477.92)(-0.678,-0.495){31}{\rule{0.641pt}{0.119pt}}
\multiput(558.67,478.17)(-21.669,-17.000){2}{\rule{0.321pt}{0.400pt}}
\multiput(535.92,459.09)(-0.496,-0.752){41}{\rule{0.120pt}{0.700pt}}
\multiput(536.17,460.55)(-22.000,-31.547){2}{\rule{0.400pt}{0.350pt}}
\multiput(513.92,426.78)(-0.496,-0.542){43}{\rule{0.120pt}{0.535pt}}
\multiput(514.17,427.89)(-23.000,-23.890){2}{\rule{0.400pt}{0.267pt}}
\multiput(673.34,651.93)(-1.310,-0.489){15}{\rule{1.122pt}{0.118pt}}
\multiput(675.67,652.17)(-20.671,-9.000){2}{\rule{0.561pt}{0.400pt}}
\multiput(652.56,644.58)(-0.611,0.495){33}{\rule{0.589pt}{0.119pt}}
\multiput(653.78,643.17)(-20.778,18.000){2}{\rule{0.294pt}{0.400pt}}
\multiput(630.04,660.92)(-0.771,-0.494){27}{\rule{0.713pt}{0.119pt}}
\multiput(631.52,661.17)(-21.519,-15.000){2}{\rule{0.357pt}{0.400pt}}
\multiput(607.85,645.92)(-0.522,-0.496){39}{\rule{0.519pt}{0.119pt}}
\multiput(608.92,646.17)(-20.923,-21.000){2}{\rule{0.260pt}{0.400pt}}
\multiput(586.92,623.35)(-0.496,-0.675){43}{\rule{0.120pt}{0.639pt}}
\multiput(587.17,624.67)(-23.000,-29.673){2}{\rule{0.400pt}{0.320pt}}
\multiput(562.85,593.92)(-0.522,-0.496){39}{\rule{0.519pt}{0.119pt}}
\multiput(563.92,594.17)(-20.923,-21.000){2}{\rule{0.260pt}{0.400pt}}
\multiput(540.46,572.92)(-0.639,-0.495){33}{\rule{0.611pt}{0.119pt}}
\multiput(541.73,573.17)(-21.732,-18.000){2}{\rule{0.306pt}{0.400pt}}
\multiput(517.85,554.92)(-0.522,-0.496){39}{\rule{0.519pt}{0.119pt}}
\multiput(518.92,555.17)(-20.923,-21.000){2}{\rule{0.260pt}{0.400pt}}
\multiput(494.98,533.92)(-0.791,-0.494){25}{\rule{0.729pt}{0.119pt}}
\multiput(496.49,534.17)(-20.488,-14.000){2}{\rule{0.364pt}{0.400pt}}
\multiput(474.92,518.78)(-0.496,-0.542){43}{\rule{0.120pt}{0.535pt}}
\multiput(475.17,519.89)(-23.000,-23.890){2}{\rule{0.400pt}{0.267pt}}
\multiput(451.92,493.62)(-0.496,-0.591){41}{\rule{0.120pt}{0.573pt}}
\multiput(452.17,494.81)(-22.000,-24.811){2}{\rule{0.400pt}{0.286pt}}
\put(594,684.17){\rule{4.500pt}{0.400pt}}
\multiput(606.66,685.17)(-12.660,-2.000){2}{\rule{2.250pt}{0.400pt}}
\multiput(592.92,684.00)(-0.496,0.520){43}{\rule{0.120pt}{0.517pt}}
\multiput(593.17,684.00)(-23.000,22.926){2}{\rule{0.400pt}{0.259pt}}
\multiput(568.85,706.92)(-0.522,-0.496){39}{\rule{0.519pt}{0.119pt}}
\multiput(569.92,707.17)(-20.923,-21.000){2}{\rule{0.260pt}{0.400pt}}
\multiput(546.77,685.92)(-0.546,-0.496){39}{\rule{0.538pt}{0.119pt}}
\multiput(547.88,686.17)(-21.883,-21.000){2}{\rule{0.269pt}{0.400pt}}
\multiput(523.85,664.92)(-0.522,-0.496){39}{\rule{0.519pt}{0.119pt}}
\multiput(524.92,665.17)(-20.923,-21.000){2}{\rule{0.260pt}{0.400pt}}
\multiput(501.77,643.92)(-0.546,-0.496){39}{\rule{0.538pt}{0.119pt}}
\multiput(502.88,644.17)(-21.883,-21.000){2}{\rule{0.269pt}{0.400pt}}
\multiput(478.92,622.92)(-0.498,-0.496){41}{\rule{0.500pt}{0.120pt}}
\multiput(479.96,623.17)(-20.962,-22.000){2}{\rule{0.250pt}{0.400pt}}
\multiput(456.77,600.92)(-0.546,-0.496){39}{\rule{0.538pt}{0.119pt}}
\multiput(457.88,601.17)(-21.883,-21.000){2}{\rule{0.269pt}{0.400pt}}
\multiput(433.85,579.92)(-0.522,-0.496){39}{\rule{0.519pt}{0.119pt}}
\multiput(434.92,580.17)(-20.923,-21.000){2}{\rule{0.260pt}{0.400pt}}
\multiput(411.77,558.92)(-0.546,-0.496){39}{\rule{0.538pt}{0.119pt}}
\multiput(412.88,559.17)(-21.883,-21.000){2}{\rule{0.269pt}{0.400pt}}
\multiput(389.92,535.04)(-0.496,-1.076){41}{\rule{0.120pt}{0.955pt}}
\multiput(390.17,537.02)(-22.000,-45.019){2}{\rule{0.400pt}{0.477pt}}
\multiput(546.28,698.93)(-2.380,-0.477){7}{\rule{1.860pt}{0.115pt}}
\multiput(550.14,699.17)(-18.139,-5.000){2}{\rule{0.930pt}{0.400pt}}
\multiput(528.11,695.58)(-1.062,0.492){19}{\rule{0.936pt}{0.118pt}}
\multiput(530.06,694.17)(-21.057,11.000){2}{\rule{0.468pt}{0.400pt}}
\multiput(507.92,703.25)(-0.496,-0.706){41}{\rule{0.120pt}{0.664pt}}
\multiput(508.17,704.62)(-22.000,-29.623){2}{\rule{0.400pt}{0.332pt}}
\multiput(484.46,673.92)(-0.639,-0.495){33}{\rule{0.611pt}{0.119pt}}
\multiput(485.73,674.17)(-21.732,-18.000){2}{\rule{0.306pt}{0.400pt}}
\multiput(462.92,654.77)(-0.496,-0.544){41}{\rule{0.120pt}{0.536pt}}
\multiput(463.17,655.89)(-22.000,-22.887){2}{\rule{0.400pt}{0.268pt}}
\multiput(439.85,631.92)(-0.522,-0.496){39}{\rule{0.519pt}{0.119pt}}
\multiput(440.92,632.17)(-20.923,-21.000){2}{\rule{0.260pt}{0.400pt}}
\multiput(417.77,610.92)(-0.546,-0.496){39}{\rule{0.538pt}{0.119pt}}
\multiput(418.88,611.17)(-21.883,-21.000){2}{\rule{0.269pt}{0.400pt}}
\multiput(395.92,588.77)(-0.496,-0.544){41}{\rule{0.120pt}{0.536pt}}
\multiput(396.17,589.89)(-22.000,-22.887){2}{\rule{0.400pt}{0.268pt}}
\multiput(372.85,565.92)(-0.521,-0.496){41}{\rule{0.518pt}{0.120pt}}
\multiput(373.92,566.17)(-21.924,-22.000){2}{\rule{0.259pt}{0.400pt}}
\multiput(350.92,542.77)(-0.496,-0.544){41}{\rule{0.120pt}{0.536pt}}
\multiput(351.17,543.89)(-22.000,-22.887){2}{\rule{0.400pt}{0.268pt}}
\multiput(328.92,518.64)(-0.496,-0.587){43}{\rule{0.120pt}{0.570pt}}
\multiput(329.17,519.82)(-23.000,-25.818){2}{\rule{0.400pt}{0.285pt}}
\multiput(489.11,715.92)(-1.062,-0.492){19}{\rule{0.936pt}{0.118pt}}
\multiput(491.06,716.17)(-21.057,-11.000){2}{\rule{0.468pt}{0.400pt}}
\multiput(466.54,704.92)(-0.927,-0.492){21}{\rule{0.833pt}{0.119pt}}
\multiput(468.27,705.17)(-20.270,-12.000){2}{\rule{0.417pt}{0.400pt}}
\multiput(444.86,692.92)(-0.827,-0.494){25}{\rule{0.757pt}{0.119pt}}
\multiput(446.43,693.17)(-21.429,-14.000){2}{\rule{0.379pt}{0.400pt}}
\multiput(422.85,678.92)(-0.522,-0.496){39}{\rule{0.519pt}{0.119pt}}
\multiput(423.92,679.17)(-20.923,-21.000){2}{\rule{0.260pt}{0.400pt}}
\multiput(400.04,657.92)(-0.771,-0.494){27}{\rule{0.713pt}{0.119pt}}
\multiput(401.52,658.17)(-21.519,-15.000){2}{\rule{0.357pt}{0.400pt}}
\multiput(377.85,642.92)(-0.522,-0.496){39}{\rule{0.519pt}{0.119pt}}
\multiput(378.92,643.17)(-20.923,-21.000){2}{\rule{0.260pt}{0.400pt}}
\multiput(355.77,621.92)(-0.546,-0.496){39}{\rule{0.538pt}{0.119pt}}
\multiput(356.88,622.17)(-21.883,-21.000){2}{\rule{0.269pt}{0.400pt}}
\multiput(333.92,599.70)(-0.496,-0.567){41}{\rule{0.120pt}{0.555pt}}
\multiput(334.17,600.85)(-22.000,-23.849){2}{\rule{0.400pt}{0.277pt}}
\multiput(311.92,574.85)(-0.496,-0.520){43}{\rule{0.120pt}{0.517pt}}
\multiput(312.17,575.93)(-23.000,-22.926){2}{\rule{0.400pt}{0.259pt}}
\multiput(288.92,550.77)(-0.496,-0.544){41}{\rule{0.120pt}{0.536pt}}
\multiput(289.17,551.89)(-22.000,-22.887){2}{\rule{0.400pt}{0.268pt}}
\multiput(266.92,526.47)(-0.496,-0.637){41}{\rule{0.120pt}{0.609pt}}
\multiput(267.17,527.74)(-22.000,-26.736){2}{\rule{0.400pt}{0.305pt}}
\multiput(1251.82,388.92)(-0.532,-0.500){503}{\rule{0.525pt}{0.120pt}}
\multiput(1252.91,389.17)(-267.910,-253.000){2}{\rule{0.263pt}{0.400pt}}
\multiput(246.00,228.92)(3.986,-0.499){183}{\rule{3.278pt}{0.120pt}}
\multiput(246.00,229.17)(732.195,-93.000){2}{\rule{1.639pt}{0.400pt}}
\put(548,120){\makebox(0,0){\popi{x}{-}}}
\put(1305,241){\makebox(0,0){\popi{y}{-}}}
\put(106,390){\makebox(0,0){\popi{U}{V}}}
\end{picture}

\caption{Grafické znázornenie závislosť potenciálu $U$ na polohe $x$ a $y$}  \label{G_2}
\end{figure}

\begin{figure}
% GNUPLOT: LaTeX picture
\setlength{\unitlength}{0.240900pt}
\ifx\plotpoint\undefined\newsavebox{\plotpoint}\fi
\begin{picture}(1500,900)(0,0)
\sbox{\plotpoint}{\rule[-0.200pt]{0.400pt}{0.400pt}}%
\put(211.0,131.0){\rule[-0.200pt]{4.818pt}{0.400pt}}
\put(191,131){\makebox(0,0)[r]{ 0.015}}
\put(1419.0,131.0){\rule[-0.200pt]{4.818pt}{0.400pt}}
\put(211.0,235.0){\rule[-0.200pt]{4.818pt}{0.400pt}}
\put(191,235){\makebox(0,0)[r]{ 0.02}}
\put(1419.0,235.0){\rule[-0.200pt]{4.818pt}{0.400pt}}
\put(211.0,339.0){\rule[-0.200pt]{4.818pt}{0.400pt}}
\put(191,339){\makebox(0,0)[r]{ 0.025}}
\put(1419.0,339.0){\rule[-0.200pt]{4.818pt}{0.400pt}}
\put(211.0,443.0){\rule[-0.200pt]{4.818pt}{0.400pt}}
\put(191,443){\makebox(0,0)[r]{ 0.03}}
\put(1419.0,443.0){\rule[-0.200pt]{4.818pt}{0.400pt}}
\put(211.0,547.0){\rule[-0.200pt]{4.818pt}{0.400pt}}
\put(191,547){\makebox(0,0)[r]{ 0.035}}
\put(1419.0,547.0){\rule[-0.200pt]{4.818pt}{0.400pt}}
\put(211.0,651.0){\rule[-0.200pt]{4.818pt}{0.400pt}}
\put(191,651){\makebox(0,0)[r]{ 0.04}}
\put(1419.0,651.0){\rule[-0.200pt]{4.818pt}{0.400pt}}
\put(211.0,755.0){\rule[-0.200pt]{4.818pt}{0.400pt}}
\put(191,755){\makebox(0,0)[r]{ 0.045}}
\put(1419.0,755.0){\rule[-0.200pt]{4.818pt}{0.400pt}}
\put(211.0,859.0){\rule[-0.200pt]{4.818pt}{0.400pt}}
\put(191,859){\makebox(0,0)[r]{ 0.05}}
\put(1419.0,859.0){\rule[-0.200pt]{4.818pt}{0.400pt}}
\put(211.0,131.0){\rule[-0.200pt]{0.400pt}{4.818pt}}
\put(211,90){\makebox(0,0){ 0.1}}
\put(211.0,839.0){\rule[-0.200pt]{0.400pt}{4.818pt}}
\put(416.0,131.0){\rule[-0.200pt]{0.400pt}{4.818pt}}
\put(416,90){\makebox(0,0){ 0.2}}
\put(416.0,839.0){\rule[-0.200pt]{0.400pt}{4.818pt}}
\put(620.0,131.0){\rule[-0.200pt]{0.400pt}{4.818pt}}
\put(620,90){\makebox(0,0){ 0.3}}
\put(620.0,839.0){\rule[-0.200pt]{0.400pt}{4.818pt}}
\put(825.0,131.0){\rule[-0.200pt]{0.400pt}{4.818pt}}
\put(825,90){\makebox(0,0){ 0.4}}
\put(825.0,839.0){\rule[-0.200pt]{0.400pt}{4.818pt}}
\put(1030.0,131.0){\rule[-0.200pt]{0.400pt}{4.818pt}}
\put(1030,90){\makebox(0,0){ 0.5}}
\put(1030.0,839.0){\rule[-0.200pt]{0.400pt}{4.818pt}}
\put(1234.0,131.0){\rule[-0.200pt]{0.400pt}{4.818pt}}
\put(1234,90){\makebox(0,0){ 0.6}}
\put(1234.0,839.0){\rule[-0.200pt]{0.400pt}{4.818pt}}
\put(1439.0,131.0){\rule[-0.200pt]{0.400pt}{4.818pt}}
\put(1439,90){\makebox(0,0){ 0.7}}
\put(1439.0,839.0){\rule[-0.200pt]{0.400pt}{4.818pt}}
\put(211.0,131.0){\rule[-0.200pt]{0.400pt}{175.375pt}}
\put(211.0,131.0){\rule[-0.200pt]{295.825pt}{0.400pt}}
\put(1439.0,131.0){\rule[-0.200pt]{0.400pt}{175.375pt}}
\put(211.0,859.0){\rule[-0.200pt]{295.825pt}{0.400pt}}
\put(30,495){\makebox(0,0){\popi{f\(s/D\)}{-}}}
\put(825,29){\makebox(0,0){\popi{s/D}{-}}}
\put(1279,819){\makebox(0,0)[r]{namerané dáta}}
\put(702,339){\makebox(0,0){$+$}}
\put(845,193){\makebox(0,0){$+$}}
\put(989,256){\makebox(0,0){$+$}}
\put(559,277){\makebox(0,0){$+$}}
\put(436,256){\makebox(0,0){$+$}}
\put(784,193){\makebox(0,0){$+$}}
\put(927,173){\makebox(0,0){$+$}}
\put(641,256){\makebox(0,0){$+$}}
\put(498,277){\makebox(0,0){$+$}}
\put(702,859){\makebox(0,0){$+$}}
\put(845,297){\makebox(0,0){$+$}}
\put(927,297){\makebox(0,0){$+$}}
\put(989,297){\makebox(0,0){$+$}}
\put(1050,297){\makebox(0,0){$+$}}
\put(1132,277){\makebox(0,0){$+$}}
\put(1193,277){\makebox(0,0){$+$}}
\put(1275,256){\makebox(0,0){$+$}}
\put(1337,256){\makebox(0,0){$+$}}
\put(1398,256){\makebox(0,0){$+$}}
\put(559,173){\makebox(0,0){$+$}}
\put(498,152){\makebox(0,0){$+$}}
\put(436,131){\makebox(0,0){$+$}}
\put(354,131){\makebox(0,0){$+$}}
\put(293,193){\makebox(0,0){$+$}}
\put(211,693){\makebox(0,0){$+$}}
\put(1349,819){\makebox(0,0){$+$}}
\put(211.0,131.0){\rule[-0.200pt]{0.400pt}{175.375pt}}
\put(211.0,131.0){\rule[-0.200pt]{295.825pt}{0.400pt}}
\put(1439.0,131.0){\rule[-0.200pt]{0.400pt}{175.375pt}}
\put(211.0,859.0){\rule[-0.200pt]{295.825pt}{0.400pt}}
\end{picture}

\caption{Závislosť neznámej funkcie $f\(s/D\)$ od $s/D$}  \label{G_3}
\end{figure}



\section{Diskusia \& Záver}

Na wikipedii\cite{C_2} som dohľadal, Dielektrická pevnosť vzduchu, kde je určená na $"3 MV"$, pre suchý vzduch. Keďže v deň merania pršalo/snežilo tak vzduch nebol suchý a najmä toto prispelo k zníženiu prieraznosti.

V druhej časti pri meraní prierazného napätie sa síce experiment podaril ale pri výpočte $f\(s/D\)$, som niekde neustále robil asi numerickú alebo systematickú chybu a hodnota vychádza o 3 rády menšia ako by mala byť a navyše ani nedopovedá predpokladu $f(0)=1$. Aj napriek tomu je vynesený graf  viď Obr, \ref{G_3}.

V poslednej časti vidíme závislosť napätia na polohe, konkrétne Obr. \ref{G_1} a Obr. \ref{G_2}.


\begin{thebibliography}{2}
\bibitem{C_1}
Dynamika rotačního pohybu [cit. 02.01.2017]Dostupné po prihlásení z Kurz: Fyzikální praktikum I:\url{https://praktikum.fjfi.cvut.cz/pluginfile.php/4352/mod_resource/content/2/Kondenzator_161002.pdf}

\bibitem{C_2}
Dielektrická pevnost [cit 14.1. 2017] dostupné na: \url{https://cs.wikipedia.org/wiki/Dielektrick\%C3\%A1_pevnost}

\end{thebibliography}

\end{document}


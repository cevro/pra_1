\documentclass[a4paper,10pt]{article}
%\usepackage[IL2]{fontenc}
\usepackage[utf8x]{inputenc}
\usepackage[czech]{babel}
\usepackage{listings}  
\usepackage{amsfonts,amsmath,amssymb,graphicx,color}
%\usepackage[total={17cm,27cm}, top=2cm, left=2cm, includefoot]{geometry}
%\usepackage{fancyhdr}
\usepackage{fkssugar}
\usepackage{hyperref}

%\usepackage{caption}
\renewcommand{\popi}[2]{$\frac{#1}{[\jd{#2}]}$}
\renewcommand{\figurename}{Obr.}
\addto\captionsczech{\renewcommand{\figurename}{Obr.}}
\addto\captionsczech{\renewcommand{\tablename}{Tab.}}

\begin{document}
\def\mean#1{\left< #1 \right>}
\noindent
{\large Fyzikální praktikum 1.} \hfil {\large FJFI ČVUT V Praze}\\
\noindent
{\large\textbf{pracovní úkol \# 2}}
\begin{center}
{\large\textit{Dynamika rotačního pohybu}}
\end{center}
\noindent
\rule{\textwidth}{1px}
\vspace{\baselineskip}

\emph{Michal Červeňák}
\par
\vspace{\baselineskip}
\begin{minipage}[l]{0.5\textwidth}%
\textit{dátum merania:}~16.12. 2016\\%
%\vspace{\baselineskip}%
\par%
\noindent%
\textit{skupina:}~4\\%
%\vspace{\baselineskip}%
\par%
\noindent%
\textit{Klasifikace:}\dotfill\\%
\end{minipage}

\section{Pracovní úkol}

\begin{enumerate}
\item DU: V přípravě odvoď ťe vzorec pro výpočet momentu setrvačnosti válce a dutého
válce. Vyjděte z definice a odvodťe vztahy (1) a (2) \cite{1}.
\item  Změřte momenty setrvačnosti přiložených rotačních objektů experimentálně a porovnejte je s
hodnotami z teoretických vzorců. Použijte disk, disk + prstenec a pomocí nich stanovte moment
setrvačnosti samotného prstence.
\item  Změřte moment setrvačnosti disku, umístěného na dráze mimo osu rotace a pomocí výsledků
z předchozího úkolu ověřte platnost Steinerovy věty.
\item  Ověřte zákon zachování momentu hybnosti. Do protokolu přiložte graf závislosti úhové rychlosti
rotace na čase.
\item  Změřte rychlost precese gyroskopu jak přímo senzorem, tak i nepřímo z měření rychlosti rotace
disku. Obě hodnoty porovnejte.
\end{enumerate}



\section{Pomôcky}
”A”base rotational adapter PASCO CI-6690, přídavny disk a prstenec, rotační dráha
s dvěma závažími, Gyroskop PASCO ME-8960, přídavn´y disk gyroskopu ME-8961, dva rotační
senzory PASCO PS-2120, USB link PASCO 2100, PC, program pro datový sběr Data
Studio, nit, posuvné měrítko, stojan s kladkou, milimetrov´e měrítko, váhy.

\section{Teória}

Moment zotrvačnosť rotujúceho disku vypočítame ako
\eq{
I = \frac{1}{2} M R^2 \,, \lbl{R_10}
}
kde $M$ je hmotnosť disku a $R$ je jeho polomer.
 
Moment zotrvačnosť rotujúcej kruhového prstenca vypočítame ako
\eq{
I = \frac{1}{2} M \(R_1^2+R_2^2\) \,, \lbl{R_11}
}
kde $M$ je hmotnosť disku a $R_{1,2}$ sú jeho polomery.

Steinerová věta nám hovorí ako vypočítať moment zotrvačnosti telesa,
keď poznáme jeho hmotnosť a moment zotrvačnosti okolo osi prechádzajúcej ťažiskom $I_0$, ktorá je vzdialená od osi otáčania o $a$
\eq{
I = I_0 + M a^2 \,. \lbl{R_12}
}

Vzťah re aparatúru na výpočet momentu zotrvačnosti z uhlového zrýchlenia $\epsilon$ 
spôsobeného podaním závažia o hmotnosti $m$ pri tiažovom zrýchlení $g="9.81 m\cdot s^{-2}"$
a kladke o polomere $r$ znie nasledovne
\eq{
I= r m\(\frac{g}{\epsilon}-r\) \,. \lbl{R_1}
}

K výpočtu teoretickej presecie $\Omega$, z momentu zotrvačnosti rotujúceho disku $I$, hmotnosti protizávažia $m$ a vzdialenosti závažia od osi precesie $d$ a úhlovej rýchlosti disku $\omega$
\eq{
\Omega = \frac{m g d}{I \omega} \lbl{R_3}
}

Moment hybnosti $L$ vypočítame ako
\eq{
L= \omega I \,. \lbl{R_2}
}

\subsubsection{Spracovanie chýb merania}

Označme $\mean{t}$ aritmetický priemer nameraných hodnôt $t_i$, a $\Delta t$ hodnotu $\mean{t}-t$, pričom 
\eq{
\mean{t} = \frac{1}{n}\sum_{i=1}^n t_i \,, \lbl{SCH_1}
}  
a chybu aritmetického priemeru 
\eq{
  \sigma_0=\sqrt{\frac{\sum_{i=1}^n \(t_i - \mean{t}\)^2}{n\(n-1\)}}\,, \lbl{SCH_2}
}
pričom $n$ je počet meraní.

\section{Postup merania}
\begin{enumerate}
\item Pomocou pusuvného meradla a metru sme zmerali všetky polomery (priemery) disku prstenca.
\item Na digitálnych váhach sme určili hmotnosť disku aj prstenca.
\item Odvážili sme aj obe použíté závažia.
\item Na kladku na aparatúre sme namotali lanko, a na jeho voľný koniec sme umiestnili pripravené závažie. \label{OP_1}
\item Následne závažie spustili a merali uhlové zrýchlenie disku 
\item Postup opakujeme pre samostatný disk, disk s prstencom, vychýlený disk a samotnú aparatúru na vychýlenie.
\item Následne boli na aparatúru umiestené 2 rovnaké závažia 
\item Pre 2 zvolené symetrické polohy závaží voči stredu boli zmerané podľa \ref{OP_1} uhlové zrýchlenia
\item Závažia boli priviazané lankom tak aby sa s nimi počas pohybu dalo pohybovať medzi krajnými polohami.
\item Aparatúra sa roztočila a niekoľko krát sa zmenila poloha závaží a pri tom sme zaznamenávali závislosť uhlovej rýchlosti na čase.
\end{enumerate}
\begin{enumerate}
\item Pomocou posuvného mradla boli odmerané všetky potrebné rozmery na giroskope $d,r_{1,2}$
\item Na koniec girsokopu boli umiestnené závažie.
\item Disk bol roztočený a pritom sa merali hodnoty precesie a uhlová rýchlosť disku.
\end{enumerate}

\section{Výsledky merania}
V tabuľke Tab. \ref{T_1} sú namerané hodnoty jednotlivých uhlových zrýchlení. 
Z nich pomocou vzťahov \ref{R_1} a \ref{SCH_1} boli vypočítané hodnoty momentu zotrvačnosti $I$ pre jednotlivé experimenty, pričom $r="3.11 cm"$.
Disk má polomer $R= "11.45 cm" $ a obruč má polomery $R_1="6.3 cm"$ a $R_2="6.5 cm"$.
Moment zotrvačnosti samostatného disku s hmotnosťou $m= "1429 g"$
sme určili
\eq{
I_{D} = "\(0.020\pm0.001\)kg m^{2}"\,,
}
pričom $m = "50 g"$.

Moment zotrvačnosti disku  prstencom o hmotnosti $m_P= "1428 g"$ 
sme určili
\eq{
I_{D+P} = "\(0.032\pm0.002\)kg m^{2}"\,,
}
pričom $m = "50 g"$.

Moment zotrvačnosti disku vychýleného od osi otáčania o vzdialenosť $a= "15 cm" $ 
sme určili
\eq{
I_{PV+A} = "\(0.093\pm0.004\)kg m^{2}"\,,
}
pričom $m = "200 g"$.

Moment zotrvačnosti aparatúry na vychýlenie disku 
sme určili
\eq{
I_{A} = "\(0.023\pm0.001\)kg m^{2}"\,,
}
pričom $m = "200 g"$.



\begin{table}[h]
\begin{center}
\begin{tabular}{| c | c | c | c |}
\hline
\popi{\epsilon_{D}}{\frac{rad}{s^2}} & \popi{\epsilon_{D+P}}{\frac{rad}{s^2}} & \popi{\epsilon_{PV+A}}{\frac{rad}{s^2}} & \popi{\epsilon_{A}}{\frac{rad}{s^2}}\\
\hline
$0.723$ & $0.480$ & $0.657$ & $2.53$\\
$0.744$ & $0.495$ & $0.651$ & $2.53$\\
$0.768$ & $0.475$ & $0.634$ & $2.48$\\
$0.749$ & $0.488$ & $0.663$ & $2.55$\\
$0.810$ & $0.475$ & $0.655$ & $2.88$\\
$0.813$ & $0.448$ & $0.633$ & - \\
$0.781$ & $0.473$ & $0.659$ & - \\
$0.777$ & $0.482$ & $0.679$ & - \\
$0.721$ & $0.482$ & $0.664$ & - \\
$0.767$ & $0.479$ & $0.670$ & - \\
\hline
\end{tabular}
\caption{
Nameraná hodnoty uhlového zrýchlenia $\epsilon_*$ pre jednotlivé merania.
} \label{T_1}
\end{center}
\end{table}


V tabuľke Tab. \ref{T_2} sú zaznamenané namerané hodnoty uhlového zrýchlenia pre závažia v jednotlivých polohách, 
$\epsilon_{17}$ pre vzdialenosť od stredu $r="17 cm"$ a $\epsilon_{7}$ pre vzdialenosť od stredu $r="7 cm"$.
Z nich podľa vzťahu \ref{R_1} boli vypočítané jednotlivé hodnoty 
momentov zotrvačnosti $I_{17} = "\(0.013\pm0.0005\) kg\cdot m^2"$ a $I="\(0.0081\pm0.0001\) kg\cdot m^2"$.
Tie boli použité na výpočet jednotlivých hodnôt momentu hybnosti z nameraných dát v tabuľke Tab. \ref{T_3}.


\begin{table}[h]
\begin{center}
\begin{tabular}{| c | c |}
\hline
\popi{\epsilon_{17}}{\frac{rad}{s^2}} & \popi{\epsilon_{7}}{\frac{rad}{s^2}}\\
\hline
$1.21$ & $1.91$\\
$1.14$ & $1.84$\\
$1.22$ & $1.85$\\
$1.23$ & $1.86$\\
$1.27$ & $1.84$\\
\hline
\end{tabular}
\caption{
Nameraná hodnoty uhlového zrýchlenia $\epsilon_*$ pre jednotlivé polohy závaží.
} \label{T_2}
\end{center}
\end{table}

\begin{table}[h]
\begin{center}
\begin{tabular}{| c | c | c | c |}
\hline
\popi{\omega_{17}}{\frac{rad}{s^2}} & \popi{\omega_{7}}{\frac{rad}{s^2}} & \popi{L_{17}}{\frac{rad kg}{s^2}} & \popi{L_{7}}{\frac{rad kg}{s^2}}\\
\hline
$0.0796$ & $0.197$ & $ 0.10$ & $1.61$\\
$0.056$  & $0.15$  & $ 0.70$ & $1.22$\\
$0.108$  & $0.152$ & $ 1.35$ & $1.27$\\
$1.9$    & $0.303$ & $23.81$ & $2.47$\\
$1.54$   & $0.351$ & $19.30$ & $2.86$\\
$1.1102$ & $0.617$ & $13.91$ & $5.03$\\
$0.0995$ & $0.152$ & $ 1.25$ & $1.24$\\
$0.0571$ & $0.204$ & $ 0.73$ & $1.66$\\
$0.0565$ & $0.324$ & $ 0.71$ & $2.64$\\
$0.0788$ & $0.189$ & $ 0.99$ & $1.54$\\
$0.199$  & $0.163$ & $ 2.49$ & $1.33$\\
\hline
\end{tabular}
\caption{
Namerané hodnoty uhlových rýchlosti pre jednotlivé vzdialenosti $\omega$ 
a vypočítané hodnoty momentu hybnosti podľa vzťahu \ref{R_2} $L$.
} \label{T_3}
\end{center}
\end{table}

Z veľkosti kladiek na prevod $d_1="5.6 cm"$ a $d_2="3.11 cm"$ sme určili prevodný pomer medzi nameranými dátami uhlovej rýchlosti rotácie disku giroskopu.
Vzdialenosť  sme určili ako súčet vzdelaností po disk od disku a hrúbku disku, teda $d = "\(3\pm0.1+7\pm0.5+13.5\pm0.5\) cm" = "23.5\pm1.1 cm"$.

V tabuľke Tab. \ref{T_4} sú namerané a vypočítane hodnoty precesie giroskopu a ich porovnanie.

\begin{table}[h]
\begin{center}
\begin{tabular}{| c | c | c | c |}
\hline
\popi{\omega}{\frac{rad}{s^2}} & \popi{\Omega_m}{s^{-1}} & \popi{\Omega_v}{s^{-1}} & \popi{\Omega_m/\Omega_v}{-}\\
\hline
$-0.265$ & $-40.891$ & $-0.25$ & $1.07$\\
$ 0.310$ & $ 27.000$ & $ 0.38$ & $0.83$\\
$-0.364$ & $-53.724$ & $-0.19$ & $1.93$\\
$ 0.281$ & $ 46.729$ & $ 0.22$ & $1.30$\\
$ 0.384$ & $ 33.393$ & $ 0.30$ & $1.27$\\
\hline
\end{tabular}
\caption{
Namerané hodnoty uhlovej rýchlosti $\omega$ pred prepočtom, 
v závislosti na hodnote precesie $\Omega_m$ , 
vypočítané hodnoty precesie podľa vzťahu \ref{R_3} $\Omega_v$ a 
ich pomer.
} \label{T_4}
\end{center}
\end{table}




\begin{figure}
% GNUPLOT: LaTeX picture
\setlength{\unitlength}{0.240900pt}
\ifx\plotpoint\undefined\newsavebox{\plotpoint}\fi
\begin{picture}(1500,900)(0,0)
\sbox{\plotpoint}{\rule[-0.200pt]{0.400pt}{0.400pt}}%
\put(191.0,131.0){\rule[-0.200pt]{4.818pt}{0.400pt}}
\put(171,131){\makebox(0,0)[r]{ 6.5}}
\put(1419.0,131.0){\rule[-0.200pt]{4.818pt}{0.400pt}}
\put(191.0,212.0){\rule[-0.200pt]{4.818pt}{0.400pt}}
\put(171,212){\makebox(0,0)[r]{ 7}}
\put(1419.0,212.0){\rule[-0.200pt]{4.818pt}{0.400pt}}
\put(191.0,293.0){\rule[-0.200pt]{4.818pt}{0.400pt}}
\put(171,293){\makebox(0,0)[r]{ 7.5}}
\put(1419.0,293.0){\rule[-0.200pt]{4.818pt}{0.400pt}}
\put(191.0,374.0){\rule[-0.200pt]{4.818pt}{0.400pt}}
\put(171,374){\makebox(0,0)[r]{ 8}}
\put(1419.0,374.0){\rule[-0.200pt]{4.818pt}{0.400pt}}
\put(191.0,455.0){\rule[-0.200pt]{4.818pt}{0.400pt}}
\put(171,455){\makebox(0,0)[r]{ 8.5}}
\put(1419.0,455.0){\rule[-0.200pt]{4.818pt}{0.400pt}}
\put(191.0,535.0){\rule[-0.200pt]{4.818pt}{0.400pt}}
\put(171,535){\makebox(0,0)[r]{ 9}}
\put(1419.0,535.0){\rule[-0.200pt]{4.818pt}{0.400pt}}
\put(191.0,616.0){\rule[-0.200pt]{4.818pt}{0.400pt}}
\put(171,616){\makebox(0,0)[r]{ 9.5}}
\put(1419.0,616.0){\rule[-0.200pt]{4.818pt}{0.400pt}}
\put(191.0,697.0){\rule[-0.200pt]{4.818pt}{0.400pt}}
\put(171,697){\makebox(0,0)[r]{ 10}}
\put(1419.0,697.0){\rule[-0.200pt]{4.818pt}{0.400pt}}
\put(191.0,778.0){\rule[-0.200pt]{4.818pt}{0.400pt}}
\put(171,778){\makebox(0,0)[r]{ 10.5}}
\put(1419.0,778.0){\rule[-0.200pt]{4.818pt}{0.400pt}}
\put(191.0,859.0){\rule[-0.200pt]{4.818pt}{0.400pt}}
\put(171,859){\makebox(0,0)[r]{ 11}}
\put(1419.0,859.0){\rule[-0.200pt]{4.818pt}{0.400pt}}
\put(191.0,131.0){\rule[-0.200pt]{0.400pt}{4.818pt}}
\put(191,90){\makebox(0,0){ 5}}
\put(191.0,839.0){\rule[-0.200pt]{0.400pt}{4.818pt}}
\put(441.0,131.0){\rule[-0.200pt]{0.400pt}{4.818pt}}
\put(441,90){\makebox(0,0){ 6}}
\put(441.0,839.0){\rule[-0.200pt]{0.400pt}{4.818pt}}
\put(690.0,131.0){\rule[-0.200pt]{0.400pt}{4.818pt}}
\put(690,90){\makebox(0,0){ 7}}
\put(690.0,839.0){\rule[-0.200pt]{0.400pt}{4.818pt}}
\put(940.0,131.0){\rule[-0.200pt]{0.400pt}{4.818pt}}
\put(940,90){\makebox(0,0){ 8}}
\put(940.0,839.0){\rule[-0.200pt]{0.400pt}{4.818pt}}
\put(1189.0,131.0){\rule[-0.200pt]{0.400pt}{4.818pt}}
\put(1189,90){\makebox(0,0){ 9}}
\put(1189.0,839.0){\rule[-0.200pt]{0.400pt}{4.818pt}}
\put(1439.0,131.0){\rule[-0.200pt]{0.400pt}{4.818pt}}
\put(1439,90){\makebox(0,0){ 10}}
\put(1439.0,839.0){\rule[-0.200pt]{0.400pt}{4.818pt}}
\put(191.0,131.0){\rule[-0.200pt]{0.400pt}{175.375pt}}
\put(191.0,131.0){\rule[-0.200pt]{300.643pt}{0.400pt}}
\put(1439.0,131.0){\rule[-0.200pt]{0.400pt}{175.375pt}}
\put(191.0,859.0){\rule[-0.200pt]{300.643pt}{0.400pt}}
\put(30,495){\makebox(0,0){\popi{\omega}{rad \cdot s^{-2}}}}
\put(815,29){\makebox(0,0){\popi{t}{s}}}
\put(1279,819){\makebox(0,0)[r]{namerná dáta}}
\put(193,676){\makebox(0,0){$+$}}
\put(208,799){\makebox(0,0){$+$}}
\put(223,809){\makebox(0,0){$+$}}
\put(238,802){\makebox(0,0){$+$}}
\put(253,801){\makebox(0,0){$+$}}
\put(267,814){\makebox(0,0){$+$}}
\put(282,796){\makebox(0,0){$+$}}
\put(297,788){\makebox(0,0){$+$}}
\put(312,785){\makebox(0,0){$+$}}
\put(327,786){\makebox(0,0){$+$}}
\put(341,775){\makebox(0,0){$+$}}
\put(356,776){\makebox(0,0){$+$}}
\put(371,775){\makebox(0,0){$+$}}
\put(386,773){\makebox(0,0){$+$}}
\put(401,786){\makebox(0,0){$+$}}
\put(416,783){\makebox(0,0){$+$}}
\put(431,764){\makebox(0,0){$+$}}
\put(446,759){\makebox(0,0){$+$}}
\put(461,764){\makebox(0,0){$+$}}
\put(476,746){\makebox(0,0){$+$}}
\put(492,744){\makebox(0,0){$+$}}
\put(507,746){\makebox(0,0){$+$}}
\put(522,747){\makebox(0,0){$+$}}
\put(537,747){\makebox(0,0){$+$}}
\put(552,762){\makebox(0,0){$+$}}
\put(568,743){\makebox(0,0){$+$}}
\put(583,739){\makebox(0,0){$+$}}
\put(598,725){\makebox(0,0){$+$}}
\put(614,595){\makebox(0,0){$+$}}
\put(632,424){\makebox(0,0){$+$}}
\put(653,226){\makebox(0,0){$+$}}
\put(675,217){\makebox(0,0){$+$}}
\put(697,218){\makebox(0,0){$+$}}
\put(719,217){\makebox(0,0){$+$}}
\put(742,215){\makebox(0,0){$+$}}
\put(764,214){\makebox(0,0){$+$}}
\put(786,212){\makebox(0,0){$+$}}
\put(809,209){\makebox(0,0){$+$}}
\put(831,209){\makebox(0,0){$+$}}
\put(854,205){\makebox(0,0){$+$}}
\put(876,202){\makebox(0,0){$+$}}
\put(899,201){\makebox(0,0){$+$}}
\put(922,204){\makebox(0,0){$+$}}
\put(944,202){\makebox(0,0){$+$}}
\put(967,201){\makebox(0,0){$+$}}
\put(989,199){\makebox(0,0){$+$}}
\put(1012,197){\makebox(0,0){$+$}}
\put(1035,194){\makebox(0,0){$+$}}
\put(1058,194){\makebox(0,0){$+$}}
\put(1080,191){\makebox(0,0){$+$}}
\put(1103,191){\makebox(0,0){$+$}}
\put(1126,192){\makebox(0,0){$+$}}
\put(1148,241){\makebox(0,0){$+$}}
\put(1169,379){\makebox(0,0){$+$}}
\put(1187,589){\makebox(0,0){$+$}}
\put(1204,686){\makebox(0,0){$+$}}
\put(1219,665){\makebox(0,0){$+$}}
\put(1235,662){\makebox(0,0){$+$}}
\put(1251,662){\makebox(0,0){$+$}}
\put(1268,657){\makebox(0,0){$+$}}
\put(1284,655){\makebox(0,0){$+$}}
\put(1300,650){\makebox(0,0){$+$}}
\put(1316,649){\makebox(0,0){$+$}}
\put(1332,647){\makebox(0,0){$+$}}
\put(1348,647){\makebox(0,0){$+$}}
\put(1364,662){\makebox(0,0){$+$}}
\put(1381,637){\makebox(0,0){$+$}}
\put(1397,634){\makebox(0,0){$+$}}
\put(1413,633){\makebox(0,0){$+$}}
\put(1430,624){\makebox(0,0){$+$}}
\put(1349,819){\makebox(0,0){$+$}}
\put(191.0,131.0){\rule[-0.200pt]{0.400pt}{175.375pt}}
\put(191.0,131.0){\rule[-0.200pt]{300.643pt}{0.400pt}}
\put(1439.0,131.0){\rule[-0.200pt]{0.400pt}{175.375pt}}
\put(191.0,859.0){\rule[-0.200pt]{300.643pt}{0.400pt}}
\end{picture}

\caption{Závislosť uhlovej $\omega$ frekvencie na čase $t$ pri zmene polohy závaží.}  \label{G_1}
\end{figure}


\begin{figure}
% GNUPLOT: LaTeX picture
\setlength{\unitlength}{0.240900pt}
\ifx\plotpoint\undefined\newsavebox{\plotpoint}\fi
\begin{picture}(1500,900)(0,0)
\sbox{\plotpoint}{\rule[-0.200pt]{0.400pt}{0.400pt}}%
\put(171.0,131.0){\rule[-0.200pt]{4.818pt}{0.400pt}}
\put(151,131){\makebox(0,0)[r]{ 0.1}}
\put(1419.0,131.0){\rule[-0.200pt]{4.818pt}{0.400pt}}
\put(171.0,235.0){\rule[-0.200pt]{4.818pt}{0.400pt}}
\put(151,235){\makebox(0,0)[r]{ 0.2}}
\put(1419.0,235.0){\rule[-0.200pt]{4.818pt}{0.400pt}}
\put(171.0,339.0){\rule[-0.200pt]{4.818pt}{0.400pt}}
\put(151,339){\makebox(0,0)[r]{ 0.3}}
\put(1419.0,339.0){\rule[-0.200pt]{4.818pt}{0.400pt}}
\put(171.0,443.0){\rule[-0.200pt]{4.818pt}{0.400pt}}
\put(151,443){\makebox(0,0)[r]{ 0.4}}
\put(1419.0,443.0){\rule[-0.200pt]{4.818pt}{0.400pt}}
\put(171.0,547.0){\rule[-0.200pt]{4.818pt}{0.400pt}}
\put(151,547){\makebox(0,0)[r]{ 0.5}}
\put(1419.0,547.0){\rule[-0.200pt]{4.818pt}{0.400pt}}
\put(171.0,651.0){\rule[-0.200pt]{4.818pt}{0.400pt}}
\put(151,651){\makebox(0,0)[r]{ 0.6}}
\put(1419.0,651.0){\rule[-0.200pt]{4.818pt}{0.400pt}}
\put(171.0,755.0){\rule[-0.200pt]{4.818pt}{0.400pt}}
\put(151,755){\makebox(0,0)[r]{ 0.7}}
\put(1419.0,755.0){\rule[-0.200pt]{4.818pt}{0.400pt}}
\put(171.0,859.0){\rule[-0.200pt]{4.818pt}{0.400pt}}
\put(151,859){\makebox(0,0)[r]{ 0.8}}
\put(1419.0,859.0){\rule[-0.200pt]{4.818pt}{0.400pt}}
\put(171.0,131.0){\rule[-0.200pt]{0.400pt}{4.818pt}}
\put(171,90){\makebox(0,0){ 0}}
\put(171.0,839.0){\rule[-0.200pt]{0.400pt}{4.818pt}}
\put(382.0,131.0){\rule[-0.200pt]{0.400pt}{4.818pt}}
\put(382,90){\makebox(0,0){ 1}}
\put(382.0,839.0){\rule[-0.200pt]{0.400pt}{4.818pt}}
\put(594.0,131.0){\rule[-0.200pt]{0.400pt}{4.818pt}}
\put(594,90){\makebox(0,0){ 2}}
\put(594.0,839.0){\rule[-0.200pt]{0.400pt}{4.818pt}}
\put(805.0,131.0){\rule[-0.200pt]{0.400pt}{4.818pt}}
\put(805,90){\makebox(0,0){ 3}}
\put(805.0,839.0){\rule[-0.200pt]{0.400pt}{4.818pt}}
\put(1016.0,131.0){\rule[-0.200pt]{0.400pt}{4.818pt}}
\put(1016,90){\makebox(0,0){ 4}}
\put(1016.0,839.0){\rule[-0.200pt]{0.400pt}{4.818pt}}
\put(1228.0,131.0){\rule[-0.200pt]{0.400pt}{4.818pt}}
\put(1228,90){\makebox(0,0){ 5}}
\put(1228.0,839.0){\rule[-0.200pt]{0.400pt}{4.818pt}}
\put(1439.0,131.0){\rule[-0.200pt]{0.400pt}{4.818pt}}
\put(1439,90){\makebox(0,0){ 6}}
\put(1439.0,839.0){\rule[-0.200pt]{0.400pt}{4.818pt}}
\put(171.0,131.0){\rule[-0.200pt]{0.400pt}{175.375pt}}
\put(171.0,131.0){\rule[-0.200pt]{305.461pt}{0.400pt}}
\put(1439.0,131.0){\rule[-0.200pt]{0.400pt}{175.375pt}}
\put(171.0,859.0){\rule[-0.200pt]{305.461pt}{0.400pt}}
\put(30,495){\makebox(0,0){\popi{\omega}{rad \cdot s^{-2}}}}
\put(805,29){\makebox(0,0){\popi{t}{s}}}
\put(1279,819){\makebox(0,0)[r]{namerná dáta}}
\put(182,485){\makebox(0,0){$+$}}
\put(203,550){\makebox(0,0){$+$}}
\put(224,631){\makebox(0,0){$+$}}
\put(245,697){\makebox(0,0){$+$}}
\put(266,729){\makebox(0,0){$+$}}
\put(287,795){\makebox(0,0){$+$}}
\put(308,811){\makebox(0,0){$+$}}
\put(330,811){\makebox(0,0){$+$}}
\put(351,779){\makebox(0,0){$+$}}
\put(372,729){\makebox(0,0){$+$}}
\put(393,713){\makebox(0,0){$+$}}
\put(414,648){\makebox(0,0){$+$}}
\put(435,566){\makebox(0,0){$+$}}
\put(456,501){\makebox(0,0){$+$}}
\put(477,436){\makebox(0,0){$+$}}
\put(499,354){\makebox(0,0){$+$}}
\put(520,321){\makebox(0,0){$+$}}
\put(541,272){\makebox(0,0){$+$}}
\put(562,239){\makebox(0,0){$+$}}
\put(583,223){\makebox(0,0){$+$}}
\put(604,223){\makebox(0,0){$+$}}
\put(625,207){\makebox(0,0){$+$}}
\put(647,223){\makebox(0,0){$+$}}
\put(668,239){\makebox(0,0){$+$}}
\put(689,272){\makebox(0,0){$+$}}
\put(710,321){\makebox(0,0){$+$}}
\put(731,354){\makebox(0,0){$+$}}
\put(752,402){\makebox(0,0){$+$}}
\put(773,451){\makebox(0,0){$+$}}
\put(794,485){\makebox(0,0){$+$}}
\put(816,517){\makebox(0,0){$+$}}
\put(837,550){\makebox(0,0){$+$}}
\put(858,582){\makebox(0,0){$+$}}
\put(879,566){\makebox(0,0){$+$}}
\put(900,582){\makebox(0,0){$+$}}
\put(921,582){\makebox(0,0){$+$}}
\put(942,533){\makebox(0,0){$+$}}
\put(964,533){\makebox(0,0){$+$}}
\put(985,501){\makebox(0,0){$+$}}
\put(1006,468){\makebox(0,0){$+$}}
\put(1027,419){\makebox(0,0){$+$}}
\put(1048,387){\makebox(0,0){$+$}}
\put(1069,354){\makebox(0,0){$+$}}
\put(1090,354){\makebox(0,0){$+$}}
\put(1111,305){\makebox(0,0){$+$}}
\put(1133,288){\makebox(0,0){$+$}}
\put(1154,288){\makebox(0,0){$+$}}
\put(1175,272){\makebox(0,0){$+$}}
\put(1196,288){\makebox(0,0){$+$}}
\put(1217,288){\makebox(0,0){$+$}}
\put(1238,321){\makebox(0,0){$+$}}
\put(1259,337){\makebox(0,0){$+$}}
\put(1281,337){\makebox(0,0){$+$}}
\put(1302,370){\makebox(0,0){$+$}}
\put(1323,402){\makebox(0,0){$+$}}
\put(1344,402){\makebox(0,0){$+$}}
\put(1365,436){\makebox(0,0){$+$}}
\put(1386,436){\makebox(0,0){$+$}}
\put(1407,451){\makebox(0,0){$+$}}
\put(1428,468){\makebox(0,0){$+$}}
\put(1349,819){\makebox(0,0){$+$}}
\put(171.0,131.0){\rule[-0.200pt]{0.400pt}{175.375pt}}
\put(171.0,131.0){\rule[-0.200pt]{305.461pt}{0.400pt}}
\put(1439.0,131.0){\rule[-0.200pt]{0.400pt}{175.375pt}}
\put(171.0,859.0){\rule[-0.200pt]{305.461pt}{0.400pt}}
\end{picture}

\caption{Závislosť uhlovej $\omega$ frekvencie na čase $t$ pri precesí.}  \label{G_1}
\end{figure}



\section{Diskusia \& Záver}
Podľa vzťahu \ref{R_10} sme vypočítali moment zotrvačnosti $I_D="93 g m^2"$ a  a podľa vzťahu \ref{R_11} obruče $I = "5,32 g m^2"$. Kde jasne vidíme, že hodnoty sa od nami nameraných výrazne líšia.
Podľa vzťahu \ref{R_11} sme sme určili teoretický $I_0$ ako $I_0 = I_{PV+A} - I_{A} - a^2 M = "0.0378 kg m^2" $, čo je opäť v rozpore s nameranými hodnotami. 

Pri overovaní ZZMH sa opäť vo výsledkoch ukazuje veľká nepresnosť a nezrovnalosť nameraných dát s teóriu.
Vyzerá to že naše meranie zaťažené systematickou chybou, ktorá podľa môjho predpokladu vznikla pri určovaní polomeru kladky. Je to jediný parameter, ktorý sa vyskytuje všade vo výpočtoch.
Pri takejto výraznej (asi) systematickej chybe je zahŕňať do odhadov nepresností aj iné chyby skoro zbytočné ale predsa ich tu spomeniem.
Predovšetkým sa jedná o trenie pri otáčaní aparatúry, hmotné lanko, nezapočítaný moment zotrvačnosti podstavca, namotávanie nitky a tým sa zväčšuje polomer kladky, nezapočítaný moment zotrvačnosti druhej kladky.


Naopak pri overovaní precesie a vzťahov pre precesiu giroskopu, vidíme veľmi peknú zhodu dát sa teóriou. 
Tu opäť môžeme diskutovať o vzniku nepresností merania. Meranie hodnoty $d$ bolo nepriame a rozdelená na cca 3 sub-merania každé s presnosťou $"0.5 cm"$, gumička na prevode dosť preklzávala, a teda neprenášala všetky otáčky. 


\begin{thebibliography}{2}
\bibitem{C_1}
Dynamika rotačního pohybu [cit. 02.01.2017]Dostupné po prihlásení z Kurz: Fyzikální praktikum I:\url{https://praktikum.fjfi.cvut.cz/pluginfile.php/133/mod_resource/content/6/Rotace_161003.pdf}

\end{thebibliography}

\end{document}


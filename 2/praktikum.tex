\documentclass[a4paper,10pt]{article}
%\usepackage[IL2]{fontenc}
\usepackage[utf8x]{inputenc}
\usepackage[czech]{babel}
\usepackage{listings}  
\usepackage{amsfonts,amsmath,amssymb,graphicx,color}
%\usepackage[total={17cm,27cm}, top=2cm, left=2cm, includefoot]{geometry}
%\usepackage{fancyhdr}
\usepackage{fkssugar}
\usepackage{hyperref}

%\usepackage{caption}
\renewcommand{\popi}[2]{$\frac{#1}{[\jd{#2}]}$}
\renewcommand{\figurename}{Obr.}
\addto\captionsczech{\renewcommand{\figurename}{Obr.}}
\addto\captionsczech{\renewcommand{\tablename}{Tab.}}

\begin{document}
\def\mean#1{\left< #1 \right>}
\noindent
{\large Fyzikální praktikum 1.} \hfil {\large FJFI ČVUT V Praze}\\
\noindent
{\large\textbf{pracovní úkol \# 2}}
\begin{center}
{\large\textit{Dynamika rotačního pohybu}}
\end{center}
\noindent
\rule{\textwidth}{1px}
\vspace{\baselineskip}

\emph{Michal Červeňák}
\par
\vspace{\baselineskip}
\begin{minipage}[l]{0.5\textwidth}%
\textit{dátum merania:}~16.12. 2016\\%
%\vspace{\baselineskip}%
\par%
\noindent%
\textit{skupina:}~4\\%
%\vspace{\baselineskip}%
\par%
\noindent%
\textit{Klasifikace:}\dotfill\\%
\end{minipage}

\section{Pracovní úkol}

\begin{enumerate}
\item DU: V přípravě odvoď ťe vzorec pro výpočet momentu setrvačnosti válce a dutého
válce. Vyjděte z definice a odvodťe vztahy (1) a (2) \cite{1}.
\item  Změřte momenty setrvačnosti přiložených rotačních objektů experimentálně a porovnejte je s
hodnotami z teoretických vzorců. Použijte disk, disk + prstenec a pomocí nich stanovte moment
setrvačnosti samotného prstence.
\item  Změřte moment setrvačnosti disku, umístěného na dráze mimo osu rotace a pomocí výsledků
z předchozího úkolu ověřte platnost Steinerovy věty.
\item  Ověřte zákon zachování momentu hybnosti. Do protokolu přiložte graf závislosti úhové rychlosti
rotace na čase.
\item  Změřte rychlost precese gyroskopu jak přímo senzorem, tak i nepřímo z měření rychlosti rotace
disku. Obě hodnoty porovnejte.
\end{enumerate}



\section{Pomôcky}
”A”base rotational adapter PASCO CI-6690, přídavny disk a prstenec, rotační dráha
s dvěma závažími, Gyroskop PASCO ME-8960, přídavn´y disk gyroskopu ME-8961, dva rotační
senzory PASCO PS-2120, USB link PASCO 2100, PC, program pro datový sběr Data
Studio, nit, posuvné měrítko, stojan s kladkou, milimetrov´e měrítko, váhy.

\section{Teória}

Moment zotrvačnosť rotujúceho disku vypočítame ako
\eq{
I = \frac{1}{2} M R^2 \,, \lbl{R_10}
}
kde $M$ je hmotnosť disku a $R$ je jeho polomer.
 
Moment zotrvačnosť rotujúcej kruhového prstenca vypočítame ako
\eq{
I = \frac{1}{2} M \(R_1^2+R_2^2\) \,, \lbl{R_11}
}
kde $M$ je hmotnosť disku a $R_{1,2}$ sú jeho polomery.

Steinerová věta nám hovorí ako vypočítať moment zotrvačnosti telesa,
keď poznáme jeho hmotnosť a moment zotrvačnosti okolo osi prechádzajúcej ťažiskom $I_0$, ktorá je vzdialená od osi otáčania o $a$
\eq{
I = I_0 + M a^2 \,. \lbl{R_12}
}

Vzťah re aparatúru na výpočet momentu zotrvačnosti z uhlového zrýchlenia $\epsilon$ 
spôsobeného podaním závažia o hmotnosti $m$ pri tiažovom zrýchlení $g="9.81 m\cdot s^{-2}"$
a kladke o polomere $r$ znie nasledovne
\eq{
I= r m\(\frac{g}{\epsilon}-r\) \,. \lbl{R_1}
}

K výpočtu teoretickej presecie $\Omega$, z momentu zotrvačnosti rotujúceho disku $I$, hmotnosti protizávažia $m$ a vzdialenosti závažia od osi precesie $d$ a úhlovej rýchlosti disku $\omega$
\eq{
\Omega = \frac{m g d}{I \omega} \lbl{R_3}
}

Moment hybnosti $L$ vypočítame ako
\eq{
L= \omega I \,. \lbl{R_2}
}

\subsubsection{Spracovanie chýb merania}

Označme $\mean{t}$ aritmetický priemer nameraných hodnôt $t_i$, a $\Delta t$ hodnotu $\mean{t}-t$, pričom 
\eq{
\mean{t} = \frac{1}{n}\sum_{i=1}^n t_i \,, \lbl{SCH_1}
}  
a chybu aritmetického priemeru 
\eq{
  \sigma_0=\sqrt{\frac{\sum_{i=1}^n \(t_i - \mean{t}\)^2}{n\(n-1\)}}\,, \lbl{SCH_2}
}
pričom $n$ je počet meraní.

\section{Postup merania}
\begin{enumerate}
\item Pomocou pusuvného meradla a metru sme zmerali všetky polomery (priemery) disku prstenca.
\item Na digitálnych váhach sme určili hmotnosť disku aj prstenca.
\item Odvážili sme aj obe použíté závažia.
\item Na kladku na aparatúre sme namotali lanko, a na jeho voľný koniec sme umiestnili pripravené závažie. \label{OP_1}
\item Následne závažie spustili a merali uhlové zrýchlenie disku 
\item Postup opakujeme pre samostatný disk, disk s prstencom, vychýlený disk a samotnú aparatúru na vychýlenie.
\item Následne boli na aparatúru umiestené 2 rovnaké závažia 
\item Pre 2 zvolené symetrické polohy závaží voči stredu boli zmerané podľa \ref{OP_1} uhlové zrýchlenia
\item Závažia boli priviazané lankom tak aby sa s nimi počas pohybu dalo pohybovať medzi krajnými polohami.
\item Aparatúra sa roztočila a niekoľko krát sa zmenila poloha závaží a pri tom sme zaznamenávali závislosť uhlovej rýchlosti na čase.
\end{enumerate}
\begin{enumerate}
\item Pomocou posuvného mradla boli odmerané všetky potrebné rozmery na giroskope $d,r_{1,2}$
\item Na koniec girsokopu boli umiestnené závažie.
\item Disk bol roztočený a pritom sa merali hodnoty precesie a uhlová rýchlosť disku.
\end{enumerate}

\section{Výsledky merania}
V tabuľke Tab. \ref{T_1} sú namerané hodnoty jednotlivých uhlových zrýchlení. 
Z nich pomocou vzťahov \ref{R_1} a \ref{SCH_1} boli vypočítané hodnoty momentu zotrvačnosti $I$ pre jednotlivé experimenty, pričom $r="3.11 cm"$.
Disk má polomer $R= "11.45 cm" $ a obruč má polomery $R_1="6.3 cm"$ a $R_2="6.5 cm"$.
Moment zotrvačnosti samostatného disku s hmotnosťou $m= "1429 g"$
sme určili
\eq{
I_{D} = "\(0.020\pm0.001\)kg m^{2}"\,,
}
pričom $m = "50 g"$.

Moment zotrvačnosti disku  prstencom o hmotnosti $m_P= "1428 g"$ 
sme určili
\eq{
I_{D+P} = "\(0.032\pm0.002\)kg m^{2}"\,,
}
pričom $m = "50 g"$.

Moment zotrvačnosti disku vychýleného od osi otáčania o vzdialenosť $a= "15 cm" $ 
sme určili
\eq{
I_{PV+A} = "\(0.093\pm0.004\)kg m^{2}"\,,
}
pričom $m = "200 g"$.

Moment zotrvačnosti aparatúry na vychýlenie disku 
sme určili
\eq{
I_{A} = "\(0.023\pm0.001\)kg m^{2}"\,,
}
pričom $m = "200 g"$.



\begin{table}[h]
\begin{center}
\begin{tabular}{| c | c | c | c |}
\hline
\popi{\epsilon_{D}}{\frac{rad}{s^2}} & \popi{\epsilon_{D+P}}{\frac{rad}{s^2}} & \popi{\epsilon_{PV+A}}{\frac{rad}{s^2}} & \popi{\epsilon_{A}}{\frac{rad}{s^2}}\\
\hline
$0.723$ & $0.480$ & $0.657$ & $2.53$\\
$0.744$ & $0.495$ & $0.651$ & $2.53$\\
$0.768$ & $0.475$ & $0.634$ & $2.48$\\
$0.749$ & $0.488$ & $0.663$ & $2.55$\\
$0.810$ & $0.475$ & $0.655$ & $2.88$\\
$0.813$ & $0.448$ & $0.633$ & - \\
$0.781$ & $0.473$ & $0.659$ & - \\
$0.777$ & $0.482$ & $0.679$ & - \\
$0.721$ & $0.482$ & $0.664$ & - \\
$0.767$ & $0.479$ & $0.670$ & - \\
\hline
\end{tabular}
\caption{
Nameraná hodnoty uhlového zrýchlenia $\epsilon_*$ pre jednotlivé merania.
} \label{T_1}
\end{center}
\end{table}


V tabuľke Tab. \ref{T_2} sú zaznamenané namerané hodnoty uhlového zrýchlenia pre závažia v jednotlivých polohách, 
$\epsilon_{17}$ pre vzdialenosť od stredu $r="17 cm"$ a $\epsilon_{7}$ pre vzdialenosť od stredu $r="7 cm"$.
Z nich podľa vzťahu \ref{R_1} boli vypočítané jednotlivé hodnoty 
momentov zotrvačnosti $I_{17} = "\(0.013\pm0.0005\) kg\cdot m^2"$ a $I="\(0.0081\pm0.0001\) kg\cdot m^2"$.
Tie boli použité na výpočet jednotlivých hodnôt momentu hybnosti z nameraných dát v tabuľke Tab. \ref{T_3}.


\begin{table}[h]
\begin{center}
\begin{tabular}{| c | c |}
\hline
\popi{\epsilon_{17}}{\frac{rad}{s^2}} & \popi{\epsilon_{7}}{\frac{rad}{s^2}}\\
\hline
$1.21$ & $1.91$\\
$1.14$ & $1.84$\\
$1.22$ & $1.85$\\
$1.23$ & $1.86$\\
$1.27$ & $1.84$\\
\hline
\end{tabular}
\caption{
Nameraná hodnoty uhlového zrýchlenia $\epsilon_*$ pre jednotlivé polohy závaží.
} \label{T_2}
\end{center}
\end{table}

\begin{table}[h]
\begin{center}
\begin{tabular}{| c | c | c | c |}
\hline
\popi{\omega_{17}}{\frac{rad}{s^2}} & \popi{\omega_{7}}{\frac{rad}{s^2}} & \popi{L_{17}}{\frac{rad kg}{s^2}} & \popi{L_{7}}{\frac{rad kg}{s^2}}\\
\hline
$0.0796$ & $0.197$ & $ 0.10$ & $1.61$\\
$0.056$  & $0.15$  & $ 0.70$ & $1.22$\\
$0.108$  & $0.152$ & $ 1.35$ & $1.27$\\
$1.9$    & $0.303$ & $23.81$ & $2.47$\\
$1.54$   & $0.351$ & $19.30$ & $2.86$\\
$1.1102$ & $0.617$ & $13.91$ & $5.03$\\
$0.0995$ & $0.152$ & $ 1.25$ & $1.24$\\
$0.0571$ & $0.204$ & $ 0.73$ & $1.66$\\
$0.0565$ & $0.324$ & $ 0.71$ & $2.64$\\
$0.0788$ & $0.189$ & $ 0.99$ & $1.54$\\
$0.199$  & $0.163$ & $ 2.49$ & $1.33$\\
\hline
\end{tabular}
\caption{
Namerané hodnoty uhlových rýchlosti pre jednotlivé vzdialenosti $\omega$ 
a vypočítané hodnoty momentu hybnosti podľa vzťahu \ref{R_2} $L$.
} \label{T_3}
\end{center}
\end{table}

Z veľkosti kladiek na prevod $d_1="5.6 cm"$ a $d_2="3.11 cm"$ sme určili prevodný pomer medzi nameranými dátami uhlovej rýchlosti rotácie disku giroskopu.
Vzdialenosť  sme určili ako súčet vzdelaností po disk od disku a hrúbku disku, teda $d = "\(3\pm0.1+7\pm0.5+13.5\pm0.5\) cm" = "23.5\pm1.1 cm"$.

V tabuľke Tab. \ref{T_4} sú namerané a vypočítane hodnoty precesie giroskopu a ich porovnanie.

\begin{table}[h]
\begin{center}
\begin{tabular}{| c | c | c | c |}
\hline
\popi{\omega}{\frac{rad}{s^2}} & \popi{\Omega_m}{s^{-1}} & \popi{\Omega_v}{s^{-1}} & \popi{\Omega_m/\Omega_v}{-}\\
\hline
$-0.265$ & $-40.891$ & $-0.25$ & $1.07$\\
$ 0.310$ & $ 27.000$ & $ 0.38$ & $0.83$\\
$-0.364$ & $-53.724$ & $-0.19$ & $1.93$\\
$ 0.281$ & $ 46.729$ & $ 0.22$ & $1.30$\\
$ 0.384$ & $ 33.393$ & $ 0.30$ & $1.27$\\
\hline
\end{tabular}
\caption{
Namerané hodnoty uhlovej rýchlosti $\omega$ pred prepočtom, 
v závislosti na hodnote precesie $\Omega_m$ , 
vypočítané hodnoty precesie podľa vzťahu \ref{R_3} $\Omega_v$ a 
ich pomer.
} \label{T_4}
\end{center}
\end{table}




\begin{figure}
% GNUPLOT: LaTeX picture
\setlength{\unitlength}{0.240900pt}
\ifx\plotpoint\undefined\newsavebox{\plotpoint}\fi
\sbox{\plotpoint}{\rule[-0.200pt]{0.400pt}{0.400pt}}%
\begin{picture}(1500,900)(0,0)
\sbox{\plotpoint}{\rule[-0.200pt]{0.400pt}{0.400pt}}%
\put(130.0,82.0){\rule[-0.200pt]{4.818pt}{0.400pt}}
\put(110,82){\makebox(0,0)[r]{ 0}}
\put(1419.0,82.0){\rule[-0.200pt]{4.818pt}{0.400pt}}
\put(130.0,193.0){\rule[-0.200pt]{4.818pt}{0.400pt}}
\put(110,193){\makebox(0,0)[r]{ 20}}
\put(1419.0,193.0){\rule[-0.200pt]{4.818pt}{0.400pt}}
\put(130.0,304.0){\rule[-0.200pt]{4.818pt}{0.400pt}}
\put(110,304){\makebox(0,0)[r]{ 40}}
\put(1419.0,304.0){\rule[-0.200pt]{4.818pt}{0.400pt}}
\put(130.0,415.0){\rule[-0.200pt]{4.818pt}{0.400pt}}
\put(110,415){\makebox(0,0)[r]{ 60}}
\put(1419.0,415.0){\rule[-0.200pt]{4.818pt}{0.400pt}}
\put(130.0,526.0){\rule[-0.200pt]{4.818pt}{0.400pt}}
\put(110,526){\makebox(0,0)[r]{ 80}}
\put(1419.0,526.0){\rule[-0.200pt]{4.818pt}{0.400pt}}
\put(130.0,637.0){\rule[-0.200pt]{4.818pt}{0.400pt}}
\put(110,637){\makebox(0,0)[r]{ 100}}
\put(1419.0,637.0){\rule[-0.200pt]{4.818pt}{0.400pt}}
\put(130.0,748.0){\rule[-0.200pt]{4.818pt}{0.400pt}}
\put(110,748){\makebox(0,0)[r]{ 120}}
\put(1419.0,748.0){\rule[-0.200pt]{4.818pt}{0.400pt}}
\put(130.0,859.0){\rule[-0.200pt]{4.818pt}{0.400pt}}
\put(110,859){\makebox(0,0)[r]{ 140}}
\put(1419.0,859.0){\rule[-0.200pt]{4.818pt}{0.400pt}}
\put(130.0,82.0){\rule[-0.200pt]{0.400pt}{4.818pt}}
\put(130,41){\makebox(0,0){ 0}}
\put(130.0,839.0){\rule[-0.200pt]{0.400pt}{4.818pt}}
\put(294.0,82.0){\rule[-0.200pt]{0.400pt}{4.818pt}}
\put(294,41){\makebox(0,0){ 500}}
\put(294.0,839.0){\rule[-0.200pt]{0.400pt}{4.818pt}}
\put(457.0,82.0){\rule[-0.200pt]{0.400pt}{4.818pt}}
\put(457,41){\makebox(0,0){ 1000}}
\put(457.0,839.0){\rule[-0.200pt]{0.400pt}{4.818pt}}
\put(621.0,82.0){\rule[-0.200pt]{0.400pt}{4.818pt}}
\put(621,41){\makebox(0,0){ 1500}}
\put(621.0,839.0){\rule[-0.200pt]{0.400pt}{4.818pt}}
\put(785.0,82.0){\rule[-0.200pt]{0.400pt}{4.818pt}}
\put(785,41){\makebox(0,0){ 2000}}
\put(785.0,839.0){\rule[-0.200pt]{0.400pt}{4.818pt}}
\put(948.0,82.0){\rule[-0.200pt]{0.400pt}{4.818pt}}
\put(948,41){\makebox(0,0){ 2500}}
\put(948.0,839.0){\rule[-0.200pt]{0.400pt}{4.818pt}}
\put(1112.0,82.0){\rule[-0.200pt]{0.400pt}{4.818pt}}
\put(1112,41){\makebox(0,0){ 3000}}
\put(1112.0,839.0){\rule[-0.200pt]{0.400pt}{4.818pt}}
\put(1275.0,82.0){\rule[-0.200pt]{0.400pt}{4.818pt}}
\put(1275,41){\makebox(0,0){ 3500}}
\put(1275.0,839.0){\rule[-0.200pt]{0.400pt}{4.818pt}}
\put(1439.0,82.0){\rule[-0.200pt]{0.400pt}{4.818pt}}
\put(1439,41){\makebox(0,0){ 4000}}
\put(1439.0,839.0){\rule[-0.200pt]{0.400pt}{4.818pt}}
\put(130.0,82.0){\rule[-0.200pt]{0.400pt}{187.179pt}}
\put(130.0,82.0){\rule[-0.200pt]{315.338pt}{0.400pt}}
\put(1439.0,82.0){\rule[-0.200pt]{0.400pt}{187.179pt}}
\put(130.0,859.0){\rule[-0.200pt]{315.338pt}{0.400pt}}
\put(1279,819){\makebox(0,0)[r]{Namerané dáta zývylosť polohy od času}}
\put(1299.0,819.0){\rule[-0.200pt]{24.090pt}{0.400pt}}
\put(1299.0,809.0){\rule[-0.200pt]{0.400pt}{4.818pt}}
\put(1399.0,809.0){\rule[-0.200pt]{0.400pt}{4.818pt}}
\put(130.0,593.0){\rule[-0.200pt]{0.400pt}{2.650pt}}
\put(130.0,593.0){\rule[-0.200pt]{2.409pt}{0.400pt}}
\put(130.0,604.0){\rule[-0.200pt]{2.409pt}{0.400pt}}
\put(137.0,471.0){\rule[-0.200pt]{0.400pt}{2.650pt}}
\put(130.0,471.0){\rule[-0.200pt]{4.095pt}{0.400pt}}
\put(130.0,482.0){\rule[-0.200pt]{4.095pt}{0.400pt}}
\put(143.0,371.0){\rule[-0.200pt]{0.400pt}{2.650pt}}
\put(133.0,371.0){\rule[-0.200pt]{4.818pt}{0.400pt}}
\put(133.0,382.0){\rule[-0.200pt]{4.818pt}{0.400pt}}
\put(150.0,282.0){\rule[-0.200pt]{0.400pt}{2.650pt}}
\put(140.0,282.0){\rule[-0.200pt]{4.818pt}{0.400pt}}
\put(140.0,293.0){\rule[-0.200pt]{4.818pt}{0.400pt}}
\put(156.0,276.0){\rule[-0.200pt]{0.400pt}{2.650pt}}
\put(146.0,276.0){\rule[-0.200pt]{4.818pt}{0.400pt}}
\put(146.0,287.0){\rule[-0.200pt]{4.818pt}{0.400pt}}
\put(163.0,343.0){\rule[-0.200pt]{0.400pt}{2.650pt}}
\put(153.0,343.0){\rule[-0.200pt]{4.818pt}{0.400pt}}
\put(153.0,354.0){\rule[-0.200pt]{4.818pt}{0.400pt}}
\put(169.0,409.0){\rule[-0.200pt]{0.400pt}{2.891pt}}
\put(159.0,409.0){\rule[-0.200pt]{4.818pt}{0.400pt}}
\put(159.0,421.0){\rule[-0.200pt]{4.818pt}{0.400pt}}
\put(176.0,487.0){\rule[-0.200pt]{0.400pt}{2.650pt}}
\put(166.0,487.0){\rule[-0.200pt]{4.818pt}{0.400pt}}
\put(166.0,498.0){\rule[-0.200pt]{4.818pt}{0.400pt}}
\put(182.0,559.0){\rule[-0.200pt]{0.400pt}{2.650pt}}
\put(172.0,559.0){\rule[-0.200pt]{4.818pt}{0.400pt}}
\put(172.0,570.0){\rule[-0.200pt]{4.818pt}{0.400pt}}
\put(189.0,631.0){\rule[-0.200pt]{0.400pt}{2.891pt}}
\put(179.0,631.0){\rule[-0.200pt]{4.818pt}{0.400pt}}
\put(179.0,643.0){\rule[-0.200pt]{4.818pt}{0.400pt}}
\put(195.0,687.0){\rule[-0.200pt]{0.400pt}{2.650pt}}
\put(185.0,687.0){\rule[-0.200pt]{4.818pt}{0.400pt}}
\put(185.0,698.0){\rule[-0.200pt]{4.818pt}{0.400pt}}
\put(202.0,737.0){\rule[-0.200pt]{0.400pt}{2.650pt}}
\put(192.0,737.0){\rule[-0.200pt]{4.818pt}{0.400pt}}
\put(192.0,748.0){\rule[-0.200pt]{4.818pt}{0.400pt}}
\put(209.0,770.0){\rule[-0.200pt]{0.400pt}{2.650pt}}
\put(199.0,770.0){\rule[-0.200pt]{4.818pt}{0.400pt}}
\put(199.0,781.0){\rule[-0.200pt]{4.818pt}{0.400pt}}
\put(215.0,776.0){\rule[-0.200pt]{0.400pt}{2.650pt}}
\put(205.0,776.0){\rule[-0.200pt]{4.818pt}{0.400pt}}
\put(205.0,787.0){\rule[-0.200pt]{4.818pt}{0.400pt}}
\put(222.0,776.0){\rule[-0.200pt]{0.400pt}{2.650pt}}
\put(212.0,776.0){\rule[-0.200pt]{4.818pt}{0.400pt}}
\put(212.0,787.0){\rule[-0.200pt]{4.818pt}{0.400pt}}
\put(228.0,754.0){\rule[-0.200pt]{0.400pt}{2.650pt}}
\put(218.0,754.0){\rule[-0.200pt]{4.818pt}{0.400pt}}
\put(218.0,765.0){\rule[-0.200pt]{4.818pt}{0.400pt}}
\put(235.0,715.0){\rule[-0.200pt]{0.400pt}{2.650pt}}
\put(225.0,715.0){\rule[-0.200pt]{4.818pt}{0.400pt}}
\put(225.0,726.0){\rule[-0.200pt]{4.818pt}{0.400pt}}
\put(241.0,665.0){\rule[-0.200pt]{0.400pt}{2.650pt}}
\put(231.0,665.0){\rule[-0.200pt]{4.818pt}{0.400pt}}
\put(231.0,676.0){\rule[-0.200pt]{4.818pt}{0.400pt}}
\put(248.0,609.0){\rule[-0.200pt]{0.400pt}{2.650pt}}
\put(238.0,609.0){\rule[-0.200pt]{4.818pt}{0.400pt}}
\put(238.0,620.0){\rule[-0.200pt]{4.818pt}{0.400pt}}
\put(254.0,543.0){\rule[-0.200pt]{0.400pt}{2.650pt}}
\put(244.0,543.0){\rule[-0.200pt]{4.818pt}{0.400pt}}
\put(244.0,554.0){\rule[-0.200pt]{4.818pt}{0.400pt}}
\put(261.0,476.0){\rule[-0.200pt]{0.400pt}{2.650pt}}
\put(251.0,476.0){\rule[-0.200pt]{4.818pt}{0.400pt}}
\put(251.0,487.0){\rule[-0.200pt]{4.818pt}{0.400pt}}
\put(267.0,415.0){\rule[-0.200pt]{0.400pt}{2.650pt}}
\put(257.0,415.0){\rule[-0.200pt]{4.818pt}{0.400pt}}
\put(257.0,426.0){\rule[-0.200pt]{4.818pt}{0.400pt}}
\put(274.0,360.0){\rule[-0.200pt]{0.400pt}{2.650pt}}
\put(264.0,360.0){\rule[-0.200pt]{4.818pt}{0.400pt}}
\put(264.0,371.0){\rule[-0.200pt]{4.818pt}{0.400pt}}
\put(281.0,310.0){\rule[-0.200pt]{0.400pt}{2.650pt}}
\put(271.0,310.0){\rule[-0.200pt]{4.818pt}{0.400pt}}
\put(271.0,321.0){\rule[-0.200pt]{4.818pt}{0.400pt}}
\put(287.0,282.0){\rule[-0.200pt]{0.400pt}{2.650pt}}
\put(277.0,282.0){\rule[-0.200pt]{4.818pt}{0.400pt}}
\put(277.0,293.0){\rule[-0.200pt]{4.818pt}{0.400pt}}
\put(294.0,260.0){\rule[-0.200pt]{0.400pt}{2.650pt}}
\put(284.0,260.0){\rule[-0.200pt]{4.818pt}{0.400pt}}
\put(284.0,271.0){\rule[-0.200pt]{4.818pt}{0.400pt}}
\put(300.0,260.0){\rule[-0.200pt]{0.400pt}{2.650pt}}
\put(290.0,260.0){\rule[-0.200pt]{4.818pt}{0.400pt}}
\put(290.0,271.0){\rule[-0.200pt]{4.818pt}{0.400pt}}
\put(307.0,282.0){\rule[-0.200pt]{0.400pt}{2.650pt}}
\put(297.0,282.0){\rule[-0.200pt]{4.818pt}{0.400pt}}
\put(297.0,293.0){\rule[-0.200pt]{4.818pt}{0.400pt}}
\put(313.0,304.0){\rule[-0.200pt]{0.400pt}{2.650pt}}
\put(303.0,304.0){\rule[-0.200pt]{4.818pt}{0.400pt}}
\put(303.0,315.0){\rule[-0.200pt]{4.818pt}{0.400pt}}
\put(320.0,348.0){\rule[-0.200pt]{0.400pt}{2.891pt}}
\put(310.0,348.0){\rule[-0.200pt]{4.818pt}{0.400pt}}
\put(310.0,360.0){\rule[-0.200pt]{4.818pt}{0.400pt}}
\put(326.0,398.0){\rule[-0.200pt]{0.400pt}{2.650pt}}
\put(316.0,398.0){\rule[-0.200pt]{4.818pt}{0.400pt}}
\put(316.0,409.0){\rule[-0.200pt]{4.818pt}{0.400pt}}
\put(339.0,509.0){\rule[-0.200pt]{0.400pt}{2.650pt}}
\put(329.0,509.0){\rule[-0.200pt]{4.818pt}{0.400pt}}
\put(329.0,520.0){\rule[-0.200pt]{4.818pt}{0.400pt}}
\put(346.0,554.0){\rule[-0.200pt]{0.400pt}{2.650pt}}
\put(336.0,554.0){\rule[-0.200pt]{4.818pt}{0.400pt}}
\put(336.0,565.0){\rule[-0.200pt]{4.818pt}{0.400pt}}
\put(353.0,593.0){\rule[-0.200pt]{0.400pt}{2.650pt}}
\put(343.0,593.0){\rule[-0.200pt]{4.818pt}{0.400pt}}
\put(343.0,604.0){\rule[-0.200pt]{4.818pt}{0.400pt}}
\put(359.0,631.0){\rule[-0.200pt]{0.400pt}{2.891pt}}
\put(349.0,631.0){\rule[-0.200pt]{4.818pt}{0.400pt}}
\put(349.0,643.0){\rule[-0.200pt]{4.818pt}{0.400pt}}
\put(366.0,654.0){\rule[-0.200pt]{0.400pt}{2.650pt}}
\put(356.0,654.0){\rule[-0.200pt]{4.818pt}{0.400pt}}
\put(356.0,665.0){\rule[-0.200pt]{4.818pt}{0.400pt}}
\put(372.0,659.0){\rule[-0.200pt]{0.400pt}{2.650pt}}
\put(362.0,659.0){\rule[-0.200pt]{4.818pt}{0.400pt}}
\put(362.0,670.0){\rule[-0.200pt]{4.818pt}{0.400pt}}
\put(379.0,648.0){\rule[-0.200pt]{0.400pt}{2.650pt}}
\put(369.0,648.0){\rule[-0.200pt]{4.818pt}{0.400pt}}
\put(369.0,659.0){\rule[-0.200pt]{4.818pt}{0.400pt}}
\put(385.0,637.0){\rule[-0.200pt]{0.400pt}{2.650pt}}
\put(375.0,637.0){\rule[-0.200pt]{4.818pt}{0.400pt}}
\put(375.0,648.0){\rule[-0.200pt]{4.818pt}{0.400pt}}
\put(392.0,604.0){\rule[-0.200pt]{0.400pt}{2.650pt}}
\put(382.0,604.0){\rule[-0.200pt]{4.818pt}{0.400pt}}
\put(382.0,615.0){\rule[-0.200pt]{4.818pt}{0.400pt}}
\put(398.0,565.0){\rule[-0.200pt]{0.400pt}{2.650pt}}
\put(388.0,565.0){\rule[-0.200pt]{4.818pt}{0.400pt}}
\put(388.0,576.0){\rule[-0.200pt]{4.818pt}{0.400pt}}
\put(405.0,509.0){\rule[-0.200pt]{0.400pt}{2.650pt}}
\put(395.0,509.0){\rule[-0.200pt]{4.818pt}{0.400pt}}
\put(395.0,520.0){\rule[-0.200pt]{4.818pt}{0.400pt}}
\put(411.0,459.0){\rule[-0.200pt]{0.400pt}{2.891pt}}
\put(401.0,459.0){\rule[-0.200pt]{4.818pt}{0.400pt}}
\put(401.0,471.0){\rule[-0.200pt]{4.818pt}{0.400pt}}
\put(418.0,415.0){\rule[-0.200pt]{0.400pt}{2.650pt}}
\put(408.0,415.0){\rule[-0.200pt]{4.818pt}{0.400pt}}
\put(408.0,426.0){\rule[-0.200pt]{4.818pt}{0.400pt}}
\put(425.0,365.0){\rule[-0.200pt]{0.400pt}{2.650pt}}
\put(415.0,365.0){\rule[-0.200pt]{4.818pt}{0.400pt}}
\put(415.0,376.0){\rule[-0.200pt]{4.818pt}{0.400pt}}
\put(431.0,326.0){\rule[-0.200pt]{0.400pt}{2.650pt}}
\put(421.0,326.0){\rule[-0.200pt]{4.818pt}{0.400pt}}
\put(421.0,337.0){\rule[-0.200pt]{4.818pt}{0.400pt}}
\put(438.0,293.0){\rule[-0.200pt]{0.400pt}{2.650pt}}
\put(428.0,293.0){\rule[-0.200pt]{4.818pt}{0.400pt}}
\put(428.0,304.0){\rule[-0.200pt]{4.818pt}{0.400pt}}
\put(444.0,276.0){\rule[-0.200pt]{0.400pt}{2.650pt}}
\put(434.0,276.0){\rule[-0.200pt]{4.818pt}{0.400pt}}
\put(434.0,287.0){\rule[-0.200pt]{4.818pt}{0.400pt}}
\put(451.0,260.0){\rule[-0.200pt]{0.400pt}{2.650pt}}
\put(441.0,260.0){\rule[-0.200pt]{4.818pt}{0.400pt}}
\put(441.0,271.0){\rule[-0.200pt]{4.818pt}{0.400pt}}
\put(457.0,265.0){\rule[-0.200pt]{0.400pt}{2.650pt}}
\put(447.0,265.0){\rule[-0.200pt]{4.818pt}{0.400pt}}
\put(447.0,276.0){\rule[-0.200pt]{4.818pt}{0.400pt}}
\put(464.0,276.0){\rule[-0.200pt]{0.400pt}{2.650pt}}
\put(454.0,276.0){\rule[-0.200pt]{4.818pt}{0.400pt}}
\put(454.0,287.0){\rule[-0.200pt]{4.818pt}{0.400pt}}
\put(470.0,298.0){\rule[-0.200pt]{0.400pt}{2.891pt}}
\put(460.0,298.0){\rule[-0.200pt]{4.818pt}{0.400pt}}
\put(460.0,310.0){\rule[-0.200pt]{4.818pt}{0.400pt}}
\put(477.0,348.0){\rule[-0.200pt]{0.400pt}{2.891pt}}
\put(467.0,348.0){\rule[-0.200pt]{4.818pt}{0.400pt}}
\put(467.0,360.0){\rule[-0.200pt]{4.818pt}{0.400pt}}
\put(483.0,371.0){\rule[-0.200pt]{0.400pt}{2.650pt}}
\put(473.0,371.0){\rule[-0.200pt]{4.818pt}{0.400pt}}
\put(473.0,382.0){\rule[-0.200pt]{4.818pt}{0.400pt}}
\put(490.0,415.0){\rule[-0.200pt]{0.400pt}{2.650pt}}
\put(480.0,415.0){\rule[-0.200pt]{4.818pt}{0.400pt}}
\put(480.0,426.0){\rule[-0.200pt]{4.818pt}{0.400pt}}
\put(497.0,465.0){\rule[-0.200pt]{0.400pt}{2.650pt}}
\put(487.0,465.0){\rule[-0.200pt]{4.818pt}{0.400pt}}
\put(487.0,476.0){\rule[-0.200pt]{4.818pt}{0.400pt}}
\put(503.0,498.0){\rule[-0.200pt]{0.400pt}{2.650pt}}
\put(493.0,498.0){\rule[-0.200pt]{4.818pt}{0.400pt}}
\put(493.0,509.0){\rule[-0.200pt]{4.818pt}{0.400pt}}
\put(510.0,537.0){\rule[-0.200pt]{0.400pt}{2.650pt}}
\put(500.0,537.0){\rule[-0.200pt]{4.818pt}{0.400pt}}
\put(500.0,548.0){\rule[-0.200pt]{4.818pt}{0.400pt}}
\put(516.0,565.0){\rule[-0.200pt]{0.400pt}{2.650pt}}
\put(506.0,565.0){\rule[-0.200pt]{4.818pt}{0.400pt}}
\put(506.0,576.0){\rule[-0.200pt]{4.818pt}{0.400pt}}
\put(523.0,593.0){\rule[-0.200pt]{0.400pt}{2.650pt}}
\put(513.0,593.0){\rule[-0.200pt]{4.818pt}{0.400pt}}
\put(513.0,604.0){\rule[-0.200pt]{4.818pt}{0.400pt}}
\put(529.0,604.0){\rule[-0.200pt]{0.400pt}{2.650pt}}
\put(519.0,604.0){\rule[-0.200pt]{4.818pt}{0.400pt}}
\put(519.0,615.0){\rule[-0.200pt]{4.818pt}{0.400pt}}
\put(536.0,615.0){\rule[-0.200pt]{0.400pt}{2.650pt}}
\put(526.0,615.0){\rule[-0.200pt]{4.818pt}{0.400pt}}
\put(526.0,626.0){\rule[-0.200pt]{4.818pt}{0.400pt}}
\put(542.0,609.0){\rule[-0.200pt]{0.400pt}{2.650pt}}
\put(532.0,609.0){\rule[-0.200pt]{4.818pt}{0.400pt}}
\put(532.0,620.0){\rule[-0.200pt]{4.818pt}{0.400pt}}
\put(549.0,598.0){\rule[-0.200pt]{0.400pt}{2.650pt}}
\put(539.0,598.0){\rule[-0.200pt]{4.818pt}{0.400pt}}
\put(539.0,609.0){\rule[-0.200pt]{4.818pt}{0.400pt}}
\put(555.0,570.0){\rule[-0.200pt]{0.400pt}{2.891pt}}
\put(545.0,570.0){\rule[-0.200pt]{4.818pt}{0.400pt}}
\put(545.0,582.0){\rule[-0.200pt]{4.818pt}{0.400pt}}
\put(562.0,543.0){\rule[-0.200pt]{0.400pt}{2.650pt}}
\put(552.0,543.0){\rule[-0.200pt]{4.818pt}{0.400pt}}
\put(552.0,554.0){\rule[-0.200pt]{4.818pt}{0.400pt}}
\put(569.0,509.0){\rule[-0.200pt]{0.400pt}{2.650pt}}
\put(559.0,509.0){\rule[-0.200pt]{4.818pt}{0.400pt}}
\put(559.0,520.0){\rule[-0.200pt]{4.818pt}{0.400pt}}
\put(575.0,409.0){\rule[-0.200pt]{0.400pt}{2.891pt}}
\put(565.0,409.0){\rule[-0.200pt]{4.818pt}{0.400pt}}
\put(565.0,421.0){\rule[-0.200pt]{4.818pt}{0.400pt}}
\put(582.0,421.0){\rule[-0.200pt]{0.400pt}{2.650pt}}
\put(572.0,421.0){\rule[-0.200pt]{4.818pt}{0.400pt}}
\put(572.0,432.0){\rule[-0.200pt]{4.818pt}{0.400pt}}
\put(588.0,393.0){\rule[-0.200pt]{0.400pt}{2.650pt}}
\put(578.0,393.0){\rule[-0.200pt]{4.818pt}{0.400pt}}
\put(578.0,404.0){\rule[-0.200pt]{4.818pt}{0.400pt}}
\put(595.0,360.0){\rule[-0.200pt]{0.400pt}{2.650pt}}
\put(585.0,360.0){\rule[-0.200pt]{4.818pt}{0.400pt}}
\put(585.0,371.0){\rule[-0.200pt]{4.818pt}{0.400pt}}
\put(601.0,337.0){\rule[-0.200pt]{0.400pt}{2.650pt}}
\put(591.0,337.0){\rule[-0.200pt]{4.818pt}{0.400pt}}
\put(591.0,348.0){\rule[-0.200pt]{4.818pt}{0.400pt}}
\put(608.0,321.0){\rule[-0.200pt]{0.400pt}{2.650pt}}
\put(598.0,321.0){\rule[-0.200pt]{4.818pt}{0.400pt}}
\put(598.0,332.0){\rule[-0.200pt]{4.818pt}{0.400pt}}
\put(614.0,310.0){\rule[-0.200pt]{0.400pt}{2.650pt}}
\put(604.0,310.0){\rule[-0.200pt]{4.818pt}{0.400pt}}
\put(604.0,321.0){\rule[-0.200pt]{4.818pt}{0.400pt}}
\put(621.0,315.0){\rule[-0.200pt]{0.400pt}{2.650pt}}
\put(611.0,315.0){\rule[-0.200pt]{4.818pt}{0.400pt}}
\put(611.0,326.0){\rule[-0.200pt]{4.818pt}{0.400pt}}
\put(627.0,326.0){\rule[-0.200pt]{0.400pt}{2.650pt}}
\put(617.0,326.0){\rule[-0.200pt]{4.818pt}{0.400pt}}
\put(617.0,337.0){\rule[-0.200pt]{4.818pt}{0.400pt}}
\put(634.0,343.0){\rule[-0.200pt]{0.400pt}{2.650pt}}
\put(624.0,343.0){\rule[-0.200pt]{4.818pt}{0.400pt}}
\put(624.0,354.0){\rule[-0.200pt]{4.818pt}{0.400pt}}
\put(641.0,365.0){\rule[-0.200pt]{0.400pt}{2.650pt}}
\put(631.0,365.0){\rule[-0.200pt]{4.818pt}{0.400pt}}
\put(631.0,376.0){\rule[-0.200pt]{4.818pt}{0.400pt}}
\put(647.0,398.0){\rule[-0.200pt]{0.400pt}{2.650pt}}
\put(637.0,398.0){\rule[-0.200pt]{4.818pt}{0.400pt}}
\put(637.0,409.0){\rule[-0.200pt]{4.818pt}{0.400pt}}
\put(654.0,426.0){\rule[-0.200pt]{0.400pt}{2.650pt}}
\put(644.0,426.0){\rule[-0.200pt]{4.818pt}{0.400pt}}
\put(644.0,437.0){\rule[-0.200pt]{4.818pt}{0.400pt}}
\put(660.0,459.0){\rule[-0.200pt]{0.400pt}{2.891pt}}
\put(650.0,459.0){\rule[-0.200pt]{4.818pt}{0.400pt}}
\put(650.0,471.0){\rule[-0.200pt]{4.818pt}{0.400pt}}
\put(667.0,498.0){\rule[-0.200pt]{0.400pt}{2.650pt}}
\put(657.0,498.0){\rule[-0.200pt]{4.818pt}{0.400pt}}
\put(657.0,509.0){\rule[-0.200pt]{4.818pt}{0.400pt}}
\put(673.0,532.0){\rule[-0.200pt]{0.400pt}{2.650pt}}
\put(663.0,532.0){\rule[-0.200pt]{4.818pt}{0.400pt}}
\put(663.0,543.0){\rule[-0.200pt]{4.818pt}{0.400pt}}
\put(680.0,548.0){\rule[-0.200pt]{0.400pt}{2.650pt}}
\put(670.0,548.0){\rule[-0.200pt]{4.818pt}{0.400pt}}
\put(670.0,559.0){\rule[-0.200pt]{4.818pt}{0.400pt}}
\put(686.0,582.0){\rule[-0.200pt]{0.400pt}{2.650pt}}
\put(676.0,582.0){\rule[-0.200pt]{4.818pt}{0.400pt}}
\put(676.0,593.0){\rule[-0.200pt]{4.818pt}{0.400pt}}
\put(693.0,587.0){\rule[-0.200pt]{0.400pt}{2.650pt}}
\put(683.0,587.0){\rule[-0.200pt]{4.818pt}{0.400pt}}
\put(683.0,598.0){\rule[-0.200pt]{4.818pt}{0.400pt}}
\put(699.0,582.0){\rule[-0.200pt]{0.400pt}{2.650pt}}
\put(689.0,582.0){\rule[-0.200pt]{4.818pt}{0.400pt}}
\put(689.0,593.0){\rule[-0.200pt]{4.818pt}{0.400pt}}
\put(706.0,576.0){\rule[-0.200pt]{0.400pt}{2.650pt}}
\put(696.0,576.0){\rule[-0.200pt]{4.818pt}{0.400pt}}
\put(696.0,587.0){\rule[-0.200pt]{4.818pt}{0.400pt}}
\put(713.0,565.0){\rule[-0.200pt]{0.400pt}{2.650pt}}
\put(703.0,565.0){\rule[-0.200pt]{4.818pt}{0.400pt}}
\put(703.0,576.0){\rule[-0.200pt]{4.818pt}{0.400pt}}
\put(719.0,548.0){\rule[-0.200pt]{0.400pt}{2.650pt}}
\put(709.0,548.0){\rule[-0.200pt]{4.818pt}{0.400pt}}
\put(709.0,559.0){\rule[-0.200pt]{4.818pt}{0.400pt}}
\put(726.0,520.0){\rule[-0.200pt]{0.400pt}{2.891pt}}
\put(716.0,520.0){\rule[-0.200pt]{4.818pt}{0.400pt}}
\put(716.0,532.0){\rule[-0.200pt]{4.818pt}{0.400pt}}
\put(732.0,493.0){\rule[-0.200pt]{0.400pt}{2.650pt}}
\put(722.0,493.0){\rule[-0.200pt]{4.818pt}{0.400pt}}
\put(722.0,504.0){\rule[-0.200pt]{4.818pt}{0.400pt}}
\put(739.0,465.0){\rule[-0.200pt]{0.400pt}{2.650pt}}
\put(729.0,465.0){\rule[-0.200pt]{4.818pt}{0.400pt}}
\put(729.0,476.0){\rule[-0.200pt]{4.818pt}{0.400pt}}
\put(745.0,432.0){\rule[-0.200pt]{0.400pt}{2.650pt}}
\put(735.0,432.0){\rule[-0.200pt]{4.818pt}{0.400pt}}
\put(735.0,443.0){\rule[-0.200pt]{4.818pt}{0.400pt}}
\put(752.0,409.0){\rule[-0.200pt]{0.400pt}{2.891pt}}
\put(742.0,409.0){\rule[-0.200pt]{4.818pt}{0.400pt}}
\put(742.0,421.0){\rule[-0.200pt]{4.818pt}{0.400pt}}
\put(758.0,387.0){\rule[-0.200pt]{0.400pt}{2.650pt}}
\put(748.0,387.0){\rule[-0.200pt]{4.818pt}{0.400pt}}
\put(748.0,398.0){\rule[-0.200pt]{4.818pt}{0.400pt}}
\put(765.0,365.0){\rule[-0.200pt]{0.400pt}{2.650pt}}
\put(755.0,365.0){\rule[-0.200pt]{4.818pt}{0.400pt}}
\put(755.0,376.0){\rule[-0.200pt]{4.818pt}{0.400pt}}
\put(771.0,354.0){\rule[-0.200pt]{0.400pt}{2.650pt}}
\put(761.0,354.0){\rule[-0.200pt]{4.818pt}{0.400pt}}
\put(761.0,365.0){\rule[-0.200pt]{4.818pt}{0.400pt}}
\put(778.0,348.0){\rule[-0.200pt]{0.400pt}{2.891pt}}
\put(768.0,348.0){\rule[-0.200pt]{4.818pt}{0.400pt}}
\put(768.0,360.0){\rule[-0.200pt]{4.818pt}{0.400pt}}
\put(785.0,348.0){\rule[-0.200pt]{0.400pt}{2.891pt}}
\put(775.0,348.0){\rule[-0.200pt]{4.818pt}{0.400pt}}
\put(775.0,360.0){\rule[-0.200pt]{4.818pt}{0.400pt}}
\put(791.0,360.0){\rule[-0.200pt]{0.400pt}{2.650pt}}
\put(781.0,360.0){\rule[-0.200pt]{4.818pt}{0.400pt}}
\put(781.0,371.0){\rule[-0.200pt]{4.818pt}{0.400pt}}
\put(798.0,371.0){\rule[-0.200pt]{0.400pt}{2.650pt}}
\put(788.0,371.0){\rule[-0.200pt]{4.818pt}{0.400pt}}
\put(788.0,382.0){\rule[-0.200pt]{4.818pt}{0.400pt}}
\put(804.0,393.0){\rule[-0.200pt]{0.400pt}{2.650pt}}
\put(794.0,393.0){\rule[-0.200pt]{4.818pt}{0.400pt}}
\put(794.0,404.0){\rule[-0.200pt]{4.818pt}{0.400pt}}
\put(811.0,415.0){\rule[-0.200pt]{0.400pt}{2.650pt}}
\put(801.0,415.0){\rule[-0.200pt]{4.818pt}{0.400pt}}
\put(801.0,426.0){\rule[-0.200pt]{4.818pt}{0.400pt}}
\put(817.0,443.0){\rule[-0.200pt]{0.400pt}{2.650pt}}
\put(807.0,443.0){\rule[-0.200pt]{4.818pt}{0.400pt}}
\put(807.0,454.0){\rule[-0.200pt]{4.818pt}{0.400pt}}
\put(824.0,476.0){\rule[-0.200pt]{0.400pt}{2.650pt}}
\put(814.0,476.0){\rule[-0.200pt]{4.818pt}{0.400pt}}
\put(814.0,487.0){\rule[-0.200pt]{4.818pt}{0.400pt}}
\put(830.0,493.0){\rule[-0.200pt]{0.400pt}{2.650pt}}
\put(820.0,493.0){\rule[-0.200pt]{4.818pt}{0.400pt}}
\put(820.0,504.0){\rule[-0.200pt]{4.818pt}{0.400pt}}
\put(837.0,515.0){\rule[-0.200pt]{0.400pt}{2.650pt}}
\put(827.0,515.0){\rule[-0.200pt]{4.818pt}{0.400pt}}
\put(827.0,526.0){\rule[-0.200pt]{4.818pt}{0.400pt}}
\put(843.0,537.0){\rule[-0.200pt]{0.400pt}{2.650pt}}
\put(833.0,537.0){\rule[-0.200pt]{4.818pt}{0.400pt}}
\put(833.0,548.0){\rule[-0.200pt]{4.818pt}{0.400pt}}
\put(850.0,548.0){\rule[-0.200pt]{0.400pt}{2.650pt}}
\put(840.0,548.0){\rule[-0.200pt]{4.818pt}{0.400pt}}
\put(840.0,559.0){\rule[-0.200pt]{4.818pt}{0.400pt}}
\put(856.0,559.0){\rule[-0.200pt]{0.400pt}{2.650pt}}
\put(846.0,559.0){\rule[-0.200pt]{4.818pt}{0.400pt}}
\put(846.0,570.0){\rule[-0.200pt]{4.818pt}{0.400pt}}
\put(863.0,559.0){\rule[-0.200pt]{0.400pt}{2.650pt}}
\put(853.0,559.0){\rule[-0.200pt]{4.818pt}{0.400pt}}
\put(853.0,570.0){\rule[-0.200pt]{4.818pt}{0.400pt}}
\put(870.0,554.0){\rule[-0.200pt]{0.400pt}{2.650pt}}
\put(860.0,554.0){\rule[-0.200pt]{4.818pt}{0.400pt}}
\put(860.0,565.0){\rule[-0.200pt]{4.818pt}{0.400pt}}
\put(876.0,543.0){\rule[-0.200pt]{0.400pt}{2.650pt}}
\put(866.0,543.0){\rule[-0.200pt]{4.818pt}{0.400pt}}
\put(866.0,554.0){\rule[-0.200pt]{4.818pt}{0.400pt}}
\put(883.0,532.0){\rule[-0.200pt]{0.400pt}{2.650pt}}
\put(873.0,532.0){\rule[-0.200pt]{4.818pt}{0.400pt}}
\put(873.0,543.0){\rule[-0.200pt]{4.818pt}{0.400pt}}
\put(902.0,454.0){\rule[-0.200pt]{0.400pt}{2.650pt}}
\put(892.0,454.0){\rule[-0.200pt]{4.818pt}{0.400pt}}
\put(892.0,465.0){\rule[-0.200pt]{4.818pt}{0.400pt}}
\put(909.0,415.0){\rule[-0.200pt]{0.400pt}{2.650pt}}
\put(899.0,415.0){\rule[-0.200pt]{4.818pt}{0.400pt}}
\put(899.0,426.0){\rule[-0.200pt]{4.818pt}{0.400pt}}
\put(915.0,393.0){\rule[-0.200pt]{0.400pt}{2.650pt}}
\put(905.0,393.0){\rule[-0.200pt]{4.818pt}{0.400pt}}
\put(905.0,404.0){\rule[-0.200pt]{4.818pt}{0.400pt}}
\put(922.0,376.0){\rule[-0.200pt]{0.400pt}{2.650pt}}
\put(912.0,376.0){\rule[-0.200pt]{4.818pt}{0.400pt}}
\put(912.0,387.0){\rule[-0.200pt]{4.818pt}{0.400pt}}
\put(928.0,387.0){\rule[-0.200pt]{0.400pt}{2.650pt}}
\put(918.0,387.0){\rule[-0.200pt]{4.818pt}{0.400pt}}
\put(918.0,398.0){\rule[-0.200pt]{4.818pt}{0.400pt}}
\put(935.0,398.0){\rule[-0.200pt]{0.400pt}{2.650pt}}
\put(925.0,398.0){\rule[-0.200pt]{4.818pt}{0.400pt}}
\put(925.0,409.0){\rule[-0.200pt]{4.818pt}{0.400pt}}
\put(942.0,426.0){\rule[-0.200pt]{0.400pt}{2.650pt}}
\put(932.0,426.0){\rule[-0.200pt]{4.818pt}{0.400pt}}
\put(932.0,437.0){\rule[-0.200pt]{4.818pt}{0.400pt}}
\put(948.0,465.0){\rule[-0.200pt]{0.400pt}{2.650pt}}
\put(938.0,465.0){\rule[-0.200pt]{4.818pt}{0.400pt}}
\put(938.0,476.0){\rule[-0.200pt]{4.818pt}{0.400pt}}
\put(955.0,498.0){\rule[-0.200pt]{0.400pt}{2.650pt}}
\put(945.0,498.0){\rule[-0.200pt]{4.818pt}{0.400pt}}
\put(945.0,509.0){\rule[-0.200pt]{4.818pt}{0.400pt}}
\put(961.0,532.0){\rule[-0.200pt]{0.400pt}{2.650pt}}
\put(951.0,532.0){\rule[-0.200pt]{4.818pt}{0.400pt}}
\put(951.0,543.0){\rule[-0.200pt]{4.818pt}{0.400pt}}
\put(968.0,559.0){\rule[-0.200pt]{0.400pt}{2.650pt}}
\put(958.0,559.0){\rule[-0.200pt]{4.818pt}{0.400pt}}
\put(958.0,570.0){\rule[-0.200pt]{4.818pt}{0.400pt}}
\put(974.0,587.0){\rule[-0.200pt]{0.400pt}{2.650pt}}
\put(964.0,587.0){\rule[-0.200pt]{4.818pt}{0.400pt}}
\put(964.0,598.0){\rule[-0.200pt]{4.818pt}{0.400pt}}
\put(981.0,620.0){\rule[-0.200pt]{0.400pt}{2.650pt}}
\put(971.0,620.0){\rule[-0.200pt]{4.818pt}{0.400pt}}
\put(971.0,631.0){\rule[-0.200pt]{4.818pt}{0.400pt}}
\put(987.0,626.0){\rule[-0.200pt]{0.400pt}{2.650pt}}
\put(977.0,626.0){\rule[-0.200pt]{4.818pt}{0.400pt}}
\put(977.0,637.0){\rule[-0.200pt]{4.818pt}{0.400pt}}
\put(994.0,643.0){\rule[-0.200pt]{0.400pt}{2.650pt}}
\put(984.0,643.0){\rule[-0.200pt]{4.818pt}{0.400pt}}
\put(984.0,654.0){\rule[-0.200pt]{4.818pt}{0.400pt}}
\put(1000.0,648.0){\rule[-0.200pt]{0.400pt}{2.650pt}}
\put(990.0,648.0){\rule[-0.200pt]{4.818pt}{0.400pt}}
\put(990.0,659.0){\rule[-0.200pt]{4.818pt}{0.400pt}}
\put(1007.0,643.0){\rule[-0.200pt]{0.400pt}{2.650pt}}
\put(997.0,643.0){\rule[-0.200pt]{4.818pt}{0.400pt}}
\put(997.0,654.0){\rule[-0.200pt]{4.818pt}{0.400pt}}
\put(1014.0,631.0){\rule[-0.200pt]{0.400pt}{2.891pt}}
\put(1004.0,631.0){\rule[-0.200pt]{4.818pt}{0.400pt}}
\put(1004.0,643.0){\rule[-0.200pt]{4.818pt}{0.400pt}}
\put(1020.0,604.0){\rule[-0.200pt]{0.400pt}{2.650pt}}
\put(1010.0,604.0){\rule[-0.200pt]{4.818pt}{0.400pt}}
\put(1010.0,615.0){\rule[-0.200pt]{4.818pt}{0.400pt}}
\put(1027.0,576.0){\rule[-0.200pt]{0.400pt}{2.650pt}}
\put(1017.0,576.0){\rule[-0.200pt]{4.818pt}{0.400pt}}
\put(1017.0,587.0){\rule[-0.200pt]{4.818pt}{0.400pt}}
\put(1033.0,543.0){\rule[-0.200pt]{0.400pt}{2.650pt}}
\put(1023.0,543.0){\rule[-0.200pt]{4.818pt}{0.400pt}}
\put(1023.0,554.0){\rule[-0.200pt]{4.818pt}{0.400pt}}
\put(1040.0,509.0){\rule[-0.200pt]{0.400pt}{2.650pt}}
\put(1030.0,509.0){\rule[-0.200pt]{4.818pt}{0.400pt}}
\put(1030.0,520.0){\rule[-0.200pt]{4.818pt}{0.400pt}}
\put(1046.0,482.0){\rule[-0.200pt]{0.400pt}{2.650pt}}
\put(1036.0,482.0){\rule[-0.200pt]{4.818pt}{0.400pt}}
\put(1036.0,493.0){\rule[-0.200pt]{4.818pt}{0.400pt}}
\put(1053.0,459.0){\rule[-0.200pt]{0.400pt}{2.891pt}}
\put(1043.0,459.0){\rule[-0.200pt]{4.818pt}{0.400pt}}
\put(1043.0,471.0){\rule[-0.200pt]{4.818pt}{0.400pt}}
\put(1059.0,443.0){\rule[-0.200pt]{0.400pt}{2.650pt}}
\put(1049.0,443.0){\rule[-0.200pt]{4.818pt}{0.400pt}}
\put(1049.0,454.0){\rule[-0.200pt]{4.818pt}{0.400pt}}
\put(1066.0,432.0){\rule[-0.200pt]{0.400pt}{2.650pt}}
\put(1056.0,432.0){\rule[-0.200pt]{4.818pt}{0.400pt}}
\put(1056.0,443.0){\rule[-0.200pt]{4.818pt}{0.400pt}}
\put(1072.0,426.0){\rule[-0.200pt]{0.400pt}{2.650pt}}
\put(1062.0,426.0){\rule[-0.200pt]{4.818pt}{0.400pt}}
\put(1062.0,437.0){\rule[-0.200pt]{4.818pt}{0.400pt}}
\put(1079.0,426.0){\rule[-0.200pt]{0.400pt}{2.650pt}}
\put(1069.0,426.0){\rule[-0.200pt]{4.818pt}{0.400pt}}
\put(1069.0,437.0){\rule[-0.200pt]{4.818pt}{0.400pt}}
\put(1086.0,432.0){\rule[-0.200pt]{0.400pt}{2.650pt}}
\put(1076.0,432.0){\rule[-0.200pt]{4.818pt}{0.400pt}}
\put(1076.0,443.0){\rule[-0.200pt]{4.818pt}{0.400pt}}
\put(1092.0,437.0){\rule[-0.200pt]{0.400pt}{2.650pt}}
\put(1082.0,437.0){\rule[-0.200pt]{4.818pt}{0.400pt}}
\put(1082.0,448.0){\rule[-0.200pt]{4.818pt}{0.400pt}}
\put(1099.0,454.0){\rule[-0.200pt]{0.400pt}{2.650pt}}
\put(1089.0,454.0){\rule[-0.200pt]{4.818pt}{0.400pt}}
\put(1089.0,465.0){\rule[-0.200pt]{4.818pt}{0.400pt}}
\put(1105.0,476.0){\rule[-0.200pt]{0.400pt}{2.650pt}}
\put(1095.0,476.0){\rule[-0.200pt]{4.818pt}{0.400pt}}
\put(1095.0,487.0){\rule[-0.200pt]{4.818pt}{0.400pt}}
\put(1112.0,498.0){\rule[-0.200pt]{0.400pt}{2.650pt}}
\put(1102.0,498.0){\rule[-0.200pt]{4.818pt}{0.400pt}}
\put(1102.0,509.0){\rule[-0.200pt]{4.818pt}{0.400pt}}
\put(1118.0,520.0){\rule[-0.200pt]{0.400pt}{2.891pt}}
\put(1108.0,520.0){\rule[-0.200pt]{4.818pt}{0.400pt}}
\put(1108.0,532.0){\rule[-0.200pt]{4.818pt}{0.400pt}}
\put(1125.0,554.0){\rule[-0.200pt]{0.400pt}{2.650pt}}
\put(1115.0,554.0){\rule[-0.200pt]{4.818pt}{0.400pt}}
\put(1115.0,565.0){\rule[-0.200pt]{4.818pt}{0.400pt}}
\put(1131.0,576.0){\rule[-0.200pt]{0.400pt}{2.650pt}}
\put(1121.0,576.0){\rule[-0.200pt]{4.818pt}{0.400pt}}
\put(1121.0,587.0){\rule[-0.200pt]{4.818pt}{0.400pt}}
\put(1138.0,598.0){\rule[-0.200pt]{0.400pt}{2.650pt}}
\put(1128.0,598.0){\rule[-0.200pt]{4.818pt}{0.400pt}}
\put(1128.0,609.0){\rule[-0.200pt]{4.818pt}{0.400pt}}
\put(1144.0,615.0){\rule[-0.200pt]{0.400pt}{2.650pt}}
\put(1134.0,615.0){\rule[-0.200pt]{4.818pt}{0.400pt}}
\put(1134.0,626.0){\rule[-0.200pt]{4.818pt}{0.400pt}}
\put(1151.0,631.0){\rule[-0.200pt]{0.400pt}{2.891pt}}
\put(1141.0,631.0){\rule[-0.200pt]{4.818pt}{0.400pt}}
\put(1141.0,643.0){\rule[-0.200pt]{4.818pt}{0.400pt}}
\put(1158.0,626.0){\rule[-0.200pt]{0.400pt}{2.650pt}}
\put(1148.0,626.0){\rule[-0.200pt]{4.818pt}{0.400pt}}
\put(1148.0,637.0){\rule[-0.200pt]{4.818pt}{0.400pt}}
\put(1164.0,620.0){\rule[-0.200pt]{0.400pt}{2.650pt}}
\put(1154.0,620.0){\rule[-0.200pt]{4.818pt}{0.400pt}}
\put(1154.0,631.0){\rule[-0.200pt]{4.818pt}{0.400pt}}
\put(1171.0,609.0){\rule[-0.200pt]{0.400pt}{2.650pt}}
\put(1161.0,609.0){\rule[-0.200pt]{4.818pt}{0.400pt}}
\put(1161.0,620.0){\rule[-0.200pt]{4.818pt}{0.400pt}}
\put(1177.0,598.0){\rule[-0.200pt]{0.400pt}{2.650pt}}
\put(1167.0,598.0){\rule[-0.200pt]{4.818pt}{0.400pt}}
\put(1167.0,609.0){\rule[-0.200pt]{4.818pt}{0.400pt}}
\put(1184.0,582.0){\rule[-0.200pt]{0.400pt}{2.650pt}}
\put(1174.0,582.0){\rule[-0.200pt]{4.818pt}{0.400pt}}
\put(1174.0,593.0){\rule[-0.200pt]{4.818pt}{0.400pt}}
\put(1190.0,565.0){\rule[-0.200pt]{0.400pt}{2.650pt}}
\put(1180.0,565.0){\rule[-0.200pt]{4.818pt}{0.400pt}}
\put(1180.0,576.0){\rule[-0.200pt]{4.818pt}{0.400pt}}
\put(1197.0,543.0){\rule[-0.200pt]{0.400pt}{2.650pt}}
\put(1187.0,543.0){\rule[-0.200pt]{4.818pt}{0.400pt}}
\put(1187.0,554.0){\rule[-0.200pt]{4.818pt}{0.400pt}}
\put(1203.0,520.0){\rule[-0.200pt]{0.400pt}{2.891pt}}
\put(1193.0,520.0){\rule[-0.200pt]{4.818pt}{0.400pt}}
\put(1193.0,532.0){\rule[-0.200pt]{4.818pt}{0.400pt}}
\put(1210.0,498.0){\rule[-0.200pt]{0.400pt}{2.650pt}}
\put(1200.0,498.0){\rule[-0.200pt]{4.818pt}{0.400pt}}
\put(1200.0,509.0){\rule[-0.200pt]{4.818pt}{0.400pt}}
\put(1216.0,482.0){\rule[-0.200pt]{0.400pt}{2.650pt}}
\put(1206.0,482.0){\rule[-0.200pt]{4.818pt}{0.400pt}}
\put(1206.0,493.0){\rule[-0.200pt]{4.818pt}{0.400pt}}
\put(1223.0,459.0){\rule[-0.200pt]{0.400pt}{2.891pt}}
\put(1213.0,459.0){\rule[-0.200pt]{4.818pt}{0.400pt}}
\put(1213.0,471.0){\rule[-0.200pt]{4.818pt}{0.400pt}}
\put(1230.0,448.0){\rule[-0.200pt]{0.400pt}{2.650pt}}
\put(1220.0,448.0){\rule[-0.200pt]{4.818pt}{0.400pt}}
\put(1220.0,459.0){\rule[-0.200pt]{4.818pt}{0.400pt}}
\put(1236.0,443.0){\rule[-0.200pt]{0.400pt}{2.650pt}}
\put(1226.0,443.0){\rule[-0.200pt]{4.818pt}{0.400pt}}
\put(1226.0,454.0){\rule[-0.200pt]{4.818pt}{0.400pt}}
\put(1243.0,443.0){\rule[-0.200pt]{0.400pt}{2.650pt}}
\put(1233.0,443.0){\rule[-0.200pt]{4.818pt}{0.400pt}}
\put(1233.0,454.0){\rule[-0.200pt]{4.818pt}{0.400pt}}
\put(1249.0,448.0){\rule[-0.200pt]{0.400pt}{2.650pt}}
\put(1239.0,448.0){\rule[-0.200pt]{4.818pt}{0.400pt}}
\put(1239.0,459.0){\rule[-0.200pt]{4.818pt}{0.400pt}}
\put(1256.0,465.0){\rule[-0.200pt]{0.400pt}{2.650pt}}
\put(1246.0,465.0){\rule[-0.200pt]{4.818pt}{0.400pt}}
\put(1246.0,476.0){\rule[-0.200pt]{4.818pt}{0.400pt}}
\put(1262.0,476.0){\rule[-0.200pt]{0.400pt}{2.650pt}}
\put(1252.0,476.0){\rule[-0.200pt]{4.818pt}{0.400pt}}
\put(1252.0,487.0){\rule[-0.200pt]{4.818pt}{0.400pt}}
\put(1269.0,493.0){\rule[-0.200pt]{0.400pt}{2.650pt}}
\put(1259.0,493.0){\rule[-0.200pt]{4.818pt}{0.400pt}}
\put(1259.0,504.0){\rule[-0.200pt]{4.818pt}{0.400pt}}
\put(1275.0,515.0){\rule[-0.200pt]{0.400pt}{2.650pt}}
\put(1265.0,515.0){\rule[-0.200pt]{4.818pt}{0.400pt}}
\put(1265.0,526.0){\rule[-0.200pt]{4.818pt}{0.400pt}}
\put(1282.0,532.0){\rule[-0.200pt]{0.400pt}{2.650pt}}
\put(1272.0,532.0){\rule[-0.200pt]{4.818pt}{0.400pt}}
\put(1272.0,543.0){\rule[-0.200pt]{4.818pt}{0.400pt}}
\put(1288.0,554.0){\rule[-0.200pt]{0.400pt}{2.650pt}}
\put(1278.0,554.0){\rule[-0.200pt]{4.818pt}{0.400pt}}
\put(1278.0,565.0){\rule[-0.200pt]{4.818pt}{0.400pt}}
\put(1295.0,570.0){\rule[-0.200pt]{0.400pt}{2.891pt}}
\put(1285.0,570.0){\rule[-0.200pt]{4.818pt}{0.400pt}}
\put(1285.0,582.0){\rule[-0.200pt]{4.818pt}{0.400pt}}
\put(1302.0,587.0){\rule[-0.200pt]{0.400pt}{2.650pt}}
\put(1292.0,587.0){\rule[-0.200pt]{4.818pt}{0.400pt}}
\put(1292.0,598.0){\rule[-0.200pt]{4.818pt}{0.400pt}}
\put(1308.0,598.0){\rule[-0.200pt]{0.400pt}{2.650pt}}
\put(1298.0,598.0){\rule[-0.200pt]{4.818pt}{0.400pt}}
\put(1298.0,609.0){\rule[-0.200pt]{4.818pt}{0.400pt}}
\put(1315.0,604.0){\rule[-0.200pt]{0.400pt}{2.650pt}}
\put(1305.0,604.0){\rule[-0.200pt]{4.818pt}{0.400pt}}
\put(1305.0,615.0){\rule[-0.200pt]{4.818pt}{0.400pt}}
\put(1321.0,609.0){\rule[-0.200pt]{0.400pt}{2.650pt}}
\put(1311.0,609.0){\rule[-0.200pt]{4.818pt}{0.400pt}}
\put(1311.0,620.0){\rule[-0.200pt]{4.818pt}{0.400pt}}
\put(1328.0,604.0){\rule[-0.200pt]{0.400pt}{2.650pt}}
\put(1318.0,604.0){\rule[-0.200pt]{4.818pt}{0.400pt}}
\put(1318.0,615.0){\rule[-0.200pt]{4.818pt}{0.400pt}}
\put(1334.0,598.0){\rule[-0.200pt]{0.400pt}{2.650pt}}
\put(1324.0,598.0){\rule[-0.200pt]{4.818pt}{0.400pt}}
\put(1324.0,609.0){\rule[-0.200pt]{4.818pt}{0.400pt}}
\put(1341.0,587.0){\rule[-0.200pt]{0.400pt}{2.650pt}}
\put(1331.0,587.0){\rule[-0.200pt]{4.818pt}{0.400pt}}
\put(1331.0,598.0){\rule[-0.200pt]{4.818pt}{0.400pt}}
\put(1347.0,554.0){\rule[-0.200pt]{0.400pt}{2.650pt}}
\put(1337.0,554.0){\rule[-0.200pt]{4.818pt}{0.400pt}}
\put(1337.0,565.0){\rule[-0.200pt]{4.818pt}{0.400pt}}
\put(1354.0,537.0){\rule[-0.200pt]{0.400pt}{2.650pt}}
\put(1344.0,537.0){\rule[-0.200pt]{4.818pt}{0.400pt}}
\put(1344.0,548.0){\rule[-0.200pt]{4.818pt}{0.400pt}}
\put(1360.0,520.0){\rule[-0.200pt]{0.400pt}{2.891pt}}
\put(1350.0,520.0){\rule[-0.200pt]{4.818pt}{0.400pt}}
\put(1350.0,532.0){\rule[-0.200pt]{4.818pt}{0.400pt}}
\put(1367.0,498.0){\rule[-0.200pt]{0.400pt}{2.650pt}}
\put(1357.0,498.0){\rule[-0.200pt]{4.818pt}{0.400pt}}
\put(1357.0,509.0){\rule[-0.200pt]{4.818pt}{0.400pt}}
\put(1374.0,487.0){\rule[-0.200pt]{0.400pt}{2.650pt}}
\put(1364.0,487.0){\rule[-0.200pt]{4.818pt}{0.400pt}}
\put(1364.0,498.0){\rule[-0.200pt]{4.818pt}{0.400pt}}
\put(1380.0,476.0){\rule[-0.200pt]{0.400pt}{2.650pt}}
\put(1370.0,476.0){\rule[-0.200pt]{4.818pt}{0.400pt}}
\put(1370.0,487.0){\rule[-0.200pt]{4.818pt}{0.400pt}}
\put(1387.0,471.0){\rule[-0.200pt]{0.400pt}{2.650pt}}
\put(1377.0,471.0){\rule[-0.200pt]{4.818pt}{0.400pt}}
\put(1377.0,482.0){\rule[-0.200pt]{4.818pt}{0.400pt}}
\put(1393.0,465.0){\rule[-0.200pt]{0.400pt}{2.650pt}}
\put(1383.0,465.0){\rule[-0.200pt]{4.818pt}{0.400pt}}
\put(1383.0,476.0){\rule[-0.200pt]{4.818pt}{0.400pt}}
\put(1400.0,465.0){\rule[-0.200pt]{0.400pt}{2.650pt}}
\put(1390.0,465.0){\rule[-0.200pt]{4.818pt}{0.400pt}}
\put(130,598){\makebox(0,0){$+$}}
\put(137,476){\makebox(0,0){$+$}}
\put(143,376){\makebox(0,0){$+$}}
\put(150,287){\makebox(0,0){$+$}}
\put(156,282){\makebox(0,0){$+$}}
\put(163,348){\makebox(0,0){$+$}}
\put(169,415){\makebox(0,0){$+$}}
\put(176,493){\makebox(0,0){$+$}}
\put(182,565){\makebox(0,0){$+$}}
\put(189,637){\makebox(0,0){$+$}}
\put(195,693){\makebox(0,0){$+$}}
\put(202,742){\makebox(0,0){$+$}}
\put(209,776){\makebox(0,0){$+$}}
\put(215,781){\makebox(0,0){$+$}}
\put(222,781){\makebox(0,0){$+$}}
\put(228,759){\makebox(0,0){$+$}}
\put(235,720){\makebox(0,0){$+$}}
\put(241,670){\makebox(0,0){$+$}}
\put(248,615){\makebox(0,0){$+$}}
\put(254,548){\makebox(0,0){$+$}}
\put(261,482){\makebox(0,0){$+$}}
\put(267,421){\makebox(0,0){$+$}}
\put(274,365){\makebox(0,0){$+$}}
\put(281,315){\makebox(0,0){$+$}}
\put(287,287){\makebox(0,0){$+$}}
\put(294,265){\makebox(0,0){$+$}}
\put(300,265){\makebox(0,0){$+$}}
\put(307,287){\makebox(0,0){$+$}}
\put(313,310){\makebox(0,0){$+$}}
\put(320,354){\makebox(0,0){$+$}}
\put(326,404){\makebox(0,0){$+$}}
\put(339,515){\makebox(0,0){$+$}}
\put(346,559){\makebox(0,0){$+$}}
\put(353,598){\makebox(0,0){$+$}}
\put(359,637){\makebox(0,0){$+$}}
\put(366,659){\makebox(0,0){$+$}}
\put(372,665){\makebox(0,0){$+$}}
\put(379,654){\makebox(0,0){$+$}}
\put(385,643){\makebox(0,0){$+$}}
\put(392,609){\makebox(0,0){$+$}}
\put(398,570){\makebox(0,0){$+$}}
\put(405,515){\makebox(0,0){$+$}}
\put(411,465){\makebox(0,0){$+$}}
\put(418,421){\makebox(0,0){$+$}}
\put(425,371){\makebox(0,0){$+$}}
\put(431,332){\makebox(0,0){$+$}}
\put(438,298){\makebox(0,0){$+$}}
\put(444,282){\makebox(0,0){$+$}}
\put(451,265){\makebox(0,0){$+$}}
\put(457,271){\makebox(0,0){$+$}}
\put(464,282){\makebox(0,0){$+$}}
\put(470,304){\makebox(0,0){$+$}}
\put(477,354){\makebox(0,0){$+$}}
\put(483,376){\makebox(0,0){$+$}}
\put(490,421){\makebox(0,0){$+$}}
\put(497,471){\makebox(0,0){$+$}}
\put(503,504){\makebox(0,0){$+$}}
\put(510,543){\makebox(0,0){$+$}}
\put(516,570){\makebox(0,0){$+$}}
\put(523,598){\makebox(0,0){$+$}}
\put(529,609){\makebox(0,0){$+$}}
\put(536,620){\makebox(0,0){$+$}}
\put(542,615){\makebox(0,0){$+$}}
\put(549,604){\makebox(0,0){$+$}}
\put(555,576){\makebox(0,0){$+$}}
\put(562,548){\makebox(0,0){$+$}}
\put(569,515){\makebox(0,0){$+$}}
\put(575,415){\makebox(0,0){$+$}}
\put(582,426){\makebox(0,0){$+$}}
\put(588,398){\makebox(0,0){$+$}}
\put(595,365){\makebox(0,0){$+$}}
\put(601,343){\makebox(0,0){$+$}}
\put(608,326){\makebox(0,0){$+$}}
\put(614,315){\makebox(0,0){$+$}}
\put(621,321){\makebox(0,0){$+$}}
\put(627,332){\makebox(0,0){$+$}}
\put(634,348){\makebox(0,0){$+$}}
\put(641,371){\makebox(0,0){$+$}}
\put(647,404){\makebox(0,0){$+$}}
\put(654,432){\makebox(0,0){$+$}}
\put(660,465){\makebox(0,0){$+$}}
\put(667,504){\makebox(0,0){$+$}}
\put(673,537){\makebox(0,0){$+$}}
\put(680,554){\makebox(0,0){$+$}}
\put(686,587){\makebox(0,0){$+$}}
\put(693,593){\makebox(0,0){$+$}}
\put(699,587){\makebox(0,0){$+$}}
\put(706,582){\makebox(0,0){$+$}}
\put(713,570){\makebox(0,0){$+$}}
\put(719,554){\makebox(0,0){$+$}}
\put(726,526){\makebox(0,0){$+$}}
\put(732,498){\makebox(0,0){$+$}}
\put(739,471){\makebox(0,0){$+$}}
\put(745,437){\makebox(0,0){$+$}}
\put(752,415){\makebox(0,0){$+$}}
\put(758,393){\makebox(0,0){$+$}}
\put(765,371){\makebox(0,0){$+$}}
\put(771,360){\makebox(0,0){$+$}}
\put(778,354){\makebox(0,0){$+$}}
\put(785,354){\makebox(0,0){$+$}}
\put(791,365){\makebox(0,0){$+$}}
\put(798,376){\makebox(0,0){$+$}}
\put(804,398){\makebox(0,0){$+$}}
\put(811,421){\makebox(0,0){$+$}}
\put(817,448){\makebox(0,0){$+$}}
\put(824,482){\makebox(0,0){$+$}}
\put(830,498){\makebox(0,0){$+$}}
\put(837,520){\makebox(0,0){$+$}}
\put(843,543){\makebox(0,0){$+$}}
\put(850,554){\makebox(0,0){$+$}}
\put(856,565){\makebox(0,0){$+$}}
\put(863,565){\makebox(0,0){$+$}}
\put(870,559){\makebox(0,0){$+$}}
\put(876,548){\makebox(0,0){$+$}}
\put(883,537){\makebox(0,0){$+$}}
\put(902,459){\makebox(0,0){$+$}}
\put(909,421){\makebox(0,0){$+$}}
\put(915,398){\makebox(0,0){$+$}}
\put(922,382){\makebox(0,0){$+$}}
\put(928,393){\makebox(0,0){$+$}}
\put(935,404){\makebox(0,0){$+$}}
\put(942,432){\makebox(0,0){$+$}}
\put(948,471){\makebox(0,0){$+$}}
\put(955,504){\makebox(0,0){$+$}}
\put(961,537){\makebox(0,0){$+$}}
\put(968,565){\makebox(0,0){$+$}}
\put(974,593){\makebox(0,0){$+$}}
\put(981,626){\makebox(0,0){$+$}}
\put(987,631){\makebox(0,0){$+$}}
\put(994,648){\makebox(0,0){$+$}}
\put(1000,654){\makebox(0,0){$+$}}
\put(1007,648){\makebox(0,0){$+$}}
\put(1014,637){\makebox(0,0){$+$}}
\put(1020,609){\makebox(0,0){$+$}}
\put(1027,582){\makebox(0,0){$+$}}
\put(1033,548){\makebox(0,0){$+$}}
\put(1040,515){\makebox(0,0){$+$}}
\put(1046,487){\makebox(0,0){$+$}}
\put(1053,465){\makebox(0,0){$+$}}
\put(1059,448){\makebox(0,0){$+$}}
\put(1066,437){\makebox(0,0){$+$}}
\put(1072,432){\makebox(0,0){$+$}}
\put(1079,432){\makebox(0,0){$+$}}
\put(1086,437){\makebox(0,0){$+$}}
\put(1092,443){\makebox(0,0){$+$}}
\put(1099,459){\makebox(0,0){$+$}}
\put(1105,482){\makebox(0,0){$+$}}
\put(1112,504){\makebox(0,0){$+$}}
\put(1118,526){\makebox(0,0){$+$}}
\put(1125,559){\makebox(0,0){$+$}}
\put(1131,582){\makebox(0,0){$+$}}
\put(1138,604){\makebox(0,0){$+$}}
\put(1144,620){\makebox(0,0){$+$}}
\put(1151,637){\makebox(0,0){$+$}}
\put(1158,631){\makebox(0,0){$+$}}
\put(1164,626){\makebox(0,0){$+$}}
\put(1171,615){\makebox(0,0){$+$}}
\put(1177,604){\makebox(0,0){$+$}}
\put(1184,587){\makebox(0,0){$+$}}
\put(1190,570){\makebox(0,0){$+$}}
\put(1197,548){\makebox(0,0){$+$}}
\put(1203,526){\makebox(0,0){$+$}}
\put(1210,504){\makebox(0,0){$+$}}
\put(1216,487){\makebox(0,0){$+$}}
\put(1223,465){\makebox(0,0){$+$}}
\put(1230,454){\makebox(0,0){$+$}}
\put(1236,448){\makebox(0,0){$+$}}
\put(1243,448){\makebox(0,0){$+$}}
\put(1249,454){\makebox(0,0){$+$}}
\put(1256,471){\makebox(0,0){$+$}}
\put(1262,482){\makebox(0,0){$+$}}
\put(1269,498){\makebox(0,0){$+$}}
\put(1275,520){\makebox(0,0){$+$}}
\put(1282,537){\makebox(0,0){$+$}}
\put(1288,559){\makebox(0,0){$+$}}
\put(1295,576){\makebox(0,0){$+$}}
\put(1302,593){\makebox(0,0){$+$}}
\put(1308,604){\makebox(0,0){$+$}}
\put(1315,609){\makebox(0,0){$+$}}
\put(1321,615){\makebox(0,0){$+$}}
\put(1328,609){\makebox(0,0){$+$}}
\put(1334,604){\makebox(0,0){$+$}}
\put(1341,593){\makebox(0,0){$+$}}
\put(1347,559){\makebox(0,0){$+$}}
\put(1354,543){\makebox(0,0){$+$}}
\put(1360,526){\makebox(0,0){$+$}}
\put(1367,504){\makebox(0,0){$+$}}
\put(1374,493){\makebox(0,0){$+$}}
\put(1380,482){\makebox(0,0){$+$}}
\put(1387,476){\makebox(0,0){$+$}}
\put(1393,471){\makebox(0,0){$+$}}
\put(1400,471){\makebox(0,0){$+$}}
\put(1349,819){\makebox(0,0){$+$}}
\put(1390.0,476.0){\rule[-0.200pt]{4.818pt}{0.400pt}}
\put(1279,778){\makebox(0,0)[r]{Fit v prvej polohe}}
\multiput(1299,778)(20.756,0.000){5}{\usebox{\plotpoint}}
\put(1399,778){\usebox{\plotpoint}}
\put(130,82){\usebox{\plotpoint}}
\put(130.00,82.00){\usebox{\plotpoint}}
\put(150.76,82.00){\usebox{\plotpoint}}
\put(171.51,82.00){\usebox{\plotpoint}}
\put(192.27,82.00){\usebox{\plotpoint}}
\put(213.02,82.00){\usebox{\plotpoint}}
\put(233.78,82.00){\usebox{\plotpoint}}
\put(254.53,82.00){\usebox{\plotpoint}}
\put(275.29,82.00){\usebox{\plotpoint}}
\put(296.04,82.00){\usebox{\plotpoint}}
\put(316.80,82.00){\usebox{\plotpoint}}
\multiput(335,82)(0.538,20.749){25}{\usebox{\plotpoint}}
\multiput(348,583)(4.132,20.340){3}{\usebox{\plotpoint}}
\put(371.09,660.98){\usebox{\plotpoint}}
\put(380.11,650.44){\usebox{\plotpoint}}
\multiput(387,634)(3.459,-20.465){4}{\usebox{\plotpoint}}
\multiput(399,563)(2.843,-20.560){4}{\usebox{\plotpoint}}
\multiput(412,469)(2.904,-20.551){5}{\usebox{\plotpoint}}
\multiput(425,377)(3.843,-20.397){3}{\usebox{\plotpoint}}
\multiput(438,308)(7.812,-19.229){2}{\usebox{\plotpoint}}
\put(459.11,284.79){\usebox{\plotpoint}}
\multiput(463,289)(4.944,20.158){2}{\usebox{\plotpoint}}
\multiput(476,342)(3.455,20.466){4}{\usebox{\plotpoint}}
\multiput(489,419)(3.138,20.517){4}{\usebox{\plotpoint}}
\multiput(502,504)(3.738,20.416){4}{\usebox{\plotpoint}}
\multiput(515,575)(6.415,19.739){2}{\usebox{\plotpoint}}
\multiput(540,618)(7.227,-19.457){2}{\usebox{\plotpoint}}
\multiput(553,583)(4.195,-20.327){3}{\usebox{\plotpoint}}
\multiput(566,520)(3.545,-20.451){4}{\usebox{\plotpoint}}
\multiput(579,445)(3.843,-20.397){3}{\usebox{\plotpoint}}
\multiput(592,376)(5.533,-20.004){3}{\usebox{\plotpoint}}
\put(615.69,316.53){\usebox{\plotpoint}}
\put(627.21,330.71){\usebox{\plotpoint}}
\multiput(630,335)(5.426,20.034){2}{\usebox{\plotpoint}}
\multiput(643,383)(4.070,20.352){3}{\usebox{\plotpoint}}
\multiput(656,448)(4.132,20.340){3}{\usebox{\plotpoint}}
\multiput(669,512)(4.844,20.182){3}{\usebox{\plotpoint}}
\put(686.34,571.86){\usebox{\plotpoint}}
\put(698.11,583.78){\usebox{\plotpoint}}
\multiput(707,579)(7.049,-19.522){2}{\usebox{\plotpoint}}
\multiput(720,543)(4.944,-20.158){2}{\usebox{\plotpoint}}
\multiput(733,490)(4.466,-20.269){3}{\usebox{\plotpoint}}
\multiput(746,431)(4.844,-20.182){3}{\usebox{\plotpoint}}
\put(763.38,368.58){\usebox{\plotpoint}}
\put(772.53,350.53){\usebox{\plotpoint}}
\put(788.17,354.69){\usebox{\plotpoint}}
\multiput(797,371)(6.006,19.867){3}{\usebox{\plotpoint}}
\multiput(810,414)(5.034,20.136){2}{\usebox{\plotpoint}}
\multiput(823,466)(5.034,20.136){3}{\usebox{\plotpoint}}
\put(842.23,532.92){\usebox{\plotpoint}}
\put(851.39,551.13){\usebox{\plotpoint}}
\put(866.43,554.57){\usebox{\plotpoint}}
\multiput(874,547)(0.580,-20.747){22}{\usebox{\plotpoint}}
\put(888.47,82.00){\usebox{\plotpoint}}
\put(909.23,82.00){\usebox{\plotpoint}}
\put(929.98,82.00){\usebox{\plotpoint}}
\put(950.74,82.00){\usebox{\plotpoint}}
\put(971.49,82.00){\usebox{\plotpoint}}
\put(992.25,82.00){\usebox{\plotpoint}}
\put(1013.00,82.00){\usebox{\plotpoint}}
\put(1033.76,82.00){\usebox{\plotpoint}}
\put(1054.51,82.00){\usebox{\plotpoint}}
\put(1075.27,82.00){\usebox{\plotpoint}}
\put(1096.03,82.00){\usebox{\plotpoint}}
\put(1116.78,82.00){\usebox{\plotpoint}}
\put(1137.54,82.00){\usebox{\plotpoint}}
\put(1158.29,82.00){\usebox{\plotpoint}}
\put(1179.05,82.00){\usebox{\plotpoint}}
\put(1199.80,82.00){\usebox{\plotpoint}}
\put(1220.56,82.00){\usebox{\plotpoint}}
\put(1241.31,82.00){\usebox{\plotpoint}}
\put(1262.07,82.00){\usebox{\plotpoint}}
\put(1282.82,82.00){\usebox{\plotpoint}}
\put(1303.58,82.00){\usebox{\plotpoint}}
\put(1324.34,82.00){\usebox{\plotpoint}}
\put(1345.09,82.00){\usebox{\plotpoint}}
\put(1365.85,82.00){\usebox{\plotpoint}}
\put(1386.60,82.00){\usebox{\plotpoint}}
\put(1400,82){\usebox{\plotpoint}}
\sbox{\plotpoint}{\rule[-0.400pt]{0.800pt}{0.800pt}}%
\sbox{\plotpoint}{\rule[-0.200pt]{0.400pt}{0.400pt}}%
\put(1279,737){\makebox(0,0)[r]{Fit v druhej polohe}}
\sbox{\plotpoint}{\rule[-0.400pt]{0.800pt}{0.800pt}}%
\put(1299.0,737.0){\rule[-0.400pt]{24.090pt}{0.800pt}}
\put(130,82){\usebox{\plotpoint}}
\multiput(901.41,82.00)(0.511,14.323){17}{\rule{0.123pt}{21.400pt}}
\multiput(898.34,82.00)(12.000,273.583){2}{\rule{0.800pt}{10.700pt}}
\put(912,397.84){\rule{3.132pt}{0.800pt}}
\multiput(912.00,398.34)(6.500,-1.000){2}{\rule{1.566pt}{0.800pt}}
\multiput(926.41,399.00)(0.509,1.236){19}{\rule{0.123pt}{2.108pt}}
\multiput(923.34,399.00)(13.000,26.625){2}{\rule{0.800pt}{1.054pt}}
\multiput(939.41,430.00)(0.509,2.270){19}{\rule{0.123pt}{3.646pt}}
\multiput(936.34,430.00)(13.000,48.432){2}{\rule{0.800pt}{1.823pt}}
\multiput(952.41,486.00)(0.509,2.643){19}{\rule{0.123pt}{4.200pt}}
\multiput(949.34,486.00)(13.000,56.283){2}{\rule{0.800pt}{2.100pt}}
\multiput(965.41,551.00)(0.511,2.570){17}{\rule{0.123pt}{4.067pt}}
\multiput(962.34,551.00)(12.000,49.559){2}{\rule{0.800pt}{2.033pt}}
\multiput(977.41,609.00)(0.509,1.484){19}{\rule{0.123pt}{2.477pt}}
\multiput(974.34,609.00)(13.000,31.859){2}{\rule{0.800pt}{1.238pt}}
\multiput(989.00,647.40)(0.847,0.520){9}{\rule{1.500pt}{0.125pt}}
\multiput(989.00,644.34)(9.887,8.000){2}{\rule{0.750pt}{0.800pt}}
\multiput(1003.41,647.29)(0.509,-0.905){19}{\rule{0.123pt}{1.615pt}}
\multiput(1000.34,650.65)(13.000,-19.647){2}{\rule{0.800pt}{0.808pt}}
\multiput(1016.41,618.67)(0.509,-1.815){19}{\rule{0.123pt}{2.969pt}}
\multiput(1013.34,624.84)(13.000,-38.837){2}{\rule{0.800pt}{1.485pt}}
\multiput(1029.41,570.35)(0.509,-2.353){19}{\rule{0.123pt}{3.769pt}}
\multiput(1026.34,578.18)(13.000,-50.177){2}{\rule{0.800pt}{1.885pt}}
\multiput(1042.41,512.23)(0.511,-2.389){17}{\rule{0.123pt}{3.800pt}}
\multiput(1039.34,520.11)(12.000,-46.113){2}{\rule{0.800pt}{1.900pt}}
\multiput(1054.41,463.46)(0.509,-1.526){19}{\rule{0.123pt}{2.538pt}}
\multiput(1051.34,468.73)(13.000,-32.731){2}{\rule{0.800pt}{1.269pt}}
\multiput(1066.00,434.08)(0.536,-0.511){17}{\rule{1.067pt}{0.123pt}}
\multiput(1066.00,434.34)(10.786,-12.000){2}{\rule{0.533pt}{0.800pt}}
\multiput(1080.41,424.00)(0.509,0.533){19}{\rule{0.123pt}{1.062pt}}
\multiput(1077.34,424.00)(13.000,11.797){2}{\rule{0.800pt}{0.531pt}}
\multiput(1093.41,438.00)(0.509,1.484){19}{\rule{0.123pt}{2.477pt}}
\multiput(1090.34,438.00)(13.000,31.859){2}{\rule{0.800pt}{1.238pt}}
\multiput(1106.41,475.00)(0.509,2.022){19}{\rule{0.123pt}{3.277pt}}
\multiput(1103.34,475.00)(13.000,43.199){2}{\rule{0.800pt}{1.638pt}}
\multiput(1119.41,525.00)(0.511,2.163){17}{\rule{0.123pt}{3.467pt}}
\multiput(1116.34,525.00)(12.000,41.805){2}{\rule{0.800pt}{1.733pt}}
\multiput(1131.41,574.00)(0.509,1.526){19}{\rule{0.123pt}{2.538pt}}
\multiput(1128.34,574.00)(13.000,32.731){2}{\rule{0.800pt}{1.269pt}}
\multiput(1144.41,612.00)(0.509,0.657){19}{\rule{0.123pt}{1.246pt}}
\multiput(1141.34,612.00)(13.000,14.414){2}{\rule{0.800pt}{0.623pt}}
\multiput(1156.00,627.08)(0.847,-0.520){9}{\rule{1.500pt}{0.125pt}}
\multiput(1156.00,627.34)(9.887,-8.000){2}{\rule{0.750pt}{0.800pt}}
\multiput(1170.41,612.76)(0.509,-1.153){19}{\rule{0.123pt}{1.985pt}}
\multiput(1167.34,616.88)(13.000,-24.881){2}{\rule{0.800pt}{0.992pt}}
\multiput(1183.41,580.19)(0.509,-1.733){19}{\rule{0.123pt}{2.846pt}}
\multiput(1180.34,586.09)(13.000,-37.093){2}{\rule{0.800pt}{1.423pt}}
\multiput(1196.41,535.72)(0.511,-1.982){17}{\rule{0.123pt}{3.200pt}}
\multiput(1193.34,542.36)(12.000,-38.358){2}{\rule{0.800pt}{1.600pt}}
\multiput(1208.41,493.72)(0.509,-1.484){19}{\rule{0.123pt}{2.477pt}}
\multiput(1205.34,498.86)(13.000,-31.859){2}{\rule{0.800pt}{1.238pt}}
\multiput(1221.41,461.32)(0.509,-0.740){19}{\rule{0.123pt}{1.369pt}}
\multiput(1218.34,464.16)(13.000,-16.158){2}{\rule{0.800pt}{0.685pt}}
\put(1233,447.34){\rule{3.132pt}{0.800pt}}
\multiput(1233.00,446.34)(6.500,2.000){2}{\rule{1.566pt}{0.800pt}}
\multiput(1247.41,450.00)(0.509,0.864){19}{\rule{0.123pt}{1.554pt}}
\multiput(1244.34,450.00)(13.000,18.775){2}{\rule{0.800pt}{0.777pt}}
\multiput(1260.41,472.00)(0.511,1.575){17}{\rule{0.123pt}{2.600pt}}
\multiput(1257.34,472.00)(12.000,30.604){2}{\rule{0.800pt}{1.300pt}}
\multiput(1272.41,508.00)(0.509,1.650){19}{\rule{0.123pt}{2.723pt}}
\multiput(1269.34,508.00)(13.000,35.348){2}{\rule{0.800pt}{1.362pt}}
\multiput(1285.41,549.00)(0.509,1.402){19}{\rule{0.123pt}{2.354pt}}
\multiput(1282.34,549.00)(13.000,30.114){2}{\rule{0.800pt}{1.177pt}}
\multiput(1298.41,584.00)(0.509,0.823){19}{\rule{0.123pt}{1.492pt}}
\multiput(1295.34,584.00)(13.000,17.903){2}{\rule{0.800pt}{0.746pt}}
\put(1310,604.34){\rule{3.132pt}{0.800pt}}
\multiput(1310.00,603.34)(6.500,2.000){2}{\rule{1.566pt}{0.800pt}}
\multiput(1324.41,602.08)(0.509,-0.616){19}{\rule{0.123pt}{1.185pt}}
\multiput(1321.34,604.54)(13.000,-13.541){2}{\rule{0.800pt}{0.592pt}}
\multiput(1337.41,581.87)(0.511,-1.304){17}{\rule{0.123pt}{2.200pt}}
\multiput(1334.34,586.43)(12.000,-25.434){2}{\rule{0.800pt}{1.100pt}}
\multiput(1349.41,550.97)(0.509,-1.443){19}{\rule{0.123pt}{2.415pt}}
\multiput(1346.34,555.99)(13.000,-30.987){2}{\rule{0.800pt}{1.208pt}}
\multiput(1362.41,515.74)(0.509,-1.319){19}{\rule{0.123pt}{2.231pt}}
\multiput(1359.34,520.37)(13.000,-28.370){2}{\rule{0.800pt}{1.115pt}}
\multiput(1375.41,485.55)(0.509,-0.864){19}{\rule{0.123pt}{1.554pt}}
\multiput(1372.34,488.77)(13.000,-18.775){2}{\rule{0.800pt}{0.777pt}}
\multiput(1387.00,468.06)(1.768,-0.560){3}{\rule{2.280pt}{0.135pt}}
\multiput(1387.00,468.34)(8.268,-5.000){2}{\rule{1.140pt}{0.800pt}}
\put(130.0,82.0){\rule[-0.400pt]{185.493pt}{0.800pt}}
\sbox{\plotpoint}{\rule[-0.500pt]{1.000pt}{1.000pt}}%
\sbox{\plotpoint}{\rule[-0.200pt]{0.400pt}{0.400pt}}%
\put(1279,696){\makebox(0,0)[r]{S^1}}
\sbox{\plotpoint}{\rule[-0.500pt]{1.000pt}{1.000pt}}%
\multiput(1299,696)(20.756,0.000){5}{\usebox{\plotpoint}}
\put(1399,696){\usebox{\plotpoint}}
\put(130,459){\usebox{\plotpoint}}
\put(130.00,459.00){\usebox{\plotpoint}}
\put(150.76,459.00){\usebox{\plotpoint}}
\put(171.51,459.00){\usebox{\plotpoint}}
\put(192.27,459.00){\usebox{\plotpoint}}
\put(213.02,459.00){\usebox{\plotpoint}}
\put(233.78,459.00){\usebox{\plotpoint}}
\put(254.53,459.00){\usebox{\plotpoint}}
\put(275.29,459.00){\usebox{\plotpoint}}
\put(296.04,459.00){\usebox{\plotpoint}}
\put(316.80,459.00){\usebox{\plotpoint}}
\put(337.55,459.00){\usebox{\plotpoint}}
\put(358.31,459.00){\usebox{\plotpoint}}
\put(379.07,459.00){\usebox{\plotpoint}}
\put(399.82,459.00){\usebox{\plotpoint}}
\put(420.58,459.00){\usebox{\plotpoint}}
\put(441.33,459.00){\usebox{\plotpoint}}
\put(462.09,459.00){\usebox{\plotpoint}}
\put(482.84,459.00){\usebox{\plotpoint}}
\put(503.60,459.00){\usebox{\plotpoint}}
\put(524.35,459.00){\usebox{\plotpoint}}
\put(545.11,459.00){\usebox{\plotpoint}}
\put(565.87,459.00){\usebox{\plotpoint}}
\put(586.62,459.00){\usebox{\plotpoint}}
\put(607.38,459.00){\usebox{\plotpoint}}
\put(628.13,459.00){\usebox{\plotpoint}}
\put(648.89,459.00){\usebox{\plotpoint}}
\put(669.64,459.00){\usebox{\plotpoint}}
\put(690.40,459.00){\usebox{\plotpoint}}
\put(711.15,459.00){\usebox{\plotpoint}}
\put(731.91,459.00){\usebox{\plotpoint}}
\put(752.66,459.00){\usebox{\plotpoint}}
\put(773.42,459.00){\usebox{\plotpoint}}
\put(794.18,459.00){\usebox{\plotpoint}}
\put(814.93,459.00){\usebox{\plotpoint}}
\put(835.69,459.00){\usebox{\plotpoint}}
\put(856.44,459.00){\usebox{\plotpoint}}
\put(877.20,459.00){\usebox{\plotpoint}}
\put(897.95,459.00){\usebox{\plotpoint}}
\put(918.71,459.00){\usebox{\plotpoint}}
\put(939.46,459.00){\usebox{\plotpoint}}
\put(960.22,459.00){\usebox{\plotpoint}}
\put(980.98,459.00){\usebox{\plotpoint}}
\put(1001.73,459.00){\usebox{\plotpoint}}
\put(1022.49,459.00){\usebox{\plotpoint}}
\put(1043.24,459.00){\usebox{\plotpoint}}
\put(1064.00,459.00){\usebox{\plotpoint}}
\put(1084.75,459.00){\usebox{\plotpoint}}
\put(1105.51,459.00){\usebox{\plotpoint}}
\put(1126.26,459.00){\usebox{\plotpoint}}
\put(1147.02,459.00){\usebox{\plotpoint}}
\put(1167.77,459.00){\usebox{\plotpoint}}
\put(1188.53,459.00){\usebox{\plotpoint}}
\put(1209.29,459.00){\usebox{\plotpoint}}
\put(1230.04,459.00){\usebox{\plotpoint}}
\put(1250.80,459.00){\usebox{\plotpoint}}
\put(1271.55,459.00){\usebox{\plotpoint}}
\put(1292.31,459.00){\usebox{\plotpoint}}
\put(1313.06,459.00){\usebox{\plotpoint}}
\put(1333.82,459.00){\usebox{\plotpoint}}
\put(1354.57,459.00){\usebox{\plotpoint}}
\put(1375.33,459.00){\usebox{\plotpoint}}
\put(1396.09,459.00){\usebox{\plotpoint}}
\put(1400,459){\usebox{\plotpoint}}
\sbox{\plotpoint}{\rule[-0.600pt]{1.200pt}{1.200pt}}%
\sbox{\plotpoint}{\rule[-0.200pt]{0.400pt}{0.400pt}}%
\put(1279,655){\makebox(0,0)[r]{S^2}}
\sbox{\plotpoint}{\rule[-0.600pt]{1.200pt}{1.200pt}}%
\put(1299.0,655.0){\rule[-0.600pt]{24.090pt}{1.200pt}}
\put(130,532){\usebox{\plotpoint}}
\put(130.0,532.0){\rule[-0.600pt]{305.943pt}{1.200pt}}
\sbox{\plotpoint}{\rule[-0.200pt]{0.400pt}{0.400pt}}%
\put(130.0,82.0){\rule[-0.200pt]{0.400pt}{187.179pt}}
\put(130.0,82.0){\rule[-0.200pt]{315.338pt}{0.400pt}}
\put(1439.0,82.0){\rule[-0.200pt]{0.400pt}{187.179pt}}
\put(130.0,859.0){\rule[-0.200pt]{315.338pt}{0.400pt}}
\end{picture}

\caption{Závislosť uhlovej $\omega$ frekvencie na čase $t$ pri zmene polohy závaží.}  \label{G_1}
\end{figure}


\begin{figure}
% GNUPLOT: LaTeX picture
\setlength{\unitlength}{0.240900pt}
\ifx\plotpoint\undefined\newsavebox{\plotpoint}\fi
\begin{picture}(1500,900)(0,0)
\sbox{\plotpoint}{\rule[-0.200pt]{0.400pt}{0.400pt}}%
\put(171.0,131.0){\rule[-0.200pt]{4.818pt}{0.400pt}}
\put(151,131){\makebox(0,0)[r]{ 0.1}}
\put(1419.0,131.0){\rule[-0.200pt]{4.818pt}{0.400pt}}
\put(171.0,235.0){\rule[-0.200pt]{4.818pt}{0.400pt}}
\put(151,235){\makebox(0,0)[r]{ 0.2}}
\put(1419.0,235.0){\rule[-0.200pt]{4.818pt}{0.400pt}}
\put(171.0,339.0){\rule[-0.200pt]{4.818pt}{0.400pt}}
\put(151,339){\makebox(0,0)[r]{ 0.3}}
\put(1419.0,339.0){\rule[-0.200pt]{4.818pt}{0.400pt}}
\put(171.0,443.0){\rule[-0.200pt]{4.818pt}{0.400pt}}
\put(151,443){\makebox(0,0)[r]{ 0.4}}
\put(1419.0,443.0){\rule[-0.200pt]{4.818pt}{0.400pt}}
\put(171.0,547.0){\rule[-0.200pt]{4.818pt}{0.400pt}}
\put(151,547){\makebox(0,0)[r]{ 0.5}}
\put(1419.0,547.0){\rule[-0.200pt]{4.818pt}{0.400pt}}
\put(171.0,651.0){\rule[-0.200pt]{4.818pt}{0.400pt}}
\put(151,651){\makebox(0,0)[r]{ 0.6}}
\put(1419.0,651.0){\rule[-0.200pt]{4.818pt}{0.400pt}}
\put(171.0,755.0){\rule[-0.200pt]{4.818pt}{0.400pt}}
\put(151,755){\makebox(0,0)[r]{ 0.7}}
\put(1419.0,755.0){\rule[-0.200pt]{4.818pt}{0.400pt}}
\put(171.0,859.0){\rule[-0.200pt]{4.818pt}{0.400pt}}
\put(151,859){\makebox(0,0)[r]{ 0.8}}
\put(1419.0,859.0){\rule[-0.200pt]{4.818pt}{0.400pt}}
\put(171.0,131.0){\rule[-0.200pt]{0.400pt}{4.818pt}}
\put(171,90){\makebox(0,0){ 0}}
\put(171.0,839.0){\rule[-0.200pt]{0.400pt}{4.818pt}}
\put(382.0,131.0){\rule[-0.200pt]{0.400pt}{4.818pt}}
\put(382,90){\makebox(0,0){ 1}}
\put(382.0,839.0){\rule[-0.200pt]{0.400pt}{4.818pt}}
\put(594.0,131.0){\rule[-0.200pt]{0.400pt}{4.818pt}}
\put(594,90){\makebox(0,0){ 2}}
\put(594.0,839.0){\rule[-0.200pt]{0.400pt}{4.818pt}}
\put(805.0,131.0){\rule[-0.200pt]{0.400pt}{4.818pt}}
\put(805,90){\makebox(0,0){ 3}}
\put(805.0,839.0){\rule[-0.200pt]{0.400pt}{4.818pt}}
\put(1016.0,131.0){\rule[-0.200pt]{0.400pt}{4.818pt}}
\put(1016,90){\makebox(0,0){ 4}}
\put(1016.0,839.0){\rule[-0.200pt]{0.400pt}{4.818pt}}
\put(1228.0,131.0){\rule[-0.200pt]{0.400pt}{4.818pt}}
\put(1228,90){\makebox(0,0){ 5}}
\put(1228.0,839.0){\rule[-0.200pt]{0.400pt}{4.818pt}}
\put(1439.0,131.0){\rule[-0.200pt]{0.400pt}{4.818pt}}
\put(1439,90){\makebox(0,0){ 6}}
\put(1439.0,839.0){\rule[-0.200pt]{0.400pt}{4.818pt}}
\put(171.0,131.0){\rule[-0.200pt]{0.400pt}{175.375pt}}
\put(171.0,131.0){\rule[-0.200pt]{305.461pt}{0.400pt}}
\put(1439.0,131.0){\rule[-0.200pt]{0.400pt}{175.375pt}}
\put(171.0,859.0){\rule[-0.200pt]{305.461pt}{0.400pt}}
\put(30,495){\makebox(0,0){\popi{\omega}{rad \cdot s^{-2}}}}
\put(805,29){\makebox(0,0){\popi{t}{s}}}
\put(1279,819){\makebox(0,0)[r]{namerná dáta}}
\put(182,485){\makebox(0,0){$+$}}
\put(203,550){\makebox(0,0){$+$}}
\put(224,631){\makebox(0,0){$+$}}
\put(245,697){\makebox(0,0){$+$}}
\put(266,729){\makebox(0,0){$+$}}
\put(287,795){\makebox(0,0){$+$}}
\put(308,811){\makebox(0,0){$+$}}
\put(330,811){\makebox(0,0){$+$}}
\put(351,779){\makebox(0,0){$+$}}
\put(372,729){\makebox(0,0){$+$}}
\put(393,713){\makebox(0,0){$+$}}
\put(414,648){\makebox(0,0){$+$}}
\put(435,566){\makebox(0,0){$+$}}
\put(456,501){\makebox(0,0){$+$}}
\put(477,436){\makebox(0,0){$+$}}
\put(499,354){\makebox(0,0){$+$}}
\put(520,321){\makebox(0,0){$+$}}
\put(541,272){\makebox(0,0){$+$}}
\put(562,239){\makebox(0,0){$+$}}
\put(583,223){\makebox(0,0){$+$}}
\put(604,223){\makebox(0,0){$+$}}
\put(625,207){\makebox(0,0){$+$}}
\put(647,223){\makebox(0,0){$+$}}
\put(668,239){\makebox(0,0){$+$}}
\put(689,272){\makebox(0,0){$+$}}
\put(710,321){\makebox(0,0){$+$}}
\put(731,354){\makebox(0,0){$+$}}
\put(752,402){\makebox(0,0){$+$}}
\put(773,451){\makebox(0,0){$+$}}
\put(794,485){\makebox(0,0){$+$}}
\put(816,517){\makebox(0,0){$+$}}
\put(837,550){\makebox(0,0){$+$}}
\put(858,582){\makebox(0,0){$+$}}
\put(879,566){\makebox(0,0){$+$}}
\put(900,582){\makebox(0,0){$+$}}
\put(921,582){\makebox(0,0){$+$}}
\put(942,533){\makebox(0,0){$+$}}
\put(964,533){\makebox(0,0){$+$}}
\put(985,501){\makebox(0,0){$+$}}
\put(1006,468){\makebox(0,0){$+$}}
\put(1027,419){\makebox(0,0){$+$}}
\put(1048,387){\makebox(0,0){$+$}}
\put(1069,354){\makebox(0,0){$+$}}
\put(1090,354){\makebox(0,0){$+$}}
\put(1111,305){\makebox(0,0){$+$}}
\put(1133,288){\makebox(0,0){$+$}}
\put(1154,288){\makebox(0,0){$+$}}
\put(1175,272){\makebox(0,0){$+$}}
\put(1196,288){\makebox(0,0){$+$}}
\put(1217,288){\makebox(0,0){$+$}}
\put(1238,321){\makebox(0,0){$+$}}
\put(1259,337){\makebox(0,0){$+$}}
\put(1281,337){\makebox(0,0){$+$}}
\put(1302,370){\makebox(0,0){$+$}}
\put(1323,402){\makebox(0,0){$+$}}
\put(1344,402){\makebox(0,0){$+$}}
\put(1365,436){\makebox(0,0){$+$}}
\put(1386,436){\makebox(0,0){$+$}}
\put(1407,451){\makebox(0,0){$+$}}
\put(1428,468){\makebox(0,0){$+$}}
\put(1349,819){\makebox(0,0){$+$}}
\put(171.0,131.0){\rule[-0.200pt]{0.400pt}{175.375pt}}
\put(171.0,131.0){\rule[-0.200pt]{305.461pt}{0.400pt}}
\put(1439.0,131.0){\rule[-0.200pt]{0.400pt}{175.375pt}}
\put(171.0,859.0){\rule[-0.200pt]{305.461pt}{0.400pt}}
\end{picture}

\caption{Závislosť uhlovej $\omega$ frekvencie na čase $t$ pri precesí.}  \label{G_1}
\end{figure}



\section{Diskusia \& Záver}
Podľa vzťahu \ref{R_10} sme vypočítali moment zotrvačnosti $I_D="93 g m^2"$ a  a podľa vzťahu \ref{R_11} obruče $I = "5,32 g m^2"$. Kde jasne vidíme, že hodnoty sa od nami nameraných výrazne líšia.
Podľa vzťahu \ref{R_11} sme sme určili teoretický $I_0$ ako $I_0 = I_{PV+A} - I_{A} - a^2 M = "0.0378 kg m^2" $, čo je opäť v rozpore s nameranými hodnotami. 

Pri overovaní ZZMH sa opäť vo výsledkoch ukazuje veľká nepresnosť a nezrovnalosť nameraných dát s teóriu.
Vyzerá to že naše meranie zaťažené systematickou chybou, ktorá podľa môjho predpokladu vznikla pri určovaní polomeru kladky. Je to jediný parameter, ktorý sa vyskytuje všade vo výpočtoch.
Pri takejto výraznej (asi) systematickej chybe je zahŕňať do odhadov nepresností aj iné chyby skoro zbytočné ale predsa ich tu spomeniem.
Predovšetkým sa jedná o trenie pri otáčaní aparatúry, hmotné lanko, nezapočítaný moment zotrvačnosti podstavca, namotávanie nitky a tým sa zväčšuje polomer kladky, nezapočítaný moment zotrvačnosti druhej kladky.


Naopak pri overovaní precesie a vzťahov pre precesiu giroskopu, vidíme veľmi peknú zhodu dát sa teóriou. 
Tu opäť môžeme diskutovať o vzniku nepresností merania. Meranie hodnoty $d$ bolo nepriame a rozdelená na cca 3 sub-merania každé s presnosťou $"0.5 cm"$, gumička na prevode dosť preklzávala, a teda neprenášala všetky otáčky. 


\begin{thebibliography}{2}
\bibitem{C_1}
Dynamika rotačního pohybu [cit. 02.01.2017]Dostupné po prihlásení z Kurz: Fyzikální praktikum I:\url{https://praktikum.fjfi.cvut.cz/pluginfile.php/133/mod_resource/content/6/Rotace_161003.pdf}

\end{thebibliography}

\end{document}


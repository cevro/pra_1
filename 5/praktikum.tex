\documentclass[a4paper,10pt]{article}
%\usepackage[IL2]{fontenc}
\usepackage[utf8x]{inputenc}
\usepackage[czech]{babel}
\usepackage{amsfonts,amsmath,amssymb,graphicx,color}
%\usepackage[total={17cm,27cm}, top=2cm, left=2cm, includefoot]{geometry}
%\usepackage{fancyhdr}
\usepackage{fkssugar}
\usepackage{hyperref}

%\usepackage{caption}
\renewcommand{\popi}[2]{$#1$[\jd{#2}]}
\renewcommand{\figurename}{Obr.}
\addto\captionsczech{\renewcommand{\figurename}{Obr.}}
\addto\captionsczech{\renewcommand{\tablename}{Tab.}}

\begin{document}
\def\mean#1{\left< #1 \right>}
\noindent
{\large Fyzikální praktikum 1.} \hfil {\large FJFI ČVUT V Praze}\\
\noindent
{\large\textbf{pracovní úkol \# 5}}
\begin{center}
{\large\textit{Měření Poissonovy konstanty a dutých objemů}}
\end{center}
\noindent
\rule{\textwidth}{1px}
\vspace{\baselineskip}

\emph{Michal Červeňák}
\par
\vspace{\baselineskip}
\begin{minipage}[l]{0.5\textwidth}%
\textit{dátum merania:}~17.10. 2016\\%
%\vspace{\baselineskip}%
\par%
\noindent%
\textit{skupina:}~4\\%
%\vspace{\baselineskip}%
\par%
\noindent%
\textit{Klasifikace:}\dotfill\\%
\end{minipage}

\section{Úkol 1}
\subsection{Pracovní úkol}

\begin{enumerate}
\item Změřte Poissonovu konstantu metodou kmitajícího pístku.
\item Změřte Poissonovu konstantu Clément-Désormesovou metodou. Nezapomeňte provést
opravu vašeho měření na systematické chyby.
\item Oba výsledky vzájemně porovnejte (procentuálně) a diskutujte, jestli je v rámci chyb
můžete považovat za stejná.
\end{enumerate}


\subsection{Postup merania}
\subsubsection{Metoda kmitajícího pístku}
\begin{enumerate}
\item ventilom bol nastavený prúd vzduchu tak aby piest kmital medzi značkami
\item bol spustený digitálny čítač kmitov a nastavený na počítanie kmitov po $t="300 s"$.
\item po uplynutí intervalu boli dáta zaznamenané a opätovné spustenie počítanie. 
\end{enumerate}


\subsubsection{ Clémentova-Désormesova metoda}

\begin{enumerate}
\item Nádoba bola natlakovaná pomocou mechu, bol uzavretý prívodný ventil
\item tlak v nádobe bol odmeraný
\item pomocou ventilu bol tlak vyrovnaný s atmosferickým, pričom bol zaznamenaný čas otvorenia ventilu.
\item počkalo sa $\sim "1 min"$ na ustálenie teplôt v nádobe s okolím a následne bol zmeraný opäť tlak v aparatúre.
\end{enumerate}



\subsection{Pomôcky}
Barometr, aparatura na měěené Poissonovy konstanty Clément-Désormesovou metodou,
aparatura pro měření Poissonovy konstanty metodou kmitajícího pístku.

\subsection{Teória}
Poissonova konstanta $\kappa$ j pomer merného tepla $C_p$ pri stálom objeme a pri stálom objeme $C_V$, teda 
\eq{
\kappa = \frac{C_p}{C_V}\,.
}

\subsubsection{ Clémentova-Désormesova metoda}
Metóda určuje Poissonova konstanta z adiabatického deja, pri ktorom vypúšťame plyn z nádoby kde je pretlak $h$. A po vypustení a ustálení teplôt $h^\prime$.
Pre výpočet $\kappa$ môžeme odvodiť vzorec
\eq{
\kappa = \frac{h}{h-h^\prime}\,. 
}

\subsubsection{Metoda kmitajícího pístku}
Pre hodnotu $\kappa$ môžeme odvodiť vzťah na závislosť do doby kmitu\eq{
\kappa = \frac{4mV}{T^2 p r^4}\,,\lbl{R_8}
}
kde
\eq{
p=b+\frac{mg}{\pi r^2}\,,
}
,pričom $b$ je atmosferický tlak, hmotnosť piestu je $m="4.59\cdot10^{-3} kg"$, objem banky je $V="1.133 l"$ a priemer piestu je $2r = "11.9\cdot10^{-3} m"$.


%\subsubsection{Spracovanie chýb merania}




\subsection{Výsledky merania}
\subsubsection{Metoda kmitajícího pístku}
V tab. \ref{T_1} sú zaznamenané počty kmitov za čas $t = "300 s"$, pre jednotlivé merania.
\begin{table}[h]

\begin{center}
\begin{tabular}{| c |}
\hline
 \popi{N}{1} \\
\hline
877\\
879\\
872\\
876\\
874\\
875\\
877\\
881\\
882\\
884\\
887\\
888\\
888\\
889\\
891\\
892\\
\hline


\end{tabular}
\caption{Namerané počty kmitov za čas $t="300 s"$} \label{T_1}
\end{center}
\end{table}
Z hodnôt v tab \ref{T_1} bol vypočítaná priemerná hodnota početu kmitov$\mean{N}="882\pm5"$.
Priemerná hodnota bola dosadená do vzťahu \ref{R_8} a bola vypočítaná Poissonova konstanta $\kappa = "1.68\pm0.01"$.

\subsubsection{ Clémentova-Désormesova metoda}

Touto metódou boli namerané 2 \uv{vzorky dát}. Prvá v pre otvárací čas pod $"200 ms"$ a druhá nad tento čas.
Dáta boli vynesené do grafu obr. \ref{G_1} a každé zvlášť preložené lineárnou funkciou. 
Následne bola vypočítaná extrapoláciou dat hodnota $\kappa_{\(0\)}$. 



\begin{figure}
% GNUPLOT: LaTeX picture
\setlength{\unitlength}{0.240900pt}
\ifx\plotpoint\undefined\newsavebox{\plotpoint}\fi
\begin{picture}(1500,900)(0,0)
\sbox{\plotpoint}{\rule[-0.200pt]{0.400pt}{0.400pt}}%
\put(191.0,131.0){\rule[-0.200pt]{4.818pt}{0.400pt}}
\put(171,131){\makebox(0,0)[r]{ 1.15}}
\put(1419.0,131.0){\rule[-0.200pt]{4.818pt}{0.400pt}}
\put(191.0,252.0){\rule[-0.200pt]{4.818pt}{0.400pt}}
\put(171,252){\makebox(0,0)[r]{ 1.2}}
\put(1419.0,252.0){\rule[-0.200pt]{4.818pt}{0.400pt}}
\put(191.0,374.0){\rule[-0.200pt]{4.818pt}{0.400pt}}
\put(171,374){\makebox(0,0)[r]{ 1.25}}
\put(1419.0,374.0){\rule[-0.200pt]{4.818pt}{0.400pt}}
\put(191.0,495.0){\rule[-0.200pt]{4.818pt}{0.400pt}}
\put(171,495){\makebox(0,0)[r]{ 1.3}}
\put(1419.0,495.0){\rule[-0.200pt]{4.818pt}{0.400pt}}
\put(191.0,616.0){\rule[-0.200pt]{4.818pt}{0.400pt}}
\put(171,616){\makebox(0,0)[r]{ 1.35}}
\put(1419.0,616.0){\rule[-0.200pt]{4.818pt}{0.400pt}}
\put(191.0,738.0){\rule[-0.200pt]{4.818pt}{0.400pt}}
\put(171,738){\makebox(0,0)[r]{ 1.4}}
\put(1419.0,738.0){\rule[-0.200pt]{4.818pt}{0.400pt}}
\put(191.0,859.0){\rule[-0.200pt]{4.818pt}{0.400pt}}
\put(171,859){\makebox(0,0)[r]{ 1.45}}
\put(1419.0,859.0){\rule[-0.200pt]{4.818pt}{0.400pt}}
\put(191.0,131.0){\rule[-0.200pt]{0.400pt}{4.818pt}}
\put(191,90){\makebox(0,0){ 0}}
\put(191.0,839.0){\rule[-0.200pt]{0.400pt}{4.818pt}}
\put(316.0,131.0){\rule[-0.200pt]{0.400pt}{4.818pt}}
\put(316,90){\makebox(0,0){ 0.05}}
\put(316.0,839.0){\rule[-0.200pt]{0.400pt}{4.818pt}}
\put(441.0,131.0){\rule[-0.200pt]{0.400pt}{4.818pt}}
\put(441,90){\makebox(0,0){ 0.1}}
\put(441.0,839.0){\rule[-0.200pt]{0.400pt}{4.818pt}}
\put(565.0,131.0){\rule[-0.200pt]{0.400pt}{4.818pt}}
\put(565,90){\makebox(0,0){ 0.15}}
\put(565.0,839.0){\rule[-0.200pt]{0.400pt}{4.818pt}}
\put(690.0,131.0){\rule[-0.200pt]{0.400pt}{4.818pt}}
\put(690,90){\makebox(0,0){ 0.2}}
\put(690.0,839.0){\rule[-0.200pt]{0.400pt}{4.818pt}}
\put(815.0,131.0){\rule[-0.200pt]{0.400pt}{4.818pt}}
\put(815,90){\makebox(0,0){ 0.25}}
\put(815.0,839.0){\rule[-0.200pt]{0.400pt}{4.818pt}}
\put(940.0,131.0){\rule[-0.200pt]{0.400pt}{4.818pt}}
\put(940,90){\makebox(0,0){ 0.3}}
\put(940.0,839.0){\rule[-0.200pt]{0.400pt}{4.818pt}}
\put(1065.0,131.0){\rule[-0.200pt]{0.400pt}{4.818pt}}
\put(1065,90){\makebox(0,0){ 0.35}}
\put(1065.0,839.0){\rule[-0.200pt]{0.400pt}{4.818pt}}
\put(1189.0,131.0){\rule[-0.200pt]{0.400pt}{4.818pt}}
\put(1189,90){\makebox(0,0){ 0.4}}
\put(1189.0,839.0){\rule[-0.200pt]{0.400pt}{4.818pt}}
\put(1314.0,131.0){\rule[-0.200pt]{0.400pt}{4.818pt}}
\put(1314,90){\makebox(0,0){ 0.45}}
\put(1314.0,839.0){\rule[-0.200pt]{0.400pt}{4.818pt}}
\put(1439.0,131.0){\rule[-0.200pt]{0.400pt}{4.818pt}}
\put(1439,90){\makebox(0,0){ 0.5}}
\put(1439.0,839.0){\rule[-0.200pt]{0.400pt}{4.818pt}}
\put(191.0,131.0){\rule[-0.200pt]{0.400pt}{175.375pt}}
\put(191.0,131.0){\rule[-0.200pt]{300.643pt}{0.400pt}}
\put(1439.0,131.0){\rule[-0.200pt]{0.400pt}{175.375pt}}
\put(191.0,859.0){\rule[-0.200pt]{300.643pt}{0.400pt}}
\put(30,495){\makebox(0,0){\popi{\kappa}{-}}}
\put(815,29){\makebox(0,0){\popi{t}{s}}}
\put(1279,819){\makebox(0,0)[r]{linenárny fit pre $t<"200 ms"$}}
\put(1299.0,819.0){\rule[-0.200pt]{24.090pt}{0.400pt}}
\put(308,603){\usebox{\plotpoint}}
\multiput(308.00,601.94)(1.505,-0.468){5}{\rule{1.200pt}{0.113pt}}
\multiput(308.00,602.17)(8.509,-4.000){2}{\rule{0.600pt}{0.400pt}}
\multiput(319.00,597.95)(2.248,-0.447){3}{\rule{1.567pt}{0.108pt}}
\multiput(319.00,598.17)(7.748,-3.000){2}{\rule{0.783pt}{0.400pt}}
\multiput(330.00,594.94)(1.505,-0.468){5}{\rule{1.200pt}{0.113pt}}
\multiput(330.00,595.17)(8.509,-4.000){2}{\rule{0.600pt}{0.400pt}}
\multiput(341.00,590.95)(2.248,-0.447){3}{\rule{1.567pt}{0.108pt}}
\multiput(341.00,591.17)(7.748,-3.000){2}{\rule{0.783pt}{0.400pt}}
\multiput(352.00,587.94)(1.505,-0.468){5}{\rule{1.200pt}{0.113pt}}
\multiput(352.00,588.17)(8.509,-4.000){2}{\rule{0.600pt}{0.400pt}}
\multiput(363.00,583.94)(1.505,-0.468){5}{\rule{1.200pt}{0.113pt}}
\multiput(363.00,584.17)(8.509,-4.000){2}{\rule{0.600pt}{0.400pt}}
\multiput(374.00,579.95)(2.248,-0.447){3}{\rule{1.567pt}{0.108pt}}
\multiput(374.00,580.17)(7.748,-3.000){2}{\rule{0.783pt}{0.400pt}}
\multiput(385.00,576.94)(1.505,-0.468){5}{\rule{1.200pt}{0.113pt}}
\multiput(385.00,577.17)(8.509,-4.000){2}{\rule{0.600pt}{0.400pt}}
\multiput(396.00,572.94)(1.505,-0.468){5}{\rule{1.200pt}{0.113pt}}
\multiput(396.00,573.17)(8.509,-4.000){2}{\rule{0.600pt}{0.400pt}}
\multiput(407.00,568.95)(2.248,-0.447){3}{\rule{1.567pt}{0.108pt}}
\multiput(407.00,569.17)(7.748,-3.000){2}{\rule{0.783pt}{0.400pt}}
\multiput(418.00,565.94)(1.505,-0.468){5}{\rule{1.200pt}{0.113pt}}
\multiput(418.00,566.17)(8.509,-4.000){2}{\rule{0.600pt}{0.400pt}}
\multiput(429.00,561.94)(1.505,-0.468){5}{\rule{1.200pt}{0.113pt}}
\multiput(429.00,562.17)(8.509,-4.000){2}{\rule{0.600pt}{0.400pt}}
\multiput(440.00,557.95)(2.248,-0.447){3}{\rule{1.567pt}{0.108pt}}
\multiput(440.00,558.17)(7.748,-3.000){2}{\rule{0.783pt}{0.400pt}}
\multiput(451.00,554.94)(1.505,-0.468){5}{\rule{1.200pt}{0.113pt}}
\multiput(451.00,555.17)(8.509,-4.000){2}{\rule{0.600pt}{0.400pt}}
\multiput(462.00,550.94)(1.505,-0.468){5}{\rule{1.200pt}{0.113pt}}
\multiput(462.00,551.17)(8.509,-4.000){2}{\rule{0.600pt}{0.400pt}}
\multiput(473.00,546.95)(2.248,-0.447){3}{\rule{1.567pt}{0.108pt}}
\multiput(473.00,547.17)(7.748,-3.000){2}{\rule{0.783pt}{0.400pt}}
\multiput(484.00,543.94)(1.505,-0.468){5}{\rule{1.200pt}{0.113pt}}
\multiput(484.00,544.17)(8.509,-4.000){2}{\rule{0.600pt}{0.400pt}}
\multiput(495.00,539.94)(1.505,-0.468){5}{\rule{1.200pt}{0.113pt}}
\multiput(495.00,540.17)(8.509,-4.000){2}{\rule{0.600pt}{0.400pt}}
\multiput(506.00,535.95)(2.248,-0.447){3}{\rule{1.567pt}{0.108pt}}
\multiput(506.00,536.17)(7.748,-3.000){2}{\rule{0.783pt}{0.400pt}}
\multiput(517.00,532.94)(1.505,-0.468){5}{\rule{1.200pt}{0.113pt}}
\multiput(517.00,533.17)(8.509,-4.000){2}{\rule{0.600pt}{0.400pt}}
\multiput(528.00,528.94)(1.505,-0.468){5}{\rule{1.200pt}{0.113pt}}
\multiput(528.00,529.17)(8.509,-4.000){2}{\rule{0.600pt}{0.400pt}}
\multiput(539.00,524.95)(2.248,-0.447){3}{\rule{1.567pt}{0.108pt}}
\multiput(539.00,525.17)(7.748,-3.000){2}{\rule{0.783pt}{0.400pt}}
\multiput(550.00,521.94)(1.505,-0.468){5}{\rule{1.200pt}{0.113pt}}
\multiput(550.00,522.17)(8.509,-4.000){2}{\rule{0.600pt}{0.400pt}}
\multiput(561.00,517.95)(2.248,-0.447){3}{\rule{1.567pt}{0.108pt}}
\multiput(561.00,518.17)(7.748,-3.000){2}{\rule{0.783pt}{0.400pt}}
\multiput(572.00,514.94)(1.358,-0.468){5}{\rule{1.100pt}{0.113pt}}
\multiput(572.00,515.17)(7.717,-4.000){2}{\rule{0.550pt}{0.400pt}}
\multiput(582.00,510.94)(1.505,-0.468){5}{\rule{1.200pt}{0.113pt}}
\multiput(582.00,511.17)(8.509,-4.000){2}{\rule{0.600pt}{0.400pt}}
\multiput(593.00,506.95)(2.248,-0.447){3}{\rule{1.567pt}{0.108pt}}
\multiput(593.00,507.17)(7.748,-3.000){2}{\rule{0.783pt}{0.400pt}}
\multiput(604.00,503.94)(1.505,-0.468){5}{\rule{1.200pt}{0.113pt}}
\multiput(604.00,504.17)(8.509,-4.000){2}{\rule{0.600pt}{0.400pt}}
\multiput(615.00,499.94)(1.505,-0.468){5}{\rule{1.200pt}{0.113pt}}
\multiput(615.00,500.17)(8.509,-4.000){2}{\rule{0.600pt}{0.400pt}}
\multiput(626.00,495.95)(2.248,-0.447){3}{\rule{1.567pt}{0.108pt}}
\multiput(626.00,496.17)(7.748,-3.000){2}{\rule{0.783pt}{0.400pt}}
\multiput(637.00,492.94)(1.505,-0.468){5}{\rule{1.200pt}{0.113pt}}
\multiput(637.00,493.17)(8.509,-4.000){2}{\rule{0.600pt}{0.400pt}}
\multiput(648.00,488.94)(1.505,-0.468){5}{\rule{1.200pt}{0.113pt}}
\multiput(648.00,489.17)(8.509,-4.000){2}{\rule{0.600pt}{0.400pt}}
\multiput(659.00,484.95)(2.248,-0.447){3}{\rule{1.567pt}{0.108pt}}
\multiput(659.00,485.17)(7.748,-3.000){2}{\rule{0.783pt}{0.400pt}}
\multiput(670.00,481.94)(1.505,-0.468){5}{\rule{1.200pt}{0.113pt}}
\multiput(670.00,482.17)(8.509,-4.000){2}{\rule{0.600pt}{0.400pt}}
\multiput(681.00,477.94)(1.505,-0.468){5}{\rule{1.200pt}{0.113pt}}
\multiput(681.00,478.17)(8.509,-4.000){2}{\rule{0.600pt}{0.400pt}}
\multiput(692.00,473.95)(2.248,-0.447){3}{\rule{1.567pt}{0.108pt}}
\multiput(692.00,474.17)(7.748,-3.000){2}{\rule{0.783pt}{0.400pt}}
\multiput(703.00,470.94)(1.505,-0.468){5}{\rule{1.200pt}{0.113pt}}
\multiput(703.00,471.17)(8.509,-4.000){2}{\rule{0.600pt}{0.400pt}}
\multiput(714.00,466.94)(1.505,-0.468){5}{\rule{1.200pt}{0.113pt}}
\multiput(714.00,467.17)(8.509,-4.000){2}{\rule{0.600pt}{0.400pt}}
\multiput(725.00,462.95)(2.248,-0.447){3}{\rule{1.567pt}{0.108pt}}
\multiput(725.00,463.17)(7.748,-3.000){2}{\rule{0.783pt}{0.400pt}}
\multiput(736.00,459.94)(1.505,-0.468){5}{\rule{1.200pt}{0.113pt}}
\multiput(736.00,460.17)(8.509,-4.000){2}{\rule{0.600pt}{0.400pt}}
\multiput(747.00,455.95)(2.248,-0.447){3}{\rule{1.567pt}{0.108pt}}
\multiput(747.00,456.17)(7.748,-3.000){2}{\rule{0.783pt}{0.400pt}}
\multiput(758.00,452.94)(1.505,-0.468){5}{\rule{1.200pt}{0.113pt}}
\multiput(758.00,453.17)(8.509,-4.000){2}{\rule{0.600pt}{0.400pt}}
\multiput(769.00,448.94)(1.505,-0.468){5}{\rule{1.200pt}{0.113pt}}
\multiput(769.00,449.17)(8.509,-4.000){2}{\rule{0.600pt}{0.400pt}}
\multiput(780.00,444.95)(2.248,-0.447){3}{\rule{1.567pt}{0.108pt}}
\multiput(780.00,445.17)(7.748,-3.000){2}{\rule{0.783pt}{0.400pt}}
\multiput(791.00,441.94)(1.505,-0.468){5}{\rule{1.200pt}{0.113pt}}
\multiput(791.00,442.17)(8.509,-4.000){2}{\rule{0.600pt}{0.400pt}}
\multiput(802.00,437.94)(1.505,-0.468){5}{\rule{1.200pt}{0.113pt}}
\multiput(802.00,438.17)(8.509,-4.000){2}{\rule{0.600pt}{0.400pt}}
\multiput(813.00,433.95)(2.248,-0.447){3}{\rule{1.567pt}{0.108pt}}
\multiput(813.00,434.17)(7.748,-3.000){2}{\rule{0.783pt}{0.400pt}}
\multiput(824.00,430.94)(1.505,-0.468){5}{\rule{1.200pt}{0.113pt}}
\multiput(824.00,431.17)(8.509,-4.000){2}{\rule{0.600pt}{0.400pt}}
\multiput(835.00,426.94)(1.505,-0.468){5}{\rule{1.200pt}{0.113pt}}
\multiput(835.00,427.17)(8.509,-4.000){2}{\rule{0.600pt}{0.400pt}}
\multiput(846.00,422.95)(2.248,-0.447){3}{\rule{1.567pt}{0.108pt}}
\multiput(846.00,423.17)(7.748,-3.000){2}{\rule{0.783pt}{0.400pt}}
\multiput(857.00,419.94)(1.505,-0.468){5}{\rule{1.200pt}{0.113pt}}
\multiput(857.00,420.17)(8.509,-4.000){2}{\rule{0.600pt}{0.400pt}}
\multiput(868.00,415.94)(1.505,-0.468){5}{\rule{1.200pt}{0.113pt}}
\multiput(868.00,416.17)(8.509,-4.000){2}{\rule{0.600pt}{0.400pt}}
\multiput(879.00,411.95)(2.248,-0.447){3}{\rule{1.567pt}{0.108pt}}
\multiput(879.00,412.17)(7.748,-3.000){2}{\rule{0.783pt}{0.400pt}}
\multiput(890.00,408.94)(1.505,-0.468){5}{\rule{1.200pt}{0.113pt}}
\multiput(890.00,409.17)(8.509,-4.000){2}{\rule{0.600pt}{0.400pt}}
\multiput(901.00,404.94)(1.505,-0.468){5}{\rule{1.200pt}{0.113pt}}
\multiput(901.00,405.17)(8.509,-4.000){2}{\rule{0.600pt}{0.400pt}}
\multiput(912.00,400.95)(2.025,-0.447){3}{\rule{1.433pt}{0.108pt}}
\multiput(912.00,401.17)(7.025,-3.000){2}{\rule{0.717pt}{0.400pt}}
\multiput(922.00,397.94)(1.505,-0.468){5}{\rule{1.200pt}{0.113pt}}
\multiput(922.00,398.17)(8.509,-4.000){2}{\rule{0.600pt}{0.400pt}}
\multiput(933.00,393.95)(2.248,-0.447){3}{\rule{1.567pt}{0.108pt}}
\multiput(933.00,394.17)(7.748,-3.000){2}{\rule{0.783pt}{0.400pt}}
\multiput(944.00,390.94)(1.505,-0.468){5}{\rule{1.200pt}{0.113pt}}
\multiput(944.00,391.17)(8.509,-4.000){2}{\rule{0.600pt}{0.400pt}}
\multiput(955.00,386.94)(1.505,-0.468){5}{\rule{1.200pt}{0.113pt}}
\multiput(955.00,387.17)(8.509,-4.000){2}{\rule{0.600pt}{0.400pt}}
\multiput(966.00,382.95)(2.248,-0.447){3}{\rule{1.567pt}{0.108pt}}
\multiput(966.00,383.17)(7.748,-3.000){2}{\rule{0.783pt}{0.400pt}}
\multiput(977.00,379.94)(1.505,-0.468){5}{\rule{1.200pt}{0.113pt}}
\multiput(977.00,380.17)(8.509,-4.000){2}{\rule{0.600pt}{0.400pt}}
\multiput(988.00,375.94)(1.505,-0.468){5}{\rule{1.200pt}{0.113pt}}
\multiput(988.00,376.17)(8.509,-4.000){2}{\rule{0.600pt}{0.400pt}}
\multiput(999.00,371.95)(2.248,-0.447){3}{\rule{1.567pt}{0.108pt}}
\multiput(999.00,372.17)(7.748,-3.000){2}{\rule{0.783pt}{0.400pt}}
\multiput(1010.00,368.94)(1.505,-0.468){5}{\rule{1.200pt}{0.113pt}}
\multiput(1010.00,369.17)(8.509,-4.000){2}{\rule{0.600pt}{0.400pt}}
\multiput(1021.00,364.94)(1.505,-0.468){5}{\rule{1.200pt}{0.113pt}}
\multiput(1021.00,365.17)(8.509,-4.000){2}{\rule{0.600pt}{0.400pt}}
\multiput(1032.00,360.95)(2.248,-0.447){3}{\rule{1.567pt}{0.108pt}}
\multiput(1032.00,361.17)(7.748,-3.000){2}{\rule{0.783pt}{0.400pt}}
\multiput(1043.00,357.94)(1.505,-0.468){5}{\rule{1.200pt}{0.113pt}}
\multiput(1043.00,358.17)(8.509,-4.000){2}{\rule{0.600pt}{0.400pt}}
\multiput(1054.00,353.94)(1.505,-0.468){5}{\rule{1.200pt}{0.113pt}}
\multiput(1054.00,354.17)(8.509,-4.000){2}{\rule{0.600pt}{0.400pt}}
\multiput(1065.00,349.95)(2.248,-0.447){3}{\rule{1.567pt}{0.108pt}}
\multiput(1065.00,350.17)(7.748,-3.000){2}{\rule{0.783pt}{0.400pt}}
\multiput(1076.00,346.94)(1.505,-0.468){5}{\rule{1.200pt}{0.113pt}}
\multiput(1076.00,347.17)(8.509,-4.000){2}{\rule{0.600pt}{0.400pt}}
\multiput(1087.00,342.94)(1.505,-0.468){5}{\rule{1.200pt}{0.113pt}}
\multiput(1087.00,343.17)(8.509,-4.000){2}{\rule{0.600pt}{0.400pt}}
\multiput(1098.00,338.95)(2.248,-0.447){3}{\rule{1.567pt}{0.108pt}}
\multiput(1098.00,339.17)(7.748,-3.000){2}{\rule{0.783pt}{0.400pt}}
\multiput(1109.00,335.94)(1.505,-0.468){5}{\rule{1.200pt}{0.113pt}}
\multiput(1109.00,336.17)(8.509,-4.000){2}{\rule{0.600pt}{0.400pt}}
\multiput(1120.00,331.95)(2.248,-0.447){3}{\rule{1.567pt}{0.108pt}}
\multiput(1120.00,332.17)(7.748,-3.000){2}{\rule{0.783pt}{0.400pt}}
\multiput(1131.00,328.94)(1.505,-0.468){5}{\rule{1.200pt}{0.113pt}}
\multiput(1131.00,329.17)(8.509,-4.000){2}{\rule{0.600pt}{0.400pt}}
\multiput(1142.00,324.94)(1.505,-0.468){5}{\rule{1.200pt}{0.113pt}}
\multiput(1142.00,325.17)(8.509,-4.000){2}{\rule{0.600pt}{0.400pt}}
\multiput(1153.00,320.95)(2.248,-0.447){3}{\rule{1.567pt}{0.108pt}}
\multiput(1153.00,321.17)(7.748,-3.000){2}{\rule{0.783pt}{0.400pt}}
\multiput(1164.00,317.94)(1.505,-0.468){5}{\rule{1.200pt}{0.113pt}}
\multiput(1164.00,318.17)(8.509,-4.000){2}{\rule{0.600pt}{0.400pt}}
\multiput(1175.00,313.94)(1.505,-0.468){5}{\rule{1.200pt}{0.113pt}}
\multiput(1175.00,314.17)(8.509,-4.000){2}{\rule{0.600pt}{0.400pt}}
\multiput(1186.00,309.95)(2.248,-0.447){3}{\rule{1.567pt}{0.108pt}}
\multiput(1186.00,310.17)(7.748,-3.000){2}{\rule{0.783pt}{0.400pt}}
\multiput(1197.00,306.94)(1.505,-0.468){5}{\rule{1.200pt}{0.113pt}}
\multiput(1197.00,307.17)(8.509,-4.000){2}{\rule{0.600pt}{0.400pt}}
\multiput(1208.00,302.94)(1.505,-0.468){5}{\rule{1.200pt}{0.113pt}}
\multiput(1208.00,303.17)(8.509,-4.000){2}{\rule{0.600pt}{0.400pt}}
\multiput(1219.00,298.95)(2.248,-0.447){3}{\rule{1.567pt}{0.108pt}}
\multiput(1219.00,299.17)(7.748,-3.000){2}{\rule{0.783pt}{0.400pt}}
\multiput(1230.00,295.94)(1.505,-0.468){5}{\rule{1.200pt}{0.113pt}}
\multiput(1230.00,296.17)(8.509,-4.000){2}{\rule{0.600pt}{0.400pt}}
\multiput(1241.00,291.94)(1.358,-0.468){5}{\rule{1.100pt}{0.113pt}}
\multiput(1241.00,292.17)(7.717,-4.000){2}{\rule{0.550pt}{0.400pt}}
\multiput(1251.00,287.95)(2.248,-0.447){3}{\rule{1.567pt}{0.108pt}}
\multiput(1251.00,288.17)(7.748,-3.000){2}{\rule{0.783pt}{0.400pt}}
\multiput(1262.00,284.94)(1.505,-0.468){5}{\rule{1.200pt}{0.113pt}}
\multiput(1262.00,285.17)(8.509,-4.000){2}{\rule{0.600pt}{0.400pt}}
\multiput(1273.00,280.94)(1.505,-0.468){5}{\rule{1.200pt}{0.113pt}}
\multiput(1273.00,281.17)(8.509,-4.000){2}{\rule{0.600pt}{0.400pt}}
\multiput(1284.00,276.95)(2.248,-0.447){3}{\rule{1.567pt}{0.108pt}}
\multiput(1284.00,277.17)(7.748,-3.000){2}{\rule{0.783pt}{0.400pt}}
\multiput(1295.00,273.94)(1.505,-0.468){5}{\rule{1.200pt}{0.113pt}}
\multiput(1295.00,274.17)(8.509,-4.000){2}{\rule{0.600pt}{0.400pt}}
\multiput(1306.00,269.94)(1.505,-0.468){5}{\rule{1.200pt}{0.113pt}}
\multiput(1306.00,270.17)(8.509,-4.000){2}{\rule{0.600pt}{0.400pt}}
\multiput(1317.00,265.95)(2.248,-0.447){3}{\rule{1.567pt}{0.108pt}}
\multiput(1317.00,266.17)(7.748,-3.000){2}{\rule{0.783pt}{0.400pt}}
\multiput(1328.00,262.94)(1.505,-0.468){5}{\rule{1.200pt}{0.113pt}}
\multiput(1328.00,263.17)(8.509,-4.000){2}{\rule{0.600pt}{0.400pt}}
\multiput(1339.00,258.95)(2.248,-0.447){3}{\rule{1.567pt}{0.108pt}}
\multiput(1339.00,259.17)(7.748,-3.000){2}{\rule{0.783pt}{0.400pt}}
\multiput(1350.00,255.94)(1.505,-0.468){5}{\rule{1.200pt}{0.113pt}}
\multiput(1350.00,256.17)(8.509,-4.000){2}{\rule{0.600pt}{0.400pt}}
\multiput(1361.00,251.94)(1.505,-0.468){5}{\rule{1.200pt}{0.113pt}}
\multiput(1361.00,252.17)(8.509,-4.000){2}{\rule{0.600pt}{0.400pt}}
\multiput(1372.00,247.95)(2.248,-0.447){3}{\rule{1.567pt}{0.108pt}}
\multiput(1372.00,248.17)(7.748,-3.000){2}{\rule{0.783pt}{0.400pt}}
\multiput(1383.00,244.94)(1.505,-0.468){5}{\rule{1.200pt}{0.113pt}}
\multiput(1383.00,245.17)(8.509,-4.000){2}{\rule{0.600pt}{0.400pt}}
\put(1279,778){\makebox(0,0)[r]{linenárny fit pre $t>"200 ms"$}}
\multiput(1299,778)(20.756,0.000){5}{\usebox{\plotpoint}}
\put(1399,778){\usebox{\plotpoint}}
\put(308,468){\usebox{\plotpoint}}
\put(308.00,468.00){\usebox{\plotpoint}}
\put(328.76,468.00){\usebox{\plotpoint}}
\put(349.51,468.00){\usebox{\plotpoint}}
\put(370.27,468.00){\usebox{\plotpoint}}
\put(391.02,468.00){\usebox{\plotpoint}}
\put(411.78,468.00){\usebox{\plotpoint}}
\put(432.49,467.00){\usebox{\plotpoint}}
\put(453.24,467.00){\usebox{\plotpoint}}
\put(474.00,467.00){\usebox{\plotpoint}}
\put(494.75,467.00){\usebox{\plotpoint}}
\put(515.51,467.00){\usebox{\plotpoint}}
\put(536.27,467.00){\usebox{\plotpoint}}
\put(556.98,466.00){\usebox{\plotpoint}}
\put(577.73,466.00){\usebox{\plotpoint}}
\put(598.49,466.00){\usebox{\plotpoint}}
\put(619.24,466.00){\usebox{\plotpoint}}
\put(640.00,466.00){\usebox{\plotpoint}}
\put(660.75,465.84){\usebox{\plotpoint}}
\put(681.46,465.00){\usebox{\plotpoint}}
\put(702.22,465.00){\usebox{\plotpoint}}
\put(722.97,465.00){\usebox{\plotpoint}}
\put(743.73,465.00){\usebox{\plotpoint}}
\put(764.48,465.00){\usebox{\plotpoint}}
\put(785.22,464.53){\usebox{\plotpoint}}
\put(805.95,464.00){\usebox{\plotpoint}}
\put(826.71,464.00){\usebox{\plotpoint}}
\put(847.46,464.00){\usebox{\plotpoint}}
\put(868.22,464.00){\usebox{\plotpoint}}
\put(888.97,464.00){\usebox{\plotpoint}}
\put(909.69,463.21){\usebox{\plotpoint}}
\put(930.44,463.00){\usebox{\plotpoint}}
\put(951.19,463.00){\usebox{\plotpoint}}
\put(971.95,463.00){\usebox{\plotpoint}}
\put(992.70,463.00){\usebox{\plotpoint}}
\put(1013.46,463.00){\usebox{\plotpoint}}
\put(1034.21,462.80){\usebox{\plotpoint}}
\put(1054.93,462.00){\usebox{\plotpoint}}
\put(1075.68,462.00){\usebox{\plotpoint}}
\put(1096.44,462.00){\usebox{\plotpoint}}
\put(1117.19,462.00){\usebox{\plotpoint}}
\put(1137.95,462.00){\usebox{\plotpoint}}
\put(1158.68,461.48){\usebox{\plotpoint}}
\put(1179.41,461.00){\usebox{\plotpoint}}
\put(1200.17,461.00){\usebox{\plotpoint}}
\put(1220.92,461.00){\usebox{\plotpoint}}
\put(1241.68,461.00){\usebox{\plotpoint}}
\put(1262.44,461.00){\usebox{\plotpoint}}
\put(1283.15,460.08){\usebox{\plotpoint}}
\put(1303.90,460.00){\usebox{\plotpoint}}
\put(1324.66,460.00){\usebox{\plotpoint}}
\put(1345.41,460.00){\usebox{\plotpoint}}
\put(1366.17,460.00){\usebox{\plotpoint}}
\put(1386.92,460.00){\usebox{\plotpoint}}
\put(1394,460){\usebox{\plotpoint}}
\sbox{\plotpoint}{\rule[-0.400pt]{0.800pt}{0.800pt}}%
\sbox{\plotpoint}{\rule[-0.200pt]{0.400pt}{0.400pt}}%
\put(1279,737){\makebox(0,0)[r]{namerané hodnoty pre $t<"200 ms"$}}
\sbox{\plotpoint}{\rule[-0.400pt]{0.800pt}{0.800pt}}%
\put(441,547){\makebox(0,0){$\ast$}}
\put(655,484){\makebox(0,0){$\ast$}}
\put(421,449){\makebox(0,0){$\ast$}}
\put(655,559){\makebox(0,0){$\ast$}}
\put(491,374){\makebox(0,0){$\ast$}}
\put(608,595){\makebox(0,0){$\ast$}}
\put(311,826){\makebox(0,0){$\ast$}}
\put(308,601){\makebox(0,0){$\ast$}}
\put(326,525){\makebox(0,0){$\ast$}}
\put(700,460){\makebox(0,0){$\ast$}}
\put(1349,737){\makebox(0,0){$\ast$}}
\sbox{\plotpoint}{\rule[-0.500pt]{1.000pt}{1.000pt}}%
\sbox{\plotpoint}{\rule[-0.200pt]{0.400pt}{0.400pt}}%
\put(1279,696){\makebox(0,0)[r]{namerané hodnoty pre $t>"200 ms"$}}
\sbox{\plotpoint}{\rule[-0.500pt]{1.000pt}{1.000pt}}%
\put(1132,441){\raisebox{-.8pt}{\makebox(0,0){$\Box$}}}
\put(910,447){\raisebox{-.8pt}{\makebox(0,0){$\Box$}}}
\put(1122,473){\raisebox{-.8pt}{\makebox(0,0){$\Box$}}}
\put(760,521){\raisebox{-.8pt}{\makebox(0,0){$\Box$}}}
\put(1394,461){\raisebox{-.8pt}{\makebox(0,0){$\Box$}}}
\put(972,473){\raisebox{-.8pt}{\makebox(0,0){$\Box$}}}
\put(950,430){\raisebox{-.8pt}{\makebox(0,0){$\Box$}}}
\put(867,386){\raisebox{-.8pt}{\makebox(0,0){$\Box$}}}
\put(882,474){\raisebox{-.8pt}{\makebox(0,0){$\Box$}}}
\put(1032,522){\raisebox{-.8pt}{\makebox(0,0){$\Box$}}}
\put(1349,696){\raisebox{-.8pt}{\makebox(0,0){$\Box$}}}
\sbox{\plotpoint}{\rule[-0.200pt]{0.400pt}{0.400pt}}%
\put(191.0,131.0){\rule[-0.200pt]{0.400pt}{175.375pt}}
\put(191.0,131.0){\rule[-0.200pt]{300.643pt}{0.400pt}}
\put(1439.0,131.0){\rule[-0.200pt]{0.400pt}{175.375pt}}
\put(191.0,859.0){\rule[-0.200pt]{300.643pt}{0.400pt}}
\end{picture}

\caption{extrapolácia nameraných hodnôt pre $t="0"$}  \label{G_1}
\end{figure}

Pre dáta s otváracím časom pod $"200 ms"$ je hodnota $\kappa_{\(0\)}= "1.36\pm0.04"$ a 
pre hodnoty s otváracím časom nad $"200 ms"$ bola vypočítaná $\kappa_{\(0\)}= "1.29\pm0.03"$.

\subsection{Diskusia}
Pri metóde kmitajúceho piestu spôsobuje hlavný zdroj nepresností a 
systematických chýb netesnosť medzi piestom a aparatúrou. 
Ďalej aj pomerne mala dierka na vypúšťanie plynu. 
Teda expanzia nieje okamžitá a vyrovnanie tlakov úplné.
Zaujímavosťou je zvyšovanie počtu kmitov s pripadajúcim časom, 
mojou teóriou na vysvetlenie tohoto javu je na zahrievanie piestu trením a teda jeho zväčšenie a teda sa zlepšilo tesnenie a neunikalo toľko plynu, pokrajoch. 

Clémentova-Désormesova metoda sa však viac približuje očakávanému výsledku \textit{$\kappa = \sim "1.40"$ pre $N_2$ alebo $O_2$}\cite{C_1}. 
Hlavné nepresnosti pri tejto metóde spočívajú v nie dokonalým vyrovnaním teplôt po vypustení plynu. 
A nevhodná funkcia na extrapoláciu dát.

\subsection{Záver}
Clémentova-Désormesovou metódou bola $\kappa_{\(0\)}= "1.36\pm0.04"$, a metódou kmitajúceho piestu $\kappa = "1.68\pm0.01"$.




%%%%%%%%%%%%%%%%%%%%%%%%%%%%%%%%%%%%%%%%%%%%%%%%%%%%%%%%%%%%%%%%%%%%%%%%%%%%%%%%%%%%%%%%%%%%%%%%%%%%%%%%%%%%%%%%%
%%%%%%%%%%%%%%%%%%%%%%%%%%%%%%%%%%%%%%%%%%%%%%%%%%%%%%%%%%%%%%%%%%%%%%%%%%%%%%%%%%%%%%%%%%%%%%%%%%%%%%%%%%%%%%%%%
%%%%%%%%%%%%%%%%%%%%%%%%%%%%%%%%%%%%%%%%%%%%%%%%%%%%%%%%%%%%%%%%%%%%%%%%%%%%%%%%%%%%%%%%%%%%%%%%%%%%%%%%%%%%%%%%%

\section{Úkol \#2}
\subsection{Pracovní úkol}
\begin{enumerate}
\item Určete objem láhve metodou vážení.
\item Určete objem ťěze láhve pomocí komprese plynu.
\item Oba výsledky vzájemně porovnejte.
\end{enumerate}

\subsection{Postup merania}
\subsubsection{Metóda kompresie plynu}
\begin{enumerate}
\item povolením ventilu bol vyrovnaný tlak v byrete s atmosferickým tlakom.
\item Vertikálnym pohybom nádoby s vodou bola hladina v byrete ustálená na úroveň $"0 \%"$.
\item Ventil pol zatvorený.
\item následne sa pohlo nádobou s vodou nahor.
\item počkalo sa na ustálenie hladín a boli odčítané hodnoty $\Delta h$ ,$V_2$.
\item Postup sa opakoval niekoľkokrát pre veľkú nádobu.
\item Meraná nádoba bola vymenená za utesnenie a postup bol zopakovaný pre meranie objemu len hadičky.
\end{enumerate}


\subsubsection{Metóda vážení}
\begin{enumerate}
\item Nádoba bola odvážená na digitálnych váhach 
\item Nádoba bola pookraj naplnená vodou a dôkladne osušení jej povrch
\item Naplnená nádoba bola opäť odvážená
\item Bola odmeraná teplota vody.
\end{enumerate}

\subsection{Pomôcky}
Fľaška (nádoba), plynová byreta s porovnávacím ramenem, katetometr,
teploměr, barometr, digitálne váhy do $"5 kg"$.

\section{Teória}

\subsubsection{Metóda kompresie plynu}
Pre metódu kompresie plynu v našom prípade môžeme odvodiť vzťah
\eq{
V = \( V_2-V_1 \)\frac{p}{\Delta p} + V_2 - V_{100} \,, \lbl{R_6}
}, kde 
\eq{
\Delta p = \Delta h \rho g\,.
}
Pričom $V_1$ je objem v byrete pri vyrovnaní tlakov.$\Delta h$ je rozdiel hladín, a $V_2$ výška hladiny po kompresií.

V našom prípade $V_1="0 \%"$. Pričom $V_{100} = "65.6 cm^3"$

\subsubsection{Metóda vážení}
Jednotkový objem vody je závisí na teplote $t$ v $\C$ podľa vzťahu
\eq{
V_v = 0.9998\cdot(1+0.00018 t) \frac{\jd{cm^3}}{\jd{g}}\,.\lbl{R_5}
} 
Potom objem metódou váženia určíme ako\eq{
V = \( m_n- m_p\) V_v \,. \lbl{R_4}
}


\subsection{Výsledky merania}
\subsubsection{Metóda kompresie plynu}
V tab. \ref{T_2} sú zaznamenané namerané hodnoty $V_2$ a $\Delta h$
z ktorých bola vypočítaná hodnota $V$. 
V prvej časti tabuľky sú hodnoty pre fľašku v druhej časti je hodnota pre samotnú hadičku.


\begin{table}[h]

\begin{center}
\begin{tabular}{| c | c | c |}
\hline
 \popi{V_2}{\%} & \popi{\Delta h}{cm} & \popi{V}{cm^3} \\
\hline
$5$   &	$3.51$ & $896.08$\\
$9$   & $5.85$  & $975.39$\\
$6$   & $5.9$   & $619.31$\\
$4.5$ & $4,10$  & $676.69$\\
$8$   & $4,68$  & $1\,089.74$\\
$8$   & $6.83$  & $728.29$\\
$9$   & $7.41$  & $757.49$\\
$7$   & $5.27$  & $833.51$\\
$8$   & $5.46$  & $925.44$\\
$6.75$& $5.85$  & $715.14$\\
\hline
$1$ &	$10.34$ & $5.58$\\							
\hline

\end{tabular}
\caption{Namerané hodnoty $V_2$, $\Delta h$ a vypočítaný objem $V$, v prvej časti pre nádobu a v druhej pre hadičku.} \label{T_2}
\end{center}
\end{table}

Objem fľašky po odčítaný objemu hadičky bol určený ako $V = "\(816.7\pm148.3\) cm^3"$.


\subsubsection{Metóda vážení}
Hmotnosť prázdnej suchej nádoby bola určená $m_p = "\(582\pm1\) g"$, jednotková hmotnosť vody pri $t="11.8 \C"$ bola určená podľa \ref{R_5} ako $V_v="1.002 \frac{cm^3}{g}"$
Hmotnosť po naplnení vodou bola určená $m_n = "\(1598\pm1\) g"$.
Podľa vzorca \ref{R_4} bol objem nádoby určený ako $V="\(1018\pm2\) cm^3"$.

\subsection{Diskusia}
Vo výsledkoch vidíme veľmi veľký rozdiel nameraných hodnôt, 
nepresnosť merania pri metóde kompresie plynu spôsobovali extrémne 
veľké netesnosti, pri ktorých pretlak z aparatúry veľmi rýchlo unikal. 
Teda toto meranie bolo zaťažené veľkou systematickou chybou. 
Aj napriek snahe merať rýchlo sa štatistická chyba pohybuje na úrovni $"18 \%"$. 
Naopak metóda váženia sa ukazuje ako veľmi presná.

\subsection{Záver}
Metódou váženia bol objem nádoby určený na $V="\(1018\pm2\) cm^3"$.
Metódou kompresie plynu bol určený objem $V = "\(816.7\pm148.3\) cm^3"$.


\begin{thebibliography}{1}
\bibitem{C_1}
Článok dostupný na \url{https://cs.wikipedia.org/wiki/Poissonova_konstanta#Hodnoty_pro_re.C3.A1ln.C3.A9_plyny}
\end{thebibliography}

\end{document}






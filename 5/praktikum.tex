\documentclass[a4paper,10pt]{article}
%\usepackage[IL2]{fontenc}
\usepackage[utf8x]{inputenc}
\usepackage[czech]{babel}
\usepackage{amsfonts,amsmath,amssymb,graphicx,color}
%\usepackage[total={17cm,27cm}, top=2cm, left=2cm, includefoot]{geometry}
%\usepackage{fancyhdr}
\usepackage{fkssugar}
\usepackage{hyperref}

%\usepackage{caption}
\renewcommand{\popi}[2]{$#1$[\jd{#2}]}
\renewcommand{\figurename}{Obr.}
\addto\captionsczech{\renewcommand{\figurename}{Obr.}}
\addto\captionsczech{\renewcommand{\tablename}{Tab.}}

\begin{document}
\def\mean#1{\left< #1 \right>}
\noindent
{\large Fyzikální praktikum 1.} \hfil {\large FJFI ČVUT V Praze}\\
\noindent
{\large\textbf{pracovní úkol \# 5}}
\begin{center}
{\large\textit{Měření Poissonovy konstanty a dutých objemů}}
\end{center}
\noindent
\rule{\textwidth}{1px}
\vspace{\baselineskip}

\emph{Michal Červeňák}
\par
\vspace{\baselineskip}
\begin{minipage}[l]{0.5\textwidth}%
\textit{dátum merania:}~17.10. 2016\\%
%\vspace{\baselineskip}%
\par%
\noindent%
\textit{skupina:}~4\\%
%\vspace{\baselineskip}%
\par%
\noindent%
\textit{Klasifikace:}\dotfill\\%
\end{minipage}

\section{Úkol 1}
\subsection{Pracovní úkol}

\begin{enumerate}
\item Změřte Poissonovu konstantu metodou kmitajícího pístku.
\item Změřte Poissonovu konstantu Clément-Désormesovou metodou. Nezapomeňte provést
opravu vašeho měření na systematické chyby.
\item Oba výsledky vzájemně porovnejte (procentuálně) a diskutujte, jestli je v rámci chyb
můžete považovat za stejná.
\end{enumerate}


\subsection{Postup merania}
\subsubsection{Metoda kmitajícího pístku}
\begin{enumerate}
\item ventilom bol nastavený prúd vzduchu tak aby piest kmital medzi značkami
\item bol spustený digitálny čítač kmitov a nastavený na počítanie kmitov po $t="300 s"$.
\item po uplynutí intervalu boli dáta zaznamenané a opätovné spustenie počítanie. 
\end{enumerate}


\subsubsection{ Clémentova-Désormesova metoda}

\begin{enumerate}
\item Nádoba bola natlakovaná pomocou mechu, bol uzavretý prívodný ventil
\item tlak v nádobe bol odmeraný
\item pomocou ventilu bol tlak vyrovnaný s atmosferickým, pričom bol zaznamenaný čas otvorenia ventilu.
\item počkalo sa $\sim "1 min"$ na ustálenie teplôt v nádobe s okolím a následne bol zmeraný opäť tlak v aparatúre.
\end{enumerate}



\subsection{Pomôcky}
Barometr, aparatura na měěené Poissonovy konstanty Clément-Désormesovou metodou,
aparatura pro měření Poissonovy konstanty metodou kmitajícího pístku.

\subsection{Teória}
Poissonova konstanta $\kappa$ j pomer merného tepla $C_p$ pri stálom objeme a pri stálom objeme $C_V$, teda 
\eq{
\kappa = \frac{C_p}{C_V}\,.
}

\subsubsection{ Clémentova-Désormesova metoda}
Metóda určuje Poissonova konstanta z adiabatického deja, pri ktorom vypúšťame plyn z nádoby kde je pretlak $h$. A po vypustení a ustálení teplôt $h^\prime$.
Pre výpočet $\kappa$ môžeme odvodiť vzorec
\eq{
\kappa = \frac{h}{h-h^\prime}\,. 
}

\subsubsection{Metoda kmitajícího pístku}
Pre hodnotu $\kappa$ môžeme odvodiť vzťah na závislosť do doby kmitu\eq{
\kappa = \frac{4mV}{T^2 p r^4}\,,\lbl{R_8}
}
kde
\eq{
p=b+\frac{mg}{\pi r^2}\,,
}
,pričom $b$ je atmosferický tlak, hmotnosť piestu je $m="4.59\cdot10^{-3} kg"$, objem banky je $V="1.133 l"$ a priemer piestu je $2r = "11.9\cdot10^{-3} m"$.


%\subsubsection{Spracovanie chýb merania}




\subsection{Výsledky merania}
\subsubsection{Metoda kmitajícího pístku}
V tab. \ref{T_1} sú zaznamenané počty kmitov za čas $t = "300 s"$, pre jednotlivé merania.
\begin{table}[h]

\begin{center}
\begin{tabular}{| c |}
\hline
 \popi{N}{1} \\
\hline
877\\
879\\
872\\
876\\
874\\
875\\
877\\
881\\
882\\
884\\
887\\
888\\
888\\
889\\
891\\
892\\
\hline


\end{tabular}
\caption{Namerané počty kmitov za čas $t="300 s"$} \label{T_1}
\end{center}
\end{table}
Z hodnôt v tab \ref{T_1} bol vypočítaná priemerná hodnota početu kmitov$\mean{N}="882\pm5"$.
Priemerná hodnota bola dosadená do vzťahu \ref{R_8} a bola vypočítaná Poissonova konstanta $\kappa = "1.68\pm0.01"$.

\subsubsection{ Clémentova-Désormesova metoda}

Touto metódou boli namerané 2 \uv{vzorky dát}. Prvá v pre otvárací čas pod $"200 ms"$ a druhá nad tento čas.
Dáta boli vynesené do grafu obr. \ref{G_1} a každé zvlášť preložené lineárnou funkciou. 
Následne bola vypočítaná extrapoláciou dat hodnota $\kappa_{\(0\)}$. 



\begin{figure}
% GNUPLOT: LaTeX picture
\setlength{\unitlength}{0.240900pt}
\ifx\plotpoint\undefined\newsavebox{\plotpoint}\fi
\begin{picture}(1500,900)(0,0)
\sbox{\plotpoint}{\rule[-0.200pt]{0.400pt}{0.400pt}}%
\put(171.0,131.0){\rule[-0.200pt]{4.818pt}{0.400pt}}
\put(151,131){\makebox(0,0)[r]{-0.5}}
\put(1419.0,131.0){\rule[-0.200pt]{4.818pt}{0.400pt}}
\put(171.0,235.0){\rule[-0.200pt]{4.818pt}{0.400pt}}
\put(151,235){\makebox(0,0)[r]{ 0}}
\put(1419.0,235.0){\rule[-0.200pt]{4.818pt}{0.400pt}}
\put(171.0,339.0){\rule[-0.200pt]{4.818pt}{0.400pt}}
\put(151,339){\makebox(0,0)[r]{ 0.5}}
\put(1419.0,339.0){\rule[-0.200pt]{4.818pt}{0.400pt}}
\put(171.0,443.0){\rule[-0.200pt]{4.818pt}{0.400pt}}
\put(151,443){\makebox(0,0)[r]{ 1}}
\put(1419.0,443.0){\rule[-0.200pt]{4.818pt}{0.400pt}}
\put(171.0,547.0){\rule[-0.200pt]{4.818pt}{0.400pt}}
\put(151,547){\makebox(0,0)[r]{ 1.5}}
\put(1419.0,547.0){\rule[-0.200pt]{4.818pt}{0.400pt}}
\put(171.0,651.0){\rule[-0.200pt]{4.818pt}{0.400pt}}
\put(151,651){\makebox(0,0)[r]{ 2}}
\put(1419.0,651.0){\rule[-0.200pt]{4.818pt}{0.400pt}}
\put(171.0,755.0){\rule[-0.200pt]{4.818pt}{0.400pt}}
\put(151,755){\makebox(0,0)[r]{ 2.5}}
\put(1419.0,755.0){\rule[-0.200pt]{4.818pt}{0.400pt}}
\put(171.0,859.0){\rule[-0.200pt]{4.818pt}{0.400pt}}
\put(151,859){\makebox(0,0)[r]{ 3}}
\put(1419.0,859.0){\rule[-0.200pt]{4.818pt}{0.400pt}}
\put(171.0,131.0){\rule[-0.200pt]{0.400pt}{4.818pt}}
\put(171,90){\makebox(0,0){ 0.05}}
\put(171.0,839.0){\rule[-0.200pt]{0.400pt}{4.818pt}}
\put(330.0,131.0){\rule[-0.200pt]{0.400pt}{4.818pt}}
\put(330,90){\makebox(0,0){ 0.1}}
\put(330.0,839.0){\rule[-0.200pt]{0.400pt}{4.818pt}}
\put(488.0,131.0){\rule[-0.200pt]{0.400pt}{4.818pt}}
\put(488,90){\makebox(0,0){ 0.15}}
\put(488.0,839.0){\rule[-0.200pt]{0.400pt}{4.818pt}}
\put(647.0,131.0){\rule[-0.200pt]{0.400pt}{4.818pt}}
\put(647,90){\makebox(0,0){ 0.2}}
\put(647.0,839.0){\rule[-0.200pt]{0.400pt}{4.818pt}}
\put(805.0,131.0){\rule[-0.200pt]{0.400pt}{4.818pt}}
\put(805,90){\makebox(0,0){ 0.25}}
\put(805.0,839.0){\rule[-0.200pt]{0.400pt}{4.818pt}}
\put(964.0,131.0){\rule[-0.200pt]{0.400pt}{4.818pt}}
\put(964,90){\makebox(0,0){ 0.3}}
\put(964.0,839.0){\rule[-0.200pt]{0.400pt}{4.818pt}}
\put(1122.0,131.0){\rule[-0.200pt]{0.400pt}{4.818pt}}
\put(1122,90){\makebox(0,0){ 0.35}}
\put(1122.0,839.0){\rule[-0.200pt]{0.400pt}{4.818pt}}
\put(1281.0,131.0){\rule[-0.200pt]{0.400pt}{4.818pt}}
\put(1281,90){\makebox(0,0){ 0.4}}
\put(1281.0,839.0){\rule[-0.200pt]{0.400pt}{4.818pt}}
\put(1439.0,131.0){\rule[-0.200pt]{0.400pt}{4.818pt}}
\put(1439,90){\makebox(0,0){ 0.45}}
\put(1439.0,839.0){\rule[-0.200pt]{0.400pt}{4.818pt}}
\put(171.0,131.0){\rule[-0.200pt]{0.400pt}{175.375pt}}
\put(171.0,131.0){\rule[-0.200pt]{305.461pt}{0.400pt}}
\put(1439.0,131.0){\rule[-0.200pt]{0.400pt}{175.375pt}}
\put(171.0,859.0){\rule[-0.200pt]{305.461pt}{0.400pt}}
\put(30,495){\makebox(0,0){\popi{\kappa}{-}}}
\put(805,29){\makebox(0,0){\popi{t}{s}}}
\put(1279,819){\makebox(0,0)[r]{vypočítané hodnoty $\kappa$}}
\put(1299.0,819.0){\rule[-0.200pt]{24.090pt}{0.400pt}}
\put(1299.0,809.0){\rule[-0.200pt]{0.400pt}{4.818pt}}
\put(1399.0,809.0){\rule[-0.200pt]{0.400pt}{4.818pt}}
\put(501.0,470.0){\rule[-0.200pt]{0.400pt}{21.440pt}}
\put(491.0,470.0){\rule[-0.200pt]{4.818pt}{0.400pt}}
\put(491.0,559.0){\rule[-0.200pt]{4.818pt}{0.400pt}}
\put(577.0,440.0){\rule[-0.200pt]{0.400pt}{30.112pt}}
\put(567.0,440.0){\rule[-0.200pt]{4.818pt}{0.400pt}}
\put(567.0,565.0){\rule[-0.200pt]{4.818pt}{0.400pt}}
\put(437.0,427.0){\rule[-0.200pt]{0.400pt}{31.799pt}}
\put(427.0,427.0){\rule[-0.200pt]{4.818pt}{0.400pt}}
\put(427.0,559.0){\rule[-0.200pt]{4.818pt}{0.400pt}}
\put(513.0,458.0){\rule[-0.200pt]{0.400pt}{24.813pt}}
\put(503.0,458.0){\rule[-0.200pt]{4.818pt}{0.400pt}}
\put(503.0,561.0){\rule[-0.200pt]{4.818pt}{0.400pt}}
\put(466.0,467.0){\rule[-0.200pt]{0.400pt}{22.404pt}}
\put(456.0,467.0){\rule[-0.200pt]{4.818pt}{0.400pt}}
\put(456.0,560.0){\rule[-0.200pt]{4.818pt}{0.400pt}}
\put(421.0,462.0){\rule[-0.200pt]{0.400pt}{25.054pt}}
\put(411.0,462.0){\rule[-0.200pt]{4.818pt}{0.400pt}}
\put(411.0,566.0){\rule[-0.200pt]{4.818pt}{0.400pt}}
\put(542.0,480.0){\rule[-0.200pt]{0.400pt}{19.513pt}}
\put(532.0,480.0){\rule[-0.200pt]{4.818pt}{0.400pt}}
\put(532.0,561.0){\rule[-0.200pt]{4.818pt}{0.400pt}}
\put(488.0,463.0){\rule[-0.200pt]{0.400pt}{22.163pt}}
\put(478.0,463.0){\rule[-0.200pt]{4.818pt}{0.400pt}}
\put(478.0,555.0){\rule[-0.200pt]{4.818pt}{0.400pt}}
\put(428.0,462.0){\rule[-0.200pt]{0.400pt}{25.054pt}}
\put(418.0,462.0){\rule[-0.200pt]{4.818pt}{0.400pt}}
\put(418.0,566.0){\rule[-0.200pt]{4.818pt}{0.400pt}}
\put(1363.0,441.0){\rule[-0.200pt]{0.400pt}{27.944pt}}
\put(1353.0,441.0){\rule[-0.200pt]{4.818pt}{0.400pt}}
\put(1353.0,557.0){\rule[-0.200pt]{4.818pt}{0.400pt}}
\put(583.0,447.0){\rule[-0.200pt]{0.400pt}{28.185pt}}
\put(573.0,447.0){\rule[-0.200pt]{4.818pt}{0.400pt}}
\put(573.0,564.0){\rule[-0.200pt]{4.818pt}{0.400pt}}
\put(881.0,402.0){\rule[-0.200pt]{0.400pt}{46.253pt}}
\put(871.0,402.0){\rule[-0.200pt]{4.818pt}{0.400pt}}
\put(871.0,594.0){\rule[-0.200pt]{4.818pt}{0.400pt}}
\put(1135.0,445.0){\rule[-0.200pt]{0.400pt}{27.944pt}}
\put(1125.0,445.0){\rule[-0.200pt]{4.818pt}{0.400pt}}
\put(1125.0,561.0){\rule[-0.200pt]{4.818pt}{0.400pt}}
\put(466.0,455.0){\rule[-0.200pt]{0.400pt}{28.426pt}}
\put(456.0,455.0){\rule[-0.200pt]{4.818pt}{0.400pt}}
\put(456.0,573.0){\rule[-0.200pt]{4.818pt}{0.400pt}}
\put(599.0,430.0){\rule[-0.200pt]{0.400pt}{38.544pt}}
\put(589.0,430.0){\rule[-0.200pt]{4.818pt}{0.400pt}}
\put(589.0,590.0){\rule[-0.200pt]{4.818pt}{0.400pt}}
\put(358.0,434.0){\rule[-0.200pt]{0.400pt}{38.785pt}}
\put(348.0,434.0){\rule[-0.200pt]{4.818pt}{0.400pt}}
\put(348.0,595.0){\rule[-0.200pt]{4.818pt}{0.400pt}}
\put(314.0,424.0){\rule[-0.200pt]{0.400pt}{38.303pt}}
\put(304.0,424.0){\rule[-0.200pt]{4.818pt}{0.400pt}}
\put(304.0,583.0){\rule[-0.200pt]{4.818pt}{0.400pt}}
\put(320.0,459.0){\rule[-0.200pt]{0.400pt}{25.054pt}}
\put(310.0,459.0){\rule[-0.200pt]{4.818pt}{0.400pt}}
\put(310.0,563.0){\rule[-0.200pt]{4.818pt}{0.400pt}}
\put(498.0,198.0){\rule[-0.200pt]{0.400pt}{136.831pt}}
\put(488.0,198.0){\rule[-0.200pt]{4.818pt}{0.400pt}}
\put(488.0,766.0){\rule[-0.200pt]{4.818pt}{0.400pt}}
\put(1303.0,449.0){\rule[-0.200pt]{0.400pt}{24.572pt}}
\put(1293.0,449.0){\rule[-0.200pt]{4.818pt}{0.400pt}}
\put(501,515){\makebox(0,0){$+$}}
\put(577,502){\makebox(0,0){$+$}}
\put(437,493){\makebox(0,0){$+$}}
\put(513,510){\makebox(0,0){$+$}}
\put(466,514){\makebox(0,0){$+$}}
\put(421,514){\makebox(0,0){$+$}}
\put(542,521){\makebox(0,0){$+$}}
\put(488,509){\makebox(0,0){$+$}}
\put(428,514){\makebox(0,0){$+$}}
\put(1363,499){\makebox(0,0){$+$}}
\put(583,505){\makebox(0,0){$+$}}
\put(881,498){\makebox(0,0){$+$}}
\put(1135,503){\makebox(0,0){$+$}}
\put(466,514){\makebox(0,0){$+$}}
\put(599,510){\makebox(0,0){$+$}}
\put(358,515){\makebox(0,0){$+$}}
\put(314,503){\makebox(0,0){$+$}}
\put(320,511){\makebox(0,0){$+$}}
\put(498,482){\makebox(0,0){$+$}}
\put(1303,500){\makebox(0,0){$+$}}
\put(1349,819){\makebox(0,0){$+$}}
\put(1293.0,551.0){\rule[-0.200pt]{4.818pt}{0.400pt}}
\put(1279,778){\makebox(0,0)[r]{fit $f(t) = \(-0.15\pm0.10\)t + \(1.335\pm0.021\)$}}
\multiput(1299,778)(20.756,0.000){5}{\usebox{\plotpoint}}
\put(1399,778){\usebox{\plotpoint}}
\put(314,510){\usebox{\plotpoint}}
\put(314.00,510.00){\usebox{\plotpoint}}
\put(334.71,509.00){\usebox{\plotpoint}}
\put(355.46,509.00){\usebox{\plotpoint}}
\put(376.22,509.00){\usebox{\plotpoint}}
\put(396.97,509.00){\usebox{\plotpoint}}
\put(417.69,508.21){\usebox{\plotpoint}}
\put(438.44,508.00){\usebox{\plotpoint}}
\put(459.19,508.00){\usebox{\plotpoint}}
\put(479.95,508.00){\usebox{\plotpoint}}
\put(500.70,508.00){\usebox{\plotpoint}}
\put(521.43,507.42){\usebox{\plotpoint}}
\put(542.17,507.00){\usebox{\plotpoint}}
\put(562.93,507.00){\usebox{\plotpoint}}
\put(583.68,507.00){\usebox{\plotpoint}}
\put(604.44,507.00){\usebox{\plotpoint}}
\put(625.15,506.00){\usebox{\plotpoint}}
\put(645.90,506.00){\usebox{\plotpoint}}
\put(666.66,506.00){\usebox{\plotpoint}}
\put(687.41,506.00){\usebox{\plotpoint}}
\put(708.16,505.78){\usebox{\plotpoint}}
\put(728.87,505.00){\usebox{\plotpoint}}
\put(749.63,505.00){\usebox{\plotpoint}}
\put(770.39,505.00){\usebox{\plotpoint}}
\put(791.14,505.00){\usebox{\plotpoint}}
\put(811.85,504.01){\usebox{\plotpoint}}
\put(832.61,504.00){\usebox{\plotpoint}}
\put(853.36,504.00){\usebox{\plotpoint}}
\put(874.12,504.00){\usebox{\plotpoint}}
\put(894.87,504.00){\usebox{\plotpoint}}
\put(915.58,503.00){\usebox{\plotpoint}}
\put(936.33,503.00){\usebox{\plotpoint}}
\put(957.09,503.00){\usebox{\plotpoint}}
\put(977.84,503.00){\usebox{\plotpoint}}
\put(998.57,502.40){\usebox{\plotpoint}}
\put(1019.31,502.00){\usebox{\plotpoint}}
\put(1040.07,502.00){\usebox{\plotpoint}}
\put(1060.82,502.00){\usebox{\plotpoint}}
\put(1081.58,502.00){\usebox{\plotpoint}}
\put(1102.29,501.00){\usebox{\plotpoint}}
\put(1123.04,501.00){\usebox{\plotpoint}}
\put(1143.80,501.00){\usebox{\plotpoint}}
\put(1164.55,501.00){\usebox{\plotpoint}}
\put(1185.31,501.00){\usebox{\plotpoint}}
\put(1206.02,500.00){\usebox{\plotpoint}}
\put(1226.77,500.00){\usebox{\plotpoint}}
\put(1247.53,500.00){\usebox{\plotpoint}}
\put(1268.29,500.00){\usebox{\plotpoint}}
\put(1289.04,500.00){\usebox{\plotpoint}}
\put(1309.75,499.00){\usebox{\plotpoint}}
\put(1330.50,499.00){\usebox{\plotpoint}}
\put(1351.26,499.00){\usebox{\plotpoint}}
\put(1363,499){\usebox{\plotpoint}}
\put(171.0,131.0){\rule[-0.200pt]{0.400pt}{175.375pt}}
\put(171.0,131.0){\rule[-0.200pt]{305.461pt}{0.400pt}}
\put(1439.0,131.0){\rule[-0.200pt]{0.400pt}{175.375pt}}
\put(171.0,859.0){\rule[-0.200pt]{305.461pt}{0.400pt}}
\end{picture}

\caption{extrapolácia nameraných hodnôt pre $t="0"$}  \label{G_1}
\end{figure}

Pre dáta s otváracím časom pod $"200 ms"$ je hodnota $\kappa_{\(0\)}= "1.36\pm0.04"$ a 
pre hodnoty s otváracím časom nad $"200 ms"$ bola vypočítaná $\kappa_{\(0\)}= "1.29\pm0.03"$.

\subsection{Diskusia}
Pri metóde kmitajúceho piestu spôsobuje hlavný zdroj nepresností a 
systematických chýb netesnosť medzi piestom a aparatúrou. 
Ďalej aj pomerne mala dierka na vypúšťanie plynu. 
Teda expanzia nieje okamžitá a vyrovnanie tlakov úplné.
Zaujímavosťou je zvyšovanie počtu kmitov s pripadajúcim časom, 
mojou teóriou na vysvetlenie tohoto javu je na zahrievanie piestu trením a teda jeho zväčšenie a teda sa zlepšilo tesnenie a neunikalo toľko plynu, pokrajoch. 

Clémentova-Désormesova metoda sa však viac približuje očakávanému výsledku \textit{$\kappa = \sim "1.40"$ pre $N_2$ alebo $O_2$}\cite{C_1}. 
Hlavné nepresnosti pri tejto metóde spočívajú v nie dokonalým vyrovnaním teplôt po vypustení plynu. 
A nevhodná funkcia na extrapoláciu dát.

\subsection{Záver}
Clémentova-Désormesovou metódou bola $\kappa_{\(0\)}= "1.36\pm0.04"$, a metódou kmitajúceho piestu $\kappa = "1.68\pm0.01"$.




%%%%%%%%%%%%%%%%%%%%%%%%%%%%%%%%%%%%%%%%%%%%%%%%%%%%%%%%%%%%%%%%%%%%%%%%%%%%%%%%%%%%%%%%%%%%%%%%%%%%%%%%%%%%%%%%%
%%%%%%%%%%%%%%%%%%%%%%%%%%%%%%%%%%%%%%%%%%%%%%%%%%%%%%%%%%%%%%%%%%%%%%%%%%%%%%%%%%%%%%%%%%%%%%%%%%%%%%%%%%%%%%%%%
%%%%%%%%%%%%%%%%%%%%%%%%%%%%%%%%%%%%%%%%%%%%%%%%%%%%%%%%%%%%%%%%%%%%%%%%%%%%%%%%%%%%%%%%%%%%%%%%%%%%%%%%%%%%%%%%%

\section{Úkol \#2}
\subsection{Pracovní úkol}
\begin{enumerate}
\item Určete objem láhve metodou vážení.
\item Určete objem ťěze láhve pomocí komprese plynu.
\item Oba výsledky vzájemně porovnejte.
\end{enumerate}

\subsection{Postup merania}
\subsubsection{Metóda kompresie plynu}
\begin{enumerate}
\item povolením ventilu bol vyrovnaný tlak v byrete s atmosferickým tlakom.
\item Vertikálnym pohybom nádoby s vodou bola hladina v byrete ustálená na úroveň $"0 \%"$.
\item Ventil pol zatvorený.
\item následne sa pohlo nádobou s vodou nahor.
\item počkalo sa na ustálenie hladín a boli odčítané hodnoty $\Delta h$ ,$V_2$.
\item Postup sa opakoval niekoľkokrát pre veľkú nádobu.
\item Meraná nádoba bola vymenená za utesnenie a postup bol zopakovaný pre meranie objemu len hadičky.
\end{enumerate}


\subsubsection{Metóda vážení}
\begin{enumerate}
\item Nádoba bola odvážená na digitálnych váhach 
\item Nádoba bola pookraj naplnená vodou a dôkladne osušení jej povrch
\item Naplnená nádoba bola opäť odvážená
\item Bola odmeraná teplota vody.
\end{enumerate}

\subsection{Pomôcky}
Fľaška (nádoba), plynová byreta s porovnávacím ramenem, katetometr,
teploměr, barometr, digitálne váhy do $"5 kg"$.

\section{Teória}

\subsubsection{Metóda kompresie plynu}
Pre metódu kompresie plynu v našom prípade môžeme odvodiť vzťah
\eq{
V = \( V_2-V_1 \)\frac{p}{\Delta p} + V_2 - V_{100} \,, \lbl{R_6}
}, kde 
\eq{
\Delta p = \Delta h \rho g\,.
}
Pričom $V_1$ je objem v byrete pri vyrovnaní tlakov.$\Delta h$ je rozdiel hladín, a $V_2$ výška hladiny po kompresií.

V našom prípade $V_1="0 \%"$. Pričom $V_{100} = "65.6 cm^3"$

\subsubsection{Metóda vážení}
Jednotkový objem vody je závisí na teplote $t$ v $\C$ podľa vzťahu
\eq{
V_v = 0.9998\cdot(1+0.00018 t) \frac{\jd{cm^3}}{\jd{g}}\,.\lbl{R_5}
} 
Potom objem metódou váženia určíme ako\eq{
V = \( m_n- m_p\) V_v \,. \lbl{R_4}
}


\subsection{Výsledky merania}
\subsubsection{Metóda kompresie plynu}
V tab. \ref{T_2} sú zaznamenané namerané hodnoty $V_2$ a $\Delta h$
z ktorých bola vypočítaná hodnota $V$. 
V prvej časti tabuľky sú hodnoty pre fľašku v druhej časti je hodnota pre samotnú hadičku.


\begin{table}[h]

\begin{center}
\begin{tabular}{| c | c | c |}
\hline
 \popi{V_2}{\%} & \popi{\Delta h}{cm} & \popi{V}{cm^3} \\
\hline
$5$   &	$3.51$ & $896.08$\\
$9$   & $5.85$  & $975.39$\\
$6$   & $5.9$   & $619.31$\\
$4.5$ & $4,10$  & $676.69$\\
$8$   & $4,68$  & $1\,089.74$\\
$8$   & $6.83$  & $728.29$\\
$9$   & $7.41$  & $757.49$\\
$7$   & $5.27$  & $833.51$\\
$8$   & $5.46$  & $925.44$\\
$6.75$& $5.85$  & $715.14$\\
\hline
$1$ &	$10.34$ & $5.58$\\							
\hline

\end{tabular}
\caption{Namerané hodnoty $V_2$, $\Delta h$ a vypočítaný objem $V$, v prvej časti pre nádobu a v druhej pre hadičku.} \label{T_2}
\end{center}
\end{table}

Objem fľašky po odčítaný objemu hadičky bol určený ako $V = "\(816.7\pm148.3\) cm^3"$.


\subsubsection{Metóda vážení}
Hmotnosť prázdnej suchej nádoby bola určená $m_p = "\(582\pm1\) g"$, jednotková hmotnosť vody pri $t="11.8 \C"$ bola určená podľa \ref{R_5} ako $V_v="1.002 \frac{cm^3}{g}"$
Hmotnosť po naplnení vodou bola určená $m_n = "\(1598\pm1\) g"$.
Podľa vzorca \ref{R_4} bol objem nádoby určený ako $V="\(1018\pm2\) cm^3"$.

\subsection{Diskusia}
Vo výsledkoch vidíme veľmi veľký rozdiel nameraných hodnôt, 
nepresnosť merania pri metóde kompresie plynu spôsobovali extrémne 
veľké netesnosti, pri ktorých pretlak z aparatúry veľmi rýchlo unikal. 
Teda toto meranie bolo zaťažené veľkou systematickou chybou. 
Aj napriek snahe merať rýchlo sa štatistická chyba pohybuje na úrovni $"18 \%"$. 
Naopak metóda váženia sa ukazuje ako veľmi presná.

\subsection{Záver}
Metódou váženia bol objem nádoby určený na $V="\(1018\pm2\) cm^3"$.
Metódou kompresie plynu bol určený objem $V = "\(816.7\pm148.3\) cm^3"$.


\begin{thebibliography}{1}
\bibitem{C_1}
Článok dostupný na \url{https://cs.wikipedia.org/wiki/Poissonova_konstanta#Hodnoty_pro_re.C3.A1ln.C3.A9_plyny}
\end{thebibliography}

\end{document}






%\documentclass[a4paper,10pt]{article}
\documentclass[10pt]{scrartcl}
%\usepackage[IL2]{fontenc}
\usepackage[utf8x]{inputenc}
\usepackage[czech]{babel}
\usepackage{listings}  
\usepackage{amsfonts,amsmath,amssymb,graphicx,color}
%\usepackage[total={17cm,27cm}, top=2cm, left=2cm, includefoot]{geometry}
%\usepackage{fancyhdr}
\usepackage{fkssugar}
\usepackage{hyperref}
\usepackage{mhchem}

%\usepackage{caption}

\input{../normalize.sty}
%  Nastaví autora, název, datum, skupinu měření apod. 
\newcommand{\FJFIInstitute}{FJFI~ČVUT~v~Praze}
%\newcommand{\Subject}{Základy fyzikálních měření}
\newcommand{\FJFISubject}{Fyzikální praktikum I}  %odkomentujte dle potřeby

%  Máte-li více spoluměřících než jednoho, vložte jen jejich příjmení
\newcommand{\FJFIAuthor}{Michal Červeňák}
\newcommand{\FJFICoauthor}{} 
\newcommand{\FJFIGroup}{Pondelok} %den, kdy chodíte na praktika, nikoli obor
\newcommand{\FJFICircle}{4} %číslo skupiny v rámci praktika, nikoli kruh

%  Tato část bude v každém protokolu jiná, nezapomeňte upravit!
\newcommand{\FJFITitle}{5. Poissonova konstanta a měření dutých objemů}
\newcommand{\FJFILabdate}{16.10.2017} %datum měření, nikoli datum odevzdání
\newcommand{\FJFIWorktime}{5 h} %jak dlouho vám trvalo vypracování protokolu


\begin{document}

\MakeFJFIHead{}

\section{Úkol 1}
\subsection{Pracovní úkol}

\begin{enumerate}
\item DU: Odvoďte rovnici pro Poissonovu konstantu (14)\cite{1}. Vyjděte z (2)\cite{1} a (13)\cite{1}.
\item Změřte Poissonovu konstantu metodou kmitajícího pístku.
\item Změřte Poissonovu konstantu Clément-Désormesovou metodou. Nezapomeňte provést opravu vašeho měření na systematické chyby.
\item Oba výsledky vzájemně porovnejte (procentuálně) a diskutujte, jestli je v rámci chyb můžete považovat za stejné.
\end{enumerate}

\subsection{Teória}
Poissonova konstanta $\kappa$ j pomer merného tepla $C_p$ pri stálom objeme a pri stálom objeme $C_V$, teda 
\eq{
\kappa = \frac{C_p}{C_V}\,.
}

\subsubsection{ Clémentova-Désormesova metoda}
Metóda určuje Poissonova konstanta z adiabatického deja, pri ktorom vypúšťame plyn z nádoby kde je pretlak $h$. A po vypustení a ustálení teplôt $h^\prime$.
Pre výpočet $\kappa$ môžeme odvodiť vzorec
\eq{
\kappa = \frac{h}{h-h^\prime} \lbl{R_7}\,. 
}

\subsubsection{Metoda kmitajícího pístku}
Pre hodnotu $\kappa$ môžeme odvodiť vzťah na závislosť do doby kmitu\eq{
\kappa = \frac{4mV}{T^2 p r^4}\,,\lbl{R_8}
}
kde
\eq{
p=b+\frac{mg}{\pi r^2}\,,
}
,pričom $b$ je atmosferický tlak, hmotnosť piestu je $m="4.59\cdot10^{-3} kg"$, objem banky je $V="1.133 l"$, priemer piestu je $2r = "11.9\cdot10^{-3} m"$ a g je tiažové zrýchlenie.

\subsubsection{Spracovanie chýb merania}

Označme $\mean{t}$ aritmetický priemer nameraných hodnôt $t_i$, a $\Delta t$ hodnotu $\mean{t}-t$, pričom 
\eq{
\mean{t} = \frac{1}{n}\sum_{i=1}^n t_i \,, \lbl{SCH_1}
}  
a chybu aritmetického priemeru 
\eq{
  \sigma_0=\sqrt{\frac{\sum_{i=1}^n \(t_i - \mean{t}\)^2}{n\(n-1\)}}\,, \lbl{SCH_2}
}
pričom $n$ je počet meraní.

Majme veličina  $ u = f(x,y,z,\ldots)$, potom podľa zákou šírenia chýb platí
\eq{
\sigma_u = \sqrt{\(\pder{f}{x}\)^2_0 \sigma_x^2 +\(\pder{f}{y}\)^2_0 \sigma_y^2 + \(\pder{f}{z}\)^2_0 \sigma_z^2 + \ldots}\,, \lbl{SCH_3}
}
kde $\sigma_i$ je stredná chyba veličiny $i$ v bode $\(x_0,y_0,z_0\)$.




\subsection{Postup merania}
\subsubsection{Metoda kmitajícího pístku}
Najskôr bolo zapnuté čerpadlo vzduchu ktoré privádza vzduch do nádoby, ventilom bol nastavený prúd vzduchu, tak aby piest kmital medzi značkami.
Bol spustený digitálny čítač kmitov a nastavený na počítanie kmitov po $t="300 s"$, po uplynutí intervalu boli dáta zaznamenané a opätovné spustenie počítanie. 

\subsubsection{ Clémentova-Désormesova metoda}

Pripravená nádoba nádoba bola natlakovaná pomocou mechu, bol uzavretý prívodný ventil tlak v nádobe bol odmeraný barometru.
Pomocou rýchlo-ventilu bola časť vzduchu odpustená, pričom bol zaznamenaný čas otvorenia ventilu.
Následne sa počkalo $\sim "1 min"$ na ustálenie teplôt v nádobe s okolím a bol zmeraný opäť tlak v aparatúre.

\subsection{Pomôcky}
Barometr, aparatura na měřené Poissonovy konstanty Clément-Désormesovou metodou,
aparatura pro měření Poissonovy konstanty metodou kmitajícího pístku.

\subsection{Výsledky merania}
\subsubsection{Metoda kmitajícího pístku}
V tab. \ref{T_1} sú zaznamenané počty kmitov za čas $t = "300 s"$, pre jednotlivé merania.

\begin{table}[h]
\begin{center}
\begin{tabular}{| c |}
\hline
 \popi{N}{-} \\
\hline
879\\
885\\
886\\
885\\
884\\
883\\
883\\
883\\
882\\
882\\
\hline
\end{tabular}
\caption{Namerané počty kmitov $N$ za čas $t="300 s"$} \label{T_1}
\end{center}
\end{table}
Z hodnôt v tab \ref{T_1} bol vypočítaná priemerná hodnota počtu kmitov$\mean{N}="883.2\pm2.0"$.
Pomocou vzťahov \ref{R_8} a \ref{SCH_3} bola vypočítaná Poissonova konstanta $\kappa = "1.62\pm0.01"$.

\subsubsection{ Clémentova-Désormesova metoda}
Namerané hodnoty výšok hladín, pred a po vypustení, na otváracom čase sú v tabuľke \ref{T_2}.
Tie boli vynesené do grafu \ref{G_1}, 
dáta boli preložené lineárnou funkciu $f(t) =  \(-0.15\pm0.10\)t  + \(1.335\pm0.021\)$, 
kde absolútny člen odpovedá hodnote $\kappa$, extrapolovanej pre nulový otvárací čas, teda pre dokonalý adiabatický dej.

\begin{figure}
% GNUPLOT: LaTeX picture
\setlength{\unitlength}{0.240900pt}
\ifx\plotpoint\undefined\newsavebox{\plotpoint}\fi
\begin{picture}(1500,900)(0,0)
\sbox{\plotpoint}{\rule[-0.200pt]{0.400pt}{0.400pt}}%
\put(171.0,131.0){\rule[-0.200pt]{4.818pt}{0.400pt}}
\put(151,131){\makebox(0,0)[r]{-0.5}}
\put(1419.0,131.0){\rule[-0.200pt]{4.818pt}{0.400pt}}
\put(171.0,235.0){\rule[-0.200pt]{4.818pt}{0.400pt}}
\put(151,235){\makebox(0,0)[r]{ 0}}
\put(1419.0,235.0){\rule[-0.200pt]{4.818pt}{0.400pt}}
\put(171.0,339.0){\rule[-0.200pt]{4.818pt}{0.400pt}}
\put(151,339){\makebox(0,0)[r]{ 0.5}}
\put(1419.0,339.0){\rule[-0.200pt]{4.818pt}{0.400pt}}
\put(171.0,443.0){\rule[-0.200pt]{4.818pt}{0.400pt}}
\put(151,443){\makebox(0,0)[r]{ 1}}
\put(1419.0,443.0){\rule[-0.200pt]{4.818pt}{0.400pt}}
\put(171.0,547.0){\rule[-0.200pt]{4.818pt}{0.400pt}}
\put(151,547){\makebox(0,0)[r]{ 1.5}}
\put(1419.0,547.0){\rule[-0.200pt]{4.818pt}{0.400pt}}
\put(171.0,651.0){\rule[-0.200pt]{4.818pt}{0.400pt}}
\put(151,651){\makebox(0,0)[r]{ 2}}
\put(1419.0,651.0){\rule[-0.200pt]{4.818pt}{0.400pt}}
\put(171.0,755.0){\rule[-0.200pt]{4.818pt}{0.400pt}}
\put(151,755){\makebox(0,0)[r]{ 2.5}}
\put(1419.0,755.0){\rule[-0.200pt]{4.818pt}{0.400pt}}
\put(171.0,859.0){\rule[-0.200pt]{4.818pt}{0.400pt}}
\put(151,859){\makebox(0,0)[r]{ 3}}
\put(1419.0,859.0){\rule[-0.200pt]{4.818pt}{0.400pt}}
\put(171.0,131.0){\rule[-0.200pt]{0.400pt}{4.818pt}}
\put(171,90){\makebox(0,0){ 0.05}}
\put(171.0,839.0){\rule[-0.200pt]{0.400pt}{4.818pt}}
\put(330.0,131.0){\rule[-0.200pt]{0.400pt}{4.818pt}}
\put(330,90){\makebox(0,0){ 0.1}}
\put(330.0,839.0){\rule[-0.200pt]{0.400pt}{4.818pt}}
\put(488.0,131.0){\rule[-0.200pt]{0.400pt}{4.818pt}}
\put(488,90){\makebox(0,0){ 0.15}}
\put(488.0,839.0){\rule[-0.200pt]{0.400pt}{4.818pt}}
\put(647.0,131.0){\rule[-0.200pt]{0.400pt}{4.818pt}}
\put(647,90){\makebox(0,0){ 0.2}}
\put(647.0,839.0){\rule[-0.200pt]{0.400pt}{4.818pt}}
\put(805.0,131.0){\rule[-0.200pt]{0.400pt}{4.818pt}}
\put(805,90){\makebox(0,0){ 0.25}}
\put(805.0,839.0){\rule[-0.200pt]{0.400pt}{4.818pt}}
\put(964.0,131.0){\rule[-0.200pt]{0.400pt}{4.818pt}}
\put(964,90){\makebox(0,0){ 0.3}}
\put(964.0,839.0){\rule[-0.200pt]{0.400pt}{4.818pt}}
\put(1122.0,131.0){\rule[-0.200pt]{0.400pt}{4.818pt}}
\put(1122,90){\makebox(0,0){ 0.35}}
\put(1122.0,839.0){\rule[-0.200pt]{0.400pt}{4.818pt}}
\put(1281.0,131.0){\rule[-0.200pt]{0.400pt}{4.818pt}}
\put(1281,90){\makebox(0,0){ 0.4}}
\put(1281.0,839.0){\rule[-0.200pt]{0.400pt}{4.818pt}}
\put(1439.0,131.0){\rule[-0.200pt]{0.400pt}{4.818pt}}
\put(1439,90){\makebox(0,0){ 0.45}}
\put(1439.0,839.0){\rule[-0.200pt]{0.400pt}{4.818pt}}
\put(171.0,131.0){\rule[-0.200pt]{0.400pt}{175.375pt}}
\put(171.0,131.0){\rule[-0.200pt]{305.461pt}{0.400pt}}
\put(1439.0,131.0){\rule[-0.200pt]{0.400pt}{175.375pt}}
\put(171.0,859.0){\rule[-0.200pt]{305.461pt}{0.400pt}}
\put(30,495){\makebox(0,0){\popi{\kappa}{-}}}
\put(805,29){\makebox(0,0){\popi{t}{s}}}
\put(1279,819){\makebox(0,0)[r]{vypočítané hodnoty $\kappa$}}
\put(1299.0,819.0){\rule[-0.200pt]{24.090pt}{0.400pt}}
\put(1299.0,809.0){\rule[-0.200pt]{0.400pt}{4.818pt}}
\put(1399.0,809.0){\rule[-0.200pt]{0.400pt}{4.818pt}}
\put(501.0,470.0){\rule[-0.200pt]{0.400pt}{21.440pt}}
\put(491.0,470.0){\rule[-0.200pt]{4.818pt}{0.400pt}}
\put(491.0,559.0){\rule[-0.200pt]{4.818pt}{0.400pt}}
\put(577.0,440.0){\rule[-0.200pt]{0.400pt}{30.112pt}}
\put(567.0,440.0){\rule[-0.200pt]{4.818pt}{0.400pt}}
\put(567.0,565.0){\rule[-0.200pt]{4.818pt}{0.400pt}}
\put(437.0,427.0){\rule[-0.200pt]{0.400pt}{31.799pt}}
\put(427.0,427.0){\rule[-0.200pt]{4.818pt}{0.400pt}}
\put(427.0,559.0){\rule[-0.200pt]{4.818pt}{0.400pt}}
\put(513.0,458.0){\rule[-0.200pt]{0.400pt}{24.813pt}}
\put(503.0,458.0){\rule[-0.200pt]{4.818pt}{0.400pt}}
\put(503.0,561.0){\rule[-0.200pt]{4.818pt}{0.400pt}}
\put(466.0,467.0){\rule[-0.200pt]{0.400pt}{22.404pt}}
\put(456.0,467.0){\rule[-0.200pt]{4.818pt}{0.400pt}}
\put(456.0,560.0){\rule[-0.200pt]{4.818pt}{0.400pt}}
\put(421.0,462.0){\rule[-0.200pt]{0.400pt}{25.054pt}}
\put(411.0,462.0){\rule[-0.200pt]{4.818pt}{0.400pt}}
\put(411.0,566.0){\rule[-0.200pt]{4.818pt}{0.400pt}}
\put(542.0,480.0){\rule[-0.200pt]{0.400pt}{19.513pt}}
\put(532.0,480.0){\rule[-0.200pt]{4.818pt}{0.400pt}}
\put(532.0,561.0){\rule[-0.200pt]{4.818pt}{0.400pt}}
\put(488.0,463.0){\rule[-0.200pt]{0.400pt}{22.163pt}}
\put(478.0,463.0){\rule[-0.200pt]{4.818pt}{0.400pt}}
\put(478.0,555.0){\rule[-0.200pt]{4.818pt}{0.400pt}}
\put(428.0,462.0){\rule[-0.200pt]{0.400pt}{25.054pt}}
\put(418.0,462.0){\rule[-0.200pt]{4.818pt}{0.400pt}}
\put(418.0,566.0){\rule[-0.200pt]{4.818pt}{0.400pt}}
\put(1363.0,441.0){\rule[-0.200pt]{0.400pt}{27.944pt}}
\put(1353.0,441.0){\rule[-0.200pt]{4.818pt}{0.400pt}}
\put(1353.0,557.0){\rule[-0.200pt]{4.818pt}{0.400pt}}
\put(583.0,447.0){\rule[-0.200pt]{0.400pt}{28.185pt}}
\put(573.0,447.0){\rule[-0.200pt]{4.818pt}{0.400pt}}
\put(573.0,564.0){\rule[-0.200pt]{4.818pt}{0.400pt}}
\put(881.0,402.0){\rule[-0.200pt]{0.400pt}{46.253pt}}
\put(871.0,402.0){\rule[-0.200pt]{4.818pt}{0.400pt}}
\put(871.0,594.0){\rule[-0.200pt]{4.818pt}{0.400pt}}
\put(1135.0,445.0){\rule[-0.200pt]{0.400pt}{27.944pt}}
\put(1125.0,445.0){\rule[-0.200pt]{4.818pt}{0.400pt}}
\put(1125.0,561.0){\rule[-0.200pt]{4.818pt}{0.400pt}}
\put(466.0,455.0){\rule[-0.200pt]{0.400pt}{28.426pt}}
\put(456.0,455.0){\rule[-0.200pt]{4.818pt}{0.400pt}}
\put(456.0,573.0){\rule[-0.200pt]{4.818pt}{0.400pt}}
\put(599.0,430.0){\rule[-0.200pt]{0.400pt}{38.544pt}}
\put(589.0,430.0){\rule[-0.200pt]{4.818pt}{0.400pt}}
\put(589.0,590.0){\rule[-0.200pt]{4.818pt}{0.400pt}}
\put(358.0,434.0){\rule[-0.200pt]{0.400pt}{38.785pt}}
\put(348.0,434.0){\rule[-0.200pt]{4.818pt}{0.400pt}}
\put(348.0,595.0){\rule[-0.200pt]{4.818pt}{0.400pt}}
\put(314.0,424.0){\rule[-0.200pt]{0.400pt}{38.303pt}}
\put(304.0,424.0){\rule[-0.200pt]{4.818pt}{0.400pt}}
\put(304.0,583.0){\rule[-0.200pt]{4.818pt}{0.400pt}}
\put(320.0,459.0){\rule[-0.200pt]{0.400pt}{25.054pt}}
\put(310.0,459.0){\rule[-0.200pt]{4.818pt}{0.400pt}}
\put(310.0,563.0){\rule[-0.200pt]{4.818pt}{0.400pt}}
\put(498.0,198.0){\rule[-0.200pt]{0.400pt}{136.831pt}}
\put(488.0,198.0){\rule[-0.200pt]{4.818pt}{0.400pt}}
\put(488.0,766.0){\rule[-0.200pt]{4.818pt}{0.400pt}}
\put(1303.0,449.0){\rule[-0.200pt]{0.400pt}{24.572pt}}
\put(1293.0,449.0){\rule[-0.200pt]{4.818pt}{0.400pt}}
\put(501,515){\makebox(0,0){$+$}}
\put(577,502){\makebox(0,0){$+$}}
\put(437,493){\makebox(0,0){$+$}}
\put(513,510){\makebox(0,0){$+$}}
\put(466,514){\makebox(0,0){$+$}}
\put(421,514){\makebox(0,0){$+$}}
\put(542,521){\makebox(0,0){$+$}}
\put(488,509){\makebox(0,0){$+$}}
\put(428,514){\makebox(0,0){$+$}}
\put(1363,499){\makebox(0,0){$+$}}
\put(583,505){\makebox(0,0){$+$}}
\put(881,498){\makebox(0,0){$+$}}
\put(1135,503){\makebox(0,0){$+$}}
\put(466,514){\makebox(0,0){$+$}}
\put(599,510){\makebox(0,0){$+$}}
\put(358,515){\makebox(0,0){$+$}}
\put(314,503){\makebox(0,0){$+$}}
\put(320,511){\makebox(0,0){$+$}}
\put(498,482){\makebox(0,0){$+$}}
\put(1303,500){\makebox(0,0){$+$}}
\put(1349,819){\makebox(0,0){$+$}}
\put(1293.0,551.0){\rule[-0.200pt]{4.818pt}{0.400pt}}
\put(1279,778){\makebox(0,0)[r]{fit $f(t) = \(-0.15\pm0.10\)t + \(1.335\pm0.021\)$}}
\multiput(1299,778)(20.756,0.000){5}{\usebox{\plotpoint}}
\put(1399,778){\usebox{\plotpoint}}
\put(314,510){\usebox{\plotpoint}}
\put(314.00,510.00){\usebox{\plotpoint}}
\put(334.71,509.00){\usebox{\plotpoint}}
\put(355.46,509.00){\usebox{\plotpoint}}
\put(376.22,509.00){\usebox{\plotpoint}}
\put(396.97,509.00){\usebox{\plotpoint}}
\put(417.69,508.21){\usebox{\plotpoint}}
\put(438.44,508.00){\usebox{\plotpoint}}
\put(459.19,508.00){\usebox{\plotpoint}}
\put(479.95,508.00){\usebox{\plotpoint}}
\put(500.70,508.00){\usebox{\plotpoint}}
\put(521.43,507.42){\usebox{\plotpoint}}
\put(542.17,507.00){\usebox{\plotpoint}}
\put(562.93,507.00){\usebox{\plotpoint}}
\put(583.68,507.00){\usebox{\plotpoint}}
\put(604.44,507.00){\usebox{\plotpoint}}
\put(625.15,506.00){\usebox{\plotpoint}}
\put(645.90,506.00){\usebox{\plotpoint}}
\put(666.66,506.00){\usebox{\plotpoint}}
\put(687.41,506.00){\usebox{\plotpoint}}
\put(708.16,505.78){\usebox{\plotpoint}}
\put(728.87,505.00){\usebox{\plotpoint}}
\put(749.63,505.00){\usebox{\plotpoint}}
\put(770.39,505.00){\usebox{\plotpoint}}
\put(791.14,505.00){\usebox{\plotpoint}}
\put(811.85,504.01){\usebox{\plotpoint}}
\put(832.61,504.00){\usebox{\plotpoint}}
\put(853.36,504.00){\usebox{\plotpoint}}
\put(874.12,504.00){\usebox{\plotpoint}}
\put(894.87,504.00){\usebox{\plotpoint}}
\put(915.58,503.00){\usebox{\plotpoint}}
\put(936.33,503.00){\usebox{\plotpoint}}
\put(957.09,503.00){\usebox{\plotpoint}}
\put(977.84,503.00){\usebox{\plotpoint}}
\put(998.57,502.40){\usebox{\plotpoint}}
\put(1019.31,502.00){\usebox{\plotpoint}}
\put(1040.07,502.00){\usebox{\plotpoint}}
\put(1060.82,502.00){\usebox{\plotpoint}}
\put(1081.58,502.00){\usebox{\plotpoint}}
\put(1102.29,501.00){\usebox{\plotpoint}}
\put(1123.04,501.00){\usebox{\plotpoint}}
\put(1143.80,501.00){\usebox{\plotpoint}}
\put(1164.55,501.00){\usebox{\plotpoint}}
\put(1185.31,501.00){\usebox{\plotpoint}}
\put(1206.02,500.00){\usebox{\plotpoint}}
\put(1226.77,500.00){\usebox{\plotpoint}}
\put(1247.53,500.00){\usebox{\plotpoint}}
\put(1268.29,500.00){\usebox{\plotpoint}}
\put(1289.04,500.00){\usebox{\plotpoint}}
\put(1309.75,499.00){\usebox{\plotpoint}}
\put(1330.50,499.00){\usebox{\plotpoint}}
\put(1351.26,499.00){\usebox{\plotpoint}}
\put(1363,499){\usebox{\plotpoint}}
\put(171.0,131.0){\rule[-0.200pt]{0.400pt}{175.375pt}}
\put(171.0,131.0){\rule[-0.200pt]{305.461pt}{0.400pt}}
\put(1439.0,131.0){\rule[-0.200pt]{0.400pt}{175.375pt}}
\put(171.0,859.0){\rule[-0.200pt]{305.461pt}{0.400pt}}
\end{picture}

\caption{Vypočítané hodnoty $\kappa$ v závislosti na otváracom čase $t$, preložené funkciou $f(t) =  \(-0.15\pm0.10\)t  + \(1.335\pm0.021\)$}  \label{G_1}
\end{figure}

\begin{table}[h]
\begin{center}
\begin{tabular}{| c | c | c | c | c | c |}
\hline
\popi{h_1}{cm} & \popi{h_2}{cm} & \popi{t}{s} & \popi{h^\prime_1}{cm} & \popi{h^\prime_2}{cm} & \popi{\kappa}{-}\\
\hline
$"10.0\pm1.0"$ & $"49.0\pm1.0"$ & $"0.154\pm0.001"$ & $"24.5\pm1.0"$ & $"34.5\pm1.0"$ & $"1.34\pm0.21"$\\
$"14.0\pm1.0"$ & $"45.2\pm1.0"$ & $"0.178\pm0.001"$ & $"26.0\pm1.0"$ & $"33.0\pm1.0"$ & $"1.29\pm0.30"$\\
$"13.0\pm1.0"$ & $"46.5\pm1.0"$ & $"0.134\pm0.001"$ & $"26.5\pm1.0"$ & $"33.0\pm1.0"$ & $"1.24\pm0.32"$\\
$"12.0\pm1.0"$ & $"47.0\pm1.0"$ & $"0.158\pm0.001"$ & $"25.5\pm1.0"$ & $"34.0\pm1.0"$ & $"1.32\pm0.25"$\\
$"11.0\pm1.0"$ & $"48.5\pm1.0"$ & $"0.143\pm0.001"$ & $"25.0\pm1.0"$ & $"34.5\pm1.0"$ & $"1.34\pm0.22"$\\
$"13.0\pm1.0"$ & $"46.5\pm1.0"$ & $"0.129\pm0.001"$ & $"25.5\pm1.0"$ & $"34.0\pm1.0"$ & $"1.34\pm0.25"$\\
$" 9.5\pm1.0"$ & $"50.0\pm1.0"$ & $"0.167\pm0.001"$ & $"24.0\pm1.0"$ & $"35.0\pm1.0"$ & $"1.37\pm0.19"$\\
$"10.0\pm1.0"$ & $"49.5\pm1.0"$ & $"0.150\pm0.001"$ & $"25.0\pm1.0"$ & $"34.5\pm1.0"$ & $"1.32\pm0.22"$\\
$"13.0\pm1.0"$ & $"46.5\pm1.0"$ & $"0.131\pm0.001"$ & $"25.5\pm1.0"$ & $"34.0\pm1.0"$ & $"1.34\pm0.25"$\\
$"12.0\pm1.0"$ & $"47.5\pm1.0"$ & $"0.426\pm0.001"$ & $"26.0\pm1.0"$ & $"33.5\pm1.0"$ & $"1.27\pm0.28"$\\
$"13.5\pm1.0"$ & $"46.0\pm1.0"$ & $"0.180\pm0.001"$ & $"26.0\pm1.0"$ & $"33.5\pm1.0"$ & $"1.30\pm0.28"$\\
$"19.0\pm1.0"$ & $"40.5\pm1.0"$ & $"0.274\pm0.001"$ & $"27.5\pm1.0"$ & $"32.0\pm1.0"$ & $"1.26\pm0.46"$\\
$"13.0\pm1.0"$ & $"46.5\pm1.0"$ & $"0.354\pm0.001"$ & $"26.0\pm1.0"$ & $"33.5\pm1.0"$ & $"1.29\pm0.28"$\\
$"15.0\pm1.0"$ & $"44.5\pm1.0"$ & $"0.143\pm0.001"$ & $"26.0\pm1.0"$ & $"33.5\pm1.0"$ & $"1.35\pm0.28"$\\
$"18.5\pm1.0"$ & $"41.0\pm1.0"$ & $"0.185\pm0.001"$ & $"27.0\pm1.0"$ & $"32.5\pm1.0"$ & $"1.32\pm0.38"$\\
$"19.0\pm1.0"$ & $"40.5\pm1.0"$ & $"0.109\pm0.001"$ & $"27.0\pm1.0"$ & $"32.5\pm1.0"$ & $"1.34\pm0.39"$\\
$"17.5\pm1.0"$ & $"42.0\pm1.0"$ & $"0.095\pm0.001"$ & $"27.5\pm1.0"$ & $"33.0\pm1.0"$ & $"1.29\pm0.38"$\\
$"12.5\pm1.0"$ & $"47.0\pm1.0"$ & $"0.097\pm0.001"$ & $"25.5\pm1.0"$ & $"34.0\pm1.0"$ & $"1.33\pm0.25"$\\
$"25.0\pm1.0"$ & $"34.5\pm1.0"$ & $"0.153\pm0.001"$ & $"29.0\pm1.0"$ & $"30.5\pm1.0"$ & $"1.19\pm1.37"$\\
$"10.0\pm1.0"$ & $"49.5\pm1.0"$ & $"0.407\pm0.001"$ & $"25.5\pm1.0"$ & $"34.0\pm1.0"$ & $"1.27\pm0.25"$\\
\hline
\end{tabular}
\caption{Namerané hodnoty výšok hladín $h_i$ pred vypustením a $h^\prime_i$ po vypustení časti vzduchu, $t$ je otvárací čas a $\kappa$ vypočítaná pomocou vzťahu \ref{R_7} a \ref{SCH_3}} \label{T_2}
\end{center}
\end{table}

Hodnoty vypočítané len za pomoci tabuľky a priemeru podľa vzťahu \ref{SCH_2} je $\kappa = "1.305\pm0.044\text{stat.}\pm0.341\text{sys.}"$.

\subsection{Diskusia}
Pri metóde kmitajúceho piestu je hlavný zdroj chýb netesnosť medzi piestom a aparatúrou. 


Clémentova-Désormesova metoda sa však viac približuje očakávanému výsledku \textit{$\kappa = \sim "1.40"$ pre $N_2$ alebo $O_2$}\cite{C_2}. 
Hlavné nepresnosti pri tejto metóde spočívajú v nie dokonalým vyrovnaním teplôt po vypustení plynu, nedostatočným tepelným izolovaním nádoby.

Zároveň je vidno, že fitom a extrapoláciou sme výsledok upresnili narozdiel od jednoduchého vypočítania priemernej hodnoty z tabuľky.

\subsection{Záver}
Clémentova-Désormesovou metódou bola $\kappa_{\(0\)}= "1.335\pm0.021"$ resp. $\kappa = "1.305\pm0.044\text{stat.}\pm0.341\text{sys.}"$, a metódou kmitajúceho piestu $\kappa = "1.62\pm0.01"$.




%%%%%%%%%%%%%%%%%%%%%%%%%%%%%%%%%%%%%%%%%%%%%%%%%%%%%%%%%%%%%%%%%%%%%%%%%%%%%%%%%%%%%%%%%%%%%%%%%%%%%%%%%%%%%%%%%
%%%%%%%%%%%%%%%%%%%%%%%%%%%%%%%%%%%%%%%%%%%%%%%%%%%%%%%%%%%%%%%%%%%%%%%%%%%%%%%%%%%%%%%%%%%%%%%%%%%%%%%%%%%%%%%%%
%%%%%%%%%%%%%%%%%%%%%%%%%%%%%%%%%%%%%%%%%%%%%%%%%%%%%%%%%%%%%%%%%%%%%%%%%%%%%%%%%%%%%%%%%%%%%%%%%%%%%%%%%%%%%%%%%

\section{Úkol \#2}
\subsection{Pracovní úkol}
\begin{enumerate}
\item Určete objem láhve metodou vážení.
\item Určete objem ťěze láhve pomocí komprese plynu.
\item Oba výsledky vzájemně porovnejte.
\end{enumerate}


\subsection{Pomôcky}
Fľaška (nádoba), plynová byreta s porovnávacím ramenem, katetometr,
teploměr, barometr, digitálne váhy do $"5 kg"$.

\section{Teória}

\subsubsection{Metóda kompresie plynu}
Pre metódu kompresie plynu v našom prípade môžeme odvodiť vzťah
\eq{
V = \( V_2-V_1 \)\frac{p}{\Delta p} + V_2 - V_{100} \,, \lbl{R_6}
}, kde 
\eq{
\Delta p = \Delta h \rho g\,,
}
pričom $V_1$ je objem v byrete pri vyrovnaní tlakov, $\Delta h$ je rozdiel hladín, a $V_2$ výška hladiny po kompresií, $g$ je tiažové zrýchlenie a $p$ je atmosferický tlak.

V našom prípade $V_1="14 \%"$ a $V_{100} = "65.6 cm^3"$

\subsubsection{Metóda vážení}
Jednotkový objem vody je závisí na teplote $t$ v $\C$ podľa vzťahu
\eq{
V_v = 0.9998\cdot(1+0.00018 t) \frac{\jd{cm^3}}{\jd{g}}\,.\lbl{R_5}
} 
Potom objem metódou váženia určíme ako\eq{
V = \( m_n- m_p\) V_v \,, \lbl{R_4}
}
kde $m_n$ je hmotnosť nádoby s vodou, $m_p$ je hmotnosť prázdnej nádoby a $V_v$ je jednotkový objem zo vzťahu \ref{R_5}.

\subsection{Postup merania}
\subsubsection{Metóda kompresie plynu}
Najskôr bol povolením ventilu vyrovnaný tlak v byrete s atmosferickým tlakom. Bola odčítaná a zaznamenaná počiatočná hodnota $V_1$, odzvdušňovací ventil pol zatvorený.
Následne sa vertikálne pohlo s nádobou s vodou, počkalo sa na ustálenie hladín a boli odčítané hodnoty $\Delta h$ reprezentované $h_1$ a $h_2$ ,$V_2$.
Postup sa opakoval niekoľkokrát pre veľkú nádobu. Meraná nádoba bola vymenená za utesnenie a postup bol zopakovaný pre meranie objemu len hadičky.


\subsubsection{Metóda vážení}
\begin{enumerate}
\item Prázdna nádoba bola odvážená na digitálnych váhach 
\item Merané nádoba bola pookraj naplnená vodou a dôkladne osušení jej povrch
\item Naplnená nádoba bola opäť odvážená
\end{enumerate}


\subsection{Výsledky merania}
\subsubsection{Metóda kompresie plynu}
V tab. \ref{T_2} sú zaznamenané namerané hodnoty $V_2$ a $\Delta h$
z ktorých bola vypočítaná hodnota $V$. 
V prvej časti tabuľky sú hodnoty pre fľašku v druhej časti je hodnota pre samotnú hadičku.


\begin{table}[h]

\begin{center}
\begin{tabular}{| c | c | c | c |}
\hline
 \popi{h_1}{mm} & \popi{h_2}{mm} & \popi{V_2}{\%} & \popi{V}{cm^3} \\
\hline
$"127.90"$ & $"127.90"$ & $"14  "$ & $"-"$\\
$"102.78"$ & $" 87.76"$ & $"18  "$ & $"1738,07"$\\
$" 87.95"$ & $" 46.29"$ & $"22  "$ & $"1240,99"$\\
$"153.53"$ & $"186.76"$ & $"10  "$ & $" 750,90"$\\
$"143.42"$ & $"169.16"$ & $"11.5"$ & $" 595,49"$\\
\hline
$"164.84"$ & $"164,84"$ & $"7	0"$ & $"-"$\\
$"162.91"$ & $"126,60"$ & $"7.5"$ & $"65.59"$\\
$"161.35"$ & $"103,88"$ & $"8  "$ & $"65.59"$\\
$"165.90"$ & $"184,78"$ & $"3.5"$ & $"65.58"$\\
$"163.40"$ & $"123,39"$ & $"7.4"$ & $"65.59"$\\
\hline
\end{tabular}
\caption{Namerané hodnoty $h_1$, $h_2$ a $V_2$ a vypočítaný objem $V$, v prvej časti pre nádobu a v druhej pre hadičku.} \label{T_2}
\end{center}
\end{table}

Objem fľašky $V_f = "1071.15\pm 514.30 cm^3"$ po odčítaný objemu hadičky $V_h="65.59\pm0,01 cm^3"$ bol určený ako $V = "1005.55\pm514.30 cm^3"$.


\subsubsection{Metóda vážení}
Hmotnosť prázdnej suchej nádoby bola určená $m_p = "\(570\pm1\) g"$, jednotková hmotnosť vody pri $t="12.5 \C"$ bola určená podľa \ref{R_5} ako $V_v="1.003 \frac{cm^3}{g}"$
Hmotnosť po naplnení vodou bola určená $m_n = "\(1582\pm1\) g"$.
Podľa vzorca \ref{R_4} bol objem nádoby určený ako $V="\(1012\pm2\) cm^3"$.

\subsection{Diskusia}
Navzdory veľkej chybe merania u metódy kompresie plynu sa stredná hodnota zhoduje s hodnotou nameranou metódou váženia, 
ktorá je zároveň omoc presnejšia.
Hlavným problémom u metódy kompresie plynu bola postupná strata tlaku 
cez odvzdušňovací ventil a iné netesnosti, pokiaľ sa pomocou katetometru odmerala výška tak časť vzduchu uniklo a teda sa hladina sa výrazne pohla, 
Je zaujímavé, že pri tomto meraní štatistická chyba dosiahla úrovne $\sim 50\%$, čo z tohoto merania robí veľmi nepresné meranie.

\subsection{Záver}
Metódou váženia bol objem nádoby určený na  $V="\(1012\pm2\) cm^3"$.
Metódou kompresie plynu bol určený objem $V = "\(1005.55\pm514.30\) cm^3"$.


\begin{thebibliography}{1}
\bibitem{C_1}
Měření Poissonovy konstanty a dutých objemů [cit. 22.10.2017] Dostupné po prihlásení na: \url{https://praktikum.fjfi.cvut.cz/pluginfile.php/4350/mod_resource/content/4/Poisson_171006.pdf}
\bibitem{C_2}
Fyzikální a jiné konstanty [cit. 15.10.2017] Jiří Bureš: \url{http://www.converter.cz/prevody/konstanty.htm}
\end{thebibliography}

\end{document}






\section{Teória}
Newtonov gravitačný zákon hovorí, že gravitačná sila $F$ medzi dvoma telesami je priamo úmerná ich hmotnostiam $m_1$ a $m_2$ a nepriamo štvorci vzdialenosti $r$, teda
\eq{
F = G\frac{m_1 m_2}{r^2}\,,
}
kde $G$ je gravitačná konštanta, ktorá udáva veľkosť interakcie.

Zo vzorcov pre torzné kyvadlo a gravitačného zákona sa dá odvodiť vzťah
\eq{
G = \frac{\pi^2 b^2}{m_2 d} \frac{d^2 + \frac{2}{5}r^2}{\(1-\frac{b^3}{b^2+4d^2}^{3/2}\)} \frac{S}{T^2 L}\,, \\ \lbl{R_2}
} pričom 
\eq[m]{
r = "9,55 mm"\,, \\
d = "50,7 mm"\,, \\
b = "45 mm"\,, \\
S = S^2 - S^1 \,, \\
m_2 = "1,24 kg"\,,
}
kde $m_2$ je hmotnosť, každej z olovených gúľ, 
$S$ je rozdeľ nulových polôh kmitov v jednotlivých polohách $S^i$,
$b$ je vzdialenosť stredov malej a veľkej gule,
$d$ je vzdialenosť stredu malej gule od stredu osi otáčania torzného kyvadla,
a $r$ je polomer malej gule.
 
Vzťah pre tlmené harmonické kmity\eq{
f(t) = A\exp\(-\delta t\) \sin\( \frac{2*\pi}{T}t + \sigma\) + S^{1\(2\)} \lbl{R_1}\,.
}


\subsubsection{Spracovanie chýb merania}

Označme $\mean{t}$ aritmetický priemer nameraných hodnôt $t_i$, a $\Delta t$ hodnotu $\mean{t}-t$, pričom 
\eq{
\mean{t} = \frac{1}{n}\sum_{i=1}^n t_i \,, \lbl{SCH_1}
}  
a chybu aritmetického priemeru 
\eq{
  \sigma_0=\sqrt{\frac{\sum_{i=1}^n \(t_i - \mean{t}\)^2}{n\(n-1\)}}\,, \lbl{SCH_2}
}
pričom $n$ je počet meraní.

Majme veličina  $ u = f(x,y,z,\ldots)$, potom podľa zákou šírenia chýb platí
\eq{
\sigma_u = \sqrt{\(\pder{f}{x}\)^2_0 \sigma_x^2 +\(\pder{f}{y}\)^2_0 \sigma_y^2 + \(\pder{f}{z}\)^2_0 \sigma_z^2 + \ldots}\,, \lbl{SCH_3}
}
kde $\sigma_i$ je stredná chyba veličiny $i$ v bode $\(x_0,y_0,z_0\)$.




Vzorec \ref{R_2} sa dá pre naše účely prepísať ako \eq{
G = \const \frac{S}{T^2 L}\,,
} z čoho môže odvodiť pre chybu merania
\eq{
\frac{\Delta G}{\mid G\mid } = \frac{\Delta S}{\mid S\mid } +\frac{2\Delta T}{\mid T\mid } +\frac{\Delta L}{\mid L\mid } \,. \lbl{R_3}
}


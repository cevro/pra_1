%\documentclass[a4paper,10pt]{article}
\documentclass[10pt]{scrartcl}
%\usepackage[IL2]{fontenc}
\usepackage[utf8x]{inputenc}
\usepackage[czech]{babel}
\usepackage{listings}  
\usepackage{amsfonts,amsmath,amssymb,graphicx,color}
%\usepackage[total={17cm,27cm}, top=2cm, left=2cm, includefoot]{geometry}
%\usepackage{fancyhdr}
\usepackage{fkssugar}
\usepackage{hyperref}
\usepackage{mhchem}

%\usepackage{caption}

\input{../normalize.sty}
%  Nastaví autora, název, datum, skupinu měření apod. 
\newcommand{\FJFIInstitute}{FJFI~ČVUT~v~Praze}
%\newcommand{\Subject}{Základy fyzikálních měření}
\newcommand{\FJFISubject}{Fyzikální praktikum I}  %odkomentujte dle potřeby

%  Máte-li více spoluměřících než jednoho, vložte jen jejich příjmení
\newcommand{\FJFIAuthor}{Michal Červeňák}
\newcommand{\FJFICoauthor}{} 
\newcommand{\FJFIGroup}{Pondelok} %den, kdy chodíte na praktika, nikoli obor
\newcommand{\FJFICircle}{4} %číslo skupiny v rámci praktika, nikoli kruh

%  Tato část bude v každém protokolu jiná, nezapomeňte upravit!
\newcommand{\FJFITitle}{5. Poissonova konstanta a měření dutých objemů}
\newcommand{\FJFILabdate}{16.10.2017} %datum měření, nikoli datum odevzdání
\newcommand{\FJFIWorktime}{5 h} %jak dlouho vám trvalo vypracování protokolu


\begin{document}

\MakeFJFIHead{}

\section{Pracovní úkol}
\begin{enumerate}
\item DU: Zopakujte si výpočet chyb nepřímého měřen´ı, vysvětlete rozdíl mezi lineárním 
a kvadratickým zákonem hromadění chyb a jejich použití.
\item DU: Odvoďťe vztah pro výpočet relativní chyby měření $G$ a zamyslete se, jak
vypadá chyba periody kmitu $T$ a chyba rozdílu vzdáleností rovnovážných poloh $S$.
\item DU: V přípravě odvoďťe vztah pro výpočet relativní chyby měření $G$ a zamyslete
se, jak vypadá chyba periody kmitu $T$ a chyba rozdílu vzdáleností
rovnovážných poloh $S$.
\item Ve spolupráci s asistentem zkontrolujte, zda je torzní kyvadlo horizontálně vyrovnané.
($"20 min"$)
\item Pomocí torzního kyvadla změřte gravitační konstantu. Do protokolu přiložte graf naměřených
dat včetně odchylek a nafitované funkce. Diskutujte, zda bylo kyvadlo rotačně vyrovnané.


\end{enumerate}

\section{Postup merania}

\begin{enumerate}
\item Najskôr bola aparatúra horizontálne vyrovnaná a dôkladne skontrolované jej vyváženie.
\item Následne boli na držiaky osadené gule, a prítlačné  na doraz do prvej polohy.
\item Aparatúra sa odaretovala a nechala sa kmitať dokedy sa pri kmitoch neprestali guličky dotýkať od stien.
\item Podobu $\sim$ 3 kmitov sa zaznamenávala zmena výchylky laserového paprsku na čase, v našom prípade každých $"20 s"$.
\item Gule boli otočné do druhej polohy a opakovalo sa meranie pre 3 kmity.
\item Pomocou metru bola odmeraná vzdialenosť kyvadla od steny, na ktorú sa odčítavala vzdialenosť. 
\end{enumerate}

\section{Pomôcky}
Torzné kyvadlo, zemniacé káble, radiátor, laser, ochranné okuliare, podstavec pod laser,
stopky, meter provázek.

\section{Teória}
Newtonov gravitačný zákon hovorí, že gravitačná sila $F$ medzi dvoma telesami je priamo úmerná ich hmotnostiam $m_1$ a $m_2$ a nepriamo štvorci vzdialenosti $r$, teda
\eq{
F = G\frac{m_1 m_2}{r^2}\,,
}
kde $G$ je gravitačná konštanta, ktorá udáva veľkosť interakcie.

Zo vzorcov pre torzné kyvadlo a gravitačného zákona sa dá odvodiť vzťah
\eq{
G = \frac{\pi^2 b^2}{m_2 d} \frac{d^2 + \frac{2}{5}r^2}{\(1-\frac{b^3}{b^2+4d^2}^{3/2}\)} \frac{S}{T^2 L}\,, \\ \lbl{R_2}
} pričom 
\eq[m]{
r = "9,55 mm"\,, \\
d = "50,7 mm"\,, \\
b = "45 mm"\,, \\
S = S^2 - S^1 \,, \\
m_2 = "1,24 kg"\,,
}
kde $m_2$ je hmotnosť, každej z olovených gúľ, 
$S$ je rozdeľ nulových polôh kmitov v jednotlivých polohách $S^i$,
$b$ je vzdialenosť stredov malej a veľkej gule,
$d$ je vzdialenosť stredu malej gule od stredu osi otáčania torzného kyvadla,
a $r$ je polomer malej gule.
 
Vzťah pre tlmené harmonické kmity\eq{
f(t) = A\exp\(-\delta t\) \sin\( \frac{2*\pi}{T}t + \sigma\) + S^{1\(2\)} \lbl{R_1}\,.
}


\subsubsection{Spracovanie chýb merania}

Označme $\mean{t}$ aritmetický priemer nameraných hodnôt $t_i$, a $\Delta t$ hodnotu $\mean{t}-t$, pričom 
\eq{
\mean{t} = \frac{1}{n}\sum_{i=1}^n t_i \,, \lbl{SCH_1}
}  
a chybu aritmetického priemeru 
\eq{
  \sigma_0=\sqrt{\frac{\sum_{i=1}^n \(t_i - \mean{t}\)^2}{n\(n-1\)}}\,, \lbl{SCH_2}
}
pričom $n$ je počet meraní.

Majme veličina  $ u = f(x,y,z,\ldots)$, potom podľa zákou šírenia chýb platí
\eq{
\sigma_u = \sqrt{\(\pder{f}{x}\)^2_0 \sigma_x^2 +\(\pder{f}{y}\)^2_0 \sigma_y^2 + \(\pder{f}{z}\)^2_0 \sigma_z^2 + \ldots}\,, \lbl{SCH_3}
}
kde $\sigma_i$ je stredná chyba veličiny $i$ v bode $\(x_0,y_0,z_0\)$.




Vzorec \ref{R_2} sa dá pre naše účely prepísať ako \eq{
G = \const \frac{S}{T^2 L}\,,
} z čoho môže odvodiť pre chybu merania
\eq{
\frac{\Delta G}{\mid G\mid } = \frac{\Delta S}{\mid S\mid } +\frac{2\Delta T}{\mid T\mid } +\frac{\Delta L}{\mid L\mid } \,. \lbl{R_3}
}




\section{Výsledky merania}

Namerané dáta sú vynesené do grafu Obr. \ref{G_1}. 
Následne pre obe polohy gulí boli dáta preložené závislosťou \eqref{R_1}, a z nej zistené hodnoty \eq[m]{
S^1 = "\(115.6\pm 0.3\) cm" \,,\\
S^2 = "\(125.6\pm 0.1\) cm" \,,\\
T = "\(493.6\pm 0.4\) s" \,.\\
}
Preloženie je v Obr. \ref{G_2} a Obr. \ref{G_3}, a v Obr. \ref{G_4} sú vykreslené spolu s dátami a fitmi aj hodnoty $S^1$ a $S^2$.

Metrom bola odmeraná hodnota $L = "\(598\pm5\) cm"$, nepresnosť merania bola odhadnutá na $"5 cm"$ hlavne z dôvodu merania popri stene a postupného merania.
Nepresnosť určovania času bola stanovená na $"5 s"$.

Po dosadení do vzorca \eqref{R_2} a \eqref{R_3}, bola dopočítaná hodnota $G = "\(5.71\pm0.95\)\cdot10^{-11} m^3 \cdot kg^{-1}\cdot s^{-2}"$.

\begin{figure}
% GNUPLOT: LaTeX picture
\setlength{\unitlength}{0.240900pt}
\ifx\plotpoint\undefined\newsavebox{\plotpoint}\fi
\begin{picture}(1500,900)(0,0)
\sbox{\plotpoint}{\rule[-0.200pt]{0.400pt}{0.400pt}}%
\put(171.0,131.0){\rule[-0.200pt]{4.818pt}{0.400pt}}
\put(151,131){\makebox(0,0)[r]{-0.5}}
\put(1419.0,131.0){\rule[-0.200pt]{4.818pt}{0.400pt}}
\put(171.0,235.0){\rule[-0.200pt]{4.818pt}{0.400pt}}
\put(151,235){\makebox(0,0)[r]{ 0}}
\put(1419.0,235.0){\rule[-0.200pt]{4.818pt}{0.400pt}}
\put(171.0,339.0){\rule[-0.200pt]{4.818pt}{0.400pt}}
\put(151,339){\makebox(0,0)[r]{ 0.5}}
\put(1419.0,339.0){\rule[-0.200pt]{4.818pt}{0.400pt}}
\put(171.0,443.0){\rule[-0.200pt]{4.818pt}{0.400pt}}
\put(151,443){\makebox(0,0)[r]{ 1}}
\put(1419.0,443.0){\rule[-0.200pt]{4.818pt}{0.400pt}}
\put(171.0,547.0){\rule[-0.200pt]{4.818pt}{0.400pt}}
\put(151,547){\makebox(0,0)[r]{ 1.5}}
\put(1419.0,547.0){\rule[-0.200pt]{4.818pt}{0.400pt}}
\put(171.0,651.0){\rule[-0.200pt]{4.818pt}{0.400pt}}
\put(151,651){\makebox(0,0)[r]{ 2}}
\put(1419.0,651.0){\rule[-0.200pt]{4.818pt}{0.400pt}}
\put(171.0,755.0){\rule[-0.200pt]{4.818pt}{0.400pt}}
\put(151,755){\makebox(0,0)[r]{ 2.5}}
\put(1419.0,755.0){\rule[-0.200pt]{4.818pt}{0.400pt}}
\put(171.0,859.0){\rule[-0.200pt]{4.818pt}{0.400pt}}
\put(151,859){\makebox(0,0)[r]{ 3}}
\put(1419.0,859.0){\rule[-0.200pt]{4.818pt}{0.400pt}}
\put(171.0,131.0){\rule[-0.200pt]{0.400pt}{4.818pt}}
\put(171,90){\makebox(0,0){ 0.05}}
\put(171.0,839.0){\rule[-0.200pt]{0.400pt}{4.818pt}}
\put(330.0,131.0){\rule[-0.200pt]{0.400pt}{4.818pt}}
\put(330,90){\makebox(0,0){ 0.1}}
\put(330.0,839.0){\rule[-0.200pt]{0.400pt}{4.818pt}}
\put(488.0,131.0){\rule[-0.200pt]{0.400pt}{4.818pt}}
\put(488,90){\makebox(0,0){ 0.15}}
\put(488.0,839.0){\rule[-0.200pt]{0.400pt}{4.818pt}}
\put(647.0,131.0){\rule[-0.200pt]{0.400pt}{4.818pt}}
\put(647,90){\makebox(0,0){ 0.2}}
\put(647.0,839.0){\rule[-0.200pt]{0.400pt}{4.818pt}}
\put(805.0,131.0){\rule[-0.200pt]{0.400pt}{4.818pt}}
\put(805,90){\makebox(0,0){ 0.25}}
\put(805.0,839.0){\rule[-0.200pt]{0.400pt}{4.818pt}}
\put(964.0,131.0){\rule[-0.200pt]{0.400pt}{4.818pt}}
\put(964,90){\makebox(0,0){ 0.3}}
\put(964.0,839.0){\rule[-0.200pt]{0.400pt}{4.818pt}}
\put(1122.0,131.0){\rule[-0.200pt]{0.400pt}{4.818pt}}
\put(1122,90){\makebox(0,0){ 0.35}}
\put(1122.0,839.0){\rule[-0.200pt]{0.400pt}{4.818pt}}
\put(1281.0,131.0){\rule[-0.200pt]{0.400pt}{4.818pt}}
\put(1281,90){\makebox(0,0){ 0.4}}
\put(1281.0,839.0){\rule[-0.200pt]{0.400pt}{4.818pt}}
\put(1439.0,131.0){\rule[-0.200pt]{0.400pt}{4.818pt}}
\put(1439,90){\makebox(0,0){ 0.45}}
\put(1439.0,839.0){\rule[-0.200pt]{0.400pt}{4.818pt}}
\put(171.0,131.0){\rule[-0.200pt]{0.400pt}{175.375pt}}
\put(171.0,131.0){\rule[-0.200pt]{305.461pt}{0.400pt}}
\put(1439.0,131.0){\rule[-0.200pt]{0.400pt}{175.375pt}}
\put(171.0,859.0){\rule[-0.200pt]{305.461pt}{0.400pt}}
\put(30,495){\makebox(0,0){\popi{\kappa}{-}}}
\put(805,29){\makebox(0,0){\popi{t}{s}}}
\put(1279,819){\makebox(0,0)[r]{vypočítané hodnoty $\kappa$}}
\put(1299.0,819.0){\rule[-0.200pt]{24.090pt}{0.400pt}}
\put(1299.0,809.0){\rule[-0.200pt]{0.400pt}{4.818pt}}
\put(1399.0,809.0){\rule[-0.200pt]{0.400pt}{4.818pt}}
\put(501.0,470.0){\rule[-0.200pt]{0.400pt}{21.440pt}}
\put(491.0,470.0){\rule[-0.200pt]{4.818pt}{0.400pt}}
\put(491.0,559.0){\rule[-0.200pt]{4.818pt}{0.400pt}}
\put(577.0,440.0){\rule[-0.200pt]{0.400pt}{30.112pt}}
\put(567.0,440.0){\rule[-0.200pt]{4.818pt}{0.400pt}}
\put(567.0,565.0){\rule[-0.200pt]{4.818pt}{0.400pt}}
\put(437.0,427.0){\rule[-0.200pt]{0.400pt}{31.799pt}}
\put(427.0,427.0){\rule[-0.200pt]{4.818pt}{0.400pt}}
\put(427.0,559.0){\rule[-0.200pt]{4.818pt}{0.400pt}}
\put(513.0,458.0){\rule[-0.200pt]{0.400pt}{24.813pt}}
\put(503.0,458.0){\rule[-0.200pt]{4.818pt}{0.400pt}}
\put(503.0,561.0){\rule[-0.200pt]{4.818pt}{0.400pt}}
\put(466.0,467.0){\rule[-0.200pt]{0.400pt}{22.404pt}}
\put(456.0,467.0){\rule[-0.200pt]{4.818pt}{0.400pt}}
\put(456.0,560.0){\rule[-0.200pt]{4.818pt}{0.400pt}}
\put(421.0,462.0){\rule[-0.200pt]{0.400pt}{25.054pt}}
\put(411.0,462.0){\rule[-0.200pt]{4.818pt}{0.400pt}}
\put(411.0,566.0){\rule[-0.200pt]{4.818pt}{0.400pt}}
\put(542.0,480.0){\rule[-0.200pt]{0.400pt}{19.513pt}}
\put(532.0,480.0){\rule[-0.200pt]{4.818pt}{0.400pt}}
\put(532.0,561.0){\rule[-0.200pt]{4.818pt}{0.400pt}}
\put(488.0,463.0){\rule[-0.200pt]{0.400pt}{22.163pt}}
\put(478.0,463.0){\rule[-0.200pt]{4.818pt}{0.400pt}}
\put(478.0,555.0){\rule[-0.200pt]{4.818pt}{0.400pt}}
\put(428.0,462.0){\rule[-0.200pt]{0.400pt}{25.054pt}}
\put(418.0,462.0){\rule[-0.200pt]{4.818pt}{0.400pt}}
\put(418.0,566.0){\rule[-0.200pt]{4.818pt}{0.400pt}}
\put(1363.0,441.0){\rule[-0.200pt]{0.400pt}{27.944pt}}
\put(1353.0,441.0){\rule[-0.200pt]{4.818pt}{0.400pt}}
\put(1353.0,557.0){\rule[-0.200pt]{4.818pt}{0.400pt}}
\put(583.0,447.0){\rule[-0.200pt]{0.400pt}{28.185pt}}
\put(573.0,447.0){\rule[-0.200pt]{4.818pt}{0.400pt}}
\put(573.0,564.0){\rule[-0.200pt]{4.818pt}{0.400pt}}
\put(881.0,402.0){\rule[-0.200pt]{0.400pt}{46.253pt}}
\put(871.0,402.0){\rule[-0.200pt]{4.818pt}{0.400pt}}
\put(871.0,594.0){\rule[-0.200pt]{4.818pt}{0.400pt}}
\put(1135.0,445.0){\rule[-0.200pt]{0.400pt}{27.944pt}}
\put(1125.0,445.0){\rule[-0.200pt]{4.818pt}{0.400pt}}
\put(1125.0,561.0){\rule[-0.200pt]{4.818pt}{0.400pt}}
\put(466.0,455.0){\rule[-0.200pt]{0.400pt}{28.426pt}}
\put(456.0,455.0){\rule[-0.200pt]{4.818pt}{0.400pt}}
\put(456.0,573.0){\rule[-0.200pt]{4.818pt}{0.400pt}}
\put(599.0,430.0){\rule[-0.200pt]{0.400pt}{38.544pt}}
\put(589.0,430.0){\rule[-0.200pt]{4.818pt}{0.400pt}}
\put(589.0,590.0){\rule[-0.200pt]{4.818pt}{0.400pt}}
\put(358.0,434.0){\rule[-0.200pt]{0.400pt}{38.785pt}}
\put(348.0,434.0){\rule[-0.200pt]{4.818pt}{0.400pt}}
\put(348.0,595.0){\rule[-0.200pt]{4.818pt}{0.400pt}}
\put(314.0,424.0){\rule[-0.200pt]{0.400pt}{38.303pt}}
\put(304.0,424.0){\rule[-0.200pt]{4.818pt}{0.400pt}}
\put(304.0,583.0){\rule[-0.200pt]{4.818pt}{0.400pt}}
\put(320.0,459.0){\rule[-0.200pt]{0.400pt}{25.054pt}}
\put(310.0,459.0){\rule[-0.200pt]{4.818pt}{0.400pt}}
\put(310.0,563.0){\rule[-0.200pt]{4.818pt}{0.400pt}}
\put(498.0,198.0){\rule[-0.200pt]{0.400pt}{136.831pt}}
\put(488.0,198.0){\rule[-0.200pt]{4.818pt}{0.400pt}}
\put(488.0,766.0){\rule[-0.200pt]{4.818pt}{0.400pt}}
\put(1303.0,449.0){\rule[-0.200pt]{0.400pt}{24.572pt}}
\put(1293.0,449.0){\rule[-0.200pt]{4.818pt}{0.400pt}}
\put(501,515){\makebox(0,0){$+$}}
\put(577,502){\makebox(0,0){$+$}}
\put(437,493){\makebox(0,0){$+$}}
\put(513,510){\makebox(0,0){$+$}}
\put(466,514){\makebox(0,0){$+$}}
\put(421,514){\makebox(0,0){$+$}}
\put(542,521){\makebox(0,0){$+$}}
\put(488,509){\makebox(0,0){$+$}}
\put(428,514){\makebox(0,0){$+$}}
\put(1363,499){\makebox(0,0){$+$}}
\put(583,505){\makebox(0,0){$+$}}
\put(881,498){\makebox(0,0){$+$}}
\put(1135,503){\makebox(0,0){$+$}}
\put(466,514){\makebox(0,0){$+$}}
\put(599,510){\makebox(0,0){$+$}}
\put(358,515){\makebox(0,0){$+$}}
\put(314,503){\makebox(0,0){$+$}}
\put(320,511){\makebox(0,0){$+$}}
\put(498,482){\makebox(0,0){$+$}}
\put(1303,500){\makebox(0,0){$+$}}
\put(1349,819){\makebox(0,0){$+$}}
\put(1293.0,551.0){\rule[-0.200pt]{4.818pt}{0.400pt}}
\put(1279,778){\makebox(0,0)[r]{fit $f(t) = \(-0.15\pm0.10\)t + \(1.335\pm0.021\)$}}
\multiput(1299,778)(20.756,0.000){5}{\usebox{\plotpoint}}
\put(1399,778){\usebox{\plotpoint}}
\put(314,510){\usebox{\plotpoint}}
\put(314.00,510.00){\usebox{\plotpoint}}
\put(334.71,509.00){\usebox{\plotpoint}}
\put(355.46,509.00){\usebox{\plotpoint}}
\put(376.22,509.00){\usebox{\plotpoint}}
\put(396.97,509.00){\usebox{\plotpoint}}
\put(417.69,508.21){\usebox{\plotpoint}}
\put(438.44,508.00){\usebox{\plotpoint}}
\put(459.19,508.00){\usebox{\plotpoint}}
\put(479.95,508.00){\usebox{\plotpoint}}
\put(500.70,508.00){\usebox{\plotpoint}}
\put(521.43,507.42){\usebox{\plotpoint}}
\put(542.17,507.00){\usebox{\plotpoint}}
\put(562.93,507.00){\usebox{\plotpoint}}
\put(583.68,507.00){\usebox{\plotpoint}}
\put(604.44,507.00){\usebox{\plotpoint}}
\put(625.15,506.00){\usebox{\plotpoint}}
\put(645.90,506.00){\usebox{\plotpoint}}
\put(666.66,506.00){\usebox{\plotpoint}}
\put(687.41,506.00){\usebox{\plotpoint}}
\put(708.16,505.78){\usebox{\plotpoint}}
\put(728.87,505.00){\usebox{\plotpoint}}
\put(749.63,505.00){\usebox{\plotpoint}}
\put(770.39,505.00){\usebox{\plotpoint}}
\put(791.14,505.00){\usebox{\plotpoint}}
\put(811.85,504.01){\usebox{\plotpoint}}
\put(832.61,504.00){\usebox{\plotpoint}}
\put(853.36,504.00){\usebox{\plotpoint}}
\put(874.12,504.00){\usebox{\plotpoint}}
\put(894.87,504.00){\usebox{\plotpoint}}
\put(915.58,503.00){\usebox{\plotpoint}}
\put(936.33,503.00){\usebox{\plotpoint}}
\put(957.09,503.00){\usebox{\plotpoint}}
\put(977.84,503.00){\usebox{\plotpoint}}
\put(998.57,502.40){\usebox{\plotpoint}}
\put(1019.31,502.00){\usebox{\plotpoint}}
\put(1040.07,502.00){\usebox{\plotpoint}}
\put(1060.82,502.00){\usebox{\plotpoint}}
\put(1081.58,502.00){\usebox{\plotpoint}}
\put(1102.29,501.00){\usebox{\plotpoint}}
\put(1123.04,501.00){\usebox{\plotpoint}}
\put(1143.80,501.00){\usebox{\plotpoint}}
\put(1164.55,501.00){\usebox{\plotpoint}}
\put(1185.31,501.00){\usebox{\plotpoint}}
\put(1206.02,500.00){\usebox{\plotpoint}}
\put(1226.77,500.00){\usebox{\plotpoint}}
\put(1247.53,500.00){\usebox{\plotpoint}}
\put(1268.29,500.00){\usebox{\plotpoint}}
\put(1289.04,500.00){\usebox{\plotpoint}}
\put(1309.75,499.00){\usebox{\plotpoint}}
\put(1330.50,499.00){\usebox{\plotpoint}}
\put(1351.26,499.00){\usebox{\plotpoint}}
\put(1363,499){\usebox{\plotpoint}}
\put(171.0,131.0){\rule[-0.200pt]{0.400pt}{175.375pt}}
\put(171.0,131.0){\rule[-0.200pt]{305.461pt}{0.400pt}}
\put(1439.0,131.0){\rule[-0.200pt]{0.400pt}{175.375pt}}
\put(171.0,859.0){\rule[-0.200pt]{305.461pt}{0.400pt}}
\end{picture}

\caption{Namereané hodnoty polohy $l$ na čase $t$}  \label{G_1}

\end{figure}

\begin{figure}
% GNUPLOT: LaTeX picture
\setlength{\unitlength}{0.240900pt}
\ifx\plotpoint\undefined\newsavebox{\plotpoint}\fi
\begin{picture}(1500,900)(0,0)
\sbox{\plotpoint}{\rule[-0.200pt]{0.400pt}{0.400pt}}%
\put(171.0,131.0){\rule[-0.200pt]{4.818pt}{0.400pt}}
\put(151,131){\makebox(0,0)[r]{ 0}}
\put(1419.0,131.0){\rule[-0.200pt]{4.818pt}{0.400pt}}
\put(171.0,252.0){\rule[-0.200pt]{4.818pt}{0.400pt}}
\put(151,252){\makebox(0,0)[r]{ 20}}
\put(1419.0,252.0){\rule[-0.200pt]{4.818pt}{0.400pt}}
\put(171.0,374.0){\rule[-0.200pt]{4.818pt}{0.400pt}}
\put(151,374){\makebox(0,0)[r]{ 40}}
\put(1419.0,374.0){\rule[-0.200pt]{4.818pt}{0.400pt}}
\put(171.0,495.0){\rule[-0.200pt]{4.818pt}{0.400pt}}
\put(151,495){\makebox(0,0)[r]{ 60}}
\put(1419.0,495.0){\rule[-0.200pt]{4.818pt}{0.400pt}}
\put(171.0,616.0){\rule[-0.200pt]{4.818pt}{0.400pt}}
\put(151,616){\makebox(0,0)[r]{ 80}}
\put(1419.0,616.0){\rule[-0.200pt]{4.818pt}{0.400pt}}
\put(171.0,738.0){\rule[-0.200pt]{4.818pt}{0.400pt}}
\put(151,738){\makebox(0,0)[r]{ 100}}
\put(1419.0,738.0){\rule[-0.200pt]{4.818pt}{0.400pt}}
\put(171.0,859.0){\rule[-0.200pt]{4.818pt}{0.400pt}}
\put(151,859){\makebox(0,0)[r]{ 120}}
\put(1419.0,859.0){\rule[-0.200pt]{4.818pt}{0.400pt}}
\put(295.0,131.0){\rule[-0.200pt]{0.400pt}{4.818pt}}
\put(295,90){\makebox(0,0){ 800}}
\put(295.0,839.0){\rule[-0.200pt]{0.400pt}{4.818pt}}
\put(449.0,131.0){\rule[-0.200pt]{0.400pt}{4.818pt}}
\put(449,90){\makebox(0,0){ 1000}}
\put(449.0,839.0){\rule[-0.200pt]{0.400pt}{4.818pt}}
\put(604.0,131.0){\rule[-0.200pt]{0.400pt}{4.818pt}}
\put(604,90){\makebox(0,0){ 1200}}
\put(604.0,839.0){\rule[-0.200pt]{0.400pt}{4.818pt}}
\put(759.0,131.0){\rule[-0.200pt]{0.400pt}{4.818pt}}
\put(759,90){\makebox(0,0){ 1400}}
\put(759.0,839.0){\rule[-0.200pt]{0.400pt}{4.818pt}}
\put(913.0,131.0){\rule[-0.200pt]{0.400pt}{4.818pt}}
\put(913,90){\makebox(0,0){ 1600}}
\put(913.0,839.0){\rule[-0.200pt]{0.400pt}{4.818pt}}
\put(1068.0,131.0){\rule[-0.200pt]{0.400pt}{4.818pt}}
\put(1068,90){\makebox(0,0){ 1800}}
\put(1068.0,839.0){\rule[-0.200pt]{0.400pt}{4.818pt}}
\put(1223.0,131.0){\rule[-0.200pt]{0.400pt}{4.818pt}}
\put(1223,90){\makebox(0,0){ 2000}}
\put(1223.0,839.0){\rule[-0.200pt]{0.400pt}{4.818pt}}
\put(1377.0,131.0){\rule[-0.200pt]{0.400pt}{4.818pt}}
\put(1377,90){\makebox(0,0){ 2200}}
\put(1377.0,839.0){\rule[-0.200pt]{0.400pt}{4.818pt}}
\put(171.0,131.0){\rule[-0.200pt]{0.400pt}{175.375pt}}
\put(171.0,131.0){\rule[-0.200pt]{305.461pt}{0.400pt}}
\put(1439.0,131.0){\rule[-0.200pt]{0.400pt}{175.375pt}}
\put(171.0,859.0){\rule[-0.200pt]{305.461pt}{0.400pt}}
\put(30,495){\makebox(0,0){\popi{l}{cm}}}
\put(805,29){\makebox(0,0){\popi{t}{s}}}
\put(1279,819){\makebox(0,0)[r]{Fit v prvej polohe}}
\put(1299.0,819.0){\rule[-0.200pt]{24.090pt}{0.400pt}}
\put(171,131){\usebox{\plotpoint}}
\multiput(171.58,131.00)(0.493,20.717){23}{\rule{0.119pt}{16.192pt}}
\multiput(170.17,131.00)(13.000,489.392){2}{\rule{0.400pt}{8.096pt}}
\multiput(184.58,654.00)(0.493,1.567){23}{\rule{0.119pt}{1.331pt}}
\multiput(183.17,654.00)(13.000,37.238){2}{\rule{0.400pt}{0.665pt}}
\multiput(197.58,694.00)(0.492,1.358){21}{\rule{0.119pt}{1.167pt}}
\multiput(196.17,694.00)(12.000,29.579){2}{\rule{0.400pt}{0.583pt}}
\multiput(209.58,726.00)(0.493,0.933){23}{\rule{0.119pt}{0.838pt}}
\multiput(208.17,726.00)(13.000,22.260){2}{\rule{0.400pt}{0.419pt}}
\multiput(222.58,750.00)(0.493,0.536){23}{\rule{0.119pt}{0.531pt}}
\multiput(221.17,750.00)(13.000,12.898){2}{\rule{0.400pt}{0.265pt}}
\multiput(235.00,764.59)(1.378,0.477){7}{\rule{1.140pt}{0.115pt}}
\multiput(235.00,763.17)(10.634,5.000){2}{\rule{0.570pt}{0.400pt}}
\multiput(248.00,767.93)(1.378,-0.477){7}{\rule{1.140pt}{0.115pt}}
\multiput(248.00,768.17)(10.634,-5.000){2}{\rule{0.570pt}{0.400pt}}
\multiput(261.58,761.51)(0.492,-0.625){21}{\rule{0.119pt}{0.600pt}}
\multiput(260.17,762.75)(12.000,-13.755){2}{\rule{0.400pt}{0.300pt}}
\multiput(273.58,745.65)(0.493,-0.893){23}{\rule{0.119pt}{0.808pt}}
\multiput(272.17,747.32)(13.000,-21.324){2}{\rule{0.400pt}{0.404pt}}
\multiput(286.58,721.63)(0.493,-1.210){23}{\rule{0.119pt}{1.054pt}}
\multiput(285.17,723.81)(13.000,-28.813){2}{\rule{0.400pt}{0.527pt}}
\multiput(299.58,689.86)(0.493,-1.448){23}{\rule{0.119pt}{1.238pt}}
\multiput(298.17,692.43)(13.000,-34.430){2}{\rule{0.400pt}{0.619pt}}
\multiput(312.58,652.35)(0.493,-1.607){23}{\rule{0.119pt}{1.362pt}}
\multiput(311.17,655.17)(13.000,-38.174){2}{\rule{0.400pt}{0.681pt}}
\multiput(325.58,610.96)(0.493,-1.726){23}{\rule{0.119pt}{1.454pt}}
\multiput(324.17,613.98)(13.000,-40.982){2}{\rule{0.400pt}{0.727pt}}
\multiput(338.58,566.50)(0.492,-1.875){21}{\rule{0.119pt}{1.567pt}}
\multiput(337.17,569.75)(12.000,-40.748){2}{\rule{0.400pt}{0.783pt}}
\multiput(350.58,522.96)(0.493,-1.726){23}{\rule{0.119pt}{1.454pt}}
\multiput(349.17,525.98)(13.000,-40.982){2}{\rule{0.400pt}{0.727pt}}
\multiput(363.58,479.48)(0.493,-1.567){23}{\rule{0.119pt}{1.331pt}}
\multiput(362.17,482.24)(13.000,-37.238){2}{\rule{0.400pt}{0.665pt}}
\multiput(376.58,440.11)(0.493,-1.369){23}{\rule{0.119pt}{1.177pt}}
\multiput(375.17,442.56)(13.000,-32.557){2}{\rule{0.400pt}{0.588pt}}
\multiput(389.58,405.88)(0.493,-1.131){23}{\rule{0.119pt}{0.992pt}}
\multiput(388.17,407.94)(13.000,-26.940){2}{\rule{0.400pt}{0.496pt}}
\multiput(402.58,377.68)(0.492,-0.884){21}{\rule{0.119pt}{0.800pt}}
\multiput(401.17,379.34)(12.000,-19.340){2}{\rule{0.400pt}{0.400pt}}
\multiput(414.00,358.92)(0.497,-0.493){23}{\rule{0.500pt}{0.119pt}}
\multiput(414.00,359.17)(11.962,-13.000){2}{\rule{0.250pt}{0.400pt}}
\multiput(427.00,345.93)(1.378,-0.477){7}{\rule{1.140pt}{0.115pt}}
\multiput(427.00,346.17)(10.634,-5.000){2}{\rule{0.570pt}{0.400pt}}
\multiput(440.00,342.59)(1.378,0.477){7}{\rule{1.140pt}{0.115pt}}
\multiput(440.00,341.17)(10.634,5.000){2}{\rule{0.570pt}{0.400pt}}
\multiput(453.00,347.58)(0.497,0.493){23}{\rule{0.500pt}{0.119pt}}
\multiput(453.00,346.17)(11.962,13.000){2}{\rule{0.250pt}{0.400pt}}
\multiput(466.58,360.00)(0.492,0.841){21}{\rule{0.119pt}{0.767pt}}
\multiput(465.17,360.00)(12.000,18.409){2}{\rule{0.400pt}{0.383pt}}
\multiput(478.58,380.00)(0.493,1.052){23}{\rule{0.119pt}{0.931pt}}
\multiput(477.17,380.00)(13.000,25.068){2}{\rule{0.400pt}{0.465pt}}
\multiput(491.58,407.00)(0.493,1.290){23}{\rule{0.119pt}{1.115pt}}
\multiput(490.17,407.00)(13.000,30.685){2}{\rule{0.400pt}{0.558pt}}
\multiput(504.58,440.00)(0.493,1.408){23}{\rule{0.119pt}{1.208pt}}
\multiput(503.17,440.00)(13.000,33.493){2}{\rule{0.400pt}{0.604pt}}
\multiput(517.58,476.00)(0.493,1.527){23}{\rule{0.119pt}{1.300pt}}
\multiput(516.17,476.00)(13.000,36.302){2}{\rule{0.400pt}{0.650pt}}
\multiput(530.58,515.00)(0.492,1.703){21}{\rule{0.119pt}{1.433pt}}
\multiput(529.17,515.00)(12.000,37.025){2}{\rule{0.400pt}{0.717pt}}
\multiput(542.58,555.00)(0.493,1.488){23}{\rule{0.119pt}{1.269pt}}
\multiput(541.17,555.00)(13.000,35.366){2}{\rule{0.400pt}{0.635pt}}
\multiput(555.58,593.00)(0.493,1.408){23}{\rule{0.119pt}{1.208pt}}
\multiput(554.17,593.00)(13.000,33.493){2}{\rule{0.400pt}{0.604pt}}
\multiput(568.58,629.00)(0.493,1.210){23}{\rule{0.119pt}{1.054pt}}
\multiput(567.17,629.00)(13.000,28.813){2}{\rule{0.400pt}{0.527pt}}
\multiput(581.58,660.00)(0.493,1.012){23}{\rule{0.119pt}{0.900pt}}
\multiput(580.17,660.00)(13.000,24.132){2}{\rule{0.400pt}{0.450pt}}
\multiput(594.58,686.00)(0.492,0.798){21}{\rule{0.119pt}{0.733pt}}
\multiput(593.17,686.00)(12.000,17.478){2}{\rule{0.400pt}{0.367pt}}
\multiput(606.00,705.58)(0.539,0.492){21}{\rule{0.533pt}{0.119pt}}
\multiput(606.00,704.17)(11.893,12.000){2}{\rule{0.267pt}{0.400pt}}
\multiput(619.00,717.60)(1.797,0.468){5}{\rule{1.400pt}{0.113pt}}
\multiput(619.00,716.17)(10.094,4.000){2}{\rule{0.700pt}{0.400pt}}
\multiput(632.00,719.94)(1.797,-0.468){5}{\rule{1.400pt}{0.113pt}}
\multiput(632.00,720.17)(10.094,-4.000){2}{\rule{0.700pt}{0.400pt}}
\multiput(645.00,715.92)(0.590,-0.492){19}{\rule{0.573pt}{0.118pt}}
\multiput(645.00,716.17)(11.811,-11.000){2}{\rule{0.286pt}{0.400pt}}
\multiput(658.58,703.29)(0.493,-0.695){23}{\rule{0.119pt}{0.654pt}}
\multiput(657.17,704.64)(13.000,-16.643){2}{\rule{0.400pt}{0.327pt}}
\multiput(671.58,684.26)(0.492,-1.013){21}{\rule{0.119pt}{0.900pt}}
\multiput(670.17,686.13)(12.000,-22.132){2}{\rule{0.400pt}{0.450pt}}
\multiput(683.58,659.88)(0.493,-1.131){23}{\rule{0.119pt}{0.992pt}}
\multiput(682.17,661.94)(13.000,-26.940){2}{\rule{0.400pt}{0.496pt}}
\multiput(696.58,630.50)(0.493,-1.250){23}{\rule{0.119pt}{1.085pt}}
\multiput(695.17,632.75)(13.000,-29.749){2}{\rule{0.400pt}{0.542pt}}
\multiput(709.58,598.24)(0.493,-1.329){23}{\rule{0.119pt}{1.146pt}}
\multiput(708.17,600.62)(13.000,-31.621){2}{\rule{0.400pt}{0.573pt}}
\multiput(722.58,564.11)(0.493,-1.369){23}{\rule{0.119pt}{1.177pt}}
\multiput(721.17,566.56)(13.000,-32.557){2}{\rule{0.400pt}{0.588pt}}
\multiput(735.58,528.88)(0.492,-1.444){21}{\rule{0.119pt}{1.233pt}}
\multiput(734.17,531.44)(12.000,-31.440){2}{\rule{0.400pt}{0.617pt}}
\multiput(747.58,495.50)(0.493,-1.250){23}{\rule{0.119pt}{1.085pt}}
\multiput(746.17,497.75)(13.000,-29.749){2}{\rule{0.400pt}{0.542pt}}
\multiput(760.58,464.01)(0.493,-1.091){23}{\rule{0.119pt}{0.962pt}}
\multiput(759.17,466.00)(13.000,-26.004){2}{\rule{0.400pt}{0.481pt}}
\multiput(773.58,436.65)(0.493,-0.893){23}{\rule{0.119pt}{0.808pt}}
\multiput(772.17,438.32)(13.000,-21.324){2}{\rule{0.400pt}{0.404pt}}
\multiput(786.58,414.41)(0.493,-0.655){23}{\rule{0.119pt}{0.623pt}}
\multiput(785.17,415.71)(13.000,-15.707){2}{\rule{0.400pt}{0.312pt}}
\multiput(799.00,398.92)(0.543,-0.492){19}{\rule{0.536pt}{0.118pt}}
\multiput(799.00,399.17)(10.887,-11.000){2}{\rule{0.268pt}{0.400pt}}
\multiput(811.00,387.94)(1.797,-0.468){5}{\rule{1.400pt}{0.113pt}}
\multiput(811.00,388.17)(10.094,-4.000){2}{\rule{0.700pt}{0.400pt}}
\multiput(824.00,385.61)(2.695,0.447){3}{\rule{1.833pt}{0.108pt}}
\multiput(824.00,384.17)(9.195,3.000){2}{\rule{0.917pt}{0.400pt}}
\multiput(837.00,388.58)(0.652,0.491){17}{\rule{0.620pt}{0.118pt}}
\multiput(837.00,387.17)(11.713,10.000){2}{\rule{0.310pt}{0.400pt}}
\multiput(850.58,398.00)(0.493,0.616){23}{\rule{0.119pt}{0.592pt}}
\multiput(849.17,398.00)(13.000,14.771){2}{\rule{0.400pt}{0.296pt}}
\multiput(863.58,414.00)(0.492,0.884){21}{\rule{0.119pt}{0.800pt}}
\multiput(862.17,414.00)(12.000,19.340){2}{\rule{0.400pt}{0.400pt}}
\multiput(875.58,435.00)(0.493,0.972){23}{\rule{0.119pt}{0.869pt}}
\multiput(874.17,435.00)(13.000,23.196){2}{\rule{0.400pt}{0.435pt}}
\multiput(888.58,460.00)(0.493,1.131){23}{\rule{0.119pt}{0.992pt}}
\multiput(887.17,460.00)(13.000,26.940){2}{\rule{0.400pt}{0.496pt}}
\multiput(901.58,489.00)(0.493,1.171){23}{\rule{0.119pt}{1.023pt}}
\multiput(900.17,489.00)(13.000,27.877){2}{\rule{0.400pt}{0.512pt}}
\multiput(914.58,519.00)(0.493,1.210){23}{\rule{0.119pt}{1.054pt}}
\multiput(913.17,519.00)(13.000,28.813){2}{\rule{0.400pt}{0.527pt}}
\multiput(927.58,550.00)(0.492,1.315){21}{\rule{0.119pt}{1.133pt}}
\multiput(926.17,550.00)(12.000,28.648){2}{\rule{0.400pt}{0.567pt}}
\multiput(939.58,581.00)(0.493,1.091){23}{\rule{0.119pt}{0.962pt}}
\multiput(938.17,581.00)(13.000,26.004){2}{\rule{0.400pt}{0.481pt}}
\multiput(952.58,609.00)(0.493,0.972){23}{\rule{0.119pt}{0.869pt}}
\multiput(951.17,609.00)(13.000,23.196){2}{\rule{0.400pt}{0.435pt}}
\multiput(965.58,634.00)(0.493,0.774){23}{\rule{0.119pt}{0.715pt}}
\multiput(964.17,634.00)(13.000,18.515){2}{\rule{0.400pt}{0.358pt}}
\multiput(978.58,654.00)(0.493,0.616){23}{\rule{0.119pt}{0.592pt}}
\multiput(977.17,654.00)(13.000,14.771){2}{\rule{0.400pt}{0.296pt}}
\multiput(991.00,670.59)(0.728,0.489){15}{\rule{0.678pt}{0.118pt}}
\multiput(991.00,669.17)(11.593,9.000){2}{\rule{0.339pt}{0.400pt}}
\multiput(1004.00,679.60)(1.651,0.468){5}{\rule{1.300pt}{0.113pt}}
\multiput(1004.00,678.17)(9.302,4.000){2}{\rule{0.650pt}{0.400pt}}
\multiput(1016.00,681.95)(2.695,-0.447){3}{\rule{1.833pt}{0.108pt}}
\multiput(1016.00,682.17)(9.195,-3.000){2}{\rule{0.917pt}{0.400pt}}
\multiput(1029.00,678.93)(0.824,-0.488){13}{\rule{0.750pt}{0.117pt}}
\multiput(1029.00,679.17)(11.443,-8.000){2}{\rule{0.375pt}{0.400pt}}
\multiput(1042.58,669.80)(0.493,-0.536){23}{\rule{0.119pt}{0.531pt}}
\multiput(1041.17,670.90)(13.000,-12.898){2}{\rule{0.400pt}{0.265pt}}
\multiput(1055.58,655.29)(0.493,-0.695){23}{\rule{0.119pt}{0.654pt}}
\multiput(1054.17,656.64)(13.000,-16.643){2}{\rule{0.400pt}{0.327pt}}
\multiput(1068.58,636.40)(0.492,-0.970){21}{\rule{0.119pt}{0.867pt}}
\multiput(1067.17,638.20)(12.000,-21.201){2}{\rule{0.400pt}{0.433pt}}
\multiput(1080.58,613.39)(0.493,-0.972){23}{\rule{0.119pt}{0.869pt}}
\multiput(1079.17,615.20)(13.000,-23.196){2}{\rule{0.400pt}{0.435pt}}
\multiput(1093.58,588.14)(0.493,-1.052){23}{\rule{0.119pt}{0.931pt}}
\multiput(1092.17,590.07)(13.000,-25.068){2}{\rule{0.400pt}{0.465pt}}
\multiput(1106.58,561.01)(0.493,-1.091){23}{\rule{0.119pt}{0.962pt}}
\multiput(1105.17,563.00)(13.000,-26.004){2}{\rule{0.400pt}{0.481pt}}
\multiput(1119.58,533.14)(0.493,-1.052){23}{\rule{0.119pt}{0.931pt}}
\multiput(1118.17,535.07)(13.000,-25.068){2}{\rule{0.400pt}{0.465pt}}
\multiput(1132.58,506.13)(0.492,-1.056){21}{\rule{0.119pt}{0.933pt}}
\multiput(1131.17,508.06)(12.000,-23.063){2}{\rule{0.400pt}{0.467pt}}
\multiput(1144.58,481.77)(0.493,-0.853){23}{\rule{0.119pt}{0.777pt}}
\multiput(1143.17,483.39)(13.000,-20.387){2}{\rule{0.400pt}{0.388pt}}
\multiput(1157.58,460.29)(0.493,-0.695){23}{\rule{0.119pt}{0.654pt}}
\multiput(1156.17,461.64)(13.000,-16.643){2}{\rule{0.400pt}{0.327pt}}
\multiput(1170.58,442.80)(0.493,-0.536){23}{\rule{0.119pt}{0.531pt}}
\multiput(1169.17,443.90)(13.000,-12.898){2}{\rule{0.400pt}{0.265pt}}
\multiput(1183.00,429.93)(0.728,-0.489){15}{\rule{0.678pt}{0.118pt}}
\multiput(1183.00,430.17)(11.593,-9.000){2}{\rule{0.339pt}{0.400pt}}
\multiput(1196.00,420.95)(2.472,-0.447){3}{\rule{1.700pt}{0.108pt}}
\multiput(1196.00,421.17)(8.472,-3.000){2}{\rule{0.850pt}{0.400pt}}
\put(1208,419.17){\rule{2.700pt}{0.400pt}}
\multiput(1208.00,418.17)(7.396,2.000){2}{\rule{1.350pt}{0.400pt}}
\multiput(1221.00,421.59)(0.950,0.485){11}{\rule{0.843pt}{0.117pt}}
\multiput(1221.00,420.17)(11.251,7.000){2}{\rule{0.421pt}{0.400pt}}
\multiput(1234.00,428.58)(0.497,0.493){23}{\rule{0.500pt}{0.119pt}}
\multiput(1234.00,427.17)(11.962,13.000){2}{\rule{0.250pt}{0.400pt}}
\multiput(1247.58,441.00)(0.493,0.616){23}{\rule{0.119pt}{0.592pt}}
\multiput(1246.17,441.00)(13.000,14.771){2}{\rule{0.400pt}{0.296pt}}
\multiput(1260.58,457.00)(0.492,0.841){21}{\rule{0.119pt}{0.767pt}}
\multiput(1259.17,457.00)(12.000,18.409){2}{\rule{0.400pt}{0.383pt}}
\multiput(1272.58,477.00)(0.493,0.853){23}{\rule{0.119pt}{0.777pt}}
\multiput(1271.17,477.00)(13.000,20.387){2}{\rule{0.400pt}{0.388pt}}
\multiput(1285.58,499.00)(0.493,0.933){23}{\rule{0.119pt}{0.838pt}}
\multiput(1284.17,499.00)(13.000,22.260){2}{\rule{0.400pt}{0.419pt}}
\multiput(1298.58,523.00)(0.493,0.933){23}{\rule{0.119pt}{0.838pt}}
\multiput(1297.17,523.00)(13.000,22.260){2}{\rule{0.400pt}{0.419pt}}
\multiput(1311.58,547.00)(0.493,0.933){23}{\rule{0.119pt}{0.838pt}}
\multiput(1310.17,547.00)(13.000,22.260){2}{\rule{0.400pt}{0.419pt}}
\multiput(1324.58,571.00)(0.493,0.893){23}{\rule{0.119pt}{0.808pt}}
\multiput(1323.17,571.00)(13.000,21.324){2}{\rule{0.400pt}{0.404pt}}
\multiput(1337.58,594.00)(0.492,0.798){21}{\rule{0.119pt}{0.733pt}}
\multiput(1336.17,594.00)(12.000,17.478){2}{\rule{0.400pt}{0.367pt}}
\multiput(1349.58,613.00)(0.493,0.655){23}{\rule{0.119pt}{0.623pt}}
\multiput(1348.17,613.00)(13.000,15.707){2}{\rule{0.400pt}{0.312pt}}
\multiput(1362.00,630.58)(0.539,0.492){21}{\rule{0.533pt}{0.119pt}}
\multiput(1362.00,629.17)(11.893,12.000){2}{\rule{0.267pt}{0.400pt}}
\multiput(1375.00,642.59)(0.824,0.488){13}{\rule{0.750pt}{0.117pt}}
\multiput(1375.00,641.17)(11.443,8.000){2}{\rule{0.375pt}{0.400pt}}
\multiput(1388.00,650.61)(2.695,0.447){3}{\rule{1.833pt}{0.108pt}}
\multiput(1388.00,649.17)(9.195,3.000){2}{\rule{0.917pt}{0.400pt}}
\put(1401,651.17){\rule{2.500pt}{0.400pt}}
\multiput(1401.00,652.17)(6.811,-2.000){2}{\rule{1.250pt}{0.400pt}}
\multiput(1413.00,649.93)(1.123,-0.482){9}{\rule{0.967pt}{0.116pt}}
\multiput(1413.00,650.17)(10.994,-6.000){2}{\rule{0.483pt}{0.400pt}}
\multiput(1426.58,578.93)(0.493,-20.361){23}{\rule{0.119pt}{15.915pt}}
\multiput(1425.17,611.97)(13.000,-480.967){2}{\rule{0.400pt}{7.958pt}}
\put(1279,778){\makebox(0,0)[r]{Namerané dáta: zývylosť polohy od času}}
\multiput(1299,778)(20.756,0.000){5}{\usebox{\plotpoint}}
\put(1399,778){\usebox{\plotpoint}}
\put(1299.00,788.00){\usebox{\plotpoint}}
\put(1299,768){\usebox{\plotpoint}}
\put(1399.00,788.00){\usebox{\plotpoint}}
\put(1399,768){\usebox{\plotpoint}}
\put(171.00,598.00){\usebox{\plotpoint}}
\put(171,610){\usebox{\plotpoint}}
\put(171.00,598.00){\usebox{\plotpoint}}
\put(181,598){\usebox{\plotpoint}}
\put(171.00,610.00){\usebox{\plotpoint}}
\put(181,610){\usebox{\plotpoint}}
\put(186.00,647.00){\usebox{\plotpoint}}
\put(186,659){\usebox{\plotpoint}}
\put(176.00,647.00){\usebox{\plotpoint}}
\put(196,647){\usebox{\plotpoint}}
\put(176.00,659.00){\usebox{\plotpoint}}
\put(196,659){\usebox{\plotpoint}}
\put(202.00,689.00){\usebox{\plotpoint}}
\put(202,701){\usebox{\plotpoint}}
\put(192.00,689.00){\usebox{\plotpoint}}
\put(212,689){\usebox{\plotpoint}}
\put(192.00,701.00){\usebox{\plotpoint}}
\put(212,701){\usebox{\plotpoint}}
\put(217.00,732.00){\usebox{\plotpoint}}
\put(217,744){\usebox{\plotpoint}}
\put(207.00,732.00){\usebox{\plotpoint}}
\put(227,732){\usebox{\plotpoint}}
\put(207.00,744.00){\usebox{\plotpoint}}
\put(227,744){\usebox{\plotpoint}}
\put(233.00,756.00){\usebox{\plotpoint}}
\put(233,768){\usebox{\plotpoint}}
\put(223.00,756.00){\usebox{\plotpoint}}
\put(243,756){\usebox{\plotpoint}}
\put(223.00,768.00){\usebox{\plotpoint}}
\put(243,768){\usebox{\plotpoint}}
\put(248.00,762.00){\usebox{\plotpoint}}
\put(248,774){\usebox{\plotpoint}}
\put(238.00,762.00){\usebox{\plotpoint}}
\put(258,762){\usebox{\plotpoint}}
\put(238.00,774.00){\usebox{\plotpoint}}
\put(258,774){\usebox{\plotpoint}}
\put(264.00,750.00){\usebox{\plotpoint}}
\put(264,762){\usebox{\plotpoint}}
\put(254.00,750.00){\usebox{\plotpoint}}
\put(274,750){\usebox{\plotpoint}}
\put(254.00,762.00){\usebox{\plotpoint}}
\put(274,762){\usebox{\plotpoint}}
\put(279.00,738.00){\usebox{\plotpoint}}
\put(279,750){\usebox{\plotpoint}}
\put(269.00,738.00){\usebox{\plotpoint}}
\put(289,738){\usebox{\plotpoint}}
\put(269.00,750.00){\usebox{\plotpoint}}
\put(289,750){\usebox{\plotpoint}}
\put(295.00,701.00){\usebox{\plotpoint}}
\put(295,713){\usebox{\plotpoint}}
\put(285.00,701.00){\usebox{\plotpoint}}
\put(305,701){\usebox{\plotpoint}}
\put(285.00,713.00){\usebox{\plotpoint}}
\put(305,713){\usebox{\plotpoint}}
\put(310.00,659.00){\usebox{\plotpoint}}
\put(310,671){\usebox{\plotpoint}}
\put(300.00,659.00){\usebox{\plotpoint}}
\put(320,659){\usebox{\plotpoint}}
\put(300.00,671.00){\usebox{\plotpoint}}
\put(320,671){\usebox{\plotpoint}}
\put(326.00,598.00){\usebox{\plotpoint}}
\put(326,610){\usebox{\plotpoint}}
\put(316.00,598.00){\usebox{\plotpoint}}
\put(336,598){\usebox{\plotpoint}}
\put(316.00,610.00){\usebox{\plotpoint}}
\put(336,610){\usebox{\plotpoint}}
\put(341.00,544.00){\usebox{\plotpoint}}
\put(341,556){\usebox{\plotpoint}}
\put(331.00,544.00){\usebox{\plotpoint}}
\put(351,544){\usebox{\plotpoint}}
\put(331.00,556.00){\usebox{\plotpoint}}
\put(351,556){\usebox{\plotpoint}}
\put(357.00,495.00){\usebox{\plotpoint}}
\put(357,507){\usebox{\plotpoint}}
\put(347.00,495.00){\usebox{\plotpoint}}
\put(367,495){\usebox{\plotpoint}}
\put(347.00,507.00){\usebox{\plotpoint}}
\put(367,507){\usebox{\plotpoint}}
\put(372.00,440.00){\usebox{\plotpoint}}
\put(372,453){\usebox{\plotpoint}}
\put(362.00,440.00){\usebox{\plotpoint}}
\put(382,440){\usebox{\plotpoint}}
\put(362.00,453.00){\usebox{\plotpoint}}
\put(382,453){\usebox{\plotpoint}}
\put(387.00,398.00){\usebox{\plotpoint}}
\put(387,410){\usebox{\plotpoint}}
\put(377.00,398.00){\usebox{\plotpoint}}
\put(397,398){\usebox{\plotpoint}}
\put(377.00,410.00){\usebox{\plotpoint}}
\put(397,410){\usebox{\plotpoint}}
\put(403.00,362.00){\usebox{\plotpoint}}
\put(403,374){\usebox{\plotpoint}}
\put(393.00,362.00){\usebox{\plotpoint}}
\put(413,362){\usebox{\plotpoint}}
\put(393.00,374.00){\usebox{\plotpoint}}
\put(413,374){\usebox{\plotpoint}}
\put(418.00,343.00){\usebox{\plotpoint}}
\put(418,355){\usebox{\plotpoint}}
\put(408.00,343.00){\usebox{\plotpoint}}
\put(428,343){\usebox{\plotpoint}}
\put(408.00,355.00){\usebox{\plotpoint}}
\put(428,355){\usebox{\plotpoint}}
\put(434.00,325.00){\usebox{\plotpoint}}
\put(434,337){\usebox{\plotpoint}}
\put(424.00,325.00){\usebox{\plotpoint}}
\put(444,325){\usebox{\plotpoint}}
\put(424.00,337.00){\usebox{\plotpoint}}
\put(444,337){\usebox{\plotpoint}}
\put(449.00,331.00){\usebox{\plotpoint}}
\put(449,343){\usebox{\plotpoint}}
\put(439.00,331.00){\usebox{\plotpoint}}
\put(459,331){\usebox{\plotpoint}}
\put(439.00,343.00){\usebox{\plotpoint}}
\put(459,343){\usebox{\plotpoint}}
\put(465.00,343.00){\usebox{\plotpoint}}
\put(465,355){\usebox{\plotpoint}}
\put(455.00,343.00){\usebox{\plotpoint}}
\put(475,343){\usebox{\plotpoint}}
\put(455.00,355.00){\usebox{\plotpoint}}
\put(475,355){\usebox{\plotpoint}}
\put(480.00,368.00){\usebox{\plotpoint}}
\put(480,380){\usebox{\plotpoint}}
\put(470.00,368.00){\usebox{\plotpoint}}
\put(490,368){\usebox{\plotpoint}}
\put(470.00,380.00){\usebox{\plotpoint}}
\put(490,380){\usebox{\plotpoint}}
\put(496.00,422.00){\usebox{\plotpoint}}
\put(496,434){\usebox{\plotpoint}}
\put(486.00,422.00){\usebox{\plotpoint}}
\put(506,422){\usebox{\plotpoint}}
\put(486.00,434.00){\usebox{\plotpoint}}
\put(506,434){\usebox{\plotpoint}}
\put(511.00,446.00){\usebox{\plotpoint}}
\put(511,459){\usebox{\plotpoint}}
\put(501.00,446.00){\usebox{\plotpoint}}
\put(521,446){\usebox{\plotpoint}}
\put(501.00,459.00){\usebox{\plotpoint}}
\put(521,459){\usebox{\plotpoint}}
\put(527.00,495.00){\usebox{\plotpoint}}
\put(527,507){\usebox{\plotpoint}}
\put(517.00,495.00){\usebox{\plotpoint}}
\put(537,495){\usebox{\plotpoint}}
\put(517.00,507.00){\usebox{\plotpoint}}
\put(537,507){\usebox{\plotpoint}}
\put(542.00,550.00){\usebox{\plotpoint}}
\put(542,562){\usebox{\plotpoint}}
\put(532.00,550.00){\usebox{\plotpoint}}
\put(552,550){\usebox{\plotpoint}}
\put(532.00,562.00){\usebox{\plotpoint}}
\put(552,562){\usebox{\plotpoint}}
\put(558.00,586.00){\usebox{\plotpoint}}
\put(558,598){\usebox{\plotpoint}}
\put(548.00,586.00){\usebox{\plotpoint}}
\put(568,586){\usebox{\plotpoint}}
\put(548.00,598.00){\usebox{\plotpoint}}
\put(568,598){\usebox{\plotpoint}}
\put(573.00,628.00){\usebox{\plotpoint}}
\put(573,641){\usebox{\plotpoint}}
\put(563.00,628.00){\usebox{\plotpoint}}
\put(583,628){\usebox{\plotpoint}}
\put(563.00,641.00){\usebox{\plotpoint}}
\put(583,641){\usebox{\plotpoint}}
\put(589.00,659.00){\usebox{\plotpoint}}
\put(589,671){\usebox{\plotpoint}}
\put(579.00,659.00){\usebox{\plotpoint}}
\put(599,659){\usebox{\plotpoint}}
\put(579.00,671.00){\usebox{\plotpoint}}
\put(599,671){\usebox{\plotpoint}}
\put(604.00,689.00){\usebox{\plotpoint}}
\put(604,701){\usebox{\plotpoint}}
\put(594.00,689.00){\usebox{\plotpoint}}
\put(614,689){\usebox{\plotpoint}}
\put(594.00,701.00){\usebox{\plotpoint}}
\put(614,701){\usebox{\plotpoint}}
\put(619.00,701.00){\usebox{\plotpoint}}
\put(619,713){\usebox{\plotpoint}}
\put(609.00,701.00){\usebox{\plotpoint}}
\put(629,701){\usebox{\plotpoint}}
\put(609.00,713.00){\usebox{\plotpoint}}
\put(629,713){\usebox{\plotpoint}}
\put(635.00,713.00){\usebox{\plotpoint}}
\put(635,726){\usebox{\plotpoint}}
\put(625.00,713.00){\usebox{\plotpoint}}
\put(645,713){\usebox{\plotpoint}}
\put(625.00,726.00){\usebox{\plotpoint}}
\put(645,726){\usebox{\plotpoint}}
\put(650.00,707.00){\usebox{\plotpoint}}
\put(650,719){\usebox{\plotpoint}}
\put(640.00,707.00){\usebox{\plotpoint}}
\put(660,707){\usebox{\plotpoint}}
\put(640.00,719.00){\usebox{\plotpoint}}
\put(660,719){\usebox{\plotpoint}}
\put(666.00,695.00){\usebox{\plotpoint}}
\put(666,707){\usebox{\plotpoint}}
\put(656.00,695.00){\usebox{\plotpoint}}
\put(676,695){\usebox{\plotpoint}}
\put(656.00,707.00){\usebox{\plotpoint}}
\put(676,707){\usebox{\plotpoint}}
\put(681.00,665.00){\usebox{\plotpoint}}
\put(681,677){\usebox{\plotpoint}}
\put(671.00,665.00){\usebox{\plotpoint}}
\put(691,665){\usebox{\plotpoint}}
\put(671.00,677.00){\usebox{\plotpoint}}
\put(691,677){\usebox{\plotpoint}}
\put(697.00,635.00){\usebox{\plotpoint}}
\put(697,647){\usebox{\plotpoint}}
\put(687.00,635.00){\usebox{\plotpoint}}
\put(707,635){\usebox{\plotpoint}}
\put(687.00,647.00){\usebox{\plotpoint}}
\put(707,647){\usebox{\plotpoint}}
\put(712.00,598.00){\usebox{\plotpoint}}
\put(712,610){\usebox{\plotpoint}}
\put(702.00,598.00){\usebox{\plotpoint}}
\put(722,598){\usebox{\plotpoint}}
\put(702.00,610.00){\usebox{\plotpoint}}
\put(722,610){\usebox{\plotpoint}}
\put(728.00,489.00){\usebox{\plotpoint}}
\put(728,501){\usebox{\plotpoint}}
\put(718.00,489.00){\usebox{\plotpoint}}
\put(738,489){\usebox{\plotpoint}}
\put(718.00,501.00){\usebox{\plotpoint}}
\put(738,501){\usebox{\plotpoint}}
\put(743.00,501.00){\usebox{\plotpoint}}
\put(743,513){\usebox{\plotpoint}}
\put(733.00,501.00){\usebox{\plotpoint}}
\put(753,501){\usebox{\plotpoint}}
\put(733.00,513.00){\usebox{\plotpoint}}
\put(753,513){\usebox{\plotpoint}}
\put(759.00,471.00){\usebox{\plotpoint}}
\put(759,483){\usebox{\plotpoint}}
\put(749.00,471.00){\usebox{\plotpoint}}
\put(769,471){\usebox{\plotpoint}}
\put(749.00,483.00){\usebox{\plotpoint}}
\put(769,483){\usebox{\plotpoint}}
\put(774.00,434.00){\usebox{\plotpoint}}
\put(774,446){\usebox{\plotpoint}}
\put(764.00,434.00){\usebox{\plotpoint}}
\put(784,434){\usebox{\plotpoint}}
\put(764.00,446.00){\usebox{\plotpoint}}
\put(784,446){\usebox{\plotpoint}}
\put(790.00,410.00){\usebox{\plotpoint}}
\put(790,422){\usebox{\plotpoint}}
\put(780.00,410.00){\usebox{\plotpoint}}
\put(800,410){\usebox{\plotpoint}}
\put(780.00,422.00){\usebox{\plotpoint}}
\put(800,422){\usebox{\plotpoint}}
\put(805.00,392.00){\usebox{\plotpoint}}
\put(805,404){\usebox{\plotpoint}}
\put(795.00,392.00){\usebox{\plotpoint}}
\put(815,392){\usebox{\plotpoint}}
\put(795.00,404.00){\usebox{\plotpoint}}
\put(815,404){\usebox{\plotpoint}}
\put(820.00,380.00){\usebox{\plotpoint}}
\put(820,392){\usebox{\plotpoint}}
\put(810.00,380.00){\usebox{\plotpoint}}
\put(830,380){\usebox{\plotpoint}}
\put(810.00,392.00){\usebox{\plotpoint}}
\put(830,392){\usebox{\plotpoint}}
\put(836.00,386.00){\usebox{\plotpoint}}
\put(836,398){\usebox{\plotpoint}}
\put(826.00,386.00){\usebox{\plotpoint}}
\put(846,386){\usebox{\plotpoint}}
\put(826.00,398.00){\usebox{\plotpoint}}
\put(846,398){\usebox{\plotpoint}}
\put(851.00,398.00){\usebox{\plotpoint}}
\put(851,410){\usebox{\plotpoint}}
\put(841.00,398.00){\usebox{\plotpoint}}
\put(861,398){\usebox{\plotpoint}}
\put(841.00,410.00){\usebox{\plotpoint}}
\put(861,410){\usebox{\plotpoint}}
\put(867.00,416.00){\usebox{\plotpoint}}
\put(867,428){\usebox{\plotpoint}}
\put(857.00,416.00){\usebox{\plotpoint}}
\put(877,416){\usebox{\plotpoint}}
\put(857.00,428.00){\usebox{\plotpoint}}
\put(877,428){\usebox{\plotpoint}}
\put(882.00,440.00){\usebox{\plotpoint}}
\put(882,453){\usebox{\plotpoint}}
\put(872.00,440.00){\usebox{\plotpoint}}
\put(892,440){\usebox{\plotpoint}}
\put(872.00,453.00){\usebox{\plotpoint}}
\put(892,453){\usebox{\plotpoint}}
\put(898.00,477.00){\usebox{\plotpoint}}
\put(898,489){\usebox{\plotpoint}}
\put(888.00,477.00){\usebox{\plotpoint}}
\put(908,477){\usebox{\plotpoint}}
\put(888.00,489.00){\usebox{\plotpoint}}
\put(908,489){\usebox{\plotpoint}}
\put(913.00,507.00){\usebox{\plotpoint}}
\put(913,519){\usebox{\plotpoint}}
\put(903.00,507.00){\usebox{\plotpoint}}
\put(923,507){\usebox{\plotpoint}}
\put(903.00,519.00){\usebox{\plotpoint}}
\put(923,519){\usebox{\plotpoint}}
\put(929.00,544.00){\usebox{\plotpoint}}
\put(929,556){\usebox{\plotpoint}}
\put(919.00,544.00){\usebox{\plotpoint}}
\put(939,544){\usebox{\plotpoint}}
\put(919.00,556.00){\usebox{\plotpoint}}
\put(939,556){\usebox{\plotpoint}}
\put(944.00,586.00){\usebox{\plotpoint}}
\put(944,598){\usebox{\plotpoint}}
\put(934.00,586.00){\usebox{\plotpoint}}
\put(954,586){\usebox{\plotpoint}}
\put(934.00,598.00){\usebox{\plotpoint}}
\put(954,598){\usebox{\plotpoint}}
\put(960.00,622.00){\usebox{\plotpoint}}
\put(960,635){\usebox{\plotpoint}}
\put(950.00,622.00){\usebox{\plotpoint}}
\put(970,622){\usebox{\plotpoint}}
\put(950.00,635.00){\usebox{\plotpoint}}
\put(970,635){\usebox{\plotpoint}}
\put(975.00,641.00){\usebox{\plotpoint}}
\put(975,653){\usebox{\plotpoint}}
\put(965.00,641.00){\usebox{\plotpoint}}
\put(985,641){\usebox{\plotpoint}}
\put(965.00,653.00){\usebox{\plotpoint}}
\put(985,653){\usebox{\plotpoint}}
\put(991.00,677.00){\usebox{\plotpoint}}
\put(991,689){\usebox{\plotpoint}}
\put(981.00,677.00){\usebox{\plotpoint}}
\put(1001,677){\usebox{\plotpoint}}
\put(981.00,689.00){\usebox{\plotpoint}}
\put(1001,689){\usebox{\plotpoint}}
\put(1006.00,683.00){\usebox{\plotpoint}}
\put(1006,695){\usebox{\plotpoint}}
\put(996.00,683.00){\usebox{\plotpoint}}
\put(1016,683){\usebox{\plotpoint}}
\put(996.00,695.00){\usebox{\plotpoint}}
\put(1016,695){\usebox{\plotpoint}}
\put(1021.00,677.00){\usebox{\plotpoint}}
\put(1021,689){\usebox{\plotpoint}}
\put(1011.00,677.00){\usebox{\plotpoint}}
\put(1031,677){\usebox{\plotpoint}}
\put(1011.00,689.00){\usebox{\plotpoint}}
\put(1031,689){\usebox{\plotpoint}}
\put(1037.00,671.00){\usebox{\plotpoint}}
\put(1037,683){\usebox{\plotpoint}}
\put(1027.00,671.00){\usebox{\plotpoint}}
\put(1047,671){\usebox{\plotpoint}}
\put(1027.00,683.00){\usebox{\plotpoint}}
\put(1047,683){\usebox{\plotpoint}}
\put(1052.00,659.00){\usebox{\plotpoint}}
\put(1052,671){\usebox{\plotpoint}}
\put(1042.00,659.00){\usebox{\plotpoint}}
\put(1062,659){\usebox{\plotpoint}}
\put(1042.00,671.00){\usebox{\plotpoint}}
\put(1062,671){\usebox{\plotpoint}}
\put(1068.00,641.00){\usebox{\plotpoint}}
\put(1068,653){\usebox{\plotpoint}}
\put(1058.00,641.00){\usebox{\plotpoint}}
\put(1078,641){\usebox{\plotpoint}}
\put(1058.00,653.00){\usebox{\plotpoint}}
\put(1078,653){\usebox{\plotpoint}}
\put(1083.00,610.00){\usebox{\plotpoint}}
\put(1083,622){\usebox{\plotpoint}}
\put(1073.00,610.00){\usebox{\plotpoint}}
\put(1093,610){\usebox{\plotpoint}}
\put(1073.00,622.00){\usebox{\plotpoint}}
\put(1093,622){\usebox{\plotpoint}}
\put(1099.00,580.00){\usebox{\plotpoint}}
\put(1099,592){\usebox{\plotpoint}}
\put(1089.00,580.00){\usebox{\plotpoint}}
\put(1109,580){\usebox{\plotpoint}}
\put(1089.00,592.00){\usebox{\plotpoint}}
\put(1109,592){\usebox{\plotpoint}}
\put(1114.00,550.00){\usebox{\plotpoint}}
\put(1114,562){\usebox{\plotpoint}}
\put(1104.00,550.00){\usebox{\plotpoint}}
\put(1124,550){\usebox{\plotpoint}}
\put(1104.00,562.00){\usebox{\plotpoint}}
\put(1124,562){\usebox{\plotpoint}}
\put(1130.00,513.00){\usebox{\plotpoint}}
\put(1130,525){\usebox{\plotpoint}}
\put(1120.00,513.00){\usebox{\plotpoint}}
\put(1140,513){\usebox{\plotpoint}}
\put(1120.00,525.00){\usebox{\plotpoint}}
\put(1140,525){\usebox{\plotpoint}}
\put(1145.00,489.00){\usebox{\plotpoint}}
\put(1145,501){\usebox{\plotpoint}}
\put(1135.00,489.00){\usebox{\plotpoint}}
\put(1155,489){\usebox{\plotpoint}}
\put(1135.00,501.00){\usebox{\plotpoint}}
\put(1155,501){\usebox{\plotpoint}}
\put(1161.00,465.00){\usebox{\plotpoint}}
\put(1161,477){\usebox{\plotpoint}}
\put(1151.00,465.00){\usebox{\plotpoint}}
\put(1171,465){\usebox{\plotpoint}}
\put(1151.00,477.00){\usebox{\plotpoint}}
\put(1171,477){\usebox{\plotpoint}}
\put(1176.00,440.00){\usebox{\plotpoint}}
\put(1176,453){\usebox{\plotpoint}}
\put(1166.00,440.00){\usebox{\plotpoint}}
\put(1186,440){\usebox{\plotpoint}}
\put(1166.00,453.00){\usebox{\plotpoint}}
\put(1186,453){\usebox{\plotpoint}}
\put(1192.00,428.00){\usebox{\plotpoint}}
\put(1192,440){\usebox{\plotpoint}}
\put(1182.00,428.00){\usebox{\plotpoint}}
\put(1202,428){\usebox{\plotpoint}}
\put(1182.00,440.00){\usebox{\plotpoint}}
\put(1202,440){\usebox{\plotpoint}}
\put(1207.00,422.00){\usebox{\plotpoint}}
\put(1207,434){\usebox{\plotpoint}}
\put(1197.00,422.00){\usebox{\plotpoint}}
\put(1217,422){\usebox{\plotpoint}}
\put(1197.00,434.00){\usebox{\plotpoint}}
\put(1217,434){\usebox{\plotpoint}}
\put(1223.00,422.00){\usebox{\plotpoint}}
\put(1223,434){\usebox{\plotpoint}}
\put(1213.00,422.00){\usebox{\plotpoint}}
\put(1233,422){\usebox{\plotpoint}}
\put(1213.00,434.00){\usebox{\plotpoint}}
\put(1233,434){\usebox{\plotpoint}}
\put(1238.00,434.00){\usebox{\plotpoint}}
\put(1238,446){\usebox{\plotpoint}}
\put(1228.00,434.00){\usebox{\plotpoint}}
\put(1248,434){\usebox{\plotpoint}}
\put(1228.00,446.00){\usebox{\plotpoint}}
\put(1248,446){\usebox{\plotpoint}}
\put(1253.00,446.00){\usebox{\plotpoint}}
\put(1253,459){\usebox{\plotpoint}}
\put(1243.00,446.00){\usebox{\plotpoint}}
\put(1263,446){\usebox{\plotpoint}}
\put(1243.00,459.00){\usebox{\plotpoint}}
\put(1263,459){\usebox{\plotpoint}}
\put(1269.00,471.00){\usebox{\plotpoint}}
\put(1269,483){\usebox{\plotpoint}}
\put(1259.00,471.00){\usebox{\plotpoint}}
\put(1279,471){\usebox{\plotpoint}}
\put(1259.00,483.00){\usebox{\plotpoint}}
\put(1279,483){\usebox{\plotpoint}}
\put(1284.00,495.00){\usebox{\plotpoint}}
\put(1284,507){\usebox{\plotpoint}}
\put(1274.00,495.00){\usebox{\plotpoint}}
\put(1294,495){\usebox{\plotpoint}}
\put(1274.00,507.00){\usebox{\plotpoint}}
\put(1294,507){\usebox{\plotpoint}}
\put(1300.00,525.00){\usebox{\plotpoint}}
\put(1300,537){\usebox{\plotpoint}}
\put(1290.00,525.00){\usebox{\plotpoint}}
\put(1310,525){\usebox{\plotpoint}}
\put(1290.00,537.00){\usebox{\plotpoint}}
\put(1310,537){\usebox{\plotpoint}}
\put(1315.00,562.00){\usebox{\plotpoint}}
\put(1315,574){\usebox{\plotpoint}}
\put(1305.00,562.00){\usebox{\plotpoint}}
\put(1325,562){\usebox{\plotpoint}}
\put(1305.00,574.00){\usebox{\plotpoint}}
\put(1325,574){\usebox{\plotpoint}}
\put(1331.00,580.00){\usebox{\plotpoint}}
\put(1331,592){\usebox{\plotpoint}}
\put(1321.00,580.00){\usebox{\plotpoint}}
\put(1341,580){\usebox{\plotpoint}}
\put(1321.00,592.00){\usebox{\plotpoint}}
\put(1341,592){\usebox{\plotpoint}}
\put(1346.00,604.00){\usebox{\plotpoint}}
\put(1346,616){\usebox{\plotpoint}}
\put(1336.00,604.00){\usebox{\plotpoint}}
\put(1356,604){\usebox{\plotpoint}}
\put(1336.00,616.00){\usebox{\plotpoint}}
\put(1356,616){\usebox{\plotpoint}}
\put(1362.00,628.00){\usebox{\plotpoint}}
\put(1362,641){\usebox{\plotpoint}}
\put(1352.00,628.00){\usebox{\plotpoint}}
\put(1372,628){\usebox{\plotpoint}}
\put(1352.00,641.00){\usebox{\plotpoint}}
\put(1372,641){\usebox{\plotpoint}}
\put(1377.00,641.00){\usebox{\plotpoint}}
\put(1377,653){\usebox{\plotpoint}}
\put(1367.00,641.00){\usebox{\plotpoint}}
\put(1387,641){\usebox{\plotpoint}}
\put(1367.00,653.00){\usebox{\plotpoint}}
\put(1387,653){\usebox{\plotpoint}}
\put(1393.00,653.00){\usebox{\plotpoint}}
\put(1393,665){\usebox{\plotpoint}}
\put(1383.00,653.00){\usebox{\plotpoint}}
\put(1403,653){\usebox{\plotpoint}}
\put(1383.00,665.00){\usebox{\plotpoint}}
\put(1403,665){\usebox{\plotpoint}}
\put(1408.00,653.00){\usebox{\plotpoint}}
\put(1408,665){\usebox{\plotpoint}}
\put(1398.00,653.00){\usebox{\plotpoint}}
\put(1418,653){\usebox{\plotpoint}}
\put(1398.00,665.00){\usebox{\plotpoint}}
\put(1418,665){\usebox{\plotpoint}}
\put(1424.00,647.00){\usebox{\plotpoint}}
\put(1424,659){\usebox{\plotpoint}}
\put(1414.00,647.00){\usebox{\plotpoint}}
\put(1434,647){\usebox{\plotpoint}}
\put(1414.00,659.00){\usebox{\plotpoint}}
\put(1434,659){\usebox{\plotpoint}}
\put(1439.00,635.00){\usebox{\plotpoint}}
\put(1439,647){\usebox{\plotpoint}}
\put(1429.00,635.00){\usebox{\plotpoint}}
\put(1439,635){\usebox{\plotpoint}}
\put(1429.00,647.00){\usebox{\plotpoint}}
\put(1439,647){\usebox{\plotpoint}}
\put(171,604){\makebox(0,0){$\times$}}
\put(186,653){\makebox(0,0){$\times$}}
\put(202,695){\makebox(0,0){$\times$}}
\put(217,738){\makebox(0,0){$\times$}}
\put(233,762){\makebox(0,0){$\times$}}
\put(248,768){\makebox(0,0){$\times$}}
\put(264,756){\makebox(0,0){$\times$}}
\put(279,744){\makebox(0,0){$\times$}}
\put(295,707){\makebox(0,0){$\times$}}
\put(310,665){\makebox(0,0){$\times$}}
\put(326,604){\makebox(0,0){$\times$}}
\put(341,550){\makebox(0,0){$\times$}}
\put(357,501){\makebox(0,0){$\times$}}
\put(372,446){\makebox(0,0){$\times$}}
\put(387,404){\makebox(0,0){$\times$}}
\put(403,368){\makebox(0,0){$\times$}}
\put(418,349){\makebox(0,0){$\times$}}
\put(434,331){\makebox(0,0){$\times$}}
\put(449,337){\makebox(0,0){$\times$}}
\put(465,349){\makebox(0,0){$\times$}}
\put(480,374){\makebox(0,0){$\times$}}
\put(496,428){\makebox(0,0){$\times$}}
\put(511,453){\makebox(0,0){$\times$}}
\put(527,501){\makebox(0,0){$\times$}}
\put(542,556){\makebox(0,0){$\times$}}
\put(558,592){\makebox(0,0){$\times$}}
\put(573,635){\makebox(0,0){$\times$}}
\put(589,665){\makebox(0,0){$\times$}}
\put(604,695){\makebox(0,0){$\times$}}
\put(619,707){\makebox(0,0){$\times$}}
\put(635,719){\makebox(0,0){$\times$}}
\put(650,713){\makebox(0,0){$\times$}}
\put(666,701){\makebox(0,0){$\times$}}
\put(681,671){\makebox(0,0){$\times$}}
\put(697,641){\makebox(0,0){$\times$}}
\put(712,604){\makebox(0,0){$\times$}}
\put(728,495){\makebox(0,0){$\times$}}
\put(743,507){\makebox(0,0){$\times$}}
\put(759,477){\makebox(0,0){$\times$}}
\put(774,440){\makebox(0,0){$\times$}}
\put(790,416){\makebox(0,0){$\times$}}
\put(805,398){\makebox(0,0){$\times$}}
\put(820,386){\makebox(0,0){$\times$}}
\put(836,392){\makebox(0,0){$\times$}}
\put(851,404){\makebox(0,0){$\times$}}
\put(867,422){\makebox(0,0){$\times$}}
\put(882,446){\makebox(0,0){$\times$}}
\put(898,483){\makebox(0,0){$\times$}}
\put(913,513){\makebox(0,0){$\times$}}
\put(929,550){\makebox(0,0){$\times$}}
\put(944,592){\makebox(0,0){$\times$}}
\put(960,628){\makebox(0,0){$\times$}}
\put(975,647){\makebox(0,0){$\times$}}
\put(991,683){\makebox(0,0){$\times$}}
\put(1006,689){\makebox(0,0){$\times$}}
\put(1021,683){\makebox(0,0){$\times$}}
\put(1037,677){\makebox(0,0){$\times$}}
\put(1052,665){\makebox(0,0){$\times$}}
\put(1068,647){\makebox(0,0){$\times$}}
\put(1083,616){\makebox(0,0){$\times$}}
\put(1099,586){\makebox(0,0){$\times$}}
\put(1114,556){\makebox(0,0){$\times$}}
\put(1130,519){\makebox(0,0){$\times$}}
\put(1145,495){\makebox(0,0){$\times$}}
\put(1161,471){\makebox(0,0){$\times$}}
\put(1176,446){\makebox(0,0){$\times$}}
\put(1192,434){\makebox(0,0){$\times$}}
\put(1207,428){\makebox(0,0){$\times$}}
\put(1223,428){\makebox(0,0){$\times$}}
\put(1238,440){\makebox(0,0){$\times$}}
\put(1253,453){\makebox(0,0){$\times$}}
\put(1269,477){\makebox(0,0){$\times$}}
\put(1284,501){\makebox(0,0){$\times$}}
\put(1300,531){\makebox(0,0){$\times$}}
\put(1315,568){\makebox(0,0){$\times$}}
\put(1331,586){\makebox(0,0){$\times$}}
\put(1346,610){\makebox(0,0){$\times$}}
\put(1362,635){\makebox(0,0){$\times$}}
\put(1377,647){\makebox(0,0){$\times$}}
\put(1393,659){\makebox(0,0){$\times$}}
\put(1408,659){\makebox(0,0){$\times$}}
\put(1424,653){\makebox(0,0){$\times$}}
\put(1439,641){\makebox(0,0){$\times$}}
\put(1349,778){\makebox(0,0){$\times$}}
\put(171.0,131.0){\rule[-0.200pt]{0.400pt}{175.375pt}}
\put(171.0,131.0){\rule[-0.200pt]{305.461pt}{0.400pt}}
\put(1439.0,131.0){\rule[-0.200pt]{0.400pt}{175.375pt}}
\put(171.0,859.0){\rule[-0.200pt]{305.461pt}{0.400pt}}
\end{picture}

\caption{Vynesená závislosť polohy $l$ na čase $t$ pre prvú polohu a fit $l\(t\) = \(65.3\pm4.9\)\exp\(-\(4.9\pm0.5\)10^{-4}t\)\(\sin\(\frac{2\pi t}{498.2\pm2.0}\) + \(4.93\pm0.07\)\) + \(115.6\pm 0.3\)$.}  \label{G_2}

\end{figure}

\begin{figure}
% GNUPLOT: LaTeX picture
\setlength{\unitlength}{0.240900pt}
\ifx\plotpoint\undefined\newsavebox{\plotpoint}\fi
\begin{picture}(1500,900)(0,0)
\sbox{\plotpoint}{\rule[-0.200pt]{0.400pt}{0.400pt}}%
\put(171.0,131.0){\rule[-0.200pt]{4.818pt}{0.400pt}}
\put(151,131){\makebox(0,0)[r]{ 90}}
\put(1419.0,131.0){\rule[-0.200pt]{4.818pt}{0.400pt}}
\put(171.0,235.0){\rule[-0.200pt]{4.818pt}{0.400pt}}
\put(151,235){\makebox(0,0)[r]{ 100}}
\put(1419.0,235.0){\rule[-0.200pt]{4.818pt}{0.400pt}}
\put(171.0,339.0){\rule[-0.200pt]{4.818pt}{0.400pt}}
\put(151,339){\makebox(0,0)[r]{ 110}}
\put(1419.0,339.0){\rule[-0.200pt]{4.818pt}{0.400pt}}
\put(171.0,443.0){\rule[-0.200pt]{4.818pt}{0.400pt}}
\put(151,443){\makebox(0,0)[r]{ 120}}
\put(1419.0,443.0){\rule[-0.200pt]{4.818pt}{0.400pt}}
\put(171.0,547.0){\rule[-0.200pt]{4.818pt}{0.400pt}}
\put(151,547){\makebox(0,0)[r]{ 130}}
\put(1419.0,547.0){\rule[-0.200pt]{4.818pt}{0.400pt}}
\put(171.0,651.0){\rule[-0.200pt]{4.818pt}{0.400pt}}
\put(151,651){\makebox(0,0)[r]{ 140}}
\put(1419.0,651.0){\rule[-0.200pt]{4.818pt}{0.400pt}}
\put(171.0,755.0){\rule[-0.200pt]{4.818pt}{0.400pt}}
\put(151,755){\makebox(0,0)[r]{ 150}}
\put(1419.0,755.0){\rule[-0.200pt]{4.818pt}{0.400pt}}
\put(171.0,859.0){\rule[-0.200pt]{4.818pt}{0.400pt}}
\put(151,859){\makebox(0,0)[r]{ 160}}
\put(1419.0,859.0){\rule[-0.200pt]{4.818pt}{0.400pt}}
\put(238.0,131.0){\rule[-0.200pt]{0.400pt}{4.818pt}}
\put(238,90){\makebox(0,0){ 2200}}
\put(238.0,839.0){\rule[-0.200pt]{0.400pt}{4.818pt}}
\put(371.0,131.0){\rule[-0.200pt]{0.400pt}{4.818pt}}
\put(371,90){\makebox(0,0){ 2400}}
\put(371.0,839.0){\rule[-0.200pt]{0.400pt}{4.818pt}}
\put(505.0,131.0){\rule[-0.200pt]{0.400pt}{4.818pt}}
\put(505,90){\makebox(0,0){ 2600}}
\put(505.0,839.0){\rule[-0.200pt]{0.400pt}{4.818pt}}
\put(638.0,131.0){\rule[-0.200pt]{0.400pt}{4.818pt}}
\put(638,90){\makebox(0,0){ 2800}}
\put(638.0,839.0){\rule[-0.200pt]{0.400pt}{4.818pt}}
\put(772.0,131.0){\rule[-0.200pt]{0.400pt}{4.818pt}}
\put(772,90){\makebox(0,0){ 3000}}
\put(772.0,839.0){\rule[-0.200pt]{0.400pt}{4.818pt}}
\put(905.0,131.0){\rule[-0.200pt]{0.400pt}{4.818pt}}
\put(905,90){\makebox(0,0){ 3200}}
\put(905.0,839.0){\rule[-0.200pt]{0.400pt}{4.818pt}}
\put(1039.0,131.0){\rule[-0.200pt]{0.400pt}{4.818pt}}
\put(1039,90){\makebox(0,0){ 3400}}
\put(1039.0,839.0){\rule[-0.200pt]{0.400pt}{4.818pt}}
\put(1172.0,131.0){\rule[-0.200pt]{0.400pt}{4.818pt}}
\put(1172,90){\makebox(0,0){ 3600}}
\put(1172.0,839.0){\rule[-0.200pt]{0.400pt}{4.818pt}}
\put(1306.0,131.0){\rule[-0.200pt]{0.400pt}{4.818pt}}
\put(1306,90){\makebox(0,0){ 3800}}
\put(1306.0,839.0){\rule[-0.200pt]{0.400pt}{4.818pt}}
\put(1439.0,131.0){\rule[-0.200pt]{0.400pt}{4.818pt}}
\put(1439,90){\makebox(0,0){ 4000}}
\put(1439.0,839.0){\rule[-0.200pt]{0.400pt}{4.818pt}}
\put(171.0,131.0){\rule[-0.200pt]{0.400pt}{175.375pt}}
\put(171.0,131.0){\rule[-0.200pt]{305.461pt}{0.400pt}}
\put(1439.0,131.0){\rule[-0.200pt]{0.400pt}{175.375pt}}
\put(171.0,859.0){\rule[-0.200pt]{305.461pt}{0.400pt}}
\put(30,495){\makebox(0,0){\popi{l}{cm}}}
\put(805,29){\makebox(0,0){\popi{t}{s}}}
\put(1279,819){\makebox(0,0)[r]{Namerané hodnoty}}
\put(1299.0,819.0){\rule[-0.200pt]{24.090pt}{0.400pt}}
\put(1299.0,809.0){\rule[-0.200pt]{0.400pt}{4.818pt}}
\put(1399.0,809.0){\rule[-0.200pt]{0.400pt}{4.818pt}}
\put(171.0,557.0){\rule[-0.200pt]{0.400pt}{25.054pt}}
\put(171.0,557.0){\rule[-0.200pt]{2.409pt}{0.400pt}}
\put(171.0,661.0){\rule[-0.200pt]{2.409pt}{0.400pt}}
\put(184.0,599.0){\rule[-0.200pt]{0.400pt}{25.054pt}}
\put(174.0,599.0){\rule[-0.200pt]{4.818pt}{0.400pt}}
\put(174.0,703.0){\rule[-0.200pt]{4.818pt}{0.400pt}}
\put(198.0,661.0){\rule[-0.200pt]{0.400pt}{25.054pt}}
\put(188.0,661.0){\rule[-0.200pt]{4.818pt}{0.400pt}}
\put(188.0,765.0){\rule[-0.200pt]{4.818pt}{0.400pt}}
\put(211.0,693.0){\rule[-0.200pt]{0.400pt}{25.054pt}}
\put(201.0,693.0){\rule[-0.200pt]{4.818pt}{0.400pt}}
\put(201.0,797.0){\rule[-0.200pt]{4.818pt}{0.400pt}}
\put(224.0,703.0){\rule[-0.200pt]{0.400pt}{25.054pt}}
\put(214.0,703.0){\rule[-0.200pt]{4.818pt}{0.400pt}}
\put(214.0,807.0){\rule[-0.200pt]{4.818pt}{0.400pt}}
\put(238.0,713.0){\rule[-0.200pt]{0.400pt}{25.054pt}}
\put(228.0,713.0){\rule[-0.200pt]{4.818pt}{0.400pt}}
\put(228.0,817.0){\rule[-0.200pt]{4.818pt}{0.400pt}}
\put(251.0,693.0){\rule[-0.200pt]{0.400pt}{25.054pt}}
\put(241.0,693.0){\rule[-0.200pt]{4.818pt}{0.400pt}}
\put(241.0,797.0){\rule[-0.200pt]{4.818pt}{0.400pt}}
\put(264.0,661.0){\rule[-0.200pt]{0.400pt}{25.054pt}}
\put(254.0,661.0){\rule[-0.200pt]{4.818pt}{0.400pt}}
\put(254.0,765.0){\rule[-0.200pt]{4.818pt}{0.400pt}}
\put(278.0,620.0){\rule[-0.200pt]{0.400pt}{25.054pt}}
\put(268.0,620.0){\rule[-0.200pt]{4.818pt}{0.400pt}}
\put(268.0,724.0){\rule[-0.200pt]{4.818pt}{0.400pt}}
\put(291.0,568.0){\rule[-0.200pt]{0.400pt}{25.054pt}}
\put(281.0,568.0){\rule[-0.200pt]{4.818pt}{0.400pt}}
\put(281.0,672.0){\rule[-0.200pt]{4.818pt}{0.400pt}}
\put(304.0,505.0){\rule[-0.200pt]{0.400pt}{25.054pt}}
\put(294.0,505.0){\rule[-0.200pt]{4.818pt}{0.400pt}}
\put(294.0,609.0){\rule[-0.200pt]{4.818pt}{0.400pt}}
\put(318.0,453.0){\rule[-0.200pt]{0.400pt}{25.054pt}}
\put(308.0,453.0){\rule[-0.200pt]{4.818pt}{0.400pt}}
\put(308.0,557.0){\rule[-0.200pt]{4.818pt}{0.400pt}}
\put(331.0,391.0){\rule[-0.200pt]{0.400pt}{25.054pt}}
\put(321.0,391.0){\rule[-0.200pt]{4.818pt}{0.400pt}}
\put(321.0,495.0){\rule[-0.200pt]{4.818pt}{0.400pt}}
\put(345.0,339.0){\rule[-0.200pt]{0.400pt}{25.054pt}}
\put(335.0,339.0){\rule[-0.200pt]{4.818pt}{0.400pt}}
\put(335.0,443.0){\rule[-0.200pt]{4.818pt}{0.400pt}}
\put(358.0,287.0){\rule[-0.200pt]{0.400pt}{25.054pt}}
\put(348.0,287.0){\rule[-0.200pt]{4.818pt}{0.400pt}}
\put(348.0,391.0){\rule[-0.200pt]{4.818pt}{0.400pt}}
\put(371.0,256.0){\rule[-0.200pt]{0.400pt}{25.054pt}}
\put(361.0,256.0){\rule[-0.200pt]{4.818pt}{0.400pt}}
\put(361.0,360.0){\rule[-0.200pt]{4.818pt}{0.400pt}}
\put(385.0,225.0){\rule[-0.200pt]{0.400pt}{25.054pt}}
\put(375.0,225.0){\rule[-0.200pt]{4.818pt}{0.400pt}}
\put(375.0,329.0){\rule[-0.200pt]{4.818pt}{0.400pt}}
\put(398.0,225.0){\rule[-0.200pt]{0.400pt}{25.054pt}}
\put(388.0,225.0){\rule[-0.200pt]{4.818pt}{0.400pt}}
\put(388.0,329.0){\rule[-0.200pt]{4.818pt}{0.400pt}}
\put(411.0,225.0){\rule[-0.200pt]{0.400pt}{25.054pt}}
\put(401.0,225.0){\rule[-0.200pt]{4.818pt}{0.400pt}}
\put(401.0,329.0){\rule[-0.200pt]{4.818pt}{0.400pt}}
\put(425.0,256.0){\rule[-0.200pt]{0.400pt}{25.054pt}}
\put(415.0,256.0){\rule[-0.200pt]{4.818pt}{0.400pt}}
\put(415.0,360.0){\rule[-0.200pt]{4.818pt}{0.400pt}}
\put(438.0,287.0){\rule[-0.200pt]{0.400pt}{25.054pt}}
\put(428.0,287.0){\rule[-0.200pt]{4.818pt}{0.400pt}}
\put(428.0,391.0){\rule[-0.200pt]{4.818pt}{0.400pt}}
\put(451.0,329.0){\rule[-0.200pt]{0.400pt}{25.054pt}}
\put(441.0,329.0){\rule[-0.200pt]{4.818pt}{0.400pt}}
\put(441.0,433.0){\rule[-0.200pt]{4.818pt}{0.400pt}}
\put(465.0,370.0){\rule[-0.200pt]{0.400pt}{25.054pt}}
\put(455.0,370.0){\rule[-0.200pt]{4.818pt}{0.400pt}}
\put(455.0,474.0){\rule[-0.200pt]{4.818pt}{0.400pt}}
\put(478.0,433.0){\rule[-0.200pt]{0.400pt}{25.054pt}}
\put(468.0,433.0){\rule[-0.200pt]{4.818pt}{0.400pt}}
\put(468.0,537.0){\rule[-0.200pt]{4.818pt}{0.400pt}}
\put(491.0,485.0){\rule[-0.200pt]{0.400pt}{25.054pt}}
\put(481.0,485.0){\rule[-0.200pt]{4.818pt}{0.400pt}}
\put(481.0,589.0){\rule[-0.200pt]{4.818pt}{0.400pt}}
\put(505.0,537.0){\rule[-0.200pt]{0.400pt}{25.054pt}}
\put(495.0,537.0){\rule[-0.200pt]{4.818pt}{0.400pt}}
\put(495.0,641.0){\rule[-0.200pt]{4.818pt}{0.400pt}}
\put(518.0,578.0){\rule[-0.200pt]{0.400pt}{25.054pt}}
\put(508.0,578.0){\rule[-0.200pt]{4.818pt}{0.400pt}}
\put(508.0,682.0){\rule[-0.200pt]{4.818pt}{0.400pt}}
\put(531.0,609.0){\rule[-0.200pt]{0.400pt}{25.054pt}}
\put(521.0,609.0){\rule[-0.200pt]{4.818pt}{0.400pt}}
\put(521.0,713.0){\rule[-0.200pt]{4.818pt}{0.400pt}}
\put(545.0,651.0){\rule[-0.200pt]{0.400pt}{25.054pt}}
\put(535.0,651.0){\rule[-0.200pt]{4.818pt}{0.400pt}}
\put(535.0,755.0){\rule[-0.200pt]{4.818pt}{0.400pt}}
\put(558.0,651.0){\rule[-0.200pt]{0.400pt}{25.054pt}}
\put(548.0,651.0){\rule[-0.200pt]{4.818pt}{0.400pt}}
\put(548.0,755.0){\rule[-0.200pt]{4.818pt}{0.400pt}}
\put(571.0,641.0){\rule[-0.200pt]{0.400pt}{25.054pt}}
\put(561.0,641.0){\rule[-0.200pt]{4.818pt}{0.400pt}}
\put(561.0,745.0){\rule[-0.200pt]{4.818pt}{0.400pt}}
\put(585.0,620.0){\rule[-0.200pt]{0.400pt}{25.054pt}}
\put(575.0,620.0){\rule[-0.200pt]{4.818pt}{0.400pt}}
\put(575.0,724.0){\rule[-0.200pt]{4.818pt}{0.400pt}}
\put(598.0,589.0){\rule[-0.200pt]{0.400pt}{25.054pt}}
\put(588.0,589.0){\rule[-0.200pt]{4.818pt}{0.400pt}}
\put(588.0,693.0){\rule[-0.200pt]{4.818pt}{0.400pt}}
\put(611.0,537.0){\rule[-0.200pt]{0.400pt}{25.054pt}}
\put(601.0,537.0){\rule[-0.200pt]{4.818pt}{0.400pt}}
\put(601.0,641.0){\rule[-0.200pt]{4.818pt}{0.400pt}}
\put(625.0,485.0){\rule[-0.200pt]{0.400pt}{25.054pt}}
\put(615.0,485.0){\rule[-0.200pt]{4.818pt}{0.400pt}}
\put(615.0,589.0){\rule[-0.200pt]{4.818pt}{0.400pt}}
\put(638.0,443.0){\rule[-0.200pt]{0.400pt}{25.054pt}}
\put(628.0,443.0){\rule[-0.200pt]{4.818pt}{0.400pt}}
\put(628.0,547.0){\rule[-0.200pt]{4.818pt}{0.400pt}}
\put(652.0,401.0){\rule[-0.200pt]{0.400pt}{25.054pt}}
\put(642.0,401.0){\rule[-0.200pt]{4.818pt}{0.400pt}}
\put(642.0,505.0){\rule[-0.200pt]{4.818pt}{0.400pt}}
\put(665.0,360.0){\rule[-0.200pt]{0.400pt}{25.054pt}}
\put(655.0,360.0){\rule[-0.200pt]{4.818pt}{0.400pt}}
\put(655.0,464.0){\rule[-0.200pt]{4.818pt}{0.400pt}}
\put(678.0,318.0){\rule[-0.200pt]{0.400pt}{25.054pt}}
\put(668.0,318.0){\rule[-0.200pt]{4.818pt}{0.400pt}}
\put(668.0,422.0){\rule[-0.200pt]{4.818pt}{0.400pt}}
\put(692.0,287.0){\rule[-0.200pt]{0.400pt}{25.054pt}}
\put(682.0,287.0){\rule[-0.200pt]{4.818pt}{0.400pt}}
\put(682.0,391.0){\rule[-0.200pt]{4.818pt}{0.400pt}}
\put(705.0,266.0){\rule[-0.200pt]{0.400pt}{25.054pt}}
\put(695.0,266.0){\rule[-0.200pt]{4.818pt}{0.400pt}}
\put(695.0,370.0){\rule[-0.200pt]{4.818pt}{0.400pt}}
\put(718.0,266.0){\rule[-0.200pt]{0.400pt}{25.054pt}}
\put(708.0,266.0){\rule[-0.200pt]{4.818pt}{0.400pt}}
\put(708.0,370.0){\rule[-0.200pt]{4.818pt}{0.400pt}}
\put(732.0,277.0){\rule[-0.200pt]{0.400pt}{25.054pt}}
\put(722.0,277.0){\rule[-0.200pt]{4.818pt}{0.400pt}}
\put(722.0,381.0){\rule[-0.200pt]{4.818pt}{0.400pt}}
\put(745.0,297.0){\rule[-0.200pt]{0.400pt}{25.054pt}}
\put(735.0,297.0){\rule[-0.200pt]{4.818pt}{0.400pt}}
\put(735.0,401.0){\rule[-0.200pt]{4.818pt}{0.400pt}}
\put(758.0,329.0){\rule[-0.200pt]{0.400pt}{25.054pt}}
\put(748.0,329.0){\rule[-0.200pt]{4.818pt}{0.400pt}}
\put(748.0,433.0){\rule[-0.200pt]{4.818pt}{0.400pt}}
\put(772.0,349.0){\rule[-0.200pt]{0.400pt}{25.054pt}}
\put(762.0,349.0){\rule[-0.200pt]{4.818pt}{0.400pt}}
\put(762.0,453.0){\rule[-0.200pt]{4.818pt}{0.400pt}}
\put(785.0,391.0){\rule[-0.200pt]{0.400pt}{25.054pt}}
\put(775.0,391.0){\rule[-0.200pt]{4.818pt}{0.400pt}}
\put(775.0,495.0){\rule[-0.200pt]{4.818pt}{0.400pt}}
\put(798.0,433.0){\rule[-0.200pt]{0.400pt}{25.054pt}}
\put(788.0,433.0){\rule[-0.200pt]{4.818pt}{0.400pt}}
\put(788.0,537.0){\rule[-0.200pt]{4.818pt}{0.400pt}}
\put(812.0,485.0){\rule[-0.200pt]{0.400pt}{25.054pt}}
\put(802.0,485.0){\rule[-0.200pt]{4.818pt}{0.400pt}}
\put(802.0,589.0){\rule[-0.200pt]{4.818pt}{0.400pt}}
\put(825.0,516.0){\rule[-0.200pt]{0.400pt}{25.054pt}}
\put(815.0,516.0){\rule[-0.200pt]{4.818pt}{0.400pt}}
\put(815.0,620.0){\rule[-0.200pt]{4.818pt}{0.400pt}}
\put(838.0,557.0){\rule[-0.200pt]{0.400pt}{25.054pt}}
\put(828.0,557.0){\rule[-0.200pt]{4.818pt}{0.400pt}}
\put(828.0,661.0){\rule[-0.200pt]{4.818pt}{0.400pt}}
\put(852.0,578.0){\rule[-0.200pt]{0.400pt}{25.054pt}}
\put(842.0,578.0){\rule[-0.200pt]{4.818pt}{0.400pt}}
\put(842.0,682.0){\rule[-0.200pt]{4.818pt}{0.400pt}}
\put(865.0,599.0){\rule[-0.200pt]{0.400pt}{25.054pt}}
\put(855.0,599.0){\rule[-0.200pt]{4.818pt}{0.400pt}}
\put(855.0,703.0){\rule[-0.200pt]{4.818pt}{0.400pt}}
\put(878.0,609.0){\rule[-0.200pt]{0.400pt}{25.054pt}}
\put(868.0,609.0){\rule[-0.200pt]{4.818pt}{0.400pt}}
\put(868.0,713.0){\rule[-0.200pt]{4.818pt}{0.400pt}}
\put(892.0,609.0){\rule[-0.200pt]{0.400pt}{25.054pt}}
\put(882.0,609.0){\rule[-0.200pt]{4.818pt}{0.400pt}}
\put(882.0,713.0){\rule[-0.200pt]{4.818pt}{0.400pt}}
\put(905.0,599.0){\rule[-0.200pt]{0.400pt}{25.054pt}}
\put(895.0,599.0){\rule[-0.200pt]{4.818pt}{0.400pt}}
\put(895.0,703.0){\rule[-0.200pt]{4.818pt}{0.400pt}}
\put(918.0,578.0){\rule[-0.200pt]{0.400pt}{25.054pt}}
\put(908.0,578.0){\rule[-0.200pt]{4.818pt}{0.400pt}}
\put(908.0,682.0){\rule[-0.200pt]{4.818pt}{0.400pt}}
\put(932.0,557.0){\rule[-0.200pt]{0.400pt}{25.054pt}}
\put(922.0,557.0){\rule[-0.200pt]{4.818pt}{0.400pt}}
\put(922.0,661.0){\rule[-0.200pt]{4.818pt}{0.400pt}}
\put(945.0,526.0){\rule[-0.200pt]{0.400pt}{25.054pt}}
\put(935.0,526.0){\rule[-0.200pt]{4.818pt}{0.400pt}}
\put(935.0,630.0){\rule[-0.200pt]{4.818pt}{0.400pt}}
\put(958.0,485.0){\rule[-0.200pt]{0.400pt}{25.054pt}}
\put(948.0,485.0){\rule[-0.200pt]{4.818pt}{0.400pt}}
\put(948.0,589.0){\rule[-0.200pt]{4.818pt}{0.400pt}}
\put(972.0,443.0){\rule[-0.200pt]{0.400pt}{25.054pt}}
\put(962.0,443.0){\rule[-0.200pt]{4.818pt}{0.400pt}}
\put(962.0,547.0){\rule[-0.200pt]{4.818pt}{0.400pt}}
\put(985.0,412.0){\rule[-0.200pt]{0.400pt}{25.054pt}}
\put(975.0,412.0){\rule[-0.200pt]{4.818pt}{0.400pt}}
\put(975.0,516.0){\rule[-0.200pt]{4.818pt}{0.400pt}}
\put(999.0,370.0){\rule[-0.200pt]{0.400pt}{25.054pt}}
\put(989.0,370.0){\rule[-0.200pt]{4.818pt}{0.400pt}}
\put(989.0,474.0){\rule[-0.200pt]{4.818pt}{0.400pt}}
\put(1012.0,349.0){\rule[-0.200pt]{0.400pt}{25.054pt}}
\put(1002.0,349.0){\rule[-0.200pt]{4.818pt}{0.400pt}}
\put(1002.0,453.0){\rule[-0.200pt]{4.818pt}{0.400pt}}
\put(1025.0,318.0){\rule[-0.200pt]{0.400pt}{25.054pt}}
\put(1015.0,318.0){\rule[-0.200pt]{4.818pt}{0.400pt}}
\put(1015.0,422.0){\rule[-0.200pt]{4.818pt}{0.400pt}}
\put(1039.0,308.0){\rule[-0.200pt]{0.400pt}{25.054pt}}
\put(1029.0,308.0){\rule[-0.200pt]{4.818pt}{0.400pt}}
\put(1029.0,412.0){\rule[-0.200pt]{4.818pt}{0.400pt}}
\put(1052.0,308.0){\rule[-0.200pt]{0.400pt}{25.054pt}}
\put(1042.0,308.0){\rule[-0.200pt]{4.818pt}{0.400pt}}
\put(1042.0,412.0){\rule[-0.200pt]{4.818pt}{0.400pt}}
\put(1065.0,318.0){\rule[-0.200pt]{0.400pt}{25.054pt}}
\put(1055.0,318.0){\rule[-0.200pt]{4.818pt}{0.400pt}}
\put(1055.0,422.0){\rule[-0.200pt]{4.818pt}{0.400pt}}
\put(1079.0,329.0){\rule[-0.200pt]{0.400pt}{25.054pt}}
\put(1069.0,329.0){\rule[-0.200pt]{4.818pt}{0.400pt}}
\put(1069.0,433.0){\rule[-0.200pt]{4.818pt}{0.400pt}}
\put(1092.0,349.0){\rule[-0.200pt]{0.400pt}{25.054pt}}
\put(1082.0,349.0){\rule[-0.200pt]{4.818pt}{0.400pt}}
\put(1082.0,453.0){\rule[-0.200pt]{4.818pt}{0.400pt}}
\put(1105.0,370.0){\rule[-0.200pt]{0.400pt}{25.054pt}}
\put(1095.0,370.0){\rule[-0.200pt]{4.818pt}{0.400pt}}
\put(1095.0,474.0){\rule[-0.200pt]{4.818pt}{0.400pt}}
\put(1119.0,401.0){\rule[-0.200pt]{0.400pt}{25.054pt}}
\put(1109.0,401.0){\rule[-0.200pt]{4.818pt}{0.400pt}}
\put(1109.0,505.0){\rule[-0.200pt]{4.818pt}{0.400pt}}
\put(1132.0,443.0){\rule[-0.200pt]{0.400pt}{25.054pt}}
\put(1122.0,443.0){\rule[-0.200pt]{4.818pt}{0.400pt}}
\put(1122.0,547.0){\rule[-0.200pt]{4.818pt}{0.400pt}}
\put(1145.0,474.0){\rule[-0.200pt]{0.400pt}{25.054pt}}
\put(1135.0,474.0){\rule[-0.200pt]{4.818pt}{0.400pt}}
\put(1135.0,578.0){\rule[-0.200pt]{4.818pt}{0.400pt}}
\put(1159.0,505.0){\rule[-0.200pt]{0.400pt}{25.054pt}}
\put(1149.0,505.0){\rule[-0.200pt]{4.818pt}{0.400pt}}
\put(1149.0,609.0){\rule[-0.200pt]{4.818pt}{0.400pt}}
\put(1172.0,568.0){\rule[-0.200pt]{0.400pt}{25.054pt}}
\put(1162.0,568.0){\rule[-0.200pt]{4.818pt}{0.400pt}}
\put(1162.0,672.0){\rule[-0.200pt]{4.818pt}{0.400pt}}
\put(1185.0,578.0){\rule[-0.200pt]{0.400pt}{25.054pt}}
\put(1175.0,578.0){\rule[-0.200pt]{4.818pt}{0.400pt}}
\put(1175.0,682.0){\rule[-0.200pt]{4.818pt}{0.400pt}}
\put(1199.0,578.0){\rule[-0.200pt]{0.400pt}{25.054pt}}
\put(1189.0,578.0){\rule[-0.200pt]{4.818pt}{0.400pt}}
\put(1189.0,682.0){\rule[-0.200pt]{4.818pt}{0.400pt}}
\put(1212.0,578.0){\rule[-0.200pt]{0.400pt}{25.054pt}}
\put(1202.0,578.0){\rule[-0.200pt]{4.818pt}{0.400pt}}
\put(1202.0,682.0){\rule[-0.200pt]{4.818pt}{0.400pt}}
\put(1225.0,568.0){\rule[-0.200pt]{0.400pt}{25.054pt}}
\put(1215.0,568.0){\rule[-0.200pt]{4.818pt}{0.400pt}}
\put(1215.0,672.0){\rule[-0.200pt]{4.818pt}{0.400pt}}
\put(1239.0,557.0){\rule[-0.200pt]{0.400pt}{25.054pt}}
\put(1229.0,557.0){\rule[-0.200pt]{4.818pt}{0.400pt}}
\put(1229.0,661.0){\rule[-0.200pt]{4.818pt}{0.400pt}}
\put(1252.0,526.0){\rule[-0.200pt]{0.400pt}{25.054pt}}
\put(1242.0,526.0){\rule[-0.200pt]{4.818pt}{0.400pt}}
\put(1242.0,630.0){\rule[-0.200pt]{4.818pt}{0.400pt}}
\put(1265.0,505.0){\rule[-0.200pt]{0.400pt}{25.054pt}}
\put(1255.0,505.0){\rule[-0.200pt]{4.818pt}{0.400pt}}
\put(1255.0,609.0){\rule[-0.200pt]{4.818pt}{0.400pt}}
\put(1279.0,474.0){\rule[-0.200pt]{0.400pt}{25.054pt}}
\put(1269.0,474.0){\rule[-0.200pt]{4.818pt}{0.400pt}}
\put(1269.0,578.0){\rule[-0.200pt]{4.818pt}{0.400pt}}
\put(1292.0,443.0){\rule[-0.200pt]{0.400pt}{25.054pt}}
\put(1282.0,443.0){\rule[-0.200pt]{4.818pt}{0.400pt}}
\put(1282.0,547.0){\rule[-0.200pt]{4.818pt}{0.400pt}}
\put(1306.0,422.0){\rule[-0.200pt]{0.400pt}{25.054pt}}
\put(1296.0,422.0){\rule[-0.200pt]{4.818pt}{0.400pt}}
\put(1296.0,526.0){\rule[-0.200pt]{4.818pt}{0.400pt}}
\put(1319.0,391.0){\rule[-0.200pt]{0.400pt}{25.054pt}}
\put(1309.0,391.0){\rule[-0.200pt]{4.818pt}{0.400pt}}
\put(1309.0,495.0){\rule[-0.200pt]{4.818pt}{0.400pt}}
\put(1332.0,360.0){\rule[-0.200pt]{0.400pt}{25.054pt}}
\put(1322.0,360.0){\rule[-0.200pt]{4.818pt}{0.400pt}}
\put(1322.0,464.0){\rule[-0.200pt]{4.818pt}{0.400pt}}
\put(1346.0,349.0){\rule[-0.200pt]{0.400pt}{25.054pt}}
\put(1336.0,349.0){\rule[-0.200pt]{4.818pt}{0.400pt}}
\put(1336.0,453.0){\rule[-0.200pt]{4.818pt}{0.400pt}}
\put(1359.0,339.0){\rule[-0.200pt]{0.400pt}{25.054pt}}
\put(1349.0,339.0){\rule[-0.200pt]{4.818pt}{0.400pt}}
\put(1349.0,443.0){\rule[-0.200pt]{4.818pt}{0.400pt}}
\put(1372.0,339.0){\rule[-0.200pt]{0.400pt}{25.054pt}}
\put(1362.0,339.0){\rule[-0.200pt]{4.818pt}{0.400pt}}
\put(1362.0,443.0){\rule[-0.200pt]{4.818pt}{0.400pt}}
\put(1386.0,349.0){\rule[-0.200pt]{0.400pt}{25.054pt}}
\put(1376.0,349.0){\rule[-0.200pt]{4.818pt}{0.400pt}}
\put(1376.0,453.0){\rule[-0.200pt]{4.818pt}{0.400pt}}
\put(171.0,609.0){\usebox{\plotpoint}}
\put(171.0,599.0){\rule[-0.200pt]{0.400pt}{4.818pt}}
\put(172.0,599.0){\rule[-0.200pt]{0.400pt}{4.818pt}}
\put(184.0,651.0){\usebox{\plotpoint}}
\put(184.0,641.0){\rule[-0.200pt]{0.400pt}{4.818pt}}
\put(185.0,641.0){\rule[-0.200pt]{0.400pt}{4.818pt}}
\put(197.0,713.0){\usebox{\plotpoint}}
\put(197.0,703.0){\rule[-0.200pt]{0.400pt}{4.818pt}}
\put(198.0,703.0){\rule[-0.200pt]{0.400pt}{4.818pt}}
\put(210.0,745.0){\rule[-0.200pt]{0.482pt}{0.400pt}}
\put(210.0,735.0){\rule[-0.200pt]{0.400pt}{4.818pt}}
\put(212.0,735.0){\rule[-0.200pt]{0.400pt}{4.818pt}}
\put(224.0,755.0){\usebox{\plotpoint}}
\put(224.0,745.0){\rule[-0.200pt]{0.400pt}{4.818pt}}
\put(225.0,745.0){\rule[-0.200pt]{0.400pt}{4.818pt}}
\put(237.0,765.0){\usebox{\plotpoint}}
\put(237.0,755.0){\rule[-0.200pt]{0.400pt}{4.818pt}}
\put(238.0,755.0){\rule[-0.200pt]{0.400pt}{4.818pt}}
\put(250.0,745.0){\rule[-0.200pt]{0.482pt}{0.400pt}}
\put(250.0,735.0){\rule[-0.200pt]{0.400pt}{4.818pt}}
\put(252.0,735.0){\rule[-0.200pt]{0.400pt}{4.818pt}}
\put(264.0,713.0){\usebox{\plotpoint}}
\put(264.0,703.0){\rule[-0.200pt]{0.400pt}{4.818pt}}
\put(265.0,703.0){\rule[-0.200pt]{0.400pt}{4.818pt}}
\put(277.0,672.0){\usebox{\plotpoint}}
\put(277.0,662.0){\rule[-0.200pt]{0.400pt}{4.818pt}}
\put(278.0,662.0){\rule[-0.200pt]{0.400pt}{4.818pt}}
\put(290.0,620.0){\rule[-0.200pt]{0.482pt}{0.400pt}}
\put(290.0,610.0){\rule[-0.200pt]{0.400pt}{4.818pt}}
\put(292.0,610.0){\rule[-0.200pt]{0.400pt}{4.818pt}}
\put(304.0,557.0){\usebox{\plotpoint}}
\put(304.0,547.0){\rule[-0.200pt]{0.400pt}{4.818pt}}
\put(305.0,547.0){\rule[-0.200pt]{0.400pt}{4.818pt}}
\put(317.0,505.0){\usebox{\plotpoint}}
\put(317.0,495.0){\rule[-0.200pt]{0.400pt}{4.818pt}}
\put(318.0,495.0){\rule[-0.200pt]{0.400pt}{4.818pt}}
\put(331.0,443.0){\usebox{\plotpoint}}
\put(331.0,433.0){\rule[-0.200pt]{0.400pt}{4.818pt}}
\put(332.0,433.0){\rule[-0.200pt]{0.400pt}{4.818pt}}
\put(344.0,391.0){\usebox{\plotpoint}}
\put(344.0,381.0){\rule[-0.200pt]{0.400pt}{4.818pt}}
\put(345.0,381.0){\rule[-0.200pt]{0.400pt}{4.818pt}}
\put(357.0,339.0){\rule[-0.200pt]{0.482pt}{0.400pt}}
\put(357.0,329.0){\rule[-0.200pt]{0.400pt}{4.818pt}}
\put(359.0,329.0){\rule[-0.200pt]{0.400pt}{4.818pt}}
\put(371.0,308.0){\usebox{\plotpoint}}
\put(371.0,298.0){\rule[-0.200pt]{0.400pt}{4.818pt}}
\put(372.0,298.0){\rule[-0.200pt]{0.400pt}{4.818pt}}
\put(384.0,277.0){\usebox{\plotpoint}}
\put(384.0,267.0){\rule[-0.200pt]{0.400pt}{4.818pt}}
\put(385.0,267.0){\rule[-0.200pt]{0.400pt}{4.818pt}}
\put(397.0,277.0){\rule[-0.200pt]{0.482pt}{0.400pt}}
\put(397.0,267.0){\rule[-0.200pt]{0.400pt}{4.818pt}}
\put(399.0,267.0){\rule[-0.200pt]{0.400pt}{4.818pt}}
\put(411.0,277.0){\usebox{\plotpoint}}
\put(411.0,267.0){\rule[-0.200pt]{0.400pt}{4.818pt}}
\put(412.0,267.0){\rule[-0.200pt]{0.400pt}{4.818pt}}
\put(424.0,308.0){\usebox{\plotpoint}}
\put(424.0,298.0){\rule[-0.200pt]{0.400pt}{4.818pt}}
\put(425.0,298.0){\rule[-0.200pt]{0.400pt}{4.818pt}}
\put(437.0,339.0){\rule[-0.200pt]{0.482pt}{0.400pt}}
\put(437.0,329.0){\rule[-0.200pt]{0.400pt}{4.818pt}}
\put(439.0,329.0){\rule[-0.200pt]{0.400pt}{4.818pt}}
\put(451.0,381.0){\usebox{\plotpoint}}
\put(451.0,371.0){\rule[-0.200pt]{0.400pt}{4.818pt}}
\put(452.0,371.0){\rule[-0.200pt]{0.400pt}{4.818pt}}
\put(464.0,422.0){\usebox{\plotpoint}}
\put(464.0,412.0){\rule[-0.200pt]{0.400pt}{4.818pt}}
\put(465.0,412.0){\rule[-0.200pt]{0.400pt}{4.818pt}}
\put(477.0,485.0){\rule[-0.200pt]{0.482pt}{0.400pt}}
\put(477.0,475.0){\rule[-0.200pt]{0.400pt}{4.818pt}}
\put(479.0,475.0){\rule[-0.200pt]{0.400pt}{4.818pt}}
\put(491.0,537.0){\usebox{\plotpoint}}
\put(491.0,527.0){\rule[-0.200pt]{0.400pt}{4.818pt}}
\put(492.0,527.0){\rule[-0.200pt]{0.400pt}{4.818pt}}
\put(504.0,589.0){\usebox{\plotpoint}}
\put(504.0,579.0){\rule[-0.200pt]{0.400pt}{4.818pt}}
\put(505.0,579.0){\rule[-0.200pt]{0.400pt}{4.818pt}}
\put(517.0,630.0){\rule[-0.200pt]{0.482pt}{0.400pt}}
\put(517.0,620.0){\rule[-0.200pt]{0.400pt}{4.818pt}}
\put(519.0,620.0){\rule[-0.200pt]{0.400pt}{4.818pt}}
\put(531.0,661.0){\usebox{\plotpoint}}
\put(531.0,651.0){\rule[-0.200pt]{0.400pt}{4.818pt}}
\put(532.0,651.0){\rule[-0.200pt]{0.400pt}{4.818pt}}
\put(544.0,703.0){\usebox{\plotpoint}}
\put(544.0,693.0){\rule[-0.200pt]{0.400pt}{4.818pt}}
\put(545.0,693.0){\rule[-0.200pt]{0.400pt}{4.818pt}}
\put(557.0,703.0){\rule[-0.200pt]{0.482pt}{0.400pt}}
\put(557.0,693.0){\rule[-0.200pt]{0.400pt}{4.818pt}}
\put(559.0,693.0){\rule[-0.200pt]{0.400pt}{4.818pt}}
\put(571.0,693.0){\usebox{\plotpoint}}
\put(571.0,683.0){\rule[-0.200pt]{0.400pt}{4.818pt}}
\put(572.0,683.0){\rule[-0.200pt]{0.400pt}{4.818pt}}
\put(584.0,672.0){\usebox{\plotpoint}}
\put(584.0,662.0){\rule[-0.200pt]{0.400pt}{4.818pt}}
\put(585.0,662.0){\rule[-0.200pt]{0.400pt}{4.818pt}}
\put(597.0,641.0){\rule[-0.200pt]{0.482pt}{0.400pt}}
\put(597.0,631.0){\rule[-0.200pt]{0.400pt}{4.818pt}}
\put(599.0,631.0){\rule[-0.200pt]{0.400pt}{4.818pt}}
\put(611.0,589.0){\usebox{\plotpoint}}
\put(611.0,579.0){\rule[-0.200pt]{0.400pt}{4.818pt}}
\put(612.0,579.0){\rule[-0.200pt]{0.400pt}{4.818pt}}
\put(624.0,537.0){\usebox{\plotpoint}}
\put(624.0,527.0){\rule[-0.200pt]{0.400pt}{4.818pt}}
\put(625.0,527.0){\rule[-0.200pt]{0.400pt}{4.818pt}}
\put(637.0,495.0){\rule[-0.200pt]{0.482pt}{0.400pt}}
\put(637.0,485.0){\rule[-0.200pt]{0.400pt}{4.818pt}}
\put(639.0,485.0){\rule[-0.200pt]{0.400pt}{4.818pt}}
\put(651.0,453.0){\usebox{\plotpoint}}
\put(651.0,443.0){\rule[-0.200pt]{0.400pt}{4.818pt}}
\put(652.0,443.0){\rule[-0.200pt]{0.400pt}{4.818pt}}
\put(664.0,412.0){\rule[-0.200pt]{0.482pt}{0.400pt}}
\put(664.0,402.0){\rule[-0.200pt]{0.400pt}{4.818pt}}
\put(666.0,402.0){\rule[-0.200pt]{0.400pt}{4.818pt}}
\put(678.0,370.0){\usebox{\plotpoint}}
\put(678.0,360.0){\rule[-0.200pt]{0.400pt}{4.818pt}}
\put(679.0,360.0){\rule[-0.200pt]{0.400pt}{4.818pt}}
\put(691.0,339.0){\usebox{\plotpoint}}
\put(691.0,329.0){\rule[-0.200pt]{0.400pt}{4.818pt}}
\put(692.0,329.0){\rule[-0.200pt]{0.400pt}{4.818pt}}
\put(704.0,318.0){\rule[-0.200pt]{0.482pt}{0.400pt}}
\put(704.0,308.0){\rule[-0.200pt]{0.400pt}{4.818pt}}
\put(706.0,308.0){\rule[-0.200pt]{0.400pt}{4.818pt}}
\put(718.0,318.0){\usebox{\plotpoint}}
\put(718.0,308.0){\rule[-0.200pt]{0.400pt}{4.818pt}}
\put(719.0,308.0){\rule[-0.200pt]{0.400pt}{4.818pt}}
\put(731.0,329.0){\usebox{\plotpoint}}
\put(731.0,319.0){\rule[-0.200pt]{0.400pt}{4.818pt}}
\put(732.0,319.0){\rule[-0.200pt]{0.400pt}{4.818pt}}
\put(744.0,349.0){\rule[-0.200pt]{0.482pt}{0.400pt}}
\put(744.0,339.0){\rule[-0.200pt]{0.400pt}{4.818pt}}
\put(746.0,339.0){\rule[-0.200pt]{0.400pt}{4.818pt}}
\put(758.0,381.0){\usebox{\plotpoint}}
\put(758.0,371.0){\rule[-0.200pt]{0.400pt}{4.818pt}}
\put(759.0,371.0){\rule[-0.200pt]{0.400pt}{4.818pt}}
\put(771.0,401.0){\usebox{\plotpoint}}
\put(771.0,391.0){\rule[-0.200pt]{0.400pt}{4.818pt}}
\put(772.0,391.0){\rule[-0.200pt]{0.400pt}{4.818pt}}
\put(784.0,443.0){\rule[-0.200pt]{0.482pt}{0.400pt}}
\put(784.0,433.0){\rule[-0.200pt]{0.400pt}{4.818pt}}
\put(786.0,433.0){\rule[-0.200pt]{0.400pt}{4.818pt}}
\put(798.0,485.0){\usebox{\plotpoint}}
\put(798.0,475.0){\rule[-0.200pt]{0.400pt}{4.818pt}}
\put(799.0,475.0){\rule[-0.200pt]{0.400pt}{4.818pt}}
\put(811.0,537.0){\usebox{\plotpoint}}
\put(811.0,527.0){\rule[-0.200pt]{0.400pt}{4.818pt}}
\put(812.0,527.0){\rule[-0.200pt]{0.400pt}{4.818pt}}
\put(824.0,568.0){\rule[-0.200pt]{0.482pt}{0.400pt}}
\put(824.0,558.0){\rule[-0.200pt]{0.400pt}{4.818pt}}
\put(826.0,558.0){\rule[-0.200pt]{0.400pt}{4.818pt}}
\put(838.0,609.0){\usebox{\plotpoint}}
\put(838.0,599.0){\rule[-0.200pt]{0.400pt}{4.818pt}}
\put(839.0,599.0){\rule[-0.200pt]{0.400pt}{4.818pt}}
\put(851.0,630.0){\usebox{\plotpoint}}
\put(851.0,620.0){\rule[-0.200pt]{0.400pt}{4.818pt}}
\put(852.0,620.0){\rule[-0.200pt]{0.400pt}{4.818pt}}
\put(864.0,651.0){\rule[-0.200pt]{0.482pt}{0.400pt}}
\put(864.0,641.0){\rule[-0.200pt]{0.400pt}{4.818pt}}
\put(866.0,641.0){\rule[-0.200pt]{0.400pt}{4.818pt}}
\put(878.0,661.0){\usebox{\plotpoint}}
\put(878.0,651.0){\rule[-0.200pt]{0.400pt}{4.818pt}}
\put(879.0,651.0){\rule[-0.200pt]{0.400pt}{4.818pt}}
\put(891.0,661.0){\usebox{\plotpoint}}
\put(891.0,651.0){\rule[-0.200pt]{0.400pt}{4.818pt}}
\put(892.0,651.0){\rule[-0.200pt]{0.400pt}{4.818pt}}
\put(904.0,651.0){\rule[-0.200pt]{0.482pt}{0.400pt}}
\put(904.0,641.0){\rule[-0.200pt]{0.400pt}{4.818pt}}
\put(906.0,641.0){\rule[-0.200pt]{0.400pt}{4.818pt}}
\put(918.0,630.0){\usebox{\plotpoint}}
\put(918.0,620.0){\rule[-0.200pt]{0.400pt}{4.818pt}}
\put(919.0,620.0){\rule[-0.200pt]{0.400pt}{4.818pt}}
\put(931.0,609.0){\usebox{\plotpoint}}
\put(931.0,599.0){\rule[-0.200pt]{0.400pt}{4.818pt}}
\put(932.0,599.0){\rule[-0.200pt]{0.400pt}{4.818pt}}
\put(944.0,578.0){\rule[-0.200pt]{0.482pt}{0.400pt}}
\put(944.0,568.0){\rule[-0.200pt]{0.400pt}{4.818pt}}
\put(946.0,568.0){\rule[-0.200pt]{0.400pt}{4.818pt}}
\put(958.0,537.0){\usebox{\plotpoint}}
\put(958.0,527.0){\rule[-0.200pt]{0.400pt}{4.818pt}}
\put(959.0,527.0){\rule[-0.200pt]{0.400pt}{4.818pt}}
\put(971.0,495.0){\rule[-0.200pt]{0.482pt}{0.400pt}}
\put(971.0,485.0){\rule[-0.200pt]{0.400pt}{4.818pt}}
\put(973.0,485.0){\rule[-0.200pt]{0.400pt}{4.818pt}}
\put(985.0,464.0){\usebox{\plotpoint}}
\put(985.0,454.0){\rule[-0.200pt]{0.400pt}{4.818pt}}
\put(986.0,454.0){\rule[-0.200pt]{0.400pt}{4.818pt}}
\put(998.0,422.0){\usebox{\plotpoint}}
\put(998.0,412.0){\rule[-0.200pt]{0.400pt}{4.818pt}}
\put(999.0,412.0){\rule[-0.200pt]{0.400pt}{4.818pt}}
\put(1011.0,401.0){\rule[-0.200pt]{0.482pt}{0.400pt}}
\put(1011.0,391.0){\rule[-0.200pt]{0.400pt}{4.818pt}}
\put(1013.0,391.0){\rule[-0.200pt]{0.400pt}{4.818pt}}
\put(1025.0,370.0){\usebox{\plotpoint}}
\put(1025.0,360.0){\rule[-0.200pt]{0.400pt}{4.818pt}}
\put(1026.0,360.0){\rule[-0.200pt]{0.400pt}{4.818pt}}
\put(1038.0,360.0){\usebox{\plotpoint}}
\put(1038.0,350.0){\rule[-0.200pt]{0.400pt}{4.818pt}}
\put(1039.0,350.0){\rule[-0.200pt]{0.400pt}{4.818pt}}
\put(1051.0,360.0){\rule[-0.200pt]{0.482pt}{0.400pt}}
\put(1051.0,350.0){\rule[-0.200pt]{0.400pt}{4.818pt}}
\put(1053.0,350.0){\rule[-0.200pt]{0.400pt}{4.818pt}}
\put(1065.0,370.0){\usebox{\plotpoint}}
\put(1065.0,360.0){\rule[-0.200pt]{0.400pt}{4.818pt}}
\put(1066.0,360.0){\rule[-0.200pt]{0.400pt}{4.818pt}}
\put(1078.0,381.0){\usebox{\plotpoint}}
\put(1078.0,371.0){\rule[-0.200pt]{0.400pt}{4.818pt}}
\put(1079.0,371.0){\rule[-0.200pt]{0.400pt}{4.818pt}}
\put(1091.0,401.0){\rule[-0.200pt]{0.482pt}{0.400pt}}
\put(1091.0,391.0){\rule[-0.200pt]{0.400pt}{4.818pt}}
\put(1093.0,391.0){\rule[-0.200pt]{0.400pt}{4.818pt}}
\put(1105.0,422.0){\usebox{\plotpoint}}
\put(1105.0,412.0){\rule[-0.200pt]{0.400pt}{4.818pt}}
\put(1106.0,412.0){\rule[-0.200pt]{0.400pt}{4.818pt}}
\put(1118.0,453.0){\usebox{\plotpoint}}
\put(1118.0,443.0){\rule[-0.200pt]{0.400pt}{4.818pt}}
\put(1119.0,443.0){\rule[-0.200pt]{0.400pt}{4.818pt}}
\put(1131.0,495.0){\rule[-0.200pt]{0.482pt}{0.400pt}}
\put(1131.0,485.0){\rule[-0.200pt]{0.400pt}{4.818pt}}
\put(1133.0,485.0){\rule[-0.200pt]{0.400pt}{4.818pt}}
\put(1145.0,526.0){\usebox{\plotpoint}}
\put(1145.0,516.0){\rule[-0.200pt]{0.400pt}{4.818pt}}
\put(1146.0,516.0){\rule[-0.200pt]{0.400pt}{4.818pt}}
\put(1158.0,557.0){\usebox{\plotpoint}}
\put(1158.0,547.0){\rule[-0.200pt]{0.400pt}{4.818pt}}
\put(1159.0,547.0){\rule[-0.200pt]{0.400pt}{4.818pt}}
\put(1171.0,620.0){\rule[-0.200pt]{0.482pt}{0.400pt}}
\put(1171.0,610.0){\rule[-0.200pt]{0.400pt}{4.818pt}}
\put(1173.0,610.0){\rule[-0.200pt]{0.400pt}{4.818pt}}
\put(1185.0,630.0){\usebox{\plotpoint}}
\put(1185.0,620.0){\rule[-0.200pt]{0.400pt}{4.818pt}}
\put(1186.0,620.0){\rule[-0.200pt]{0.400pt}{4.818pt}}
\put(1198.0,630.0){\usebox{\plotpoint}}
\put(1198.0,620.0){\rule[-0.200pt]{0.400pt}{4.818pt}}
\put(1199.0,620.0){\rule[-0.200pt]{0.400pt}{4.818pt}}
\put(1211.0,630.0){\rule[-0.200pt]{0.482pt}{0.400pt}}
\put(1211.0,620.0){\rule[-0.200pt]{0.400pt}{4.818pt}}
\put(1213.0,620.0){\rule[-0.200pt]{0.400pt}{4.818pt}}
\put(1225.0,620.0){\usebox{\plotpoint}}
\put(1225.0,610.0){\rule[-0.200pt]{0.400pt}{4.818pt}}
\put(1226.0,610.0){\rule[-0.200pt]{0.400pt}{4.818pt}}
\put(1238.0,609.0){\usebox{\plotpoint}}
\put(1238.0,599.0){\rule[-0.200pt]{0.400pt}{4.818pt}}
\put(1239.0,599.0){\rule[-0.200pt]{0.400pt}{4.818pt}}
\put(1251.0,578.0){\rule[-0.200pt]{0.482pt}{0.400pt}}
\put(1251.0,568.0){\rule[-0.200pt]{0.400pt}{4.818pt}}
\put(1253.0,568.0){\rule[-0.200pt]{0.400pt}{4.818pt}}
\put(1265.0,557.0){\usebox{\plotpoint}}
\put(1265.0,547.0){\rule[-0.200pt]{0.400pt}{4.818pt}}
\put(1266.0,547.0){\rule[-0.200pt]{0.400pt}{4.818pt}}
\put(1278.0,526.0){\usebox{\plotpoint}}
\put(1278.0,516.0){\rule[-0.200pt]{0.400pt}{4.818pt}}
\put(1279.0,516.0){\rule[-0.200pt]{0.400pt}{4.818pt}}
\put(1292.0,495.0){\usebox{\plotpoint}}
\put(1292.0,485.0){\rule[-0.200pt]{0.400pt}{4.818pt}}
\put(1293.0,485.0){\rule[-0.200pt]{0.400pt}{4.818pt}}
\put(1305.0,474.0){\usebox{\plotpoint}}
\put(1305.0,464.0){\rule[-0.200pt]{0.400pt}{4.818pt}}
\put(1306.0,464.0){\rule[-0.200pt]{0.400pt}{4.818pt}}
\put(1318.0,443.0){\rule[-0.200pt]{0.482pt}{0.400pt}}
\put(1318.0,433.0){\rule[-0.200pt]{0.400pt}{4.818pt}}
\put(1320.0,433.0){\rule[-0.200pt]{0.400pt}{4.818pt}}
\put(1332.0,412.0){\usebox{\plotpoint}}
\put(1332.0,402.0){\rule[-0.200pt]{0.400pt}{4.818pt}}
\put(1333.0,402.0){\rule[-0.200pt]{0.400pt}{4.818pt}}
\put(1345.0,401.0){\usebox{\plotpoint}}
\put(1345.0,391.0){\rule[-0.200pt]{0.400pt}{4.818pt}}
\put(1346.0,391.0){\rule[-0.200pt]{0.400pt}{4.818pt}}
\put(1358.0,391.0){\rule[-0.200pt]{0.482pt}{0.400pt}}
\put(1358.0,381.0){\rule[-0.200pt]{0.400pt}{4.818pt}}
\put(1360.0,381.0){\rule[-0.200pt]{0.400pt}{4.818pt}}
\put(1372.0,391.0){\usebox{\plotpoint}}
\put(1372.0,381.0){\rule[-0.200pt]{0.400pt}{4.818pt}}
\put(1373.0,381.0){\rule[-0.200pt]{0.400pt}{4.818pt}}
\put(1385.0,401.0){\usebox{\plotpoint}}
\put(1385.0,391.0){\rule[-0.200pt]{0.400pt}{4.818pt}}
\put(171,609){\makebox(0,0){$+$}}
\put(184,651){\makebox(0,0){$+$}}
\put(198,713){\makebox(0,0){$+$}}
\put(211,745){\makebox(0,0){$+$}}
\put(224,755){\makebox(0,0){$+$}}
\put(238,765){\makebox(0,0){$+$}}
\put(251,745){\makebox(0,0){$+$}}
\put(264,713){\makebox(0,0){$+$}}
\put(278,672){\makebox(0,0){$+$}}
\put(291,620){\makebox(0,0){$+$}}
\put(304,557){\makebox(0,0){$+$}}
\put(318,505){\makebox(0,0){$+$}}
\put(331,443){\makebox(0,0){$+$}}
\put(345,391){\makebox(0,0){$+$}}
\put(358,339){\makebox(0,0){$+$}}
\put(371,308){\makebox(0,0){$+$}}
\put(385,277){\makebox(0,0){$+$}}
\put(398,277){\makebox(0,0){$+$}}
\put(411,277){\makebox(0,0){$+$}}
\put(425,308){\makebox(0,0){$+$}}
\put(438,339){\makebox(0,0){$+$}}
\put(451,381){\makebox(0,0){$+$}}
\put(465,422){\makebox(0,0){$+$}}
\put(478,485){\makebox(0,0){$+$}}
\put(491,537){\makebox(0,0){$+$}}
\put(505,589){\makebox(0,0){$+$}}
\put(518,630){\makebox(0,0){$+$}}
\put(531,661){\makebox(0,0){$+$}}
\put(545,703){\makebox(0,0){$+$}}
\put(558,703){\makebox(0,0){$+$}}
\put(571,693){\makebox(0,0){$+$}}
\put(585,672){\makebox(0,0){$+$}}
\put(598,641){\makebox(0,0){$+$}}
\put(611,589){\makebox(0,0){$+$}}
\put(625,537){\makebox(0,0){$+$}}
\put(638,495){\makebox(0,0){$+$}}
\put(652,453){\makebox(0,0){$+$}}
\put(665,412){\makebox(0,0){$+$}}
\put(678,370){\makebox(0,0){$+$}}
\put(692,339){\makebox(0,0){$+$}}
\put(705,318){\makebox(0,0){$+$}}
\put(718,318){\makebox(0,0){$+$}}
\put(732,329){\makebox(0,0){$+$}}
\put(745,349){\makebox(0,0){$+$}}
\put(758,381){\makebox(0,0){$+$}}
\put(772,401){\makebox(0,0){$+$}}
\put(785,443){\makebox(0,0){$+$}}
\put(798,485){\makebox(0,0){$+$}}
\put(812,537){\makebox(0,0){$+$}}
\put(825,568){\makebox(0,0){$+$}}
\put(838,609){\makebox(0,0){$+$}}
\put(852,630){\makebox(0,0){$+$}}
\put(865,651){\makebox(0,0){$+$}}
\put(878,661){\makebox(0,0){$+$}}
\put(892,661){\makebox(0,0){$+$}}
\put(905,651){\makebox(0,0){$+$}}
\put(918,630){\makebox(0,0){$+$}}
\put(932,609){\makebox(0,0){$+$}}
\put(945,578){\makebox(0,0){$+$}}
\put(958,537){\makebox(0,0){$+$}}
\put(972,495){\makebox(0,0){$+$}}
\put(985,464){\makebox(0,0){$+$}}
\put(999,422){\makebox(0,0){$+$}}
\put(1012,401){\makebox(0,0){$+$}}
\put(1025,370){\makebox(0,0){$+$}}
\put(1039,360){\makebox(0,0){$+$}}
\put(1052,360){\makebox(0,0){$+$}}
\put(1065,370){\makebox(0,0){$+$}}
\put(1079,381){\makebox(0,0){$+$}}
\put(1092,401){\makebox(0,0){$+$}}
\put(1105,422){\makebox(0,0){$+$}}
\put(1119,453){\makebox(0,0){$+$}}
\put(1132,495){\makebox(0,0){$+$}}
\put(1145,526){\makebox(0,0){$+$}}
\put(1159,557){\makebox(0,0){$+$}}
\put(1172,620){\makebox(0,0){$+$}}
\put(1185,630){\makebox(0,0){$+$}}
\put(1199,630){\makebox(0,0){$+$}}
\put(1212,630){\makebox(0,0){$+$}}
\put(1225,620){\makebox(0,0){$+$}}
\put(1239,609){\makebox(0,0){$+$}}
\put(1252,578){\makebox(0,0){$+$}}
\put(1265,557){\makebox(0,0){$+$}}
\put(1279,526){\makebox(0,0){$+$}}
\put(1292,495){\makebox(0,0){$+$}}
\put(1306,474){\makebox(0,0){$+$}}
\put(1319,443){\makebox(0,0){$+$}}
\put(1332,412){\makebox(0,0){$+$}}
\put(1346,401){\makebox(0,0){$+$}}
\put(1359,391){\makebox(0,0){$+$}}
\put(1372,391){\makebox(0,0){$+$}}
\put(1386,401){\makebox(0,0){$+$}}
\put(1349,819){\makebox(0,0){$+$}}
\put(1386.0,391.0){\rule[-0.200pt]{0.400pt}{4.818pt}}
\put(1279,778){\makebox(0,0)[r]{fit}}
\multiput(1299,778)(20.756,0.000){5}{\usebox{\plotpoint}}
\put(1399,778){\usebox{\plotpoint}}
\put(171,597){\usebox{\plotpoint}}
\multiput(171,597)(4.424,20.278){3}{\usebox{\plotpoint}}
\multiput(183,652)(5.760,19.940){2}{\usebox{\plotpoint}}
\multiput(196,697)(6.908,19.572){2}{\usebox{\plotpoint}}
\put(211.14,736.23){\usebox{\plotpoint}}
\put(223.16,752.58){\usebox{\plotpoint}}
\put(241.26,751.30){\usebox{\plotpoint}}
\put(253.12,734.79){\usebox{\plotpoint}}
\multiput(257,728)(6.908,-19.572){2}{\usebox{\plotpoint}}
\multiput(269,694)(5.579,-19.992){2}{\usebox{\plotpoint}}
\multiput(281,651)(5.034,-20.136){2}{\usebox{\plotpoint}}
\multiput(294,599)(4.424,-20.278){3}{\usebox{\plotpoint}}
\multiput(306,544)(4.276,-20.310){3}{\usebox{\plotpoint}}
\multiput(318,487)(4.693,-20.218){3}{\usebox{\plotpoint}}
\multiput(331,431)(4.844,-20.182){2}{\usebox{\plotpoint}}
\multiput(343,381)(5.579,-19.992){2}{\usebox{\plotpoint}}
\multiput(355,338)(7.093,-19.506){2}{\usebox{\plotpoint}}
\put(372.74,295.28){\usebox{\plotpoint}}
\put(385.19,279.11){\usebox{\plotpoint}}
\put(403.88,276.97){\usebox{\plotpoint}}
\multiput(417,293)(8.430,18.967){2}{\usebox{\plotpoint}}
\multiput(429,320)(6.563,19.690){2}{\usebox{\plotpoint}}
\multiput(441,356)(5.461,20.024){2}{\usebox{\plotpoint}}
\multiput(453,400)(5.322,20.061){2}{\usebox{\plotpoint}}
\multiput(466,449)(4.754,20.204){3}{\usebox{\plotpoint}}
\multiput(478,500)(4.937,20.160){2}{\usebox{\plotpoint}}
\multiput(490,549)(5.135,20.110){3}{\usebox{\plotpoint}}
\multiput(502,596)(6.415,19.739){2}{\usebox{\plotpoint}}
\put(520.76,651.36){\usebox{\plotpoint}}
\multiput(527,668)(9.601,18.402){2}{\usebox{\plotpoint}}
\put(553.53,702.00){\usebox{\plotpoint}}
\put(571.58,695.05){\usebox{\plotpoint}}
\put(583.32,678.18){\usebox{\plotpoint}}
\multiput(588,670)(8.253,-19.044){2}{\usebox{\plotpoint}}
\put(607.21,620.84){\usebox{\plotpoint}}
\multiput(613,603)(5.702,-19.957){3}{\usebox{\plotpoint}}
\multiput(625,561)(5.348,-20.055){2}{\usebox{\plotpoint}}
\multiput(637,516)(5.760,-19.940){2}{\usebox{\plotpoint}}
\multiput(650,471)(5.702,-19.957){2}{\usebox{\plotpoint}}
\multiput(662,429)(6.250,-19.792){2}{\usebox{\plotpoint}}
\multiput(674,391)(8.027,-19.141){2}{\usebox{\plotpoint}}
\put(695.42,344.56){\usebox{\plotpoint}}
\put(708.01,328.24){\usebox{\plotpoint}}
\put(726.61,323.67){\usebox{\plotpoint}}
\put(742.00,336.51){\usebox{\plotpoint}}
\put(752.63,354.27){\usebox{\plotpoint}}
\multiput(760,369)(7.812,19.229){2}{\usebox{\plotpoint}}
\multiput(773,401)(6.563,19.690){2}{\usebox{\plotpoint}}
\multiput(785,437)(5.964,19.880){2}{\usebox{\plotpoint}}
\multiput(797,477)(5.964,19.880){2}{\usebox{\plotpoint}}
\multiput(809,517)(6.718,19.638){2}{\usebox{\plotpoint}}
\multiput(822,555)(6.732,19.634){2}{\usebox{\plotpoint}}
\put(841.71,608.64){\usebox{\plotpoint}}
\put(850.42,627.46){\usebox{\plotpoint}}
\put(860.97,645.20){\usebox{\plotpoint}}
\put(876.55,658.31){\usebox{\plotpoint}}
\put(895.99,657.01){\usebox{\plotpoint}}
\put(910.01,641.81){\usebox{\plotpoint}}
\multiput(920,626)(8.430,-18.967){2}{\usebox{\plotpoint}}
\put(937.06,585.93){\usebox{\plotpoint}}
\multiput(944,568)(7.227,-19.457){2}{\usebox{\plotpoint}}
\multiput(957,533)(6.732,-19.634){2}{\usebox{\plotpoint}}
\multiput(969,498)(6.732,-19.634){2}{\usebox{\plotpoint}}
\put(986.17,449.21){\usebox{\plotpoint}}
\multiput(993,431)(8.740,-18.825){2}{\usebox{\plotpoint}}
\put(1011.93,392.63){\usebox{\plotpoint}}
\put(1023.32,375.35){\usebox{\plotpoint}}
\put(1038.86,362.23){\usebox{\plotpoint}}
\put(1058.34,363.22){\usebox{\plotpoint}}
\put(1073.21,377.27){\usebox{\plotpoint}}
\put(1084.65,394.56){\usebox{\plotpoint}}
\multiput(1092,407)(8.430,18.967){2}{\usebox{\plotpoint}}
\put(1110.78,450.96){\usebox{\plotpoint}}
\multiput(1116,464)(7.493,19.356){2}{\usebox{\plotpoint}}
\put(1133.79,508.81){\usebox{\plotpoint}}
\multiput(1141,526)(7.708,19.271){2}{\usebox{\plotpoint}}
\put(1157.74,566.28){\usebox{\plotpoint}}
\multiput(1165,582)(10.925,17.648){2}{\usebox{\plotpoint}}
\put(1190.87,618.58){\usebox{\plotpoint}}
\put(1209.35,626.61){\usebox{\plotpoint}}
\put(1228.25,619.75){\usebox{\plotpoint}}
\put(1242.08,604.38){\usebox{\plotpoint}}
\multiput(1251,591)(10.559,-17.869){2}{\usebox{\plotpoint}}
\put(1272.65,550.26){\usebox{\plotpoint}}
\put(1281.02,531.28){\usebox{\plotpoint}}
\multiput(1288,515)(8.176,-19.077){2}{\usebox{\plotpoint}}
\put(1306.11,474.30){\usebox{\plotpoint}}
\multiput(1313,460)(9.282,-18.564){2}{\usebox{\plotpoint}}
\put(1335.06,419.23){\usebox{\plotpoint}}
\put(1347.62,402.73){\usebox{\plotpoint}}
\put(1364.65,391.34){\usebox{\plotpoint}}
\put(1384.78,391.70){\usebox{\plotpoint}}
\put(1386,392){\usebox{\plotpoint}}
\put(171.0,131.0){\rule[-0.200pt]{0.400pt}{175.375pt}}
\put(171.0,131.0){\rule[-0.200pt]{305.461pt}{0.400pt}}
\put(1439.0,131.0){\rule[-0.200pt]{0.400pt}{175.375pt}}
\put(171.0,859.0){\rule[-0.200pt]{305.461pt}{0.400pt}}
\end{picture}

\caption{Vynesená závislosť polohy $l$ na čase $t$ pre druhú polohu a fit $l(t) = \(70.9\pm0.5\)\exp\(-\(4.9\pm0.5\)10^{-4}t\)\(\sin\(\frac{2\pi t}{488.9\pm0.5}\)+ \(4.79\pm0.04\)\) + \(125.6\pm 0.1\)$.}  \label{G_3}
\end{figure}


\begin{figure}
% GNUPLOT: LaTeX picture
\setlength{\unitlength}{0.240900pt}
\ifx\plotpoint\undefined\newsavebox{\plotpoint}\fi
\sbox{\plotpoint}{\rule[-0.200pt]{0.400pt}{0.400pt}}%
\begin{picture}(1500,900)(0,0)
\sbox{\plotpoint}{\rule[-0.200pt]{0.400pt}{0.400pt}}%
\put(151.0,131.0){\rule[-0.200pt]{4.818pt}{0.400pt}}
\put(131,131){\makebox(0,0)[r]{ 24}}
\put(1419.0,131.0){\rule[-0.200pt]{4.818pt}{0.400pt}}
\put(151.0,235.0){\rule[-0.200pt]{4.818pt}{0.400pt}}
\put(131,235){\makebox(0,0)[r]{ 25}}
\put(1419.0,235.0){\rule[-0.200pt]{4.818pt}{0.400pt}}
\put(151.0,339.0){\rule[-0.200pt]{4.818pt}{0.400pt}}
\put(131,339){\makebox(0,0)[r]{ 26}}
\put(1419.0,339.0){\rule[-0.200pt]{4.818pt}{0.400pt}}
\put(151.0,443.0){\rule[-0.200pt]{4.818pt}{0.400pt}}
\put(131,443){\makebox(0,0)[r]{ 27}}
\put(1419.0,443.0){\rule[-0.200pt]{4.818pt}{0.400pt}}
\put(151.0,547.0){\rule[-0.200pt]{4.818pt}{0.400pt}}
\put(131,547){\makebox(0,0)[r]{ 28}}
\put(1419.0,547.0){\rule[-0.200pt]{4.818pt}{0.400pt}}
\put(151.0,651.0){\rule[-0.200pt]{4.818pt}{0.400pt}}
\put(131,651){\makebox(0,0)[r]{ 29}}
\put(1419.0,651.0){\rule[-0.200pt]{4.818pt}{0.400pt}}
\put(151.0,755.0){\rule[-0.200pt]{4.818pt}{0.400pt}}
\put(131,755){\makebox(0,0)[r]{ 30}}
\put(1419.0,755.0){\rule[-0.200pt]{4.818pt}{0.400pt}}
\put(151.0,859.0){\rule[-0.200pt]{4.818pt}{0.400pt}}
\put(131,859){\makebox(0,0)[r]{ 31}}
\put(1419.0,859.0){\rule[-0.200pt]{4.818pt}{0.400pt}}
\put(151.0,131.0){\rule[-0.200pt]{0.400pt}{4.818pt}}
\put(151,90){\makebox(0,0){ 0}}
\put(151.0,839.0){\rule[-0.200pt]{0.400pt}{4.818pt}}
\put(312.0,131.0){\rule[-0.200pt]{0.400pt}{4.818pt}}
\put(312,90){\makebox(0,0){ 20}}
\put(312.0,839.0){\rule[-0.200pt]{0.400pt}{4.818pt}}
\put(473.0,131.0){\rule[-0.200pt]{0.400pt}{4.818pt}}
\put(473,90){\makebox(0,0){ 40}}
\put(473.0,839.0){\rule[-0.200pt]{0.400pt}{4.818pt}}
\put(634.0,131.0){\rule[-0.200pt]{0.400pt}{4.818pt}}
\put(634,90){\makebox(0,0){ 60}}
\put(634.0,839.0){\rule[-0.200pt]{0.400pt}{4.818pt}}
\put(795.0,131.0){\rule[-0.200pt]{0.400pt}{4.818pt}}
\put(795,90){\makebox(0,0){ 80}}
\put(795.0,839.0){\rule[-0.200pt]{0.400pt}{4.818pt}}
\put(956.0,131.0){\rule[-0.200pt]{0.400pt}{4.818pt}}
\put(956,90){\makebox(0,0){ 100}}
\put(956.0,839.0){\rule[-0.200pt]{0.400pt}{4.818pt}}
\put(1117.0,131.0){\rule[-0.200pt]{0.400pt}{4.818pt}}
\put(1117,90){\makebox(0,0){ 120}}
\put(1117.0,839.0){\rule[-0.200pt]{0.400pt}{4.818pt}}
\put(1278.0,131.0){\rule[-0.200pt]{0.400pt}{4.818pt}}
\put(1278,90){\makebox(0,0){ 140}}
\put(1278.0,839.0){\rule[-0.200pt]{0.400pt}{4.818pt}}
\put(1439.0,131.0){\rule[-0.200pt]{0.400pt}{4.818pt}}
\put(1439,90){\makebox(0,0){ 160}}
\put(1439.0,839.0){\rule[-0.200pt]{0.400pt}{4.818pt}}
\put(151.0,131.0){\rule[-0.200pt]{0.400pt}{175.375pt}}
\put(151.0,131.0){\rule[-0.200pt]{310.279pt}{0.400pt}}
\put(1439.0,131.0){\rule[-0.200pt]{0.400pt}{175.375pt}}
\put(151.0,859.0){\rule[-0.200pt]{310.279pt}{0.400pt}}
\put(30,495){\makebox(0,0){\popi{t}{\C}}}
\put(795,29){\makebox(0,0){\popi{t}{s}}}
\put(1279,819){\makebox(0,0)[r]{file u 1:2}}
\put(151,235){\makebox(0,0){$+$}}
\put(151,235){\makebox(0,0){$+$}}
\put(232,235){\makebox(0,0){$+$}}
\put(312,235){\makebox(0,0){$+$}}
\put(393,225){\makebox(0,0){$+$}}
\put(473,256){\makebox(0,0){$+$}}
\put(554,318){\makebox(0,0){$+$}}
\put(634,391){\makebox(0,0){$+$}}
\put(715,412){\makebox(0,0){$+$}}
\put(795,443){\makebox(0,0){$+$}}
\put(876,474){\makebox(0,0){$+$}}
\put(956,526){\makebox(0,0){$+$}}
\put(1037,703){\makebox(0,0){$+$}}
\put(1117,734){\makebox(0,0){$+$}}
\put(1198,745){\makebox(0,0){$+$}}
\put(1278,765){\makebox(0,0){$+$}}
\put(1359,786){\makebox(0,0){$+$}}
\put(1439,797){\makebox(0,0){$+$}}
\put(1349,819){\makebox(0,0){$+$}}
\put(151.0,131.0){\rule[-0.200pt]{0.400pt}{175.375pt}}
\put(151.0,131.0){\rule[-0.200pt]{310.279pt}{0.400pt}}
\put(1439.0,131.0){\rule[-0.200pt]{0.400pt}{175.375pt}}
\put(151.0,859.0){\rule[-0.200pt]{310.279pt}{0.400pt}}
\end{picture}

\caption{Namerané hodnoty polohy $l$ na čase $t$ bez chybových úsečiek, so zanesenými hodnotami nulových polôh $S^1$ a $S^2$ pre jednotlivé polohy.}  \label{G_4}

\end{figure}
\section{Diskusia}

Hodnota nami odmeraného $G$ sa rádovo zhoduje s tabuľkovým\cite{C_2} \eq{
G_{tab}="\(6.674\,08\pm0,000\,31\)\cdot10^{-11} m^3 \cdot kg^{-1}\cdot s^{-2}"\,.
}
zároveň vidíme, že tento experiment bol zaťažený systematickými chybami. 

V prvej rade sa jedná o chyba pri zisťovaní dĺžky,
či už vzdialenosti aparatúry od premietacej steny alebo určovaní polohy laseru.
Pri fitovaní cez niekoľkých periód sa však táto chyba fixuje a jej vplyv n výsledok je malý.

Ďalším zdrojom možnej systematickej chyby je počiatočné veľké kmity, kde kyvadlo narážalo na okraje, tieto kyvy boli neboli zanášané do fitu ale predovšetkým na prvý kyv, mohli mať vplyv.

Ďalej nerovnosti na guliach vďaka, ktorým sa nedali narovnako prisunúť k aparatúre a teda vzdialenosť $b$, mohla byť pre každú z polôh odlišná.

Posledným výrazným problémom je uzemnenie aparatúry o radiátor, predovšetkým počas vykurovacej sezóny, môže na radiátore vznikať statický náboj a ten sa preniesť na aparatúru a tým ovplyvniť meranie.


\section{Záver}
Gravitačnú konštantu sme pomocou Cavendishového experimentu určili\eq{
G = \(8.21\cdot10^{-11}\pm0.86\cdot10^{-11}\) \jd{m^3 \cdot kg^{-1}\cdot s^{-2}}\,.
}

\begin{thebibliography}{2}
\bibitem{C_1}
Cavendishův experiment [cit. 15.10.2017] Dostupné po prihlásení na: \url{https://praktikum.fjfi.cvut.cz/mod/resource/view.php?id=16}
\bibitem{C_2}
Fyzikální a jiné konstanty [cit. 15.10.2017] Jiří Bureš: \url{http://www.converter.cz/prevody/konstanty.htm}

\end{thebibliography}

\end{document}




